
\newpage
\section{Unfitted cut discontinuous Galerkin method for the Poisson problem}%
\label{sec:elliptic}

\subsection{Poisson problem}%
\label{sub:possion_problem}
We will first consider the continuous Poisson problem. Let $f \in H^1( \Omega ) $ and $g \in H^{\frac{1}{2}}( \Gamma ) $ and $\Omega  \in \mathbb{R} ^{d}$ . We then define the strong formulation of the Possion problem to be \[
\begin{split}
    -\Delta u &= f \quad  \text{in }\Omega  \\
     u &= g  \quad \text{on } \Gamma    \\
\end{split} .
\]
Let us define the Hilbert spaces $V=H^{1}( \Omega ) $,   $V_{g} = \left\{ v \in H^{1}( \Omega ): v \mid _{\Gamma } = g \right\} $, the bilinear form $a: V \times V  \to \mathbb{R}  $ and the linear form $l: V'\to \mathbb{R}  $ s.t. \[
a( u,v) = ( \nabla u, \nabla v) _{\Omega }, \quad l( v) = (f,v)_{\Omega }.
\]
We say the weak formulation is to find a $u \in V_{g}$ so this equation holds  \[
a( u,v) = l( v), \quad  \forall v \in V
\]
\subsection{CutFEM }%
\label{sub:cutfem}

One of the key element to unfitted methods is that is relying on a background mesh. Let $\widetilde{\Omega }$ as a background domain with a corresponding shape regular and quasi-uniform background mesh $\widetilde{\mathcal{T}}_{h} $. We will assume we
have a physical domain $\Omega
\subset \widetilde{\Omega }$ with a corresponding $\Gamma \in C^2 $ boundary. For $\mathcal{T}_{h} $ we define a the so-called active mesh
$\mathcal{T} _{h} \subset \widetilde{\mathcal{T} }_{h}$ consisting of those elements that intersect with the interior of $\Omega^{\circ } = \Omega  \setminus \Gamma   $. That is,\[
\mathcal{T }_{h} = \left\{ T \in  \mathcal{T}_{h}  \mid  T \cap \Omega \neq \emptyset   \right\}.
\]
The set of interior facets in $\mathcal{T}_{h} $ is defined as $\mathcal{F}_{h} $.
Let the submesh $\mathcal{T}_{\Gamma } \subset \mathcal{T}_{h}  $ consisting of all cut elements, \[
\mathcal{T} _{\Gamma  } = \left\{ T \in \mathcal{T} _{h}  \mid  T \cap  \Gamma  \neq \emptyset  \right\}.
\]

\todo[inline]{ Wrap the above definitions into definitions blocks.}

We denote the discrete to function space $V_{h}$ to be the broken polynomial space of order $k$, that is, $V_{h} := \mathcal{P}^{k}( \mathcal{T}_{h} )  $.


Let the bilinear form $a_{h}: V_{h} \times V_{h} \to \mathbb{R} $  and the linear form $l_{h}: V_{h} \to R$. For the discontinuous Galerkin Poisson formulation (Poisson DG) to be

\begin{equation}
\label{eq:poisson_DG}
\begin{split}
    a_{h}( v,w)  = &( \nabla v, \nabla w)_{\mathcal{T} _{h} \cap \Omega } - ( \partial _{n} v,w)_{\Gamma } - ( v, \partial _{n} w)_{\Gamma } + \beta ( h^{-1} v,w)_{\Gamma } \\
    & - ( \mean{ \partial _{n} v }  , \jump{ w }  ) _{\mathcal{F} _{h} \cap \Omega } - ( \jump{ v }  , \mean{ \partial _{n} w }  )_{\mathcal{F} _{h}} + \beta ( h^{-1} \jump{ v }  , \jump{ w }  )_{\mathcal{F}_{h}\cap \Omega  } \\
    l_{h}( v)  &=  ( f,v) _{\mathcal{T} _{h} \cap \Omega } - ( \partial _{n} v,g) _{\Gamma } + \beta ( h^{-1} g,v)
\end{split}
\end{equation}






