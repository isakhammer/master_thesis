
\newpage
\section{Unfitted cut discontinuous Galerkin method for the Poisson problem}%
\label{sec:elliptic}

\subsection{Poisson problem}%
\label{sub:possion_problem}
We will first consider the continuous Poisson problem. Let $f \in H^1( \Omega ) $ and $g \in H^{\frac{1}{2}}( \Gamma ) $ and $\Omega  \in \mathbb{R} ^{d}$ . We then define the strong formulation of the Possion problem to be \[
\begin{split}
    -\Delta u &= f \quad  \text{in }\Omega  \\
     u &= g  \quad \text{on } \Gamma    \\
\end{split} .
\]
Let us define the Hilbert spaces $V=H^{1}( \Omega ) $,   $V_{g} = \left\{ v \in H^{1}( \Omega ): v \mid _{\Gamma } = g \right\} $, the bilinear form $a: V \times V  \to \mathbb{R}  $ and the linear form $l: V'\to \mathbb{R}  $ s.t. \[
a( u,v) = ( \nabla u, \nabla v) _{\Omega }, \quad l( v) = (f,v)_{\Omega }.
\]
We say the weak formulation is to find a $u \in V_{g}$ so this equation holds  \[
a( u,v) = l( v), \quad  \forall v \in V
\]
\subsection{CutFEM }%
\label{sub:cutfem}

One of the key element to unfitted methods is that is relying on a background mesh. We denote the shape regular and quasi-uniform background mesh as $\widetilde{\mathcal{T}}_{h} $ which is covering $\Omega $.









