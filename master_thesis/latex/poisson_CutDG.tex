
\newpage
\section{Unfitted cut discontinuous Galerkin method for the Poisson problem}%
\label{sec:elliptic}

\subsection{Poisson problem}%
\label{sub:possion_problem}
We will first consider the continuous Poisson problem. Let $f \in H^1( \Omega ) $ and $g \in H^{\frac{1}{2}}( \Gamma ) $ and $\Omega  \in \mathbb{R} ^{d}$ . We then define the strong formulation of the Possion problem to be \[
\begin{split}
    -\Delta u &= f \quad  \text{in }\Omega  \\
     u &= g  \quad \text{on } \Gamma    \\
\end{split} .
\]
Let us define the Hilbert spaces $V=H^{1}( \Omega ) $,   $V_{g} = \left\{ v \in H^{1}( \Omega ): v \mid _{\Gamma } = g \right\} $, the bilinear form $a: V \times V  \to \mathbb{R}  $ and the linear form $l: V'\to \mathbb{R}  $ s.t. \[
a( u,v) = ( \nabla u, \nabla v) _{\Omega }, \quad l( v) = (f,v)_{\Omega }.
\]
We say the weak formulation is to find a $u \in V_{g}$ so this equation holds  \[
a( u,v) = l( v), \quad  \forall v \in V
\]
\subsection{CutFEM }%
\label{sub:cutfem}

One of the key element to unfitted methods is that is relying on a background mesh. Let $\widetilde{\Omega }$ as a background domain with a corresponding shape regular and quasi-uniform background mesh $\widetilde{\mathcal{T}}_{h} $. We will assume we
have a physical domain $\Omega
\subset \widetilde{\Omega }$ with a corresponding $\Gamma \in C^2 $ boundary. For $\mathcal{T}_{h} $ we define a the so-called active mesh
$\mathcal{T} _{h} \subset \widetilde{\mathcal{T} }_{h}$ consisting of those elements that intersect with the interior of $\Omega^{\circ } = \Omega  \setminus \Gamma   $. That is,\[
\mathcal{T }_{h} = \left\{ T \in  \mathcal{T}_{h}  \mid  T \cap \Omega \neq \emptyset   \right\}.
\]
The set of interior facets in $\mathcal{T}_{h} $ is defined as $\mathcal{F}_{h} $.
Let the submesh $\mathcal{T}_{\Gamma } \subset \mathcal{T}_{h}  $ consisting of all cut elements, \[
\mathcal{T} _{\Gamma  } = \left\{ T \in \mathcal{T} _{h}  \mid  T \cap  \Gamma  \neq \emptyset  \right\}.
\]

\todo[inline]{ Wrap the above definitions into definitions blocks.}

We denote the discrete to function space $V_{h}$ to be the broken polynomial space of order $k$, that is, $V_{h} := \mathcal{P}^{k}( \mathcal{T}_{h} )  $.


Let the bilinear form $a_{h}: V_{h} \times V_{h} \to \mathbb{R} $  and the linear form $l_{h}: V_{h} \to \mathbb{R} $. We denote the symmetric interior penalty discontinuous Galerkin Poisson (DG Poisson)  formulation to be

\begin{equation}
\label{eq:poisson_DG}
\begin{split}
    a_{h}( v,w)  = &( \nabla v, \nabla w)_{\mathcal{T} _{h} \cap \Omega } - ( \partial _{n} v,w)_{\Gamma } - ( v, \partial _{n} w)_{\Gamma } + \beta ( h^{-1} v,w)_{\Gamma } \\
    & - ( \mean{ \partial _{n} v }  , \jump{ w }  ) _{\mathcal{F} _{h} \cap \Omega } - ( \jump{ v }  , \mean{ \partial _{n} w }  )_{\mathcal{F} _{h}} + \beta ( h^{-1} \jump{ v }  , \jump{ w }  )_{\mathcal{F}_{h}\cap \Omega  } \\
    l_{h}( v)  &=  ( f,v) _{\mathcal{T} _{h} \cap \Omega } - ( \partial _{n} v,g) _{\Gamma } + \beta ( h^{-1} g,v)_{\Gamma }
\end{split}
\end{equation}
For more information about the derivation, see \cite[Chapter 4.2]{pietro2012}. Remark that this formulation is defined on unfitted meshed and, thus, imposing Nitsche penalty to fulfill the boundary condition.
\todo[inline]{ Need source of the derivation or maybe need to to it myself }
However, to show well-posedness will we later see that it is necessary to add a stability term . Hence, the symmetric interior penalty cut discontinuous Galerkin Poisson (CutDG Poisson) formulation,
\[
A_{h}( u, v) := a_{h}( u, v) + g_{h}( u_{h}, v) = l_{h}(v)
\]
We denote the stability term $g_{h}: V_{h} \times V_{h} \to \mathbb{R} $ as the so-called ghost penalty term.

\begin{definition}[CutDG Poisson problem]
We denote the CutDG Poisson problem as follows; to find a unique $u_{h} \in V_{h}$ s.t.  \[
A_{h}(u_{h}, v ) = l_{h}( v)  \quad \forall  v \in V_{h}.
\]
\end{definition}

Our main goal is to determine the necessary stability criteria for the ghost penalty term for the CutDG Poisson problem to be well-posed. We can then apply these criteria to engineer a ghost penalty to handle this problem. For convenience will we make some assumptions on our problem.


\begin{assumption}[G1]
    \label{as:G1}
    The boundary $\Gamma $ is $C^{2}$.
\end{assumption}

\begin{remark}
In most applications is $\Gamma $ represented with a level-set function, that is a sufficiently smooth function $\phi ( x)  = 0$. There exists several methods to discretize this properly s.t. it can we can compute the integral over the cut elements.
    However, to simplify the analysis will we assume the numerical contributions of the cut elements consisting of $\mathcal{T}_{h} \cap \Gamma   $ and $\mathcal{F}_{h} \cap \Omega  $ to be exact.
\end{remark}


\begin{assumption}[G1]
    $\mathcal{T}_{h} $ is mesh conform, quasi-uniform and shape regular.
\end{assumption}

\begin{assumption}[G3]
    For $T \in \mathcal{T} _{\Gamma   }$ there is a path $P$ of $diam(P) \lesssim h$ which contains $T$ and an element $T'$ with a "fat" interaction satisfying $\left\lvert T' \cap \Omega  \right\rvert \ge c_{s} \left\lvert T'  \right\rvert _{d}$
    \todo[inline]{ Don't understand this definition}
    \todo[inline]{ Define patch in mathematical background. }
\end{assumption}



\subsection{Norms}%
\label{sub:norms}



