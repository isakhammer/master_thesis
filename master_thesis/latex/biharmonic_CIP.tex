
\newpage
\section{Continuous Interior Penalty Biharmonic Problem on a Polygonal Domain}%
\label{sec:CIP_biharmonic_problem}

Recall the biharmonic problem.
Let $\Omega \subseteq    \mathbb{R} ^d$ be a bounded polygonal domain and $\Gamma $ be its corresponding boundary. Also let $\mathcal{T}_{h} = \left\{ T \right\} $ be a shape-regular fitted mesh s.t. $\mathcal{T}_{h} = \Omega $. Let the BH have the form,
\begin{equation}
\label{eq:bi_problem}
\begin{split}
    \Delta^2  u  + \alpha  u  & = f( x)  \quad \text{in } \Omega,   \\
    \partial _{n} u & = g_{1}(x)   \quad \text{on } \partial \Omega,  \\
    \partial _{n} \Delta  u & = g_{2}( x)   \quad \text{on } \partial \Omega .  \\
\end{split}
\end{equation}
Here is $\Delta ^2 = \Delta  \left( \Delta  \right) $ the biharmonic operator, also known as the bilaplacian. We will assume for the strong form that $u \in H^{4}\left( \Omega  \right) $, $\alpha  \in  \mathbb{R} $ and $f \in L^{2}\left( \Omega  \right)
$. The functions $g_{1},g_{2}: \Omega  \to \mathbb{R}$ are denoted as boundary conditions similar to the CH problem if we do not take account of the non-linearity.

\begin{remark}
It is worth noting that the BH problem is closely related to the Kirchhoff's plate problem by changing the boundary conditions s.t. $u = \partial _{n } u = 0$ on $\Gamma $, which is in the literature known as so-called clamped boundary conditions.
Many of the papers we refer to may consider clamped boundary condition and not the CH boundary conditions. The main difference relies on if the problem is treated with homogeneous or non-homogeneous boundary conditions and if the discrete space is
imposing the Dirichlet and Neumann conditions strongly in the discrete solution space or weakly using the Nitsche's method \cite{nitsche1971variationsprinzip}.
\end{remark}


\subsection{Continuous Interior Penalty formulations}%
\label{sub:continuous_interior_penalty_formulations}

We define the two relevant CIP formulations for the \eqref{eq:bi_problem} as follows. Let $$V_{h} = \left\{ v \in C^{0}\left( \Omega  \right): v_{T} = v | _{T} \in \mathcal{P} _{2}\left( T \right), \forall T \in
\mathcal{T}_{h}    \right\}$$

The general idea is to construct a bilinear problem $a_{h}: V_{h} \times  V_{h} = \mathbb{R} $ and $l_{h}: V^{h} \to \mathbb{R} $ s.t. there exists an $u \in V_{h}$ s.t. $a_{h}( u_{h}, v_{h}) = l_{h}( v) $ for all $v \in V_{h}$. For the CIP
formulation we will investigate two formulations of the forms $a_{h}$ and $l_{h}$.

\begin{enumerate}[label=\arabic*)]
    \item The Hessian problem formulation is,
\begin{equation}
    \label{eq:cip_biharmonic_form}
\begin{split}
a_{h}^{H} \left( u, v \right)   =&
    \left( \alpha  u, v \right) _{\Omega }   +  \left( D^2 u, D^2v \right) _{\mathcal{T} _{h}} \\
 & +
  \left( \mean{  \partial _{n n} u }, \jump{ \partial _{n }u} \right)_{\mathcal{F}_{h}}  +
 \left( \mean{ \partial _{n n} v }, \jump{ \partial _{n}u }      \right)_{\mathcal{F}_{h}}  + \frac{\gamma }{h}  \left( \jump{ \partial _{n} u}, \jump{ \partial _{n} v   }   \right)_{\mathcal{F}_{h}} \\
 l_{h}^{H}( v) & =  \left( f, v \right) _{\Omega }  - \left(g_{2}, v  \right) _{\Gamma } - ( g_{1}, \partial _{nn} v)_{\Gamma } +  \frac{\gamma }{h} ( g_{1} , \partial _{n} v)_{\Gamma }
\end{split}
\end{equation}

With the corresponding energy norms,
\begin{equation}
\label{eq:a_cip_energy_norm}
    \begin{split}
 \| v \|_{ a_{h} }^{ 2 }& =  \| v\|_{ \Omega  }^{2  }  +  \| D ^2 v \|_{ \mathcal{T} _{h}  }^{ 2 }  + \|  h^{-\frac{1}{2}} \jump{ \partial _{n} v    }\|_{  \mathcal{F} _{h} }^{2  },  \quad v \in V_{h}  \\
   \| v \|_{ a_{h},* }^{ 2 } =& \| v \|_{ a_{h} }^{ 2 }  + \| h^{\frac{1}{2}}  \mean{     \partial _{nn } v}  \|_{ \mathcal{F}_{h}   }^{  2}, \quad  v\in V \oplus V_{h}.
    \end{split}
\end{equation}


\item The Laplace formulation is
    \begin{equation}
        \label{eq:cip_laplace_form}
        \begin{split}
            a_{h}^{L} \left( u, v \right)   =&
            \left( \alpha  u, v \right) _{\Omega }   +  \left( \Delta  u, \Delta v \right) _{ \Omega } \\
                                             & + \left( \mean{  \Delta  u }, \jump{ \partial _{n }v} \right)_{\mathcal{F}_{h}  }  + \left( \mean{ \Delta  v }, \jump{ \partial _{n}u }      \right)_{\mathcal{F}_{h}  }  + \frac{\gamma }{h}  \left( \jump{ \partial _{n} u}, \jump{ \partial _{n} v   }   \right)_{\mathcal{F}_{h} } \\
                                             & + \left(   \Delta  u ,  \partial _{n }v \right)_{\Gamma   }  + \left(  \Delta  v ,  \partial _{n}u       \right)_{\Gamma  }  + \frac{\gamma }{h}  \left(  \partial _{n} u,  \partial _{n} v      \right)_{ \Gamma } \\
                                             l^{L}_{h}( v)  =&  \left( f, v \right) _{\Omega } - ( g_{2},  v )_{\Gamma } -  ( g_{1}, \Delta  v  )_{\Gamma }  + \frac{\gamma }{h} ( g_{1}, \partial _{n} v  )_{\Gamma }
                                         \end{split}
                                     \end{equation}
                                     With the corresponding energy norms
                                     \begin{equation}
                                         \label{eq:a_cip_energy_norm}
                                         \begin{split}
                                             \| v \|_{ a_{h} }^{ 2 }& =  \| v\|_{ \Omega  }^{2  }  +  \| \Delta   v \|_{ \mathcal{T} _{h}  }^{ 2 }  + \|  h^{-\frac{1}{2}} \jump{ \partial _{n} v    }\|_{  \mathcal{F} _{h} }^{2  },  \quad v \in V_{h}  \\
                                             \| v \|_{ a_{h},* }^{ 2 } &= \| v \|_{ a_{h} }^{ 2 }  + \| h^{\frac{1}{2}}  \mean{     \partial _{nn } v}  \|_{ \mathcal{F}_{h}   }^{  2}, \quad  v\in V \oplus V_{h}.
                                         \end{split}
                                     \end{equation}


\end{enumerate}


The Hessian formulation is well investigated by Susanne Brenner in several papers for \cite{brenner2012, brenner2012quadratic, brenner2012quadratic_kirk} with a corresponding analysis and numerical validation. Similarly, variants of the Laplace formulation can be found here
\cite{feng2007fully, georgoulis2009discontinuous}. In these article there is good also evidence  that both formulation have the following expected a priori estimates. Let  $u \in H^{s}( \Omega ) $, and $u_{h}\in  V_{h}  $ of order $k$. Then with $r = \min\left\{ s,
k+1 \right\}$ the a priori estimates are   \[
    \begin{split}
\| u - u_{h} \|_{ a_{h},*  }^{  }  & \lesssim  h^{r-1} \| u \|_{ r, \Omega  }^{  } \\
\| u - u_{h} \|_{ \Omega   }^{  }  & \lesssim  h^{r} \| u \|_{ r,\Omega  }^{  }
    \end{split}
\]

\begin{remark}

As the author knows the exact Laplace formulation \eqref{eq:cip_laplace_form} is not found in the literature, but it is expected to have the same well-posedness and convergence properties as the
Hessian formulation. Hence, in this master report will we focus on the Hessian formulation for the analysis. However, for the numerical results will will both be presented.
\end{remark}


\subsection{  Construction for Hessian CIP}%
\label{sub:continious_weak_form_of_biharmonic_equation}

Assume $u \in H^{4}( \Omega ) $ and $V = \left\{ v \in H^{1}( \Omega )  \mid  v \mid _{T} \in H^{m}( T) \ \forall T \in \mathcal{T} _{h}   \right\} $. We will start constructing a local theory for a triangle $K$ and then extend it to the full mesh
$\mathcal{T}_{h} $. Using Greens Theorem is it obvious that \(
\left( \Delta ^2 u,v \right) _{T }   = \left( \partial _{n} \Delta u, v \right) _{\partial T  } - \left( \nabla \left( \Delta  u \right) , \nabla v \right) _{T }
\).
We can expand the second term in the following way.
\begin{equation*}
    \begin{split}
( \nabla ( \Delta u ) , \nabla v ) _{T } & = \sum_{i = 1}^{ d}  ( \Delta  \partial _{x_{i}} u, \partial _{x_{i}}v ) _{T }  = \sum_{i = 1}^{d}  ( \nabla \cdot ( \nabla \partial _{x_{i}} u ) , \partial _{x_{i}} v )_{T }  \\
&= \sum_{i = 1}^{d}  ( \partial_n  \partial _{x_{i}} u, \nabla  \partial _{x_{i}} v ) _{\partial T } -   ( \nabla \partial _{x_{i}} u, \nabla \partial _{x_{i}} v )_{T }
= (  \partial_n\nabla u, \nabla v ) _{\partial_{} T  } - ( D^2 u, D^2v ) _{T } \\
    \end{split}
.\end{equation*}
Hence, the boundary condition of $\Delta u$ is integrated into the formulation.  It can be denoted that $D^2$ is the Hessian matrix operator. Also remark that we apply the notation
$( D^2u, D^2v )_{\Omega } = \int_{\Omega }^{} D^{2}u : D^2v  dx$ for the inner product $D^2u:D^2v$.

We want to decompose the evaluation of $\nabla  u $ on the boundary $\partial T$ in the tangential and normal direction. Pick a facet  $F \in \partial T$, then we define the following decomposition of linear transformation $\nabla u = P_{F}\nabla u  + Q_{F}  \nabla u  $ s.t. the
orthogonality, $
P_{F} \nabla u  \cdot Q_{F}  \nabla u = 0$, holds. The normal projection matrix is defined as $Q_{F} = n \otimes n $ and the tangential decomposition follows from $ P_{F} = I - Q_{F} = I - n \otimes n  =  \sum_{i=1}^{d-1} t_{i} \otimes t_i$, which
is a orthonormal basis $t_{i}$, $i = 1, \ldots, d-1$ for the space orthogonal to the outer normal vector $n$ on a facet $F$. Let $ a_{1}, a_{2}, a_{3} \in R^{d}$ be any vectors, then it is well known that the following identity holds $ ( a_{1}
\otimes a_{2}  ) a_{3} = ( a_{2}^{T}  a_{3}) a_{1} $. Hence, we have
\begin{equation}
\label{eq:projection}
    \begin{split}
   Q_{F} \nabla u & = ( n \otimes n ) \nabla u =  (n^{T} \nabla u)n \\
   P_{F} \nabla u & =( I - n \otimes n ) \nabla u =   \nabla u  - (n^{T}  \nabla u)n =  \sum_{ i =1 }^{d-1} ( t_{i}^{T}  \nabla u ) t_{i}
    \end{split}
\end{equation}

Given that $u$ is evaluated only on $\partial T$ can we write
$\nabla u = \left( n^{T} \nabla u   \right) n + \sum_i^{d-1} \left( t_i^{T} \nabla u   \right) t_i$ s.t.
\[
    \begin{split}
(  \partial_n\nabla u, \nabla v ) _{\partial_{} T  } & =  ( \partial _{n} ( \partial_{n}u \cdot n), \partial _{n} v \cdot n )_{\partial T}   +\sum_{i=1}^{d-1} ( \partial _{n} ( \partial_{t_{i}}u \cdot t_{i}), \partial _{t_{i}} v \cdot t_{i} )_{\partial T} \\
& =  ( \partial _{nn} u ), \partial _{n} v  )_{\partial T}+\sum_{i=1}^{d-1} ( \partial _{n t_{i}}u , \partial _{t_{i}} v  )_{\partial T}
    \end{split}
\]
Here we used that $n^{T} n = 1$ and $t_{i}^{T} t_{i} = 1$.
We applied the simple relation,
    \begin{align*}
\partial_n (\partial_n u)  & = n^T \nabla (\partial_n u)  = n ^T (D^2 u \ n)  = n^{T} D^2 u \ n = \partial _{nn} u, \\
\partial_n (\partial_{t_{i}} u)  & = t_{i}^T \nabla (\partial_n u)  = t_i^T (D^2 u \ n )   = n^{T} D^2 u \ t_{i} = \partial _{n t_{i}} u.
    \end{align*}
We may also deduce the relationship $\partial _{nt_{i}} u = \partial _{t_{i}n}u$ which arise from the fact that $n^{T} D^2u \ t_{i} = ( n^{T} D^2u \ v )^T = t_{i}^{T}  D^2u \  n$, where we utilized the symmetry $D^2u = ( D^2u) ^{T} $.
Adding all these calculations together we have the following local identity,
\[
( \Delta ^2 u, v) _{T}   = ( D^2 u, D^2v)_{T } + ( \partial _{n}  \Delta u, v )_{\partial T} -( \partial _{n} ( \partial_{n}u ), \partial _{n} v  )_{\partial T}-\sum_{i=1}^{d-1} ( \partial _{n} ( \partial_{t_{i}}u ), \partial _{t_{i}} v  )_{\partial T}
\]

For global continuity we add all the triangles in the mesh $\mathcal{T} _{h}$.

\begin{equation}
\label{eq:bi_basic_dg2}
\left( \Delta  ^{2} u,v \right) _{\Omega } = \sum_{T \in  \mathcal{T} _{h}}^{}  ( D^2 u, D^2v)_{T } + ( \partial _{n}  \Delta u, v )_{\partial T} -( \partial _{n} ( \partial_{n}u ), \partial _{n} v  )_{\partial T}-\sum_{i=1}^{d-1} ( \partial _{n t_{i}} u , \partial _{t_{i}} v  )_{\partial T}
\end{equation}
Our goal is to simplify the equation above so we can take account for discontinuities of the derivatives.
By integrating over exterior facets $\mathcal{F} _{h}^{ext}$ and interior facets $\mathcal{F} _{h}^{int}$ we will get e more suitable formulation which makes it easier to control the jumps between the elements, hence makes it possible to penalize discontinuities.

\begin{equation*}
    \begin{split}
 ( \Delta  ^{2} u,v ) _{\Omega }  =&\sum_{T\in \mathcal{T} _{h}}^{} ( D^2u,D^2v ) _{T }  + (\partial _{n} \Delta  u,v)_{\partial T} - (\partial _{nn} u, \partial _{n}v )_{\partial T}  - \sum_{i=1}^{d-1} ( \partial _{t_{i}n}u , \partial _{t_{i}} v  )_{\partial T}   \\
= &\sum_{T\in \mathcal{T} _{h}}^{} ( D^2u,D^2v ) _{T }  + \sum_{F \in \mathcal{F}_{h}^{ext} }^{}  (\partial _{n} \Delta  u,v)_{F} - (\underbrace{\partial _{nn} u}_{ \partial _{n} g_{1}} , \partial _{n}v )_{F}  - \sum_{i=1}^{d-1} ( \underbrace{\partial _{ t_{i}n} u }_{  \partial_{ t_{i} } g_{1},  }, \partial _{t_{i}} v
)_{F}     \\
   &  + \sum_{F \in \mathcal{F} _{h}^{int}}^{} \underbrace{\left( (\partial _{n^{+}} \Delta  u^{+}
        ,v^{+} )_{F}
+ \left(\partial _{n^{-}} \Delta  u^{+} ,v^{-}\right)_{F}  \right)}_{(I)}    \\
    &\quad \quad  -
\underbrace{\left( \left(\partial _{n^{+}n^{+}} u^{+}, v^{+} \right) _{F} + \left(\partial _{n^{-}n^{-}} u^{-}, v^{-}
\right) _{F} \right) }_{(II)} \\
   &  \quad \quad - \sum_{i=1}^{d-1}\underbrace{( (\partial _{n^{+}t_{i}} u^{+}, \partial_{t_{i}} v^{+} )_{F} +  \left(\partial _{n^{-}t_{i}} u^{-},
        \partial_{t_{i}} v^{-}
\right)_{F} ) }_{(III)} \\
    \end{split}
.\end{equation*}

Where integration over all interior facets $ \forall F \in \mathcal{F}_{h}^{int}$ is computed in this way.
\begin{equation*}
    \begin{split}
        (I) &  =    \left(\partial _{n^{+}} \Delta  u^{+} ,v^{+}\right)_{F} +
        \left(\partial _{n^{-}} \Delta  u^{-} ,v^{-}\right)_{F}  \\
            & =   \int_{F}^{}
            \jump{ \partial _{n} \Delta  u \cdot v } =
            \int_{F}^{}
            \mean{ \partial _{n} \Delta  u } \underbrace{\jump{ v }}_{= 0}    + \underbrace{\jump{ \partial _{n} \Delta  u
            }}_{= 0}    \mean{ v } = 0 \\
            (II) &  =     \left(\partial _{n^{+}n^{+}} u^{+}, \partial_{n^{+}} v^{+} \right)_{F} +  \left(\partial _{n^{-}n^{-}} u^{-}, \partial_{n^{-}} v^{-} \right)_{F}    \\
                 &= \int_{F}^{} \jump{ \partial _{nn} u \cdot  \partial_{n} v }   = \int_{F}^{}
                       \mean{ \partial _{nn} u    } \underbrace{\jump{ \partial_{n} v }  }_{\neq 0}    + \underbrace{\jump{ \partial
                               _{nn}  u
                       }}_{= 0}    \mean{ \partial _{n}v } \\
            (III) &  =     \left(\partial _{n^{+}t_{i}} u^{+}, \partial_{t_{i}} v^{+}
                \right)_{F} +  \left(\partial _{n^{-}t_{i}} u^{-}, \partial_{t_{i}} v^{-}
                \right)_{F}   \\
                 &  =   \int_{F}^{}
                 \jump{ \partial _{nt_{i}} u \cdot  \partial_{t_{i}} v } =
                 \int_{F}^{}
                 \mean{ \partial _{nt_{i}} u    } \underbrace{\jump{ \partial_{t_{i}} v }  }_{= 0}    + \underbrace{\jump{ \partial
                         _{nt_{i}}  u
                 }}_{= 0}    \mean{ \partial _{t_{i}}v }  = 0
                   \end{split}
.\end{equation*}
Observe that the cancellations in the term $(I)$ and term $(III)$  appears of the continuity of $v\in V $ and $u\in H^{4}( \Omega ) $ which makes the jumps and derivative jumps zero. On the other hand, the second term $(II)$  is does not vanish since the discontinuity in normal vector for $v \in V$ is a jump. It can also be raised that $\mean{
\partial _{nn} u } = \partial _{nn} u  $ holds by the continuity of $H^{4}( \Omega  ) $.

% \red{
% Remark that we can simplify the exterior boundary integral terms for each $F \in \mathcal{F} ^{ext}_{h}$ using $t_{i}t_{i}^{T}=1$ and $ n^{T} n = 1$, then apply \eqref{eq:projection}},
% \[
% \red{
%     \begin{split}
% ( \partial _{n} g_{1}, \partial _{n} v)_{F} + \sum_{{i} =1}^{d-1} (\partial _{t_{i}} g_{1} , \partial _{t_{i}} v )_{F } & =( n^{T} \nabla  g_{1}n , n^{T}\nabla  v n)_{F} + \sum_{{i} =1}^{d-1} ( t_{i}^{T} \nabla  g_{1} \ t_{i} , t_{i}^{T} \nabla  v \ t_{i} )_{F } \\
%                                                                           & = ( (I - n  \otimes n ) \nabla _{} g_{1}  , (I - n  \otimes n ) \nabla  v  )_{F }
%     \end{split}
% }
% \]
Hence, we have the following identity.
\begin{equation}
\label{eq:bi_basic_dg_full_1}
\red{
\begin{split}
    \left( \Delta  ^{2} u, v \right) _{\Omega }  =&   \left( D^2u, D^2v \right)_{\mathcal{T} _{h}} +  \left(g_{2}, v  \right) _{\mathcal{F}^{ext}_{h} }  -  ( \mean{ \partial _{nn} u }   , \jump{ \partial_{n} v } )_{\mathcal{F}_{h}^{int} } \\
                                                  & - ( \partial _{n} g_{1} , \partial _{n} v)_{\mathcal{F}^{ext}_{h} } - \sum_{i =1  }^{d-1} ( \partial   _{t_{i}} g_{1}  ,  \partial   _{t_{i}}  v  )_{ \mathcal{F}^{ext} _{h}  }
\end{split}
}
\end{equation}


We will now assemble the Hessian CIP formulation. Assume $u,v \in V_{h}$, then we have the linear form
\begin{equation}
\begin{split}
a_{h}^{H} \left( u, v \right)   =&
    \left( \alpha  u, v \right) _{\Omega }   +  \left( D^2 u, D^2v \right) _{\mathcal{T} _{h}} \\
 & +
  \left( \mean{  \partial _{n n} u }, \jump{ \partial _{n }u} \right)_{\mathcal{F}_{h}}  +
 \left( \mean{ \partial _{n n} v }, \jump{ \partial _{n}u }      \right)_{\mathcal{F}_{h}}  + \frac{\gamma }{h}  \left( \jump{ \partial _{n} u}, \jump{ \partial _{n} v   }   \right)_{\mathcal{F}_{h}} \\
 l_{h}^{H}( v_{h}) & =  \left( f, v \right) _{\Omega }  - \left(g_{2}, v  \right) _{\Gamma } - ( g_{1}, \partial _{nn} v)_{\Gamma } +  \frac{\gamma }{h} ( g_{1} , \partial _{n} v)_{\Gamma }
\end{split}
\end{equation}
Hence, we are finished.

\begin{remark}
Remark that we imposed the Neumann condition $g_{1}$ using the Nitsche's method and thus imposing a regularisation term \cite{nitsche1971variationsprinzip}. Observe we added n symmetry term $( \mean{ \partial _{n n} v }, \jump{ \partial _{n}u }      )_{\mathcal{F}_{h}}$,
which is consistent for zero, also involving penalties for the interior facets. Technically is the interior regularisation equivalent to do a Nitsche's method inbetween all interior elements, but with boundary conditions of each triangle weakly
imposed to zero. Thus, the penalty parameter $\gamma$  is the same interior and exterior elements. From \cite{brenner2012quadratic, brenner2012} is it experimentally shown that $\gamma = 2k ( k-1 ) $ where $k$ is the polynomial order $k\ge 2$ may be
a good choice for most applications.
\end{remark}


\subsection{Construction of Laplacian CIP}%
\label{sub:construction_of_laplacian_cip}

 Again, assume $u \in H^{4}( \Omega ) $ and $ v \in  V = \left\{ v \in H^{1}( \Omega )  \mid  v \mid _{T} \in H^{m}( T) \ \forall T \in \mathcal{T} _{h}   \right\} $. Similarly, we start constructing a local theory for a triangle $T$ and then extend it to the full mesh
$\mathcal{T}_{h} $. Using Greens Theorem is it obvious that \(
\left( \Delta ^2 u,v \right) _{T }   = ( \partial _{n} \Delta u, v ) _{\partial T  } - ( \nabla \left( \Delta  u \right) , \nabla v ) _{T }
\).
The main difference here is that we straight away do a new iteration of the Greens theorem.
\begin{equation*}
( \nabla ( \Delta u ) , \nabla v ) _{T }  =  + ( \partial _{n}v , \Delta v)_{\partial T} - ( \Delta u, \Delta v ) _{T}
.\end{equation*}

Hence, we have a local identity s.t. \[
( \Delta ^2 u, v ) _{T} = ( \Delta u, \Delta v) +  ( \partial _{n} \Delta u, v)_{\partial T} - ( \partial _{n} v, \Delta u) _{\partial T}
\]
Now, doing a summation over all elements we get \[
    \begin{split}
( \Delta ^2 u, v ) _{\Omega } & = \sum_{T \in \mathcal{T}_{h} }^{}  ( \Delta u, \Delta v)_{T}
+  ( \partial _{n} \Delta u, v)_{\partial T} - ( \partial _{n} v, \Delta u) _{\partial T} \\
 & =   ( \Delta u, \Delta v)_{\mathcal{T} _{h}} +  \sum_{F \in \mathcal{F}_{h}^{ext} }^{}
  \overbrace{( \partial _{n} \Delta u, v)_{ F}}^{=( g_{2},v)_{F} }  - ( \partial _{n} v, \Delta u) _{F} \\
  &   \quad + \sum_{F \in \mathcal{F}_{h}^{int} }^{} \underbrace{( ( \partial _{n^{+}} \Delta u, v)_{ F} + ( \partial _{n^{-}} \Delta u, v)_{ F} )}_{(I)}  - \underbrace{( ( \partial _{n^{+}} v, \Delta u) _{F} + ( \partial _{n^{-}} v, \Delta u) _{F}
  )}_{(II)}   \\
    \end{split}
\]

Now, decomposing the terms as utilizing the regularity of $u$ and $v$ is it easy to see that.    \[
\begin{split}
    (I) & = ( \partial _{n^{+}} \Delta u, v)_{ F} + ( \partial _{n^{-}} \Delta u, v)_{ F}  = \int_{F}^{} \left[ \partial _{n} \Delta u \cdot  v \right] =  (  \mean{ \partial _{n^{+}} \Delta u } , \underbrace{\jump{v  }}_{ = 0}      )_{ F} + (  \underbrace{\jump{ \partial _{n}
    \Delta u }}_{=0}  , \mean{v  }     )_{ F} \\
    (II) &=  ( \partial _{n^{+}} v, \Delta u) _{F} + ( \partial _{n^{-}} v, \Delta u) _{F} = \int_{F}^{} \jump{ \partial _{n} v, \Delta u } =  ( \underbrace{\jump{ \partial _{n} v}}_{ \neq 0 } , \mean{ \Delta u })_{F}  + ( \mean{ \partial _{n} v}, \underbrace{\jump{ \Delta u
    }}_{=0} )_{F}
\end{split} .
\]

Hence, we end up with the identity,
\[
( \Delta ^2 u, v ) _{\Omega } = ( \Delta u, \Delta v)_{\mathcal{T} _{h} }  +  ( \jump{ \partial _{n} v} , \mean{ \Delta u })_{\mathcal{F}_{h} }  + ( g_{2} , v )_{\Gamma } - ( \partial _{n} v, \Delta u)_{\Gamma }.
\]

Finally, can we construct the Laplace CIP formulation. Let $u,v \in V_{h}$, then can we write the following statement.

\begin{equation}
    \begin{split}
        a_{h}^{L} \left( u, v \right)   =&
        \left( \alpha  u, v \right) _{\Omega }   +  \left( \Delta  u, \Delta v \right) _{ \Omega } \\
                                         & + \left( \mean{  \Delta  u }, \jump{ \partial _{n }v} \right)_{\mathcal{F}_{h}  }  + \left( \mean{ \Delta  v }, \jump{ \partial _{n}u }      \right)_{\mathcal{F}_{h}  }  + \frac{\gamma }{h}  \left( \jump{ \partial _{n} u}, \jump{ \partial _{n} v   }   \right)_{\mathcal{F}_{h} } \\
                                         & + \left(   \Delta  u ,  \partial _{n }v \right)_{\Gamma   }  + \left(  \Delta  v ,  \partial _{n}u       \right)_{\Gamma  }  + \frac{\gamma }{h}  \left(  \partial _{n} u,  \partial _{n} v      \right)_{ \Gamma } \\
        l^{L}_{h}( v_{h}) & =  \left( f, v \right) _{\Omega } - ( g_{2},  v )_{\Gamma } -  ( g_{1}, \Delta  v  )_{\Gamma }  + \frac{\gamma }{h} ( g_{1}, \partial _{n} v  )_{\Gamma }
    \end{split}
\end{equation}


Similarly, as in the previous section, is the boundary condition $g_{1}$ imposed weakly using the Nitsche's method, hence, the regularization term.


\subsection{Note on the Biharmonic Mixed Formulation}%
\label{subsec:biharmonic_mixed_formulation}

It is easy to see that the biharmonic problem can be rewritten into an equivalent mixed formulation , that is, to find $\sigma, \tau  \in H^2( \Omega ) $ s.t. \[
    \begin{split}
\Delta \sigma  & = f \quad  \text{in } \Omega \\
\sigma   & = \Delta u  \text{ in } \Omega \\
\partial _{n} \sigma  & = g_{1} \text{ on } \Gamma  \\
\partial _{n} u   & = g_{2} \text{ on } \Gamma
    \end{split}
\]
The goal is to obtain an useful weak formulation. Using Greens theorem on the first equation we get,
\[
( \sigma, v)_{\Omega } = ( \nabla  u , \nabla v  )_{\Omega } - ( \nabla _{n} u , v) _{\Gamma }.
\]
Similarly for the second equation we obtain
\[
( \nabla \sigma , \nabla \varphi  )_{\Omega} - ( \partial _{n} \sigma ,  \varphi )_{\Gamma } = ( f,\varphi ) _{\Omega}
\]
Putting it all together we have the following mixed weak formulation; Find $( u, \sigma ) \in H^{1}( \Omega ) \times H^{1}( \Omega )  $ s.t. \[
    \begin{split}
     ( \nabla  u , \nabla v  )_{\Omega } -( \sigma, v)_{\Omega }  & =   ( g_{1} , v) _{\Gamma } \quad  \forall v \in H^{1}( \Omega ) \\
( \nabla \sigma , \nabla \varphi  )_{\Omega}  & = ( f,\varphi ) _{\Omega} + ( g_{2} ,  \varphi )_{\Gamma } \quad  \forall \varphi \in H^{1}( \Omega )
    \end{split}
\]
Now we want to relate this formulation to the abstract saddle point problem (SPP) (find references).
Let $V = H^{1}( \Omega ) $  and $W=H^{1}( \Omega ) $ be  Hilbert spaces and define the bilinear form $a: V\times V \to \mathbb{R}  $ and $b: V \times W \to \mathbb{R} $ s.t. $a( \sigma,v ) = - ( \sigma , v) _{\Omega }  $ and $b( u,v) = ( \nabla u,
\nabla v)_{\Omega  }  $. We also may define the linear forms, $G,F: V \to \mathbb{R} $ s.t. $ G( v)  = ( g_{1}, v) _{\Gamma } $ and $F( \varphi ) = ( f, \varphi )_{\Omega } + ( g_{2}, \varphi )_{\Gamma } $.

Hence, we can connect it to the SPP. We want to find $( u,\sigma ) \in V \times W$ s.t.  \[
    \begin{cases}
       a( \sigma ,v) + b ( u, v )  & = G( v)   \quad  \forall v \in V \\
       b( u, \varphi  )  & = F( \varphi )     \quad \forall \phi \in W
    \end{cases}
\]
This is useful since we can now apply standard saddle point theory to do an analysis for the problem. We will see that it is now easier to handle the boundary constraints naturally, but with the cost of a more challenging time discretization
procedure.
For more information about the biharmonic mixed formulation, see \cite{babuvska1980analysis,cai2023nitsche}.
However, in this master thesis is the focus on solving the biharmonic equation avoiding the mixed formulation using the CIP formulation, which does in fact handle the downsides with the SPP problem.


% A well known and mature application for SPP is the well known Stokes equation, hence, a good place to start \cite{john2016finite, knabner2003numerical}.



