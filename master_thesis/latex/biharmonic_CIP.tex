
\newpage
\section{CIP biharmonic problem }%
\label{sec:CIP_biharmonic_problem}


\subsection{Strong form of the biharmonic problem}%
\label{sub:strong_form_of_the_biharmonic_equation}

Let $\Omega \subseteq    \mathbb{R} ^2$ be a bounded polygonal domain and $\partial \Omega $ be its corresponding boundary. Let the inhomogeneous fourth order biharmonic equation have the form,

\begin{equation}
\label{eq:bi_problem}
\begin{split}
    \Delta^2  u  + \alpha  u  & = f( x)  \quad \text{in } \Omega,   \\
    \partial _{n} u & = 0  \quad \text{on } \partial \Omega,  \\
    \partial _{n} \Delta  u & = g(x)  \quad \text{on } \partial \Omega .  \\
\end{split}
\end{equation}
Here is $\Delta ^2 = \Delta  \left( \Delta  \right) $ the biharmonic operator, also known as the bilaplacian. We will assume for the strong form that $u \in H^{4}\left( \Omega  \right) $, $\alpha  \in  \mathbb{R} $ and $f \in L^{2}\left( \Omega  \right)
$. We may consider the functions $g( x ) $ as a time independent boundary conditions. Such problems as \eqref{eq:bi_problem} are often associated with the Cahn-Hilliard model
for phase separation \cite{cahnhilliard1957} .

\subsection{  Weak form biharmonic equation in $H^{4}\left( \Omega  \right) $}%
\label{sub:continious_weak_form_of_biharmonic_equation}


The goal is to find a useful full weak formulation of \eqref{eq:bi_problem}. Now, let the solution space be on the form,
\begin{equation*}
V = \left\{ v \in H^2\left( \Omega  \right) : \partial _{n} v = 0  \text{ on }
\partial \Omega  \right\}.
\end{equation*}

Let $u,v \in  V$, then the derivation of the general weak form is,
\[
\begin{split}
\left( \Delta ^2 u,v \right) _{\Omega }  &  = \left( \partial _{n} \Delta u, v \right) _{\partial \Omega } - \left( \nabla \left( \Delta  u \right) , \nabla v \right) _{\Omega }  \\
\end{split}
\]
In fact, the simplest formulation has the form,
\[
  \left( \nabla \left( \Delta u \right) , \nabla v \right) _{\Omega } =   \left( \Delta u, \partial _{n} v \right) _{\partial \Omega } - \left( \Delta u, \Delta v \right)_{\Omega },
\]
A major issue with this formulation is that we do not have boundary condition for $\Delta u$. Instead, we can expand the term in the following fashion.

\begin{equation*}
    \begin{split}
\left( \nabla \left( \Delta u \right) , \nabla v \right) _{\Omega } & = \sum_{i = 1}^{ d}  \left( \Delta  \partial _{x_{i}} u, \partial _{x_{i}}v \right) _{\Omega }  \\
&= \sum_{i = 1}^{d}  \left( \nabla \cdot \left( \nabla \partial _{x_{i}} u \right) , \partial _{x_{i}} v \right)_{\Omega }  \\
&= \sum_{i = 1}^{d}  \left( \partial_n  \partial _{x_{i}} u, \nabla  \partial _{x_{i}} v \right) _{\partial \Omega} -   \left( \nabla \partial _{x_{i}} u, \nabla \partial _{x_{i}} v \right)_{\Omega }  \\
&= \left(  \partial_n\nabla u, \nabla v \right) _{\partial \Omega } - \left( D^2 u, D^2v \right) _{\Omega } \\
&= \left( \partial _{nn} u, \partial _{n} v  \right)_{\partial \Omega }   + \left( \partial _{nt} u, \partial _{t} v \right) _{\partial \Omega } - \left( D^2u, D^2v \right) _{\Omega } .
    \end{split}
.\end{equation*}
Hence, the boundary condition of $\Delta u$ is integrated into the formulation.  It can be denoted that $D^2$ is the Hessian matrix operator such that
$$( D^2u, D^2v )_{\Omega } = \int_{\Omega }^{} D^{2}u : D^2v  dx,$$
where $D^2u:D^2v$ is the inner product and similarly for $\partial _{nn} u = n\cdot D^2 u \cdot n$. Thus, we now have a weak form identity,
\begin{equation}
\label{eq:weak_form_identity}
\left( \Delta ^2 u, v \right) _{ \Omega } = \left( D^2u, D^2v \right) _{\Omega} +   \left( \partial _{n} \Delta u, v  \right) _{\partial \Omega }  - (\partial _{nn} u, \partial _{n} v )_{\partial \Omega } - \left( \partial _{nt} u, \partial _{t}v
\right) _{\partial \Omega }
.\end{equation}

Using weak form identity \eqref{eq:weak_form_identity} and the boundary conditions stated in the strong form \eqref{eq:bi_problem} can we write

\begin{equation}
\begin{split}
\left( \Delta ^2 u, v \right) _{ \Omega } & = \left( D^2u, D^2v \right) _{\Omega} +   \underbrace{\left( \partial _{n} \Delta u, v  \right) _{\partial \Omega }}_{ = \left( g,v \right) _{\partial \Omega }}   - \underbrace{(\partial _{nn} u, \partial
    _{n} v )_{\partial \Omega }}_{ = 0}  - \underbrace{\left( \partial _{nt} u, \partial _{t}v \right) _{\partial \Omega }}_{ = 0} \\
    &= \left( D^2u, D^2v \right) _{\Omega } + \left( g,v \right) _{\partial \Omega }  \\
\end{split}
.\end{equation}
\todo[inline]{ Is it a way to prove $\partial _{nn} u = 0$ and $\partial _{nt} u = 0$ on $\partial \Omega $? Does this differ from plate problem vs cahn hilliard?}

Finally, we can define the following bilinear functional $a:V\times V \to  \mathbb{R} $ and the linear functional $F: V \to \mathbb{R} $ s.t.
\begin{equation}
\label{eq:weak_formulation}
\begin{split}
a\left( u,v \right)_{\Omega } & =    \left( D ^2 u , D ^2 v\right)_{\Omega }  +
\alpha \left( u, v \right)_{\Omega }   , \\
F\left( v \right)_{\Omega } & = \left( f,v \right)_{\Omega } - \left(g,v \right)_{\partial \Omega }.
\end{split}
\end{equation}

Thus, we have now the necessary definitions to define the biharmonic problem.

We define the biharmonic problem to solve for $u \in V  $ s.t.
\begin{equation}
    \label{eq:bi_weak1}
a\left( u,v \right) = F(v)\quad \forall v \in
V,
\end{equation}
where $ V = \left\{ v \in H^2\left( \Omega  \right) : \partial _{n} v = 0  \text{ on }
\partial \Omega  \right\}$.

A problem that appear in this formulation is that the solution is only unique for $\alpha  > 0$. However, for $\alpha  = 0$ is it necessary to apply the solvability condition,
\begin{equation*}
 \int_{\Omega }^{} f dx = \int_{\partial \Omega }^{} g ds
.\end{equation*}
This condition easily arise when using the substitution $v=1$ in \eqref{eq:bi_weak1}. To handle this, can we extended the solution space \[
V^{*} = \begin{cases}
    V \quad & \alpha  > 0 \\
    \left\{ v \in V: \int_{\Omega }^{} v dx  = 0\right\} \quad & \alpha  = 0,
\end{cases}
\]
Thus, the unique solution in $v \in V^{*}$ belongs to $H^{3 }(\Omega ) $ and we get the following
elliptic regularity estimate \cite{gu2012c0},
\begin{equation*}
\label{eq:bi_harmonic_ellitpic_regularity}
\left| u \right| _{H^{3 }\left( \Omega  \right) }  \le C_{\Omega } \left( \| f \|_{  L^{2}( \Omega ) }^{  } + ( 1 + \alpha  ^{\frac{1}{2}}
) \cdot \| w  \|_{ H^{4}\left( \Omega  \right)  }^{  }    \right) \quad w\in H^{4}\left( \Omega  \right).
\end{equation*}

Finally, we can define the weak formulation of the biharmonic problem.

\begin{definition}[Biharmonic problem]
    \label{def:biharmonics_problem}

We define the biharmonic problem to find $u \in V^*  $ s.t.
\begin{equation}
a\left( u,v \right) = F(v)\quad \forall v \in
V^* .
\end{equation}

\end{definition}

% \subsection{Constructing Continuous Interior Penalty Method}%
% \label{sub:constructing_continious_interior_penalty_method}

%  Let us assume that $u,v \in
% H^{4}\left( T  \right) $. Using that the weak form identity \eqref{eq:weak_form_identity} also holds for a triangle $T$ can we write
% \begin{equation}
% \label{eq:bi_basic_dg}
% \left( \Delta  ^{2} u,v \right) _{T} =  \left( D^2u,D^2v \right) _{T } - \left(\partial _{nt} u, \partial _{t}v
% \right)_{\partial T} - \left(\partial _{nn} u, \partial _{n}v \right)_{\partial T} + \left(\partial _{n} \Delta  u,v
% \right)_{\partial T}
% .\end{equation}
% For global continuity, let  $v \in V =  \left\{ v \in H^{1}\left( \Omega  \right): v_{T} \in  H^{4}\left( T \right), \ \forall T \in
% \mathcal{T}_{h}    \right\} $ and $u \in  H^{4}\left( \Omega  \right) $ such that,

% \begin{equation}
% \label{eq:bi_basic_dg2}
% \left( \Delta  ^{2} u,v \right) _{\Omega } = \sum_{T \in  \mathcal{T} _{h}}^{}  \left( D^2u,D^2v \right) _{T } - \left(\partial _{nt} u, \partial _{t}v
% \right)_{\partial T} - \left(\partial _{nn} u, \partial _{n}v \right)_{\partial T} + \left(\partial _{n} \Delta  u,v
% \right)_{\partial T}.
% \end{equation}
% However, this expression can be written to distinguish integrating over triangles $\mathcal{T} _{h}$ , integrating over exterior facets $\mathcal{F} _{h}^{ext}$ and then integrate interior facets $\mathcal{F} _{h}^{int}$.

% \begin{equation}
% \label{eq:bi_basic_dg_full_1}
% \begin{split}
% \left( \Delta  ^{2} u, v \right) _{\Omega } =& \sum_{T \in  \mathcal{T} _{h}}^{} \left( D^2u, D^2v \right)_{T}    \\
% & + \sum_{F \in \mathcal{F}_{h}^{ext}}  \left(\partial _{n} \Delta u, v  \right) _{F} - \left(\partial _{nt} u, \partial _{t} v \right) _{F}-
% \left( \partial _{nn} u, \partial _{n} v \right)_{F}  \\
% & + \sum_{F \in \mathcal{F}_{h}  ^{int}}^{} \left(\partial _{nn} u , \jump{ \partial _{n} v }
% \right)_{F} \\
% & = \sum_{T \in  \mathcal{T} _{h}}^{} \left( D^2u, D^2v \right)_{T} + \sum_{F \in
% \mathcal{F} ^{ext}_{h}}^{} \left(g, v  \right) _{F}
%   + \sum_{F \in \mathcal{F}_{h}  ^{int}}^{} \left( \partial _{nn} u , \jump{ \partial_{n} v } \right)_{F}
% \end{split}
% \end{equation}
% Keep in mind that any jump over a interior facet $F \subset \mathcal{F} _{h}^{int}   $, visualized in figure \ref{fig:normal}, is defined as $\jump{ a } =    a^{+} - a^{-} $
% and likewise for the mean, $\mean{ a  } = \frac{1}{2}(   a^{+}
% + a^{-})$.    The equivalence of \eqref{eq:bi_basic_dg2} and \eqref{eq:bi_basic_dg_full_1} comes from the following argumentation.

% \begin{equation*}
%     \begin{split}
%  \left( \Delta  ^{2} u,v \right) _{\Omega } & =\sum_{T\in \mathcal{T} _{h}}^{} \left( D^2u,D^2v \right) _{T } - \left(\partial _{nt} u, \partial _{t}v
% \right)_{\partial T} - \left(\partial _{nn} u, \partial _{n}v \right)_{\partial T} + \left(\partial _{n} \Delta  u,v
% \right)_{\partial T} \\
% &= \sum_{T\in \mathcal{T} _{h}}^{} \left( D^2u,D^2v \right) _{T } \\
% &  \quad + \sum_{F \in \mathcal{F}_{h}^{ext} }^{} \underbrace{\left( \partial _{n} \Delta  u, v  \right)_{F}}_{= \left( g,v \right)_{F} }  -  \left(
% \partial _{nt} u, \partial _{t} v \right) _{F}  - \underbrace{\left( \partial _{nn} u, \partial _{n} v \right)_{F}}_{ = 0}    \\
% & \quad  + \sum_{F \in \mathcal{F} _{h}^{int}}^{} \underbrace{\left( \left(\partial _{n^{+}} \Delta  u^{+}
%         ,v^{+}\right)_{F}
% + \left(\partial _{n^{-}} \Delta  u^{+} ,v^{-}\right)_{F}  \right)}_{(I)} \\
%  & \quad \quad \quad  \quad +
% \underbrace{\left( \left(\partial _{n^{+}t} u^{+}, \partial_{t} v^{+} \right)_{F} +  \left(\partial _{n^{-}t} u^{-},
%         \partial_{t} v^{-}
% \right)_{F}  \right) }_{(II)} \\
%  & \quad \quad \quad  \quad  +
% \underbrace{\left( \left(\partial _{n^{+}n^{+}} u^{+}, v^{+} \right) _{F} + \left(\partial _{n^{-}n^{-}} u^{-}, v^{-}
% \right) _{F} \right) }_{(III)}
%     \end{split}
% .\end{equation*}

% Where integration over all interior facets $ \forall F \in \mathcal{F}_{h}^{int}$ is computed in this way.
% \begin{equation*}
%     \begin{split}
%         (I) &  =    \left(\partial _{n^{+}} \Delta  u^{+} ,v^{+}\right)_{F} +
%         \left(\partial _{n^{-}} \Delta  u^{-} ,v^{-}\right)_{F}  \\
%         & =   \int_{F}^{}
%         \jump{ \partial _{n} \Delta  u \cdot v } =
%          \int_{F}^{}
%          \mean{ \partial _{n} \Delta  u } \underbrace{\jump{ v }}_{= 0}    + \underbrace{\jump{ \partial _{n} \Delta  u
%          }}_{= 0}    \mean{ v } = 0 \\
%         (II) &  =     \left(\partial _{n^{+}t} u^{+}, \partial_{t} v^{+}
%         \right)_{F} +  \left(\partial _{n^{-}t} u^{-}, \partial_{t} v^{-}
% \right)_{F}   \\
% &  =   \int_{F}^{}
%         \jump{ \partial _{nt} u \cdot  \partial_{t} v } =
%          \int_{F}^{}
%          \mean{ \partial _{nt} u    } \underbrace{\jump{ \partial_{t} v }  }_{= 0}    + \underbrace{\jump{ \partial
%                  _{nt}  u
%          }}_{= 0}    \mean{ \partial _{t}v }  = 0\\
%         (III) &  =     \left(\partial _{n^{+}n^{+}} u^{+}, \partial_{n^{+}} v^{+} \right)_{F} +  \left(\partial _{n^{-}n^{-}} u^{-}, \partial_{n^{-}} v^{-} \right)_{F}    =    \int_{F}^{} \jump{ \partial _{nn} u \cdot  \partial_{n} v }  \\
%         & = \int_{F}^{}
%         \mean{ \partial _{nn} u    } \underbrace{\jump{ \partial_{n} v }  }_{\neq 0}    + \underbrace{\jump{ \partial
%                  _{nn}  u
%          }}_{= 0}    \mean{ \partial _{n}v }   =  \left( \partial _{nn} u, \jump{ \partial_{n} v } \right)_{F}   \end{split}
% .\end{equation*}
% Observe that the cancellations in the term $(I)$ appears of the continuity of $v\in V $ and $u\in H^{4}\left( \Omega  \right) $ which makes the jumps zero. For the second term $(II)$ does the terms become zero cancelled because the tangential
% derivative at the facet has no jump. However, The third term $(III)$  is fairly interesting since the discontinuity in
% normal vector for $v \in V$ is a jump, while the second term is still continuous. It can also be raised that $\mean{
% \partial _{nn} u } = \partial _{nn} u  $ holds by the continuity of $H^{4}\left( \Omega  \right) $. Hence,
% \eqref{eq:bi_basic_dg2} and \eqref{eq:bi_basic_dg_full_1} is equivalent.

% \subsection{Formulation of the Continious Interior Penalty Method}%
% \label{sub:formulation_of_continious_interior_penalty_method}


% We can finally start defining the fully discrete formulation. Let the basis be a $\mathcal{P}_{2} $ Lagrange finite element space so,
% \[
% V_{h} = \left\{ v \in C^{0}\left( \Omega  \right): v_{T} = v | _{T} \in \mathcal{P} _{2}\left( T \right), \forall T \in
% \mathcal{T}_{h}    \right\}
% \]
% and
% \[
% V_{h}^{*} = \begin{cases}
%     V_{h} & \text{ if } \alpha  > 0 \\
%     \left\{ v \in V_{h}: \int_{\Omega }^{} v dx   = 0   \right\} &  \text{ if } \alpha   = 0
% \end{cases}
% \]
% Now, if we choose $u \in V_{h}$, then we must take account that the jump is discrete.
%  Finally, the CIP formulation can be stated as follows.
% The discretized numerical problem is to solve $w_{h} \in V_{h}^{*}$ such that
% \begin{equation}
% \label{eq:CP_A_F}
% \mathcal{A}\left( w_{h}, v_{h} \right)   = F\left( v_{h} \right), \quad \forall v_{h} \in V_{h}^{*}  .
% \end{equation}
% where
% \begin{equation}
% \label{eq:CP_A_h_1}
% \begin{split}
% \mathcal{A} \left( w_{h}, v_{h} \right)   =&
%   \quad  \left( \alpha  w_{h}, v_{h} \right) _{\Omega }\\
% &  + \sum_{T \in \mathcal{T} _{h}}^{} \left( D^2 w_{h}, D^2v_{h} \right) _{T} \\
%  & +
%   \sum_{F \in \mathcal{F}_{h}^{int} }^{}
%   \left( \mean{  \partial _{n n} w_{h} }, \jump{ \partial _{n }v_{h}} \right)_{F}  +
%  \left( \mean{ \partial _{n n} v_{h} }, \jump{ \partial _{n}w }      \right)_{F} \\
% & \quad \quad \quad \quad  + \frac{\gamma}{h}  \left( \jump{ \partial _{n} w_{h}}, \jump{ \partial _{n} v_{h}   }   \right)_{F}
% \end{split}
% \end{equation}
% and
% \begin{equation}
% \label{eq:CP_F_h}
% F\left( v_{h} \right)  = \left( f, v_{h} \right) _{\Omega } +  \sum_ {F \in \mathcal{F}_{h} ^{ext}}^{} - \left(g, v_{h}  \right) _{F}.
% \end{equation}
% Notice that the regulation term determined by respectively a global tuning parameter $\gamma >0 $. Another key component to the formulation
% in \eqref{eq:CP_A_h_1} after introduction of $ w_{h}, v_{h} \in V^{*}_{h}$  is that we expanded $\left( \partial _{nn}w, \jump{ \partial _{n} v }  \right)_{F} \to \left( \mean{ \partial _{nn}w_{h} }  , \jump{ \partial _{n} v_{h} }  \right)_{F} $ since we can longer not guarantee a
% continuous jump. For symmetric purposes we also added $ \left( \mean{ \partial _{nn} v_{h}}  , \jump{ \partial _{n} w_{h} }  \right)_{F} $. For convenience will we introduce the compact notation of \eqref{eq:CP_A_h_1},

% \begin{equation}
% \label{eq:CP_A_h}
% \begin{split}
% \mathcal{A} \left( w_{h}, v_{h} \right)   =&
%   \quad  \left( \alpha  w_{h}, v_{h} \right) _{\Omega }\\
% &  +  \left( D^2 w_{h}, D^2v_{h} \right) _{\mathcal{T} _{h}} \\
%  & +
%   \left( \mean{  \partial _{n n} w_{h} }, \jump{ \partial _{n }v_{h}} \right)_{\mathcal{F}_{h}}  +
%  \left( \mean{ \partial _{n n} v_{h} }, \jump{ \partial _{n}w }      \right)_{\mathcal{F}_{h}}
%  \\
%  & + \frac{\gamma }{h}  \left( \jump{ \partial _{n} w_{h}}, \jump{ \partial _{n} v_{h}   }   \right)_{\mathcal{F}_{h}} \\
% \end{split}
% .
% \end{equation}

% \todo[inline]{ I think I need to work on how $ \mathcal{A}$ incorporates the $\partial _{n} v \mid _{\partial \Omega } = 0$ boundary condition.   }


\subsection{Constructing Continuous Interior Penalty Method}%
\label{sub:constructing_continious_interior_penalty_method}

 Let us assume that $u,v \in
H^{4}\left( T  \right) $. Using that the weak form identity \eqref{eq:weak_form_identity} also holds for a triangle $T$ can we write
\begin{equation}
\label{eq:bi_basic_dg}
\left( \Delta  ^{2} u,v \right) _{T} =  \left( D^2u,D^2v \right) _{T } - \left(\partial _{nt} u, \partial _{t}v
\right)_{\partial T} - \left(\partial _{nn} u, \partial _{n}v \right)_{\partial T} + \left(\partial _{n} \Delta  u,v
\right)_{\partial T}
.\end{equation}
For global continuity, let  $v \in V =  \left\{ v \in H^{1}\left( \Omega  \right): v_{T} \in  H^{4}\left( T \right), \ \forall T \in
\mathcal{T}_{h}    \right\} $ and $u \in  H^{4}\left( \Omega  \right) $ such that,

\begin{equation}
\label{eq:bi_basic_dg2}
\left( \Delta  ^{2} u,v \right) _{\Omega } = \sum_{T \in  \mathcal{T} _{h}}^{}  \left( D^2u,D^2v \right) _{T } - \left(\partial _{nt} u, \partial _{t}v
\right)_{\partial T} - \left(\partial _{nn} u, \partial _{n}v \right)_{\partial T} + \left(\partial _{n} \Delta  u,v
\right)_{\partial T}.
\end{equation}
However, this expression can be written to distinguish integrating over triangles $\mathcal{T} _{h}$ , integrating over exterior facets $\mathcal{F} _{h}^{ext}$ and then integrate interior facets $\mathcal{F} _{h}^{int}$.

\begin{equation}
\label{eq:bi_basic_dg_full_1}
\begin{split}
\left( \Delta  ^{2} u, v \right) _{\Omega } =& \sum_{T \in  \mathcal{T} _{h}}^{} \left( D^2u, D^2v \right)_{T}    \\
& + \sum_{F \in \mathcal{F}_{h}^{ext}}  \left(\partial _{n} \Delta u, v  \right) _{F} - \left(\partial _{nt} u, \partial _{t} v \right) _{F}-
\left( \partial _{nn} u, \partial _{n} v \right)_{F}  \\
& + \sum_{F \in \mathcal{F}_{h}  ^{int}}^{} \left(\partial _{nn} u , \jump{ \partial _{n} v }
\right)_{F} \\
& = \sum_{T \in  \mathcal{T} _{h}}^{} \left( D^2u, D^2v \right)_{T} + \sum_{F \in
\mathcal{F} ^{ext}_{h}}^{} \left(g, v  \right) _{F}
  + \sum_{F \in \mathcal{F}_{h}  ^{int}}^{} \left( \partial _{nn} u , \jump{ \partial_{n} v } \right)_{F}
\end{split}
\end{equation}
Keep in mind that any jump over a interior facet $F \subset \mathcal{F} _{h}^{int}   $, visualized in figure \ref{fig:normal}, is defined as $\jump{ a } =    a^{+} - a^{-} $
and likewise for the mean, $\mean{ a  } = \frac{1}{2}(   a^{+}
+ a^{-})$.    The equivalence of \eqref{eq:bi_basic_dg2} and \eqref{eq:bi_basic_dg_full_1} comes from the following argumentation.

\begin{equation*}
    \begin{split}
 \left( \Delta  ^{2} u,v \right) _{\Omega } & =\sum_{T\in \mathcal{T} _{h}}^{} \left( D^2u,D^2v \right) _{T } - \left(\partial _{nt} u, \partial _{t}v
\right)_{\partial T} - \left(\partial _{nn} u, \partial _{n}v \right)_{\partial T} + \left(\partial _{n} \Delta  u,v
\right)_{\partial T} \\
&= \sum_{T\in \mathcal{T} _{h}}^{} \left( D^2u,D^2v \right) _{T } \\
&  \quad + \sum_{F \in \mathcal{F}_{h}^{ext} }^{} \underbrace{\left( \partial _{n} \Delta  u, v  \right)_{F}}_{= \left( g,v \right)_{F} }  -  \left(
\partial _{nt} u, \partial _{t} v \right) _{F}  - \underbrace{\left( \partial _{nn} u, \partial _{n} v \right)_{F}}_{ = 0}    \\
& \quad  + \sum_{F \in \mathcal{F} _{h}^{int}}^{} \underbrace{\left( \left(\partial _{n^{+}} \Delta  u^{+}
        ,v^{+}\right)_{F}
+ \left(\partial _{n^{-}} \Delta  u^{+} ,v^{-}\right)_{F}  \right)}_{(I)} \\
 & \quad \quad \quad  \quad +
\underbrace{\left( \left(\partial _{n^{+}t} u^{+}, \partial_{t} v^{+} \right)_{F} +  \left(\partial _{n^{-}t} u^{-},
        \partial_{t} v^{-}
\right)_{F}  \right) }_{(II)} \\
 & \quad \quad \quad  \quad  +
\underbrace{\left( \left(\partial _{n^{+}n^{+}} u^{+}, v^{+} \right) _{F} + \left(\partial _{n^{-}n^{-}} u^{-}, v^{-}
\right) _{F} \right) }_{(III)}
    \end{split}
.\end{equation*}

Where integration over all interior facets $ \forall F \in \mathcal{F}_{h}^{int}$ is computed in this way.
\begin{equation*}
    \begin{split}
        (I) &  =    \left(\partial _{n^{+}} \Delta  u^{+} ,v^{+}\right)_{F} +
        \left(\partial _{n^{-}} \Delta  u^{-} ,v^{-}\right)_{F}  \\
        & =   \int_{F}^{}
        \jump{ \partial _{n} \Delta  u \cdot v } =
         \int_{F}^{}
         \mean{ \partial _{n} \Delta  u } \underbrace{\jump{ v }}_{= 0}    + \underbrace{\jump{ \partial _{n} \Delta  u
         }}_{= 0}    \mean{ v } = 0 \\
        (II) &  =     \left(\partial _{n^{+}t} u^{+}, \partial_{t} v^{+}
        \right)_{F} +  \left(\partial _{n^{-}t} u^{-}, \partial_{t} v^{-}
\right)_{F}   \\
&  =   \int_{F}^{}
        \jump{ \partial _{nt} u \cdot  \partial_{t} v } =
         \int_{F}^{}
         \mean{ \partial _{nt} u    } \underbrace{\jump{ \partial_{t} v }  }_{= 0}    + \underbrace{\jump{ \partial
                 _{nt}  u
         }}_{= 0}    \mean{ \partial _{t}v }  = 0\\
        (III) &  =     \left(\partial _{n^{+}n^{+}} u^{+}, \partial_{n^{+}} v^{+} \right)_{F} +  \left(\partial _{n^{-}n^{-}} u^{-}, \partial_{n^{-}} v^{-} \right)_{F}    =    \int_{F}^{} \jump{ \partial _{nn} u \cdot  \partial_{n} v }  \\
        & = \int_{F}^{}
        \mean{ \partial _{nn} u    } \underbrace{\jump{ \partial_{n} v }  }_{\neq 0}    + \underbrace{\jump{ \partial
                 _{nn}  u
         }}_{= 0}    \mean{ \partial _{n}v }   =  \left( \partial _{nn} u, \jump{ \partial_{n} v } \right)_{F}   \end{split}
.\end{equation*}
Observe that the cancellations in the term $(I)$ appears of the continuity of $v\in V $ and $u\in H^{4}\left( \Omega  \right) $ which makes the jumps zero. For the second term $(II)$ does the terms become zero cancelled because the tangential
derivative at the facet has no jump. However, The third term $(III)$  is fairly interesting since the discontinuity in
normal vector for $v \in V$ is a jump, while the second term is still continuous. It can also be raised that $\mean{
\partial _{nn} u } = \partial _{nn} u  $ holds by the continuity of $H^{4}\left( \Omega  \right) $. Hence,
\eqref{eq:bi_basic_dg2} and \eqref{eq:bi_basic_dg_full_1} is equivalent.

\subsection{Formulation of the Continious Interior Penalty Method}%
\label{sub:formulation_of_continious_interior_penalty_method}


We can finally start defining the fully discrete formulation. Let the basis be a $\mathcal{P}_{2} $ Lagrange finite element space so,
\[
V_{h} = \left\{ v \in C^{0}\left( \Omega  \right): v_{T} = v | _{T} \in \mathcal{P} _{2}\left( T \right), \forall T \in
\mathcal{T}_{h}    \right\}
\]
and
\[
V_{h}^{*} = \begin{cases}
    V_{h} & \text{ if } \alpha  > 0 \\
    \left\{ v \in V_{h}: \int_{\Omega }^{} v dx   = 0   \right\} &  \text{ if } \alpha   = 0
\end{cases}
\]
Now, if we choose $u \in V_{h}$, then we must take account that the jump is discrete.
 Finally, the CIP formulation can be stated as follows.
The discretized numerical problem is to solve $w_{h} \in V_{h}^{*}$ such that
\begin{equation}
\label{eq:CP_A_F}
\mathcal{A}\left( w_{h}, v_{h} \right)   = F\left( v_{h} \right), \quad \forall v_{h} \in V_{h}^{*}  .
\end{equation}
where
\begin{equation}
\label{eq:CP_A_h_1}
\begin{split}
\mathcal{A} \left( w_{h}, v_{h} \right)   =&
  \quad  \left( \alpha  w_{h}, v_{h} \right) _{\Omega }\\
&  + \sum_{T \in \mathcal{T} _{h}}^{} \left( D^2 w_{h}, D^2v_{h} \right) _{T} \\
 & +
  \sum_{F \in \mathcal{F}_{h}^{int} }^{}
  \left( \mean{  \partial _{n n} w_{h} }, \jump{ \partial _{n }v_{h}} \right)_{F}  +
 \left( \mean{ \partial _{n n} v_{h} }, \jump{ \partial _{n}w }      \right)_{F} \\
& \quad \quad \quad \quad  + \frac{\gamma}{h}  \left( \jump{ \partial _{n} w_{h}}, \jump{ \partial _{n} v_{h}   }   \right)_{F}
\end{split}
\end{equation}
and
\begin{equation}
\label{eq:CP_F_h}
F\left( v_{h} \right)  = \left( f, v_{h} \right) _{\Omega } +  \sum_ {F \in \mathcal{F}_{h} ^{ext}}^{} - \left(g, v_{h}  \right) _{F}.
\end{equation}
Notice that the regulation term determined by respectively a global tuning parameter $\gamma >0 $. Another key component to the formulation
in \eqref{eq:CP_A_h_1} after introduction of $ w_{h}, v_{h} \in V^{*}_{h}$  is that we expanded $\left( \partial _{nn}w, \jump{ \partial _{n} v }  \right)_{F} \to \left( \mean{ \partial _{nn}w_{h} }  , \jump{ \partial _{n} v_{h} }  \right)_{F} $ since we can longer not guarantee a
continuous jump. For symmetric purposes we also added $ \left( \mean{ \partial _{nn} v_{h}}  , \jump{ \partial _{n} w_{h} }  \right)_{F} $. For convenience will we introduce the compact notation of \eqref{eq:CP_A_h_1},

\begin{equation}
\label{eq:CP_A_h}
\begin{split}
\mathcal{A} \left( w_{h}, v_{h} \right)   =&
  \quad  \left( \alpha  w_{h}, v_{h} \right) _{\Omega }\\
&  +  \left( D^2 w_{h}, D^2v_{h} \right) _{\mathcal{T} _{h}} \\
 & +
  \left( \mean{  \partial _{n n} w_{h} }, \jump{ \partial _{n }v_{h}} \right)_{\mathcal{F}_{h}}  +
 \left( \mean{ \partial _{n n} v_{h} }, \jump{ \partial _{n}w }      \right)_{\mathcal{F}_{h}}
 \\
 & + \frac{\gamma }{h}  \left( \jump{ \partial _{n} w_{h}}, \jump{ \partial _{n} v_{h}   }   \right)_{\mathcal{F}_{h}} \\
\end{split}
.
\end{equation}

\subsection{ Stability Results}%
\label{sub:error_and_stability_analysis_of_c0ip}

To guarantee convergence and stability we may want to check coercivity and boundedness of the method.

First of all, let us now establish some important inequalities.
\[
\begin{split}
    \textbf{Cauchy-Schwarz inequality: } & \| ab \|_{  }^{  }  \le \| a \|_{  }^{  } \| b \|_{  }^{  }   \\
    \textbf{Inverse inequality: } & \frac{1}{h}\| \partial _{nn}  v_{h} \|_{\mathcal{F}_{h}   }^{2  }  \le C_{j} \| D ^2 v_{h} \|_{ \mathcal{T} _{h} }^{ 2 }   \\
    \textbf{Youngs epsilon inequality: } & 2ab =   2\sqrt{\varepsilon }a\cdot    \frac{b}{\sqrt{\varepsilon } } \le \varepsilon a^2+ b^2 \frac{1}{\varepsilon }
\end{split}
\]

Let the energy norm be on the form,
\begin{equation}
\label{eq:A_energy_norm}
    \begin{split}
\| v_{h} \|_{ h }^{2  } & = \| v_{h} \|_{ a_{h} }^{ 2 } =  \| v_{h} \|_{ \Omega  }^{2  }  +  \| D ^2 v_{h} \|_{ \mathcal{T} _{h}  }^{ 2 }  + \|  h^{-\frac{1}{2}} \jump{ \partial _{n} v_{h}    }\|_{  \mathcal{F} _{h} }^{2  }, \\
\| v \|_{ h }^{ 2 }  &= \| v \|_{ a_{h},* }^{ 2 } = \| v \|_{ a_{h} }^{ 2 }  + \| h^{\frac{1}{2}} \left\{ \partial _{nn } v\right\}  \|_{ \mathcal{F}_{h}   }^{  2}, \quad  v\in V \oplus V_{h}.
    \end{split}
\end{equation}
The method is said to be coercive if $\mathcal{A} _{h}\left( v_{h}, v_{h} \right) \ge  C \| v_{h} \|_{ a_{h} }^{  } $. Similarly, it is bounded if $ \mathcal{A} _{h} \left( v_{h}, u_{h} \right) \le  C \| u_{h} \|_{  a_{h}}^{ 2 }  \| v_{h} \|_{ a_{h}
}^{ 2 } $ and then, according to Lax Milgram the problem is said to be well posed.

\subsubsection{Coercivity}%
\label{ssub:coercitivity}


Suppose we have the CIP problem described in \eqref{eq:CP_A_F}. Then is the coercivity be computed such that,
\[
    \begin{split}
\mathcal{A} \left( v_{h}, v_{h} \right)  =& \quad  \alpha \|  v_{h}  v_{h} \|_{ \Omega  }^{  } +  \| D^2v_{h} \|_{ \mathcal{T} _{h} }^{2  }   \\
& \quad + 2 \left(  \mean{ \partial _{nn} v_{h} }    ,  \jump{ \partial _{n}v_{h} }     \right) _{\mathcal{F} _{h}} +  \frac{\gamma}{h} \|  \jump{ \partial _{n} v_{h} }
  \|_{ \mathcal{F} _{h} }^{ 2 } \\
\quad \textit{Cauchy-Schwarz inequality} \quad
 \ge& \quad  \alpha \| v_{h}  \|_{\Omega   }^{  } \| v_{h} \|_{\Omega   }^{  } +   \| D ^2 v_{h} \|_{ \mathcal{T} _{h} }^{2  } \\
& \quad -2 \| h^{\frac{1}{2}} \mean{ \partial _{nn}v_{h} }    \|_{  \mathcal{F} _{h}}^{  } \| h^{-\frac{1}{2}} \jump{ \partial _{n}v_{h} }    \|_{  \mathcal{F} _{h}}^{  } + \gamma \| h^{-\frac{1}{2}}  \jump{ \partial _{n}v_{h} }   \|_{ \mathcal{F} _{h}  }^{ 2 } \\
\quad \textit{Inverse inequality} \quad
   \ge & \quad  \alpha \| v_{h}  \|_{\Omega   }^{  } \| v_{h} \|_{\Omega   }^{  } + \| D ^2 v_{h}  \|_{ \mathcal{T} _{h}  }^{ 2  }  \\
 &  \quad - 2 C^{\frac{1}{2}}_{j} \|   D ^2 v_{h}    \|_{ \mathcal{T} _{h}  }^{  } \| h^{-\frac{1}{2}} \jump{ \partial _{n} v_{h} }   \|_{ \mathcal{F} _{h} }^{  }  + \gamma \| h^{ -\frac{1}{2}} \jump{
 \partial _{n } v_{h}}   \|_{ \mathcal{F}_{h}}^{2}  \\
\quad \textit{ Youngs epsilon inequality} \quad
    \ge  &  \quad  \alpha \| v_{h}  \|_{\Omega   }^{  } \| v_{h} \|_{\Omega   }^{  } +  \| D ^2 v_{h} \|_{ \mathcal{T}_{h}  }^{2  } - \varepsilon C_{j} \| D ^2 v_{h} \|_{ \mathcal{T} _{h} }^{2  } \\
  & \quad  - \frac{1}{\varepsilon } \| h^{\frac{1}{2}} \jump{ \partial _{n} v_{h} }   \|_{ \mathcal{F} _{h} }^{2  }  + \gamma \|
  h^{-\frac{1}{2}} \jump{ \partial _{n} v_{h}}   \|_{ \mathcal{F} _{h} }^{2  }  \\
   =&  \quad  \alpha \| v_{h}  \|_{\Omega   }^{  } \| v_{h} \|_{\Omega   }^{  } +\left( 1 - \varepsilon C_{j} \right) \| D ^2 v_{h} \|_{\mathcal{T} _{h}  }^{ 2 }  \\
  & \quad + \left( \gamma  - \frac{1}{\varepsilon } \right) \| h^{-\frac{1}{2}} \jump{ \partial _{n} v_{h} }   \|_{ \mathcal{T} _{h} }^{ 2 } \\
  (\varepsilon  = \frac{1}{2 C_{j} })  \implies  \quad \quad =& \quad  \alpha \| v_{h}  \|_{\Omega   }^{  } \| v_{h} \|_{\Omega   }^{  } +\frac{1}{2} \| D ^2 v_{h} \|_{ \mathcal{T} _{h} }^{ 2 }  + \underbrace{\left( \gamma -2 C_{j} \right)}_{ \ge  \frac{1}{2}}  \| h^{\frac{1}{2}} \jump{ \partial _{n} v_{h} }   \|_{
  \mathcal{F} _{h} }^{2  } \\
   \ge & \quad  C \| v_{h} \|_{ a_{h} }^{  2}
    \end{split}
\]
This holds if $C=\min\left\{  \alpha , 1 /2\right\}$.
Observe that for the first inequality is the standard \textbf{Cauchy-Schwarz inequality} such that $$\left( \mean{ \partial_{nn} v_{h} }  , \jump{ \partial _{n} v_{h} }   \right) _{\mathcal{F} _{h}} \ge - \| h^{-\frac{1}{2}} \mean{ \partial _{nn}
v_{h} }    \|_{\mathcal{F}_{h}   }^{  } \| \mean{ \partial _{n}v_{h} }   \|_{ \mathcal{F}_{h}   }^{  } .  $$ On the second inequality the \textbf{Inverse inequality} was applied,
\[
- \| h^{\frac{1}{2}} \mean{ \partial _{nn}v }   \|_{ \mathcal{F} _{h}  }^{  }\ge - C_{j}^{\frac{1}{2}} \| D ^2 v_{h} \|_{ \mathcal{T} _{h} }^{  }
\]
The next step is then to use the \textbf{Youngs epsilon inequality} to separate the facets and triangulation norms, \[
 - 2 C^{\frac{1}{2}}_{j} \|  D ^2 v_{h}    \|_{ \mathcal{T} _{h}  }^{  } \| h^{\frac{1}{2}} \jump{ \partial _{n} v_{h} }   \|_{ \mathcal{F} _{h} }^{  } \ge- \varepsilon C_{j} \| D ^2 v_{h} \|_{ \mathcal{T} _{h} }^{2  } -
 \frac{1}{\varepsilon } \| h^{\frac{1}{2}} \jump{ \partial _{n} v_{h} }   \|_{ \mathcal{F} _{h} }^{2  }
.\]
The last step was to choose a $\varepsilon $ and $\gamma $ as some positive constant so that the second term is restricted to be multiplied with something bigger than $\frac{1}{2}$. Thus, the term fulfils coercivity of the \eqref{eq:A_energy_norm}.
Hence, the CIP method is coercive.

\subsubsection{Boundedness}%
\label{ssub:bounded}
We want the CIP method to be bounded.


\begin{equation*}
    \begin{split}
\mathcal{A} \left( w_{h}, v_{h} \right)   =& \quad \left( \alpha w_{h}, v_{h} \right) _{\Omega } +
    \left( D ^2 w_{h}, D ^2v_{h} \right) _{\mathcal{T} _{h}}
   \\
    &\quad  +
  \left( \mean{  \partial _{n n} w_{h} }, \jump{ \partial _{n }v_{h}} \right)_{\mathcal{F}_{h}}  +
 \left( \mean{ \partial _{n n} v_{h} }, \jump{ \partial _{n}w }      \right)_{\mathcal{F}_{h}} \\
 & \quad + \frac{\gamma }{h}  \left( \jump{ \partial _{n} w_{h}}, \jump{ \partial _{n} v_{h}   }   \right)_{\mathcal{F}_{h}} \\
\quad \textit{Cauchy-Schwarz inequality }\quad  \le& \quad  \alpha  \|  w_{h} \|_{\Omega   }^{  } \| v_{h} \|_{ \Omega  }^{  }     +
\| D ^2w_{h} \|_{\mathcal{T} _{h}   }^{  }  \| D ^2v_{h} \|_{\mathcal{T} _{h}   }^{  } \\
& \quad  + \| h^{\frac{1}{2}}\mean{ \partial _{nn} w_{h} } \|_{ \mathcal{F}_{h}  }^{  } \| h^{-\frac{1}{2}}\jump{ \partial _{n} v_{h} } \|_{ \mathcal{F}_{h}  }^{  }    \\
& \quad  + \| h^{\frac{1}{2}}\mean{ \partial _{nn} v_{h} }
\|_{ \mathcal{F}_{h}  }^{  } \| h^{-\frac{1}{2}}\jump{ \partial _{n} w_{h} } \|_{ \mathcal{F}_{h}  }^{  }  \\
& \quad + \gamma \| h^{-1} \jump{ \partial _{n} v_{h}}   \|_{ \mathcal{F} _{h} }^{  }   \|  \jump{ \partial _{n} w_{h}}   \|_{ \mathcal{F} _{h} }^{  } \\
\quad \textit{Inverse inequality }\quad  \le & \quad   \alpha  \|  w_{h} \|_{\Omega   }^{  } \| v_{h} \|_{ \Omega  }^{  }  +
\| D ^2w_{h} \|_{\mathcal{T} _{h}   }^{  }  \| D ^2v_{h} \|_{\mathcal{T} _{h}   }^{  } \\
& \quad + C_{j}^{\frac{1}{2}} \| D ^2 w_{h} \|_{\mathcal{T} _{h}  }^{  }  \| h^{-\frac{1}{2}}\jump{ \partial _{n} v_{h} } \|_{ \mathcal{F}_{h}  }^{  }
 \\
& \quad +  C_{j}^{\frac{1}{2}} \| D ^2 w_{h} \|_{\mathcal{T} _{h}  }^{  }
 \| h^{-\frac{1}{2}}\jump{ \partial _{n} w_{h} } \|_{ \mathcal{F}_{h}  }^{  }\\
 & \quad  + \gamma \| h^{-1} \jump{ \partial _{n} v_{h}}   \|_{ \mathcal{F} _{h} }^{  }   \|  \jump{ \partial _{n} w_{h}}   \|_{ \mathcal{F} _{h} }^{  } \\
\textit{ Using \eqref{eq:bounded_ineq}}  \quad  \le & \quad \alpha  \|  w_{h} \|_{a_{h}   }^{  } \| v_{h} \|_{ a_{h}   }^{  } + \| w_{h} \|_{ a_{h} }^{  } \| v_{h} \|_{ a_{h} }^{  }  + 2C_{j}^{\frac{1}{2}} \| w_{h} \|_{ a_{h} }^{  } \| v_{h} \|_{
a_{h} }^{  }  \\
 & \quad + \gamma \| v_{h} \|_{ a_{h} }^{  } \| w_{h} \|_{ a_{h} }^{  } \\
 \le& \quad   \left( \alpha + 1 + 2C_{j}^{\frac{1}{2}} + \gamma  \right)  \| v_{h} \|_{a_{h}  }^{  }  \| w_{h} \|_{ a_{h} }^{  }  \le  K  \| v_{h} \|_{a_{h}  }^{  }  \| w_{h} \|_{ a_{h} }^{  }
\end{split}
\end{equation*}

Thus, the CIP method is shown to be bounded.
Again, the first step was to apply the \textbf{Cauchy-Schwarz inequality} for every term. On the second inequality the \textbf{Inverse inequality} was applied so that
\[
\| h^{\frac{1}{2}} \mean{ \partial _{nn}v_{h} }   \|_{ \mathcal{F} _{h}  }^{  }\le   C_{j}^{\frac{1}{2}} \| D ^2 v_{h} \|_{ \mathcal{T} _{h} }^{  } \quad \text{and} \quad   \| h^{\frac{1}{2}} \mean{ \partial _{nn}w_{h} }   \|_{ \mathcal{F} _{h}
}^{  }\le   C_{j}^{\frac{1}{2}} \| D ^2 w_{h} \|_{ \mathcal{T} _{h} }^{  }.
\]
The second step can we luckily observe that all terms invidually is less than the norm, that is,

\begin{equation}
\label{eq:bounded_ineq}
\begin{split}
\| w_{h} \|_{\Omega    }^{  }  \| v_{h} \|_{ \Omega    }^{  } & \le \| w_{h} \|_{ a_{h} }^{  } \| v_{h} \|_{ a_{h} }^{  }, \\
\| D ^2w_{h} \|_{\mathcal{T}_{h}   }^{  }  \| D ^2v_{h} \|_{\mathcal{T}_{h}   }^{  } & \le \| w_{h} \|_{ a_{h} }^{  } \| v_{h} \|_{ a_{h} }^{  }, \\
\|  D ^2 w_{h} \|_{ \mathcal{T} _{h} }^{ } \| h^{-\frac{1}{2}} \jump{ \partial _{n} v_{h} }   \|_{ \mathcal{F} _{h} }^{  }  & \le  \| w_{h} \|_{ a_{h} }^{  } \| v_{h} \|_{ a_{h} }^{  }, \\
   \|  D ^2 v_{h} \|_{ \mathcal{T} _{h} }^{ } \| h^{-\frac{1}{2}} \jump{ \partial _{n} w_{h} }   \|_{ \mathcal{F} _{h} }^{  }   & \le \| w_{h} \|_{ a_{h} }^{  } \| v_{h} \|_{ a_{h} }^{  }, \\
 \gamma \| h^{-1 } \jump{ \partial _{n} v_{h} }    \|_{ \mathcal{F} _{h}  }^{  }  \| \jump{ \partial _{n} w_{h} }    \|_{\mathcal{F}_{h}   }^{  }   & \le \gamma \| w_{h} \|_{ a_{h} }^{  }  \| v_{h} \|_{ a_{h} }^{  }.
\end{split}
\end{equation}
Hence, the CIP method is does fulfills the Lax Milgram criteria because it is both bounded and unique.

\subsection{Interpolations Estimates}%
\label{sub:clements_lemma}


We want to compute the expected convergence rate of the energy norm \eqref{eq:A_energy_norm}. An important tool in the process is the Cléments interpolation operator, $C_{h}$.
It is used for interpolation on non smooth functions and is defined as a local $L^{2}$ projection onto the so-called macroelements, that is, $C_{h}: H^{m} \left( \Omega  \right) \mapsto V_{h}$. For further detailed information, please investigate \cite{ern04}.


Recall the definition \eqref{eq:mixed_derivative} and let us define the integral norm notation,
\[
\| u \|_{ m,2,T }^{  } = \left( \sum_{ \left\lvert \alpha  \right\rvert \le m}^{} \int_{T}^{}  \left\lvert  \partial ^{\alpha } u \right\rvert^{2} dx   \right)^{\frac{1}{2}}
\]
We denote a patch, $\omega \left( T \right) $, as the set of elements in $\mathcal{T} _{h}$  sharing at least one vertex with $T \in \mathcal{T} _{h}$ . And similarly we denote a another patch, $\omega \left( F \right) $, as the set of all elements in $\mathcal{T}_{h} $
sharing at least one vertex with $F \in  \mathcal{F} _{h}$.
Furthermore, we also introduce the notation $\partial T$ for integration along the facets for a triangle $T$.

Now, let the interpolation estimate have the form $u - C_{h}u$.
The stability and interpolation properties of the Cléments interpolation operator has proven to be useful. In fact, Cléments lemma says that the operator satisfies \cite{ern04},
\[
 \| C_{h} v \|_{H^{m}\left( \Omega  \right)   }^{  } \lesssim \| v \|_{ H^{m}\left( \Omega  \right)  }^{  } \quad \forall v \in H^{m}\left( \Omega  \right),
\]
and if the following conditions for $m,l$ and $k$ is satisfied, it exists error estimates such that,
\[
    \begin{split}
      m\le l \le k+1  \implies \| v - C_{h} v \|_{ m,2,T   }^{  }  &  \lesssim h^{l-m}_{T} \| v \|_{l,2,\omega \left( T \right)  }^{  } \quad  \forall T \in \mathcal{T} _{h}, \forall v \in H^{l}\left( \omega \left( T \right)
      \right), \\
      m +\frac{1}{2}\le l \le k+1  \implies \| v - C_{h} v \|_{ m,2,F }^{  } & \lesssim h^{l-m- \frac{1}{2}}_{T} \| v \|_{l,2,\omega \left( F \right)  }^{  } \quad  \forall \partial T \in \mathcal{T} _{h}, \forall v \in H^{l}\left( \omega \left( F
      \right)  \right).
    \end{split}
\]
We will use these estimates to compute convergence rate.

Firstly and foremost, let the energy norm error be formulated as, \[
    \begin{split}
\| u - C_{h}u \|_{ a_{h},* }^{ 2 }  =&  \underbrace{\| D^2( u - C_{h}u ) \|_{\Omega   }^{2  }}_{(I)}  + \gamma \underbrace{\| h^{-\frac{1}{2}} \jump{ \partial _{n}\left( u - C_{h}u \right)  }    \|_{ \mathcal{F} _{h}  }^{2  }}_{(II)} \\
& +  \underbrace{\alpha ^2 \| u - C_{h}u \|_{\Omega   }^{2}}_{(III)}  + \underbrace{\| h^{\frac{1}{2}} \mean{ \partial _{nn}\left(  u - C_{h}u\right)  }   \|_{\mathcal{F} _{h}  }^{ 2 }}_{(IV)}.
    \end{split}
\]


Observe that by summing over triangles the jump and mean terms (respectively term $II$ and
$IV$) is notable simplified, \[
    \begin{split}
\sum_{F \in \mathcal{F}_{h} }^{}  \| \jump{ v }   \|_{F  }^{  } & =\sum_{F \in \mathcal{F}_{h} }^{}  \|  v^{+} - v^{-}    \|_{F  }^{  } \le \sum_{F \in \mathcal{F}_{h} }^{} \| v^{+} \|_{  F}^{  }  + \| v^{-} \|_{F }^{  } \le \sum_{T \in
\mathcal{T}_{h} }^{} \| v \|_{ \partial T }^{  } \\
\sum_{F \in \mathcal{F}_{h} }^{}  \| \mean{ v }   \|_{F  }^{  } & =\sum_{F \in \mathcal{F}_{h} }^{} \frac{1}{2} \|  v^{+} + v^{-}    \|_{F  }^{  } \le \sum_{F \in \mathcal{F}_{h} }^{} \frac{1}{2} \| v^{+} \|_{  F}^{  }  +\frac{1}{2} \| v^{-} \|_{F }^{  } \le \sum_{T \in
\mathcal{T}_{h} }^{} \| v \|_{ \partial T }^{  }. \\
    \end{split}
\]
Using this fact and applying Cléments lemma we get,
\begin{equation*}
    \begin{split}
(I) & \le  \| D^2\left( u - C_{h}u \right)  \|_{ \mathcal{T} _{h} }^{ 2 } = \sum_{T \in \mathcal{T} _{h}}^{} \| D^2 \left( u - C_{h}u \right)  \|_{ T }^{ 2 } \\
 & \lesssim \sum_{T \in \mathcal{T} _{h}}^{}  \| \left( u - C_{h}u \right)  \|_{2,T  }^{  2} \lesssim  \sum_{T \in \mathcal{T} _{h}}^{}  h_{T}^{2\left( l-2 \right) } \| u \|_{l, \omega \left( T \right)   }^{2  }, \\
    (II) & \le  \sum_{T \in \mathcal{T} _{h}}^{}  h^{-1} \| \partial _{n} \left( u - C_{h} \right)  \|_{ \partial T  }^{  } \le  \sum_{T \in \mathcal{T} _{h}}^{}  h^{-1} \left( h^{l -1 -\frac{1}{2}} \| u \|_{ l, \omega \left( T \right)  }^{  }
    \right)^{2},  \\
    & \lesssim \sum_{T \in \mathcal{T} _{h}}^{}  h^{2(l-2)} \| u \|_{ l, \omega \left( F \right)  }^{ 2 } \\
 (III)  &\le   \alpha^2  \sum_{T \in \mathcal{T} _{h}}^{}  \| u - C_{h} \|_{ T }^{ 2 } \lesssim   \sum_{T \in \mathcal{T} _{h}}^{} h^{2l} \| u \|_{ \omega \left( T \right)  }^{ 2 }, \\
(IV) & \le \sum_{T \in \mathcal{T} _{h}}^{}  h \| \partial _{nn} \left( u - C_{h} \right)  \|_{\partial T  }^{2  } \lesssim  \sum_{T \in  \mathcal{T} _{h}}^{} h \left( h^{l -2 -\frac{1}{2}} \| u \|_{ l, \omega \left( F \right)  }^{  }  \right)^{2}
\lesssim
\sum_{T  \in  \mathcal{T} _{h}}^{} h^{2(l-2)}  \| u \|_{ l, \omega \left( F \right)  }^{ 2 }.
 \end{split}
\end{equation*}
 The result then follows easily,
\[
    \begin{split}
        (I) + (II) + (III) +  (IV)  &  \lesssim  \sum_{T \in \mathcal{T} _{h}}^{}  h_{T}^{2\left( l-2 \right) } \| u \|_{l, \omega \left( T \right)   }^{  } +   2\cdot h^{2(l-2)}  \| u \|_{ l, \omega \left( F \right)  }^{ 2 } + h_{T}^{2l} \| u \|_{l, \omega \left( T \right)}
        \\
        &  \lesssim  h^{2\left( l-2 \right) } \| u \|_{ H^{l}\left( \Omega  \right)  }^{2  }.
    \end{split}
\]
Ergo, we now have a convergence rate estimate,
\[
\| u - C_{h}u \|_{a_{h},*  }^{  } \lesssim  h^{l(l-2)} \| u \|_{H^{l} \left( \Omega  \right)  }^{  }.
\]



It is easy to that we must require $u$ to be at least be in $H^{3}\left( \Omega  \right) $ since,
\[
u \in H^3\left( \Omega  \right)  \implies \begin{cases}
    \Delta u \in  H^{1}\left( \Omega  \right ) \\
    \nabla u \in H^2\left( \Omega  \right), \partial _{n} u \in H^{\frac{3}{2}}\left( \Gamma  \right) \\
    D^2u \in H^{1} \left( \Omega  \right), \quad  \partial _{nn}u \in  H^{\frac{1}{2}}\left( \Gamma  \right)
\end{cases}.
\]
Thus, if we let $u_{h} \in \mathcal{P} ^{k}$ and $u \in  H^{l}\left( \Omega  \right) $, then a reasonable assumption is that $3 \le  l \le k+1$.



\subsection{A Priori Estimates}%
\label{sub:apriori_estimates}

We will now introduce the notion of an a priori estimate, which can be used the estimate the size of a solution even before we have a solution. Since we have discrete coercivity, then $V_{h} \not \subseteq  V$, thus the standard method does not work.
Firstly, we want to use the results from \ref{sub:error_and_stability_analysis_of_c0ip}. We have shown that,
\begin{equation*}
    \begin{split}
    \textit{Discrete coercivity } \quad & \hat{\alpha } \| u_{h} \|_{ a_{h} }^{ 2 }  \le  \mathcal{A} \left( u_{h}, u_{h} \right) \quad  \forall u_{h} \in  V_{h} \\
    \textit{Boundedness (semi-discrete) }\quad  & \mathcal{A} \left( v,w_{h} \right)  \le  \widetilde{C} \| v \|_{ a_{h,*} }^{  }  \| w_{h} \|_{ a_{h} }^{  } \quad \forall v \in  V_{h} \oplus H^{4}\left( \Omega  \right),  \forall w_{h} \in V_{h} \\
    \textit{Boundedness (fully discrete) }\quad  & \mathcal{A} \left( v_{h},w_{h} \right)  \le  \overline{C}  \| v \|_{ a_{h} }^{  }  \| w_{h} \|_{ a_{h} }^{  } \quad \forall v_{h}, w_{h} \in V_{h}
    \end{split}
\end{equation*}

Let the difference have the form $u - u_{h} = (u - v_{h} )  + (v_{h} - u_{h})$ and the define identity,
$$
\| w_{h} \|_{ a_{h,*} }^{  }  \le  D \| w_{h} \|_{ a_{h} }^{  }, \forall w_{h} \in V_{h} .
$$
Thus, the norm can now be computed such that,\[
\| u - u_{h} \|_{ a_{h,*} }^{  } \le \|u - v_{h}  \|_{a_{h,*}  }^{  }  + \| v_{h} - u_{h} \|_{a_{h,*}  }^{  } \\
\le \| u -v_{h} \|_{ a_{h,*} }^{  }  + D \| u_{h} - v_{h} \|_{a_{h}  }^{  }.
\]
Finally, following the same procedure as in Ceas Lemma we get,
\[
    \begin{split}
\| u_{h} - v_{h} \|_{a_{h}  }^{2  } \hat{\alpha } & \le  \mathcal{A} \left( u_{h} - v_{h}, u_{h} - v_{h} \right) \\
& =  \mathcal{A} \left( u_{h} -u, u_{h} -v_{h} \right) + \mathcal{A} \left( u - v_{h}, u_{h} - v_{h} \right) \\
 &  \le  \mathcal{A}  \left( u - v_{h}, u_{h} - v_{h} \right)   \\
 &\le  \widetilde{C} \| u - v_{h} \|_{ a_{h,*} }^{  } \| u- v_{h} \|_{ a_{h} }^{  }.
    \end{split}
\]

Observe that we now have $ \| u_{h} - v_{h} \|_{ a_{h} }^{  }   \le \frac{\widetilde{C}}{\hat{\alpha }}  \| u - v_{h} \|_{ a_{h, *} }^{  }$ and $\| u - u_{h} \|_{ a_{h,*} }^{  }   \le \left( 1 + D \widetilde{C} /\hat{\alpha } \right)\cdot  \| u -
v_{h} \|_{ a_{h,*} }^{  } $. Hence, we have derived a equivalent Céa's lemma for the CIP method.
\[
    \begin{split}
\| u_{h} - v_{h} \|_{ a_{h} }^{  }  & \le \frac{\widetilde{C}}{\hat{\alpha }}  \inf_{v_{h} \in  V_{h}} \|  v_{h} - u \|_{ a_{h, *} }^{  } \\
\| u - u_{h} \|_{ a_{h,*} }^{  }  & \le \left( 1 + D \widetilde{C} /\hat{\alpha } \right)\cdot \inf_{v_{h} \in  V_{h}}   \| u - v_{h} \|_{ a_{h,*} }^{  }
    \end{split}
\]
By combing Cléments lemma and Céa's lemma can we now, in fact, conclude that the energy norm has the convergence rate estimate,
\begin{equation}
\label{eq:conv_estimate}
\| u - u_{h} \|_{ a_{h}, * }^{  } \lesssim h^{k-1}.
\end{equation}


