
\newpage
\section{Application to the Cahn Hilliard equation }%
\label{sec:cahn_hilliard_equation}

In this section, we will demonstrate that the proposed cut finite element method can be used to solve the Cahn-Hilliard problem. Firstly, we will recall the strong form and illustrate how it can be recast into a weak form following the approach in
\cite{feng2007fully}. Subsequently, we will derive a simplistic numerical time iteration scheme to demonstrate that the solution to the problem can indeed be found. Again, all numerical experiments are conducted using the open-source finite element method (FEM) framework, Gridap, as documented in \cite{verdugo22}.


\subsection{Revisiting the strong formulation}%
\label{sub:revisiting_the_strong_formulation}

We will do a short revisit of the strong Cahn-Hilliard equation. Recall that,
\begin{equation}
\label{eq:strong_ch}
    \begin{split}
\partial _{t} u + \Delta  \left(  \varepsilon  \Delta u - \frac{1}{\varepsilon }f( u) \right)   &=0  \quad \text{ in } \Omega  \\
\partial _{n} u &= 0 \quad \text{ on } \Gamma  \\
\partial _{n}    \Delta u       &= 0 \quad \text{ on } \Gamma  \\
    \end{split}
\end{equation}
Where we defined $f( u) = u( u^{2} -1)  = F' ( u)  $ where $F( u) = ( 1 / 4 ) ( u^2 -1 ) ^{2} $. With the corresponding energy.
\begin{equation}
E( u)  = \int_{\Omega }^{} \frac{\varepsilon }{2} \abs{ \nabla u } ^2 + \frac{1}{\varepsilon } F( u) dx
\end{equation}

We also recall that the energy functional monotonically decreasing and that the global mass concentration is conserved, i.e.
\begin{equation}
\label{eq:mass_cons_energy_decrease}
\frac{d}{dt} E( u)  <  0 \text{ and }\frac{d}{dt} \int_{\Omega }^{}  u dx = 0.
\end{equation}

Now assume that $u \in L^2( [0,T], H^{4}( \Omega ) ) $ and $v_{h} \in  V_{h}$ of order $k$. It is clear that the initial weak formulation is,
\begin{equation}
    ( \partial_{t} u,v_{h} )_{\Omega }  + ( \Delta ^2u, v_{h})_{\Omega } - ( f( u), v_{h} )_{\Omega } = 0.
\end{equation}

As we can see did the biharmonic equation appear and, thus, we can apply the cut finite element bilinear form proposed in Equation \eqref{eq:discrete_CutCIP_prob}. By also utilizing the  can we discretize the problem.
\begin{equation}
    \begin{split}
        & \text{Find  }u_{h} \in L^{2}( [0,T],V_{h})  \text{ such that } \forall v_{h} \in V_{h} \\
        & ( \partial_{t} u_{h},v_{h} )_{\Omega }   + \frac{1}{\varepsilon }A_{h}( u_{h},v_{h})   + \varepsilon c_{h}( u_{h},v_{h})  = 0.
    \end{split}
\end{equation}
Here we have followed the nonlinear weak formulation \cite[Equation
4.2]{feng2007fully} such that
\begin{equation}
    c_{h}( u_{h}, v_{h})  = ( f( u_{h}) ,\Delta v_{h})_{\Omega } +  ( f( u_{h}) , \partial _{n}v)_{\mathcal{F}_{h} }
\end{equation}





\subsection{Development of an Implicit-Explicit Scheme}
\label{sub:implicit_explicit_scheme}

The primary aim is to formulate an uncomplicated time iteration scheme. Define the index $m$ to range over the set ${0, 1, \ldots, M}$, which corresponds to uniformly distributed time points $t_{m}$ that satisfy the boundary conditions $t_{0} = 0$
and $t_{M} = T$. Here, each time step is an element of the function space, i.e. $u^{m}_{h} \in V_{h}$  with the initial condition defined as $u^{0} = u( t_{0},x )$.
For the Implicit explicit scheme (IMEX) sceheme we have the following discretization
\[
( \overline{\partial } _{t} u^{m}_{h}, v_{h}   )_{\Omega } + \varepsilon a^{m}_{h}( u_{h}^{m} , v) + \frac{1}{\varepsilon } c_{h} (  u_{h}^{m-1}, v_{h})  = l_{h}( v) , \quad \forall v_{h}, u^{m}_{h} \in V^{}_{h}.
\]

Here we have define the forward difference operator with time step $\tau $
\[
\overline{\partial } _{t} u_{h}^{m} = \frac{u_{h}^{m} - u_{h}^{m-1}}{ \tau }
\]
, then we have the following scheme.
\[
( u_{h}^{m},v )_{\Omega }  + \tau \varepsilon a_{h}( u_{h}^{m} , v)   = \tau  l_{h}( v) +   ( u_{h}^{m-1},v )_{\Omega } - \frac{\tau}{\varepsilon } c_{h} (  u_{h}^{m-1}, v) .
\]

 % Recall that $V_{h}$ is spanned by the orthonormal polynomial basis $ \left\{ \phi _{j} \right\}_{i=1}^{ N}  $ with $N$ degrees of freedom  s.t.  $v = \sum_{i}^{N} V_{i} \phi_{i}   $ and $u_{h} = \sum_{i}^{N} U_{i} \phi_{i}   $.
 %    Let  $U^{m} = \left[ U_{1}, \ldots, U^{N} \right]^{T} $ and $V= \left[ V_{1}, \ldots, V_{M} \right]^{T} $. Writing the matrices $[ A ]_{i,j} =  a( \phi _{i}, \phi _{j})  $ and $\left[ F \right] _{j} = l_{h}( \phi _{j})  $ and $ \left[ C \right]_{ij}
 %    = c_{h}( U^{m-1} \phi _{j}, \phi _{j}) $  .
 %    Subsequently, we can express the system in an equivalent explicit system.
 %    \[
 %        \begin{split}
 %            U^{m}  + \tau \varepsilon A U^{m} & = ( I  + \tau \varepsilon A ) U^{m}   \\
 %            b(U^{m-1}, V) &:=   \tau  F +   ( u_{h}^{m-1},v )_{\Omega } - \frac{\tau}{\varepsilon } c_{h} (  u_{h}^{m-1}, v)
 %        \end{split}
 %    \]
 %    s.t. $  ( I  + \tau \varepsilon A ) U^{m} = b(U^{m-1}, V )  $


\subsection{Demonstration on the physical Cahn-Hilliard problem}%
\label{sub:demonstration_on_the_physical_cahn_hilliard}

On the physical CH \eqref{eq:strongch} presented we have no analytical solution to check the solution. However, a way to check that the system does behave like expected based on the physical properties \eqref{eq:mass_cons_energy_decrease}. In other
words, we can check that our discrete solution satisfies \[
    \begin{split}
 E( u_{h}( \cdot , t_{2}) ) & \le  E( u_{h}( \cdot , t_{1}) )   \text{ for } 0 < t_{1} < t_{2},  \\
\int_{\Omega }^{} u_{h} ( x,t)  dx & = \int_{\Omega }^{} u_{0}(x)  dx,
    \end{split}
\]
where $u_{0}( x) = u_{h}( x,t) $. Let us define
\[
 e_{u} = \frac{ \int_{\Omega }^{}  u_h(x,t)- u(0,x) dx}{ \int_{\Omega }^{}  u(x,0) dx}
\]


\begin{figure}[H]
    \centering
    \subfloat[$t=0\tau$]{\label{sub:fig:ill_c0}
        \includegraphics[width=0.3\textwidth]{results/illustration/c0.png}
    }\hfill
    \subfloat[$t=2\tau$]{\label{sub:fig:ill_c2}
        \includegraphics[width=0.3\textwidth]{results/illustration/c2.png}
    }\hfill
    \subfloat[$t=10\tau$]{\label{sub:fig:ill_c10}
        \includegraphics[width=0.3\textwidth]{results/illustration/c10.png}
    }
    \vspace{10pt}
    \subfloat[t=50$\tau$]{\label{sub:fig:ill_c50}
        \includegraphics[width=0.3\textwidth]{results/illustration/c50.png}
    }\hfill
    \subfloat[$t=200\tau$]{\label{sub:fig:ill_c200}
        \includegraphics[width=0.3\textwidth]{results/illustration/c200.png}
    }\hfill
    \subfloat[$t=1000\tau$]{\label{sub:fig:ill_c1000}
        \includegraphics[width=0.3\textwidth]{results/illustration/c1000.png}
    } \\
    \vspace{10pt}
    \centering
    \subfloat[$t=0\tau$]{\label{sub:fig:ill_f0}
        \includegraphics[width=0.3\textwidth]{results/illustration/f0.png}
    }\hfill
    \subfloat[$t=2\tau$]{\label{sub:fig:ill_f2}
        \includegraphics[width=0.3\textwidth]{results/illustration/f2.png}
    }\hfill
    \subfloat[$t=10\tau$]{\label{sub:fig:ill_f10}
        \includegraphics[width=0.3\textwidth]{results/illustration/f10.png}
    }
    \vspace{10pt}
    \subfloat[$t=50\tau$]{\label{sub:fig:ill_f50}
        \includegraphics[width=0.3\textwidth]{results/illustration/f50.png}
    }\hfill
    \subfloat[$t=200\tau$]{\label{sub:fig:ill_f200}
        \includegraphics[width=0.3\textwidth]{results/illustration/f200.png}
    }\hfill
    \subfloat[$t=1000\tau$]{\label{sub:fig:ill_f1000}
        \includegraphics[width=0.3\textwidth]{results/illustration/f1000.png}
    }
    \caption{Illustration of simulations of the Cahn-Hilliard equation on the circle and a flower domain for $t\in \left[ 0, 1000\tau  \right] $.}
    \label{sub:fig:ill_circle_flower}
\end{figure}



We ran the simulation on the for the flower domain \eqref{eq:flower} and the circular domain \eqref{eq:circle}, illustrated in Figure \ref{sub:fig:ill_circle_flower} . The corresponding plots of the mass conservation and energy decrease are presented
in the Figure \ref{fig:physical_CH_plot}, and confirm the expected physical properties of the Cahn-Hilliard equation. The relative error $e_{u}$ shows small deviations of the numerical solution in the order of $10^{ -10 }$ demonstrating that the mass is conserved. We also observe that the energy functional
$E(u_h)$ decreases over time, signifying the system's tendency to seek a state of minimal energy.


\begin{figure}[h!]
% Recommended preamble:
% \usetikzlibrary{arrows.meta}
% \usetikzlibrary{backgrounds}
% \usepgfplotslibrary{patchplots}
% \usepgfplotslibrary{fillbetween}
% \pgfplotsset{%
%     layers/standard/.define layer set={%
%         background,axis background,axis grid,axis ticks,axis lines,axis tick labels,pre main,main,axis descriptions,axis foreground%
%     }{
%         grid style={/pgfplots/on layer=axis grid},%
%         tick style={/pgfplots/on layer=axis ticks},%
%         axis line style={/pgfplots/on layer=axis lines},%
%         label style={/pgfplots/on layer=axis descriptions},%
%         legend style={/pgfplots/on layer=axis descriptions},%
%         title style={/pgfplots/on layer=axis descriptions},%
%         colorbar style={/pgfplots/on layer=axis descriptions},%
%         ticklabel style={/pgfplots/on layer=axis tick labels},%
%         axis background@ style={/pgfplots/on layer=axis background},%
%         3d box foreground style={/pgfplots/on layer=axis foreground},%
%     },
% }

\begin{tikzpicture}[/tikz/background rectangle/.style={fill={rgb,1:red,1.0;green,1.0;blue,1.0}, fill opacity={1.0}, draw opacity={1.0}}, show background rectangle]
\begin{axis}[point meta max={nan}, point meta min={nan}, legend cell align={left}, legend columns={1}, title={}, title style={at={{(0.5,1)}}, anchor={south}, font={{\fontsize{14 pt}{18.2 pt}\selectfont}}, color={rgb,1:red,0.0;green,0.0;blue,0.0}, draw opacity={1.0}, rotate={0.0}, align={center}}, legend style={color={rgb,1:red,0.0;green,0.0;blue,0.0}, draw opacity={1.0}, line width={1}, solid, fill={rgb,1:red,1.0;green,1.0;blue,1.0}, fill opacity={1.0}, text opacity={1.0}, font={{\fontsize{8 pt}{10.4 pt}\selectfont}}, text={rgb,1:red,0.0;green,0.0;blue,0.0}, cells={anchor={center}}, at={(1.02, 1)}, anchor={north west}}, axis background/.style={fill={rgb,1:red,1.0;green,1.0;blue,1.0}, opacity={1.0}}, anchor={north west}, xshift={1.0mm}, yshift={-1.0mm}, width={150.4mm}, height={36.099999999999994mm}, scaled x ticks={false}, xlabel={$t/\tau$}, x tick style={color={rgb,1:red,0.0;green,0.0;blue,0.0}, opacity={1.0}}, x tick label style={color={rgb,1:red,0.0;green,0.0;blue,0.0}, opacity={1.0}, rotate={0}}, xlabel style={at={(ticklabel cs:0.5)}, anchor=near ticklabel, at={{(ticklabel cs:0.5)}}, anchor={near ticklabel}, font={{\fontsize{11 pt}{14.3 pt}\selectfont}}, color={rgb,1:red,0.0;green,0.0;blue,0.0}, draw opacity={1.0}, rotate={0.0}}, xmode={log}, log basis x={10}, xmajorgrids={true}, xmin={0.8707036374807742}, xmax={115.99813719881251}, xticklabels={{$10^{0}$,$10^{1}$,$10^{2}$}}, xtick={{1.0,10.0,100.0}}, xtick align={inside}, xticklabel style={font={{\fontsize{8 pt}{10.4 pt}\selectfont}}, color={rgb,1:red,0.0;green,0.0;blue,0.0}, draw opacity={1.0}, rotate={0.0}}, x grid style={color={rgb,1:red,0.0;green,0.0;blue,0.0}, draw opacity={0.1}, line width={0.5}, solid}, axis x line*={left}, x axis line style={color={rgb,1:red,0.0;green,0.0;blue,0.0}, draw opacity={1.0}, line width={1}, solid}, scaled y ticks={false}, ylabel={$\Delta u^m$}, y tick style={color={rgb,1:red,0.0;green,0.0;blue,0.0}, opacity={1.0}}, y tick label style={color={rgb,1:red,0.0;green,0.0;blue,0.0}, opacity={1.0}, rotate={0}}, ylabel style={at={(ticklabel cs:0.5)}, anchor=near ticklabel, at={{(ticklabel cs:0.5)}}, anchor={near ticklabel}, font={{\fontsize{11 pt}{14.3 pt}\selectfont}}, color={rgb,1:red,0.0;green,0.0;blue,0.0}, draw opacity={1.0}, rotate={0.0}}, ymajorgrids={true}, ymin={-1.8092279519419343e-14}, ymax={6.211682635000637e-13}, yticklabels={{$0$,$1.0\times10^{^{-13}}$,$2.0\times10^{^{-13}}$,$3.0\times10^{^{-13}}$,$4.0\times10^{^{-13}}$,$5.0\times10^{^{-13}}$,$6.0\times10^{^{-13}}$}}, ytick={{0.0,1.0e-13,2.0e-13,3.0000000000000003e-13,4.0e-13,5.0e-13,6.000000000000001e-13}}, ytick align={inside}, yticklabel style={font={{\fontsize{8 pt}{10.4 pt}\selectfont}}, color={rgb,1:red,0.0;green,0.0;blue,0.0}, draw opacity={1.0}, rotate={0.0}}, y grid style={color={rgb,1:red,0.0;green,0.0;blue,0.0}, draw opacity={0.1}, line width={0.5}, solid}, axis y line*={left}, y axis line style={color={rgb,1:red,0.0;green,0.0;blue,0.0}, draw opacity={1.0}, line width={1}, solid}, colorbar={false}]
    \addplot[color={rgb,1:red,1.0;green,0.0;blue,0.0}, name path={e8e1944d-357f-4256-a0ef-de3fb5aa5457}, draw opacity={1.0}, line width={1}, solid]
        table[row sep={\\}]
        {
            \\
            1.0  1.750929943592765e-14  \\
            2.0  1.0744342835682876e-14  \\
            3.0  1.631548356529622e-14  \\
            4.0  2.4672194659716234e-14  \\
            5.0  2.7855703648066716e-14  \\
            6.000000000000001  3.1835089883504815e-14  \\
            7.0  3.9793862354381025e-14  \\
            8.0  4.3773248589819127e-14  \\
            9.0  3.422272162476768e-14  \\
            10.0  2.7855703648066716e-14  \\
            11.000000000000002  6.605781150827249e-14  \\
            12.0  3.263096713059244e-14  \\
            13.0  2.8651580895154336e-14  \\
            14.0  1.1938158706314307e-14  \\
            15.0  2.7855703648066716e-14  \\
            15.999999999999998  1.910105393010289e-14  \\
            17.0  4.456912583690674e-14  \\
            18.0  3.9793862354381025e-14  \\
            19.0  4.934438931943247e-14  \\
            20.0  6.685368875536012e-14  \\
            21.000000000000004  7.958772470876205e-14  \\
            22.0  4.775263482525723e-14  \\
            23.0  4.934438931943247e-14  \\
            24.0  5.730316179030867e-14  \\
            25.0  8.436298819128776e-14  \\
            26.0  9.391351515633921e-14  \\
            27.0  1.0187228762721541e-13  \\
            28.000000000000004  6.048667077865915e-14  \\
            29.0  4.2977371342731504e-14  \\
            30.0  7.640421572041156e-14  \\
            30.999999999999996  1.0983106009809162e-13  \\
            32.0  1.2256509605149356e-13  \\
            33.0  8.675061993255062e-14  \\
            34.0  7.640421572041156e-14  \\
            35.0  1.0425991936847828e-13  \\
            35.99999999999999  1.0823930560391639e-13  \\
            37.0  8.197535645002491e-14  \\
            38.0  5.093614381360771e-14  \\
            39.0  7.79959702145868e-14  \\
            40.0  8.038360195584967e-14  \\
            41.00000000000001  1.130145690864421e-13  \\
            42.0  3.740623061311816e-14  \\
            43.0  6.3670179767009634e-15  \\
            44.0  0.0  \\
            45.0  1.0346404212139066e-14  \\
            46.0  5.770110041385249e-14  \\
            47.0  6.963925912016679e-14  \\
            48.0  1.0744342835682876e-13  \\
            49.0  1.0505579661556589e-13  \\
            50.0  1.5519606318208598e-13  \\
            51.0  1.9419404828937938e-13  \\
            52.0  2.0812190011341274e-13  \\
            53.0  1.9419404828937938e-13  \\
            54.0  2.2006005881972705e-13  \\
            55.00000000000001  2.220497519374461e-13  \\
            56.00000000000001  2.417477138028647e-13  \\
            57.0  2.775621899218076e-13  \\
            57.99999999999999  2.99846752840261e-13  \\
            59.0  3.0273180786095363e-13  \\
            59.99999999999999  3.214349231675127e-13  \\
            60.99999999999999  3.315823580678799e-13  \\
            62.0  2.9407664279887576e-13  \\
            63.0  3.1208336551423317e-13  \\
            64.0  3.3939190355492714e-13  \\
            65.0  3.79484219876966e-13  \\
            66.0  4.584750366504123e-13  \\
            67.0  4.3942372504825243e-13  \\
            68.0  4.4004550414753964e-13  \\
            69.0  3.539166633142762e-13  \\
            70.0  4.1099598462884147e-13  \\
            70.99999999999999  4.207952232336078e-13  \\
            72.0  3.5435190868377725e-13  \\
            73.0  3.8634866113309676e-13  \\
            74.0  3.697969015100714e-13  \\
            75.0  3.349150940400593e-13  \\
            76.0  3.256132787147227e-13  \\
            77.0  3.537176940025043e-13  \\
            78.0  3.4570917920368515e-13  \\
            79.0  3.5050931385018236e-13  \\
            80.0  3.663025029720773e-13  \\
            81.00000000000001  3.6970985243617116e-13  \\
            82.0  3.6421332519847233e-13  \\
            83.0  3.8823886959492983e-13  \\
            84.0  3.319802966914237e-13  \\
            85.0  3.730177172443791e-13  \\
            86.0  3.2725477553684095e-13  \\
            87.0  3.7878782728576433e-13  \\
            88.0  4.055989420470286e-13  \\
            89.0  4.3897604409676564e-13  \\
            90.0  4.374340319305334e-13  \\
            91.0  4.5141162608250974e-13  \\
            92.0  5.064266407874415e-13  \\
            93.0  4.988160646121662e-13  \\
            94.0  5.342326021075652e-13  \\
            95.0  6.030759839806444e-13  \\
            95.99999999999999  6.007878368952675e-13  \\
            97.0  5.248810444542857e-13  \\
            98.0  5.615908824762022e-13  \\
            99.0  5.414949819872398e-13  \\
            100.0  4.818041884556682e-13  \\
            101.0  4.701644837170118e-13  \\
        }
        ;
    \addplot[color={rgb,1:red,0.0;green,0.0;blue,1.0}, name path={abdf1c6f-a02e-4c76-a718-32f1b2304057}, draw opacity={1.0}, line width={1}, solid]
        table[row sep={\\}]
        {
            \\
            1.0  4.567439002177944e-15  \\
            2.0  7.03960528727286e-15  \\
            3.0  6.873516596284571e-15  \\
            4.0  1.0629676223250488e-14  \\
            5.0  8.662164037696912e-15  \\
            6.000000000000001  1.3184886853839548e-14  \\
            7.0  1.620003539793464e-14  \\
            8.0  2.095272717083029e-14  \\
            9.0  2.304799988791332e-14  \\
            10.0  2.8260629574315002e-14  \\
            11.000000000000002  3.209344552019859e-14  \\
            12.0  3.4035405599446276e-14  \\
            13.0  3.91458268606244e-14  \\
            14.0  4.517612394881458e-14  \\
            15.0  4.793575142985076e-14  \\
            15.999999999999998  5.2024088438793255e-14  \\
            17.0  5.376163166759382e-14  \\
            18.0  5.682788442430069e-14  \\
            19.0  6.009855403145469e-14  \\
            20.0  6.398247418995006e-14  \\
            21.000000000000004  6.755976907277474e-14  \\
            22.0  7.134148080604655e-14  \\
            23.0  6.704872694665693e-14  \\
            24.0  7.399889986185917e-14  \\
            25.0  7.645190206722467e-14  \\
            26.0  8.013140537527291e-14  \\
            27.0  8.360649183287403e-14  \\
            28.000000000000004  8.66727445895809e-14  \\
            29.0  9.15787490003119e-14  \\
            30.0  8.99434141967349e-14  \\
            30.999999999999996  9.260083325254753e-14  \\
            32.0  9.341850065433603e-14  \\
            33.0  9.832450506506702e-14  \\
            34.0  9.995983986864402e-14  \\
            35.0  9.955100616774977e-14  \\
            35.99999999999999  9.852892191551415e-14  \\
            37.0  1.0343492632624514e-13  \\
            38.0  1.0650117908295202e-13  \\
            39.0  1.1099834979278876e-13  \\
            40.0  1.154955205026255e-13  \\
            41.00000000000001  1.0588792853161064e-13  \\
            42.0  1.1345135199815425e-13  \\
            43.0  1.1426901939994274e-13  \\
            44.0  1.2183244286648638e-13  \\
            45.0  1.1569993735307262e-13  \\
            46.0  1.2305894396916912e-13  \\
            47.0  1.1835735640888525e-13  \\
            48.0  1.2673844727721737e-13  \\
            49.0  1.253075293240875e-13  \\
            50.0  1.3348420334197247e-13  \\
            51.0  1.361416223977851e-13  \\
            52.0  1.4431829641567009e-13  \\
            53.0  1.4901988397595396e-13  \\
            54.0  1.5167730303176659e-13  \\
            55.00000000000001  1.592407264983102e-13  \\
            56.00000000000001  1.5617447374160334e-13  \\
            57.0  1.6046722760099296e-13  \\
            57.99999999999999  1.6189814555412283e-13  \\
            59.0  1.6271581295591134e-13  \\
            59.99999999999999  1.625113961054642e-13  \\
            60.99999999999999  1.6925715217021933e-13  \\
            62.0  1.700748195720078e-13  \\
            63.0  1.6762181736664232e-13  \\
            64.0  1.6803065106753657e-13  \\
            65.0  1.7763824303855143e-13  \\
            66.0  1.739587397305032e-13  \\
            67.0  1.7804707673944568e-13  \\
            68.0  1.8172658004749393e-13  \\
            69.0  1.8029566209436406e-13  \\
            70.0  1.774338261881043e-13  \\
            70.99999999999999  1.805000789448112e-13  \\
            72.0  1.805000789448112e-13  \\
            73.0  1.8131774634659968e-13  \\
            74.0  1.8274866429972955e-13  \\
            75.0  1.9276508997163868e-13  \\
            76.0  1.8745025186001342e-13  \\
            77.0  1.9317392367253293e-13  \\
            78.0  1.8867675296269618e-13  \\
            79.0  1.9521809217700417e-13  \\
            80.0  2.0237268194265353e-13  \\
            81.00000000000001  2.0809635375517302e-13  \\
            82.0  2.035991830453363e-13  \\
            83.0  2.005329302886294e-13  \\
            84.0  2.0135059769041791e-13  \\
            85.0  2.0911843800740864e-13  \\
            86.0  2.170906951748465e-13  \\
            87.0  2.1320677501635114e-13  \\
            88.0  2.1422885926858676e-13  \\
            89.0  2.1341119186679828e-13  \\
            90.0  2.0748310320383165e-13  \\
            91.0  2.1749952887574076e-13  \\
            92.0  2.2608503659452002e-13  \\
            93.0  2.1054935596053854e-13  \\
            94.0  2.144332761190339e-13  \\
            95.0  2.1545536037126952e-13  \\
            95.99999999999999  2.1647744462350514e-13  \\
            97.0  2.1647744462350514e-13  \\
            98.0  2.2158786588468326e-13  \\
            99.0  2.2240553328647177e-13  \\
            100.0  2.2894687250077976e-13  \\
            101.0  2.301733736034625e-13  \\
        }
        ;
    \addplot[color={rgb,1:red,0.2422;green,0.6433;blue,0.3044}, name path={849b4d3e-eb4b-4345-a107-8dc22edda7f3}, only marks, draw opacity={1.0}, line width={0}, solid, mark={*}, mark size={1.5 pt}, mark repeat={1}, mark options={color={rgb,1:red,0.0;green,0.0;blue,0.0}, draw opacity={0.5}, fill={rgb,1:red,1.0;green,0.0;blue,0.0}, fill opacity={0.5}, line width={0.75}, rotate={0}, solid}]
        table[row sep={\\}]
        {
            \\
            1.0  1.750929943592765e-14  \\
            2.0  1.0744342835682876e-14  \\
            3.0  1.631548356529622e-14  \\
            4.0  2.4672194659716234e-14  \\
            5.0  2.7855703648066716e-14  \\
            6.000000000000001  3.1835089883504815e-14  \\
            7.0  3.9793862354381025e-14  \\
            8.0  4.3773248589819127e-14  \\
            9.0  3.422272162476768e-14  \\
            10.0  2.7855703648066716e-14  \\
            11.000000000000002  6.605781150827249e-14  \\
            12.0  3.263096713059244e-14  \\
            13.0  2.8651580895154336e-14  \\
            14.0  1.1938158706314307e-14  \\
            15.0  2.7855703648066716e-14  \\
            15.999999999999998  1.910105393010289e-14  \\
            17.0  4.456912583690674e-14  \\
            18.0  3.9793862354381025e-14  \\
            19.0  4.934438931943247e-14  \\
            20.0  6.685368875536012e-14  \\
            21.000000000000004  7.958772470876205e-14  \\
            22.0  4.775263482525723e-14  \\
            23.0  4.934438931943247e-14  \\
            24.0  5.730316179030867e-14  \\
            25.0  8.436298819128776e-14  \\
            26.0  9.391351515633921e-14  \\
            27.0  1.0187228762721541e-13  \\
            28.000000000000004  6.048667077865915e-14  \\
            29.0  4.2977371342731504e-14  \\
            30.0  7.640421572041156e-14  \\
            30.999999999999996  1.0983106009809162e-13  \\
            32.0  1.2256509605149356e-13  \\
            33.0  8.675061993255062e-14  \\
            34.0  7.640421572041156e-14  \\
            35.0  1.0425991936847828e-13  \\
            35.99999999999999  1.0823930560391639e-13  \\
            37.0  8.197535645002491e-14  \\
            38.0  5.093614381360771e-14  \\
            39.0  7.79959702145868e-14  \\
            40.0  8.038360195584967e-14  \\
            41.00000000000001  1.130145690864421e-13  \\
            42.0  3.740623061311816e-14  \\
            43.0  6.3670179767009634e-15  \\
            44.0  0.0  \\
            45.0  1.0346404212139066e-14  \\
            46.0  5.770110041385249e-14  \\
            47.0  6.963925912016679e-14  \\
            48.0  1.0744342835682876e-13  \\
            49.0  1.0505579661556589e-13  \\
            50.0  1.5519606318208598e-13  \\
            51.0  1.9419404828937938e-13  \\
            52.0  2.0812190011341274e-13  \\
            53.0  1.9419404828937938e-13  \\
            54.0  2.2006005881972705e-13  \\
            55.00000000000001  2.220497519374461e-13  \\
            56.00000000000001  2.417477138028647e-13  \\
            57.0  2.775621899218076e-13  \\
            57.99999999999999  2.99846752840261e-13  \\
            59.0  3.0273180786095363e-13  \\
            59.99999999999999  3.214349231675127e-13  \\
            60.99999999999999  3.315823580678799e-13  \\
            62.0  2.9407664279887576e-13  \\
            63.0  3.1208336551423317e-13  \\
            64.0  3.3939190355492714e-13  \\
            65.0  3.79484219876966e-13  \\
            66.0  4.584750366504123e-13  \\
            67.0  4.3942372504825243e-13  \\
            68.0  4.4004550414753964e-13  \\
            69.0  3.539166633142762e-13  \\
            70.0  4.1099598462884147e-13  \\
            70.99999999999999  4.207952232336078e-13  \\
            72.0  3.5435190868377725e-13  \\
            73.0  3.8634866113309676e-13  \\
            74.0  3.697969015100714e-13  \\
            75.0  3.349150940400593e-13  \\
            76.0  3.256132787147227e-13  \\
            77.0  3.537176940025043e-13  \\
            78.0  3.4570917920368515e-13  \\
            79.0  3.5050931385018236e-13  \\
            80.0  3.663025029720773e-13  \\
            81.00000000000001  3.6970985243617116e-13  \\
            82.0  3.6421332519847233e-13  \\
            83.0  3.8823886959492983e-13  \\
            84.0  3.319802966914237e-13  \\
            85.0  3.730177172443791e-13  \\
            86.0  3.2725477553684095e-13  \\
            87.0  3.7878782728576433e-13  \\
            88.0  4.055989420470286e-13  \\
            89.0  4.3897604409676564e-13  \\
            90.0  4.374340319305334e-13  \\
            91.0  4.5141162608250974e-13  \\
            92.0  5.064266407874415e-13  \\
            93.0  4.988160646121662e-13  \\
            94.0  5.342326021075652e-13  \\
            95.0  6.030759839806444e-13  \\
            95.99999999999999  6.007878368952675e-13  \\
            97.0  5.248810444542857e-13  \\
            98.0  5.615908824762022e-13  \\
            99.0  5.414949819872398e-13  \\
            100.0  4.818041884556682e-13  \\
            101.0  4.701644837170118e-13  \\
        }
        ;
    \addplot[color={rgb,1:red,0.7644;green,0.4441;blue,0.8243}, name path={7fec3f10-9da4-4c6c-8e4d-71404640b17e}, only marks, draw opacity={1.0}, line width={0}, solid, mark={*}, mark size={1.5 pt}, mark repeat={1}, mark options={color={rgb,1:red,0.0;green,0.0;blue,0.0}, draw opacity={0.5}, fill={rgb,1:red,0.0;green,0.0;blue,1.0}, fill opacity={0.5}, line width={0.75}, rotate={0}, solid}]
        table[row sep={\\}]
        {
            \\
            1.0  4.567439002177944e-15  \\
            2.0  7.03960528727286e-15  \\
            3.0  6.873516596284571e-15  \\
            4.0  1.0629676223250488e-14  \\
            5.0  8.662164037696912e-15  \\
            6.000000000000001  1.3184886853839548e-14  \\
            7.0  1.620003539793464e-14  \\
            8.0  2.095272717083029e-14  \\
            9.0  2.304799988791332e-14  \\
            10.0  2.8260629574315002e-14  \\
            11.000000000000002  3.209344552019859e-14  \\
            12.0  3.4035405599446276e-14  \\
            13.0  3.91458268606244e-14  \\
            14.0  4.517612394881458e-14  \\
            15.0  4.793575142985076e-14  \\
            15.999999999999998  5.2024088438793255e-14  \\
            17.0  5.376163166759382e-14  \\
            18.0  5.682788442430069e-14  \\
            19.0  6.009855403145469e-14  \\
            20.0  6.398247418995006e-14  \\
            21.000000000000004  6.755976907277474e-14  \\
            22.0  7.134148080604655e-14  \\
            23.0  6.704872694665693e-14  \\
            24.0  7.399889986185917e-14  \\
            25.0  7.645190206722467e-14  \\
            26.0  8.013140537527291e-14  \\
            27.0  8.360649183287403e-14  \\
            28.000000000000004  8.66727445895809e-14  \\
            29.0  9.15787490003119e-14  \\
            30.0  8.99434141967349e-14  \\
            30.999999999999996  9.260083325254753e-14  \\
            32.0  9.341850065433603e-14  \\
            33.0  9.832450506506702e-14  \\
            34.0  9.995983986864402e-14  \\
            35.0  9.955100616774977e-14  \\
            35.99999999999999  9.852892191551415e-14  \\
            37.0  1.0343492632624514e-13  \\
            38.0  1.0650117908295202e-13  \\
            39.0  1.1099834979278876e-13  \\
            40.0  1.154955205026255e-13  \\
            41.00000000000001  1.0588792853161064e-13  \\
            42.0  1.1345135199815425e-13  \\
            43.0  1.1426901939994274e-13  \\
            44.0  1.2183244286648638e-13  \\
            45.0  1.1569993735307262e-13  \\
            46.0  1.2305894396916912e-13  \\
            47.0  1.1835735640888525e-13  \\
            48.0  1.2673844727721737e-13  \\
            49.0  1.253075293240875e-13  \\
            50.0  1.3348420334197247e-13  \\
            51.0  1.361416223977851e-13  \\
            52.0  1.4431829641567009e-13  \\
            53.0  1.4901988397595396e-13  \\
            54.0  1.5167730303176659e-13  \\
            55.00000000000001  1.592407264983102e-13  \\
            56.00000000000001  1.5617447374160334e-13  \\
            57.0  1.6046722760099296e-13  \\
            57.99999999999999  1.6189814555412283e-13  \\
            59.0  1.6271581295591134e-13  \\
            59.99999999999999  1.625113961054642e-13  \\
            60.99999999999999  1.6925715217021933e-13  \\
            62.0  1.700748195720078e-13  \\
            63.0  1.6762181736664232e-13  \\
            64.0  1.6803065106753657e-13  \\
            65.0  1.7763824303855143e-13  \\
            66.0  1.739587397305032e-13  \\
            67.0  1.7804707673944568e-13  \\
            68.0  1.8172658004749393e-13  \\
            69.0  1.8029566209436406e-13  \\
            70.0  1.774338261881043e-13  \\
            70.99999999999999  1.805000789448112e-13  \\
            72.0  1.805000789448112e-13  \\
            73.0  1.8131774634659968e-13  \\
            74.0  1.8274866429972955e-13  \\
            75.0  1.9276508997163868e-13  \\
            76.0  1.8745025186001342e-13  \\
            77.0  1.9317392367253293e-13  \\
            78.0  1.8867675296269618e-13  \\
            79.0  1.9521809217700417e-13  \\
            80.0  2.0237268194265353e-13  \\
            81.00000000000001  2.0809635375517302e-13  \\
            82.0  2.035991830453363e-13  \\
            83.0  2.005329302886294e-13  \\
            84.0  2.0135059769041791e-13  \\
            85.0  2.0911843800740864e-13  \\
            86.0  2.170906951748465e-13  \\
            87.0  2.1320677501635114e-13  \\
            88.0  2.1422885926858676e-13  \\
            89.0  2.1341119186679828e-13  \\
            90.0  2.0748310320383165e-13  \\
            91.0  2.1749952887574076e-13  \\
            92.0  2.2608503659452002e-13  \\
            93.0  2.1054935596053854e-13  \\
            94.0  2.144332761190339e-13  \\
            95.0  2.1545536037126952e-13  \\
            95.99999999999999  2.1647744462350514e-13  \\
            97.0  2.1647744462350514e-13  \\
            98.0  2.2158786588468326e-13  \\
            99.0  2.2240553328647177e-13  \\
            100.0  2.2894687250077976e-13  \\
            101.0  2.301733736034625e-13  \\
        }
        ;
\end{axis}
\begin{axis}[point meta max={nan}, point meta min={nan}, legend cell align={left}, legend columns={1}, title={}, title style={at={{(0.5,1)}}, anchor={south}, font={{\fontsize{14 pt}{18.2 pt}\selectfont}}, color={rgb,1:red,0.0;green,0.0;blue,0.0}, draw opacity={1.0}, rotate={0.0}, align={center}}, legend style={color={rgb,1:red,0.0;green,0.0;blue,0.0}, draw opacity={1.0}, line width={1}, solid, fill={rgb,1:red,1.0;green,1.0;blue,1.0}, fill opacity={1.0}, text opacity={1.0}, font={{\fontsize{8 pt}{10.4 pt}\selectfont}}, text={rgb,1:red,0.0;green,0.0;blue,0.0}, cells={anchor={center}}, at={(1.02, 1)}, anchor={north west}}, axis background/.style={fill={rgb,1:red,1.0;green,1.0;blue,1.0}, opacity={1.0}}, anchor={north west}, xshift={1.0mm}, yshift={-39.099999999999994mm}, width={150.4mm}, height={36.099999999999994mm}, scaled x ticks={false}, xlabel={$t/\tau$}, x tick style={color={rgb,1:red,0.0;green,0.0;blue,0.0}, opacity={1.0}}, x tick label style={color={rgb,1:red,0.0;green,0.0;blue,0.0}, opacity={1.0}, rotate={0}}, xlabel style={at={(ticklabel cs:0.5)}, anchor=near ticklabel, at={{(ticklabel cs:0.5)}}, anchor={near ticklabel}, font={{\fontsize{11 pt}{14.3 pt}\selectfont}}, color={rgb,1:red,0.0;green,0.0;blue,0.0}, draw opacity={1.0}, rotate={0.0}}, xmode={log}, log basis x={10}, xmajorgrids={true}, xmin={0.8707036374807742}, xmax={115.99813719881251}, xticklabels={{$10^{0}$,$10^{1}$,$10^{2}$}}, xtick={{1.0,10.0,100.0}}, xtick align={inside}, xticklabel style={font={{\fontsize{8 pt}{10.4 pt}\selectfont}}, color={rgb,1:red,0.0;green,0.0;blue,0.0}, draw opacity={1.0}, rotate={0.0}}, x grid style={color={rgb,1:red,0.0;green,0.0;blue,0.0}, draw opacity={0.1}, line width={0.5}, solid}, axis x line*={left}, x axis line style={color={rgb,1:red,0.0;green,0.0;blue,0.0}, draw opacity={1.0}, line width={1}, solid}, scaled y ticks={false}, ylabel={$\delta u^m$}, y tick style={color={rgb,1:red,0.0;green,0.0;blue,0.0}, opacity={1.0}}, y tick label style={color={rgb,1:red,0.0;green,0.0;blue,0.0}, opacity={1.0}, rotate={0}}, ylabel style={at={(ticklabel cs:0.5)}, anchor=near ticklabel, at={{(ticklabel cs:0.5)}}, anchor={near ticklabel}, font={{\fontsize{11 pt}{14.3 pt}\selectfont}}, color={rgb,1:red,0.0;green,0.0;blue,0.0}, draw opacity={1.0}, rotate={0.0}}, ymajorgrids={true}, ymin={-5.84297322513522e-14}, ymax={6.046605880151778e-14}, yticklabels={{$-5.00\times10^{^{-14}}$,$-2.50\times10^{^{-14}}$,$0$,$2.50\times10^{^{-14}}$,$5.00\times10^{^{-14}}$}}, ytick={{-5.0e-14,-2.5e-14,0.0,2.5e-14,5.0e-14}}, ytick align={inside}, yticklabel style={font={{\fontsize{8 pt}{10.4 pt}\selectfont}}, color={rgb,1:red,0.0;green,0.0;blue,0.0}, draw opacity={1.0}, rotate={0.0}}, y grid style={color={rgb,1:red,0.0;green,0.0;blue,0.0}, draw opacity={0.1}, line width={0.5}, solid}, axis y line*={left}, y axis line style={color={rgb,1:red,0.0;green,0.0;blue,0.0}, draw opacity={1.0}, line width={1}, solid}, colorbar={false}]
    \addplot[color={rgb,1:red,1.0;green,0.0;blue,0.0}, name path={8fe765bb-dfab-4022-9e4f-5d6cd0b35362}, draw opacity={1.0}, line width={1}, solid]
        table[row sep={\\}]
        {
            \\
            1.0  1.750929943592765e-14  \\
            2.0  -1.3032489921059785e-14  \\
            3.0  1.1440735426884544e-14  \\
            4.0  7.113152895845607e-15  \\
            5.0  1.4922698382892884e-15  \\
            6.000000000000001  1.1440735426884544e-15  \\
            7.0  1.3529913200489548e-14  \\
            8.0  2.885055020692624e-15  \\
            9.0  -1.2087385690143236e-14  \\
            10.0  -5.07371745018358e-15  \\
            11.000000000000002  3.278019411442137e-14  \\
            12.0  -3.969437769849507e-14  \\
            13.0  5.770110041385248e-15  \\
            14.0  -1.4922698382892883e-14  \\
            15.0  8.257226438534061e-15  \\
            15.999999999999998  -4.476809514867865e-16  \\
            17.0  3.233251316293458e-14  \\
            18.0  -9.550526965051445e-15  \\
            19.0  -1.989693117719051e-15  \\
            20.0  1.50967965306933e-14  \\
            21.000000000000004  -3.7058034317517325e-15  \\
            22.0  -2.570434796453299e-14  \\
            23.0  1.8302067787518834e-14  \\
            24.0  6.665471944358821e-15  \\
            25.0  2.1812010802995098e-14  \\
            26.0  2.5866010530347665e-15  \\
            27.0  2.2657630378025695e-14  \\
            28.000000000000004  -3.287967877030732e-14  \\
            29.0  6.81469892818775e-15  \\
            30.0  1.7658526419756578e-15  \\
            30.999999999999996  1.0644858179796924e-14  \\
            32.0  3.20838015232197e-14  \\
            33.0  -1.0794085163625853e-14  \\
            34.0  -1.9200538585988843e-14  \\
            35.0  -3.929643907495126e-15  \\
            35.99999999999999  4.1783555472100076e-15  \\
            37.0  -1.9747704193361582e-14  \\
            38.0  2.2881470853769087e-15  \\
            39.0  -6.3670179767009634e-15  \\
            40.0  2.1438943343422775e-14  \\
            41.00000000000001  8.505938078248944e-15  \\
            42.0  -5.506475703287474e-14  \\
            43.0  -3.138740893201803e-14  \\
            44.0  2.13892010154798e-15  \\
            45.0  6.0188216811001296e-15  \\
            46.0  2.5318844922974926e-14  \\
            47.0  3.765494225283304e-14  \\
            48.0  2.939771581429898e-14  \\
            49.0  6.764956600244774e-15  \\
            50.0  3.407349464093875e-14  \\
            51.0  3.743110177708965e-14  \\
            52.0  3.6510868710144586e-14  \\
            53.0  -1.1565091246741984e-14  \\
            54.0  8.306968766477038e-15  \\
            55.00000000000001  4.55142300678233e-15  \\
            56.00000000000001  1.84544036668442e-14  \\
            57.0  2.5343716086946415e-14  \\
            57.99999999999999  1.5942416105723896e-14  \\
            59.0  1.3977594151976333e-14  \\
            59.99999999999999  1.1515348918799009e-14  \\
            60.99999999999999  1.8342483428972502e-14  \\
            62.0  -9.774367440794839e-15  \\
            63.0  7.187766387760072e-15  \\
            64.0  4.00425739940959e-15  \\
            65.0  5.457976933543072e-14  \\
            66.0  4.2635392838123544e-14  \\
            67.0  3.855030415580661e-15  \\
            68.0  -7.411606863503465e-15  \\
            69.0  -3.9992831666152926e-14  \\
            70.0  4.027885005182504e-14  \\
            70.99999999999999  -2.5160291252656688e-14  \\
            72.0  -4.223900866232795e-14  \\
            73.0  5.7615605787700495e-15  \\
            74.0  1.4473851595594933e-15  \\
            75.0  -3.446521547348969e-14  \\
            76.0  -1.856943280021233e-14  \\
            77.0  4.080269894297451e-14  \\
            78.0  -5.614665266563448e-15  \\
            79.0  -2.179335743001648e-15  \\
            80.0  2.0751877438710417e-14  \\
            81.00000000000001  2.173117952008776e-15  \\
            82.0  -1.6825342426711725e-14  \\
            83.0  7.57948722031101e-15  \\
            84.0  -8.975381298210782e-15  \\
            85.0  2.4631779018262565e-14  \\
            86.0  -2.5328171609464235e-14  \\
            87.0  1.8199474236136447e-14  \\
            88.0  3.2773976323428496e-14  \\
            89.0  4.968636782404043e-14  \\
            90.0  -1.7658526419756577e-14  \\
            91.0  4.1733813144157096e-14  \\
            92.0  4.290897564180991e-14  \\
            93.0  -1.5556913064165832e-14  \\
            94.0  3.675958034985947e-14  \\
            95.0  5.710108358304033e-14  \\
            95.99999999999999  -3.366622933090563e-14  \\
            97.0  -4.130478556564893e-14  \\
            98.0  1.8995351483224067e-14  \\
            99.0  -1.0408582122067786e-14  \\
            100.0  -3.031794888124404e-14  \\
            101.0  -9.270726370372203e-15  \\
        }
        ;
    \addplot[color={rgb,1:red,0.0;green,0.0;blue,1.0}, name path={a8c8cbeb-8c49-416a-8615-f814c496431b}, draw opacity={1.0}, line width={1}, solid]
        table[row sep={\\}]
        {
            \\
            1.0  4.567439002177944e-15  \\
            2.0  2.536046550859642e-15  \\
            3.0  -3.8328159458835894e-17  \\
            4.0  3.353713952648141e-15  \\
            5.0  -1.3798137405180922e-15  \\
            6.000000000000001  4.1330531949778044e-15  \\
            7.0  2.9640443314833093e-15  \\
            8.0  5.8642083972018924e-15  \\
            9.0  1.5970066441181623e-15  \\
            10.0  3.794487786424754e-15  \\
            11.000000000000002  3.1812372350833795e-15  \\
            12.0  4.720751640013288e-15  \\
            13.0  4.548274922448526e-15  \\
            14.0  5.647015493601822e-15  \\
            15.0  2.8362837999538565e-15  \\
            15.999999999999998  3.634787122012938e-15  \\
            17.0  2.114436796812447e-15  \\
            18.0  1.6608869098828888e-16  \\
            19.0  5.672567599907713e-15  \\
            20.0  2.804343667071493e-15  \\
            21.000000000000004  6.036685114766654e-15  \\
            22.0  6.324146310707923e-16  \\
            23.0  -2.2358093017654274e-15  \\
            24.0  3.8328159458835895e-15  \\
            25.0  3.3664900058010863e-15  \\
            26.0  5.084869154872229e-15  \\
            27.0  5.040152968836921e-15  \\
            28.000000000000004  3.1301330224715983e-15  \\
            29.0  7.346230562943547e-16  \\
            30.0  3.7050554143541366e-16  \\
            30.999999999999996  2.2421973283419e-15  \\
            32.0  3.538966723365848e-15  \\
            33.0  1.4309179531298734e-15  \\
            34.0  2.0313924513183025e-15  \\
            35.0  2.734075374730294e-15  \\
            35.99999999999999  -1.2456651824121667e-15  \\
            37.0  4.618543214789726e-15  \\
            38.0  3.0438946636892175e-15  \\
            39.0  2.4242560857713707e-15  \\
            40.0  -7.537871360237726e-16  \\
            41.00000000000001  2.076108637353611e-15  \\
            42.0  5.7811640517077475e-16  \\
            43.0  5.966416822425455e-15  \\
            44.0  1.210531036241567e-15  \\
            45.0  7.537871360237726e-16  \\
            46.0  2.8762089660568106e-15  \\
            47.0  2.1799140692212917e-15  \\
            48.0  2.775597547477366e-15  \\
            49.0  2.6670010956773313e-15  \\
            50.0  2.764418500968539e-15  \\
            51.0  3.9861285837189335e-15  \\
            52.0  6.1349010233799205e-15  \\
            53.0  4.625330493027227e-15  \\
            54.0  3.358105720919466e-15  \\
            55.00000000000001  6.69067493130105e-15  \\
            56.00000000000001  -6.084595314090199e-16  \\
            57.0  1.9104191980263518e-15  \\
            57.99999999999999  1.0061141857944422e-15  \\
            59.0  2.714911295000876e-15  \\
            59.99999999999999  3.45911639115994e-15  \\
            60.99999999999999  1.3494706142798472e-15  \\
            62.0  1.8908558666359043e-15  \\
            63.0  -1.054024385117987e-15  \\
            64.0  2.5041064179772784e-15  \\
            65.0  4.147426254774867e-15  \\
            66.0  -1.86849777361825e-16  \\
            67.0  7.521901293796545e-16  \\
            68.0  4.0484118428395414e-15  \\
            69.0  2.379539899736062e-16  \\
            70.0  -1.485216179029891e-15  \\
            70.99999999999999  2.8746119594126925e-16  \\
            72.0  2.967238344771546e-15  \\
            73.0  2.213451208747773e-15  \\
            74.0  1.4979922321828362e-15  \\
            75.0  4.660065387536798e-15  \\
            76.0  -1.210531036241567e-15  \\
            77.0  4.372604191595529e-15  \\
            78.0  -2.076108637353611e-15  \\
            79.0  4.01487470331306e-15  \\
            80.0  7.662437878478942e-15  \\
            81.00000000000001  1.6289467770005256e-15  \\
            82.0  -1.2424711691239304e-15  \\
            83.0  1.6768569763240704e-15  \\
            84.0  -1.050830371829751e-15  \\
            85.0  5.589523254413568e-15  \\
            86.0  2.644643002659677e-15  \\
            87.0  3.920651311310089e-15  \\
            88.0  -3.4175942184128676e-16  \\
            89.0  -2.3731518731595895e-15  \\
            90.0  -1.1546358036974314e-15  \\
            91.0  5.145555407348719e-15  \\
            92.0  3.7641446601865084e-15  \\
            93.0  -4.163396321216049e-15  \\
            94.0  1.740737242088797e-16  \\
            95.0  -2.980014397924491e-15  \\
            95.99999999999999  1.88606484670355e-15  \\
            97.0  1.2935753817357116e-15  \\
            98.0  4.604170154992662e-15  \\
            99.0  4.8165720386603774e-15  \\
            100.0  2.3843309196684166e-15  \\
            101.0  1.5011862454710726e-15  \\
        }
        ;
    \addplot[color={rgb,1:red,0.2422;green,0.6433;blue,0.3044}, name path={d370584e-6773-4f6e-8d90-98b8e3f6b52f}, only marks, draw opacity={1.0}, line width={0}, solid, mark={*}, mark size={1.5 pt}, mark repeat={1}, mark options={color={rgb,1:red,0.0;green,0.0;blue,0.0}, draw opacity={0.5}, fill={rgb,1:red,1.0;green,0.0;blue,0.0}, fill opacity={0.5}, line width={0.75}, rotate={0}, solid}]
        table[row sep={\\}]
        {
            \\
            1.0  1.750929943592765e-14  \\
            2.0  -1.3032489921059785e-14  \\
            3.0  1.1440735426884544e-14  \\
            4.0  7.113152895845607e-15  \\
            5.0  1.4922698382892884e-15  \\
            6.000000000000001  1.1440735426884544e-15  \\
            7.0  1.3529913200489548e-14  \\
            8.0  2.885055020692624e-15  \\
            9.0  -1.2087385690143236e-14  \\
            10.0  -5.07371745018358e-15  \\
            11.000000000000002  3.278019411442137e-14  \\
            12.0  -3.969437769849507e-14  \\
            13.0  5.770110041385248e-15  \\
            14.0  -1.4922698382892883e-14  \\
            15.0  8.257226438534061e-15  \\
            15.999999999999998  -4.476809514867865e-16  \\
            17.0  3.233251316293458e-14  \\
            18.0  -9.550526965051445e-15  \\
            19.0  -1.989693117719051e-15  \\
            20.0  1.50967965306933e-14  \\
            21.000000000000004  -3.7058034317517325e-15  \\
            22.0  -2.570434796453299e-14  \\
            23.0  1.8302067787518834e-14  \\
            24.0  6.665471944358821e-15  \\
            25.0  2.1812010802995098e-14  \\
            26.0  2.5866010530347665e-15  \\
            27.0  2.2657630378025695e-14  \\
            28.000000000000004  -3.287967877030732e-14  \\
            29.0  6.81469892818775e-15  \\
            30.0  1.7658526419756578e-15  \\
            30.999999999999996  1.0644858179796924e-14  \\
            32.0  3.20838015232197e-14  \\
            33.0  -1.0794085163625853e-14  \\
            34.0  -1.9200538585988843e-14  \\
            35.0  -3.929643907495126e-15  \\
            35.99999999999999  4.1783555472100076e-15  \\
            37.0  -1.9747704193361582e-14  \\
            38.0  2.2881470853769087e-15  \\
            39.0  -6.3670179767009634e-15  \\
            40.0  2.1438943343422775e-14  \\
            41.00000000000001  8.505938078248944e-15  \\
            42.0  -5.506475703287474e-14  \\
            43.0  -3.138740893201803e-14  \\
            44.0  2.13892010154798e-15  \\
            45.0  6.0188216811001296e-15  \\
            46.0  2.5318844922974926e-14  \\
            47.0  3.765494225283304e-14  \\
            48.0  2.939771581429898e-14  \\
            49.0  6.764956600244774e-15  \\
            50.0  3.407349464093875e-14  \\
            51.0  3.743110177708965e-14  \\
            52.0  3.6510868710144586e-14  \\
            53.0  -1.1565091246741984e-14  \\
            54.0  8.306968766477038e-15  \\
            55.00000000000001  4.55142300678233e-15  \\
            56.00000000000001  1.84544036668442e-14  \\
            57.0  2.5343716086946415e-14  \\
            57.99999999999999  1.5942416105723896e-14  \\
            59.0  1.3977594151976333e-14  \\
            59.99999999999999  1.1515348918799009e-14  \\
            60.99999999999999  1.8342483428972502e-14  \\
            62.0  -9.774367440794839e-15  \\
            63.0  7.187766387760072e-15  \\
            64.0  4.00425739940959e-15  \\
            65.0  5.457976933543072e-14  \\
            66.0  4.2635392838123544e-14  \\
            67.0  3.855030415580661e-15  \\
            68.0  -7.411606863503465e-15  \\
            69.0  -3.9992831666152926e-14  \\
            70.0  4.027885005182504e-14  \\
            70.99999999999999  -2.5160291252656688e-14  \\
            72.0  -4.223900866232795e-14  \\
            73.0  5.7615605787700495e-15  \\
            74.0  1.4473851595594933e-15  \\
            75.0  -3.446521547348969e-14  \\
            76.0  -1.856943280021233e-14  \\
            77.0  4.080269894297451e-14  \\
            78.0  -5.614665266563448e-15  \\
            79.0  -2.179335743001648e-15  \\
            80.0  2.0751877438710417e-14  \\
            81.00000000000001  2.173117952008776e-15  \\
            82.0  -1.6825342426711725e-14  \\
            83.0  7.57948722031101e-15  \\
            84.0  -8.975381298210782e-15  \\
            85.0  2.4631779018262565e-14  \\
            86.0  -2.5328171609464235e-14  \\
            87.0  1.8199474236136447e-14  \\
            88.0  3.2773976323428496e-14  \\
            89.0  4.968636782404043e-14  \\
            90.0  -1.7658526419756577e-14  \\
            91.0  4.1733813144157096e-14  \\
            92.0  4.290897564180991e-14  \\
            93.0  -1.5556913064165832e-14  \\
            94.0  3.675958034985947e-14  \\
            95.0  5.710108358304033e-14  \\
            95.99999999999999  -3.366622933090563e-14  \\
            97.0  -4.130478556564893e-14  \\
            98.0  1.8995351483224067e-14  \\
            99.0  -1.0408582122067786e-14  \\
            100.0  -3.031794888124404e-14  \\
            101.0  -9.270726370372203e-15  \\
        }
        ;
    \addplot[color={rgb,1:red,0.7644;green,0.4441;blue,0.8243}, name path={f8b852a0-678f-4c88-a49c-b384d5c92581}, only marks, draw opacity={1.0}, line width={0}, solid, mark={*}, mark size={1.5 pt}, mark repeat={1}, mark options={color={rgb,1:red,0.0;green,0.0;blue,0.0}, draw opacity={0.5}, fill={rgb,1:red,0.0;green,0.0;blue,1.0}, fill opacity={0.5}, line width={0.75}, rotate={0}, solid}]
        table[row sep={\\}]
        {
            \\
            1.0  4.567439002177944e-15  \\
            2.0  2.536046550859642e-15  \\
            3.0  -3.8328159458835894e-17  \\
            4.0  3.353713952648141e-15  \\
            5.0  -1.3798137405180922e-15  \\
            6.000000000000001  4.1330531949778044e-15  \\
            7.0  2.9640443314833093e-15  \\
            8.0  5.8642083972018924e-15  \\
            9.0  1.5970066441181623e-15  \\
            10.0  3.794487786424754e-15  \\
            11.000000000000002  3.1812372350833795e-15  \\
            12.0  4.720751640013288e-15  \\
            13.0  4.548274922448526e-15  \\
            14.0  5.647015493601822e-15  \\
            15.0  2.8362837999538565e-15  \\
            15.999999999999998  3.634787122012938e-15  \\
            17.0  2.114436796812447e-15  \\
            18.0  1.6608869098828888e-16  \\
            19.0  5.672567599907713e-15  \\
            20.0  2.804343667071493e-15  \\
            21.000000000000004  6.036685114766654e-15  \\
            22.0  6.324146310707923e-16  \\
            23.0  -2.2358093017654274e-15  \\
            24.0  3.8328159458835895e-15  \\
            25.0  3.3664900058010863e-15  \\
            26.0  5.084869154872229e-15  \\
            27.0  5.040152968836921e-15  \\
            28.000000000000004  3.1301330224715983e-15  \\
            29.0  7.346230562943547e-16  \\
            30.0  3.7050554143541366e-16  \\
            30.999999999999996  2.2421973283419e-15  \\
            32.0  3.538966723365848e-15  \\
            33.0  1.4309179531298734e-15  \\
            34.0  2.0313924513183025e-15  \\
            35.0  2.734075374730294e-15  \\
            35.99999999999999  -1.2456651824121667e-15  \\
            37.0  4.618543214789726e-15  \\
            38.0  3.0438946636892175e-15  \\
            39.0  2.4242560857713707e-15  \\
            40.0  -7.537871360237726e-16  \\
            41.00000000000001  2.076108637353611e-15  \\
            42.0  5.7811640517077475e-16  \\
            43.0  5.966416822425455e-15  \\
            44.0  1.210531036241567e-15  \\
            45.0  7.537871360237726e-16  \\
            46.0  2.8762089660568106e-15  \\
            47.0  2.1799140692212917e-15  \\
            48.0  2.775597547477366e-15  \\
            49.0  2.6670010956773313e-15  \\
            50.0  2.764418500968539e-15  \\
            51.0  3.9861285837189335e-15  \\
            52.0  6.1349010233799205e-15  \\
            53.0  4.625330493027227e-15  \\
            54.0  3.358105720919466e-15  \\
            55.00000000000001  6.69067493130105e-15  \\
            56.00000000000001  -6.084595314090199e-16  \\
            57.0  1.9104191980263518e-15  \\
            57.99999999999999  1.0061141857944422e-15  \\
            59.0  2.714911295000876e-15  \\
            59.99999999999999  3.45911639115994e-15  \\
            60.99999999999999  1.3494706142798472e-15  \\
            62.0  1.8908558666359043e-15  \\
            63.0  -1.054024385117987e-15  \\
            64.0  2.5041064179772784e-15  \\
            65.0  4.147426254774867e-15  \\
            66.0  -1.86849777361825e-16  \\
            67.0  7.521901293796545e-16  \\
            68.0  4.0484118428395414e-15  \\
            69.0  2.379539899736062e-16  \\
            70.0  -1.485216179029891e-15  \\
            70.99999999999999  2.8746119594126925e-16  \\
            72.0  2.967238344771546e-15  \\
            73.0  2.213451208747773e-15  \\
            74.0  1.4979922321828362e-15  \\
            75.0  4.660065387536798e-15  \\
            76.0  -1.210531036241567e-15  \\
            77.0  4.372604191595529e-15  \\
            78.0  -2.076108637353611e-15  \\
            79.0  4.01487470331306e-15  \\
            80.0  7.662437878478942e-15  \\
            81.00000000000001  1.6289467770005256e-15  \\
            82.0  -1.2424711691239304e-15  \\
            83.0  1.6768569763240704e-15  \\
            84.0  -1.050830371829751e-15  \\
            85.0  5.589523254413568e-15  \\
            86.0  2.644643002659677e-15  \\
            87.0  3.920651311310089e-15  \\
            88.0  -3.4175942184128676e-16  \\
            89.0  -2.3731518731595895e-15  \\
            90.0  -1.1546358036974314e-15  \\
            91.0  5.145555407348719e-15  \\
            92.0  3.7641446601865084e-15  \\
            93.0  -4.163396321216049e-15  \\
            94.0  1.740737242088797e-16  \\
            95.0  -2.980014397924491e-15  \\
            95.99999999999999  1.88606484670355e-15  \\
            97.0  1.2935753817357116e-15  \\
            98.0  4.604170154992662e-15  \\
            99.0  4.8165720386603774e-15  \\
            100.0  2.3843309196684166e-15  \\
            101.0  1.5011862454710726e-15  \\
        }
        ;
\end{axis}
\begin{axis}[point meta max={nan}, point meta min={nan}, legend cell align={left}, legend columns={1}, title={}, title style={at={{(0.5,1)}}, anchor={south}, font={{\fontsize{14 pt}{18.2 pt}\selectfont}}, color={rgb,1:red,0.0;green,0.0;blue,0.0}, draw opacity={1.0}, rotate={0.0}, align={center}}, legend style={color={rgb,1:red,0.0;green,0.0;blue,0.0}, draw opacity={1.0}, line width={1}, solid, fill={rgb,1:red,1.0;green,1.0;blue,1.0}, fill opacity={1.0}, text opacity={1.0}, font={{\fontsize{8 pt}{10.4 pt}\selectfont}}, text={rgb,1:red,0.0;green,0.0;blue,0.0}, cells={anchor={center}}, at={(1.02, 1)}, anchor={north west}}, axis background/.style={fill={rgb,1:red,1.0;green,1.0;blue,1.0}, opacity={1.0}}, anchor={north west}, xshift={1.0mm}, yshift={-77.19999999999999mm}, width={150.4mm}, height={36.099999999999994mm}, scaled x ticks={false}, xlabel={$t/\tau$}, x tick style={color={rgb,1:red,0.0;green,0.0;blue,0.0}, opacity={1.0}}, x tick label style={color={rgb,1:red,0.0;green,0.0;blue,0.0}, opacity={1.0}, rotate={0}}, xlabel style={at={(ticklabel cs:0.5)}, anchor=near ticklabel, at={{(ticklabel cs:0.5)}}, anchor={near ticklabel}, font={{\fontsize{11 pt}{14.3 pt}\selectfont}}, color={rgb,1:red,0.0;green,0.0;blue,0.0}, draw opacity={1.0}, rotate={0.0}}, xmode={log}, log basis x={10}, xmajorgrids={true}, xmin={0.8704463225992706}, xmax={117.18126362509537}, xticklabels={{$10^{0}$,$10^{1}$,$10^{2}$}}, xtick={{1.0,10.0,100.0}}, xtick align={inside}, xticklabel style={font={{\fontsize{8 pt}{10.4 pt}\selectfont}}, color={rgb,1:red,0.0;green,0.0;blue,0.0}, draw opacity={1.0}, rotate={0.0}}, x grid style={color={rgb,1:red,0.0;green,0.0;blue,0.0}, draw opacity={0.1}, line width={0.5}, solid}, axis x line*={left}, x axis line style={color={rgb,1:red,0.0;green,0.0;blue,0.0}, draw opacity={1.0}, line width={1}, solid}, scaled y ticks={false}, ylabel={$E^m$}, y tick style={color={rgb,1:red,0.0;green,0.0;blue,0.0}, opacity={1.0}}, y tick label style={color={rgb,1:red,0.0;green,0.0;blue,0.0}, opacity={1.0}, rotate={0}}, ylabel style={at={(ticklabel cs:0.5)}, anchor=near ticklabel, at={{(ticklabel cs:0.5)}}, anchor={near ticklabel}, font={{\fontsize{11 pt}{14.3 pt}\selectfont}}, color={rgb,1:red,0.0;green,0.0;blue,0.0}, draw opacity={1.0}, rotate={0.0}}, ymajorgrids={true}, ymin={0.23585576615676734}, ymax={1.0339436136571276}, yticklabels={{$0.4$,$0.6$,$0.8$,$1.0$}}, ytick={{0.4,0.6000000000000001,0.8,1.0}}, ytick align={inside}, yticklabel style={font={{\fontsize{8 pt}{10.4 pt}\selectfont}}, color={rgb,1:red,0.0;green,0.0;blue,0.0}, draw opacity={1.0}, rotate={0.0}}, y grid style={color={rgb,1:red,0.0;green,0.0;blue,0.0}, draw opacity={0.1}, line width={0.5}, solid}, axis y line*={left}, y axis line style={color={rgb,1:red,0.0;green,0.0;blue,0.0}, draw opacity={1.0}, line width={1}, solid}, colorbar={false}]
    \addplot[color={rgb,1:red,1.0;green,0.0;blue,0.0}, name path={5c50a9ca-d16c-415b-9eda-a4093c6da9ae}, draw opacity={1.0}, line width={1}, solid]
        table[row sep={\\}]
        {
            \\
            1.0  1.01135622174674  \\
            2.0  0.7168059286278647  \\
            3.0  0.6996797675978398  \\
            4.0  0.688027586066897  \\
            5.0  0.6762621705500135  \\
            6.0  0.6640123379717233  \\
            7.0  0.6514087166175337  \\
            8.0  0.6386683826230077  \\
            9.0  0.6260155648516823  \\
            10.0  0.6136485193489647  \\
            11.0  0.6017234170779991  \\
            12.0  0.5903508230741573  \\
            13.0  0.5796000473040726  \\
            14.000000000000002  0.569506451594987  \\
            15.000000000000002  0.5600788657238133  \\
            16.0  0.5513060655086555  \\
            17.0  0.5431625107176021  \\
            18.000000000000004  0.5356139122304764  \\
            19.0  0.5286219289381372  \\
            20.0  0.5221466596453406  \\
            21.0  0.5161469922665837  \\
            22.0  0.5105802242040551  \\
            23.0  0.5054021934858668  \\
            24.0  0.5005683189667889  \\
            25.0  0.49603549784659917  \\
            26.0  0.4917644600371073  \\
            27.000000000000004  0.48772168049906073  \\
            28.0  0.48387994152456887  \\
            29.000000000000004  0.4802173137174501  \\
            30.0  0.4767152032547668  \\
            31.0  0.47335671944668495  \\
            32.0  0.4701263717290625  \\
            33.0  0.4670108331054088  \\
            34.0  0.4639993801316949  \\
            35.00000000000001  0.46108312734894746  \\
            35.99999999999999  0.4582538752227135  \\
            37.0  0.45550396978479296  \\
            38.0  0.4528271519485793  \\
            39.0  0.45021913374545364  \\
            40.0  0.4476771904630225  \\
            41.0  0.44519929955931936  \\
            42.0  0.4427838166221053  \\
            43.0  0.44043004217074005  \\
            44.0  0.43813898406240337  \\
            45.0  0.4359131931839157  \\
            46.0  0.43375521055401833  \\
            47.0  0.43166543595644774  \\
            48.00000000000001  0.42964091449861075  \\
            49.0  0.42767575194205165  \\
            49.99999999999999  0.4257625126236845  \\
            51.0  0.42389365746971797  \\
            52.0  0.42206271011623886  \\
            53.00000000000001  0.42026522351718293  \\
            54.0  0.4184994667981517  \\
            55.0  0.41676646549475127  \\
            56.00000000000001  0.41506901435425214  \\
            57.00000000000001  0.4134098081286251  \\
            57.99999999999999  0.4117896510354008  \\
            59.0  0.4102068272307848  \\
            59.99999999999999  0.40865772963534697  \\
            61.0  0.40713795859181373  \\
            62.0  0.40564327451257887  \\
            63.0  0.40417033885288917  \\
            64.0  0.40271731661051396  \\
            65.0  0.4012842037284867  \\
            66.0  0.3998725661740698  \\
            67.0  0.39848455766733365  \\
            68.00000000000001  0.3971216378803723  \\
            69.00000000000001  0.39578379000701247  \\
            70.0  0.39446965578596027  \\
            70.99999999999999  0.3931772351015078  \\
            71.99999999999999  0.39190451283538524  \\
            73.0  0.39064971296198764  \\
            74.0  0.3894113219854961  \\
            75.0  0.3881881909329506  \\
            76.0  0.3869798683492954  \\
            77.0  0.38578700011664613  \\
            78.0  0.3846114170471341  \\
            79.0  0.3834556293830853  \\
            80.0  0.38232187419601854  \\
            81.0  0.38121128844228186  \\
            82.0  0.38012370525312367  \\
            83.0  0.3790580239824326  \\
            84.0  0.3780127406501853  \\
            85.0  0.3769863410561816  \\
            86.0  0.3759775059540227  \\
            87.0  0.3749851823781874  \\
            88.0  0.37400857900634327  \\
            89.0  0.37304713138471424  \\
            90.0  0.37210047037221883  \\
            91.0  0.37116840709241766  \\
            92.0  0.37025092356941647  \\
            93.0  0.3693481432015775  \\
            94.0  0.36846025994291187  \\
            95.00000000000001  0.36758742845798215  \\
            96.0  0.36672964589718954  \\
            97.0  0.36588666988120633  \\
            97.99999999999999  0.3650580041143552  \\
            98.99999999999999  0.3642429487309649  \\
            100.0  0.36344068028677634  \\
            101.0  0.36265031793947916  \\
            102.0  0.3618709495110627  \\
        }
        ;
    \addplot[color={rgb,1:red,0.0;green,0.0;blue,1.0}, name path={eb6bd54a-a8b8-4d4d-8146-9ef731233c71}, draw opacity={1.0}, line width={1}, solid]
        table[row sep={\\}]
        {
            \\
            1.0  0.691486688773568  \\
            2.0  0.494915508732498  \\
            3.0  0.48319778248753004  \\
            4.0  0.4752290908874254  \\
            5.0  0.46725393170868934  \\
            6.0  0.4590113203712845  \\
            7.0  0.4505666347740428  \\
            8.0  0.44203862476740124  \\
            9.0  0.4335567323124081  \\
            10.0  0.42524603725559834  \\
            11.0  0.41721506295710015  \\
            12.0  0.40954498983276116  \\
            13.0  0.4022845472537381  \\
            14.000000000000002  0.39545367158532546  \\
            15.000000000000002  0.3890533378881676  \\
            16.0  0.38307496470536057  \\
            17.0  0.3775046246263573  \\
            18.000000000000004  0.3723225685335079  \\
            19.0  0.36750235643075446  \\
            20.0  0.3630130924491887  \\
            21.0  0.3588241133614915  \\
            22.0  0.3549087753166257  \\
            23.0  0.3512451901498941  \\
            24.0  0.34781454705219034  \\
            25.0  0.3445990039469754  \\
            26.0  0.34158062936266453  \\
            27.000000000000004  0.3387418251992612  \\
            28.0  0.33606683788459113  \\
            29.000000000000004  0.3335432271816649  \\
            30.0  0.3311617594043113  \\
            31.0  0.32891411999877534  \\
            32.0  0.3267901102587565  \\
            33.0  0.32477695885080404  \\
            34.0  0.32286107401732655  \\
            35.00000000000001  0.32103010837708035  \\
            35.99999999999999  0.3192737868742977  \\
            37.0  0.31758366061854804  \\
            38.0  0.315952499787298  \\
            39.0  0.3143737597408235  \\
            40.0  0.31284126279557234  \\
            41.0  0.31134911561310585  \\
            42.0  0.3098918411817597  \\
            43.0  0.30846466639325953  \\
            44.0  0.3070638429381533  \\
            45.0  0.30568681418939153  \\
            46.0  0.304332059737034  \\
            47.0  0.30299865233061085  \\
            48.00000000000001  0.30168588685252157  \\
            49.0  0.3003934482597056  \\
            49.99999999999999  0.2991221670571698  \\
            51.0  0.29787467429833153  \\
            52.0  0.29665492501499824  \\
            53.00000000000001  0.29546633552847185  \\
            54.0  0.29430986991922276  \\
            55.0  0.2931837724969536  \\
            56.00000000000001  0.29208484411627766  \\
            57.00000000000001  0.2910098050678055  \\
            57.99999999999999  0.28995596837180754  \\
            59.0  0.288921411914212  \\
            59.99999999999999  0.2879049657399  \\
            61.0  0.2869061199522267  \\
            62.0  0.2859248455358481  \\
            63.0  0.28496133151195346  \\
            64.0  0.28401569952030564  \\
            65.0  0.2830877909265195  \\
            66.0  0.28217709538926794  \\
            67.0  0.28128282283990974  \\
            68.00000000000001  0.2804040652776229  \\
            69.00000000000001  0.27953998080416825  \\
            70.0  0.2786899477263606  \\
            70.99999999999999  0.27785365591901934  \\
            71.99999999999999  0.27703111701305283  \\
            73.0  0.27622259121444004  \\
            74.0  0.27542845209746497  \\
            75.0  0.2746490341731904  \\
            76.0  0.273884515242367  \\
            77.0  0.2731348675502015  \\
            78.0  0.27239987867327364  \\
            79.0  0.2716792162343961  \\
            80.0  0.2709725031779559  \\
            81.0  0.2702793793625517  \\
            82.0  0.2695995395253314  \\
            83.0  0.2689327486741843  \\
            84.0  0.2682788408765248  \\
            85.0  0.26763770736728026  \\
            86.0  0.2670092775567047  \\
            87.0  0.2663934944790995  \\
            88.0  0.2657902859497693  \\
            89.0  0.2651995340250054  \\
            90.0  0.2646210469020738  \\
            91.0  0.26405453769091564  \\
            92.0  0.2634996130793409  \\
            93.0  0.2629557725499573  \\
            94.0  0.2624224167227208  \\
            95.00000000000001  0.26189886238568943  \\
            96.0  0.26138436183161096  \\
            97.0  0.26087812471946564  \\
            97.99999999999999  0.26037934127392337  \\
            98.99999999999999  0.2598872058975528  \\
            100.0  0.25940094010729453  \\
            101.0  0.2589198131703238  \\
            102.0  0.2584431580671549  \\
        }
        ;
    \addplot[color={rgb,1:red,0.2422;green,0.6433;blue,0.3044}, name path={9aedccd2-9994-42b4-a795-26ee75230969}, only marks, draw opacity={1.0}, line width={0}, solid, mark={*}, mark size={1.5 pt}, mark repeat={1}, mark options={color={rgb,1:red,0.0;green,0.0;blue,0.0}, draw opacity={0.5}, fill={rgb,1:red,1.0;green,0.0;blue,0.0}, fill opacity={0.5}, line width={0.75}, rotate={0}, solid}]
        table[row sep={\\}]
        {
            \\
            1.0  1.01135622174674  \\
            2.0  0.7168059286278647  \\
            3.0  0.6996797675978398  \\
            4.0  0.688027586066897  \\
            5.0  0.6762621705500135  \\
            6.0  0.6640123379717233  \\
            7.0  0.6514087166175337  \\
            8.0  0.6386683826230077  \\
            9.0  0.6260155648516823  \\
            10.0  0.6136485193489647  \\
            11.0  0.6017234170779991  \\
            12.0  0.5903508230741573  \\
            13.0  0.5796000473040726  \\
            14.000000000000002  0.569506451594987  \\
            15.000000000000002  0.5600788657238133  \\
            16.0  0.5513060655086555  \\
            17.0  0.5431625107176021  \\
            18.000000000000004  0.5356139122304764  \\
            19.0  0.5286219289381372  \\
            20.0  0.5221466596453406  \\
            21.0  0.5161469922665837  \\
            22.0  0.5105802242040551  \\
            23.0  0.5054021934858668  \\
            24.0  0.5005683189667889  \\
            25.0  0.49603549784659917  \\
            26.0  0.4917644600371073  \\
            27.000000000000004  0.48772168049906073  \\
            28.0  0.48387994152456887  \\
            29.000000000000004  0.4802173137174501  \\
            30.0  0.4767152032547668  \\
            31.0  0.47335671944668495  \\
            32.0  0.4701263717290625  \\
            33.0  0.4670108331054088  \\
            34.0  0.4639993801316949  \\
            35.00000000000001  0.46108312734894746  \\
            35.99999999999999  0.4582538752227135  \\
            37.0  0.45550396978479296  \\
            38.0  0.4528271519485793  \\
            39.0  0.45021913374545364  \\
            40.0  0.4476771904630225  \\
            41.0  0.44519929955931936  \\
            42.0  0.4427838166221053  \\
            43.0  0.44043004217074005  \\
            44.0  0.43813898406240337  \\
            45.0  0.4359131931839157  \\
            46.0  0.43375521055401833  \\
            47.0  0.43166543595644774  \\
            48.00000000000001  0.42964091449861075  \\
            49.0  0.42767575194205165  \\
            49.99999999999999  0.4257625126236845  \\
            51.0  0.42389365746971797  \\
            52.0  0.42206271011623886  \\
            53.00000000000001  0.42026522351718293  \\
            54.0  0.4184994667981517  \\
            55.0  0.41676646549475127  \\
            56.00000000000001  0.41506901435425214  \\
            57.00000000000001  0.4134098081286251  \\
            57.99999999999999  0.4117896510354008  \\
            59.0  0.4102068272307848  \\
            59.99999999999999  0.40865772963534697  \\
            61.0  0.40713795859181373  \\
            62.0  0.40564327451257887  \\
            63.0  0.40417033885288917  \\
            64.0  0.40271731661051396  \\
            65.0  0.4012842037284867  \\
            66.0  0.3998725661740698  \\
            67.0  0.39848455766733365  \\
            68.00000000000001  0.3971216378803723  \\
            69.00000000000001  0.39578379000701247  \\
            70.0  0.39446965578596027  \\
            70.99999999999999  0.3931772351015078  \\
            71.99999999999999  0.39190451283538524  \\
            73.0  0.39064971296198764  \\
            74.0  0.3894113219854961  \\
            75.0  0.3881881909329506  \\
            76.0  0.3869798683492954  \\
            77.0  0.38578700011664613  \\
            78.0  0.3846114170471341  \\
            79.0  0.3834556293830853  \\
            80.0  0.38232187419601854  \\
            81.0  0.38121128844228186  \\
            82.0  0.38012370525312367  \\
            83.0  0.3790580239824326  \\
            84.0  0.3780127406501853  \\
            85.0  0.3769863410561816  \\
            86.0  0.3759775059540227  \\
            87.0  0.3749851823781874  \\
            88.0  0.37400857900634327  \\
            89.0  0.37304713138471424  \\
            90.0  0.37210047037221883  \\
            91.0  0.37116840709241766  \\
            92.0  0.37025092356941647  \\
            93.0  0.3693481432015775  \\
            94.0  0.36846025994291187  \\
            95.00000000000001  0.36758742845798215  \\
            96.0  0.36672964589718954  \\
            97.0  0.36588666988120633  \\
            97.99999999999999  0.3650580041143552  \\
            98.99999999999999  0.3642429487309649  \\
            100.0  0.36344068028677634  \\
            101.0  0.36265031793947916  \\
            102.0  0.3618709495110627  \\
        }
        ;
    \addplot[color={rgb,1:red,0.7644;green,0.4441;blue,0.8243}, name path={ef618aa3-d10b-4495-a150-dd16d89860c0}, only marks, draw opacity={1.0}, line width={0}, solid, mark={*}, mark size={1.5 pt}, mark repeat={1}, mark options={color={rgb,1:red,0.0;green,0.0;blue,0.0}, draw opacity={0.5}, fill={rgb,1:red,0.0;green,0.0;blue,1.0}, fill opacity={0.5}, line width={0.75}, rotate={0}, solid}]
        table[row sep={\\}]
        {
            \\
            1.0  0.691486688773568  \\
            2.0  0.494915508732498  \\
            3.0  0.48319778248753004  \\
            4.0  0.4752290908874254  \\
            5.0  0.46725393170868934  \\
            6.0  0.4590113203712845  \\
            7.0  0.4505666347740428  \\
            8.0  0.44203862476740124  \\
            9.0  0.4335567323124081  \\
            10.0  0.42524603725559834  \\
            11.0  0.41721506295710015  \\
            12.0  0.40954498983276116  \\
            13.0  0.4022845472537381  \\
            14.000000000000002  0.39545367158532546  \\
            15.000000000000002  0.3890533378881676  \\
            16.0  0.38307496470536057  \\
            17.0  0.3775046246263573  \\
            18.000000000000004  0.3723225685335079  \\
            19.0  0.36750235643075446  \\
            20.0  0.3630130924491887  \\
            21.0  0.3588241133614915  \\
            22.0  0.3549087753166257  \\
            23.0  0.3512451901498941  \\
            24.0  0.34781454705219034  \\
            25.0  0.3445990039469754  \\
            26.0  0.34158062936266453  \\
            27.000000000000004  0.3387418251992612  \\
            28.0  0.33606683788459113  \\
            29.000000000000004  0.3335432271816649  \\
            30.0  0.3311617594043113  \\
            31.0  0.32891411999877534  \\
            32.0  0.3267901102587565  \\
            33.0  0.32477695885080404  \\
            34.0  0.32286107401732655  \\
            35.00000000000001  0.32103010837708035  \\
            35.99999999999999  0.3192737868742977  \\
            37.0  0.31758366061854804  \\
            38.0  0.315952499787298  \\
            39.0  0.3143737597408235  \\
            40.0  0.31284126279557234  \\
            41.0  0.31134911561310585  \\
            42.0  0.3098918411817597  \\
            43.0  0.30846466639325953  \\
            44.0  0.3070638429381533  \\
            45.0  0.30568681418939153  \\
            46.0  0.304332059737034  \\
            47.0  0.30299865233061085  \\
            48.00000000000001  0.30168588685252157  \\
            49.0  0.3003934482597056  \\
            49.99999999999999  0.2991221670571698  \\
            51.0  0.29787467429833153  \\
            52.0  0.29665492501499824  \\
            53.00000000000001  0.29546633552847185  \\
            54.0  0.29430986991922276  \\
            55.0  0.2931837724969536  \\
            56.00000000000001  0.29208484411627766  \\
            57.00000000000001  0.2910098050678055  \\
            57.99999999999999  0.28995596837180754  \\
            59.0  0.288921411914212  \\
            59.99999999999999  0.2879049657399  \\
            61.0  0.2869061199522267  \\
            62.0  0.2859248455358481  \\
            63.0  0.28496133151195346  \\
            64.0  0.28401569952030564  \\
            65.0  0.2830877909265195  \\
            66.0  0.28217709538926794  \\
            67.0  0.28128282283990974  \\
            68.00000000000001  0.2804040652776229  \\
            69.00000000000001  0.27953998080416825  \\
            70.0  0.2786899477263606  \\
            70.99999999999999  0.27785365591901934  \\
            71.99999999999999  0.27703111701305283  \\
            73.0  0.27622259121444004  \\
            74.0  0.27542845209746497  \\
            75.0  0.2746490341731904  \\
            76.0  0.273884515242367  \\
            77.0  0.2731348675502015  \\
            78.0  0.27239987867327364  \\
            79.0  0.2716792162343961  \\
            80.0  0.2709725031779559  \\
            81.0  0.2702793793625517  \\
            82.0  0.2695995395253314  \\
            83.0  0.2689327486741843  \\
            84.0  0.2682788408765248  \\
            85.0  0.26763770736728026  \\
            86.0  0.2670092775567047  \\
            87.0  0.2663934944790995  \\
            88.0  0.2657902859497693  \\
            89.0  0.2651995340250054  \\
            90.0  0.2646210469020738  \\
            91.0  0.26405453769091564  \\
            92.0  0.2634996130793409  \\
            93.0  0.2629557725499573  \\
            94.0  0.2624224167227208  \\
            95.00000000000001  0.26189886238568943  \\
            96.0  0.26138436183161096  \\
            97.0  0.26087812471946564  \\
            97.99999999999999  0.26037934127392337  \\
            98.99999999999999  0.2598872058975528  \\
            100.0  0.25940094010729453  \\
            101.0  0.2589198131703238  \\
            102.0  0.2584431580671549  \\
        }
        ;
\end{axis}
\begin{axis}[point meta max={nan}, point meta min={nan}, legend cell align={left}, legend columns={1}, title={}, title style={at={{(0.5,1)}}, anchor={south}, font={{\fontsize{14 pt}{18.2 pt}\selectfont}}, color={rgb,1:red,0.0;green,0.0;blue,0.0}, draw opacity={1.0}, rotate={0.0}, align={center}}, legend style={color={rgb,1:red,0.0;green,0.0;blue,0.0}, draw opacity={1.0}, line width={1}, solid, fill={rgb,1:red,1.0;green,1.0;blue,1.0}, fill opacity={1.0}, text opacity={1.0}, font={{\fontsize{8 pt}{10.4 pt}\selectfont}}, text={rgb,1:red,0.0;green,0.0;blue,0.0}, cells={anchor={center}}, at={(1.02, 1)}, anchor={north west}}, axis background/.style={fill={rgb,1:red,1.0;green,1.0;blue,1.0}, opacity={1.0}}, anchor={north west}, xshift={1.0mm}, yshift={-115.29999999999998mm}, width={150.4mm}, height={36.099999999999994mm}, scaled x ticks={false}, xlabel={$t/\tau$}, x tick style={color={rgb,1:red,0.0;green,0.0;blue,0.0}, opacity={1.0}}, x tick label style={color={rgb,1:red,0.0;green,0.0;blue,0.0}, opacity={1.0}, rotate={0}}, xlabel style={at={(ticklabel cs:0.5)}, anchor=near ticklabel, at={{(ticklabel cs:0.5)}}, anchor={near ticklabel}, font={{\fontsize{11 pt}{14.3 pt}\selectfont}}, color={rgb,1:red,0.0;green,0.0;blue,0.0}, draw opacity={1.0}, rotate={0.0}}, xmode={log}, log basis x={10}, xmajorgrids={true}, xmin={0.8707036374807742}, xmax={115.99813719881251}, xticklabels={{$10^{0}$,$10^{1}$,$10^{2}$}}, xtick={{1.0,10.0,100.0}}, xtick align={inside}, xticklabel style={font={{\fontsize{8 pt}{10.4 pt}\selectfont}}, color={rgb,1:red,0.0;green,0.0;blue,0.0}, draw opacity={1.0}, rotate={0.0}}, x grid style={color={rgb,1:red,0.0;green,0.0;blue,0.0}, draw opacity={0.1}, line width={0.5}, solid}, axis x line*={left}, x axis line style={color={rgb,1:red,0.0;green,0.0;blue,0.0}, draw opacity={1.0}, line width={1}, solid}, scaled y ticks={false}, ylabel={$\delta E^m$}, y tick style={color={rgb,1:red,0.0;green,0.0;blue,0.0}, opacity={1.0}}, y tick label style={color={rgb,1:red,0.0;green,0.0;blue,0.0}, opacity={1.0}, rotate={0}}, ylabel style={at={(ticklabel cs:0.5)}, anchor=near ticklabel, at={{(ticklabel cs:0.5)}}, anchor={near ticklabel}, font={{\fontsize{11 pt}{14.3 pt}\selectfont}}, color={rgb,1:red,0.0;green,0.0;blue,0.0}, draw opacity={1.0}, rotate={0.0}}, ymode={log}, log basis y={10}, ymajorgrids={true}, ymin={0.0003930751516650648}, ymax={0.35718080819986003}, yticklabels={{$10^{-3}$,$10^{-2}$,$10^{-1}$}}, ytick={{0.001,0.01,0.1}}, ytick align={inside}, yticklabel style={font={{\fontsize{8 pt}{10.4 pt}\selectfont}}, color={rgb,1:red,0.0;green,0.0;blue,0.0}, draw opacity={1.0}, rotate={0.0}}, y grid style={color={rgb,1:red,0.0;green,0.0;blue,0.0}, draw opacity={0.1}, line width={0.5}, solid}, axis y line*={left}, y axis line style={color={rgb,1:red,0.0;green,0.0;blue,0.0}, draw opacity={1.0}, line width={1}, solid}, colorbar={false}]
    \addplot[color={rgb,1:red,1.0;green,0.0;blue,0.0}, name path={14eefb30-274c-435a-8269-3697ff1eb603}, draw opacity={1.0}, line width={1}, solid]
        table[row sep={\\}]
        {
            \\
            1.0  0.29455029311887526  \\
            2.0  0.017126161030024956  \\
            3.0  0.011652181530942829  \\
            4.0  0.011765415516883482  \\
            5.0  0.012249832578290154  \\
            6.000000000000001  0.012603621354189576  \\
            7.0  0.01274033399452601  \\
            8.0  0.01265281777132543  \\
            9.0  0.01236704550271761  \\
            10.0  0.011925102270965549  \\
            11.000000000000002  0.01137259400384183  \\
            12.0  0.010750775770084764  \\
            13.0  0.010093595709085523  \\
            14.0  0.00942758587117376  \\
            15.0  0.008772800215157806  \\
            15.999999999999998  0.008143554791053331  \\
            17.0  0.007548598487125746  \\
            18.0  0.006991983292339188  \\
            19.0  0.006475269292796626  \\
            20.0  0.005999667378756901  \\
            21.000000000000004  0.005566768062528538  \\
            22.0  0.005178030718188342  \\
            23.0  0.0048338745190779076  \\
            24.0  0.004532821120189712  \\
            25.0  0.0042710378094918955  \\
            26.0  0.004042779538046548  \\
            27.0  0.0038417389744918595  \\
            28.000000000000004  0.00366262780711879  \\
            29.0  0.0035021104626832833  \\
            30.0  0.003358483808081847  \\
            30.999999999999996  0.0032303477176224282  \\
            32.0  0.003115538623653713  \\
            33.0  0.003011452973713913  \\
            34.0  0.0029162527827474305  \\
            35.0  0.002829252126233972  \\
            35.99999999999999  0.00274990543792053  \\
            37.0  0.002676817836213652  \\
            38.0  0.0026080182031256727  \\
            39.0  0.002541943282431125  \\
            40.0  0.0024778909037031505  \\
            41.00000000000001  0.002415482937214053  \\
            42.0  0.0023537744513652537  \\
            43.0  0.0022910581083366854  \\
            44.0  0.0022257908784876768  \\
            45.0  0.0021579826298973592  \\
            46.0  0.002089774597570593  \\
            47.0  0.002024521457836992  \\
            48.0  0.001965162556559097  \\
            49.0  0.001913239318367177  \\
            50.0  0.0018688551539665088  \\
            51.0  0.0018309473534791065  \\
            52.0  0.0017974865990559308  \\
            53.0  0.0017657567190312307  \\
            54.0  0.0017330013034004277  \\
            55.00000000000001  0.0016974511404991288  \\
            56.00000000000001  0.0016592062256270412  \\
            57.0  0.0016201570932242726  \\
            57.99999999999999  0.0015828238046160514  \\
            59.0  0.0015490975954378094  \\
            59.99999999999999  0.0015197710435332357  \\
            60.99999999999999  0.0014946840792348648  \\
            62.0  0.0014729356596897003  \\
            63.0  0.0014530222423752015  \\
            64.0  0.0014331128820272432  \\
            65.0  0.0014116375544169224  \\
            66.0  0.0013880085067361492  \\
            67.0  0.0013629197869613474  \\
            68.0  0.001337847873359832  \\
            69.0  0.001314134221052199  \\
            70.0  0.0012924206844524555  \\
            70.99999999999999  0.0012727222661225723  \\
            72.0  0.0012547998733976051  \\
            73.0  0.0012383909764915324  \\
            74.0  0.001223131052545512  \\
            75.0  0.0012083225836552192  \\
            76.0  0.0011928682326492468  \\
            77.0  0.0011755830695120428  \\
            78.0  0.0011557876640487796  \\
            79.0  0.001133755187066765  \\
            80.0  0.0011105857537366814  \\
            81.00000000000001  0.0010875831891581922  \\
            82.0  0.0010656812706910657  \\
            83.0  0.0010452833322472932  \\
            84.0  0.0010263995940036907  \\
            85.0  0.0010088351021589337  \\
            86.0  0.0009923235758352855  \\
            87.0  0.0009766033718441314  \\
            88.0  0.0009614476216290302  \\
            89.0  0.0009466610124954089  \\
            90.0  0.0009320632798011652  \\
            91.0  0.0009174835230011946  \\
            92.0  0.0009027803678389734  \\
            93.0  0.0008878832586656227  \\
            94.0  0.0008728314849297258  \\
            95.0  0.0008577825607926082  \\
            95.99999999999999  0.00084297601598321  \\
            97.0  0.0008286657668511443  \\
            98.0  0.000815055383390273  \\
            99.0  0.0008022684441885697  \\
            100.0  0.0007903623472971799  \\
            101.0  0.0007793684284164715  \\
        }
        ;
    \addplot[color={rgb,1:red,0.0;green,0.0;blue,1.0}, name path={9aa61cf8-97f2-453a-9ead-331ce053dff1}, draw opacity={1.0}, line width={1}, solid]
        table[row sep={\\}]
        {
            \\
            1.0  0.19657118004107  \\
            2.0  0.011717726244967974  \\
            3.0  0.00796869160010466  \\
            4.0  0.007975159178736035  \\
            5.0  0.008242611337404837  \\
            6.000000000000001  0.008444685597241708  \\
            7.0  0.008528010006641551  \\
            8.0  0.008481892454993145  \\
            9.0  0.008310695056809758  \\
            10.0  0.008030974298498195  \\
            11.000000000000002  0.007670073124338983  \\
            12.0  0.007260442579023085  \\
            13.0  0.006830875668412617  \\
            14.0  0.006400333697157867  \\
            15.0  0.0059783731828070286  \\
            15.999999999999998  0.005570340079003255  \\
            17.0  0.005182056092849385  \\
            18.0  0.004820212102753463  \\
            19.0  0.004489263981565739  \\
            20.0  0.00418897908769722  \\
            21.000000000000004  0.003915338044865824  \\
            22.0  0.0036635851667315977  \\
            23.0  0.003430643097703745  \\
            24.0  0.0032155431052149352  \\
            25.0  0.003018374584310868  \\
            26.0  0.002838804163403319  \\
            27.0  0.0026749873146700853  \\
            28.000000000000004  0.0025236107029262267  \\
            29.0  0.0023814677773535964  \\
            30.0  0.0022476394055359616  \\
            30.999999999999996  0.0021240097400188618  \\
            32.0  0.0020131514079524404  \\
            33.0  0.001915884833477488  \\
            34.0  0.0018309656402462071  \\
            35.0  0.001756321502782654  \\
            35.99999999999999  0.0016901262557496577  \\
            37.0  0.0016311608312500336  \\
            38.0  0.0015787400464745138  \\
            39.0  0.0015324969452511494  \\
            40.0  0.0014921471824664856  \\
            41.00000000000001  0.0014572744313461339  \\
            42.0  0.001427174788500185  \\
            43.0  0.001400823455106226  \\
            44.0  0.0013770287487617772  \\
            45.0  0.0013547544523575539  \\
            46.0  0.001333407406423126  \\
            47.0  0.001312765478089284  \\
            48.0  0.0012924385928159698  \\
            49.0  0.001271281202535779  \\
            50.0  0.0012474927588382867  \\
            51.0  0.0012197492833332935  \\
            52.0  0.0011885894865263924  \\
            53.0  0.001156465609249091  \\
            54.0  0.0011260974222691567  \\
            55.00000000000001  0.0010989283806759342  \\
            56.00000000000001  0.0010750390484721706  \\
            57.0  0.0010538366959979562  \\
            57.99999999999999  0.0010345564575955368  \\
            59.0  0.0010164461743120135  \\
            59.99999999999999  0.000998845787673286  \\
            60.99999999999999  0.0009812744163785747  \\
            62.0  0.0009635140238946627  \\
            63.0  0.0009456319916478217  \\
            64.0  0.0009279085937861642  \\
            65.0  0.000910695537251538  \\
            66.0  0.0008942725493581993  \\
            67.0  0.000878757562286836  \\
            68.0  0.0008640844734546516  \\
            69.0  0.000850033077807677  \\
            70.0  0.0008362918073412384  \\
            70.99999999999999  0.0008225389059665078  \\
            72.0  0.0008085257986127914  \\
            73.0  0.0007941391169750678  \\
            74.0  0.0007794179242745503  \\
            75.0  0.0007645189308234102  \\
            76.0  0.0007496476921655115  \\
            77.0  0.0007349888769278556  \\
            78.0  0.0007206624388775285  \\
            79.0  0.0007067130564402113  \\
            80.0  0.0006931238154042063  \\
            81.00000000000001  0.0006798398372203085  \\
            82.0  0.0006667908511470699  \\
            83.0  0.0006539077976595031  \\
            84.0  0.0006411335092445514  \\
            85.0  0.000628429810575537  \\
            86.0  0.0006157830776052364  \\
            87.0  0.0006032085293302125  \\
            88.0  0.0005907519247638771  \\
            89.0  0.0005784871229316257  \\
            90.0  0.0005665092111581393  \\
            91.0  0.0005549246115747475  \\
            92.0  0.0005438405293836102  \\
            93.0  0.0005333558272364591  \\
            94.0  0.0005235543370313844  \\
            95.0  0.0005145005540784786  \\
            95.99999999999999  0.0005062371121453135  \\
            97.0  0.0004987834455422724  \\
            98.0  0.0004921353763705505  \\
            99.0  0.000486265790258289  \\
            100.0  0.0004811269369707216  \\
            101.0  0.00047665510316891035  \\
        }
        ;
    \addplot[color={rgb,1:red,0.2422;green,0.6433;blue,0.3044}, name path={ae8585df-03cd-4d23-a5b7-703a4b1e2958}, only marks, draw opacity={1.0}, line width={0}, solid, mark={*}, mark size={1.5 pt}, mark repeat={1}, mark options={color={rgb,1:red,0.0;green,0.0;blue,0.0}, draw opacity={0.5}, fill={rgb,1:red,1.0;green,0.0;blue,0.0}, fill opacity={0.5}, line width={0.75}, rotate={0}, solid}]
        table[row sep={\\}]
        {
            \\
            1.0  0.29455029311887526  \\
            2.0  0.017126161030024956  \\
            3.0  0.011652181530942829  \\
            4.0  0.011765415516883482  \\
            5.0  0.012249832578290154  \\
            6.000000000000001  0.012603621354189576  \\
            7.0  0.01274033399452601  \\
            8.0  0.01265281777132543  \\
            9.0  0.01236704550271761  \\
            10.0  0.011925102270965549  \\
            11.000000000000002  0.01137259400384183  \\
            12.0  0.010750775770084764  \\
            13.0  0.010093595709085523  \\
            14.0  0.00942758587117376  \\
            15.0  0.008772800215157806  \\
            15.999999999999998  0.008143554791053331  \\
            17.0  0.007548598487125746  \\
            18.0  0.006991983292339188  \\
            19.0  0.006475269292796626  \\
            20.0  0.005999667378756901  \\
            21.000000000000004  0.005566768062528538  \\
            22.0  0.005178030718188342  \\
            23.0  0.0048338745190779076  \\
            24.0  0.004532821120189712  \\
            25.0  0.0042710378094918955  \\
            26.0  0.004042779538046548  \\
            27.0  0.0038417389744918595  \\
            28.000000000000004  0.00366262780711879  \\
            29.0  0.0035021104626832833  \\
            30.0  0.003358483808081847  \\
            30.999999999999996  0.0032303477176224282  \\
            32.0  0.003115538623653713  \\
            33.0  0.003011452973713913  \\
            34.0  0.0029162527827474305  \\
            35.0  0.002829252126233972  \\
            35.99999999999999  0.00274990543792053  \\
            37.0  0.002676817836213652  \\
            38.0  0.0026080182031256727  \\
            39.0  0.002541943282431125  \\
            40.0  0.0024778909037031505  \\
            41.00000000000001  0.002415482937214053  \\
            42.0  0.0023537744513652537  \\
            43.0  0.0022910581083366854  \\
            44.0  0.0022257908784876768  \\
            45.0  0.0021579826298973592  \\
            46.0  0.002089774597570593  \\
            47.0  0.002024521457836992  \\
            48.0  0.001965162556559097  \\
            49.0  0.001913239318367177  \\
            50.0  0.0018688551539665088  \\
            51.0  0.0018309473534791065  \\
            52.0  0.0017974865990559308  \\
            53.0  0.0017657567190312307  \\
            54.0  0.0017330013034004277  \\
            55.00000000000001  0.0016974511404991288  \\
            56.00000000000001  0.0016592062256270412  \\
            57.0  0.0016201570932242726  \\
            57.99999999999999  0.0015828238046160514  \\
            59.0  0.0015490975954378094  \\
            59.99999999999999  0.0015197710435332357  \\
            60.99999999999999  0.0014946840792348648  \\
            62.0  0.0014729356596897003  \\
            63.0  0.0014530222423752015  \\
            64.0  0.0014331128820272432  \\
            65.0  0.0014116375544169224  \\
            66.0  0.0013880085067361492  \\
            67.0  0.0013629197869613474  \\
            68.0  0.001337847873359832  \\
            69.0  0.001314134221052199  \\
            70.0  0.0012924206844524555  \\
            70.99999999999999  0.0012727222661225723  \\
            72.0  0.0012547998733976051  \\
            73.0  0.0012383909764915324  \\
            74.0  0.001223131052545512  \\
            75.0  0.0012083225836552192  \\
            76.0  0.0011928682326492468  \\
            77.0  0.0011755830695120428  \\
            78.0  0.0011557876640487796  \\
            79.0  0.001133755187066765  \\
            80.0  0.0011105857537366814  \\
            81.00000000000001  0.0010875831891581922  \\
            82.0  0.0010656812706910657  \\
            83.0  0.0010452833322472932  \\
            84.0  0.0010263995940036907  \\
            85.0  0.0010088351021589337  \\
            86.0  0.0009923235758352855  \\
            87.0  0.0009766033718441314  \\
            88.0  0.0009614476216290302  \\
            89.0  0.0009466610124954089  \\
            90.0  0.0009320632798011652  \\
            91.0  0.0009174835230011946  \\
            92.0  0.0009027803678389734  \\
            93.0  0.0008878832586656227  \\
            94.0  0.0008728314849297258  \\
            95.0  0.0008577825607926082  \\
            95.99999999999999  0.00084297601598321  \\
            97.0  0.0008286657668511443  \\
            98.0  0.000815055383390273  \\
            99.0  0.0008022684441885697  \\
            100.0  0.0007903623472971799  \\
            101.0  0.0007793684284164715  \\
        }
        ;
    \addplot[color={rgb,1:red,0.7644;green,0.4441;blue,0.8243}, name path={3346dcbb-686c-497a-920c-27eba63b00be}, only marks, draw opacity={1.0}, line width={0}, solid, mark={*}, mark size={1.5 pt}, mark repeat={1}, mark options={color={rgb,1:red,0.0;green,0.0;blue,0.0}, draw opacity={0.5}, fill={rgb,1:red,0.0;green,0.0;blue,1.0}, fill opacity={0.5}, line width={0.75}, rotate={0}, solid}]
        table[row sep={\\}]
        {
            \\
            1.0  0.19657118004107  \\
            2.0  0.011717726244967974  \\
            3.0  0.00796869160010466  \\
            4.0  0.007975159178736035  \\
            5.0  0.008242611337404837  \\
            6.000000000000001  0.008444685597241708  \\
            7.0  0.008528010006641551  \\
            8.0  0.008481892454993145  \\
            9.0  0.008310695056809758  \\
            10.0  0.008030974298498195  \\
            11.000000000000002  0.007670073124338983  \\
            12.0  0.007260442579023085  \\
            13.0  0.006830875668412617  \\
            14.0  0.006400333697157867  \\
            15.0  0.0059783731828070286  \\
            15.999999999999998  0.005570340079003255  \\
            17.0  0.005182056092849385  \\
            18.0  0.004820212102753463  \\
            19.0  0.004489263981565739  \\
            20.0  0.00418897908769722  \\
            21.000000000000004  0.003915338044865824  \\
            22.0  0.0036635851667315977  \\
            23.0  0.003430643097703745  \\
            24.0  0.0032155431052149352  \\
            25.0  0.003018374584310868  \\
            26.0  0.002838804163403319  \\
            27.0  0.0026749873146700853  \\
            28.000000000000004  0.0025236107029262267  \\
            29.0  0.0023814677773535964  \\
            30.0  0.0022476394055359616  \\
            30.999999999999996  0.0021240097400188618  \\
            32.0  0.0020131514079524404  \\
            33.0  0.001915884833477488  \\
            34.0  0.0018309656402462071  \\
            35.0  0.001756321502782654  \\
            35.99999999999999  0.0016901262557496577  \\
            37.0  0.0016311608312500336  \\
            38.0  0.0015787400464745138  \\
            39.0  0.0015324969452511494  \\
            40.0  0.0014921471824664856  \\
            41.00000000000001  0.0014572744313461339  \\
            42.0  0.001427174788500185  \\
            43.0  0.001400823455106226  \\
            44.0  0.0013770287487617772  \\
            45.0  0.0013547544523575539  \\
            46.0  0.001333407406423126  \\
            47.0  0.001312765478089284  \\
            48.0  0.0012924385928159698  \\
            49.0  0.001271281202535779  \\
            50.0  0.0012474927588382867  \\
            51.0  0.0012197492833332935  \\
            52.0  0.0011885894865263924  \\
            53.0  0.001156465609249091  \\
            54.0  0.0011260974222691567  \\
            55.00000000000001  0.0010989283806759342  \\
            56.00000000000001  0.0010750390484721706  \\
            57.0  0.0010538366959979562  \\
            57.99999999999999  0.0010345564575955368  \\
            59.0  0.0010164461743120135  \\
            59.99999999999999  0.000998845787673286  \\
            60.99999999999999  0.0009812744163785747  \\
            62.0  0.0009635140238946627  \\
            63.0  0.0009456319916478217  \\
            64.0  0.0009279085937861642  \\
            65.0  0.000910695537251538  \\
            66.0  0.0008942725493581993  \\
            67.0  0.000878757562286836  \\
            68.0  0.0008640844734546516  \\
            69.0  0.000850033077807677  \\
            70.0  0.0008362918073412384  \\
            70.99999999999999  0.0008225389059665078  \\
            72.0  0.0008085257986127914  \\
            73.0  0.0007941391169750678  \\
            74.0  0.0007794179242745503  \\
            75.0  0.0007645189308234102  \\
            76.0  0.0007496476921655115  \\
            77.0  0.0007349888769278556  \\
            78.0  0.0007206624388775285  \\
            79.0  0.0007067130564402113  \\
            80.0  0.0006931238154042063  \\
            81.00000000000001  0.0006798398372203085  \\
            82.0  0.0006667908511470699  \\
            83.0  0.0006539077976595031  \\
            84.0  0.0006411335092445514  \\
            85.0  0.000628429810575537  \\
            86.0  0.0006157830776052364  \\
            87.0  0.0006032085293302125  \\
            88.0  0.0005907519247638771  \\
            89.0  0.0005784871229316257  \\
            90.0  0.0005665092111581393  \\
            91.0  0.0005549246115747475  \\
            92.0  0.0005438405293836102  \\
            93.0  0.0005333558272364591  \\
            94.0  0.0005235543370313844  \\
            95.0  0.0005145005540784786  \\
            95.99999999999999  0.0005062371121453135  \\
            97.0  0.0004987834455422724  \\
            98.0  0.0004921353763705505  \\
            99.0  0.000486265790258289  \\
            100.0  0.0004811269369707216  \\
            101.0  0.00047665510316891035  \\
        }
        ;
\end{axis}
\begin{axis}[point meta max={nan}, point meta min={nan}, legend cell align={left}, legend columns={1}, title={}, title style={at={{(0.5,1)}}, anchor={south}, font={{\fontsize{14 pt}{18.2 pt}\selectfont}}, color={rgb,1:red,0.0;green,0.0;blue,0.0}, draw opacity={1.0}, rotate={0.0}, align={center}}, legend style={color={rgb,1:red,0.0;green,0.0;blue,0.0}, draw opacity={1.0}, line width={1}, solid, fill={rgb,1:red,1.0;green,1.0;blue,1.0}, fill opacity={1.0}, text opacity={1.0}, font={{\fontsize{8 pt}{10.4 pt}\selectfont}}, text={rgb,1:red,0.0;green,0.0;blue,0.0}, cells={anchor={center}}, at={(1.02, 1)}, anchor={north west}}, axis background/.style={fill={rgb,1:red,1.0;green,1.0;blue,1.0}, opacity={1.0}}, anchor={north west}, xshift={1.0mm}, yshift={-153.39999999999998mm}, width={150.4mm}, height={36.099999999999994mm}, scaled x ticks={false}, xlabel={$t/\tau$}, x tick style={color={rgb,1:red,0.0;green,0.0;blue,0.0}, opacity={1.0}}, x tick label style={color={rgb,1:red,0.0;green,0.0;blue,0.0}, opacity={1.0}, rotate={0}}, xlabel style={at={(ticklabel cs:0.5)}, anchor=near ticklabel, at={{(ticklabel cs:0.5)}}, anchor={near ticklabel}, font={{\fontsize{11 pt}{14.3 pt}\selectfont}}, color={rgb,1:red,0.0;green,0.0;blue,0.0}, draw opacity={1.0}, rotate={0.0}}, xmode={log}, log basis x={10}, xmajorgrids={true}, xmin={0.8704463225992706}, xmax={117.18126362509537}, xticklabels={{$10^{0}$,$10^{1}$,$10^{2}$}}, xtick={{1.0,10.0,100.0}}, xtick align={inside}, xticklabel style={font={{\fontsize{8 pt}{10.4 pt}\selectfont}}, color={rgb,1:red,0.0;green,0.0;blue,0.0}, draw opacity={1.0}, rotate={0.0}}, x grid style={color={rgb,1:red,0.0;green,0.0;blue,0.0}, draw opacity={0.1}, line width={0.5}, solid}, axis x line*={left}, x axis line style={color={rgb,1:red,0.0;green,0.0;blue,0.0}, draw opacity={1.0}, line width={1}, solid}, scaled y ticks={false}, ylabel={$E_1^m$}, y tick style={color={rgb,1:red,0.0;green,0.0;blue,0.0}, opacity={1.0}}, y tick label style={color={rgb,1:red,0.0;green,0.0;blue,0.0}, opacity={1.0}, rotate={0}}, ylabel style={at={(ticklabel cs:0.5)}, anchor=near ticklabel, at={{(ticklabel cs:0.5)}}, anchor={near ticklabel}, font={{\fontsize{11 pt}{14.3 pt}\selectfont}}, color={rgb,1:red,0.0;green,0.0;blue,0.0}, draw opacity={1.0}, rotate={0.0}}, ymajorgrids={true}, ymin={0.0234467711013705}, ymax={0.49612432607875223}, yticklabels={{$0.1$,$0.2$,$0.3$,$0.4$}}, ytick={{0.1,0.2,0.30000000000000004,0.4}}, ytick align={inside}, yticklabel style={font={{\fontsize{8 pt}{10.4 pt}\selectfont}}, color={rgb,1:red,0.0;green,0.0;blue,0.0}, draw opacity={1.0}, rotate={0.0}}, y grid style={color={rgb,1:red,0.0;green,0.0;blue,0.0}, draw opacity={0.1}, line width={0.5}, solid}, axis y line*={left}, y axis line style={color={rgb,1:red,0.0;green,0.0;blue,0.0}, draw opacity={1.0}, line width={1}, solid}, colorbar={false}]
    \addplot[color={rgb,1:red,1.0;green,0.0;blue,0.0}, name path={af69248f-7f3d-455f-83c2-08f5e531f094}, draw opacity={1.0}, line width={1}, solid]
        table[row sep={\\}]
        {
            \\
            1.0  0.48274665942844897  \\
            2.0  0.06579901340241394  \\
            3.0  0.05108123060571343  \\
            4.0  0.052026903358996644  \\
            5.0  0.05618759582107911  \\
            6.0  0.06158197749732928  \\
            7.0  0.06761782290889347  \\
            8.0  0.0739972328456438  \\
            9.0  0.0805049042987615  \\
            10.0  0.08696160038861434  \\
            11.0  0.09321850914538878  \\
            12.0  0.09916008934253578  \\
            13.0  0.10470501563084636  \\
            14.000000000000002  0.10980332600816096  \\
            15.000000000000002  0.11443117797351098  \\
            16.0  0.11858494921539528  \\
            17.0  0.12227583254319409  \\
            18.000000000000004  0.12552533210260738  \\
            19.0  0.1283614716680329  \\
            20.0  0.13081562038262662  \\
            21.0  0.1329202417445825  \\
            22.0  0.1347077559198547  \\
            23.0  0.1362102020475155  \\
            24.0  0.13745917792066614  \\
            25.0  0.13848567957875302  \\
            26.0  0.1393196217871389  \\
            27.000000000000004  0.13998901954177073  \\
            28.0  0.14051909432114015  \\
            29.000000000000004  0.1409316944889563  \\
            30.0  0.14124529359033078  \\
            31.0  0.1414755486998313  \\
            32.0  0.14163606839696108  \\
            33.0  0.14173890161310332  \\
            34.0  0.14179455417480968  \\
            35.00000000000001  0.14181184326938304  \\
            35.99999999999999  0.1417980140987669  \\
            37.0  0.1417590961655505  \\
            38.0  0.14170011647724104  \\
            39.0  0.14162504215725555  \\
            40.0  0.1415367503400405  \\
            41.0  0.1414373046883674  \\
            42.0  0.1413284209870913  \\
            43.0  0.1412116956741562  \\
            44.0  0.1410882979837244  \\
            45.0  0.1409583118739898  \\
            46.0  0.14082031869337863  \\
            47.0  0.1406717284504398  \\
            48.00000000000001  0.14050978981706852  \\
            49.0  0.1403326593130597  \\
            49.99999999999999  0.14013996970281975  \\
            51.0  0.13993284135623002  \\
            52.0  0.1397135754721356  \\
            53.00000000000001  0.13948518855944436  \\
            54.0  0.13925080549075936  \\
            55.0  0.13901294255353588  \\
            56.00000000000001  0.13877287057387708  \\
            57.00000000000001  0.13853038996742362  \\
            57.99999999999999  0.13828423878427895  \\
            59.0  0.1380329488925579  \\
            59.99999999999999  0.13777565162470365  \\
            61.0  0.13751247915959858  \\
            62.0  0.137244574133072  \\
            63.0  0.13697388102804606  \\
            64.0  0.13670282803798742  \\
            65.0  0.1364339239319363  \\
            66.0  0.13616930577977043  \\
            67.0  0.13591035469052012  \\
            68.00000000000001  0.13565753161946914  \\
            69.00000000000001  0.13541048460811925  \\
            70.0  0.1351683205833177  \\
            70.99999999999999  0.13492988526482436  \\
            71.99999999999999  0.13469397667482946  \\
            73.0  0.13445951370538098  \\
            74.0  0.13422570130636644  \\
            75.0  0.1339921742728886  \\
            76.0  0.13375903094192165  \\
            77.0  0.13352667544458666  \\
            78.0  0.13329549526511394  \\
            79.0  0.13306553659513662  \\
            80.0  0.13283638423006064  \\
            81.0  0.13260732883046197  \\
            82.0  0.1323776911723012  \\
            83.0  0.13214708076709467  \\
            84.0  0.13191547264870182  \\
            85.0  0.1316831358894812  \\
            86.0  0.13145050365006367  \\
            87.0  0.13121805393365793  \\
            88.0  0.1309862320991637  \\
            89.0  0.13075541678664643  \\
            90.0  0.13052591397597793  \\
            91.0  0.13029795863403631  \\
            92.0  0.13007170874594964  \\
            93.0  0.12984722891007336  \\
            94.0  0.12962447305772626  \\
            95.00000000000001  0.12940328091425715  \\
            96.0  0.12918339744738894  \\
            97.0  0.12896451201990536  \\
            97.99999999999999  0.1287463024533477  \\
            98.99999999999999  0.12852846682991265  \\
            100.0  0.1283107343916692  \\
            101.0  0.12809285980930177  \\
            102.0  0.1278746131379242  \\
        }
        ;
    \addplot[color={rgb,1:red,0.0;green,0.0;blue,1.0}, name path={2c8b253c-9c32-41d7-a0f1-55d82c408060}, draw opacity={1.0}, line width={1}, solid]
        table[row sep={\\}]
        {
            \\
            1.0  0.32548160087919675  \\
            2.0  0.04662902748201897  \\
            3.0  0.03682443775167376  \\
            4.0  0.03747651357723076  \\
            5.0  0.040239827761236406  \\
            6.0  0.04379729498174154  \\
            7.0  0.047751322663152004  \\
            8.0  0.05190520099393914  \\
            9.0  0.05612578886649848  \\
            10.0  0.06030884699955043  \\
            11.0  0.06436893332053131  \\
            12.0  0.06823728446655365  \\
            13.0  0.07186287869503367  \\
            14.000000000000002  0.07521342475841547  \\
            15.000000000000002  0.0782737281069778  \\
            16.0  0.08104123185376713  \\
            17.0  0.08352096284586089  \\
            18.000000000000004  0.08572244989982793  \\
            19.0  0.08765930650608779  \\
            20.0  0.0893498584238125  \\
            21.0  0.09081667324723124  \\
            22.0  0.09208446188110853  \\
            23.0  0.09317749884388725  \\
            24.0  0.09411785198981504  \\
            25.0  0.0949248336839835  \\
            26.0  0.09561531305690034  \\
            27.000000000000004  0.0962042689698791  \\
            28.0  0.09670503784446671  \\
            29.000000000000004  0.09712898505829656  \\
            30.0  0.09748483903056816  \\
            31.0  0.09777845742424301  \\
            32.0  0.09801368673374378  \\
            33.0  0.09819394667127324  \\
            34.0  0.09832334693613225  \\
            35.00000000000001  0.09840671619355545  \\
            35.99999999999999  0.09844900833912328  \\
            37.0  0.09845477270027961  \\
            38.0  0.09842795659503169  \\
            39.0  0.09837198707717394  \\
            40.0  0.0982899986950434  \\
            41.0  0.09818509225965773  \\
            42.0  0.0980605368772598  \\
            43.0  0.09791984830749377  \\
            44.0  0.09776670391908891  \\
            45.0  0.09760470690141825  \\
            46.0  0.09743709394361719  \\
            47.0  0.09726654909286983  \\
            48.00000000000001  0.09709525203819734  \\
            49.0  0.09692510759469321  \\
            49.99999999999999  0.09675789889216614  \\
            51.0  0.09659511020352328  \\
            52.0  0.09643746525626593  \\
            53.00000000000001  0.09628459964061425  \\
            54.0  0.09613527798303095  \\
            55.0  0.0959880327593573  \\
            56.00000000000001  0.09584169835208309  \\
            57.00000000000001  0.09569557910838566  \\
            57.99999999999999  0.09554939201232311  \\
            59.0  0.09540315369796347  \\
            59.99999999999999  0.09525705881618436  \\
            61.0  0.09511135401506526  \\
            62.0  0.0949662255842407  \\
            63.0  0.09482172984826308  \\
            64.0  0.09467778445435614  \\
            65.0  0.09453421209921486  \\
            66.0  0.09439080634127624  \\
            67.0  0.0942473882013728  \\
            68.00000000000001  0.0941038382667144  \\
            69.00000000000001  0.09396010483540813  \\
            70.0  0.09381619434867045  \\
            70.99999999999999  0.09367214980765982  \\
            71.99999999999999  0.0935280233169364  \\
            73.0  0.09338385154928033  \\
            74.0  0.09323964371377903  \\
            75.0  0.09309538707802686  \\
            76.0  0.09295106661919517  \\
            77.0  0.09280668849756514  \\
            78.0  0.09266229643582613  \\
            79.0  0.0925179751927618  \\
            80.0  0.0923738416310802  \\
            81.0  0.09223002768127793  \\
            82.0  0.09208666019144059  \\
            83.0  0.09194384148865796  \\
            84.0  0.09180163283628587  \\
            85.0  0.09166004172979082  \\
            86.0  0.09151901349604777  \\
            87.0  0.09137842780421138  \\
            88.0  0.09123810092831437  \\
            89.0  0.09109779435928402  \\
            90.0  0.09095722946570019  \\
            91.0  0.09081610668524001  \\
            92.0  0.09067412678660497  \\
            93.0  0.09053101149189131  \\
            94.0  0.09038652118584246  \\
            95.00000000000001  0.09024046823078348  \\
            96.0  0.09009272518014629  \\
            97.0  0.08994322772413249  \\
            97.99999999999999  0.08979197247731247  \\
            98.99999999999999  0.08963900981190187  \\
            100.0  0.08948443198087123  \\
            101.0  0.0893283569031559  \\
            102.0  0.08917090832241885  \\
        }
        ;
    \addplot[color={rgb,1:red,0.2422;green,0.6433;blue,0.3044}, name path={02c56399-7937-4f1c-a990-79d9733c44f5}, only marks, draw opacity={1.0}, line width={0}, solid, mark={*}, mark size={1.5 pt}, mark repeat={1}, mark options={color={rgb,1:red,0.0;green,0.0;blue,0.0}, draw opacity={0.5}, fill={rgb,1:red,1.0;green,0.0;blue,0.0}, fill opacity={0.5}, line width={0.75}, rotate={0}, solid}]
        table[row sep={\\}]
        {
            \\
            1.0  0.48274665942844897  \\
            2.0  0.06579901340241394  \\
            3.0  0.05108123060571343  \\
            4.0  0.052026903358996644  \\
            5.0  0.05618759582107911  \\
            6.0  0.06158197749732928  \\
            7.0  0.06761782290889347  \\
            8.0  0.0739972328456438  \\
            9.0  0.0805049042987615  \\
            10.0  0.08696160038861434  \\
            11.0  0.09321850914538878  \\
            12.0  0.09916008934253578  \\
            13.0  0.10470501563084636  \\
            14.000000000000002  0.10980332600816096  \\
            15.000000000000002  0.11443117797351098  \\
            16.0  0.11858494921539528  \\
            17.0  0.12227583254319409  \\
            18.000000000000004  0.12552533210260738  \\
            19.0  0.1283614716680329  \\
            20.0  0.13081562038262662  \\
            21.0  0.1329202417445825  \\
            22.0  0.1347077559198547  \\
            23.0  0.1362102020475155  \\
            24.0  0.13745917792066614  \\
            25.0  0.13848567957875302  \\
            26.0  0.1393196217871389  \\
            27.000000000000004  0.13998901954177073  \\
            28.0  0.14051909432114015  \\
            29.000000000000004  0.1409316944889563  \\
            30.0  0.14124529359033078  \\
            31.0  0.1414755486998313  \\
            32.0  0.14163606839696108  \\
            33.0  0.14173890161310332  \\
            34.0  0.14179455417480968  \\
            35.00000000000001  0.14181184326938304  \\
            35.99999999999999  0.1417980140987669  \\
            37.0  0.1417590961655505  \\
            38.0  0.14170011647724104  \\
            39.0  0.14162504215725555  \\
            40.0  0.1415367503400405  \\
            41.0  0.1414373046883674  \\
            42.0  0.1413284209870913  \\
            43.0  0.1412116956741562  \\
            44.0  0.1410882979837244  \\
            45.0  0.1409583118739898  \\
            46.0  0.14082031869337863  \\
            47.0  0.1406717284504398  \\
            48.00000000000001  0.14050978981706852  \\
            49.0  0.1403326593130597  \\
            49.99999999999999  0.14013996970281975  \\
            51.0  0.13993284135623002  \\
            52.0  0.1397135754721356  \\
            53.00000000000001  0.13948518855944436  \\
            54.0  0.13925080549075936  \\
            55.0  0.13901294255353588  \\
            56.00000000000001  0.13877287057387708  \\
            57.00000000000001  0.13853038996742362  \\
            57.99999999999999  0.13828423878427895  \\
            59.0  0.1380329488925579  \\
            59.99999999999999  0.13777565162470365  \\
            61.0  0.13751247915959858  \\
            62.0  0.137244574133072  \\
            63.0  0.13697388102804606  \\
            64.0  0.13670282803798742  \\
            65.0  0.1364339239319363  \\
            66.0  0.13616930577977043  \\
            67.0  0.13591035469052012  \\
            68.00000000000001  0.13565753161946914  \\
            69.00000000000001  0.13541048460811925  \\
            70.0  0.1351683205833177  \\
            70.99999999999999  0.13492988526482436  \\
            71.99999999999999  0.13469397667482946  \\
            73.0  0.13445951370538098  \\
            74.0  0.13422570130636644  \\
            75.0  0.1339921742728886  \\
            76.0  0.13375903094192165  \\
            77.0  0.13352667544458666  \\
            78.0  0.13329549526511394  \\
            79.0  0.13306553659513662  \\
            80.0  0.13283638423006064  \\
            81.0  0.13260732883046197  \\
            82.0  0.1323776911723012  \\
            83.0  0.13214708076709467  \\
            84.0  0.13191547264870182  \\
            85.0  0.1316831358894812  \\
            86.0  0.13145050365006367  \\
            87.0  0.13121805393365793  \\
            88.0  0.1309862320991637  \\
            89.0  0.13075541678664643  \\
            90.0  0.13052591397597793  \\
            91.0  0.13029795863403631  \\
            92.0  0.13007170874594964  \\
            93.0  0.12984722891007336  \\
            94.0  0.12962447305772626  \\
            95.00000000000001  0.12940328091425715  \\
            96.0  0.12918339744738894  \\
            97.0  0.12896451201990536  \\
            97.99999999999999  0.1287463024533477  \\
            98.99999999999999  0.12852846682991265  \\
            100.0  0.1283107343916692  \\
            101.0  0.12809285980930177  \\
            102.0  0.1278746131379242  \\
        }
        ;
    \addplot[color={rgb,1:red,0.7644;green,0.4441;blue,0.8243}, name path={88e5e1c5-a1ba-4c83-8d05-08305b608d96}, only marks, draw opacity={1.0}, line width={0}, solid, mark={*}, mark size={1.5 pt}, mark repeat={1}, mark options={color={rgb,1:red,0.0;green,0.0;blue,0.0}, draw opacity={0.5}, fill={rgb,1:red,0.0;green,0.0;blue,1.0}, fill opacity={0.5}, line width={0.75}, rotate={0}, solid}]
        table[row sep={\\}]
        {
            \\
            1.0  0.32548160087919675  \\
            2.0  0.04662902748201897  \\
            3.0  0.03682443775167376  \\
            4.0  0.03747651357723076  \\
            5.0  0.040239827761236406  \\
            6.0  0.04379729498174154  \\
            7.0  0.047751322663152004  \\
            8.0  0.05190520099393914  \\
            9.0  0.05612578886649848  \\
            10.0  0.06030884699955043  \\
            11.0  0.06436893332053131  \\
            12.0  0.06823728446655365  \\
            13.0  0.07186287869503367  \\
            14.000000000000002  0.07521342475841547  \\
            15.000000000000002  0.0782737281069778  \\
            16.0  0.08104123185376713  \\
            17.0  0.08352096284586089  \\
            18.000000000000004  0.08572244989982793  \\
            19.0  0.08765930650608779  \\
            20.0  0.0893498584238125  \\
            21.0  0.09081667324723124  \\
            22.0  0.09208446188110853  \\
            23.0  0.09317749884388725  \\
            24.0  0.09411785198981504  \\
            25.0  0.0949248336839835  \\
            26.0  0.09561531305690034  \\
            27.000000000000004  0.0962042689698791  \\
            28.0  0.09670503784446671  \\
            29.000000000000004  0.09712898505829656  \\
            30.0  0.09748483903056816  \\
            31.0  0.09777845742424301  \\
            32.0  0.09801368673374378  \\
            33.0  0.09819394667127324  \\
            34.0  0.09832334693613225  \\
            35.00000000000001  0.09840671619355545  \\
            35.99999999999999  0.09844900833912328  \\
            37.0  0.09845477270027961  \\
            38.0  0.09842795659503169  \\
            39.0  0.09837198707717394  \\
            40.0  0.0982899986950434  \\
            41.0  0.09818509225965773  \\
            42.0  0.0980605368772598  \\
            43.0  0.09791984830749377  \\
            44.0  0.09776670391908891  \\
            45.0  0.09760470690141825  \\
            46.0  0.09743709394361719  \\
            47.0  0.09726654909286983  \\
            48.00000000000001  0.09709525203819734  \\
            49.0  0.09692510759469321  \\
            49.99999999999999  0.09675789889216614  \\
            51.0  0.09659511020352328  \\
            52.0  0.09643746525626593  \\
            53.00000000000001  0.09628459964061425  \\
            54.0  0.09613527798303095  \\
            55.0  0.0959880327593573  \\
            56.00000000000001  0.09584169835208309  \\
            57.00000000000001  0.09569557910838566  \\
            57.99999999999999  0.09554939201232311  \\
            59.0  0.09540315369796347  \\
            59.99999999999999  0.09525705881618436  \\
            61.0  0.09511135401506526  \\
            62.0  0.0949662255842407  \\
            63.0  0.09482172984826308  \\
            64.0  0.09467778445435614  \\
            65.0  0.09453421209921486  \\
            66.0  0.09439080634127624  \\
            67.0  0.0942473882013728  \\
            68.00000000000001  0.0941038382667144  \\
            69.00000000000001  0.09396010483540813  \\
            70.0  0.09381619434867045  \\
            70.99999999999999  0.09367214980765982  \\
            71.99999999999999  0.0935280233169364  \\
            73.0  0.09338385154928033  \\
            74.0  0.09323964371377903  \\
            75.0  0.09309538707802686  \\
            76.0  0.09295106661919517  \\
            77.0  0.09280668849756514  \\
            78.0  0.09266229643582613  \\
            79.0  0.0925179751927618  \\
            80.0  0.0923738416310802  \\
            81.0  0.09223002768127793  \\
            82.0  0.09208666019144059  \\
            83.0  0.09194384148865796  \\
            84.0  0.09180163283628587  \\
            85.0  0.09166004172979082  \\
            86.0  0.09151901349604777  \\
            87.0  0.09137842780421138  \\
            88.0  0.09123810092831437  \\
            89.0  0.09109779435928402  \\
            90.0  0.09095722946570019  \\
            91.0  0.09081610668524001  \\
            92.0  0.09067412678660497  \\
            93.0  0.09053101149189131  \\
            94.0  0.09038652118584246  \\
            95.00000000000001  0.09024046823078348  \\
            96.0  0.09009272518014629  \\
            97.0  0.08994322772413249  \\
            97.99999999999999  0.08979197247731247  \\
            98.99999999999999  0.08963900981190187  \\
            100.0  0.08948443198087123  \\
            101.0  0.0893283569031559  \\
            102.0  0.08917090832241885  \\
        }
        ;
\end{axis}
\begin{axis}[point meta max={nan}, point meta min={nan}, legend cell align={left}, legend columns={1}, title={}, title style={at={{(0.5,1)}}, anchor={south}, font={{\fontsize{14 pt}{18.2 pt}\selectfont}}, color={rgb,1:red,0.0;green,0.0;blue,0.0}, draw opacity={1.0}, rotate={0.0}, align={center}}, legend style={color={rgb,1:red,0.0;green,0.0;blue,0.0}, draw opacity={1.0}, line width={1}, solid, fill={rgb,1:red,1.0;green,1.0;blue,1.0}, fill opacity={1.0}, text opacity={1.0}, font={{\fontsize{8 pt}{10.4 pt}\selectfont}}, text={rgb,1:red,0.0;green,0.0;blue,0.0}, cells={anchor={center}}, at={(1.02, 1)}, anchor={north west}}, axis background/.style={fill={rgb,1:red,1.0;green,1.0;blue,1.0}, opacity={1.0}}, anchor={north west}, xshift={1.0mm}, yshift={-191.49999999999997mm}, width={145.4mm}, height={36.099999999999994mm}, scaled x ticks={false}, xlabel={$t/\\tau$}, x tick style={color={rgb,1:red,0.0;green,0.0;blue,0.0}, opacity={1.0}}, x tick label style={color={rgb,1:red,0.0;green,0.0;blue,0.0}, opacity={1.0}, rotate={0}}, xlabel style={at={(ticklabel cs:0.5)}, anchor=near ticklabel, at={{(ticklabel cs:0.5)}}, anchor={near ticklabel}, font={{\fontsize{11 pt}{14.3 pt}\selectfont}}, color={rgb,1:red,0.0;green,0.0;blue,0.0}, draw opacity={1.0}, rotate={0.0}}, xmode={log}, log basis x={10}, xmajorgrids={true}, xmin={0.8704463225992706}, xmax={117.18126362509537}, xticklabels={{$10^{0}$,$10^{1}$,$10^{2}$}}, xtick={{1.0,10.0,100.0}}, xtick align={inside}, xticklabel style={font={{\fontsize{8 pt}{10.4 pt}\selectfont}}, color={rgb,1:red,0.0;green,0.0;blue,0.0}, draw opacity={1.0}, rotate={0.0}}, x grid style={color={rgb,1:red,0.0;green,0.0;blue,0.0}, draw opacity={0.1}, line width={0.5}, solid}, axis x line*={left}, x axis line style={color={rgb,1:red,0.0;green,0.0;blue,0.0}, draw opacity={1.0}, line width={1}, solid}, scaled y ticks={false}, ylabel={$E_2^m$}, y tick style={color={rgb,1:red,0.0;green,0.0;blue,0.0}, opacity={1.0}}, y tick label style={color={rgb,1:red,0.0;green,0.0;blue,0.0}, opacity={1.0}, rotate={0}}, ylabel style={at={(ticklabel cs:0.5)}, anchor=near ticklabel, at={{(ticklabel cs:0.5)}}, anchor={near ticklabel}, font={{\fontsize{11 pt}{14.3 pt}\selectfont}}, color={rgb,1:red,0.0;green,0.0;blue,0.0}, draw opacity={1.0}, rotate={0.0}}, ymajorgrids={true}, ymin={0.15482020978031436}, ymax={0.6654589551898722}, yticklabels={{$0.2$,$0.3$,$0.4$,$0.5$,$0.6$}}, ytick={{0.2,0.30000000000000004,0.4,0.5,0.6000000000000001}}, ytick align={inside}, yticklabel style={font={{\fontsize{8 pt}{10.4 pt}\selectfont}}, color={rgb,1:red,0.0;green,0.0;blue,0.0}, draw opacity={1.0}, rotate={0.0}}, y grid style={color={rgb,1:red,0.0;green,0.0;blue,0.0}, draw opacity={0.1}, line width={0.5}, solid}, axis y line*={left}, y axis line style={color={rgb,1:red,0.0;green,0.0;blue,0.0}, draw opacity={1.0}, line width={1}, solid}, colorbar={false}]
    \addplot[color={rgb,1:red,1.0;green,0.0;blue,0.0}, name path={9bd4c918-d54f-4b7a-98ee-49e0fca0963b}, draw opacity={1.0}, line width={1}, solid]
        table[row sep={\\}]
        {
            \\
            1.0  0.5286095623182909  \\
            2.0  0.6510069152254507  \\
            3.0  0.6485985369921259  \\
            4.0  0.6360006827079028  \\
            5.0  0.6200745747289331  \\
            6.0  0.6024303604743951  \\
            7.0  0.5837908937086412  \\
            8.0  0.5646711497773633  \\
            9.0  0.545510660552921  \\
            10.0  0.5266869189603494  \\
            11.0  0.5085049079326106  \\
            12.0  0.49119073373162164  \\
            13.0  0.4748950316732266  \\
            14.000000000000002  0.4597031255868268  \\
            15.000000000000002  0.445647687750303  \\
            16.0  0.4327211162932604  \\
            17.0  0.42088667817440784  \\
            18.000000000000004  0.41008858012786864  \\
            19.0  0.4002604572701045  \\
            20.0  0.3913310392627143  \\
            21.0  0.38322675052200195  \\
            22.0  0.37587246828420084  \\
            23.0  0.3691919914383496  \\
            24.0  0.363109141046124  \\
            25.0  0.35754981826784654  \\
            26.0  0.3524448382499679  \\
            27.000000000000004  0.3477326609572915  \\
            28.0  0.3433608472034292  \\
            29.000000000000004  0.3392856192284936  \\
            30.0  0.33546990966443563  \\
            31.0  0.33188117074685314  \\
            32.0  0.3284903033321019  \\
            33.0  0.3252719314923056  \\
            34.0  0.3222048259568849  \\
            35.00000000000001  0.3192712840795638  \\
            35.99999999999999  0.31645586112394647  \\
            37.0  0.3137448736192439  \\
            38.0  0.3111270354713385  \\
            39.0  0.3085940915881992  \\
            40.0  0.3061404401229822  \\
            41.0  0.303761994870951  \\
            42.0  0.30145539563501406  \\
            43.0  0.2992183464965857  \\
            44.0  0.2970506860786777  \\
            45.0  0.2949548813099255  \\
            46.0  0.2929348918606399  \\
            47.0  0.2909937075060083  \\
            48.00000000000001  0.28913112468154223  \\
            49.0  0.28734309262899155  \\
            49.99999999999999  0.28562254292086514  \\
            51.0  0.28396081611348867  \\
            52.0  0.2823491346441033  \\
            53.00000000000001  0.2807800349577391  \\
            54.0  0.27924866130739207  \\
            55.0  0.2777535229412159  \\
            56.00000000000001  0.27629614378037504  \\
            57.00000000000001  0.27487941816120093  \\
            57.99999999999999  0.27350541225112146  \\
            59.0  0.2721738783382276  \\
            59.99999999999999  0.27088207801064346  \\
            61.0  0.26962547943221477  \\
            62.0  0.268398700379506  \\
            63.0  0.267196457824843  \\
            64.0  0.2660144885725261  \\
            65.0  0.2648502797965506  \\
            66.0  0.26370326039429937  \\
            67.0  0.2625742029768133  \\
            68.00000000000001  0.26146410626090333  \\
            69.00000000000001  0.260373305398893  \\
            70.0  0.2593013352026439  \\
            70.99999999999999  0.25824734983668335  \\
            71.99999999999999  0.25721053616055595  \\
            73.0  0.2561901992566068  \\
            74.0  0.25518562067912975  \\
            75.0  0.25419601666006186  \\
            76.0  0.253220837407374  \\
            77.0  0.2522603246720591  \\
            78.0  0.2513159217820209  \\
            79.0  0.250390092787949  \\
            80.0  0.24948548996595812  \\
            81.0  0.24860395961182052  \\
            82.0  0.24774601408082209  \\
            83.0  0.246910943215338  \\
            84.0  0.24609726800148266  \\
            85.0  0.24530320516670057  \\
            86.0  0.24452700230395913  \\
            87.0  0.24376712844452925  \\
            88.0  0.24302234690717978  \\
            89.0  0.24229171459806761  \\
            90.0  0.2415745563962406  \\
            91.0  0.24087044845838157  \\
            92.0  0.24017921482346788  \\
            93.0  0.2395009142915038  \\
            94.0  0.2388357868851851  \\
            95.00000000000001  0.2381841475437253  \\
            96.0  0.2375462484498007  \\
            97.0  0.2369221578613014  \\
            97.99999999999999  0.23631170166100723  \\
            98.99999999999999  0.2357144819010518  \\
            100.0  0.2351299458951068  \\
            101.0  0.23455745813017745  \\
            102.0  0.23399633637313785  \\
        }
        ;
    \addlegendentry {Circle}
    \addplot[color={rgb,1:red,0.0;green,0.0;blue,1.0}, name path={00a440a3-9341-4b02-a5c7-d2ba65659b2a}, draw opacity={1.0}, line width={1}, solid]
        table[row sep={\\}]
        {
            \\
            1.0  0.36600508789437036  \\
            2.0  0.4482864812504783  \\
            3.0  0.4463733447358563  \\
            4.0  0.43775257731019424  \\
            5.0  0.4270141039474528  \\
            6.0  0.41521402538954394  \\
            7.0  0.40281531211089006  \\
            8.0  0.39013342377346216  \\
            9.0  0.37743094344590905  \\
            10.0  0.3649371902560481  \\
            11.0  0.35284612963657014  \\
            12.0  0.341307705366209  \\
            13.0  0.3304216685587052  \\
            14.000000000000002  0.32024024682690955  \\
            15.000000000000002  0.3107796097811898  \\
            16.0  0.30203373285159324  \\
            17.0  0.29398366178049673  \\
            18.000000000000004  0.28660011863368  \\
            19.0  0.2798430499246671  \\
            20.0  0.27366323402537573  \\
            21.0  0.26800744011426075  \\
            22.0  0.26282431343551793  \\
            23.0  0.2580676913060073  \\
            24.0  0.2536966950623751  \\
            25.0  0.249674170262992  \\
            26.0  0.2459653163057641  \\
            27.000000000000004  0.24253755622938214  \\
            28.0  0.23936180004012464  \\
            29.000000000000004  0.23641424212336842  \\
            30.0  0.23367692037374369  \\
            31.0  0.2311356625745318  \\
            32.0  0.22877642352501276  \\
            33.0  0.22658301217953092  \\
            34.0  0.2245377270811943  \\
            35.00000000000001  0.22262339218352467  \\
            35.99999999999999  0.2208247785351742  \\
            37.0  0.21912888791826848  \\
            38.0  0.2175245431922659  \\
            39.0  0.21600177266364867  \\
            40.0  0.21455126410052958  \\
            41.0  0.21316402335344853  \\
            42.0  0.21183130430449965  \\
            43.0  0.21054481808576553  \\
            44.0  0.20929713901906458  \\
            45.0  0.20808210728797338  \\
            46.0  0.20689496579341682  \\
            47.0  0.20573210323774083  \\
            48.00000000000001  0.20459063481432432  \\
            49.0  0.20346834066501227  \\
            49.99999999999999  0.20236426816500316  \\
            51.0  0.20127956409480838  \\
            52.0  0.20021745975873198  \\
            53.00000000000001  0.19918173588785798  \\
            54.0  0.19817459193619177  \\
            55.0  0.19719573973759583  \\
            56.00000000000001  0.1962431457641944  \\
            57.00000000000001  0.19531422595941947  \\
            57.99999999999999  0.19440657635948416  \\
            59.0  0.19351825821624816  \\
            59.99999999999999  0.19264790692371586  \\
            61.0  0.19179476593716138  \\
            62.0  0.19095861995160737  \\
            63.0  0.19013960166369023  \\
            64.0  0.18933791506594982  \\
            65.0  0.18855357882730425  \\
            66.0  0.1877862890479914  \\
            67.0  0.18703543463853703  \\
            68.00000000000001  0.18630022701090854  \\
            69.00000000000001  0.18557987596876052  \\
            70.0  0.1848737533776898  \\
            70.99999999999999  0.1841815061113596  \\
            71.99999999999999  0.1835030936961167  \\
            73.0  0.18283873966516007  \\
            74.0  0.18218880838368598  \\
            75.0  0.18155364709516322  \\
            76.0  0.18093344862317198  \\
            77.0  0.1803281790526362  \\
            78.0  0.17973758223744765  \\
            79.0  0.17916124104163467  \\
            80.0  0.1785986615468758  \\
            81.0  0.1780493516812738  \\
            82.0  0.17751287933389037  \\
            83.0  0.17698890718552626  \\
            84.0  0.1764772080402392  \\
            85.0  0.1759776656374892  \\
            86.0  0.17549026406065704  \\
            87.0  0.1750150666748884  \\
            88.0  0.17455218502145492  \\
            89.0  0.17410173966572165  \\
            90.0  0.17366381743637338  \\
            91.0  0.17323843100567546  \\
            92.0  0.17282548629273672  \\
            93.0  0.17242476105806653  \\
            94.0  0.17203589553687867  \\
            95.00000000000001  0.17165839415490605  \\
            96.0  0.17129163665146452  \\
            97.0  0.17093489699533285  \\
            97.99999999999999  0.17058736879661088  \\
            98.99999999999999  0.1702481960856512  \\
            100.0  0.16991650812642314  \\
            101.0  0.1695914562671679  \\
            102.0  0.16927224974473581  \\
        }
        ;
    \addlegendentry {Flower}
    \addplot[color={rgb,1:red,0.2422;green,0.6433;blue,0.3044}, name path={0009c21e-01a4-4113-b179-be22b77ef293}, only marks, draw opacity={1.0}, line width={0}, solid, mark={*}, mark size={1.5 pt}, mark repeat={1}, mark options={color={rgb,1:red,0.0;green,0.0;blue,0.0}, draw opacity={0.5}, fill={rgb,1:red,1.0;green,0.0;blue,0.0}, fill opacity={0.5}, line width={0.75}, rotate={0}, solid}, forget plot]
        table[row sep={\\}]
        {
            \\
            1.0  0.5286095623182909  \\
            2.0  0.6510069152254507  \\
            3.0  0.6485985369921259  \\
            4.0  0.6360006827079028  \\
            5.0  0.6200745747289331  \\
            6.0  0.6024303604743951  \\
            7.0  0.5837908937086412  \\
            8.0  0.5646711497773633  \\
            9.0  0.545510660552921  \\
            10.0  0.5266869189603494  \\
            11.0  0.5085049079326106  \\
            12.0  0.49119073373162164  \\
            13.0  0.4748950316732266  \\
            14.000000000000002  0.4597031255868268  \\
            15.000000000000002  0.445647687750303  \\
            16.0  0.4327211162932604  \\
            17.0  0.42088667817440784  \\
            18.000000000000004  0.41008858012786864  \\
            19.0  0.4002604572701045  \\
            20.0  0.3913310392627143  \\
            21.0  0.38322675052200195  \\
            22.0  0.37587246828420084  \\
            23.0  0.3691919914383496  \\
            24.0  0.363109141046124  \\
            25.0  0.35754981826784654  \\
            26.0  0.3524448382499679  \\
            27.000000000000004  0.3477326609572915  \\
            28.0  0.3433608472034292  \\
            29.000000000000004  0.3392856192284936  \\
            30.0  0.33546990966443563  \\
            31.0  0.33188117074685314  \\
            32.0  0.3284903033321019  \\
            33.0  0.3252719314923056  \\
            34.0  0.3222048259568849  \\
            35.00000000000001  0.3192712840795638  \\
            35.99999999999999  0.31645586112394647  \\
            37.0  0.3137448736192439  \\
            38.0  0.3111270354713385  \\
            39.0  0.3085940915881992  \\
            40.0  0.3061404401229822  \\
            41.0  0.303761994870951  \\
            42.0  0.30145539563501406  \\
            43.0  0.2992183464965857  \\
            44.0  0.2970506860786777  \\
            45.0  0.2949548813099255  \\
            46.0  0.2929348918606399  \\
            47.0  0.2909937075060083  \\
            48.00000000000001  0.28913112468154223  \\
            49.0  0.28734309262899155  \\
            49.99999999999999  0.28562254292086514  \\
            51.0  0.28396081611348867  \\
            52.0  0.2823491346441033  \\
            53.00000000000001  0.2807800349577391  \\
            54.0  0.27924866130739207  \\
            55.0  0.2777535229412159  \\
            56.00000000000001  0.27629614378037504  \\
            57.00000000000001  0.27487941816120093  \\
            57.99999999999999  0.27350541225112146  \\
            59.0  0.2721738783382276  \\
            59.99999999999999  0.27088207801064346  \\
            61.0  0.26962547943221477  \\
            62.0  0.268398700379506  \\
            63.0  0.267196457824843  \\
            64.0  0.2660144885725261  \\
            65.0  0.2648502797965506  \\
            66.0  0.26370326039429937  \\
            67.0  0.2625742029768133  \\
            68.00000000000001  0.26146410626090333  \\
            69.00000000000001  0.260373305398893  \\
            70.0  0.2593013352026439  \\
            70.99999999999999  0.25824734983668335  \\
            71.99999999999999  0.25721053616055595  \\
            73.0  0.2561901992566068  \\
            74.0  0.25518562067912975  \\
            75.0  0.25419601666006186  \\
            76.0  0.253220837407374  \\
            77.0  0.2522603246720591  \\
            78.0  0.2513159217820209  \\
            79.0  0.250390092787949  \\
            80.0  0.24948548996595812  \\
            81.0  0.24860395961182052  \\
            82.0  0.24774601408082209  \\
            83.0  0.246910943215338  \\
            84.0  0.24609726800148266  \\
            85.0  0.24530320516670057  \\
            86.0  0.24452700230395913  \\
            87.0  0.24376712844452925  \\
            88.0  0.24302234690717978  \\
            89.0  0.24229171459806761  \\
            90.0  0.2415745563962406  \\
            91.0  0.24087044845838157  \\
            92.0  0.24017921482346788  \\
            93.0  0.2395009142915038  \\
            94.0  0.2388357868851851  \\
            95.00000000000001  0.2381841475437253  \\
            96.0  0.2375462484498007  \\
            97.0  0.2369221578613014  \\
            97.99999999999999  0.23631170166100723  \\
            98.99999999999999  0.2357144819010518  \\
            100.0  0.2351299458951068  \\
            101.0  0.23455745813017745  \\
            102.0  0.23399633637313785  \\
        }
        ;
    \addplot[color={rgb,1:red,0.7644;green,0.4441;blue,0.8243}, name path={0994cc57-8069-4235-bb09-89ea6c2eade1}, only marks, draw opacity={1.0}, line width={0}, solid, mark={*}, mark size={1.5 pt}, mark repeat={1}, mark options={color={rgb,1:red,0.0;green,0.0;blue,0.0}, draw opacity={0.5}, fill={rgb,1:red,0.0;green,0.0;blue,1.0}, fill opacity={0.5}, line width={0.75}, rotate={0}, solid}, forget plot]
        table[row sep={\\}]
        {
            \\
            1.0  0.36600508789437036  \\
            2.0  0.4482864812504783  \\
            3.0  0.4463733447358563  \\
            4.0  0.43775257731019424  \\
            5.0  0.4270141039474528  \\
            6.0  0.41521402538954394  \\
            7.0  0.40281531211089006  \\
            8.0  0.39013342377346216  \\
            9.0  0.37743094344590905  \\
            10.0  0.3649371902560481  \\
            11.0  0.35284612963657014  \\
            12.0  0.341307705366209  \\
            13.0  0.3304216685587052  \\
            14.000000000000002  0.32024024682690955  \\
            15.000000000000002  0.3107796097811898  \\
            16.0  0.30203373285159324  \\
            17.0  0.29398366178049673  \\
            18.000000000000004  0.28660011863368  \\
            19.0  0.2798430499246671  \\
            20.0  0.27366323402537573  \\
            21.0  0.26800744011426075  \\
            22.0  0.26282431343551793  \\
            23.0  0.2580676913060073  \\
            24.0  0.2536966950623751  \\
            25.0  0.249674170262992  \\
            26.0  0.2459653163057641  \\
            27.000000000000004  0.24253755622938214  \\
            28.0  0.23936180004012464  \\
            29.000000000000004  0.23641424212336842  \\
            30.0  0.23367692037374369  \\
            31.0  0.2311356625745318  \\
            32.0  0.22877642352501276  \\
            33.0  0.22658301217953092  \\
            34.0  0.2245377270811943  \\
            35.00000000000001  0.22262339218352467  \\
            35.99999999999999  0.2208247785351742  \\
            37.0  0.21912888791826848  \\
            38.0  0.2175245431922659  \\
            39.0  0.21600177266364867  \\
            40.0  0.21455126410052958  \\
            41.0  0.21316402335344853  \\
            42.0  0.21183130430449965  \\
            43.0  0.21054481808576553  \\
            44.0  0.20929713901906458  \\
            45.0  0.20808210728797338  \\
            46.0  0.20689496579341682  \\
            47.0  0.20573210323774083  \\
            48.00000000000001  0.20459063481432432  \\
            49.0  0.20346834066501227  \\
            49.99999999999999  0.20236426816500316  \\
            51.0  0.20127956409480838  \\
            52.0  0.20021745975873198  \\
            53.00000000000001  0.19918173588785798  \\
            54.0  0.19817459193619177  \\
            55.0  0.19719573973759583  \\
            56.00000000000001  0.1962431457641944  \\
            57.00000000000001  0.19531422595941947  \\
            57.99999999999999  0.19440657635948416  \\
            59.0  0.19351825821624816  \\
            59.99999999999999  0.19264790692371586  \\
            61.0  0.19179476593716138  \\
            62.0  0.19095861995160737  \\
            63.0  0.19013960166369023  \\
            64.0  0.18933791506594982  \\
            65.0  0.18855357882730425  \\
            66.0  0.1877862890479914  \\
            67.0  0.18703543463853703  \\
            68.00000000000001  0.18630022701090854  \\
            69.00000000000001  0.18557987596876052  \\
            70.0  0.1848737533776898  \\
            70.99999999999999  0.1841815061113596  \\
            71.99999999999999  0.1835030936961167  \\
            73.0  0.18283873966516007  \\
            74.0  0.18218880838368598  \\
            75.0  0.18155364709516322  \\
            76.0  0.18093344862317198  \\
            77.0  0.1803281790526362  \\
            78.0  0.17973758223744765  \\
            79.0  0.17916124104163467  \\
            80.0  0.1785986615468758  \\
            81.0  0.1780493516812738  \\
            82.0  0.17751287933389037  \\
            83.0  0.17698890718552626  \\
            84.0  0.1764772080402392  \\
            85.0  0.1759776656374892  \\
            86.0  0.17549026406065704  \\
            87.0  0.1750150666748884  \\
            88.0  0.17455218502145492  \\
            89.0  0.17410173966572165  \\
            90.0  0.17366381743637338  \\
            91.0  0.17323843100567546  \\
            92.0  0.17282548629273672  \\
            93.0  0.17242476105806653  \\
            94.0  0.17203589553687867  \\
            95.00000000000001  0.17165839415490605  \\
            96.0  0.17129163665146452  \\
            97.0  0.17093489699533285  \\
            97.99999999999999  0.17058736879661088  \\
            98.99999999999999  0.1702481960856512  \\
            100.0  0.16991650812642314  \\
            101.0  0.1695914562671679  \\
            102.0  0.16927224974473581  \\
        }
        ;
\end{axis}
\end{tikzpicture}

\caption{Evolution of the error measure, $e_{L^{1}(\Omega)}$ and the energy functional $E(u_h)$ over time for two different domains: a circular domain and a flower-shaped domain.  }
\label{fig:physical_CH_plot}
\end{figure}


\subsection{Note on the manufactured solution}%
\label{sub:the_problem}

While the report is not consisting of a full worthy experiments of convergence analysis, we still present a framework for manufactured solutions.

Recall the strong form Cahn-Hilliard equation. Let $ u( x,0) =  u_{0}$ then is the dynamics on the form,
\begin{subequations}
    \label{eq:ch_gen}
    \begin{align}
    \label{eq:ch_gen:a}
        \partial _{t} u + \Delta  \left(  \varepsilon  \Delta u - \frac{1}{\varepsilon }f( u) \right)   &= g_{0}( x)   \quad \text{ in } \Omega  \\
        \partial _{n} u &= g_{1}( x)  \quad \text{ on } \Gamma  \\
        \partial _{n}    \Delta u   &= g_{2}(x)  \quad \text{ on } \Gamma
    \end{align}
\end{subequations}
where we defined $f( u) = F'( u) =u( u^2 -1)  $ for $F( u) = \frac{1}{4}( u^{2} - 1)^{2} $ and the domain $\Omega \subset \mathbb{R} ^{d} $  for $d = 2,3$. In contrast to the standard version presented in the introduction \eqref{eq:strongch}, is this version
generalized to also holds for for functions $g_{0},g_{1},g_{2}: \Omega \to\mathbb{R}   $. While the standard version may be physical correct, this version creates flexibility so we can easily construct manufactured solution on complex domains.

    Designing a manufactured solution using $g_{0}( \cdot ) $ may be temping with the formulation \eqref{eq:ch_gen:a}. However, observe that expanding the Laplacian we get,
    \begin{equation}
    \begin{split}
        \Delta  \left(  \varepsilon  \Delta u - \frac{1}{\varepsilon }f( u) \right) & = \varepsilon \Delta^2 u - \frac{1}{\varepsilon } \Delta f( u) \\
                                                                                    &= \varepsilon \Delta ^2 u  - \frac{3}{\varepsilon }( 2u \| \nabla u \|_{ 2 }^{ 2 } + u^{2}  \Delta u )   \\
    \end{split}
    \end{equation}
Here we applied the chain rule twice and inserted the derivatives.
\begin{equation}
    \label{eq:nonlinear_laplace}
    \begin{split}
\Delta f( u)  &= \nabla \cdot \nabla f( u)  = \nabla \cdot  \left[ f' ( u) \partial _{x_{1}}u, \ldots, f' ( u) \partial _{x_{d}}u \right] ^{T} \\
& =  f'' ( u)( ( \partial _{x_{1}}u )^{2} + \ldots +( \partial _{x_{d}}u )^{2} ) +  f' ( u)( \partial _{x_{1} x_{1}}u + \ldots +   \partial _{x_{d} x_{d}}u ) \\
&=  f'' ( u) \| \nabla u \|_{ 2 }^{ 2 } + f' ( u)  \Delta u  = 6u \| \nabla u \|_{ 2 }^{ 2 } + 3u^{2}  \Delta u
    \end{split}
\end{equation}

Our goal is to write the Cahn Hilliard equation on weak form.
Assume that $\Omega  \subset \mathbb{R} ^{d}$ with a $\Gamma $ in $C^2$.
 Let $u \in  H^{4}( \Omega ) $ and $v \in V_{h} $.
Now, expanding the first Laplace operator from a weak point of view is it clear that
\[
 ( \Delta ( \varepsilon  \Delta u - \frac{1}{\varepsilon } f( u) ) ,v )_{\Omega } = \varepsilon ( \Delta^{2} u ,v )_{\Omega } - \frac{1}{\varepsilon } ( \Delta f( u)  ,v )_{\Omega }.
\]
Hence, this makes it natural to associate the biharmonic $( \Delta ^2 u,v)_{\Omega } $ with bilinear forms $A_{h}( \cdot ,\cdot ) $ , however, in this section will we only consider the Laplace
variant $a^{L}( \cdot ,\cdot ) $.

We will now find seek to find a bilinear form of the nonlinear term,
\begin{lemma}[Semi-linear form]
    Let $u \in H^4( \Omega ) $ be solution to \eqref{eq:ch_gen} and $v_{h} \in V_{h}$ the test function.
Then can we rewrite the nonlinear term into the corresponding semi-linear form $c_{h}( \cdot ,\cdot )  $ for the nonlinear term $( -\Delta f( u) , v_{h})_{\Omega }$ into two consistent formulations.
\begin{align}
    \label{eq:ch:1}
      c^{1}_{h} ( u,v_{h})  & = ( f' ( u) \nabla u, v_{h} )_{\Omega }  - ( f'( u)  g_{1}   ,  v_{h})_{\Gamma } \\
    \label{eq:ch:2}
        c^{2}_{h} ( u,v_{h})  & = -( f( u), \Delta v_{h} )_{\Omega }+  ( f( u) , \jump{ \partial _{n}v_{h} }  )_{\mathcal{F} _{h}^{int}} + ( f(u), \partial _{n} v_{h})_{\Gamma  }  - ( f'( u)  g_{1}   ,  v_{h})_{\Gamma }
\end{align}
% \begin{remark}
%     Be aware that the both formulations are consistent and if we replace $u \in H^{4}( \Omega ) $ with  $u_{h} \in  V_{h}$ we have two different discrete formulations.
% \end{remark}

\end{lemma}
% \todo[inline]{ I criticize \cite[ Remark 4.1d]{feng2007fully} which says that says that finding this weak form is not possible for conforming methods (I guess $C^{0}$ is a conforming method??). }

\begin{proof}

         \textbf{Derivation of \eqref{eq:ch:1}.  }  We want to construct the first formulation. Let $T$ be an element in $\mathcal{T}_{h}$. From Greens theorem is it easy to see that
            \begin{equation}
            \label{eq:1_gr}
-(\Delta f( u) , v_{h})_{T } = (\nabla f( u), \nabla v_{h}  )_{T } - ( \partial _{n}  f( u), v_{h} )_{\partial T }
            \end{equation}
            First by utilizing that $\nabla f( u) = f' ( u) \nabla u $ and $\partial _{n}f( u)  = f' ( u)  \partial _{n}u$  and doing a summation over the triangles  is it clear that \[
            ( -\Delta f( u),v_{h} )_{\Omega  } =(f' ( u) \nabla u, \nabla v_{h}  )_{\Omega  } - (   f' ( u)\partial _{n}u, v_{h} )_{\partial \mathcal{T}_{h}  }
            \]
            Iterating over the facets is it clear that \[
                \begin{split}
            (   f' ( u)\partial _{n}u, v_{h} )_{\partial \mathcal{T}_{h}  } & = \sum_{F \in \mathcal{F}_{h}  }^{} \int_{F}^{}   \jump{ f' ( u)\partial _{n}u, v_{h} } \\
                                                                        & =  ( \jump{ f' ( u) \partial _{n}u },  \mean{v_{h}}    )_{\mathcal{F}^{int}_{h} } + ( \mean{ f' ( u) \partial _{n}u }, \jump{ v_{h} }    )_{\mathcal{F}^{int}_{h} } +  ( f' ( u)
                                                                        \partial _{n}u, v_{h}) _{\Gamma } \\
                                                                        & =  ( f' ( u) \partial _{n}u, v_{h}) _{\Gamma }
                \end{split}
            \]
            The jump terms vanishes by the regularity of $u$ and $v_{h}$. Hence, by inserting $g_{1}$ we have shown that the first formulation holds.

         \textbf{Derivation of \eqref{eq:ch:2}.  }  Applying a extra iteration of Greens theorem on \eqref{eq:1_gr} we get the following terms.
\[
    \begin{split}
-(\Delta f( u) , v_{h})_{T }  = -( f( u), \Delta v_{h} )_{T} + (f( u), \partial _{n} v_{h}  )_{\partial T} - (   f'( u)\partial _{n}u, v_{h} )_{\partial T } .
    \end{split}
\]
Now, by doing a summation of all triangles it is clear that this holds.
\begin{equation}
\label{eq:f_g2}
-(\Delta f( u) , v_{h})_{\Omega  }  = -( f( u), \Delta v_{h} )_{\Omega } + (f( u), \partial _{n} v_{h}  )_{\partial \mathcal{T}_{h} } - (   f'( u)\partial _{n}u, v_{h} )_{\partial \mathcal{T}_{h}  }
\end{equation}
It comes evident from the first step of the proof that $ (   f'( u)\partial _{n}u, v_{h} )_{\partial \mathcal{T}_{h}  } = (   f'( u)\partial _{n}u, v_{h} )_{\Gamma }$, hence, we only need to compute the term $(f( u), \partial _{n} v_{h}  )_{\partial
\mathcal{T}_{h} }$ on the facets. \[
    \begin{split}
(f( u), \partial _{n} v_{h}  )_{\partial
\mathcal{T}_{h} } & = \sum_{F\in \mathcal{F} _{h}}^{} \int_{F}^{}\jump{ f( u), \partial _{n} v_{h}  } \\
& =  (\jump{ f( u)  }  , \mean{ \partial _{n} v_{h} }    )_{ \mathcal{F}_{h}^{int} } +(\mean{ f( u)  }  , \jump{ \partial _{n} v_{h} }    )_{ \mathcal{F}_{h}^{int} } + (f( u), \partial _{n} v_{h}  )_{\Gamma } \\
&=  (\mean{ f( u)  }  , \jump{ \partial _{n} v_{h} }    )_{ \mathcal{F}_{h}^{int} } + (f( u), \partial _{n} v_{h}  )_{\Gamma }
    \end{split}
\]
Again one of the jump terms vanishes because of the regularity of $u$.
Inserting the result into \eqref{eq:f_g2} have we shown that the second formulation also holds.

The consistency follows in a weak sense for $u \in H^{4}( \Omega ) $.
\end{proof}

Hence, we have a scheme for nonlinear spatial discretization. Let $u_{h} \in \left[ 0,T \right] \times V_{h}  $ and $v_{h} \in V_{h}$. We define the following CIP discretization
\begin{equation}
    \label{eq:cip_ch}
    \begin{split}
        a_{h} \left( u_{h}, v_{h} \right)   =& ( \Delta  u_{h}, \Delta v_{h} ) _{ \Omega } \\
                                     & + \left( \mean{  \Delta  u_{h} }, \jump{ \partial _{n }v_{h}} \right)_{\mathcal{F}_{h}  }  + \left( \mean{ \Delta  v_{h} }, \jump{ \partial _{n}u_{h} }      \right)_{\mathcal{F}_{h}  }  + \frac{\gamma }{h}
                                     \left( \jump{ \partial _{n} u_{h}}, \jump{ \partial _{n} v_{h}   }   \right)_{\mathcal{F}_{h} } \\
                                     & + \left(   \Delta  u_{h} ,  \partial _{n }v_{h} \right)_{\Gamma   }  + \left(  \Delta  v_{h} ,  \partial _{n}u_{h}       \right)_{\Gamma  }  + \frac{\gamma }{h}  \left(  \partial _{n} u_{h},  \partial _{n} v_{h}      \right)_{ \Gamma } \\
    c^{}_{h} ( u_{h},v_{h})  & = -( f( u_{h}), \Delta v_{h} )_{\mathcal{T} _{h}}+  ( f( u_{h}) , \jump{ \partial _{n}v_{h} }  )_{\mathcal{F} _{h}^{int}} + ( f(u_{h}), \partial _{n} v_{h})_{\Gamma  }  + ( f'( u_{h})  g_{1}   ,  v_{h})_{\Gamma } \\
    l_{h}( v_{h}) & =  \left( g_{0}, v_{h} \right) _{\Omega } -  \varepsilon ( g_{2},  v_{h} )_{\Gamma }  -  \varepsilon ( g_{1}, \Delta  v_{h}  )_{\Gamma }  + \varepsilon \frac{\gamma }{h} ( g_{1}, \partial _{n} v_{h}  )_{\Gamma }
    \end{split}
\end{equation}
such that that the following relationship holds.
\[
    ( \partial _{t}u_{h}, v_{h})_\Omega + \varepsilon  a_{h}( u_{h},v_{h}) + \frac{1}{\varepsilon }c_{h}( u_{h},v_{h})   =  l_{h}(v_{h}) \quad  \forall u_{h}, v_{h} \in V_{h}
\]


