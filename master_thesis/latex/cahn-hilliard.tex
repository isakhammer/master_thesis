
\newpage
\section{Cahn Hilliard equation }%
\label{sec:cahn_hilliard_equation}


\begin{itemize}
    \item Explain the full discretization scheme including Follow \cite{feng2007fully}
    \item CutFEMCIP for spatial discretization
    \item implicit Euler for temporal discretization
    \item linearization of nonlinearity (IMEX)
\end{itemize}


\subsection{The linear Cahn-Hilliard}%

We will consider linear Cahn Hilliard problem. Let $ u( x,0) =  u_{0}$ then is the formulations

\begin{equation}
\label{eq:ch_exact}
    \begin{split}
        \partial _{t} u  + \varepsilon \Delta^2   u -   & =  g_{0}(x)  \quad \text{in} \Omega \\
        \partial _{n} u & =  g_{1}(x)  \quad \text{ on } \Gamma \\
         \partial _{n} \Delta u & = g_{2}(x)  \quad \text{ on } \Gamma  \\
    \end{split}
\end{equation}



We can see that the simplest weak formulation for $u,v \in H^{4}( \Omega ) $ is on the form,
\[
( \partial _{t}u, v )_{\Omega }  - \varepsilon ( \Delta^2 u, v)_{\Omega } = ( h,v)_{\Omega }   \\
\]
We can easily observe that the biharmonic equation is a term of this equation. Combining the theory from \ref{sec:cutcip_biharmonic_problem} and can we see that the Laplace formulation



\subsection{The general Cahn-Hilliard problem}%
\label{sub:the_problem}

Recall the strong form Cahn-Hilliard equation. Let $ u( x,0) =  u_{0}$ then is the dynamics on the form,

\begin{equation}
\label{eq:ch_gen}
    \begin{split}
\partial _{t} u + \Delta  \left(  \varepsilon  \Delta u - \frac{1}{\varepsilon }f( u) \right)   &= g_{0}( x)   \quad \text{ in } \Omega  \\
\partial _{n} u &= g_{1}( x)  \quad \text{ on } \Gamma  \\
\partial _{n}    \Delta u   &= g_{2}(x)  \quad \text{ on } \Gamma  \\
    \end{split}
\end{equation}
where we used the notation $f( u) = F'( u) =u( u^2 -1)  $ for $F( u) = \frac{1}{4}( u^{2} - 1)^{2} $. $\Omega \subset \mathbb{R} ^{d} $  for $d = 2,3$. In contrast to the standard version presented in the introduction \eqref{eq:strongch}, is this version
generalized to also holds for for functions $g_{0},g_{1},g_{2}: \Omega \to\mathbb{R}   $. While the standard version may be physical correct, this version creates flexibility so we can easily construct manufactured solution on complex domains.

\begin{remark}
    Computing manufactured solution using $g_{0}$ may be temping with the formulation presented in formulation \eqref{eq:ch_gen}. However, observed that expanding the Laplacian we get,
\[
    \begin{split}
        \Delta  \left(  \varepsilon  \Delta u - \frac{1}{\varepsilon }f( u) \right) & = \varepsilon \Delta^2 u - \frac{1}{\varepsilon } \Delta f( u) \\
                                                                                    &= \varepsilon \Delta ^2 u  - \frac{3}{\varepsilon }( 2u \| \nabla u \|_{ 2 }^{ 2 } + u^{2}  \Delta u )   \\
    \end{split}
\]
Here we applied the chain rule twice and inserted the derivatives.
\[
    \begin{split}
\Delta f( u)  &= \nabla \cdot \nabla f( u)  = \nabla \cdot  \left[ f' ( u) \partial _{x_{1}}u, \ldots, f' ( u) \partial _{x_{d}}u \right] ^{T} \\
& =  f'' ( u)( ( \partial _{x_{1}}u )^{2} + \ldots +( \partial _{x_{d}}u )^{2} ) +  f' ( u)( \partial _{x_{1} x_{1}}u + \ldots +   \partial _{x_{d} x_{d}}u ) \\
&=  f'' ( u) \| \nabla u \|_{ 2 }^{ 2 } + f' ( u)  \Delta u  = 6u \| \nabla u \|_{ 2 }^{ 2 } + 3u^{2}  \Delta u
    \end{split}
\]
\end{remark}


\subsection{ CIP method for the Cahn Hilliard Equation on a Polygonal mesh}%
\label{sub:writing_the_cahn_hilliard_equation_of_weak_form}

Our goal is to write the Cahn Hilliard equation on weak form.
Assume that $\Omega  \subset \mathbb{R} ^{d}$ is a polygon.
 Let $u \in  H^{4}( \Omega ) $ and $v \in V_{h} $ where
\[
V_{h} = \left\{ v \in C^{0}( \Omega )   \mid v\in \mathcal{P} ^{k} ( T) \quad \forall T \in \mathcal{T} _{h} \right\}.
\]
Now, expanding the first Laplace operator from a weak point of view is it clear that
\[
 ( \Delta ( \varepsilon  \Delta u - \frac{1}{\varepsilon } f( u) ) ,v )_{\Omega } = \varepsilon ( \Delta^{2} u ,v )_{\Omega } - \frac{1}{\varepsilon } ( \Delta f( u)  ,v )_{\Omega }.
\]
Hence, this makes it natural to associate the biharmonic $( \Delta ^2 u,v)_{\Gamma } $ with bilinear forms $a_h^{H} $ in  and $a^{H}_{h} $ from Section \ref{sec:CIP_biharmonic_problem}, however, in this section will we only consider the Laplace
variant $a^{L}$.
We will now find seek to find a bilinear form of the nonlinear term,

\begin{lemma}[Semi-linear form]
    Let $u \in H^4( \Omega ) $ be solution to \eqref{eq:ch_gen} and $v \in V_{h}$ the test function.
Then can we rewrite the nonlinear term into the corresponding semi-linear form $c_{h}:V_{h} \times  V_{h} \to \mathbb{R} $  for the nonlinear term $( -\Delta f( u) , v)_{\Omega }$ into two equivalent formulations.
\begin{enumerate}[label=\arabic*)]
    \item  $c^{}_{h} ( u,v)  = ( f' ( u) \nabla u, v )  - ( f'( u)  g_{1}   ,  v)_{\Gamma }$
    \item
        $c^{}_{h} ( u,v)  = -( f( u), \Delta v )_{\Omega }+  ( f( u) , \jump{ \partial _{n}v }  )_{\mathcal{F} _{h}^{int}} + ( f(u), \partial _{n} v)_{\Gamma  }  - ( f'( u)  g_{1}   ,  v)_{\Gamma }$
\end{enumerate}

\end{lemma}
\todo[inline]{ I criticize \cite[ Remark 4.1d]{feng2007fully} which says that says that finding this weak form is not possible for conforming methods (I guess $C^{0}$ is a conforming method??). }

\begin{proof}

    We will first show the first formulation, and then the second. The equivalence follows in a weak sense for $u \in H^{4}( \Omega ) $  when this is shown.
    \begin{enumerate}[label=\arabic*)]
        \item We want to construct the first formulation. Let $T$ be an element in $\mathcal{T}_{h}$. From Greens theorem is it easy to see that
            \begin{equation}
            \label{eq:1_gr}
-(\Delta f( u) , v)_{T } = (\nabla f( u), \nabla v  )_{T } - ( \partial _{n}  f( u), v )_{\partial T }
            \end{equation}
            First by utilizing that $\nabla f( u) = f' ( u) \nabla u $ and $\partial _{n}f( u)  = f' ( u)  \partial _{n}u$  and doing a summation over the triangles  is it clear that \[
            ( -\Delta f( u),v )_{\Omega  } =(f' ( u) \nabla u, \nabla v  )_{\Omega  } - (   f' ( u)\partial _{n}u, v )_{\partial \mathcal{T}_{h}  }
            \]
            Iterating over the facets is it clear that \[
                \begin{split}
            (   f' ( u)\partial _{n}u, v )_{\partial \mathcal{T}_{h}  } & = \sum_{F \in \mathcal{F}_{h}  }^{} \int_{F}^{}   \jump{ f' ( u)\partial _{n}u, v } \\
                                                                        & =  ( \jump{ f' ( u) \partial _{n}u },  \mean{v}    )_{\mathcal{F}^{int}_{h} } + ( \mean{ f' ( u) \partial _{n}u }, \jump{ v }    )_{\mathcal{F}^{int}_{h} } +  ( f' ( u) \partial _{n}u, v) _{\Gamma } \\
                                                                        & =  ( f' ( u) \partial _{n}u, v) _{\Gamma }
                \end{split}
            \]
            The jump terms vanishes by the regularity of $u$ and $v$. Hence, by inserting $g_{1}$ we have shown that the first formulation holds.
        \item  Doing a new iteration of Greens thorem on \eqref{eq:1_gr} we get the following terms.
\[
    \begin{split}
-(\Delta f( u) , v)_{T }  = -( f( u), \Delta v )_{T} + (f( u), \partial _{n} v  )_{\partial T} - (   f'( u)\partial _{n}u, v )_{\partial T } .
    \end{split}
\]
Now, by doing a summation of all triangles it is clear that this holds.
\begin{equation}
\label{eq:f_g2}
-(\Delta f( u) , v)_{\Omega  }  = -( f( u), \Delta v )_{\Omega } + (f( u), \partial _{n} v  )_{\partial \mathcal{T}_{h} } - (   f'( u)\partial _{n}u, v )_{\partial \mathcal{T}_{h}  }
\end{equation}
It comes evident from the first step of the proof that $ (   f'( u)\partial _{n}u, v )_{\partial \mathcal{T}_{h}  } = (   f'( u)\partial _{n}u, v )_{\Gamma }$, hence, we only need to compute the term $(f( u), \partial _{n} v  )_{\partial
\mathcal{T}_{h} }$ on the facets. \[
    \begin{split}
(f( u), \partial _{n} v  )_{\partial
\mathcal{T}_{h} } & = \sum_{F\in \mathcal{F} _{h}}^{} \int_{F}^{}\jump{ f( u), \partial _{n} v  } \\
& =  (\jump{ f( u)  }  , \mean{ \partial _{n} v }    )_{ \mathcal{F}_{h}^{int} } +(\mean{ f( u)  }  , \jump{ \partial _{n} v }    )_{ \mathcal{F}_{h}^{int} } + (f( u), \partial _{n} v  )_{\Gamma } \\
&=  (\mean{ f( u)  }  , \jump{ \partial _{n} v }    )_{ \mathcal{F}_{h}^{int} } + (f( u), \partial _{n} v  )_{\Gamma }
    \end{split}
\]
Again one of the jump terms vanishes because of the regularity of $u$.
Inserting the result into \eqref{eq:f_g2} have we shown that the second formulation also holds.
    \end{enumerate}
\end{proof}

Hence, we have a scheme for nonlinear spatial discretization. Let $u \in \left[ 0,T \right] \times V_{h}  $ and $v \in V_{h}$. We define the following CIP discretization
\begin{equation}
    \label{eq:cip_ch}
    \begin{split}
        a_{h} \left( u, v \right)   =& ( \Delta  u, \Delta v ) _{ \Omega } \\
                                     & + \left( \mean{  \Delta  u }, \jump{ \partial _{n }v} \right)_{\mathcal{F}_{h}  }  + \left( \mean{ \Delta  v }, \jump{ \partial _{n}u }      \right)_{\mathcal{F}_{h}  }  + \frac{\gamma }{h}  \left( \jump{ \partial _{n} u}, \jump{ \partial _{n} v   }   \right)_{\mathcal{F}_{h} } \\
                                     & + \left(   \Delta  u ,  \partial _{n }v \right)_{\Gamma   }  + \left(  \Delta  v ,  \partial _{n}u       \right)_{\Gamma  }  + \frac{\gamma }{h}  \left(  \partial _{n} u,  \partial _{n} v      \right)_{ \Gamma } \\
    c^{}_{h} ( u,v)  & = -( f( u), \Delta v )_{\mathcal{T} _{h}}+  ( f( u) , \jump{ \partial _{n}v }  )_{\mathcal{F} _{h}^{int}} + ( f(u), \partial _{n} v)_{\Gamma  }  + ( f'( u)  g_{1}   ,  v)_{\Gamma } \\
    l_{h}( v_{}) & =  \left( g_{0}, v \right) _{\Omega } -  \varepsilon ( g_{2},  v )_{\Gamma }  -  \varepsilon ( g_{1}, \Delta  v  )_{\Gamma }  + \varepsilon \frac{\gamma }{h} ( g_{1}, \partial _{n} v  )_{\Gamma }
    \end{split}
\end{equation}
s.t. that the following relation ship holds.
\[
    ( \partial _{t}u, v)_\Omega + \varepsilon  a_{h}( u,v) + \frac{1}{\varepsilon }c_{h}( u,v)   =  l_{h}(v)
\]


\subsection{Development of an Implicit-Explicit Scheme}
\label{sub:implicit_explicit_scheme}

The primary aim is to formulate an uncomplicated time iteration scheme. Define the index $m$ to range over the set ${0, 1, \ldots, M}$, which corresponds to uniformly distributed time points $t_{m}$ that satisfy the boundary conditions $t_{0} = 0$
and $t_{M} = T$. Here, each time step is an element of the function space, i.e. $u^{m}_{h} \in V_{h}$  with the initial condition defined as $u^{0} = u( t_{0},x )$.
For the Implicit explicit scheme (IMEX) sceheme we have the following discretization
\[
( \overline{\partial } _{t} u^{m}_{h}, v   )_{\Omega } + \varepsilon a^{m}_{h}( u_{h}^{m} , v) + \frac{1}{\varepsilon } c_{h} (  u_{h}^{m-1}, v)  = l_{h}( v) , \quad \forall v \in V_{h}^{m}.
\]

Here we have define the forward difference operator with time step $\tau $
\[
\overline{\partial } _{t} u_{h}^{m} = \frac{u_{h}^{m} - u_{h}^{m-1}}{ \tau }
\]
, then we have the following scheme.
\[
( u_{h}^{m},v )_{\Omega }  + \tau \varepsilon a_{h}( u_{h}^{m} , v)   = \tau  l_{h}( v) +   ( u_{h}^{m-1},v )_{\Omega } - \frac{\tau}{\varepsilon } c_{h} (  u_{h}^{m-1}, v) .
\]

 % Recall that $V_{h}$ is spanned by the orthonormal polynomial basis $ \left\{ \phi _{j} \right\}_{i=1}^{ N}  $ with $N$ degrees of freedom  s.t.  $v = \sum_{i}^{N} V_{i} \phi_{i}   $ and $u_{h} = \sum_{i}^{N} U_{i} \phi_{i}   $.
 %    Let  $U^{m} = \left[ U_{1}, \ldots, U^{N} \right]^{T} $ and $V= \left[ V_{1}, \ldots, V_{M} \right]^{T} $. Writing the matrices $[ A ]_{i,j} =  a( \phi _{i}, \phi _{j})  $ and $\left[ F \right] _{j} = l_{h}( \phi _{j})  $ and $ \left[ C \right]_{ij}
 %    = c_{h}( U^{m-1} \phi _{j}, \phi _{j}) $  .
 %    Subsequently, we can express the system in an equivalent explicit system.
 %    \[
 %        \begin{split}
 %            U^{m}  + \tau \varepsilon A U^{m} & = ( I  + \tau \varepsilon A ) U^{m}   \\
 %            b(U^{m-1}, V) &:=   \tau  F +   ( u_{h}^{m-1},v )_{\Omega } - \frac{\tau}{\varepsilon } c_{h} (  u_{h}^{m-1}, v)
 %        \end{split}
 %    \]
 %    s.t. $  ( I  + \tau \varepsilon A ) U^{m} = b(U^{m-1}, V )  $


\subsection{Introduction to numerical methods for Cahn Hilliard}%
\label{sub:introduction_to_numerical_methods_for_cahn_hilliard}

Present numerical results for CH in separate chapter including
\begin{itemize}
    \item Your numerical EOC for  manufactured solution for \textbf{linear} 4th order parabolic
    \item Your numerical EOC for  manufactured solution for \textbf{full} C-H
    \item Nice numerical example in 2d (flower)
    \item Nice numerical example in 3d (popcorn/doughnut geometries)
\end{itemize}


