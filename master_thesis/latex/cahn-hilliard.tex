
\newpage
\section{Cahn Hilliard equation }%
\label{sec:cahn_hilliard_equation}


\begin{itemize}
    \item Explain the full discretization scheme including Follow \cite{feng2007fully}
    \item CutFEMCIP for spatial discretization
    \item implicit Euler for temporal discretization
    \item linearization of nonlinearity (IMEX)
\end{itemize}


\subsection{The Problem}%
\label{sub:the_problem}

We will consider General Cahn Hilliard problem is on the form
\begin{equation}
\label{eq:ch_exact}
    \begin{split}
        \partial _{t} u & - \Delta ( \varepsilon  \Delta u - \frac{1}{\varepsilon } f( u) ) =  h(x)  \quad \text{in} \Omega \\
        \partial _{n} u & =  g_{1}(x)  \quad \text{ on } \Gamma \\
        \partial _{n} \Delta u & = g_{2}(x)  \quad \text{ on } \Gamma  \\
        u(t=0) & = u_{0}   \text{ on } \Omega
    \end{split}
\end{equation}

Where $f( s)  = F' ( s)=s( s^2 -1)  $ and $F( s)  = \frac{1}{4} ( s^2 - 1)^2 $. $\Omega \subset \mathbb{R} ^{d} $  for $d = 2,3$.



\subsection{ CIP method for the Cahn Hilliard Equation}%
\label{sub:writing_the_cahn_hilliard_equation_of_weak_form}

Our goal is to write the Cahn Hilliard equation on weak form.
Assume that $\Omega  \subset \mathbb{R} ^{d}$ is a polygon.
 Let $u \in \left[ 0,T \right] \times H^{4}( \Omega ) $ and $v \in V$ where \[
V = \left\{ v \in H^{1}( \Omega )   \mid  v \mid _{T} \in H^{m}( T)  \forall T \in \mathcal{T} _{h} \right\}
\]

Expanding the first Laplace operator can we observe that,
\[
- ( \Delta ( \varepsilon  \Delta u - \frac{1}{\varepsilon } f( u) ) ,v )_{\Omega } = (  \varepsilon   \Delta^{2} u ,v )_{\Omega } - \frac{1}{\varepsilon } ( \Delta f( u)  ,v )_{\Omega }.
\]


\[
V_{h} = \left\{ v \in C^{0}( \Omega )   \mid v\in \mathcal{P} ^{k} ( T) \quad \forall T \in \mathcal{T} _{h} \right\}.
\]


\subsection{Weak form of the nonlinear term}%
\label{sub:derivation_of_the_cahn_hilliard_cip_formulation}

\begin{lemma}[Nonlinear weak form]
    Let $u \in H^2( \Omega ) $ be the exact solution to \eqref{eq:ch_exact} and $v \in V_{h}$.
Then is $( -\Delta f( u) , v)_{\Omega } = c_{h}(u,v )   $ s.t.
\[
    c^{m}_{h} ( u,v) = - ( f( u), \Delta v )_{ \Omega } + ( f( \mean{ u }  ), \jump{ \partial _{n} v }   )_{\mathcal{F}_{h} }  \red{- ( f' (  u  )  \partial _{n}u ,  v      )_{\mathcal{F}_{h} ^{ext}}}
\]

\end{lemma}

\begin{proof}
  We can easily see that \[
    \begin{split}
-(\Delta f( u) , v)_{T } &= -(\nabla f( u), \nabla v  )_{T } + ( \partial _{n}  f( u), v )_{\partial T }.\\
                    & = -( f( u), \Delta v )_{T} + (f( u), \partial _{n} v  )_{\partial T} + (   f'( u)\partial _{n}u, v )_{\partial T } .
    \end{split}
\]
And by doing the summation of the triangles we get the following.
\[
\sum_{T \in  \mathcal{T} _{h} }^{} -(\Delta f( u) , v)_{T } = \sum_{T \in \mathcal{T} _{h}}^{}  -( f( u), \Delta v )_{T} + (f( u), \partial _{n} v  )_{\partial T} + (   f'( u)\partial _{n}u, v )_{\partial T }
\]

Here we see that \[
    \begin{split}
\sum_{T \in \mathcal{T}_{h} }^{} ( f( u) , \partial _{n}v)_{\partial T} & =  \sum_{F \in \mathcal{F}_{h} }^{} \int_{F}^{}  \jump{    f( u) \partial _{n}v}  = \sum_{F}^{} ( \jump{f( u)} , \mean{ \partial _{n}v }  )_{F} + \sum_{F}^{} ( \mean{f( u)}
, \jump{ \partial _{n}v }  )_{F} \\
 & =   \sum_{F}^{} ( \mean{f( u)} , \jump{ \partial _{n}v }  )_{F} = \sum_{F}^{} ( f(\mean{ u}) , \jump{ \partial _{n}v }  )_{F}
    \end{split}
\]

Here we used that $ \jump{ f( u)  } = 0  $ and $\mean{ f( u)  } = f( u) = f( \mean{ u }  )    $  since $u \in H^2( \Omega ) $. Also remark that $\jump{ \partial _{n}v } \neq 0   $ from $v \in V_{h}$. We can do similar procedure for the other term

\[
    \begin{split}
\sum_{T \in \mathcal{T} _{h}}^{} (   f'( u)\partial _{n}u, v )_{\partial T } & = \sum_{F \in \mathcal{F} _{h}}^{} (\mean{ f'( u)\partial _{n}u }  , \jump{ v }   )_{F} +  (\jump{ f'( u)\partial _{n}u }  , \mean{ v }   )_{F} \\
                                                                             & = \sum_{F \in \mathcal{F} _{h}}^{} (\mean{ f'( u)} \mean{ \partial _{n}u }  , \jump{ v }   )_{F} = \red{\sum_{F \in \mathcal{F} _{h}^{ext}}^{} ( f'( u)  \partial _{n}u   ,  v
                                                                             )_{F} = 0} \\
    \end{split}
\]
Again, here a similar identities $\jump{  f' ( u) \partial _{n} u} = 0  $ since $u \in H^2( \Omega ) $. Finally, the last term is vanished by $\left[ v \right] = 0 $ on interior facets \red{and on external facets we  imposed the boundary condition $\partial _{n}u =0$ on $\Gamma $}.

\end{proof}


\subsection{Implicit Explicit Scheme}%
\label{sub:implicit_explicit_scheme}


For the Implicit explicit scheme (IMEX) sceheme we have the following discretization
\[
( \overline{\partial } _{t} U^{m}, v   )_{\Omega } + \varepsilon a^{m}_{h}( U^{M} , v) + \frac{1}{\varepsilon } ( P^{m} U^{m-1}, v) _{\Omega } =0, \quad \forall v \in V_{h}^{m}.
\]


\subsection{Introduction to numerical methods for Cahn Hilliard}%
\label{sub:introduction_to_numerical_methods_for_cahn_hilliard}

Present numerical results for CH in separate chapter including
\begin{itemize}
    \item Your numerical EOC for  manufactured solution for \textbf{linear} 4th order parabolic
    \item Your numerical EOC for  manufactured solution for \textbf{full} C-H
    \item Nice numerical example in 2d (flower)
    \item Nice numerical example in 3d (popcorn/doughnut geometries)
\end{itemize}


