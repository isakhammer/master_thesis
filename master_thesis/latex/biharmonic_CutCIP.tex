
\newpage
\section{CutCIP Biharmonic problem}%
\label{sec:cutcip_biharmonic_problem}

\subsection{Weak formulation}%
\label{sub:weak_formulation}


Let $\Omega \subseteq    \mathbb{R} ^d$ be a physical mesh, $\Gamma  $ be a $C^2$ boundary and the function $\alpha: \Omega \to \mathbb{R}  $. We define the strong biharmonic problem to be on the form

\begin{equation}
\begin{split}
    \Delta^2  u  + \alpha  u  & = f( x)  \quad \text{in } \Omega,   \\
    \partial _{n} u & = g_{1}(x)   \quad \text{on } \Gamma ,  \\
    \partial _{n} \Delta  u & = g_2(x)  \quad \text{on } \Gamma  .  \\
\end{split}
\end{equation}
The goal is to write the problem on a weak form.
Let $u \in H^{4}( \Omega ) $ be a solution of the strong problem and $v \in H^{2}( \Omega ) $ be a test function. We argue that this holds,
    \[
        \begin{split}
(\Delta ^2u,v )_{\Omega } & = ( \partial _{n} \Delta u, v)_{\Gamma } - ( \nabla ( \Delta u) , \nabla v) _{\Gamma } \\
&= ( D^2u, D^2v)_{\Omega } + ( \partial _{n} \Delta u ,v)_{\Gamma } - ( \partial _{nn} u, \partial _{n}v)_{\Gamma } - ( \partial _{tn} u, \partial _{t} v)_{\Gamma }.        \\
        \end{split}
    \]
    Define the space $V_{g} = \left\{ u \in H^{2}( \Omega ) :  \partial _{n}u  \mid _{\Gamma } = g_1(x )  \right \} $ and $V = H^2( \Omega ) $.
    % By applying the boundary conditions can we see that $( \partial _{n} \Delta u, v)_\Gamma   = ( g,v)_{\Gamma }$ which occurs naturally. However, the Neumann boundary condition can be imposed by assuming $u \in V_{0}$ and
    %  as a consequence is the following terms $ ( \partial _{nn} u, \partial _{n}v)_{\Gamma }
    % = 0$   and $( \partial _{tn} u, \partial _{t}v)_{\Gamma } = 0$. This follows from the fact that that the Neumann boundary condition is homogeneous s.t. $\partial _{n} v = 0$ and $\partial _{t} (\partial _{n}u) = 0 $.
    % \todo[inline]{ But what if the Neumann conditions is not homogeneous? And why is it necessary to include it in the function space when it comes in "naturally"?  }
    Let $u,v \in V$ then is it natural to define an general bilinear form  \[
    a( u,v) = ( D^2u, D^2v)_{\Omega } + ( \partial _{n} \Delta u ,v)_{\Gamma } - ( \partial _{nn} u, \partial _{n}v)_{\Gamma } - ( \partial _{tn} u, \partial _{t} v)_{\Gamma }.
    \]
    where $\alpha >0$ and a corresponding linear form $l: V \to \mathbb{R} $,
    \[
    l( v) = ( f ,v)_{\Gamma } +  ( h_{2},v)_{\Gamma } .
    \]
    We define the weak problem is to find a $u \in  V_{g}$ s.t. \[
    a( u,v) = l(v) \quad  \forall v \in V_{}
    \]

\subsection{Initial discrete formulation}%
\label{sub:initial_discrete_formulation}

We want to make a cutFEM version of the CIP problem. Let $\widetilde{\mathcal{T}_{h} } $ be a shape-regular and quasi-uniform background mesh. Let us denote the active set $\mathcal{T} _{h} \subseteq \widetilde{\mathcal{T}_{h}}$ which intersects the interior of the active domain $\Omega $, that is  \[
\mathcal{T} _{h} = \left\{ T \in \widetilde{\mathcal{T} _{h}}  \mid  T \cap (\Omega \setminus \Gamma ) \neq \emptyset    \right\} .
\]
With a corresponding set of interior facets, \[
    \mathcal{F} _{h} = \left\{ F = T^{+} \cap T^{-}  \mid  T^{+}, T^{-} \in \mathcal{T} _{h} \right\},
\]
and a set of cut elements \[
\mathcal{T} _{\Gamma } = \left\{ T \in \mathcal{T} _{h}   \mid  T \cap \Gamma \neq \Omega \right\}.
\]
We denote the $C^{0}$ polynomial space of order $k$ as
\[
V_{h} = \left\{ v \in C^{0}\left( \Omega  \right): v_{T} = v | _{T} \in \mathcal{P} ^{k}\left( T \right), \forall T \in
\mathcal{T}_{h}    \right\}
\]
From the previous chapter can we write the CIP method. The bilinear form $a_{h}:  V_{h}\times  V_{h} \to \mathbb{R} $ is defined as

\begin{equation}
\label{eq:Bi_a_h}
\begin{split}
a_{h} \left( u, v \right)   =&   \left( \alpha  u, v \right) _{\mathcal{T} _{h} \cap \Omega }   +  \left( D^2 u, D^2v \right) _{\mathcal{T} _{h} \cap \Omega} \\
 & +
  \left( \mean{  \partial _{n n} u }, \jump{ \partial _{n }v} \right)_{\mathcal{F}_{h}^{int} \cap \Omega}  +
 \left( \mean{ \partial _{n n} v }, \jump{ \partial _{n}u }      \right)_{\mathcal{F}_{h}^{int} \cap \Omega} \\
 & + ( \partial _{nn} u, \partial _{n} v)_{\Gamma } + ( \partial _{nn} v, \partial _{n} u)_{\Gamma }
 \\
 & + \frac{\gamma }{h}  \left( \jump{ \partial _{n} u}, \jump{ \partial _{n} v_{}   }   \right)_{\mathcal{F}_{h}^{int} \cap \Omega} +  \frac{\gamma }{h}  \left(  \partial _{n} u,  \partial _{n} v_{}      \right)_{\Gamma } \\
\end{split}
.
\end{equation}
\todo[inline]{TODO: Remove $\mathcal{F} ^{int}_{h}$ and use $\mathcal{F} _{h}$ instead.   }
Similarly the linear form is defined as
 \[
l_{h}( v) =  ( f,v)_{\mathcal{T} _{h} \cap \Omega } - (g_{2},v) - ( g_{1}, \partial _{nn}v) _{\Gamma } + \frac{\gamma }{h}  ( g_{1}, \partial _{n} v)   .
\]
To make sure the problem is stabilized will we add a ghost-penalty. That is, we define the discrete problem to find a $u_{h} \in V_{h}$ s.t. \[
A_{h}( u_{h} ,v ) := a_{h}( u_{h}, v)  + g_{h}( u_{h},v) = l_{h} ( v) \quad  \forall v \in  V_{h}.
\]
We define the underlying norms for $ v \in V_{h} $ as
    \begin{align}
        \label{eq:bi_ah_norm}
        \| v \|_{ a_{h} }^{ 2 } & =    \| |\alpha |^{\frac{1}{2}}  v \|_{ \mathcal{T} _{h} \cap \Omega  }^{ 2}  + \| D^2 v \|_{\mathcal{T} _{h} \cap \Omega   }^{ 2 } + \gamma \| h^{-\frac{1}{2}} \jump{ \partial _{n} v }   \|_{ \mathcal{F}_{h}^{int}\cap \Omega    }^{ 2
        } + \gamma \| h^{-\frac{1}{2}}  \partial _{n} v    \|_{ \Gamma   }^{ 2 },    \\
        \label{eq:bi_gh_norm}
\abs{ v } _{g_{h}}^{2} & = g( v,v) \\
        \label{eq:bi_Ah_norm}
\| v \|_{A_{h}  }^{  2}  & = \| v \|_{ a_{h} }^{ 2 } + \abs{ v } _{g_{h}}^{2}
    \end{align}
and for $v \in V + V_{h}$ we get, \[
\| v \|_{ a_{h}, * }^{  2} =\| v \|_{ a_{h} }^{ 2 } +  \| h^{\frac{1}{2}} \mean{ \partial _{nn} v }   \|_{\mathcal{F} _{h}^{int} \cap \Omega   }^{  2} +  \| h^{\frac{1}{2}} \partial _{nn} v    \|_{ \Gamma }^{  2} .
\]
\begin{remark}
Note that it holds that $\mathcal{T} _{h} \cap  \Omega   = \Omega  $ and $\mathcal{T} _{h} \cap  \Gamma  = \Gamma $. Depending on context, we choose the best suitable notation.
\end{remark}

\subsection{Stability estimate}%
\label{sub:stability_estimate}

Similarly for the Poisson problem will we have the following assumptions for the computational mesh;

\begin{enumerate}[label=\textbf{S.\arabic*}]
    \item\label{as:s1} Boundary $\Gamma $ is of $C^2$
    \item\label{as:s2} The mesh $\mathcal{T} _{h}$ is quasi-uniform.
    \item \label{as:s3}For a $T \in \mathcal{T} _{\Gamma }$ there exists a path $P$ of $diam(P) \lesssim h$ which contains $T$ and an element $T'$ with a so-called fat intersection $
    \abs{ T' \cap \Omega  } _{d} \ge \abs{ T' } _{d}$.
\end{enumerate}

From basic theory we have the following inverse estimate for $ v \in \mathcal{P}^{k}( T)$ s.t. \[
     \| \partial _{nn}  v \|_{F   }^{ }  \lesssim  \| h_{T}^{-\frac{1}{2}} D ^2 v \|_{ T }^{  },
\]
where the hidden constant depend on dimension $d$, order $k$ and the shape regularity. Similarly for cut elements is it easy to see that this must hold,
\begin{equation*}
     \| \partial _{nn}  v \|_{F \cap \Omega    }^{  }  \lesssim\| \partial _{nn}  v \|_{F }^{  }  \lesssim   \| h_{T}^{-\frac{1}{2}} D ^2 v \|_{ T }^{  }.
\end{equation*}
A useful variant is the following inequality that is,
\begin{equation*}
\| \partial _{nn} v \|_{ \Gamma \cap T  }^{  } \lesssim h^{-\frac{1}{2}} \| D^2 v \|_{ T }^{  }.
\end{equation*}
Summation the inverse inequalities over $\mathcal{F}_{h} $ and $\mathcal{T}_{h} $ implies that
\begin{align}
\label{eq:bi_cut_inverse_1}
\| \partial _{nn} v \|_{ \mathcal{T} _{h} \cap \Gamma  }^{  } &\lesssim h^{-\frac{1}{2}} \| D^2 v \|_{ \mathcal{T}_h }^{  }, \\
\label{eq:bi_cut_inverse_2}
\| \partial _{nn}  v \|_{ \mathcal{F}_h \cap \Omega    }^{  }  &  \lesssim   h^{-\frac{1}{2}} \| D^2 v \|_{ \mathcal{T}_h  }^{  }.
\end{align}
In fact, combining the inequalities we get the identity,
\begin{equation}
\label{eq:bi_identity}
h\| \partial _{nn}  v \|_{ \mathcal{F}_h \cap \Omega    }^{2 } + h\| \partial _{nn} v \|_{ \mathcal{T} _{h} \cap \Gamma  }^{2  } \lesssim \| D^2 v \|_{ \mathcal{T} _{h}  }^{2  }.
\end{equation}

We may introduce our first assumption on the ghost penalty.
\begin{assumption}[EP1]
    \label{as:bi_EP1}
    The ghost penalty $g_{h}$ extends the $H^{1}$ norm s.t. \[
    \| D^2 v \|_{ \mathcal{T} _{h} }^{ 2 } \lesssim  \| D^2 v \|_{ \Omega  }^{ 2 } + \abs{ v } _{g_{h}}^{2}.
    \]
\end{assumption}


Combing the results we get the following convenient corollary.

\begin{corollary}
    \label{cor:bi_inverse_thm}
    Let $g_{h}$ satisfy Assumption \ref{as:bi_EP1} then
    \[
            h\| \partial _{nn}  v \|_{ \mathcal{F}_h^{int} \cap \Omega    }^{2 } + h\| \partial _{nn} v \|_{ \mathcal{T} _{h} \cap \Gamma  }^{2  }   \lesssim  \| D^2 v \|_{ \Omega  }^{ 2 } + \abs{ v } _{g_{h}}^{2} \\
              \lesssim \| v \|_{ A_{h} }^{  2}
    \]
\end{corollary}
\begin{proof}
    The first inequality is a direct result of \eqref{eq:bi_identity} and Assumption \ref{as:bi_EP1}. The second inequality is simply a results of the definition \eqref{eq:bi_Ah_norm}.
\end{proof}

\begin{lemma}
    The discrete form $A_{h}$ is coercive, that is, \[
    \| v \|_{ A_{h} }^{ 2 }  \lesssim A_{h}( v,v) \forall v \in V_{h}
    \]
\end{lemma}

\begin{proof}
    Let $v \in V^{h}$.
    Observe that \[
    A_{h}( v,v) = a_{h}( v,v)  + \abs{ v }_{g_{h}}^{2}
    \]
    Thus, since the second term already is part of the $\| \cdot  \|_{ A_{h} }^{  } $ norm is a good start to focus on the $a_{h}$ term, that is,
    \[
    \begin{split}
       a_{h}( v,v) &=   \|\ |\alpha|^{\frac{1}{2}} \cdot v  \|_{   \Omega   }^{2} + \| D^2v \|_{   \Omega  }^{2  } + 2 ( \mean{ \partial _{nn} v }, \jump{ \partial _{n} v }    )_{\mathcal{F} ^{int}_{h} \cap \Omega }  + 2 (  \partial _{nn} v ,
       \partial _{n} v  )_{\Gamma } \\
                   & \quad+ \frac{\gamma }{h}  \|  \jump{ \partial _{n} v }\|_{\mathcal{F} _{h}^{int}  }^{ 2 } + \frac{\gamma }{h}  \| \partial _{n} v \|_{ \Gamma  }^{ 2 }
    \end{split}
    \]
    We will first focus on the symmetry terms. Using Cauchy-Schwarz we observe that \[
        \begin{split}
    ( \mean{ \partial _{nn} v }  , \jump{ \partial _{n} v }  )_{\mathcal{F}^{int}_{h}\cap \Omega  } & \ge - \| h^{\frac{1}{2}}\mean{ \partial _{nn} v }   \|_{ \mathcal{F}^{int}_{h}\cap \Omega   }^{  }  \|h^{-\frac{1}{2}} \jump{ \partial _{n} v }   \|_{
    \mathcal{F}^{int}_{h}\cap \Omega   }^{  } \\
    (  \partial _{nn} v   ,  \partial _{n} v   )_{\Gamma   } & \ge - \| h^{\frac{1}{2}} \partial _{nn} v    \|_{ \Gamma    }^{  }  \|h^{-\frac{1}{2}}  \partial _{n} v    \|_{ \Gamma    }^{  }
        \end{split}
    \]
    Using inverse-inequalities \eqref{eq:bi_cut_inverse_1} and \eqref{eq:bi_cut_inverse_2} and the Corollary \ref{cor:bi_inverse_thm} can we easily observe that \[
        \begin{split}
     \| \mean{ \partial _{nn}v } \|_{ \mathcal{T}_{h} \cap \Omega    }^{  2} & \le C_{1} \| D^2 v \|_{ \mathcal{T}_{h}   }^{2  } \le  C  (\| D^2 v \|_{ \Omega  }^{ 2 }  + \abs{ v } _{ g_{h} }^{2  } )  \\
     \|  \partial _{nn}v  \|_{ \Gamma     }^{ 2 } & \le C_{2} \| D^2 v \|_{ \mathcal{T} _{h}  }^{2  } \le C  (\| D^2 v \|_{ \Omega  }^{ 2 }  + \abs{ v } _{ g_{h} }^{2  } )
        \end{split}
    \]
    Thus, by applying Youngs $\varepsilon $-inequality, $2ab \le  \varepsilon^{-1} a^{2} + \varepsilon b^{2} $, is it natural to see that,
    \[
        \begin{split}
- C_{1}^{\frac{1}{2}} \| D^2 v    \|_{ \mathcal{T} _{h}   }^{  }  \|h^{-\frac{1}{2}} \jump{ \partial _{n} v }   \|_{ \mathcal{F}^{int}_{h}\cap \Omega   }^{  }
& \ge - \frac{1}{\varepsilon } C  (\| D^2 v \|_{ \Omega  }^{ 2 }  + \abs{ v } _{ g_{h} }^{2  } ) -  \varepsilon \|h^{-\frac{1}{2}} \jump{ \partial _{n} v }   \|_{ \mathcal{F}^{int}_{h}\cap \Omega   }^{2  } \\
- C_{2}^{\frac{1}{2}}  \| D^2 v \|_{ \mathcal{T} _{h} }^{  } \| h^{-\frac{1}{2}}  \partial _{n} v    \|_{ \Gamma    }^{  }
& \ge - \frac{1}{\varepsilon } C  (\| D^2 v \|_{ \Omega  }^{ 2 }  + \abs{ v } _{ g_{h} }^{2  } ) -  \varepsilon \|h^{-\frac{1}{2}}  \partial _{n} v    \|_{ \Gamma    }^{2  } \\
        \end{split}
    \]
    Combining these ideas do we end up with the following inequality,
    \[
    \begin{split}
       a_{h}( v,v)  \ge& \     \|\ |\alpha|^{\frac{1}{2}} \  v  \|_{   \Omega   }^{2} +\| D^2v  \|_{   \Omega   }^{2} -  \frac{1}{\varepsilon } 4C  (\| D^2 v \|_{ \Omega  }^{ 2 }  + \abs{ v } _{ g_{h} }^{2  } )  \\
                       & + (\gamma - 2\varepsilon  )\left( \|h^{-\frac{1}{2}}  \jump{ \partial _{n} v }\|_{\mathcal{F} _{h}^{int} \cap \Omega   }^{ 2 } + \| h^{-\frac{1}{2}} \partial _{n} v \|_{ \Gamma  }^{ 2} \right)        \\
    \end{split}
    \]
    This inequality is useful, since if we apply it on the $\| \cdot  \|_{ A_{h} }^{  } $ we have a extra ghost penalty term s.t.,

    \[
        \begin{split}
     A_{h}( v,v) & = a( v,v) + \abs{ v }_{g_{h}}^{2} \\
     &\ge  \   \| \ |\alpha|^{\frac{1}{2}} \  v  \|_{   \Omega   }^{2} + (1  - \frac{1}{\varepsilon } 4C)  (\| D^2 v \|_{ \Omega  }^{ 2 }  + \abs{ v } _{ g_{h} }^{2  } )  \\
                       & + (\gamma - 2\varepsilon  )\left( \|h^{-\frac{1}{2}}  \jump{ \partial _{n} v }\|_{\mathcal{F} _{h}^{int}\cap \Omega   }^{ 2 } + \| h^{-\frac{1}{2}} \partial _{n} v \|_{ \Gamma  }^{ 2} \right)        .
        \end{split}
    \]
    Setting $\varepsilon = 8C$, $\gamma = \frac{33}{2} C$ and $C=1$  we simplify the problem to \[
        \begin{split}
           A_{h}( v,v)  \ge& \   \| \ |\alpha|^{\frac{1}{2}} \    v  \|_{  \Omega   }^{2} + \frac{1}{2}  (\| D^2 v \|_{ \Omega  }^{ 2 }  + \abs{ v } _{ g_{h} }^{2  } )  \\
                       & + \frac{1}{2}\left( \|h^{-\frac{1}{2}}  \jump{ \partial _{n} v }\|_{\mathcal{F} _{h}^{int}\cap \Omega   }^{ 2 } + \| h^{-\frac{1}{2}} \partial _{n} v \|_{ \Gamma  }^{ 2} \right) \\
                       \ge& \frac{1}{2}\| v \|_{ A_{h} }^{2  }
        \end{split}
    \]
    Hence, proof is complete.
    \todo[inline]{ Is it allowed to set constants like this? }
\end{proof}


\begin{lemma}
    The discrete form $A_{h}$ is bounded, that is,
    \begin{equation}
    \label{eq:bi_A_h_bounded}
     A_{h}( v,w) \lesssim \| v \|_{A_{h}  }^{  }\| v \|_{A_{h}  }^{  }  \forall w \in V_{h}
    \end{equation}
    Moreover, for $v \in V_{h} + V$  and $w \in V_{h}$ the discrete form $a_{h}$ satisfies
    \begin{equation}
        \label{eq:bi_a_h_bounded}
        a_{h} ( v,w) \lesssim \| v \|_{ a_{h},* }^{  } \| w \|_{ A_{h} }^{  }
    \end{equation}
\end{lemma}

\begin{proof}
    We will divide the proof in two steps.
    \begin{enumerate}[label=\arabic*)]
        \item The goal is to prove the inequality \eqref{eq:bi_A_h_bounded}. \[
                \abs{ A_{h}( v ,w ) } \lesssim   \abs{a_{h}( v, w) }   + \abs{g_{h}( v,w)  }          \]
                By assumption is the ghost penalty $g_{h}$ positive semi-definite, thus, it fulfills the Cauchy-Schwartz inequality \[
                \abs{ g_{h}(v,w ) } \lesssim \abs{ v } _{g_{h}}\abs{ w }_{g_{h}}
                \]
                Hence, by definition is $\abs{ g_{h}(v,w ) } \lesssim \| v \|_{ A_{h} }^{  } \| w \|_{ A_{h} }^{  } $. Now it remains to show that the bilinear term $ a_{h}$ is bounded.
                \begin{equation}
                    \begin{split}
                        \abs{ a_{h} \left( v, w \right) }   \le  &   \abs{\left( \alpha  v, w \right) _{\mathcal{T} _{h} \cap \Omega }  }    +  \abs{\left( D^2 v, D^2w \right) _{\mathcal{T} _{h} \cap \Omega}  }  \\
                                                     & + \abs{\left( \mean{  \partial _{n n} v }, \jump{ \partial _{n }w} \right)_{\mathcal{F}_{h}^{int} \cap \Omega}  }   +
                                                     \abs{\left( \mean{ \partial _{nn } v }, \jump{ \partial _{n}w }      \right)_{\mathcal{F}_{h}^{int} \cap \Omega}  } \\
                                                     & + \abs{\left(  \partial _{n n} v ,  \partial _{n }w \right)_{\Gamma }}     +
                                                     \abs{\left(  \partial _{n n} w ,  \partial _{n}v       \right)_{\Gamma }  }
                                                     \\
                                                     & + \frac{\gamma }{h} \abs{ \left( \jump{ \partial _{n} v}, \jump{ \partial _{n} w   }   \right)_{\mathcal{F}_{h}^{int} \cap \Omega}  } + \frac{\gamma }{h} \abs{ \left(  \partial _{n} v,  \partial _{n} w
                                                     \right)_{\Gamma }  }
                    \end{split}
                \end{equation}
                The strategy is to bound each term individually using Cauchy-Schwartz. We can easily see that $\abs{\left( \alpha  v, w \right) _{\mathcal{T} _{h} \cap \Omega }  }   \lesssim \| v \|_{a_{h}  }^{  } \| w \|_{ a_{h} }^{  } $ and that
                $\abs{\left( D^2 v, D^2w \right) _{\mathcal{T} _{h} \cap \Omega}  } \lesssim \| v \|_{a_{h}  }^{  } \| w \|_{ a_{h} }^{  } $ using Cauchy Schwartz. For the symmetric terms we also apply the inverse inequality
                \eqref{eq:bi_cut_inverse_2}.
                \[
                    \begin{split}
                    \abs{\left( \mean{ \partial _{n n} v }, \jump{ \partial _{n}w }      \right)_{\mathcal{F}_{h}^{int} \cap \Omega}  } & \lesssim  \|\mean{ \partial _{n n} v }  \|_{ \mathcal{F}_{h}^{int} \cap \Omega}^{  }\|\jump{ \partial _{n} w }  \|_{
                    \mathcal{F}_{h}^{int} \cap \Omega}^{  } \\
                    & \lesssim  \|h^{\frac{1}{2}} \partial _{n n} v  \|_{ \mathcal{F}_{h}^{int} \cap \Omega}^{  }\| h^{-\frac{1}{2}} \jump{ \partial _{n} w }     \|_{\mathcal{F}_{h}^{int} \cap \Omega}^{  } \\
                    & \lesssim  \| v \|_{A_{h}  }^{  } \|w    \|_{ a_{h}}^{  }
                    \end{split}
                \]
                Here we used the Corollary \ref{cor:bi_inverse_thm} s.t.  $\|h^{\frac{1}{2}} \partial _{n n}  v \|_{\mathcal{F}_{h}^{int} \cap \Omega} \lesssim \| v \|_{ A_{h}  }^{  }  $.
              The boundedness interior penalty  inequality is showed in this manner, \[
             \frac{\gamma }{h} \abs{ \left( \jump{ \partial _{n} v}, \jump{ \partial _{n} w   }   \right)_{\mathcal{F}_{h}^{int} \cap \Omega}  }  \lesssim  \|h^{-\frac{1}{2}} \jump{ \partial _{n} v}  \|_{ \mathcal{F}_{h}^{int} \cap \Omega }^{  }
             \|h^{-\frac{1}{2}} \jump{ \partial _{n} w}  \|_{ \mathcal{F}_{h}^{int} \cap \Omega }^{  }  \lesssim  \| v  \|_{ a_{h} }^{  }
             \| w  \|_{ a_{h} }^{  }.
             \]
              Now it remains to handle boundary terms. \[
                \abs{ ( \partial _{nn} v, \partial _{n} w)_{\Gamma } }  \lesssim \| h^{\frac{1}{2}}\partial _{nn} v \|_{\Gamma   }^{  } \| h^{-\frac{1}{2}} \partial _{n}w \|_{\Gamma   }^{  }  \lesssim \|  v \|_{A_{h}  }^{  } \| w \|_{ a_{h}   }^{  }
             \]
             Again, here we used the Corollary \ref{cor:bi_inverse_thm}.
             Finally, using the definition of the norm is it easily to see that,
             \[
\frac{\gamma }{h} \abs{ \left(  \partial _{n} v,  \partial _{n} w \right)_{\Gamma }  } \lesssim \gamma  \| h^{-\frac{1}{2}} \partial _{n} v \|_{  \Gamma }^{  } \| h^{-\frac{1}{2}} \partial _{n} w \|_{\Gamma   }^{  } \lesssim \| v \|_{ a_{h} }^{  }
\| w \|_{ a_{h} }^{  } .
             \]

             Obviously is $\| v \|_{a_{h}  }^{  } \lesssim \| v \|_{A_{h}  }^{  }$. Hence, we have showed that all terms in $a_{h}$ is bounded in the $\|\cdot   \|_{A_{h}  }^{  } $ norm.
         \item The goal is to prove \eqref{eq:bi_a_h_bounded} using many of the same ideas as in the first part. Let $v \in V_{h} +V $ and $v \in V_{h}$. Next step is to show that the bilinear term $ a_{h}$ is bounded.
                \begin{equation}
                    \begin{split}
                        \abs{ a_{h} \left( v, w \right) }   \le  &   \abs{\left( \alpha  v, w \right) _{\mathcal{T} _{h} \cap \Omega }  }    +  \abs{\left( D^2 v, D^2w \right) _{\mathcal{T} _{h} \cap \Omega}  }  \\
                                                     & + \abs{\left( \mean{  \partial _{n n} v }, \jump{ \partial _{n }w} \right)_{\mathcal{F}_{h}^{int} \cap \Omega}  }   + \abs{\left( \mean{ \partial _{nn } w }, \jump{ \partial _{n}v }
                                                     \right)_{\mathcal{F}_{h}^{int} \cap \Omega}  } \\
                                                     & + \abs{\left(  \partial _{n n} v ,  \partial _{n }w \right)_{\Gamma }}     +
                                                     \abs{\left(  \partial _{n n} w ,  \partial _{n}v       \right)_{\Gamma }  }
                                                     \\
                                                     & + \frac{\gamma }{h} \abs{ \left( \jump{ \partial _{n} v}, \jump{ \partial _{n} w   }   \right)_{\mathcal{F}_{h}^{int} \cap \Omega}  } + \frac{\gamma }{h} \abs{ \left(  \partial _{n} v,  \partial _{n} w
                                                     \right)_{\Gamma }  }
                    \end{split}
                \end{equation}
                     We can easily observe from the first part that this must holds, \[
    \abs{\left( \alpha  v, w \right) _{\mathcal{T} _{h} \cap \Omega }  }    +  \abs{\left( D^2 v, D^2w \right) _{\mathcal{T} _{h} \cap \Omega}  } \lesssim \| v \|_{ a_{h},* }^{  } \| w \|_{A_{h}  }^{  }.
    \]
    And for the symmetric interior terms,
    \[
        \begin{split}
            \abs{\left( \mean{  \partial _{n n} v }, \jump{ \partial _{n }w} \right)_{\mathcal{F}_{h}^{int} \cap \Omega}  } &\lesssim \| h^{\frac{1}{2}}\mean{ \partial _{nn}v }   \|_{ \mathcal{F}_{h}^{int} \cap \Omega  }^{  } \| h^{-\frac{1}{2}}\jump{ \partial _{n }w}
    \|_{\mathcal{F}_{h}^{int} \cap \Omega  }^{  } \lesssim \| v  \|_{ a_{h},*  }^{  }\|  w \|_{A_{h}  }^{  }, \\
            \abs{\left( \mean{  \partial _{n n} w }, \jump{ \partial _{n }v} \right)_{\mathcal{F}_{h}^{int} \cap \Omega}  } &\lesssim \| h^{\frac{1}{2}}\mean{ \partial _{nn}w }   \|_{ \mathcal{F}_{h}^{int} \cap \Omega  }^{  } \| h^{-\frac{1}{2}}\jump{ \partial _{n }v}
    \|_{\mathcal{F}_{h}^{int} \cap \Omega  }^{  } \lesssim \| w  \|_{A_{h}  }^{  }\|  v \|_{ a_{h},*  }^{  }.
        \end{split}
    \]
    Remark that for $\| h^{\frac{1}{2}} \mean{ \partial _{nn} v }   \|_{\mathcal{F}^{int} _{h} \cap \Omega   }^{  } $ is the norm incorporated in the definition of $\| \cdot  \|_{ a_{h},* }^{  } $, but for $\| h^{\frac{1}{2}} \mean{ \partial _{nn} w }   \|_{
    \mathcal{F}_{h}^{int}\cap \Omega   }^{  } $ was the Corollary \ref{cor:bi_inverse_thm} applied. The jump terms $\|h^{-\frac{1}{2}} \jump{\partial _{n} v  }   \|_{\mathcal{F}^{int}_{h}\cap \Omega    }^{  } $
is incorporated in the $\| \cdot  \|_{ a_{h} }^{  } $ norm, thus, this also holds, \[
\frac{\gamma }{h}\abs{  ( \jump{ \partial _{n} v }  , \jump{\partial _{n} w }  )_{\mathcal{F}^{int} _{h} \cap \Omega } } \lesssim \| h^{-\frac{1}{2}} \jump{\partial _{n} v },      \|_{ \mathcal{F}^{int} _{h}\cap \Omega   }^{  }     \|
h^{-\frac{1}{2}} \jump{\partial _{n} w } \|_{\mathcal{F}^{int} _{h}\cap \Omega
}^{  } \lesssim \| v \|_{ a_{h},* }^{  } \| w \|_{ A_{h} }^{  }
\]
Finally, the boundary terms, \[
    \begin{split}
( \partial _{nn} v, \partial _{n} w)_\Gamma & \lesssim \| h^{\frac{1}{2}} \partial _{nn} v \|_{ \Gamma   }^{  }  \|h^{-\frac{1}{2}} \partial _{n} w \|_{ \Gamma   }^{  } \lesssim \| v \|_{ a_{h},* }^{  } \| w \|_{ A_{h} }^{  }   , \\
( \partial _{nn} w, \partial _{n} v)_\Gamma & \lesssim \| h^{\frac{1}{2}} \partial _{nn} w \|_{ \Gamma   }^{  }  \|h^{-\frac{1}{2}} \partial _{n} v \|_{ \Gamma   }^{  } \lesssim \| w \|_{ A_{h} }^{  } \| v \|_{ a_{h},* }^{  }   , \\
    \end{split}
\]
and \[
\frac{\gamma }{h} \abs{ (  \partial _{n} v, \partial _{n}w)_{\mathcal{F}^{int} _{h} \cap \Omega }   }\lesssim \| h^{- \frac{1}{2}} \partial _{n} v \|_{ \Gamma   }^{  }  \| h^{-\frac{1}{2}} \partial _{n} w \|_{ \Gamma   }^{  } \lesssim \| v \|_{ a_{h},*
}^{  } \| w \|_{ A_{h}  }^{  },
\]
where we just applied the definition of the norms $\| \cdot  \|_{a_{h}  }^{  } $ and $\| \cdot  \|_{a_{h},*  }^{  } $. We have now shown that every term is bounded.

    \end{enumerate}
    Hence, the proof is complete.
\end{proof}

    \todo[inline]{ Why is not $\| \mean{ \partial _{nn} v}   \|_{\mathcal{F}_h^{int}  }^{  } \lesssim \|  \partial _{nn} v   \|_{ \mathcal{F}^{int} }^{  }  $ interesting for $v \in V + V_h$, but is often used for $v \in V$?  Ex. the mean is conserved in the $\|
    \cdot \|_{a_h,*}$ norm}

\newpage
\section{A priori estimates}%
\label{sec:a_priori_estimates}

\newpage
\section{Condition number}%
\label{sec:condition_number}


\newpage
\section{Constructing Ghost Penalties}%
\label{sec:constructing_ghost_penalties}

We have the following assumptions for the ghost penalty.
\begin{enumerate}[label=\textbf{EP\arabic*}]
    \item\label{as:EP1} The ghost penalty $g_{h}$ extends the $H^{1}$ norm s.t. \[
    \| D^2v \|_{ \mathcal{T} _{h} }^{ 2 }  \lesssim \| D^2 v \|_{ \Omega  }^{  2} + \abs{ v } _{g_{h}}^2
    \]
\end{enumerate}


We will construct face-based ghost penalties.

\begin{lemma}
    Let $T_{1},T_{2 } \in  \mathcal{T} _{h}$ be two elements sharing a common face $F$. Then for $v \in V_{h}$  we have \[
    \| v \|_{ T_{1} }^{  }  \lesssim \| v \|_{ T_{2} }^{  } \sum_{0\le j\le k}  {h^{2j +1}}^{} ( \jump{ \partial _{n}^{j} v }, \jump{ \partial ^{j}_{n} v }    )_{F}
    \]

\end{lemma}
\begin{proof}
    See \cite[Lemma 2.19]{gurkan2019stabilized}.
\end{proof}
The goal in this chapter is to engineer an ghost penalty which fulfills these assumptions.
We denote the multi-index $\alpha  = ( \alpha _{1}, \ldots, \alpha _{d})  $ of order $\abs{ \alpha  } = \sum_{i}^{}  \alpha _{i} = k $   and the normal vectors $n^{\alpha } = n_{1}^{\alpha _{1}} \ldots n_{d}^{\alpha _{d}}$.
Recall the notation for the derivates $D^{\alpha } v$ , that is \[
D ^{0} v  = v, \quad   D ^{1}v  = \nabla v \text{ and }  D ^{2} v  = J(\nabla v) = Hess(v).
\]
\todo[inline]{ But $\alpha = ( \alpha _{1}, \ldots, \alpha _{d}) $ cannot be a scalar? Ex. How can $D^{\alpha } = D^{1}$ be a consistent notation.  }
where $J$ is the Jacobian operator. Remark that equivalent element-wise is the following, \[
\left[ \nabla v \right] _{i} = \partial_{x_{i}} v, \quad  \left[ D^2 v \right] _{i,j} = \partial_{x_{i}x_{j}} v,
\]
Let us introduce the notation \[
\partial _{n}^{j} v = \sum_{\abs{ \alpha  } =j }^{k} \frac{D ^{\alpha }v( x) n^{\alpha }}{\alpha !}, \quad \abs{ \alpha  } = \sum_{j}^{d} \alpha_{i}
\]
An useful result that may help us design ghost penalty is the following estimate.

\begin{lemma}
    \label{lemma:local_facet_estimate}
    Let $T_{1}, T_{2} \in  \mathcal{T} _{h} $ share a common facet $F \in \mathcal{F}_{h} $. Then for $v_{} \in  V_{h}$  does this hold \[
    \| v \|_{ T_{1} }^{  2}  \lesssim  \| v \|_{ T_{2} }^{2  }  + \sum_{j=0}^{k}  h^{2j +1} ( \jump{ \partial _{n}^{j} v}, \jump{ \partial _{n}^{j} v}    )_{F}
    \]
        Here is $k$ the polynomial degree.
\end{lemma}

\begin{proof}
    For a detailed proof, see \cite{gurkan2019stabilized}.
\end{proof}

We will now introduce the so-called ghost penalty faces, that is, \[
\mathcal{F} ^{g}_{h} = \left\{ F\in \mathcal{F} _{h} : T^{+}\cap \Gamma \neq \emptyset  \vee T^{-}\cap \Gamma \neq \emptyset  \right\}.
\]

\begin{proposition}
    \label{prop:hessian_change}
    The following identity holds for $j=0,1,2$.
    $$\partial ^{j}_{n} (D^2v) = D^2 ( \partial ^{j}_{n} v)  $$
\end{proposition}

\begin{proof}
        Recall that $\left[ D^2 v \right]_{i,j} = \partial _{x_{i}x_{j}} v $. We will compute each index individually.
    \begin{enumerate}[label=\arabic*)]
        \item $j = 0$ is trivially true.
        \item Let $j=1$. We can then easily see that, \[
        \partial ^{}_{n} ( \partial _{x_{i} x_{j}} v)  = \nabla  ( \partial _{x_{i} x_{j}} v)   n = \partial _{x_{i} x_{j}} (\nabla  v) n =\partial _{x_{i} x_{j}} (\partial _{n} v)
        \]
        Hence, $\partial _{n} (D^2v) = D^2( \partial _{n}v)$.
        \item Let $j=2$. Similarly can we see that, \[
                \begin{split}
                \partial^{2} _{n} (\partial _{x_{i} x_{j}} v) & = n^{T}  D^2(\partial _{x_{i} x_{j}} v) n = n^{T}  J( \nabla (\partial _{x_{i} x_{j}}v) ) n \\
                & =  n^{T}  J(\partial _{x_{i} x_{j}}(\nabla v) ) n = \partial _{x_{i} x_{j}} n^{T} J(\nabla v) n
                \end{split}
            \]
            Thus, taking account for all elements in the matrix we get $\partial^{2} _{n} (D^2v) = D^2( \partial^{2} _{n}v)$.
    \end{enumerate}
    Proof is complete.

\end{proof}

This set is simply all facets that belong to all elements of the active mesh $\mathcal{T} _{h}$  intersected with $\Gamma $.

\begin{lemma}
    \label{lemma:inv_gh_lemma}
    Assume \ref{as:s2} and \ref{as:s3}.
    For $v \in  V_{h}$ it holds that
        \begin{align}
            \label{eq:inv_gh_1}
        \| v \|_{ \mathcal{T} _{h} }^{ 2 }  & \lesssim  \| v \|_{ \Omega  }^{ 2 }  + \sum_{j=0}^{k} h^{2j+1} ( \jump{ \partial ^{j}_{n} v }, \jump{ \partial ^{j}_{n} v}    )_{\mathcal{F}_{h}^{g}}\\
            \label{eq:inv_gh_2}
        \| D ^2 v \|_{ \mathcal{T} _{h} }^{ 2 }  & \lesssim  \| D^2 v \|_{ \Omega  }^{ 2 }  + h^{-2}\sum_{j=0}^{k} h^{2j-1} ( \jump{ \partial ^{j}_{n} v }, \jump{ \partial ^{j}_{n}v }    )_{\mathcal{F}_{h}^{g}}
        \end{align}
        Here is $k$ the polynomial degree.
\end{lemma}

\begin{proof}
    We will dive the proof in two parts where we first prove \eqref{eq:inv_gh_1} and then in the second part prove \eqref{eq:inv_gh_2}.
    \begin{enumerate}[label=\arabic*)]
        \item
            First of all, notice that there is a patch $P(T) $ consisting of $\left\{ T_{i} \right\}_{i=1}^{l} $ mesh elements s.t. each pair $ \left\{ T_{i}, T_{i+1} \right\} $ share a facet $F_{i}$ and the last element $T_{l}$ has a so-called "fat"
            intersection according to Assumption \ref{as:G3}.

            Let us define the following norm \[
            g_{F_{i}}^{L^{2}}( v,v)  = \sum_{j=0}^{k} h^{2j+1}( \jump{ \partial ^{j}_{n}v }, \jump{ \partial ^{j}_{n}v }    )_{F_{i} }
            \]
            where $F_{i} \in  \mathcal{F} ^{g}_{h}$ and polynomial degree $ k$. Using Lemma \ref{lemma:local_facet_estimate} can we see that \[
            \| v \|_{ T_{i} }^{  } \lesssim \| v \|_{ T_{i+1} }^{ 2 } + g_{F_{i}}^{L^{2}}( v,v).
            \]
    Consequently, using induction over each pair $\left\{ T_{i}, T_{i+1} \right\} $ with a corresponding $F_{i}$, we obtain
            \[
                \begin{split}
            \| v \|_{ T_{1} }^{2  }  & \le  C \| v \|_{ T_{2} }^{ 2 } + g_{F_{1}}^{L^{2}}( v,v)\\
              & \le  C( C( \| v \|_{ T_{3} }^{ 2 } + g_{F_{2}}^{L^{2}}( v,v) ) + g_{F_{1}}^{L^{2}}( v,v) )\\
              & \lesssim    \| v \|_{ T_{l} }^{ 2 }  + \sum_{i=1}^{l-1} g_{F_{i}}^{L^{2}}( v,v)  \\
              & \lesssim    \| v \|_{ T_{l} \cap \Omega  }^{ 2 }  + \sum_{i=1}^{l-1} g_{F_{i}}^{L^{2}}( v,v)
                \end{split}
            \]
            Here the last steps arise from the fact that $\|  v \|_{ T_{l} }^{  } \lesssim  \|  v \|_{ T_{l} \cap \Omega  }^{  }  $, which is a consequence of the Assumption \ref{as:s3}.
Summation over the intersected triangles $\mathcal{T} _{\Gamma }$ implies,
            \[
                    \| v \|_{ \mathcal{T} _{\Gamma } }^{2  } \lesssim \| v \|_{ \mathcal{T} _{\Gamma}\cap \Omega  }^{2  }+ \sum_{i=1}^{l-1} g_{F_{l}}^{L^{2}}( v,v) \\
                     = \| v \|_{ \mathcal{T}_{\Gamma } \cap \Omega   }^{ 2 }  + \sum_{j=0}^{k} h^{2j+1} ( \jump{ \partial ^{j}_{n} v }, \jump{ \partial ^{j}_{n}v }    )_{\mathcal{F}_{h}^{g}}
        \]
        And as a trivial extension this now also holds for the active mesh $\mathcal{T} _{h}$ , that is, \[
                    \| v \|_{ \mathcal{T} _{h } }^{2  } \lesssim  \| v \|_{ \mathcal{T}_{h } \cap \Omega   }^{ 2 }  + \sum_{j=0}^{k} h^{2j+1} ( \jump{ \partial ^{j}_{n} v }, \jump{ \partial ^{j}_{n}v }    )_{\mathcal{F}_{h}^{g}}.
        \]
        Hence, \eqref{eq:inv_gh_1} holds and the first part of the proof is complete.

    \item We will simply start by replacing $v$  by $D^2 v$ and use the Proposition \ref{prop:hessian_change}.
        \[
            \begin{split}
                    \| D^2 v \|_{ \mathcal{T} _{h} }^{2  }&  \lesssim \| D^2 v \|_{ \Omega  }^{ 2 }  + \sum_{j=0}^{k} h^{2j+1} ( \jump{   \partial ^{j}_{n} D^2 v }, \jump{  \partial ^{j}_{n} D^2 v}    )_{\mathcal{F}_{h}^{g}} \\
                    &=  \| D^2 v \|_{ \Omega  }^{ 2 }  + \sum_{j=0}^{k} h^{2j+1} ( \jump{   D^2 \partial ^{j}_{n}  v }, \jump{  D^2 \partial ^{j}_{n}  v}    )_{\mathcal{F}_{h}^{g}}
            \end{split}
        \]
        Remark that $\|  D^2 v \|_{ T_{l} }^{  } \lesssim  \|  D^2 v \|_{ T_{l} \cap \Omega  }^{  }  $ also holds based on Assumption \ref{as:s3}.
        Let us denote the tangential operator $P_{F} := I - n_{F} \oplus n_{F} $. We apply this so we can to decompose the Hessian s.t. \[
        D^2 v = ( \partial _{nn}v)n_{F} + P_{F} D^2 v.
        \]
        \todo[inline]{TODO: Finish rest of proof! }
        Recall from basic finite element theory that $\| D^2v \|_{F  }^{  }\le h^{-\frac{3}{2}} \| v \|_{ T }^{  }    $ for $v \in  V_{h}$.
        That is, applying the decomposition we get the following estimates \[
        \| \jump{ P_{F}   D^2 \partial _{n}^{j} v }\|_{ F }^{ 2 } = \| P_{F} D^2 \jump{ \partial _{n}^{j} v }   \|_{ F  }^{ 2} \lesssim h^{-2} \|  \jump{ \partial ^{j}_{n} v }   \|_{ F }^{  2}
        \]
        and \[
            h^{2j +1} \| \jump{\partial ^{j}_{n} \nabla v}  \|_{ F }^{  } \lesssim h^{2j +1 } \| \jump{ \partial ^{j+1}_{n} v }   \|_{F  }^{ 2 } + \| \jump{ \partial ^{j}_{n}v }   \|_{ F }^{ 2 }
        \]
        Thus, fulfilling \eqref{eq:inv_gh_2}.
        % \todo[inline]{ Finish this proof. }

    \end{enumerate}




\end{proof}



Finally, we now have the tools we need to construct an candidate for the ghost penalty for which satisfies all assumptions.

\begin{proposition}[Face-based ghost penalty]
    For any set of positive parameters $\left\{ \gamma _{j} \right\} _{j=0}^{k}$, the ghost penalty defined as \[
    g^{1}_{h}( v,w)  := \sum_{j=0}^{k} \sum_{F \in \mathcal{F} _{h}^{g}}^{} \gamma _{j} h^{2j-1}_{F} ( \jump{ \partial ^{j}_{n} v }, \jump{ \partial ^{j}_{n} w }  ) _{F} \text{ for any } v,w \in V_{h},
    \]
    satisfies the assumptions described in the equations \eqref{eq:EP1}, \eqref{eq:EP2}, \eqref{eq:EP3} and \eqref{eq:EP4}.
\end{proposition}


\begin{proof}
    Using Lemma \ref{lemma:inv_gh_lemma} can we see that \eqref{eq:EP1} and \eqref{eq:EP3} is satisfied, that is \[
    \begin{split}
        \| \nabla v \|_{ \mathcal{T} _{h} }^{  } & \lesssim \| \nabla v \|_{ \Omega  }^{  }  + \abs{ v } _{g_{h}^{1}} \\
        \|  v \|_{ \mathcal{T} _{h} }^{  } & \lesssim \|  v \|_{ \Omega  }^{  }  + \abs{ v } _{g_{h}^{1}}
    \end{split}
    \]
    Thus, it now only reminds to show that $g_{h}^{1}$  satisfies \eqref{eq:EP2} and \eqref{eq:EP4} starting with the first on. As described in the assumption, let $v \in H^{s}( \Omega ) $ and $r = \min \{s, k+1\} $. Then it is easy to see that \[
        \begin{split}
        \abs{ \pi ^{e}_{h} v }_{g_{h}^{l}}^{2} & = \sum_{j=0}^{k} h^{2j-1} \| \jump{ \partial ^{j}_{n}\pi ^{e}_{h} v  }  \|_{\mathcal{F} _{h}^{g}  }^{2  } \\
        &= \sum_{j=0}^{r-1} h^{2j -1} \| \jump{ \partial ^{j}_{n} ( \pi _{h}^{e} v - v^{e}) }    \|_{\mathcal{F} _{h}^{g}  }^{  } +  \sum_{j=r}^{k} h^{2j-1} \| \jump{ \partial ^{j}_{n} \pi ^{e}_{h} v }   \|_{\mathcal{F} _{h}^{g}  }^{ 2 }    \\
        & \lesssim h^{2r- 2}\| v \|_{ r,\Omega  }^{ 2 } + h^{2j-2}\| D^{r} \pi ^{e}_{h} v \|_{ \mathcal{T} _{h} }^{ 2 }   \\
        & \lesssim h^{2r -2 } \| v \|_{ r,\Omega  }^{  }
        \end{split}
    \]
    Remark that we on the first sum in the second equality combined the fact that $\jump{ \partial _{n}^{j} v^{e} }| _{F} = 0 $ for $0\lesssim j \lesssim r-1$ and the approximation \[
    \| v-\pi _{h}^{e} v \|_{ \mathcal{F} _{h},r  }^{  } \lesssim h^{s-r -\frac{1}{2}} \| v \|_{ s,\Omega  }^{  }, \quad 0 \le r \le  s-\frac{1}{2}
\]. The proof of the approximation can be found in \cite[Chapter 2.41]{gurkan2019stabilized}.
\todo[inline]{ Show approximation in detail. Maybe do a longer recap of a priori results.}

The second sum was carried by the following inverse estimate, \[
\| \partial ^{j}_{n}v \|_{ F }^{  }  \lesssim h^{r - j - \frac{1}{2}} \| D^{r} v \|_{ T }^{  } \quad  \forall v \in V^{h},
\]
and the stability of the projection operator $\pi _{h}$ and the Sobolev extension in the $H^{r}$ norm for $r=0$.
\end{proof}



