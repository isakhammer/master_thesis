
\newpage
\section{Unfitted cut continuous interior penalty method for the biharmonic problem }%
\label{sec:biharmonic_problem}


\subsection{Strong form of the biharmonic problem}%
\label{sub:strong_form_of_the_biharmonic_equation}

Let $\Omega \subseteq    \mathbb{R} ^2$ be a bounded polygonal domain and $\partial \Omega $ be its corresponding boundary. Let the inhomogeneous fourth order biharmonic equation have the form,

\begin{equation}
\label{eq:bi_problem}
\begin{split}
    \Delta^2  u  + \alpha  u  & = f( x)  \quad \text{in } \Omega,   \\
    \partial _{n} u & = 0  \quad \text{on } \partial \Omega,  \\
    \partial _{n} \Delta  u & = g(x)  \quad \text{on } \partial \Omega .  \\
\end{split}
\end{equation}
Here is $\Delta ^2 = \Delta  \left( \Delta  \right) $ the biharmonic operator, also known as the bilaplacian. We will assume for the strong form that $u \in H^{4}\left( \Omega  \right) $, $\alpha  \in  \mathbb{R} $ and $f \in L^{2}\left( \Omega  \right)
$. We may consider the functions $g( x ) $ as a time independent boundary conditions. Such problems as \eqref{eq:bi_problem} are often associated with the Cahn-Hilliard model
for phase separation \cite{cahnhilliard1957} .

\subsection{  Weak form biharmonic equation in $H^{4}\left( \Omega  \right) $}%
\label{sub:continious_weak_form_of_biharmonic_equation}


The goal is to find a useful full weak formulation of \eqref{eq:bi_problem}. Now, let the solution space be on the form,
\begin{equation*}
V = \left\{ v \in H^2\left( \Omega  \right) : \partial _{n} v = 0  \text{ on }
\partial \Omega  \right\}.
\end{equation*}

Let $u,v \in  V$, then the derivation of the general weak form is,
\[
\begin{split}
\left( \Delta ^2 u,v \right) _{\Omega }  &  = \left( \partial _{n} \Delta u, v \right) _{\partial \Omega } - \left( \nabla \left( \Delta  u \right) , \nabla v \right) _{\Omega }  \\
\end{split}
\]
In fact, the simplest formulation has the form,
\[
  \left( \nabla \left( \Delta u \right) , \nabla v \right) _{\Omega } =   \left( \Delta u, \partial _{n} v \right) _{\partial \Omega } - \left( \Delta u, \Delta v \right)_{\Omega },
\]
A major issue with this formulation is that we do not have boundary condition for $\Delta u$. Instead, we can expand the term in the following fashion.

\begin{equation*}
    \begin{split}
\left( \nabla \left( \Delta u \right) , \nabla v \right) _{\Omega } & = \sum_{i = 1}^{ d}  \left( \Delta  \partial _{x_{i}} u, \partial _{x_{i}}v \right) _{\Omega }  \\
&= \sum_{i = 1}^{d}  \left( \nabla \cdot \left( \nabla \partial _{x_{i}} u \right) , \partial _{x_{i}} v \right)_{\Omega }  \\
&= \sum_{i = 1}^{d}  \left( \partial_n  \partial _{x_{i}} u, \nabla  \partial _{x_{i}} v \right) _{\partial \Omega} -   \left( \nabla \partial _{x_{i}} u, \nabla \partial _{x_{i}} v \right)_{\Omega }  \\
&= \left(  \partial_n\nabla u, \nabla v \right) _{\partial \Omega } - \left( D^2 u, D^2v \right) _{\Omega } \\
&= \left( \partial _{nn} u, \partial _{n} v  \right)_{\partial \Omega }   + \left( \partial _{nt} u, \partial _{t} v \right) _{\partial \Omega } - \left( D^2u, D^2v \right) _{\Omega } .
    \end{split}
.\end{equation*}
Hence, the boundary condition of $\Delta u$ is integrated into the formulation.  It can be denoted that $D^2$ is the Hessian matrix operator such that
$$( D^2u, D^2v )_{\Omega } = \int_{\Omega }^{} D^{2}u : D^2v  dx,$$
where $D^2u:D^2v$ is the inner product and similarly for $\partial _{nn} u = n\cdot D^2 u \cdot n$. Thus, we now have a weak form identity,
\begin{equation}
\label{eq:weak_form_identity}
\left( \Delta ^2 u, v \right) _{ \Omega } = \left( D^2u, D^2v \right) _{\Omega} +   \left( \partial _{n} \Delta u, v  \right) _{\partial \Omega }  - (\partial _{nn} u, \partial _{n} v )_{\partial \Omega } - \left( \partial _{nt} u, \partial _{t}v
\right) _{\partial \Omega }
.\end{equation}

Using weak form identity \eqref{eq:weak_form_identity} and the boundary conditions stated in the strong form \eqref{eq:bi_problem} can we write

\begin{equation}
\begin{split}
\left( \Delta ^2 u, v \right) _{ \Omega } & = \left( D^2u, D^2v \right) _{\Omega} +   \underbrace{\left( \partial _{n} \Delta u, v  \right) _{\partial \Omega }}_{ = \left( g,v \right) _{\partial \Omega }}   - \underbrace{(\partial _{nn} u, \partial
    _{n} v )_{\partial \Omega }}_{ = 0}  - \underbrace{\left( \partial _{nt} u, \partial _{t}v \right) _{\partial \Omega }}_{ = 0} \\
    &= \left( D^2u, D^2v \right) _{\Omega } + \left( g,v \right) _{\partial \Omega }  \\
\end{split}
.\end{equation}
\todo[inline]{ Is it a way to prove $\partial _{nn} u = 0$ and $\partial _{nt} u = 0$ on $\partial \Omega $? Does this differ from plate problem vs cahn hilliard?}

Finally, we can define the following bilinear functional $a:V\times V \to  \mathbb{R} $ and the linear functional $F: V \to \mathbb{R} $ s.t.
\begin{equation}
\label{eq:weak_formulation}
\begin{split}
a\left( u,v \right)_{\Omega } & =    \left( D ^2 u , D ^2 v\right)_{\Omega }  +
\alpha \left( u, v \right)_{\Omega }   , \\
F\left( v \right)_{\Omega } & = \left( f,v \right)_{\Omega } - \left(g,v \right)_{\partial \Omega }.
\end{split}
\end{equation}

Thus, we have now the necessary definitions to define the biharmonic problem.

We define the biharmonic problem to solve for $u \in V  $ s.t.
\begin{equation}
    \label{eq:bi_weak1}
a\left( u,v \right) = F(v)\quad \forall v \in
V,
\end{equation}
where $ V = \left\{ v \in H^2\left( \Omega  \right) : \partial _{n} v = 0  \text{ on }
\partial \Omega  \right\}$.

A problem that appear in this formulation is that the solution is only unique for $\alpha  > 0$. However, for $\alpha  = 0$ is it necessary to apply the solvability condition,
\begin{equation*}
 \int_{\Omega }^{} f dx = \int_{\partial \Omega }^{} g ds
.\end{equation*}
This condition easily arise when using the substitution $v=1$ in \eqref{eq:bi_weak1}. To handle this, can we extended the solution space \[
V^{*} = \begin{cases}
    V \quad & \alpha  > 0 \\
    \left\{ v \in V: \int_{\Omega }^{} v dx  = 0\right\} \quad & \alpha  = 0,
\end{cases}
\]
Thus, the unique solution in $v \in V^{*}$ belongs to $H^{3 }(\Omega ) $ and we get the following
elliptic regularity estimate \cite{gu2012c0},
\begin{equation*}
\label{eq:bi_harmonic_ellitpic_regularity}
\left| u \right| _{H^{3 }\left( \Omega  \right) }  \le C_{\Omega } \left( \| f \|_{  L^{2}( \Omega ) }^{  } + ( 1 + \alpha  ^{\frac{1}{2}}
) \cdot \| w  \|_{ H^{4}\left( \Omega  \right)  }^{  }    \right) \quad w\in H^{4}\left( \Omega  \right).
\end{equation*}

Finally, we can define the weak formulation of the biharmonic problem.

\begin{definition}[Biharmonic problem]
    \label{def:biharmonics_problem}

We define the biharmonic problem to find $u \in V^*  $ s.t.
\begin{equation}
a\left( u,v \right) = F(v)\quad \forall v \in
V^* .
\end{equation}

\end{definition}

\subsection{Constructing Continuous Interior Penalty Method}%
\label{sub:constructing_continious_interior_penalty_method}

 Let us assume that $u,v \in
H^{4}\left( T  \right) $. Using that the weak form identity \eqref{eq:weak_form_identity} also holds for a triangle $T$ can we write
\begin{equation}
\label{eq:bi_basic_dg}
\left( \Delta  ^{2} u,v \right) _{T} =  \left( D^2u,D^2v \right) _{T } - \left(\partial _{nt} u, \partial _{t}v
\right)_{\partial T} - \left(\partial _{nn} u, \partial _{n}v \right)_{\partial T} + \left(\partial _{n} \Delta  u,v
\right)_{\partial T}
.\end{equation}
For global continuity, let  $v \in V =  \left\{ v \in H^{1}\left( \Omega  \right): v_{T} \in  H^{4}\left( T \right), \ \forall T \in
\mathcal{T}_{h}    \right\} $ and $u \in  H^{4}\left( \Omega  \right) $ such that,

\begin{equation}
\label{eq:bi_basic_dg2}
\left( \Delta  ^{2} u,v \right) _{\Omega } = \sum_{T \in  \mathcal{T} _{h}}^{}  \left( D^2u,D^2v \right) _{T } - \left(\partial _{nt} u, \partial _{t}v
\right)_{\partial T} - \left(\partial _{nn} u, \partial _{n}v \right)_{\partial T} + \left(\partial _{n} \Delta  u,v
\right)_{\partial T}.
\end{equation}
However, this expression can be written to distinguish integrating over triangles $\mathcal{T} _{h}$ , integrating over exterior facets $\mathcal{F} _{h}^{ext}$ and then integrate interior facets $\mathcal{F} _{h}^{int}$.

\begin{equation}
\label{eq:bi_basic_dg_full_1}
\begin{split}
\left( \Delta  ^{2} u, v \right) _{\Omega } =& \sum_{T \in  \mathcal{T} _{h}}^{} \left( D^2u, D^2v \right)_{T}    \\
& + \sum_{F \in \mathcal{F}_{h}^{ext}}  \left(\partial _{n} \Delta u, v  \right) _{F} - \left(\partial _{nt} u, \partial _{t} v \right) _{F}-
\left( \partial _{nn} u, \partial _{n} v \right)_{F}  \\
& + \sum_{F \in \mathcal{F}_{h}  ^{int}}^{} \left(\partial _{nn} u , \jump{ \partial _{n} v }
\right)_{F} \\
& = \sum_{T \in  \mathcal{T} _{h}}^{} \left( D^2u, D^2v \right)_{T} + \sum_{F \in
\mathcal{F} ^{ext}_{h}}^{} \left(g, v  \right) _{F}
  + \sum_{F \in \mathcal{F}_{h}  ^{int}}^{} \left( \partial _{nn} u , \jump{ \partial_{n} v } \right)_{F}
\end{split}
\end{equation}
Keep in mind that any jump over a interior facet $F \subset \mathcal{F} _{h}^{int}   $, visualized in figure \ref{fig:normal}, is defined as $\jump{ a } =    a^{+} - a^{-} $
and likewise for the mean, $\mean{ a  } = \frac{1}{2}(   a^{+}
+ a^{-})$.    The equivalence of \eqref{eq:bi_basic_dg2} and \eqref{eq:bi_basic_dg_full_1} comes from the following argumentation.

\begin{equation*}
    \begin{split}
 \left( \Delta  ^{2} u,v \right) _{\Omega } & =\sum_{T\in \mathcal{T} _{h}}^{} \left( D^2u,D^2v \right) _{T } - \left(\partial _{nt} u, \partial _{t}v
\right)_{\partial T} - \left(\partial _{nn} u, \partial _{n}v \right)_{\partial T} + \left(\partial _{n} \Delta  u,v
\right)_{\partial T} \\
&= \sum_{T\in \mathcal{T} _{h}}^{} \left( D^2u,D^2v \right) _{T } \\
&  \quad + \sum_{F \in \mathcal{F}_{h}^{ext} }^{} \underbrace{\left( \partial _{n} \Delta  u, v  \right)_{F}}_{= \left( g,v \right)_{F} }  -  \left(
\partial _{nt} u, \partial _{t} v \right) _{F}  - \underbrace{\left( \partial _{nn} u, \partial _{n} v \right)_{F}}_{ = 0}    \\
& \quad  + \sum_{F \in \mathcal{F} _{h}^{int}}^{} \underbrace{\left( \left(\partial _{n^{+}} \Delta  u^{+}
        ,v^{+}\right)_{F}
+ \left(\partial _{n^{-}} \Delta  u^{+} ,v^{-}\right)_{F}  \right)}_{(I)} \\
 & \quad \quad \quad  \quad +
\underbrace{\left( \left(\partial _{n^{+}t} u^{+}, \partial_{t} v^{+} \right)_{F} +  \left(\partial _{n^{-}t} u^{-},
        \partial_{t} v^{-}
\right)_{F}  \right) }_{(II)} \\
 & \quad \quad \quad  \quad  +
\underbrace{\left( \left(\partial _{n^{+}n^{+}} u^{+}, v^{+} \right) _{F} + \left(\partial _{n^{-}n^{-}} u^{-}, v^{-}
\right) _{F} \right) }_{(III)}
    \end{split}
.\end{equation*}

Where integration over all interior facets $ \forall F \in \mathcal{F}_{h}^{int}$ is computed in this way.
\begin{equation*}
    \begin{split}
        (I) &  =    \left(\partial _{n^{+}} \Delta  u^{+} ,v^{+}\right)_{F} +
        \left(\partial _{n^{-}} \Delta  u^{-} ,v^{-}\right)_{F}  \\
        & =   \int_{F}^{}
        \jump{ \partial _{n} \Delta  u \cdot v } =
         \int_{F}^{}
         \mean{ \partial _{n} \Delta  u } \underbrace{\jump{ v }}_{= 0}    + \underbrace{\jump{ \partial _{n} \Delta  u
         }}_{= 0}    \mean{ v } = 0 \\
        (II) &  =     \left(\partial _{n^{+}t} u^{+}, \partial_{t} v^{+}
        \right)_{F} +  \left(\partial _{n^{-}t} u^{-}, \partial_{t} v^{-}
\right)_{F}   \\
&  =   \int_{F}^{}
        \jump{ \partial _{nt} u \cdot  \partial_{t} v } =
         \int_{F}^{}
         \mean{ \partial _{nt} u    } \underbrace{\jump{ \partial_{t} v }  }_{= 0}    + \underbrace{\jump{ \partial
                 _{nt}  u
         }}_{= 0}    \mean{ \partial _{t}v }  = 0\\
        (III) &  =     \left(\partial _{n^{+}n^{+}} u^{+}, \partial_{n^{+}} v^{+} \right)_{F} +  \left(\partial _{n^{-}n^{-}} u^{-}, \partial_{n^{-}} v^{-} \right)_{F}    =    \int_{F}^{} \jump{ \partial _{nn} u \cdot  \partial_{n} v }  \\
        & = \int_{F}^{}
        \mean{ \partial _{nn} u    } \underbrace{\jump{ \partial_{n} v }  }_{\neq 0}    + \underbrace{\jump{ \partial
                 _{nn}  u
         }}_{= 0}    \mean{ \partial _{n}v }   =  \left( \partial _{nn} u, \jump{ \partial_{n} v } \right)_{F}   \end{split}
.\end{equation*}
Observe that the cancellations in the term $(I)$ appears of the continuity of $v\in V $ and $u\in H^{4}\left( \Omega  \right) $ which makes the jumps zero. For the second term $(II)$ does the terms become zero cancelled because the tangential
derivative at the facet has no jump. However, The third term $(III)$  is fairly interesting since the discontinuity in
normal vector for $v \in V$ is a jump, while the second term is still continuous. It can also be raised that $\mean{
\partial _{nn} u } = \partial _{nn} u  $ holds by the continuity of $H^{4}\left( \Omega  \right) $. Hence,
\eqref{eq:bi_basic_dg2} and \eqref{eq:bi_basic_dg_full_1} is equivalent.

\subsection{Formulation of the Continious Interior Penalty Method}%
\label{sub:formulation_of_continious_interior_penalty_method}


We can finally start defining the fully discrete formulation. Let the basis be a $\mathcal{P}_{2} $ Lagrange finite element space so,
\[
V_{h} = \left\{ v \in C^{0}\left( \Omega  \right): v_{T} = v | _{T} \in \mathcal{P} _{2}\left( T \right), \forall T \in
\mathcal{T}_{h}    \right\}
\]
and
\[
V_{h}^{*} = \begin{cases}
    V_{h} & \text{ if } \alpha  > 0 \\
    \left\{ v \in V_{h}: \int_{\Omega }^{} v dx   = 0   \right\} &  \text{ if } \alpha   = 0
\end{cases}
\]
Now, if we choose $u \in V_{h}$, then we must take account that the jump is discrete.
 Finally, the CIP formulation can be stated as follows.
The discretized numerical problem is to solve $w_{h} \in V_{h}^{*}$ such that
\begin{equation}
\label{eq:CP_A_F}
\mathcal{A}\left( w_{h}, v_{h} \right)   = F\left( v_{h} \right), \quad \forall v_{h} \in V_{h}^{*}  .
\end{equation}
where
\begin{equation}
\label{eq:CP_A_h_1}
\begin{split}
\mathcal{A} \left( w_{h}, v_{h} \right)   =&
  \quad  \left( \alpha  w_{h}, v_{h} \right) _{\Omega }\\
&  + \sum_{T \in \mathcal{T} _{h}}^{} \left( D^2 w_{h}, D^2v_{h} \right) _{T} \\
 & +
  \sum_{F \in \mathcal{F}_{h}^{int} }^{}
  \left( \mean{  \partial _{n n} w_{h} }, \jump{ \partial _{n }v_{h}} \right)_{F}  +
 \left( \mean{ \partial _{n n} v_{h} }, \jump{ \partial _{n}w }      \right)_{F} \\
& \quad \quad \quad \quad  + \frac{\gamma}{h}  \left( \jump{ \partial _{n} w_{h}}, \jump{ \partial _{n} v_{h}   }   \right)_{F}
\end{split}
\end{equation}
and
\begin{equation}
\label{eq:CP_F_h}
F\left( v_{h} \right)  = \left( f, v_{h} \right) _{\Omega } +  \sum_ {F \in \mathcal{F}_{h} ^{ext}}^{} - \left(g, v_{h}  \right) _{F}.
\end{equation}
Notice that the regulation term determined by respectively a global tuning parameter $\gamma >0 $. Another key component to the formulation
in \eqref{eq:CP_A_h_1} after introduction of $ w_{h}, v_{h} \in V^{*}_{h}$  is that we expanded $\left( \partial _{nn}w, \jump{ \partial _{n} v }  \right)_{F} \to \left( \mean{ \partial _{nn}w_{h} }  , \jump{ \partial _{n} v_{h} }  \right)_{F} $ since we can longer not guarantee a
continuous jump. For symmetric purposes we also added $ \left( \mean{ \partial _{nn} v_{h}}  , \jump{ \partial _{n} w_{h} }  \right)_{F} $. For convenience will we introduce the compact notation of \eqref{eq:CP_A_h_1},

\begin{equation}
\label{eq:CP_A_h}
\begin{split}
\mathcal{A} \left( w_{h}, v_{h} \right)   =&
  \quad  \left( \alpha  w_{h}, v_{h} \right) _{\Omega }\\
&  +  \left( D^2 w_{h}, D^2v_{h} \right) _{\mathcal{T} _{h}} \\
 & +
  \left( \mean{  \partial _{n n} w_{h} }, \jump{ \partial _{n }v_{h}} \right)_{\mathcal{F}_{h}}  +
 \left( \mean{ \partial _{n n} v_{h} }, \jump{ \partial _{n}w }      \right)_{\mathcal{F}_{h}}
 \\
 & + \frac{\gamma }{h}  \left( \jump{ \partial _{n} w_{h}}, \jump{ \partial _{n} v_{h}   }   \right)_{\mathcal{F}_{h}} \\
\end{split}
.
\end{equation}

\todo[inline]{ I think I need to work on how $ \mathcal{A}$ incorporates the $\partial _{n} v \mid _{\partial \Omega } = 0$ boundary condition.   }

