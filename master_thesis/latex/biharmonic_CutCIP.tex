
\newpage
\section{CutCIP Biharmonic problem}%
\label{sec:cutcip_biharmonic_problem}

\subsection{Weak formulation}%
\label{sub:weak_formulation}


Let $\Omega \subseteq    \mathbb{R} ^d$ be a physical mesh with and $\Gamma  $ be a $C^2$ boundary. We define the strong biharmonic problem to be on the form

\begin{equation}
\begin{split}
    \Delta^2  u  + \alpha  u  & = f( x)  \quad \text{in } \Omega,   \\
    \partial _{n} u & = 0  \quad \text{on } \Gamma ,  \\
    \partial _{n} \Delta  u & = g(x)  \quad \text{on } \Gamma  .  \\
\end{split}
\end{equation}
We want to write the problem on the weak form.
Let $u \in H^{4}( \Omega ) $ be a solution of the strong problem and $v \in H^{2}( \Omega ) $ be a test function. We argue that this holds,
    \[
        \begin{split}
(\Delta ^2u,v )_{\Omega } & = ( \partial _{n} \Delta u, v)_{\Gamma } - ( \nabla ( \Delta u) , \nabla v) _{\Gamma } \\
&= ( D^2u, D^2v)_{\Omega } + ( \partial _{n} \Delta u ,v)_{\Gamma } - ( \partial _{nn} u, \partial _{n}v)_{\Gamma } - ( \partial _{tn} u, \partial _{t} v)_{\Gamma }.        \\
        \end{split}
    \]
    Define the space $V_{0} = \left\{ u \in H^{2}( \Omega ) :  \partial _{n}u  \mid _{\Gamma } = 0 \right \} $ and $V = H^2( \Omega ) $ .
    By applying the boundary conditions can we see that $( \partial _{n} \Delta u, v)_\Gamma   = ( g,v)_{\Gamma }$ which occurs naturally. However, the neumann boundary condition can be imposed by assuming $u \in V_{0}$.
    A consequence of this choice is that we force $ ( \partial _{nn} u, \partial _{n}v)_{\Gamma }
    = 0$ and $( \partial _{tn} u, \partial _{t}v)_{\Gamma } = 0$ since $\partial _{n} v = 0$ and that $\partial _{t} (\partial _{n}u) = 0 $. The latter is a consequence of that the Neumann boundary condition is homogeneous.
    \todo[inline]{ But what if the Neumann conditions is not homogeneous? And why is it necessary to include it in the function space when it comes in "naturally"?  }

    Finally, if we $u,v \in V_{0}$ then is it natural to define an general bilinear form $ a: V_{0} \times  V_{0} \to   \mathbb{R} $ as \[
    a( u,v) = \alpha ( u,v)_{\Omega } + ( D^2u, D^2v)_{\Omega }
    \]
    where $\alpha >0$ and a corresponding linear form $F: V \to \mathbb{R} $,
    \[
    F( v) = ( \partial _{n} \Delta u ,v)_{\Gamma }.
    \]

    We define the weak problem is to find a $u \in  V_{0}$ s.t. \[
    a( u,v) = F(v) \forall v \in V_{0}
    \]




\subsection{Initial discrete formulation}%
\label{sub:initial_discrete_formulation}



We want to make a cutFEM version of the CIP problem. That is, using the conventions of

\[
V_{h} = \left\{ v \in C^{0}\left( \Omega  \right): v_{T} = v | _{T} \in \mathcal{P} _{2}\left( T \right), \forall T \in
\mathcal{T}_{h}    \right\}
\]



