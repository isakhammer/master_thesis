
\newpage
\section{CutCIP Biharmonic problem}%
\label{sec:cutcip_biharmonic_problem}

\subsection{Weak formulation}%
\label{sub:weak_formulation}


Let $\Omega \subseteq    \mathbb{R} ^d$ be a physical mesh, $\Gamma  $ be a $C^2$ boundary and the constant $\alpha >0$. We define the strong biharmonic problem to be on the form

\begin{equation}
\begin{split}
    \Delta^2  u  + \alpha  u  & = f( x)  \quad \text{in } \Omega,   \\
    \partial _{n} u & = g_{1}(x)   \quad \text{on } \Gamma ,  \\
    \partial _{n} \Delta  u & = g_2(x)  \quad \text{on } \Gamma  .  \\
\end{split}
\end{equation}
The goal is to write the problem on a weak form.
Let $u \in H^{4}( \Omega ) $ be a solution of the strong problem and $v \in H^{2}( \Omega ) $ be a test function. We argue that this holds,
    \[
        \begin{split}
(\Delta ^2u,v )_{\Omega } & = ( \partial _{n} \Delta u, v)_{\Gamma } - ( \nabla ( \Delta u) , \nabla v) _{\Gamma } \\
&= ( D^2u, D^2v)_{\Omega } + ( \partial _{n} \Delta u ,v)_{\Gamma } - ( \partial _{nn} u, \partial _{n}v)_{\Gamma } - ( \partial _{tn} u, \partial _{t} v)_{\Gamma }.        \\
        \end{split}
    \]
    Define the space $V_{g} = \left\{ u \in H^{2}( \Omega ) :  \partial _{n}u  \mid _{\Gamma } = g_1(x )  \right \} $ and $V = H^2( \Omega ) $.
    % By applying the boundary conditions can we see that $( \partial _{n} \Delta u, v)_\Gamma   = ( g,v)_{\Gamma }$ which occurs naturally. However, the Neumann boundary condition can be imposed by assuming $u \in V_{0}$ and
    %  as a consequence is the following terms $ ( \partial _{nn} u, \partial _{n}v)_{\Gamma }
    % = 0$   and $( \partial _{tn} u, \partial _{t}v)_{\Gamma } = 0$. This follows from the fact that that the Neumann boundary condition is homogeneous s.t. $\partial _{n} v = 0$ and $\partial _{t} (\partial _{n}u) = 0 $.
    % \todo[inline]{ But what if the Neumann conditions is not homogeneous? And why is it necessary to include it in the function space when it comes in "naturally"?  }
    Let $u,v \in V$ then is it natural to define an general bilinear form  \[
    a( u,v) = ( D^2u, D^2v)_{\Omega } + ( \partial _{n} \Delta u ,v)_{\Gamma } - ( \partial _{nn} u, \partial _{n}v)_{\Gamma } - ( \partial _{tn} u, \partial _{t} v)_{\Gamma }.
    \]
    where $\alpha >0$ and a corresponding linear form $l: V \to \mathbb{R} $,
    \[
    l( v) = ( f ,v)_{\Gamma } +  ( h_{2},v)_{\Gamma } .
    \]
    We define the weak problem is to find a $u \in  V_{g}$ s.t. \[
    a( u,v) = l(v) \quad  \forall v \in V_{}
    \]

\subsection{Initial discrete formulation}%
\label{sub:initial_discrete_formulation}

We want to make a cutFEM version of the CIP problem. Let $\widetilde{\mathcal{T}_{h} } $ be a shape-regular and quasi-uniform background mesh. Let us denote the active set $\mathcal{T} _{h} \subseteq \widetilde{\mathcal{T}_{h}}$ which intersects the interior of the active domain $\Omega $, that is  \[
\mathcal{T} _{h} = \left\{ T \in \widetilde{\mathcal{T} _{h}}  \mid  T \cap (\Omega \setminus \Gamma ) \neq \emptyset    \right\} .
\]
With a corresponding set of interior facets, \[
    \mathcal{F} _{h} = \left\{ F = T^{+} \cap T^{-}  \mid  T^{+}, T^{-} \in \mathcal{T} _{h} \right\},
\]
and a set of cut elements \[
\mathcal{T} _{\Gamma } = \left\{ T \in \mathcal{T} _{h}   \mid  T \cap \Gamma \neq \Omega \right\}.
\]
We denote the $C^{0}$ polynomial space of order $k$ as
\[
V_{h} = \left\{ v \in C^{0}\left( \Omega  \right): v_{T} = v | _{T} \in \mathcal{P} ^{k}\left( T \right), \forall T \in
\mathcal{T}_{h}    \right\}
\]
From the previous chapter can we write the CIP method. The bilinear form $a_{h}:  V_{h}\times  V_{h} \to \mathbb{R} $ is defined as

\begin{equation}
\label{eq:Bi_a_h}
\begin{split}
a_{h} \left( u, v \right)   =&   \left( \alpha  u, v \right) _{\mathcal{T} _{h} \cap \Omega }   +  \left( D^2 u, D^2v \right) _{\mathcal{T} _{h} \cap \Omega} \\
 & +
  \left( \mean{  \partial _{n n} u }, \jump{ \partial _{n }v} \right)_{\mathcal{F}_{h}^{int} \cap \Omega}  +
 \left( \mean{ \partial _{n n} v }, \jump{ \partial _{n}u }      \right)_{\mathcal{F}_{h}^{int} \cap \Omega} \\
 & + ( \partial _{nn} u, \partial _{n} v)_{\Gamma } + ( \partial _{nn} v, \partial _{n} u)_{\Gamma }
 \\
 & + \frac{\gamma }{h}  \left( \jump{ \partial _{n} u}, \jump{ \partial _{n} v_{}   }   \right)_{\mathcal{F}_{h}^{int} \cap \Omega} +  \frac{\gamma }{h}  \left(  \partial _{n} u,  \partial _{n} v_{}      \right)_{\Gamma } \\
\end{split}
.
\end{equation}
Similarly the linear form is defined as
 \[
l_{h}( v) =  ( f,v)_{\mathcal{T} _{h} \cap \Omega } - ( g_{2}, v) _{\Gamma }  .
\]
To make sure the problem is stabilized will we add a ghost-penalty. That is, we define the discrete problem to find a $u_{h} \in V_{h}$ s.t. \[
A_{h}( u_{h} ,v ) := a_{h}( u_{h}, v)  + g_{h}( u_{h},v) = l_{h} ( v) \quad  \forall v \in  V_{h}.
\]
We define the underlying norms for $ v \in V_{h} $ as
    \begin{align}
        \label{eq:bi_ah_norm}
        \| v \|_{ a_{h} }^{ 2 } & =  \alpha  \| v \|_{ \mathcal{T} _{h} \cap \Omega  }^{ 2}  + \| D^2 v \|_{\mathcal{T} _{h} \cap \Omega   }^{ 2 } + \gamma \| h^{-\frac{1}{2}} \jump{ \partial _{n} v }   \|_{ \mathcal{F}_{h}^{int}\cap \Omega    }^{ 2
        } + \gamma \| h^{-\frac{1}{2}}  \partial _{n} v    \|_{ \Gamma   }^{ 2 },    \\
        \label{eq:bi_gh_norm}
\abs{ v } _{g_{h}}^{2} & = g( v,v) \\
        \label{eq:bi_Ah_norm}
\| v \|_{A_{h}  }^{  2}  & = \| v \|_{ a_{h} }^{ 2 } + \abs{ v } _{g_{h}}^{2}
    \end{align}
and for $v \in V + V_{h}$ we get, \[
\| v \|_{ a_{h}, * }^{  2} =\| v \|_{ a_{h} }^{ 2 } +\abs{ v } _{g_{h}}^{2} +  \| h^{\frac{1}{2}} \mean{ \partial _{nn} v }   \|_{\mathcal{F} _{h}^{int} \cap \Omega   }^{  2} +  \| h^{\frac{1}{2}} \partial _{nn} v    \|_{ \Gamma }^{  2} .
\]
\begin{remark}
Note that it holds that $\mathcal{T} _{h} \cap  \Omega   = \Omega  $ and $\mathcal{T} _{h} \cap  \Gamma  = \Gamma $. Depending on context, we choose the best suitable notation.
\end{remark}

\subsection{Stability estimate}%
\label{sub:stability_estimate}

Similarly for the Poisson problem will we have the following assumptions for the computational mesh;

\begin{enumerate}[label=\textbf{S.\arabic*}]
    \item\label{as:s1} Boundary $\Gamma $ is of $C^2$
    \item\label{as:s2} The mesh $\mathcal{T} _{h}$ is quasi-uniform.
    \item \label{as:s3}For a $T \in \mathcal{T} _{\Gamma }$ there exists a path $P$ of $diam(P) \lesssim h$ which contains $T$ and an element $T'$ with a so-called fat intersection $
    \abs{ T' \cap \Omega  } _{d} \ge \abs{ T' } _{d}$.
\end{enumerate}

From basic theory we have the following inverse estimate for $ v \in \mathcal{P}^{k}( T)$ s.t. \[
     \| \partial _{nn}  v \|_{F   }^{ }  \lesssim  \| h_{T}^{-\frac{1}{2}} D ^2 v \|_{ T }^{  },
\]
where the hidden constant depend on dimension $d$, order $k$ and the shape regularity. Similarly for cut elements is it easy to see that this must hold,
\begin{equation*}
     \| \partial _{nn}  v \|_{F \cap \Omega    }^{  }  \lesssim\| \partial _{nn}  v \|_{F }^{  }  \lesssim   \| h_{T}^{-\frac{1}{2}} D ^2 v \|_{ T }^{  }.
\end{equation*}
A useful variant is the following inequality that is,
\begin{equation*}
\| \partial _{nn} v \|_{ \Gamma \cap T  }^{  } \lesssim h^{-\frac{1}{2}} \| D^2 v \|_{ T }^{  }.
\end{equation*}
\todo[inline]{I Hope this one is correct?}
Summation the inverse inequalities over $\mathcal{F}_{h} $ and $\mathcal{T}_{h} $ implies that
\begin{align}
\label{eq:bi_cut_inverse_1}
\| \partial _{nn} v \|_{ \mathcal{T} _{h} \cap \Gamma  }^{  } &\lesssim h^{-\frac{1}{2}} \| D^2 v \|_{ \mathcal{T}_h }^{  }, \\
\label{eq:bi_cut_inverse_2}
\| \partial _{nn}  v \|_{ \mathcal{F}_h \cap \Omega    }^{  }  &  \lesssim   h^{-\frac{1}{2}} \| D^2 v \|_{ \mathcal{T}_h  }^{  }.
\end{align}
In fact, combining the inequalities we get the identity,
\begin{equation}
\label{eq:bi_identity}
h\| \partial _{nn}  v \|_{ \mathcal{F}_h \cap \Omega    }^{2 } + h\| \partial _{nn} v \|_{ \mathcal{T} _{h} \cap \Gamma  }^{2  } \lesssim \| D^2 v \|_{ \mathcal{T} _{h}  }^{2  }.
\end{equation}

We may introduce our first assumption on the ghost penalty.
\begin{assumption}[EP1]
    \label{as:bi_EP1}
    The ghost penalty $g_{h}$ extends the $H^{1}$ norm s.t. \[
    \| D^2 v \|_{ \mathcal{T} _{h} }^{ 2 } \lesssim  \| D^2 v \|_{ \Omega  }^{ 2 } + \abs{ v } _{g_{h}}^{2}.
    \]
\end{assumption}


Combing the results we get the following convenient corollary.

\begin{corollary}
    \label{cor:bi_inverse_thm}
    Let $g_{h}$ satisfy Assumption \ref{as:bi_EP1} then
    \[
        \begin{split}
            h\| \partial _{nn}  v \|_{ \mathcal{F}_h \cap \Omega    }^{2 } + h\| \partial _{nn} v \|_{ \mathcal{T} _{h} \cap \Gamma  }^{2  } &  \lesssim  \| D^2 v \|_{ \Omega  }^{ 2 } + \abs{ v } _{g_{h}}^{2} \\
            &  \lesssim \| v \|_{ A_{h} }^{  2}
        \end{split}
    \]
\end{corollary}
\begin{proof}
    The first inequality is a direct result of \eqref{eq:bi_identity} and Assumption \ref{as:bi_EP1}. The second inequality is simply a results of the definition \eqref{eq:bi_Ah_norm}.
\end{proof}


\begin{lemma}
    The discrete form $A_{h}$ is coercive, that is, \[
    \| v \|_{ A_{h} }^{ 2 }  \lesssim A_{h}( v,v) \forall v \in V_{h}
    \]
\end{lemma}

\begin{proof}
    \todo[inline]{ TODO }
\end{proof}

\begin{lemma}
    The discrete form $A_{h}$ is bounded, that is,
    \begin{equation}
    \label{eq:bi_A_h_bounded}
     A_{h}( v,w) \lesssim \| v \|_{A_{h}  }^{  }\| v \|_{A_{h}  }^{  }  \forall w \in V_{h}
    \end{equation}
    Moreover, for $v \in V_{h} + V$  and $w \in V_{h}$ the discrete form $a_{h}$ satisfies
    \begin{equation}
        \label{eq:bi_a_h_bounded}
        a_{h} ( v,w) \lesssim \| v \|_{ a_{h},* }^{  } \| w \|_{ A_{h} }^{  }
    \end{equation}
\end{lemma}

\begin{proof}
    We will divide the proof in two steps.
    \begin{enumerate}[label=\arabic*)]
        \item The goal is to prove the inequality \eqref{eq:bi_A_h_bounded}. \[
                \abs{ A_{h}( v ,w ) } \lesssim   \abs{a_{h}( v, w) }   + \abs{g_{h}( v,w)  }          \]
                By assumption is the ghost penalty $g_{h}$ positive semi-definite, thus, it fulfills the Cauchy-Schwartz inequality \[
                \abs{ g_{h}(v,w ) } \lesssim \abs{ v } _{g_{h}}\abs{ w }_{g_{h}}
                \]
                Hence, by definition is $\abs{ g_{h}(v,w ) } \lesssim \| v \|_{ A_{h} }^{  } \| w \|_{ A_{h} }^{  } $. Now it remains to show that the bilinear term $ a_{h}$ is bounded.
                \begin{equation}
                    \begin{split}
                        \abs{ a_{h} \left( v, w \right) }   \le  &   \abs{\left( \alpha  v, w \right) _{\mathcal{T} _{h} \cap \Omega }  }    +  \abs{\left( D^2 v, D^2w \right) _{\mathcal{T} _{h} \cap \Omega}  }  \\
                                                     & + \abs{\left( \mean{  \partial _{n n} v }, \jump{ \partial _{n }w} \right)_{\mathcal{F}_{h} \cap \Omega}  }   +
                                                     \abs{\left( \mean{ \partial _{n n} v }, \jump{ \partial _{n}w }      \right)_{\mathcal{F}_{h} \cap \Omega}  } \\
                                                     & + \abs{\left(  \partial _{n n} v ,  \partial _{n }w \right)_{\Gamma }}     +
                                                     \abs{\left(  \partial _{n n} v ,  \partial _{n}w       \right)_{\Gamma }  }
                                                     \\
                                                     & + \frac{\gamma }{h} \abs{ \left( \jump{ \partial _{n} v}, \jump{ \partial _{n} w   }   \right)_{\mathcal{F}_{h} \cap \Omega}  }  \\
                    \end{split}
                \end{equation}
                The strategy is to bound each term individually using Cauchy-Schwartz. We can easily see that $\abs{\left( \alpha  v, w \right) _{\mathcal{T} _{h} \cap \Omega }  }   \lesssim \| v \|_{a_{h}  }^{  } \| w \|_{ a_{h} }^{  } $ and that
                $\abs{\left( D^2 v, D^2w \right) _{\mathcal{T} _{h} \cap \Omega}  } \lesssim \| v \|_{a_{h}  }^{  } \| w \|_{ a_{h} }^{  } $ using Cauchy Schwartz. For the symmetric terms we also apply the inverse inequality
                \eqref{eq:bi_cut_inverse_2}.
                \[
                    \begin{split}
                    \abs{\left( \mean{ \partial _{n n} v }, \jump{ \partial _{n}w }      \right)_{\mathcal{F}_{h} \cap \Omega}  } & \lesssim  \|\mean{ \partial _{n n} v }  \|_{ \mathcal{F}_{h} \cap \Omega}^{  }\|\jump{ \partial _{n} w }  \|_{
                    \mathcal{F}_{h} \cap \Omega}^{  } \\
                    & \lesssim  \|h^{\frac{1}{2}} \partial _{n n} v  \|_{ \mathcal{F}_{h} \cap \Omega}^{  }\| h^{-\frac{1}{2}} \jump{ \partial _{n} w }     \|_{\mathcal{F}_{h} \cap \Omega}^{  } \\
                    & \lesssim  \| v \|_{A_{h}  }^{  } \|w    \|_{ a_{h}}^{  }
                    \end{split}
                \]
                Here we used the Corollary \ref{cor:bi_inverse_thm} s.t.  $\|h^{\frac{1}{2}} \partial _{n n}  v \|_{\mathcal{F}_{h} \cap \Omega} \lesssim \| v \|_{ A_{h}  }^{  }  $.
              Finally, the penalty term \[
             \frac{\gamma }{h} \abs{ \left( \jump{ \partial _{n} v}, \jump{ \partial _{n} w   }   \right)_{\mathcal{F}_{h} \cap \Omega}  }  \lesssim  \|h^{-\frac{1}{2}} \jump{ \partial _{n} v}  \|_{ \mathcal{F}_{h} \cap \Omega }^{  }
             \|h^{-\frac{1}{2}} \jump{ \partial _{n} w}  \|_{ \mathcal{F}_{h} \cap \Omega }^{  }  \lesssim  \| v  \|_{ a_{h} }^{  }
             \| w  \|_{ a_{h} }^{  }
             \]
             Obviously is $\| v \|_{a_{h}  }^{  } \lesssim \| v \|_{A_{h}  }^{  }$. Now it remains to handle boundary terms. \[
                 \begin{split}
                \abs{ ( \partial _{n} v, \partial _{n} w)_{\Gamma } } & \lesssim \| h^{\frac{1}{2}}\partial _{n} v \|_{\Gamma   }^{  } \| h^{-\frac{1}{2}} \partial _{n}w \|_{\Gamma   }^{  } \\
                & \lesssim \| h^{\frac{1}{2}}\partial _{n} v \|_{\Gamma   }^{  } \| w \|_{ A_{h}   }^{  }
                 \end{split}
             \]


             Hence, $a_{h}$ is bounded in the $\|\cdot   \|_{A_{h}  }^{  } $ norm.
    \end{enumerate}
    \todo[inline]{ Finish proof. }
\end{proof}

