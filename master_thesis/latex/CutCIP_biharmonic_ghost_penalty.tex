
\newpage
\subsection{Constructing Ghost Penalties}%
\label{sec:constructing_ghost_penalties}

We have the following assumptions for the ghost penalty.
\begin{enumerate}[label=\textbf{EP\arabic*}]
    \item \label{as:EP1} The ghost penalty $g_{h}$ extends the $H^{1}$ norm such that
        \begin{equation}
    \| D^2v_{h} \|_{ \mathcal{T} _{h} }^{ 2 }  \lesssim \| D^2 v_{h} \|_{ \Omega  }^{  2} + \abs{ v_{h} } _{g_{h}}^2 \quad \forall v_{h} \in V_{h}
        \end{equation}
\item \label{as:EP2} For $v \in H^{s}( \Omega ) $ and $r = \min \{s,k+2 \} $, the semi-norm $\abs{ \cdot  }_{g_{h}} $ satisfies the following estimate, \[
    \abs{ \pi _{h}^{e} v } _{g_{h}} \lesssim  h^{r-2} \| v \|_{ r,\Omega  }^{  }.
    \]
\end{enumerate}
The goal in this chapter is to engineer an ghost penalty which fulfills these assumptions.

Let us denote the generalization of the normal derivative,
\begin{equation}
\partial _{n}^{j} v = \sum_{\abs{ \alpha  } =j }^{k} \frac{D ^{\alpha }v( x) n^{\alpha }}{\alpha !}, \quad \abs{ \alpha  } = \sum_{i=0}^{d} \alpha_{i}.
\end{equation}
We denote the multi-index $\alpha  = ( \alpha _{1}, \ldots, \alpha _{d})  $ of order $\abs{ \alpha  } = \sum_{i}^{}  \alpha _{i} = k $   and the normal vectors $n^{\alpha } = n_{1}^{\alpha _{1}} \ldots n_{d}^{\alpha _{d}}$.
Recall the notation for the derivates $D^{\alpha } v$
\footnote{
Remark that the operator $D^{\alpha }$  is related to with the derivate operator $\partial ^{\alpha } $ introduced in \eqref{eq:der}. For instance, for $d=2$ we have $\alpha = ( \alpha _{1}, \alpha _{2}) $  such that $ D^{1} v =  \nabla v  = \left[ \partial ^{( 1,0 )} v,\partial ^{( 0,1 )}v
\right]^{T}$ and $   D^2 v  = \begin{bmatrix}
\partial ^{( 2,0 )} v &  \partial ^{( 1,1) }v \\
\partial ^{( 1,1 )} v &  \partial ^{( 0,2) }v
\end{bmatrix}
 $.
},
that is
\begin{equation}
D ^{0} v  = v, \quad   D ^{1}v  = \nabla v \text{ and }  D ^{2} v  = J(\nabla v) = \mathrm{Hess}(v).
\end{equation}
where $J$ is the Jacobian operator .

The following result is the backbone of the face-based ghost penalty.
\begin{lemma}
    Let $T_{1},T_{2 } \in  \mathcal{T} _{h}$ be two elements sharing a common face $F$. Then for $v_{h} \in V_{h}$ with polynomial degree $k$  we have
    \begin{equation}
    \| v_{h} \|_{ T_{1} }^{  }  \lesssim \| v_{h} \|_{ T_{2} }^{  } + \sum_{0\le j\le k}  {h^{2j +1}}^{} ( \jump{ \partial _{n}^{j} v_{h} }, \jump{ \partial ^{j}_{n} v_{h} }    )_{F}
    \end{equation}

\end{lemma}
\begin{proof}
    See \cite[Lemma 2.19]{gurkan2019stabilized}.
\end{proof}

An useful result that may help us design ghost penalty is the following estimate.

\begin{lemma}
    \label{lemma:bi_local_facet_estimate}
    Let $T_{1}, T_{2} \in  \mathcal{T} _{h} $ share a common facet $F \in \mathcal{F}_{h} $ and let $v_{} \in  V_{h}$ with polynomial degree $k$. Then does this hold,
    \begin{equation}
    \| v_{h} \|_{ T_{1} }^{  2}  \lesssim  \| v_{h} \|_{ T_{2} }^{2  }  + \sum_{j=0}^{k}  h^{2j +1} ( \jump{ \partial _{n}^{j} v_{h}}, \jump{ \partial _{n}^{j} v_{h}}    )_{F}.
    \end{equation}

\end{lemma}

\begin{proof}
    For a detailed proof, see \cite{gurkan2019stabilized}.
\end{proof}

We will now introduce the so-called ghost penalty faces, that is, \[
\mathcal{F} ^{g}_{h} = \left\{ F\in \mathcal{F} _{h} : T^{+}\cap \Gamma \neq \emptyset  \vee T^{-}\cap \Gamma \neq \emptyset  \right\}.
\]
This set is simply all facets that belong to all elements of the active mesh $\mathcal{T} _{h}$  intersected with $\Gamma $, i.e., all triangles to the cut cells $\mathcal{T} _{\Gamma }$. For an illustration, see Figure \ref{fig:illustration_F_g}.




\begin{proposition}
    \label{prop:hessian_change}
    The following identity holds for $j=0,1,2$.
    $$\partial ^{j}_{n} (D^2v) = D^2 ( \partial ^{j}_{n} v)  $$
\end{proposition}

\begin{proof}
        Recall that $\left[ D^2 v \right]_{i,j} = \partial _{x_{i}x_{j}} v $. We will compute each index individually.
    \begin{enumerate}[label=\arabic*)]
        \item $j = 0$ is trivially true.
        \item Let $j=1$. We can then easily see that, \[
        \partial ^{}_{n} ( \partial _{x_{i} x_{j}} v)  = \nabla  ( \partial _{x_{i} x_{j}} v)   n = \partial _{x_{i} x_{j}} (\nabla  v) n =\partial _{x_{i} x_{j}} (\partial _{n} v)
        \]
        Hence, $\partial _{n} (D^2v) = D^2( \partial _{n}v)$.
        \item Let $j=2$. Similarly can we see that, \[
                \begin{split}
                \partial^{2} _{n} (\partial _{x_{i} x_{j}} v) & = n^{T}  D^2(\partial _{x_{i} x_{j}} v) n = n^{T}  J( \nabla (\partial _{x_{i} x_{j}}v) ) n \\
                & =  n^{T}  J(\partial _{x_{i} x_{j}}(\nabla v) ) n = \partial _{x_{i} x_{j}} n^{T} J(\nabla v) n
                \end{split}
            \]
            Thus, taking account for all elements in the matrix we get $\partial^{2} _{n} (D^2v) = D^2( \partial^{2} _{n}v)$.
    \end{enumerate}
    Proof is complete.

\end{proof}


\begin{figure}
    \centering
    \begin{tikzpicture}
        \coordinate (center) at (0, 0);

        % Reference hexagon vertices
        \coordinate (A1) at (0:2.5);
        \coordinate (A2) at (60:2.5);
        \coordinate (A3) at (120:2.5);
        \coordinate (A4) at (180:2.5);
        \coordinate (A5) at (240:2.5);
        \coordinate (A6) at (300:2.5);
        \coordinate (A7) at (300:2.5);


        \fill[green!40] (center) --(A1) -- (A2) -- (A3) -- (A4)  -- cycle;
        \fill[blue!30] (center) -- (A4) --(A5)--(A6)-- (A7) -- (A1) -- cycle;

        % Draw the individual edges
        \draw[dotted, line width=1.5pt] (center) -- (A1);
        \draw[dotted, line width=1.5pt] (center) -- (A2);
        \draw[dotted, line width=1.5pt] (center) -- (A3) -- (A4) -- (center);
        \draw (A2) -- (A3);
        \draw (A4) -- (A5);
        \draw (center) -- (A5) -- (A6) ;
        \draw (center) -- (A6) -- (A1);


        \begin{scope}[shift={(2.5,0)}]
            % Shifted
            \coordinate (center) at (0, 0);

            % Reference hexagon vertices
            \coordinate (B1) at (0:2.5);
            \coordinate (B2) at (60:2.5);
            \coordinate (B3) at (120:2.5);
            \coordinate (B5) at (240:2.5);
            \coordinate (B6) at (300:2.5);

            \fill[green!40] (center) --(B6) -- (B1) -- (B2) --(B3) -- cycle;
            \fill[blue!30] (center) -- (B5) --(B6) -- cycle;


            % Draw the individual edges
            \draw[dotted, line width=1.5pt] (center) -- (B2);
            \draw[dotted, line width=1.5pt] (center) -- (B3); % double
            \draw[dotted, line width=1.5pt] (center) -- (B1);
            \draw (B1) -- (B2);
            \draw (B2) -- (B3);
            \draw (B5) -- (B6);
            \draw[dotted, line width=1.5pt] (center) -- (B6) -- (B1);
        \end{scope}


        \coordinate (Ti) at (-0,-1.0);
        \coordinate (Tg) at (3.0,2.0);
        \node[below] at (Tg) {$\mathcal{T}_{\Gamma }$};
        \node[below] at (Ti) {$\mathcal{T}_{int }$};

        \coordinate (C1) at (-3,0);
        \coordinate (C2) at (5.2,-0.5);
        \draw[-, line width=2pt, >=stealth] ($(C2)$) to[bend right] node[midway, yshift=-0.3cm] {$\Gamma $} ($(C1)$);

        \begin{scope}[shift={(6.0,1.7)}]
            \draw[dotted, line width=1.5pt] (-1.5,0) -- (-1,0);
            \node[anchor=west] at (-1.0,0) {$\mathcal{F}_h^g$};
        \end{scope}

        % Symbol visualization
        % \draw[dotted, line width=1.5pt] (3.4,3.1) -- (3.9,3.1);
        % \node[anchor=west] at (4.0,3.1) {$\mathcal{F}_h^g$};
        %% Legend
        \draw (4.4,1.3) rectangle (5.8,2.1); % Legend box

\end{tikzpicture}

\caption{Illustration of $\mathcal{F} _{h}^{g}$ denoted as the dotted lines. The set is defined as all facets which belongs to cut cells $\mathcal{T} _{\Gamma }$ sharing a node with interior elements $\mathcal{T} _{int }$ .  }
\label{fig:illustration_F_g}
\end{figure}





\begin{lemma}
    \label{lemma:bi_inv_gh_lemma}
    For $v_{h} \in  V_{h}$ of polynomial degree $k$.
        \begin{align}
            \label{eq:bi_inv_gh_1}
        \| v_h \|_{ \mathcal{T} _{h} }^{ 2 }  & \lesssim  \| v_h \|_{ \Omega  }^{ 2 }  + \sum_{j=0}^{k} h^{2j+1} ( \jump{ \partial ^{j}_{n} v }, \jump{ \partial ^{j}_{n} v_h}    )_{\mathcal{F}_{h}^{g}}\\
            \label{eq:bi_inv_gh_2}
        \| D ^2 v_h \|_{ \mathcal{T} _{h} }^{ 2 }  & \lesssim  \| D^2 v_h \|_{ \Omega  }^{ 2 }  + \sum_{j=0}^{k} h^{2j-3} ( \jump{ \partial ^{j}_{n} v_h }, \jump{ \partial ^{j}_{n}v_h }    )_{\mathcal{F}_{h}^{g}}
        \end{align}
\end{lemma}

\begin{proof}
    We will dive the proof in two parts where we first prove \eqref{eq:bi_inv_gh_1} and then in the second part prove \eqref{eq:bi_inv_gh_2}.
    \begin{enumerate}[label=\arabic*)]
        \item
            First of all, notice that there is a patch $P(T) $ consisting of $\left\{ T_{i} \right\}_{i=1}^{l} $ mesh elements s.t. each pair $ \left\{ T_{i}, T_{i+1} \right\} $ share a facet $F_{i}$ and the last element $T_{l}$ has a so-called "fat"
            intersection.

            Let us define the following norm \[
            g_{F_{i}}^{L^{2}}( v_{h},v_{h})  = \sum_{j=0}^{k} h^{2j+1}( \jump{ \partial ^{j}_{n}v_{h} }, \jump{ \partial ^{j}_{n}v_{h} }    )_{F_{i} }
            \]
            where $F_{i} \in  \mathcal{F} ^{g}_{h}$ and polynomial degree $ k$. Using Lemma \ref{lemma:bi_local_facet_estimate} can we see that \[
            \| v_{h} \|_{ T_{i} }^{  } \lesssim \| v_{h} \|_{ T_{i+1} }^{ 2 } + g_{F_{i}}^{L^{2}}( v,v).
            \]
    Consequently, using induction over each pair $\left\{ T_{i}, T_{i+1} \right\} $ with a corresponding $F_{i}$, we obtain
            \[
                \begin{split}
            \| v_{h} \|_{ T_{1} }^{2  }  & \le  C \| v_{h} \|_{ T_{2} }^{ 2 } + g_{F_{1}}^{L^{2}}( v_{h},v_{h})\\
              & \le  C( C( \| v_{h} \|_{ T_{3} }^{ 2 } + g_{F_{2}}^{L^{2}}( v_{h},v_{h}) ) + g_{F_{1}}^{L^{2}}( v_{h},v_{h}) )\\
              & \lesssim    \| v_{h} \|_{ T_{l} }^{ 2 }  + \sum_{i=1}^{l-1} g_{F_{i}}^{L^{2}}( v_{h},v_{h})  \\
              & \lesssim    \| v_{h} \|_{ T_{l} \cap \Omega  }^{ 2 }  + \sum_{i=1}^{l-1} g_{F_{i}}^{L^{2}}( v_{h},v_{h})
                \end{split}
            \]
            Here the last steps arise from the fact that $\|  v_{h} \|_{ T_{l} }^{  } \lesssim  \|  v_{h} \|_{ T_{l} \cap \Omega  }^{  }  $, which is a consequence of the fat intersection property.
Summation over the intersected triangles $\mathcal{T} _{\Gamma }$ implies,
            \[
                    \| v_{h} \|_{ \mathcal{T} _{\Gamma } }^{2  } \lesssim \| v_{h} \|_{ \mathcal{T} _{\Gamma}\cap \Omega  }^{2  }+ \sum_{i=1}^{l-1} g_{F_{l}}^{L^{2}}( v_{h},v_{h}) \\
                     = \| v_{h} \|_{ \mathcal{T}_{\Gamma } \cap \Omega   }^{ 2 }  + \sum_{j=0}^{k} h^{2j+1} ( \jump{ \partial ^{j}_{n} v_{h} }, \jump{ \partial ^{j}_{n}v_{h} }    )_{\mathcal{F}_{h}^{g}}
        \]
        And as a trivial extension this now also holds for the active mesh $\mathcal{T} _{h}$ , that is, \[
                    \| v_{h} \|_{ \mathcal{T} _{h } }^{2  } \lesssim  \| v_{h} \|_{ \mathcal{T}_{h } \cap \Omega   }^{ 2 }  + \sum_{j=0}^{k} h^{2j+1} ( \jump{ \partial ^{j}_{n} v_{h} }, \jump{ \partial ^{j}_{n}v_{h} }    )_{\mathcal{F}_{h}^{g}}.
        \]
        Hence, \eqref{eq:bi_inv_gh_1} holds and the first part of the proof is complete.

    \item We will simply start by replacing $v$  by $D^2 v$ and use the Proposition \ref{prop:hessian_change}.
        \[
            \begin{split}
                    \| D^2 v_{h} \|_{ \mathcal{T} _{h} }^{2  }&  \lesssim \| D^2 v_{h} \|_{ \Omega  }^{ 2 }  + \sum_{j=0}^{k} h^{2j+1} ( \jump{   \partial ^{j}_{n} D^2 v_{h} }, \jump{  \partial ^{j}_{n} D^2 v_{h}}    )_{\mathcal{F}_{h}^{g}} \\
                    &=  \| D^2 v_{h} \|_{ \Omega  }^{ 2 }  + \sum_{j=0}^{k} h^{2j+1} ( \jump{   D^2 \partial ^{j}_{n}  v_{h} }, \jump{  D^2 \partial ^{j}_{n}  v_{h}}    )_{\mathcal{F}_{h}^{g}}
            \end{split}
        \]
        Remark that $\|  D^2 v_{h} \|_{ T_{l} }^{  } \lesssim  \|  D^2 v_{h} \|_{ T_{l} \cap \Omega  }^{  }  $ using the fat-intersection property.
        Thus, it remains to show that \[
        \sum_{j=0}^{k} h^{2j+1} ( \jump{   D^2 \partial ^{j}_{n}  v_{h} }, \jump{  D^2 \partial ^{j}_{n}  v_{h}}    )_{\mathcal{F}_{h}^{g}} \lesssim  \sum_{j=0}^{k} h^{2j-3} ( \jump{    \partial ^{j}_{n}  v_{h} }, \jump{  \partial ^{j}_{n}  v_{h}}
        )_{\mathcal{F}_{h}^{g}}.
        \]
        Recall the decomposition procedure done in \eqref{eq:projection} for a facet $F$ , where $P_{F} := I - n_{} \oplus n_{} $ and $Q_{F} = n_{} \oplus n_{}$. We apply this so we can to decompose the Hessian evaluated on the facet  such that, \[
        D^2 v_{h}  \mid _{F} = Q_{F}D^2v_{h} + P_{F} D^2 v_{h} .
        \]
        % Recall from basic finite element theory .
        \red{
        Recall from \eqref{eq:fund_inv_est} and \eqref{eq:degrade} the inverse estimates $\| D^2v_h \|_{T  }^{ 2 }  \le h^{-4} \| v_h \|_{ T }^{2  }    $ and $ \| D^2v_h \|_{F  }^{2  } \le h^{-1} \| D^2 v_h  \|_{T  }^{ 2 }   $.}
        That is, applying the decomposition we get the following estimates \[
            \begin{split}
        \| \jump{ P_{F}   D^2 \partial _{n}^{j} v_{h} }\|_{ F }^{ 2 } & = \| P_{F} D^2 \jump{ \partial _{n}^{j} v_{h} }   \|_{ F  }^{ 2} = \| (I- n\otimes n) D^2 \jump{ \partial _{n}^{j} v_{h} }   \|_{ F  }^{ 2}    \\
        & \lesssim  \|  (I- n\otimes n) \|_{l^{\infty} }^{  } \| D^2 \jump{ \partial _{n}^{j} v_{h} }   \|_{ F  }^{ 2} \lesssim   \|  D^2 \jump{ \partial _{n}^{j} v_{h} }   \|_{ F  }^{ 2} \lesssim h^{-4} \|  \jump{ \partial ^{j}_{n} v_{h} }   \|_{ F }^{  2}. \\
        \| \jump{ Q_{F}   D^2 \partial _{n}^{j} v_{h} }\|_{ F }^{ 2 } & = \| Q_{F} D^2 \jump{ \partial _{n}^{j} v_{h} }   \|_{ F  }^{ 2} = \| n\otimes n D^2 \jump{ \partial _{n}^{j} v_{h} }   \|_{ F  }^{ 2}    \\
        & \lesssim  \| n \otimes n \|_{l^{\infty} }^{  }  \|  D^2 \jump{ \partial _{n}^{j} v_{h} }   \|_{ F  }^{ 2} \lesssim h^{-4} \|  \jump{ \partial ^{j}_{n} v_{h} }   \|_{ F }^{  2}.
            \end{split}
        \]
        Here is $\| \cdot  \|_{l^{\infty}  }^{  } $ denoted as the discrete matrix inf-norm.
        Thus, using this on the decomposition can we admit that
         \[
             \begin{split}
            h^{2j +1} \| \jump{D^2 \partial ^{j}_{n}  v_{h}}  \|_{ F }^{2  } & \lesssim h^{2j +1} \| Q_{F} \jump{ \partial ^{j}_{n}  v_{h} }  \|_{ F }^{2  } + h^{2j +1} \| P_{F} D^2 \jump{ \partial ^{j}_{n}  v_{h} }       \|_{ F }^{2  } \\
            &  \lesssim   h^{2j -3}  \| \jump{ \partial ^{j}_{n}v_{h} }   \|_{ F }^{ 2 }
             \end{split}
        \]
        % \red{Seems like the projection argument is not necessarry. You can simply prove it using direcetly the estimate $\| D^2v \|_{F  }^{ 2 }\le h^{-2} \| \nabla v \|_{ F }^{ 2 } \le h^{-4} \| v \|_{ F }^{2  }    $ s.t.
        % $$
        % \| D^2 \jump{ \partial _{n}^{j} v } \|_{F  }^{ 2 } \le h^{-4} \| \jump{ \partial _{n}^{j} v } \|_{ F }^{2  }
    % $$ }

        Thus, fulfilling \eqref{eq:bi_inv_gh_2}.
        \[
            \begin{split}
         \sum_{j=0}^{k} h^{2j+1} ( \jump{   D^2 \partial ^{j}_{n}  v_{h} }, \jump{  D^2 \partial ^{j}_{n}  v_{h}}    )_{\mathcal{F}_{h}^{g}} & \lesssim  \sum_{j=0}^{k} h^{2j-3}  (\jump{    \partial ^{j}_{n}  v_{h} }, \jump{ \partial ^{j}_{n}  v_{h} }  )_{\mathcal{F} _{h}^{g}  }^{2  }
            \end{split}
        \]

    \end{enumerate}
    Hence, the proof is complete.

\end{proof}


Finally, we now have the tools we need to construct an candidate for the ghost penalty for which satisfies all assumptions.

\begin{proposition}[Face-based ghost penalty]
    Let $k\ge  2$ be the order of the polynomial basis in $V_{h}$ .
    For any set of positive parameters $\left\{ \gamma _{j} \right\} _{j=0}^{k}$, the ghost penalty defined as
    \begin{equation}
        \label{eq:ghost_penalty}
    g^{}_{h}( v_{h},w_{h})  := \sum_{j=1}^{k} \sum_{F \in \mathcal{F} _{h}^{g}}^{} \gamma _{j} h^{2j-3}_{F} ( \jump{ \partial ^{j}_{n} v_{h} }, \jump{ \partial ^{j}_{n} w_{h} }  ) _{F} \text{ for any } v_{h},w_{h} \in V_{h},
    \end{equation}
    satisfies the Assumption \ref{as:EP1} and \ref{as:EP2}.
\end{proposition}

\begin{proof}
    First of all, from Lemma \ref{lemma:bi_inv_gh_lemma} is it clear that \(
    \| D^2v_{h} \|_{ \mathcal{T} _{h} }^{  }  \lesssim \| D^2 v_{h} \|_{ \Omega   }^{  } + \abs{ v_{h} } _{g_{h}}.
    \)
    Thus, the Assumption \ref{as:EP1} holds and it only remains to check Assumption \ref{as:EP2}, that is, $ \abs{ C^{e}_{h} v }_{g_{h}} \lesssim h^{r-2} \| v \|_{r, \Omega   }^{  }$.
    Let $v  \in H^{s}( \Omega ) $, $s\ge 3$,  and $r = \min\{s,k+1\} $
    From definition is \[
        \begin{split}
        \abs{ C^{e}_{h} v }_{g_{h}}^{2} & = \sum_{j=1}^{k}  \gamma _{j} h^{2j-3} ( \jump{ \partial ^{j}_{n} v }, \jump{ \partial ^{j}_{n} v }  ) _{\mathcal{F}_{h}^{g} } \\
& = \sum_{j=1}^{r-1}  \gamma _{j} h^{2j-3} ( \jump{ \partial ^{j}_{n} v }, \jump{ \partial ^{j}_{n} v }  ) _{\mathcal{F}_{h}^{g} } + \sum_{j=r}^{k}  \gamma _{j} h^{2j-3} ( \jump{ \partial ^{j}_{n} v }, \jump{ \partial ^{j}_{n} v }  ) _{\mathcal{F}_{h}^{g} } \\
&= \sum_{j=r}^{k}  \gamma _{j} h^{2j-3} ( \jump{ \partial ^{j}_{n} v }, \jump{ \partial ^{j}_{n} v }  ) _{\mathcal{F}_{h}^{g} } \\
        \end{split}
    \]
    \red{TODO: Finish this one.}

    where the jump vanishes for the for the first $r-1$ terms, that is,  $\jump{ \partial ^{j}_{n} v } = 0  \  \forall s \le r-1$.





\end{proof}
