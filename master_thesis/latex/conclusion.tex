
\newpage
\section{Conclusion}%
\label{sec:conclusion}

We first presented the construction on the properties of the continuous interior penalty methods for the biharmonic equation with Cahn-Hilliard boundary conditions based on \cite{feng2007fully, brenner2012}.Following the cut finite element framework developed in
\cite{gurkan2019stabilized}, we presented a stable extension the discrete formulation to be adopted to a unfitted mesh. We denote the method as the so-called cut continuous interior penalty method which inherits the face-based ghost penalty and
provide theoretical proof for the stability and a priori analysis with corresponding numerical experiments. Finally, we then demonstrated that the cut continuous interior penalty methods can successfully be applied to the nonlinear Cahn-Hilliard
equation.
