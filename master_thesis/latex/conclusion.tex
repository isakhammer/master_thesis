
\newpage
\section{Conclusion}%
\label{sec:conclusion}

We first presented the construction on the properties of the continuous interior penalty methods for the biharmonic equation with Cahn-Hilliard boundary conditions based on \cite{feng2007fully, brenner2012}.
Next, we constructed a stabilized cut finite element method for an unfitted mesh based the framework established in \cite{gurkan2019stabilized}.
This method involves the use of a so-called ghost penalty, which ensures stabilization by integrating
outside of the physical domain.
Theoretical evidence was provided to validate its stability, and an a priori analysis was conducted alongside corresponding numerical experiments.
Finally, we then demonstrated that the cut continuous interior penalty methods can successfully be applied to the nonlinear Cahn-Hilliard equation.
