
\newpage
\section{Biharmonic Mixed Formulation}%
\label{sec:biharmonic_mixed_formulation}


\subsection{Introduction}%
\label{sub:introduction}

Recall the biharmonic problem formulation to find a $u \in H^4( \Omega ) $  \[
    \begin{split}
\Delta ^2 u & = f \quad  \text{in } \Omega \\
\partial _{n} u & = h_{1} \quad \text{ on } \Gamma  \\
\partial _{n}( \Delta u)  & = h_{2} \quad \text{ on } \Gamma
    \end{split}
\]
For $f: \Omega  \to  \mathbb{R} $ and $h_{1},h_{2}: \Omega  \to \mathbb{R}  $. It is easy to see
that the formulation can be rewritten in an equivalent mixed formulation , that is, to find $\sigma, \tau  \in H^2( \Omega ) $ s.t. \[
    \begin{split}
\Delta \sigma  & = f \quad  \text{in } \Omega \\
\sigma   & = \Delta u  \text{ in } \Omega \\
\partial _{n} \sigma  & = h_{1} \text{ on } \Gamma  \\
\partial _{n} u   & = h_{2} \text{ on } \Gamma
    \end{split}
\]
\todo[inline]{ Add a inkscape-picture illustrating $\Omega $ and $\Gamma $. }
In contrast to the CIP formulation, we will see that it is now easier to handle the constraints.
The goal is to obtain an useful weak formulation. Using Greens theorem on the first equation we get,
\[
( \sigma, v)_{\Omega } = ( \nabla  u , \nabla v  )_{\Omega } - ( \nabla _{n} u , v) _{\Gamma }.
\]
Similarly for the second equation we obtain
\[
( \nabla \sigma , \nabla \varphi  )_{\Omega} - ( \nabla _{n} \sigma ,  \varphi )_{\Gamma } = ( f,\varphi ) _{\Omega}
\]
Putting it all together we have the following mixed weak formulation; Find $( u, \sigma ) \in H^{1}( \Omega ) \times H^{1}( \Omega )  $ s.t. \[
    \begin{split}
     ( \nabla  u , \nabla v  )_{\Omega } -( \sigma, v)_{\Omega }  & =   ( h_{1} , v) _{\Gamma } \quad  \forall v \in H^{1}( \Omega ) \\
( \nabla \sigma , \nabla \varphi  )_{\Omega}  & = ( f,\varphi ) _{\Omega} + ( h_{2} ,  \varphi )_{\Gamma } \quad  \forall \varphi \in H^{1}( \Omega )
    \end{split}
\]
Now we want to relate this formulation to the abstract saddle point problem (SPP) (find references).
Let $V = H^{1}( \Omega ) $  and $W=H^{1}( \Omega ) $ be  Hilbert spaces and define the bilinear form $a: V\times V \to \mathbb{R}  $ and $b: V \times W \to \mathbb{R} $ s.t. $a( \sigma,v ) = - ( \sigma , v) _{\Omega }  $ and $b( u,v) = ( \nabla u,
\nabla v)_{\Omega  }  $. We also may define the linear forms, $G,F: V \to \mathbb{R} $ s.t. $ G( v)  = ( h_{1}, v) _{\Gamma } $ and $F( \varphi ) = ( f, \varphi )_{\Omega } + ( h_{2}, \varphi )_{\Gamma } $.

Hence, we can connect it to the SPP. We want to find $( u,\sigma ) \in V \times W$ s.t.  \[
    \begin{cases}
       a( \sigma ,v) + b ( u, v )  & = G( v)   \quad  \forall v \in V \\
       b( u, \varphi  )  & = F( \varphi )     \quad \forall \phi \in W
    \end{cases}
\]

This is useful since we can now apply standard saddle point theory to do an analysis for the problem.

There is several interesting papers on this subject.
\begin{itemize}
    \item General saddle point theory for the stokes equation \cite{john2016finite, knabner2003numerical}.
    \item Classical formulations \cite{babuvska1980analysis}
    \item  A Nitsche mixed extended finite element method for the biharmonic interface problem \cite{cai2023nitsche}
        \begin{enumerate}[label=\arabic*)]
            \item Applying Nitsche based penalty based on a so called Ciarlet–Raviart formulation
        \end{enumerate}
\end{itemize}






