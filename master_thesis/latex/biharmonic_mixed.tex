
\newpage
\section{Biharmonic Mixed Formulation}%
\label{sec:biharmonic_mixed_formulation}


\subsection{Introduction}%
\label{sub:introduction}

Recall the biharmonic problem formulation $u \in H^4( \Omega ) $  \[
    \begin{split}
\Delta ^2 u & = f \quad  \text{in } \Omega \\
\partial _{n} u & = h_{1}(x ) \\
\partial _{n}( \Delta u)  & = h_{2}( x)
    \end{split}
\]
For $f: \Omega  \to  \mathbb{R} $ and $h_{1},h_{2}: \Omega  \to \mathbb{R}  $. It is easy to see
that the formulation can be rewritten in an equivalent mixed formulation, that is, \[
    \begin{split}
\Delta \sigma  & = f \quad  \text{in } \Omega \\
\sigma   & = \Delta u  \text{ in } \Omega \\
\partial _{n} \sigma  & = h_{1}(x ) \text{ on } \Gamma  \\
\partial _{n} \tau   & = h_{2}( x) \text{ on } \Gamma
    \end{split}
\]



We say that we have a mixed formulation if the problem is written on the form; Give $f \in V' $  and $g \in W' $, then we want to find $( u, \phi ) \in V \times W$  s.t. \[
    \begin{cases}
       a( u,v) + b ( v, \phi )  & = ( f,v)  \quad  \forall v \in V, \\
       b( u, \phi )  & = ( g, \phi )    \quad \forall \phi \in W
    \end{cases}
\]
Where $V$  and $W$  is Hilbert spaces and $a: V\times V \to \mathbb{R}  $ and $b: V \times W \to \mathbb{R} $.

There is several interesting papers on this subject.
\begin{itemize}
    \item Classical formulations \cite{babuvska1980analysis}
    \item  A Nitsche mixed extended finite element method for the biharmonic interface problem \cite{cai2023nitsche}
        \begin{enumerate}[label=\arabic*)]
            \item Applying nitsche based penalty based on a so called Ciarlet–Raviart formulation
        \end{enumerate}
\end{itemize}






