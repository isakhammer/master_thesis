\section{Introduction}\label{sec:introduction}

% \subsection{The importance of Cahn-Hilliard}%
% \label{sub:the_importance_of_cahn_hilliard}


The first application of the Cahn-Hilliard equation (CH) appeared when modelling phase separation of two-component incompressible fluids \cite{cahn1958free, cahn1959free, cahn1961spinodal}, but was quickly generalized to handle multi-component system
as well \cite{bosch2015fractional, eyre1993systems, toth2016phase}. In engineering, CH is the key component in
the phase-field model, which is a mathematical framework to model transitions and interface dynamics in materials and fluid dynamics \cite{steinbach2009phase, chen2002phase}.
From this has the equation found many interesting applications for a big variety of problems. To mention a few, we have
multiphase fluid dynamical problems \cite{badalassi2003computation, li2016lattice, kim2012phase, shen2010phase}, solidification of binary alloys \cite{kim1999phase, echebarria2004quantitative}, and continuum modelling of fracture dynamics in
materials \cite{kuhn2010continuum, li2015phase}. Perhaps an unexpected application is that CH can be used to for inpainting when recovering damaged parts of an image \cite{bertozzi2006inpainting, burger2009cahn, bosch2015fractional, brkic2020image}.
CH has shown be be essential in many areas of biology and medicine. Formation of so-called lipids rafts on a biological membrane is modelled as a phase separation problem and has been shown to have major impact on the behaviour of the system.

Many have also seen

biology \cite{bressloff2014stochastic} thus, and comes in many forms \cite{miranville2017cahn}.


% \subsection{The equations}%
% \label{sub:the_equations}


% The goal of the thesis implement an modern unfitted finite element method (FEM) to solve the Cahn-Hilliard equation. The plan is to first derive the stability and well-posedness for the method for the fourth order Biharmonic equation. Once this framework is established will we aim for solving the Chan-Hilliard equation using well known techniques for time-discretization and nonlinear problems.

% We can describe the Cahn-Hilliard problem in this formulation. Let $\Omega  $ be a domain with boundary $\Gamma $. Assume that $c: \Omega  \to \left[ 0,1 \right] $ and let $h_{1},h_{2}:\Omega  \to \mathbb{R} $.


% \[
%     \begin{split}
% \partial _{t} c &= \Delta ( f(c) \Delta c)     \quad  \text{in } \Omega \\
% \partial _{n} c & = h_{1} \quad \text{ on } \Gamma  \\
% \partial _{n} \Delta c  & = h_{2} \quad \text{ on } \Gamma
%     \end{split}
% \]
% where $f( c) $ is a nonlinear functional. First of all, a simplified training problem is the so-called biharmonic problem.
% \[
%     \begin{split}
% \Delta ^2 u & = f \quad  \text{in } \Omega \\
% \partial _{n} u & = h_{1} \quad \text{ on } \Gamma  \\
% \partial _{n} \Delta u  & = h_{2} \quad \text{ on } \Gamma
%     \end{split}
% \]



% \begin{enumerate}[label=(\alph*)]
%     \item Derive an unfitted cut continuous interior penalty method (CutCIP) for the biharmonic equation and show that the method is well-posed.
%     \item Implement a time discretization scheme of an linear Cahn-Hilliard equation.
%     \item Implement nonlinear iteration methods for the (nonlinear) Cahn-Hilliard Equation.
% \end{enumerate}


% \subsection{Numerical Methods to solve the problem}%
% \label{sub:numerical_methods_to_solve_the_problem}


% \subsection{FEM}

% First signs of finite element was in 1941 when Hrennikoff used the method solve problems in linear elasticity \cite{hrennikoff1941solution}. It then took offspring to applications in aerospace application in the 50's when it was formalized the method
% could solve dynamical problems on complex structures as a competitor to the finite difference method
% \cite{argyris1960energy, turner1956stiffness, liu2022eighty}. To expect good numerical results does he finite element methods requires a lot of DOFs and thus did the popularity climb with the rise of the supercomputers. Thus it has proven to be
% useful found a great
% variety of applications in multi-physics problems because of its simplicity based on the abstract mathematical generalisation of both functions spaces and computational domains.

% CIP formulation of the Cahn-Hilliard was presented in \cite{wells2006discontinuous} which avoids high derivatives along boundaries, thus no need for global $C^{1}$ elements.


% \subsection{CutFEM}%
% \label{sub:cutfem}
% The goal of this thesis is to make a foundation of a numerical framework, which later can be developed to handle moving domains. We will use CutFEM method \cite{burman2015cutfem}.
% Alternatives to smooth boundaries is isogeometric analysis, which is based on NURBS, and does in fact handle complex geometries quite well \cite{hughes2005isogeometric}.

% Mixed CutFEM formulation of Cahn-Hilliard \cite{karatzas2021reduced} and a mixed Hybrid High-Order method \cite{chave2016hybrid}. Isogeometric formulation of Cahn-Hilliard \cite{kastner2016isogeometric, gomez2008isogeometric} and unfitted versions \cite{zhao2017variational}. Isogeoemtric formulation on moving
% surfaces \cite{zimmermann2019isogeometric}. Virtual $C^{1}$  elements has also been considered \cite{antonietti2016c}.



% \subsection{ODE integrators}%
% \label{sub:ode_integrators}

% Introduced unconditionally stable, second order accurate, numerically dissipative with parameters $\alpha$ and $ \beta $,  Newmans Method \cite{newmark1959method}
% Improved versions of the Newmark method by introducing the numerically dissipative parameters such as the Hilber method and the Bossak Method \cite{hilber1977improved, wood1980alpha}. Combing these ideas the Generalized $\alpha $ method from which all the other methods can be derived \cite{chung1993time}.


