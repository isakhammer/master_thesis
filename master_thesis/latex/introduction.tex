\section{Introduction}\label{sec:introduction}
The goal of the thesis implement an modern unfitted finite element method (FEM) to solve the Cahn-Hilliard equation. The plan is to first derive the stability and well-posedness for the method for the fourth order Biharmonic equation. Once this framework is established will we aim for solving the Chan-Hilliard equation using well known techniques for time-discretization and nonlinear problems.

We can describe the Cahn-Hilliard problem in this formulation. Let $\Omega  $ be a domain with boundary $\Gamma $. Assume that $c: \Omega  \to \left[ 0,1 \right] $ and let $h_{1},h_{2}:\Omega  \to \mathbb{R} $.

\[
    \begin{split}
\partial _{t} c &= \Delta ( f(c) \Delta c)     \quad  \text{in } \Omega \\
\partial _{n} c & = h_{1} \quad \text{ on } \Gamma  \\
\partial _{n} \Delta c  & = h_{2} \quad \text{ on } \Gamma
    \end{split}
\]
where $f( c) $ is a nonlinear functional. First of all, a simplified training problem is the so-called biharmonic problem.
\[
    \begin{split}
\Delta ^2 u & = f \quad  \text{in } \Omega \\
\partial _{n} u & = h_{1} \quad \text{ on } \Gamma  \\
\partial _{n} \Delta u  & = h_{2} \quad \text{ on } \Gamma
    \end{split}
\]
Such problems are often associated with the Cahn-Hilliard model
for phase separation \cite{cahnhilliard1957} .



\begin{enumerate}[label=(\alph*)]
    \item Derive an unfitted cut continuous interior penalty method (CutCIP) for the biharmonic equation and show that the method is well-posed.
    \item Implement a time discretization scheme of an linear Cahn-Hilliard equation.
    \item Implement nonlinear iteration methods for the (nonlinear) Cahn-Hilliard Equation.
\end{enumerate}


\subsection{CutFEM}%
\label{sub:cutfem}
The goal of this thesis is to make a foundation of a numerical framework, which later can be developed to handle moving domains. We will use CutFEM method \cite{burman2015cutfem}.
Alternatives to smooth boundaries is isogeometric analysis, which is based on NURBS, and does in fact handle complex geometries quite well \cite{hughes2005isogeometric}.




