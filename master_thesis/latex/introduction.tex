\section{Introduction}\label{sec:introduction}

% \subsection{The importance of Cahn-Hilliard}%
% \label{sub:the_importance_of_cahn_hilliard}


The first application of the Cahn-Hilliard equation (CH) appeared when modelling phase separation of two-component incompressible fluids \cite{cahn1958free, cahn1959free, cahn1961spinodal}, but was quickly generalized to handle multi-component system
as well \cite{bosch2015fractional, eyre1993systems, toth2016phase, miranville2017cahn}. In engineering, CH is the key component in
the phase-field model, which is a mathematical framework to model transitions and interface dynamics in materials and fluid dynamics \cite{steinbach2009phase, chen2002phase}.
From this has the equation found many interesting applications for a big variety of problems. To mention a few, we have
multiphase fluid dynamical problems \cite{badalassi2003computation, li2016lattice, kim2012phase, shen2010phase}, solidification of binary or multicomponent alloys \cite{kim1999phase, echebarria2004quantitative}, and continuum modelling of fracture dynamics in
materials \cite{kuhn2010continuum, li2015phase}. Perhaps an unexpected application is that CH can be used to for in painting when recovering damaged parts of an image \cite{bertozzi2006inpainting, burger2009cahn, bosch2015fractional, brkic2020image}
and modelling the origin of the irregular structure in the Saturn's rings \cite{tremaine2003origin}.
CH has also shown be be essential in many areas of biology and medicine. From a macroscopic point of view is that CH is a great tool to model tumor growth, wound healing and brain tumors \cite{agosti2017cahn, cristini2009nonlinear}.
On the microscopic level on the bio membrane is there a ongoing hypothesis about the existence of the accumulation of lipids into so-called lipid rafts which serve as a rigid platform for proteins with
special properties such as signalling and intercellular trafficking \cite{ levental2020lipid, hancock2006lipid, munro2003lipid, simons1997functional}. It turns out that the hypothesis can tested by modelling the problem as a separation problem using
CH \cite{miller2020divide, garcke2016coupled, yushutin2019computational}.

\subsection{The Equation}%
\label{sub:the_equations}

We can describe the Cahn-Hilliard equation comes in many variants depending on its application \cite{miranville2017cahn}. Let $\Omega  $ be a domain with boundary $\Gamma $. Assume that $u: \Omega  \to \mathbb{R}  $ and let $g_{1},g_{2}:\Omega  \to \mathbb{R} $.

\[
    \begin{split}
\partial _{t} u &= \Delta ( f(c) +  \Delta c)     \quad  \text{in } \Omega \\
\partial _{n} u & = g_{1} \quad \text{ on } \Gamma  \\
\partial _{n} \Delta u  & = g_{2} \quad \text{ on } \Gamma
    \end{split}
\]
where $f( u) $ is a nonlinear functional.
\todo[inline]{ Add a paragraph to describe the strong form of this problem and its boundary conditions. (Maybe a short derivation based on Energy functional?) }

\subsection{Numerical Methods}%
\label{sub:numerical_methods}

\todo[inline]{ Describe the most common ways to solve this problem. FVM? FDM? FEM? One paragraph}

First signs of finite element was in 1941 when Hrennikoff used the method solve problems in linear elasticity \cite{hrennikoff1941solution}. It then took offspring to applications in aerospace application in the 50's when it was formalized the method
could solve dynamical problems on complex structures as a competitor to the finite difference method
\cite{argyris1960energy, turner1956stiffness, liu2022eighty}. To expect good numerical results does he finite element methods requires a lot of DOFs and thus did the popularity climb with the rise of the supercomputers. Thus it has proven to be
useful found a great variety of applications in multi-physics problems because of its simplicity based on the abstract mathematical generalisation of both functions spaces and computational domains.

\todo[inline]{ Describe why we ended up with a finite element. What methods are used in FEM? Methods to handle the non-linearities? Methods for time discretization? Methods to handle smooth geometries? Methods to handle moving domains?}
CIP formulation of the Cahn-Hilliard was presented in \cite{wells2006discontinuous} which avoids high derivatives along boundaries, thus no need for global $C^{1}$ elements.

Alternatives to smooth boundaries is isogeometric analysis, which is based on NURBS, and does in fact handle complex geometries quite well \cite{hughes2005isogeometric}.

Mixed CutFEM formulation of Cahn-Hilliard \cite{karatzas2021reduced} and a mixed Hybrid High-Order method \cite{chave2016hybrid}. Isogeometric formulation of Cahn-Hilliard \cite{kastner2016isogeometric, gomez2008isogeometric} and unfitted versions \cite{zhao2017variational}. Isogeoemtric formulation on moving
surfaces \cite{zimmermann2019isogeometric}. Virtual $C^{1}$  elements has also been considered \cite{antonietti2016c}.

Introduced unconditionally stable, second order accurate, numerically dissipative with parameters $\alpha$ and $ \beta $,  Newmans Method \cite{newmark1959method}
Improved versions of the Newmark method by introducing the numerically dissipative parameters such as the Hilber method and the Bossak Method \cite{hilber1977improved, wood1980alpha}. Combing these ideas the Generalized $\alpha $ method from which all the other methods can be derived \cite{chung1993time}.

% \subsection{CutFEM}%
% \label{sub:cutfem}
% The goal of this thesis is to make a foundation of a numerical framework, which later can be developed to handle moving domains. We will use CutFEM method \cite{burman2015cutfem}.


% \subsection{ODE integrators}%
% \label{sub:ode_integrators}



\subsection{Outline of the report}%
\label{sub:outline_of_the_report}
\todo[inline]{ Describe the workflow of this method. }


The goal of the thesis implement an modern unfitted finite element method (FEM) to solve the Cahn-Hilliard equation. The plan is to first derive the stability and well-posedness for the method for the fourth order Biharmonic equation,
\[
    \begin{split}
\Delta ^2 u & = f \quad  \text{in } \Omega \\
\partial _{n} u & = g_{1} \quad \text{ on } \Gamma  \\
\partial _{n} \Delta u  & = g_{2} \quad \text{ on } \Gamma
    \end{split}
\]
\begin{enumerate}[label=(\alph*)]
    \item Derive an unfitted cut continuous interior penalty method (CutCIP) for the biharmonic equation and show that the method is well-posed.
    \item Implement a time discretization scheme of an linear Cahn-Hilliard equation.
    \item Implement nonlinear iteration methods for the (nonlinear) Cahn-Hilliard Equation.
\end{enumerate}
