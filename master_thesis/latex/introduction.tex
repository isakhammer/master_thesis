\section{Introduction}\label{sec:introduction}


The first application of the Cahn-Hilliard equation (CH) appeared when modelling phase separation of two-component incompressible fluids \cite{cahn1958free, cahn1959free, cahn1961spinodal}, but was quickly generalized to handle multi-component system
as well \cite{bosch2015fractional, eyre1993systems, toth2016phase, miranville2017cahn}. In engineering, CH is the key component in
the phase-field model, which is a mathematical framework to model transitions and interface dynamics in materials and fluid dynamics \cite{steinbach2009phase, chen2002phase}.
From this has the equation found many interesting applications for a big variety of problems. To mention a few, we have
multiphase fluid dynamical problems \cite{badalassi2003computation, li2016lattice, kim2012phase, shen2010phase}, solidification of binary or multicomponent alloys \cite{kim1999phase, echebarria2004quantitative}, and continuum modelling of fracture dynamics in
materials \cite{kuhn2010continuum, li2015phase}. Perhaps an unexpected application is that CH can be used to for in painting when recovering damaged parts of an image \cite{bertozzi2006inpainting, burger2009cahn, bosch2015fractional, brkic2020image}
and modelling the origin of the irregular structure in the Saturn's rings \cite{tremaine2003origin}.
CH has also shown be be essential in many areas of biology and medicine. From a macroscopic point of view is that CH is a great tool to model tumor growth, wound healing and brain tumors \cite{agosti2017cahn, cristini2009nonlinear}.
On the microscopic level on the bio membrane is there a ongoing hypothesis about the existence of the accumulation of lipids into so-called lipid rafts which serve as a rigid platform for proteins with
special properties such as signalling and intercellular trafficking \cite{ levental2020lipid, hancock2006lipid, munro2003lipid, simons1997functional}. It turns out that the hypothesis can tested by modelling the problem as a separation problem using
CH \cite{miller2020divide, garcke2016coupled, yushutin2019computational}.

\subsection{The Cahn Hilliard Equation}%
\label{sub:the_equations}

% \todo[inline]{ Derivation of the Cahn-Hilliard equation from its energy functional. }
The Cahn-Hilliard equation comes in many variants depending on its application, but we will in this report on the binary mixture version \cite{miranville2017cahn}. Let $\Omega \subset  \mathbb{R} ^{d} $ be a compact set for $d=2,3$ with boundary
$\Gamma $. We define the time duration parameter $T \in  [0,\infty) $ and the
so-called unknown phase-field function as the
mapping $u: \left[ 0,T \right] \times \Omega  \to \left[ -1,1 \right]  $, which is denoted as the local difference of a binary mixture of two concentrations $c_{A}, c_{B} \in \left[ 0,1\right] $ s.t. $u = c_{A} -c_{B}$ and $c_{A} + c_{B} = 1$. Remark that if there exist a
local point so $u$ has the extreme value $-1$, then it implies that the particular point has $100\%$ concentration $c_{A}$ and vice-versa for $c_{B}$. On the other hand, if $u$ is zero it implies that the mixture is $50\% - 50\%$.

For a isotropic
binary mixture nonuniform, the standard Ginzburg-Landay free energy functional is given by \[
E( u)  = \int_{\Omega }^{} \frac{\varepsilon ^2}{2} \abs{ \nabla u } ^2 + F( u) dx
\]
The nonlinear function $F( u) $ is denoted as the (Helmholtz) free energy density associated with the interaction dynamics between the components and thus comes in many forms depending on the thermodynamic properties, see \cite{miranville2017cahn}.
However, we will in this article assume that $F( u) = ( 1 / 4 ) ( 1- u^2) ^{4} $.
We choose to define the chemical potential $\mu $ as the variational derivative,
\[
\mu = \frac{ \delta E( u) }{ \delta  u} = f( u)  - \varepsilon ^{2} \Delta c .
\]
where we used the notation $f( u) = F'( u) $.
First of all, to require local mass conservation we may enforce the continuity equation, that is \[
\partial _{t} u + \nabla \cdot \mathcal{J}  = 0,
\]
where $J$ denotes the flux governed by the physical dynamics. Hence, this naturally leads the no-flux and the Neumann boundary conditions,
\begin{equation}
\label{eq:conservation}
    \begin{split}
\mathcal{J}  \cdot n & = 0 \text{ on } \Gamma \\
\partial _{n} u & = 0 \text{ on } \Gamma
    \end{split}
\end{equation}
 A well accepted law for the flux  is to proportional to the gradient of the chemical energy, $\mathcal{J} = - M  \nabla \mu  $ for a parameter $M$.
Thus, we finally have the strong form Cahn-Hilliard equation. Let $ u( x,0) =  u_{0}$ then is the dynamics on the form,

\begin{equation}
\label{eq:strongch}
    \begin{split}
\partial _{t} u  = \Delta ( f( u)  - \varepsilon ^2 \Delta u ) &\quad \text{ in } \Omega  \\
\partial _{n} u = 0 \quad &\text{ on } \Gamma  \\
\partial _{n}(f( u)  - \varepsilon ^2 \Delta u )  = 0 \quad &\text{ on } \Gamma  \\
    \end{split}
\end{equation}

Based on these laws and the boundary conditions, it becomes evident that the energy functional serves as a Lyapunov function in the sense that its time derivative is monotonically decreasing and that the global mass concentration is zero, i.e.
\[
    \begin{split}
\frac{d}{dt} E( u)  \le  0 \text{ and }\frac{d}{dt} \int_{\Omega }^{}  u dx = 0.
    \end{split}
\]
Remark that the computation of the inequality does utilize the assumption of $M$ to be constant and both equations require the no-flux boundary condition, $\mathcal{J} \cdot n = 0$.
For details, see \cite[Equation 17 ]{lee2014physical} and \cite[Equation 1.7]{garcke2020weak}.
This is useful since we expect $E( u( \cdot , t_{2}) ) \le  E( u( \cdot , t_{1}) ) $ for $0 < t_{1} < t_{2} $ and that the global mass is conserved, \[
\int_{\Omega }^{} u ( x,t)  dx = \int_{\Omega }^{} u_{0}(x)  dx.
\]
The properties under consideration serve as a theoretical foundation for establishing the existence, uniqueness, and long-term behavior of the CH problem. Consequently, these properties are well-comprehended from a mathematical standpoint. For
references, see \cite{abels2007convergence, cherfils2011cahn, elliott1986cahn}.

\subsection{Numerical Methods}%
\label{sub:numerical_methods}

One of the key challenges with the CH problem is that it involves fourth order spatial derivatives. It has for simple domains successfully been implemented using finite difference methods (FDM)  \cite{furihata2001stable,
cheng2019energy} and Spectral Methods \cite{liu2003phase, he2009class}. However, these methods are generally constrained to simple domains (with some notable exceptions \cite{li2013conservative, shen2009efficient, feng2009fourier}).

As a further evolution to address the CH problem, it is common to consider a corresponding biharmonic (BH) problem as a numerical testbed in the spatial direction. This problem is defined as follows, \[
\begin{split}
    \Delta ^2 u & = f( x) \quad \text{in }  \Omega, \\
    \partial _{n} u & = g_{1} \quad \text{on } \Gamma,   \\
    \partial _{n} \Delta  u & = g_{2} \quad \text{on } \Gamma.   \\
\end{split}
\]
Generally is this problem since it is likely to provide a nice spatial-integration framework before moving onto solving the non-linearities and time-integration.

The early Finite Element Methods (FEM) for CH was proposed in \cite{elliott1987numerical, elliott1986cahn} utilizing global $C^{1}$ and $C^{2}$ in one spatial dimension, but later is has been shown that making $C^{1}$ (or higher order) elements are far from trivial. For
reference, see \cite{kapl2021family, percell1976cubic, argyris1968tuba}.

To avoid the need for global $C^1$ continuity has it been research on two families of methods.
% Paragraph of CIP
The first involves the Continuous Interior Penalty (CIP) methods, which use the standard weak formulation but penalize the discontinuity of the derivative between elements as a form of regularization. The method has it been design several interesting
stable variants, that is \cite{brenner2012, brenner2012quadratic, brenner2012quadratic_kirk, mu2014weak, georgoulis2009discontinuous}.
The CIP formulation has also then been adapted to solve CH \cite{wells2006discontinuous}.
% The theory of CIP method is closely related to the family of discontinuous Galerkin methods which is well documented.

% Paragraph of Mixed
A another popular variant is to rewrite the BH problem in a mixed formulation as a system of second-order problems \cite{falk1978approximation, ciarlet1974mixed, gudi2008mixed, cheng2000some}.


% Paragraph of time integration

% Paragraph of isogeometric analysis
Alternatives to smooth boundaries is isogeometric analysis, which is based on NURBS, and does in fact handle complex geometries quite well \cite{hughes2005isogeometric}.
Mixed CutFEM formulation of Cahn-Hilliard \cite{karatzas2021reduced} and a mixed Hybrid High-Order method \cite{chave2016hybrid}. Isogeometric formulation of Cahn-Hilliard \cite{kastner2016isogeometric, gomez2008isogeometric} and unfitted versions \cite{zhao2017variational}. Isogeoemtric formulation on moving
surfaces \cite{zimmermann2019isogeometric}. Virtual $C^{1}$  elements has also been considered \cite{antonietti2016c}.

% Paragraph of CutFEM

% Introduced unconditionally stable, second order accurate, numerically dissipative with parameters $\alpha$ and $ \beta $,  Newmans Method \cite{newmark1959method}
% Improved versions of the Newmark method by introducing the numerically dissipative parameters such as the Hilber method and the Bossak Method \cite{hilber1977improved, wood1980alpha}. Combing these ideas the Generalized $\alpha $ method from which all the other methods can be derived \cite{chung1993time}.

% \subsection{CutFEM}%
% \label{sub:cutfem}
% The goal of this thesis is to make a foundation of a numerical framework, which later can be developed to handle moving domains. We will use CutFEM method \cite{burman2015cutfem}.

% \subsection{ODE integrators}%
% \label{sub:ode_integrators}

% First signs of finite element was in 1941 when Hrennikoff used the method solve problems in linear elasticity \cite{hrennikoff1941solution}. It then took offspring to applications in aerospace application in the 50's when it was formalized the method
% could solve dynamical problems on complex structures as a competitor to the finite difference method
% \cite{argyris1960energy, turner1956stiffness, liu2022eighty}. To expect good numerical results does he finite element methods requires a lot of DOFs and thus did the popularity climb with the rise of the supercomputers. Thus it has proven to be
% useful found a great variety of applications in multi-physics problems because of its simplicity based on the abstract mathematical generalisation of both functions spaces and computational domains.

\subsection{Outline of the report}%
\label{sub:outline_of_the_report}
\todo[inline]{ Describe the workflow of this method. }


The goal of the thesis implement an modern unfitted finite element method (FEM) to solve the Cahn-Hilliard equation. The plan is to first derive the stability and well-posedness for the method for the fourth order Biharmonic equation,
\[
    \begin{split}
\Delta ^2 u & = f \quad  \text{in } \Omega \\
\partial _{n} u & = g_{1} \quad \text{ on } \Gamma  \\
\partial _{n} \Delta u  & = g_{2} \quad \text{ on } \Gamma
    \end{split}
\]
\begin{enumerate}[label=(\alph*)]
    \item Derive an unfitted cut continuous interior penalty method (CutCIP) for the biharmonic equation and show that the method is well-posed.
    \item Implement a time discretization scheme of an linear Cahn-Hilliard equation.
    \item Implement nonlinear iteration methods for the (nonlinear) Cahn-Hilliard Equation.
\end{enumerate}
