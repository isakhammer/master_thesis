
\newpage
\section{Numerical results}%
\label{sec:numerical_results}


\begin{table}[h!]
    \caption{Convergence analysis of the numerical method}
    \label{table:CutFEM_error1}
  \begin{tabular}{rrrrrrrrr}
    \hline\hline
    \textbf{$n$} & \textbf{$\Vert e \Vert_{L^2}$} & \textbf{EOC} & \textbf{$ \Vert e \Vert_{H^1}$} & \textbf{EOC} & \textbf{$\Vert e \Vert_{ a_h,* }$} & \textbf{EOC} & \textbf{Cond number} & \textbf{ndofs} \\\hline
    8 & 1.6E+01 &  & 1.3E+02 &  & 3.6E+03 &  & 3.0E+05 & 2.4E+02 \\
    16 & 3.7E+00 & 2.14 & 3.1E+01 & 2.09 & 1.0E+03 & 1.78 & 2.2E+06 & 8.3E+02 \\
    32 & 8.3E-01 & 2.15 & 7.4E+00 & 2.09 & 2.9E+02 & 1.83 & 1.6E+07 & 3.0E+03 \\
    64 & 2.3E-01 & 1.87 & 1.8E+00 & 2.03 & 8.0E+01 & 1.86 & 1.3E+08 & 1.1E+04 \\
    128 & 5.0E-02 & 2.19 & 4.4E-01 & 2.04 & 2.6E+01 & 1.63 & 1.0E+09 & 4.3E+04 \\
    256 & 1.3E-02 & 1.92 & 1.1E-01 & 2.01 & 8.4E+00 & 1.63 & 7.9E+09 & 1.7E+05 \\\hline\hline
  \end{tabular}

\end{table}

\begin{figure}[h!]
    \centering
    \begin{table}
  \begin{tabular}{rrrrrrrrr}
    \noalign{\hrule height 2pt}
    \textbf{$n$} & \textbf{$\Vert e \Vert_{L^2}$} & \textbf{EOC} & \textbf{$ \Vert e \Vert_{H^1}$} & \textbf{EOC} & \textbf{$\Vert e \Vert_{ a_h,* }$} & \textbf{EOC} & \textbf{Cond number} & \textbf{ndofs} \\\noalign{\hrule height 2pt}
    8 & 8.5E-01 & NaN & 2.2E+00 & NaN & 7.9E+01 & NaN & 5.4E+05 & 2.4E+02 \\
    16 & 1.3E-02 & 6.04 & 8.2E-02 & 4.72 & 2.4E+00 & 5.04 & 1.1E+06 & 6.8E+02 \\
    32 & 3.5E-03 & 1.89 & 2.2E-02 & 1.91 & 1.1E+00 & 1.14 & 7.6E+06 & 2.3E+03 \\
    64 & 8.4E-04 & 2.05 & 5.6E-03 & 1.97 & 6.2E-01 & 0.82 & 5.9E+07 & 8.9E+03 \\
    128 & 2.1E-04 & 2.00 & 1.4E-03 & 1.99 & 2.7E-01 & 1.20 & 4.7E+08 & 3.4E+04 \\
    256 & 5.4E-05 & 1.96 & 3.5E-04 & 1.99 & 1.3E-01 & 1.06 & 3.7E+09 & 1.3E+05 \\\noalign{\hrule height 2pt}
  \end{tabular}
\end{table}

    \caption{The plot presents the error in $L^2$, $H^1$ the energy norm ($\Vert e \Vert_{a_h,*}$) with order 2 of the CutCIP method (Laplace).}
    \label{fig:CutFEM_error1}
\end{figure}

\begin{figure}[h!]
    \centering
    \begin{subfigure}{0.49\textwidth}
        \centering
        % Recommended preamble:
% \usetikzlibrary{arrows.meta}
% \usetikzlibrary{backgrounds}
% \usepgfplotslibrary{patchplots}
% \usepgfplotslibrary{fillbetween}
% \pgfplotsset{%
%     layers/standard/.define layer set={%
%         background,axis background,axis grid,axis ticks,axis lines,axis tick labels,pre main,main,axis descriptions,axis foreground%
%     }{
%         grid style={/pgfplots/on layer=axis grid},%
%         tick style={/pgfplots/on layer=axis ticks},%
%         axis line style={/pgfplots/on layer=axis lines},%
%         label style={/pgfplots/on layer=axis descriptions},%
%         legend style={/pgfplots/on layer=axis descriptions},%
%         title style={/pgfplots/on layer=axis descriptions},%
%         colorbar style={/pgfplots/on layer=axis descriptions},%
%         ticklabel style={/pgfplots/on layer=axis tick labels},%
%         axis background@ style={/pgfplots/on layer=axis background},%
%         3d box foreground style={/pgfplots/on layer=axis foreground},%
%     },
% }

\begin{tikzpicture}[/tikz/background rectangle/.style={fill={rgb,1:red,1.0;green,1.0;blue,1.0}, fill opacity={1.0}, draw opacity={1.0}}, show background rectangle]
\begin{axis}[point meta max={nan}, point meta min={nan}, legend cell align={left}, legend columns={1}, title={}, title style={at={{(0.5,1)}}, anchor={south}, font={{\fontsize{14 pt}{18.2 pt}\selectfont}}, color={rgb,1:red,0.0;green,0.0;blue,0.0}, draw opacity={1.0}, rotate={0.0}, align={center}}, legend style={color={rgb,1:red,0.0;green,0.0;blue,0.0}, draw opacity={1.0}, line width={1}, solid, fill={rgb,1:red,1.0;green,1.0;blue,1.0}, fill opacity={1.0}, text opacity={1.0}, font={{\fontsize{8 pt}{10.4 pt}\selectfont}}, text={rgb,1:red,0.0;green,0.0;blue,0.0}, cells={anchor={center}}, at={(1.02, 1)}, anchor={north west}}, axis background/.style={fill={rgb,1:red,1.0;green,1.0;blue,1.0}, opacity={1.0}}, anchor={north west}, xshift={1.0mm}, yshift={-1.0mm}, width={145.4mm}, height={99.6mm}, scaled x ticks={false}, xlabel={$\delta$}, x tick style={color={rgb,1:red,0.0;green,0.0;blue,0.0}, opacity={1.0}}, x tick label style={color={rgb,1:red,0.0;green,0.0;blue,0.0}, opacity={1.0}, rotate={0}}, xlabel style={at={(ticklabel cs:0.5)}, anchor=near ticklabel, at={{(ticklabel cs:0.5)}}, anchor={near ticklabel}, font={{\fontsize{11 pt}{14.3 pt}\selectfont}}, color={rgb,1:red,0.0;green,0.0;blue,0.0}, draw opacity={1.0}, rotate={0.0}}, xmajorgrids={true}, xmin={-0.001471665988344504}, xmax={0.05052719893316125}, xticklabels={{$0.00$,$0.01$,$0.02$,$0.03$,$0.04$,$0.05$}}, xtick={{0.0,0.010000000000000002,0.020000000000000004,0.030000000000000006,0.04000000000000001,0.05000000000000001}}, xtick align={inside}, xticklabel style={font={{\fontsize{8 pt}{10.4 pt}\selectfont}}, color={rgb,1:red,0.0;green,0.0;blue,0.0}, draw opacity={1.0}, rotate={0.0}}, x grid style={color={rgb,1:red,0.0;green,0.0;blue,0.0}, draw opacity={0.1}, line width={0.5}, solid}, axis x line*={left}, x axis line style={color={rgb,1:red,0.0;green,0.0;blue,0.0}, draw opacity={1.0}, line width={1}, solid}, scaled y ticks={false}, ylabel={$\kappa(A)$}, y tick style={color={rgb,1:red,0.0;green,0.0;blue,0.0}, opacity={1.0}}, y tick label style={color={rgb,1:red,0.0;green,0.0;blue,0.0}, opacity={1.0}, rotate={0}}, ylabel style={at={(ticklabel cs:0.5)}, anchor=near ticklabel, at={{(ticklabel cs:0.5)}}, anchor={near ticklabel}, font={{\fontsize{11 pt}{14.3 pt}\selectfont}}, color={rgb,1:red,0.0;green,0.0;blue,0.0}, draw opacity={1.0}, rotate={0.0}}, ymode={log}, log basis y={10}, ymajorgrids={true}, ymin={100000.0}, ymax={1.0e25}, yticklabels={{$10^{5}$,$10^{10}$,$10^{15}$,$10^{20}$,$10^{25}$}}, ytick={{100000.0,1.0e10,1.0e15,1.0e20,1.0e25}}, ytick align={inside}, yticklabel style={font={{\fontsize{8 pt}{10.4 pt}\selectfont}}, color={rgb,1:red,0.0;green,0.0;blue,0.0}, draw opacity={1.0}, rotate={0.0}}, y grid style={color={rgb,1:red,0.0;green,0.0;blue,0.0}, draw opacity={0.1}, line width={0.5}, solid}, axis y line*={left}, y axis line style={color={rgb,1:red,0.0;green,0.0;blue,0.0}, draw opacity={1.0}, line width={1}, solid}, colorbar={false}]
    [\addlegendimage{empty legend}] \addlegendentry[font={{\fontsize{11 pt}{14.3 pt}\selectfont}}, text={rgb,1:red,0.0;green,0.0;blue,0.0}] {\hspace{-.6cm}{\textbf{$(\gamma, \gamma_1, \gamma_2)$}}}
    \addplot[color={rgb,1:red,0.0;green,0.0;blue,1.0}, name path={1a807371-3e5a-480f-a163-3ad0d4bc4382}, draw opacity={1.0}, line width={1}, solid]
        table[row sep={\\}]
        {
            \\
            0.0  1.2679162980675659e8  \\
            9.830768125213776e-5  1.2680687823756537e8  \\
            0.0001966153625042755  1.2681967925832012e8  \\
            0.0002949230437564133  1.268300474760143e8  \\
            0.000393230725008551  1.2683796829979104e8  \\
            0.0004915384062606888  1.2684342530531715e8  \\
            0.0005898460875128266  1.2686139370184006e8  \\
            0.0006881537687649644  1.2686075219695534e8  \\
            0.000786461450017102  1.2686087327645683e8  \\
            0.0008847691312692399  1.2686551185390782e8  \\
            0.0009830768125213777  1.2687006593177739e8  \\
            0.0010813844937735155  1.2687452326801112e8  \\
            0.0011796921750256532  1.2688212369912691e8  \\
            0.0012779998562777908  1.2689312690197854e8  \\
            0.0013763075375299288  1.2691868029945637e8  \\
            0.0014746152187820663  1.268948378561596e8  \\
            0.001572922900034204  1.268864637049064e8  \\
            0.0016712305812863419  1.2688485812855718e8  \\
            0.0017695382625384799  1.2688312736136235e8  \\
            0.0018678459437906176  1.2688131391746148e8  \\
            0.0019661536250427554  1.268794147065554e8  \\
            0.002064461306294893  1.2687739397205383e8  \\
            0.002162768987547031  1.2688088258017486e8  \\
            0.0022610766687991683  1.2692645215918219e8  \\
            0.0023593843500513065  1.2692413911040716e8  \\
            0.0024576920313034442  1.2692398832661444e8  \\
            0.0025559997125555816  1.269269892649186e8  \\
            0.00265430739380772  1.2692989382748559e8  \\
            0.0027526150750598576  1.2693270517181793e8  \\
            0.0028509227563119953  1.2681420801618162e8  \\
            0.0029492304375641327  1.2681117902990691e8  \\
            0.003047538118816271  1.2692941868769366e8  \\
            0.003145845800068408  1.2693142044087845e8  \\
            0.003244153481320546  1.269324550420301e8  \\
            0.0033424611625726837  1.269325467894505e8  \\
            0.003440768843824822  1.2693168378286676e8  \\
            0.0035390765250769597  1.2692988380661857e8  \\
            0.0036373842063290975  1.2692711745184304e8  \\
            0.0037356918875812353  1.2692340243518572e8  \\
            0.0038339995688333726  1.2691872031523156e8  \\
            0.003932307250085511  1.269131198436436e8  \\
            0.004030614931337649  1.2690655253652686e8  \\
            0.004128922612589786  1.2689903957986815e8  \\
            0.004227230293841924  1.2689056517732508e8  \\
            0.004325537975094062  1.2688115686955453e8  \\
            0.0044238456563462  1.2687081353307994e8  \\
            0.0045221533375983365  1.2685949935676442e8  \\
            0.004620461018850475  1.26847263760272e8  \\
            0.004718768700102613  1.2683406896650636e8  \\
            0.004817076381354751  1.268379918831678e8  \\
            0.0049153840626068885  1.2686427163687037e8  \\
            0.005013691743859026  1.2688838170129517e8  \\
            0.005111999425111163  1.2691036404801688e8  \\
            0.005210307106363301  1.269301693815558e8  \\
            0.00530861478761544  1.2694783651350005e8  \\
            0.005406922468867577  1.2696335059300832e8  \\
            0.005505230150119715  1.2697671251653418e8  \\
            0.005603537831371853  1.2698791790333259e8  \\
            0.005701845512623991  1.269969788222526e8  \\
            0.0058001531938761276  1.2700385276300646e8  \\
            0.005898460875128265  1.2700857062642296e8  \\
            0.005996768556380403  1.2664088894956867e8  \\
            0.006095076237632542  1.266343595107889e8  \\
            0.006193383918884679  1.2662770360723044e8  \\
            0.006291691600136816  1.2662094723486093e8  \\
            0.006389999281388954  1.2697991483699545e8  \\
            0.006488306962641092  1.2697958544586308e8  \\
            0.00658661464389323  1.2697650243541273e8  \\
            0.0066849223251453675  1.2697063451527533e8  \\
            0.006783230006397506  1.2696200822375764e8  \\
            0.006881537687649644  1.2695062911494075e8  \\
            0.006979845368901782  1.2693650430743295e8  \\
            0.0070781530501539194  1.2691963811192304e8  \\
            0.007176460731406057  1.2693305212879306e8  \\
            0.007274768412658195  1.2693280796359102e8  \\
            0.007373076093910333  1.2692919465120995e8  \\
            0.0074713837751624705  1.2692227018117693e8  \\
            0.007569691456414607  1.2691205077709544e8  \\
            0.007667999137666745  1.2689852351900998e8  \\
            0.007766306818918883  1.2691102383482742e8  \\
            0.007864614500171022  1.2692565000905915e8  \\
            0.007962922181423158  1.26938104415268e8  \\
            0.008061229862675297  1.2682072588068998e8  \\
            0.008159537543927434  1.268208816359404e8  \\
            0.008257845225179573  1.2682566042205875e8  \\
            0.00835615290643171  1.2684250308653902e8  \\
            0.008454460587683848  1.2693088688273773e8  \\
            0.008552768268935985  1.269399754459724e8  \\
            0.008651075950188124  1.2694605770944552e8  \\
            0.008749383631440262  1.2694902324280314e8  \\
            0.0088476913126924  1.2691493939967619e8  \\
            0.008945998993944536  1.2695526709271704e8  \\
            0.009044306675196673  1.2692535098695725e8  \\
            0.009142614356448812  1.268969891858638e8  \\
            0.00924092203770095  1.2686631983072469e8  \\
            0.009339229718953087  1.2683335931294784e8  \\
            0.009437537400205226  1.2679808573500594e8  \\
            0.009535845081457363  1.2691457744014263e8  \\
            0.009634152762709501  1.2677819149765758e8  \\
            0.009732460443961638  1.2673591514611788e8  \\
            0.009830768125213777  1.2669131036908022e8  \\
            0.009929075806465914  1.2668183338087782e8  \\
            0.010027383487718053  1.2672392789225487e8  \\
            0.010125691168970191  1.2676337640629393e8  \\
            0.010223998850222326  1.2680245102453563e8  \\
            0.010322306531474465  1.2684203794530605e8  \\
            0.010420614212726602  1.268791336031587e8  \\
            0.01051892189397874  1.2691363324757831e8  \\
            0.01061722957523088  1.2694551005074945e8  \\
            0.010715537256483016  1.2697476808748105e8  \\
            0.010813844937735155  1.270014295688094e8  \\
            0.010912152618987292  1.2702548401844417e8  \\
            0.01101046030023943  1.2707019471737559e8  \\
            0.011108767981491567  1.270968728707832e8  \\
            0.011207075662743706  1.271130796875566e8  \\
            0.011305383343995843  1.2712664076581012e8  \\
            0.011403691025247981  1.2713755380314519e8  \\
            0.011501998706500118  1.2714657780016026e8  \\
            0.011600306387752255  1.2718115048598807e8  \\
            0.011698614069004394  1.2716756670566939e8  \\
            0.01179692175025653  1.2718605969899234e8  \\
            0.01189522943150867  1.2720067421609372e8  \\
            0.011993537112760806  1.2721139470795906e8  \\
            0.012091844794012945  1.2721819566707517e8  \\
            0.012190152475265083  1.2722489664015596e8  \\
            0.012288460156517219  1.2725279445333336e8  \\
            0.012386767837769357  1.2724510068277211e8  \\
            0.012485075519021494  1.272325270863088e8  \\
            0.012583383200273633  1.2721510286478981e8  \\
            0.012681690881525771  1.272289479085141e8  \\
            0.012779998562777908  1.2727708865360117e8  \\
            0.012878306244030047  1.272826139495292e8  \\
            0.012976613925282184  1.2728590084512493e8  \\
            0.013074921606534323  1.2728689230527009e8  \\
            0.01317322928778646  1.2728561693524328e8  \\
            0.013271536969038598  1.2708821070758194e8  \\
            0.013369844650290735  1.2703291388034308e8  \\
            0.013468152331542874  1.269730366860631e8  \\
            0.013566460012795012  1.2690859529498897e8  \\
            0.01366476769404715  1.2683962113573629e8  \\
            0.013763075375299288  1.2676616361977181e8  \\
            0.013861383056551425  1.266882301415422e8  \\
            0.013959690737803563  1.2660586792578913e8  \\
            0.0140579984190557  1.2652435050580086e8  \\
            0.014156306100307839  1.2730123018261448e8  \\
            0.014254613781559976  1.2729528592984582e8  \\
            0.014352921462812114  1.272652629568292e8  \\
            0.014451229144064253  1.2725547638022996e8  \\
            0.01454953682531639  1.2724768187737249e8  \\
            0.014647844506568529  1.2728299930100085e8  \\
            0.014746152187820665  1.272645657338248e8  \\
            0.014844459869072804  1.2724530143745503e8  \\
            0.014942767550324941  1.2722379244803067e8  \\
            0.015041075231577076  1.2720007764367078e8  \\
            0.015139382912829215  1.2717414361346298e8  \\
            0.015237690594081352  1.2714600236859223e8  \\
            0.01533599827533349  1.2711562243336181e8  \\
            0.015434305956585627  1.270830135993837e8  \\
            0.015532613637837766  1.270481566520487e8  \\
            0.015630921319089903  1.2701108606556112e8  \\
            0.015729229000342043  1.2697174577807891e8  \\
            0.01582753668159418  1.2693016851687552e8  \\
            0.015925844362846317  1.2688633322805431e8  \\
            0.016024152044098454  1.2684023107537624e8  \\
            0.016122459725350594  1.2679187127645646e8  \\
            0.01622076740660273  1.2674121712638816e8  \\
            0.016319075087854868  1.2724222514829445e8  \\
            0.01641738276910701  1.2722587366506271e8  \\
            0.016515690450359145  1.2719687354773055e8  \\
            0.016613998131611282  1.2715546551256423e8  \\
            0.01671230581286342  1.27101842151658e8  \\
            0.01681061349411556  1.2703620319124736e8  \\
            0.016908921175367696  1.2695877029113722e8  \\
            0.017007228856619833  1.2686970742690125e8  \\
            0.01710553653787197  1.26774241521045e8  \\
            0.01720384421912411  1.2666806073835672e8  \\
            0.017302151900376248  1.2661605265231161e8  \\
            0.017400459581628384  1.2676120597538169e8  \\
            0.017498767262880525  1.2688843941699417e8  \\
            0.01759707494413266  1.269973690494709e8  \\
            0.0176953826253848  1.2717776010081863e8  \\
            0.017793690306636935  1.2717370852318032e8  \\
            0.017891997987889072  1.2717156281895271e8  \\
            0.01799030566914121  1.271697670536278e8  \\
            0.018088613350393346  1.27204444893315e8  \\
            0.018186921031645487  1.2719167297226985e8  \\
            0.018285228712897623  1.271801009062305e8  \\
            0.01838353639414976  1.2716846437758105e8  \\
            0.0184818440754019  1.2715467240473701e8  \\
            0.018580151756654038  1.2713871029787466e8  \\
            0.018678459437906175  1.2712059213003783e8  \\
            0.01877676711915831  1.2717503729809974e8  \\
            0.018875074800410452  1.2717315923645e8  \\
            0.01897338248166259  1.2711852772018988e8  \\
            0.019071690162914726  1.2711526317217165e8  \\
            0.019169997844166863  1.2711145368334706e8  \\
            0.019268305525419003  1.2710709204337019e8  \\
            0.01936661320667114  1.2710220310771118e8  \\
            0.019464920887923277  1.2710430816268575e8  \\
            0.019563228569175417  1.2715332987747718e8  \\
            0.019661536250427554  1.271468255186224e8  \\
            0.01975984393167969  1.2713976790084428e8  \\
            0.019858151612931828  1.2713218854169784e8  \\
            0.019956459294183968  1.2712408619975346e8  \\
            0.020054766975436105  1.2712775750198524e8  \\
            0.020153074656688242  1.2713583155146545e8  \\
            0.020251382337940382  1.2714351188002416e8  \\
            0.02034969001919252  1.2715001423139885e8  \\
            0.020447997700444653  1.2715474677262338e8  \\
            0.020546305381696793  1.2717486057944643e8  \\
            0.02064461306294893  1.2718180709994152e8  \\
            0.020742920744201067  1.2718712280984163e8  \\
            0.020841228425453204  1.2719084639809515e8  \\
            0.020939536106705344  1.2719292840441015e8  \\
            0.02103784378795748  1.2719774812249024e8  \\
            0.021136151469209618  1.2720194367785849e8  \\
            0.02123445915046176  1.2709302422202489e8  \\
            0.021332766831713895  1.2708146625571348e8  \\
            0.021431074512966032  1.270705399734262e8  \\
            0.02152938219421817  1.2705911809673493e8  \\
            0.02162768987547031  1.2704712742177565e8  \\
            0.021725997556722446  1.2720533057719435e8  \\
            0.021824305237974583  1.2720769395666325e8  \\
            0.02192261291922672  1.2720978257848662e8  \\
            0.02202092060047886  1.2721161192434609e8  \\
            0.022119228281730997  1.2721315334640552e8  \\
            0.022217535962983134  1.2721446170349877e8  \\
            0.022315843644235275  1.2721546692915003e8  \\
            0.02241415132548741  1.2715500733522075e8  \\
            0.02251245900673955  1.2715283261737026e8  \\
            0.022610766687991685  1.2715543073452346e8  \\
            0.022709074369243826  1.2716371680681695e8  \\
            0.022807382050495963  1.2717838716671684e8  \\
            0.0229056897317481  1.2717733942785907e8  \\
            0.023003997413000236  1.2717640477940845e8  \\
            0.023102305094252377  1.2715514587632628e8  \\
            0.02320061277550451  1.271472608622306e8  \\
            0.02329892045675665  1.271355211515707e8  \\
            0.023397228138008788  1.2711490994436511e8  \\
            0.023495535819260924  1.2711372573817095e8  \\
            0.02359384350051306  1.2710055869364823e8  \\
            0.0236921511817652  1.2707929300447051e8  \\
            0.02379045886301734  1.2707374182415417e8  \\
            0.023888766544269475  1.2706527463643624e8  \\
            0.023987074225521612  1.2705388821098809e8  \\
            0.024085381906773753  1.2703958895298919e8  \\
            0.02418368958802589  1.2702238474995825e8  \\
            0.024281997269278027  1.270022764197992e8  \\
            0.024380304950530167  1.2697929595522225e8  \\
            0.024478612631782304  1.2695343467314838e8  \\
            0.024576920313034437  1.2695343468666992e8  \\
            0.024675227994286578  1.2697929680767381e8  \\
            0.024773535675538715  1.2700227728397162e8  \\
            0.02487184335679085  1.2702238463671592e8  \\
            0.02497015103804299  1.2703957446918225e8  \\
            0.02506845871929513  1.2705387461288795e8  \\
            0.025166766400547266  1.2706526037861331e8  \\
            0.025265074081799403  1.2707374112936433e8  \\
            0.025363381763051543  1.2707930498605336e8  \\
            0.02546168944430368  1.2710055952856505e8  \\
            0.025559997125555817  1.27113739668396e8  \\
            0.025658304806807954  1.2711489646259616e8  \\
            0.025756612488060094  1.2713552194158237e8  \\
            0.02585492016931223  1.2714725810247228e8  \\
            0.025953227850564368  1.2715515749033844e8  \\
            0.026051535531816508  1.2717640444411887e8  \\
            0.026149843213068645  1.2717732622053736e8  \\
            0.026248150894320782  1.2717840059395334e8  \\
            0.02634645857557292  1.2716371728947647e8  \\
            0.02644476625682506  1.271554180002317e8  \\
            0.026543073938077196  1.271528187601028e8  \\
            0.026641381619329333  1.2715500840889744e8  \\
            0.02673968930058147  1.2721546687551771e8  \\
            0.02683799698183361  1.2721444908300272e8  \\
            0.026936304663085747  1.2721316645332544e8  \\
            0.027034612344337884  1.2721161248730293e8  \\
            0.027132920025590024  1.272097984087594e8  \\
            0.02723122770684216  1.2720769272148667e8  \\
            0.0273295353880943  1.2720533184198777e8  \\
            0.027427843069346435  1.270471410805855e8  \\
            0.027526150750598576  1.2705911895050174e8  \\
            0.027624458431850712  1.270705521305853e8  \\
            0.02772276611310285  1.270814520597821e8  \\
            0.027821073794354986  1.2709300926632164e8  \\
            0.027919381475607127  1.2720194326657638e8  \\
            0.028017689156859264  1.2719776180188474e8  \\
            0.0281159968381114  1.271929281459719e8  \\
            0.02821430451936354  1.2719083352920726e8  \\
            0.028312612200615678  1.2718713518524115e8  \\
            0.028410919881867815  1.271818048179778e8  \\
            0.02850922756311995  1.2717484542346e8  \\
            0.028607535244372092  1.2715475893868661e8  \\
            0.02870584292562423  1.2715002618596244e8  \\
            0.028804150606876366  1.2714351251935878e8  \\
            0.028902458288128506  1.2713582979072577e8  \\
            0.029000765969380643  1.271277677778379e8  \\
            0.02909907365063278  1.2712407467938222e8  \\
            0.029197381331884917  1.2713220015404381e8  \\
            0.029295689013137057  1.2713976702737619e8  \\
            0.029393996694389194  1.271468108042564e8  \\
            0.02949230437564133  1.271533296320874e8  \\
            0.029590612056893468  1.27104309145371e8  \\
            0.029688919738145608  1.2710221550052404e8  \\
            0.029787227419397745  1.2710710573169018e8  \\
            0.029885535100649882  1.2711144168320289e8  \\
            0.029983842781902015  1.2711524922197352e8  \\
            0.030082150463154152  1.2711851447836854e8  \\
            0.030180458144406293  1.2717314359607518e8  \\
            0.03027876582565843  1.2717504824307133e8  \\
            0.030377073506910567  1.2712059285984166e8  \\
            0.030475381188162703  1.2713870934345406e8  \\
            0.030573688869414844  1.2715467134429541e8  \\
            0.03067199655066698  1.2716846325971636e8  \\
            0.030770304231919118  1.2718008978539449e8  \\
            0.030868611913171255  1.2719167319059417e8  \\
            0.030966919594423395  1.2720445757551125e8  \\
            0.031065227275675532  1.2716978136304875e8  \\
            0.03116353495692767  1.271715622707009e8  \\
            0.031261842638179806  1.2717370821332286e8  \\
            0.03136015031943194  1.271777616709108e8  \\
            0.031458458000684086  1.2699735505224599e8  \\
            0.03155676568193622  1.268884279301714e8  \\
            0.03165507336318836  1.2676120638719128e8  \\
            0.0317533810444405  1.2661605291758308e8  \\
            0.031851688725692634  1.2666806255276377e8  \\
            0.03194999640694477  1.2677424159028119e8  \\
            0.03204830408819691  1.2686970766403064e8  \\
            0.03214661176944905  1.2695875706149387e8  \\
            0.03224491945070119  1.2703620280607507e8  \\
            0.032343227131953325  1.2710184094282074e8  \\
            0.03244153481320546  1.2715547982505365e8  \\
            0.0325398424944576  1.271968743959589e8  \\
            0.032638150175709736  1.2722587503274837e8  \\
            0.03273645785696187  1.2724223886033522e8  \\
            0.03283476553821402  1.2674121547791763e8  \\
            0.032933073219466154  1.2679185745159847e8  \\
            0.03303138090071829  1.2684023123427941e8  \\
            0.03312968858197043  1.2688633333258152e8  \\
            0.033227996263222564  1.2693018061716147e8  \\
            0.0333263039444747  1.2697176007689394e8  \\
            0.03342461162572684  1.2701108572955732e8  \\
            0.033522919306978975  1.27048169482642e8  \\
            0.03362122698823112  1.2708301328630145e8  \\
            0.033719534669483256  1.2711560855398385e8  \\
            0.03381784235073539  1.271459896521182e8  \\
            0.03391615003198753  1.2717414424566773e8  \\
            0.03401445771323967  1.2720007663076149e8  \\
            0.034112765394491804  1.2722379191107188e8  \\
            0.03421107307574394  1.2724530205327258e8  \\
            0.034309380756996084  1.2726457892385378e8  \\
            0.03440768843824822  1.2728301226156503e8  \\
            0.03450599611950036  1.2724769598770966e8  \\
            0.034604303800752495  1.2725546251139979e8  \\
            0.03470261148200463  1.2726526125114335e8  \\
            0.03480091916325677  1.2729528724851012e8  \\
            0.034899226844508906  1.2730122963747579e8  \\
            0.03499753452576105  1.2652435101565625e8  \\
            0.035095842207013186  1.266058546590367e8  \\
            0.03519414988826532  1.2668821745488602e8  \\
            0.03529245756951746  1.2676616536589041e8  \\
            0.0353907652507696  1.268396350868848e8  \\
            0.03548907293202173  1.2690860776151532e8  \\
            0.03558738061327387  1.2697303573140776e8  \\
            0.03568568829452601  1.2703292751465523e8  \\
            0.035783995975778145  1.2708821001187693e8  \\
            0.03588230365703028  1.2728563148277788e8  \\
            0.03598061133828242  1.2728689253171189e8  \\
            0.036078919019534555  1.272859016588281e8  \\
            0.03617722670078669  1.272826147863187e8  \\
            0.036275534382038836  1.272770740222524e8  \\
            0.03637384206329097  1.2722893421203224e8  \\
            0.03647214974454311  1.2721510347251007e8  \\
            0.03657045742579525  1.272325130000217e8  \\
            0.036668765107047384  1.2724510008925042e8  \\
            0.03676707278829952  1.272527925852454e8  \\
            0.03686538046955166  1.2722489818974042e8  \\
            0.0369636881508038  1.2721819603234261e8  \\
            0.03706199583205594  1.2721139387778775e8  \\
            0.037160303513308075  1.2720066290359971e8  \\
            0.03725861119456021  1.2718605774787673e8  \\
            0.03735691887581235  1.2716756536974028e8  \\
            0.037455226557064486  1.2718113685214144e8  \\
            0.03755353423831662  1.2714657812955031e8  \\
            0.03765184191956877  1.2713755590201297e8  \\
            0.037750149600820904  1.2712662930688858e8  \\
            0.03784845728207304  1.2711306557963291e8  \\
            0.03794676496332518  1.2709688754351963e8  \\
            0.038045072644577314  1.2707019501711814e8  \\
            0.03814338032582945  1.2702549848992755e8  \\
            0.03824168800708159  1.2700144371276848e8  \\
            0.038339995688333725  1.2697476950795296e8  \\
            0.03843830336958587  1.2694549963930607e8  \\
            0.038536611050838006  1.2691363420774844e8  \\
            0.03863491873209014  1.2687913326156135e8  \\
            0.03873322641334228  1.2684205187825856e8  \\
            0.038831534094594417  1.2680246511151214e8  \\
            0.03892984177584655  1.2676337713780147e8  \\
            0.03902814945709869  1.2672391497585924e8  \\
            0.039126457138350834  1.2668183186627029e8  \\
            0.03922476481960297  1.2669131069365768e8  \\
            0.03932307250085511  1.2673591479597078e8  \\
            0.039421380182107245  1.2677817781070042e8  \\
            0.03951968786335938  1.269145628673008e8  \\
            0.03961799554461152  1.2679808652729273e8  \\
            0.039716303225863656  1.2683335928989369e8  \\
            0.0398146109071158  1.2686630709625702e8  \\
            0.039912918588367936  1.2689697492853531e8  \\
            0.04001122626962007  1.2692536368750337e8  \\
            0.04010953395087221  1.2695527753444603e8  \\
            0.04020784163212435  1.269149256187344e8  \\
            0.040306149313376484  1.2694903789948523e8  \\
            0.04040445699462862  1.2694606074237168e8  \\
            0.040502764675880765  1.2693997552269638e8  \\
            0.0406010723571329  1.2693090038387202e8  \\
            0.04069938003838504  1.2684250329672605e8  \\
            0.040797687719637175  1.2682566150244153e8  \\
            0.040895995400889305  1.268208842421794e8  \\
            0.04099430308214144  1.2682072625379023e8  \\
            0.041092610763393586  1.2693811909392442e8  \\
            0.04119091844464572  1.2692566518894942e8  \\
            0.04128922612589786  1.2691101070343417e8  \\
            0.04138753380715  1.268985084386492e8  \\
            0.041485841488402134  1.2691204940541214e8  \\
            0.04158414916965427  1.2692228521893536e8  \\
            0.04168245685090641  1.2692919587189914e8  \\
            0.04178076453215855  1.2693279484268971e8  \\
            0.04187907221341069  1.2693305387610033e8  \\
            0.041977379894662825  1.269196405591142e8  \\
            0.04207568757591496  1.2693649021505758e8  \\
            0.0421739952571671  1.269506299977259e8  \\
            0.042272302938419236  1.269620080765999e8  \\
            0.04237061061967137  1.2697062168427856e8  \\
            0.04246891830092352  1.2697648900509666e8  \\
            0.042567225982175654  1.2697958591200314e8  \\
            0.04266553366342779  1.2697990238690537e8  \\
            0.04276384134467993  1.2662096190707198e8  \\
            0.042862149025932064  1.2662771645572516e8  \\
            0.0429604567071842  1.2663435921444814e8  \\
            0.04305876438843634  1.2664087514159267e8  \\
            0.043157072069688475  1.2700858586803393e8  \\
            0.04325537975094062  1.2700386637100466e8  \\
            0.043353687432192756  1.2699698079277524e8  \\
            0.04345199511344489  1.2698791859153803e8  \\
            0.04355030279469703  1.2697671368419439e8  \\
            0.043648610475949166  1.269633494361212e8  \\
            0.0437469181572013  1.2694783630061872e8  \\
            0.04384522583845344  1.2693018547036557e8  \\
            0.043943533519705584  1.2691035257894453e8  \\
            0.04404184120095772  1.268883828559377e8  \\
            0.04414014888220986  1.2686427011619428e8  \\
            0.044238456563461995  1.2683797823795944e8  \\
            0.04433676424471413  1.268340706305334e8  \\
            0.04443507192596627  1.2684726326715882e8  \\
            0.044533379607218405  1.2685949883288786e8  \\
            0.04463168728847055  1.268708158098006e8  \\
            0.044729994969722686  1.2688115618161584e8  \\
            0.04482830265097482  1.2689057988193582e8  \\
            0.04492661033222696  1.2689904046217944e8  \\
            0.0450249180134791  1.2690655292956926e8  \\
            0.045123225694731234  1.2691310375978938e8  \\
            0.04522153337598337  1.269187333500742e8  \\
            0.045319841057235515  1.2692340333679618e8  \\
            0.04541814873848765  1.2692710593554339e8  \\
            0.04551645641973979  1.269298709318532e8  \\
            0.045614764100991925  1.2693170012738134e8  \\
            0.04571307178224406  1.2693256208612083e8  \\
            0.0458113794634962  1.2693246914718652e8  \\
            0.045909687144748336  1.2693142255688916e8  \\
            0.04600799482600047  1.2692940716112758e8  \\
            0.04610630250725262  1.2681118062314059e8  \\
            0.046204610188504754  1.2681420885617736e8  \\
            0.04630291786975689  1.2693271861246762e8  \\
            0.04640122555100902  1.2692990880319624e8  \\
            0.04649953323226116  1.2692699001502796e8  \\
            0.0465978409135133  1.2692400228944905e8  \\
            0.04669614859476544  1.2692413778864643e8  \\
            0.046794456276017575  1.2692643687952192e8  \\
            0.04689276395726971  1.2688089815553662e8  \\
            0.04699107163852185  1.2687739289525306e8  \\
            0.047089379319773986  1.2687940117339122e8  \\
            0.04718768700102612  1.2688132629194702e8  \\
            0.047285994682278266  1.2688313989964883e8  \\
            0.0473843023635304  1.2688485689847554e8  \\
            0.04748261004478254  1.2688646327622293e8  \\
            0.04758091772603468  1.2689484972956343e8  \\
            0.047679225407286814  1.2691868156765595e8  \\
            0.04777753308853895  1.2689311237854658e8  \\
            0.04787584076979109  1.2688213681113982e8  \\
            0.047974148451043225  1.2687452359710544e8  \\
            0.04807245613229537  1.2687008048246601e8  \\
            0.048170763813547506  1.2686551209580536e8  \\
            0.04826907149479964  1.2686085893245949e8  \\
            0.04836737917605178  1.2686075192939766e8  \\
            0.048465686857303916  1.268613917547832e8  \\
            0.04856399453855605  1.2684343866989249e8  \\
            0.04866230221980819  1.2683795479134473e8  \\
            0.048760609901060334  1.2683003439729318e8  \\
            0.04885891758231247  1.2681966817841457e8  \\
            0.04895722526356461  1.2680687821802719e8  \\
            0.049055532944816745  1.2679164293105379e8  \\
        }
        ;
    \addlegendentry { $1.0 \cdot 10^{1}$, $0.5 \cdot 10^{1}$, $1.0 \cdot 10^{-1}$ }
    \addplot[color={rgb,1:red,1.0;green,0.0;blue,0.0}, name path={c38c5060-ddb9-4ad2-be7a-eea552e31edc}, draw opacity={1.0}, line width={1}, solid]
        table[row sep={\\}]
        {
            \\
            0.0  4.639960145909961e10  \\
            9.830768125213776e-5  9.69178605738127e10  \\
            0.0001966153625042755  2.3845800281330045e11  \\
            0.0002949230437564133  8.588075792498015e11  \\
            0.000393230725008551  8.999021832965525e12  \\
            0.0004915384062606888  2.0666571731531807e12  \\
            0.0005898460875128266  1.017062911302021e20  \\
            0.0006881537687649644  4.7248203449960904e16  \\
            0.000786461450017102  1.7069988999386722e15  \\
            0.0008847691312692399  1.9643235063560697e14  \\
            0.0009830768125213777  3.940762364824056e13  \\
            0.0010813844937735155  1.703346067867858e14  \\
            0.0011796921750256532  1.4051487203854045e15  \\
            0.0012779998562777908  4.967321087455724e16  \\
            0.0013763075375299288  6.283364725795125e21  \\
            0.0014746152187820663  1.477792836788004e15  \\
            0.001572922900034204  7.068992923196961e13  \\
            0.0016712305812863419  9.96599858508195e12  \\
            0.0017695382625384799  7.420062712563103e12  \\
            0.0018678459437906176  1.067675150836421e13  \\
            0.0019661536250427554  1.6226775583079125e13  \\
            0.002064461306294893  2.621296205352156e13  \\
            0.002162768987547031  4.544428351724183e13  \\
            0.0022610766687991683  7.731077504613927e19  \\
            0.0023593843500513065  4.231759112000662e16  \\
            0.0024576920313034442  1.7689119045593385e15  \\
            0.0025559997125555816  1.25795146486825e15  \\
            0.00265430739380772  4.820611509344994e15  \\
            0.0027526150750598576  2.894921536916048e16  \\
            0.0028509227563119953  3.00909546885669e22  \\
            0.0029492304375641327  2.0366264101228446e20  \\
            0.003047538118816271  1.371863214591281e19  \\
            0.003145845800068408  2.1304347895988244e18  \\
            0.003244153481320546  5.122750816232713e17  \\
            0.0033424611625726837  1.6129810673280906e17  \\
            0.003440768843824822  6.0992565451765304e16  \\
            0.0035390765250769597  2.6333190136894776e16  \\
            0.0036373842063290975  1.2570639739160104e16  \\
            0.0037356918875812353  6.492764341479366e15  \\
            0.0038339995688333726  3.573223013934013e15  \\
            0.003932307250085511  2.0718947893266148e15  \\
            0.004030614931337649  1.2550573710375718e15  \\
            0.004128922612589786  7.890337431207952e14  \\
            0.004227230293841924  5.1216452033663344e14  \\
            0.004325537975094062  3.4181715078350444e14  \\
            0.0044238456563462  2.3376082291073222e14  \\
            0.0045221533375983365  1.6335123759218703e14  \\
            0.004620461018850475  1.1636565624871417e14  \\
            0.004718768700102613  8.433673824812103e13  \\
            0.004817076381354751  6.208169752627569e13  \\
            0.0049153840626068885  4.634841398400729e13  \\
            0.005013691743859026  3.5049403470804047e13  \\
            0.005111999425111163  2.6818078310229883e13  \\
            0.005210307106363301  2.074233313754407e13  \\
            0.00530861478761544  1.620325006599659e13  \\
            0.005406922468867577  1.2774239616507043e13  \\
            0.005505230150119715  1.0157006204851385e13  \\
            0.005603537831371853  8.140190064995386e12  \\
            0.005701845512623991  6.572162087339613e12  \\
            0.0058001531938761276  1.224101242527807e13  \\
            0.005898460875128265  4.58503310297416e13  \\
            0.005996768556380403  7.545277656915696e18  \\
            0.006095076237632542  1.8306624354425164e16  \\
            0.006193383918884679  6.327544719949214e16  \\
            0.006291691600136816  1.2793564132534536e20  \\
            0.006389999281388954  2.7845820659836008e13  \\
            0.006488306962641092  8.2138546318288545e12  \\
            0.00658661464389323  1.789190546927232e13  \\
            0.0066849223251453675  7.808092606453948e13  \\
            0.006783230006397506  5.3073503966841425e14  \\
            0.006881537687649644  9.434284575678236e15  \\
            0.006979845368901782  2.5393934902466208e17  \\
            0.0070781530501539194  2.2691283763109913e19  \\
            0.007176460731406057  6.473626036188701e11  \\
            0.007274768412658195  1.4465518100578137e12  \\
            0.007373076093910333  3.616338754895206e12  \\
            0.0074713837751624705  1.0508096392843707e13  \\
            0.007569691456414607  3.775049439153802e13  \\
            0.007667999137666745  1.8718645908292072e14  \\
            0.007766306818918883  1.6082900528759302e15  \\
            0.007864614500171022  4.410072695102857e16  \\
            0.007962922181423158  2.963173307681006e22  \\
            0.008061229862675297  1.5722202238076215e11  \\
            0.008159537543927434  1.3980385139975787e11  \\
            0.008257845225179573  1.245724194608263e11  \\
            0.00835615290643171  1.0e23  \\
            0.008454460587683848  2.0094364898147443e17  \\
            0.008552768268935985  3.6648052201373755e15  \\
            0.008651075950188124  3.301760886504717e14  \\
            0.008749383631440262  5.8393557045292836e13  \\
            0.0088476913126924  3.0001803791229145e19  \\
            0.008945998993944536  5.2973656654397304e16  \\
            0.009044306675196673  2.0771849094108578e15  \\
            0.009142614356448812  2.3314645128041425e14  \\
            0.00924092203770095  4.465418516280445e13  \\
            0.009339229718953087  1.1822278522721092e13  \\
            0.009437537400205226  3.888728715051765e12  \\
            0.009535845081457363  1.494389644034858e12  \\
            0.009634152762709501  4.409191144261708e16  \\
            0.009732460443961638  2.0771133811491047e12  \\
            0.009830768125213777  1.5528830465653903e11  \\
            0.009929075806465914  8.385732493560904e10  \\
            0.010027383487718053  4.759729443400236e10  \\
            0.010125691168970191  2.818651455347003e10  \\
            0.010223998850222326  1.74615963878381e10  \\
            0.010322306531474465  1.6050643801337774e10  \\
            0.010420614212726602  1.4769635642333878e10  \\
            0.01051892189397874  1.3605205325244919e10  \\
            0.01061722957523088  1.2545502761199474e10  \\
            0.010715537256483016  1.158000642431987e10  \\
            0.010813844937735155  1.3495873791192976e10  \\
            0.010912152618987292  1.966120350575815e10  \\
            0.01101046030023943  2.9643884791277203e10  \\
            0.011108767981491567  1.1363529173413788e16  \\
            0.011207075662743706  3.7307303820103555e12  \\
            0.011305383343995843  1.6929121301994012e11  \\
            0.011403691025247981  2.4435495867974133e11  \\
            0.011501998706500118  4.907071937906927e11  \\
            0.011600306387752255  1.2893344500815478e22  \\
            0.011698614069004394  9.292833745415535e19  \\
            0.01179692175025653  5.166960812637816e16  \\
            0.01189522943150867  1.8669263157600505e15  \\
            0.011993537112760806  2.6338612511318888e14  \\
            0.012091844794012945  5.305282294726201e15  \\
            0.012190152475265083  3.896044768877705e18  \\
            0.012288460156517219  1.0e23  \\
            0.012386767837769357  1.2752821139260682e17  \\
            0.012485075519021494  3.2610566057270345e15  \\
            0.012583383200273633  4.0801006870555694e14  \\
            0.012681690881525771  1.0490191572497877e14  \\
            0.012779998562777908  8.98558799231956e16  \\
            0.012878306244030047  3.0421366678636188e13  \\
            0.012976613925282184  1.9996980981346498e15  \\
            0.013074921606534323  5.833294761118557e12  \\
            0.01317322928778646  1.466631931549802e12  \\
            0.013271536969038598  7.799480940155404e17  \\
            0.013369844650290735  7.221004693921882e15  \\
            0.013468152331542874  5.2011145587628594e14  \\
            0.013566460012795012  8.213723019429105e13  \\
            0.01366476769404715  5.861735279254524e13  \\
            0.013763075375299288  3.3219900516945456e14  \\
            0.013861383056551425  3.6970672299547455e15  \\
            0.013959690737803563  2.0332413963680416e17  \\
            0.0140579984190557  1.0e23  \\
            0.014156306100307839  4.459132708126265e16  \\
            0.014254613781559976  1.0e23  \\
            0.014352921462812114  7.070997787135313e10  \\
            0.014451229144064253  4.1955520878766365e10  \\
            0.01454953682531639  2.583946413963402e10  \\
            0.014647844506568529  2.5672347459312314e17  \\
            0.014746152187820665  3.004217089703152e13  \\
            0.014844459869072804  1.1418058050280513e12  \\
            0.014942767550324941  1.5040693796475717e11  \\
            0.015041075231577076  3.454904785415566e10  \\
            0.015139382912829215  1.5498160627433846e10  \\
            0.015237690594081352  1.8723230661061665e10  \\
            0.01533599827533349  2.2741902331936188e10  \\
            0.015434305956585627  2.7781087557167603e10  \\
            0.015532613637837766  3.4142409068295254e10  \\
            0.015630921319089903  7.430053153033447e10  \\
            0.015729229000342043  1.9108986250628244e11  \\
            0.01582753668159418  6.001720933570117e11  \\
            0.015925844362846317  2.554550978287962e12  \\
            0.016024152044098454  1.8269434238065527e13  \\
            0.016122459725350594  3.886286835241201e14  \\
            0.01622076740660273  4.276766993279511e17  \\
            0.016319075087854868  2.333929259183853e11  \\
            0.01641738276910701  3.09908064735197e11  \\
            0.016515690450359145  4.166944238279801e11  \\
            0.016613998131611282  5.680397103903696e11  \\
            0.01671230581286342  7.862141065595032e11  \\
            0.01681061349411556  1.106723912957384e12  \\
            0.016908921175367696  1.587627363403267e12  \\
            0.017007228856619833  2.3265994050542456e12  \\
            0.01710553653787197  5.837867331452696e12  \\
            0.01720384421912411  1.847395760360295e13  \\
            0.017302151900376248  7.552368000981453e13  \\
            0.017400459581628384  4.648322864106595e14  \\
            0.017498767262880525  6.125480688821749e15  \\
            0.01759707494413266  5.6795011895640845e17  \\
            0.0176953826253848  9.23498469815075e13  \\
            0.017793690306636935  2.089996062776401e14  \\
            0.017891997987889072  5.839518086297736e14  \\
            0.01799030566914121  2.933676306454614e15  \\
            0.018088613350393346  1.0e23  \\
            0.018186921031645487  4.4945014228577785e15  \\
            0.018285228712897623  6.138115803891745e15  \\
            0.01838353639414976  1.1594106624689242e16  \\
            0.0184818440754019  2.915120050735041e16  \\
            0.018580151756654038  1.0397885127006277e17  \\
            0.018678459437906175  3.311521664852422e18  \\
            0.01877676711915831  1.0842538139649974e19  \\
            0.018875074800410452  3.2627742896753566e22  \\
            0.01897338248166259  7.767972051844303e11  \\
            0.019071690162914726  9.95127428676866e11  \\
            0.019169997844166863  1.2868565180188398e12  \\
            0.019268305525419003  1.6810850254778367e12  \\
            0.01936661320667114  2.220373551253645e12  \\
            0.019464920887923277  2.9679484060950347e12  \\
            0.019563228569175417  1.5761503242819222e15  \\
            0.019661536250427554  1.503886725647448e13  \\
            0.01975984393167969  7.702725280798741e12  \\
            0.019858151612931828  1.0937911221671605e13  \\
            0.019956459294183968  1.5835046817917607e13  \\
            0.020054766975436105  2.3425500122803e13  \\
            0.020153074656688242  3.55074065834086e13  \\
            0.020251382337940382  5.5325230490090664e13  \\
            0.02034969001919252  8.896351925627253e13  \\
            0.020447997700444653  1.483554075872993e14  \\
            0.020546305381696793  8.069165498569363e18  \\
            0.02064461306294893  1.9847041563852084e16  \\
            0.020742920744201067  1.0001711441936626e15  \\
            0.020841228425453204  1.9198107364892828e15  \\
            0.020939536106705344  4.4067698108755885e15  \\
            0.02103784378795748  1.1399333443299084e16  \\
            0.021136151469209618  3.4614390297273628e16  \\
            0.02123445915046176  6.123809470953853e19  \\
            0.021332766831713895  7.097154844465725e17  \\
            0.021431074512966032  6.944546431972005e18  \\
            0.02152938219421817  2.491552523321387e20  \\
            0.02162768987547031  1.0e23  \\
            0.021725997556722446  7.266175025208652e13  \\
            0.021824305237974583  1.1473383829116781e14  \\
            0.02192261291922672  1.7962846400781938e15  \\
            0.02202092060047886  1.4044086193645358e15  \\
            0.022119228281730997  7.165985976762076e15  \\
            0.022217535962983134  1.9548849328461405e17  \\
            0.022315843644235275  2.116921797384778e22  \\
            0.02241415132548741  8.437699856942208e11  \\
            0.02251245900673955  5.69820460101781e11  \\
            0.022610766687991685  9.388201520215134e11  \\
            0.022709074369243826  3.36751142191123e12  \\
            0.022807382050495963  1.3322677745345252e18  \\
            0.0229056897317481  3.220851680058081e15  \\
            0.023003997413000236  1.3257887197894666e16  \\
            0.023102305094252377  2.7480436284669754e13  \\
            0.02320061277550451  4.280126306871381e13  \\
            0.02329892045675665  2.186543572472715e14  \\
            0.023397228138008788  2.2008544781517042e15  \\
            0.023495535819260924  4.725169347148843e16  \\
            0.02359384350051306  3.309179421633442e19  \\
            0.0236921511817652  1.3871116239791196e11  \\
            0.02379045886301734  8.717107669058075e10  \\
            0.023888766544269475  5.8739971621009705e10  \\
            0.023987074225521612  5.213814377289781e10  \\
            0.024085381906773753  7.404752314953891e10  \\
            0.02418368958802589  2.3802484430132053e11  \\
            0.024281997269278027  1.0195523229830544e11  \\
            0.024380304950530167  3.35510370516035e10  \\
            0.024478612631782304  6.0876920182003555e10  \\
            0.024576920313034437  6.0876919632543205e10  \\
            0.024675227994286578  3.3551037475988266e10  \\
            0.024773535675538715  1.0195522858137186e11  \\
            0.02487184335679085  2.38024874696688e11  \\
            0.02497015103804299  7.404752450015689e10  \\
            0.02506845871929513  5.2138144254800186e10  \\
            0.025166766400547266  5.873997211358435e10  \\
            0.025265074081799403  8.717107521031052e10  \\
            0.025363381763051543  1.3871115857782956e11  \\
            0.02546168944430368  3.3077665774771966e19  \\
            0.025559997125555817  4.725167329083897e16  \\
            0.025658304806807954  2.200853307721612e15  \\
            0.025756612488060094  2.1865453734394438e14  \\
            0.02585492016931223  4.28012658198572e13  \\
            0.025953227850564368  2.7480434107135723e13  \\
            0.026051535531816508  1.3257918692462232e16  \\
            0.026149843213068645  3.2208591288606135e15  \\
            0.026248150894320782  1.3322590502076946e18  \\
            0.02634645857557292  3.367511544552502e12  \\
            0.02644476625682506  9.388201282775691e11  \\
            0.026543073938077196  5.698204705089503e11  \\
            0.026641381619329333  8.437699990973594e11  \\
            0.02673968930058147  2.103455450891781e22  \\
            0.02683799698183361  1.9548937717556704e17  \\
            0.026936304663085747  7.165973777583817e15  \\
            0.027034612344337884  1.4044033018624382e15  \\
            0.027132920025590024  1.7962648608145888e15  \\
            0.02723122770684216  1.1473372548603144e14  \\
            0.0273295353880943  7.266172537696772e13  \\
            0.027427843069346435  1.0e23  \\
            0.027526150750598576  2.491110741526511e20  \\
            0.027624458431850712  6.945339994750908e18  \\
            0.02772276611310285  7.096794918487866e17  \\
            0.027821073794354986  6.122934959289001e19  \\
            0.027919381475607127  3.4614495739798284e16  \\
            0.028017689156859264  1.1399328634213006e16  \\
            0.0281159968381114  4.406771203603912e15  \\
            0.02821430451936354  1.9198107140766705e15  \\
            0.028312612200615678  1.0001704665088338e15  \\
            0.028410919881867815  1.9847109761221916e16  \\
            0.02850922756311995  8.069458626984405e18  \\
            0.028607535244372092  1.483554598252925e14  \\
            0.02870584292562423  8.896349743706273e13  \\
            0.028804150606876366  5.5325235047589234e13  \\
            0.028902458288128506  3.550741159016042e13  \\
            0.029000765969380643  2.3425502780302965e13  \\
            0.02909907365063278  1.583504655537e13  \\
            0.029197381331884917  1.09379107752493e13  \\
            0.029295689013137057  7.702725408543031e12  \\
            0.029393996694389194  1.5038868217959627e13  \\
            0.02949230437564133  1.5761495777523015e15  \\
            0.029590612056893468  2.9679483271594126e12  \\
            0.029688919738145608  2.2203734836400337e12  \\
            0.029787227419397745  1.6810850586730032e12  \\
            0.029885535100649882  1.2868564919771514e12  \\
            0.029983842781902015  9.951274205710504e11  \\
            0.030082150463154152  7.76797197347261e11  \\
            0.030180458144406293  3.2588491326882107e22  \\
            0.03027876582565843  1.0844787366133443e19  \\
            0.030377073506910567  3.311618172337213e18  \\
            0.030475381188162703  1.0398171694034584e17  \\
            0.030573688869414844  2.91510302453079e16  \\
            0.03067199655066698  1.1594099262441772e16  \\
            0.030770304231919118  6.138085796085018e15  \\
            0.030868611913171255  4.494528112828064e15  \\
            0.030966919594423395  1.0e23  \\
            0.031065227275675532  2.9336654325903255e15  \\
            0.03116353495692767  5.839508325701066e14  \\
            0.031261842638179806  2.089995384249182e14  \\
            0.03136015031943194  9.234982984065325e13  \\
            0.031458458000684086  5.679471312892871e17  \\
            0.03155676568193622  6.12547936182111e15  \\
            0.03165507336318836  4.64832525031075e14  \\
            0.0317533810444405  7.552367286050195e13  \\
            0.031851688725692634  1.847395750701505e13  \\
            0.03194999640694477  5.837867297691228e12  \\
            0.03204830408819691  2.326599799916077e12  \\
            0.03214661176944905  1.5876273305389885e12  \\
            0.03224491945070119  1.1067239000482603e12  \\
            0.032343227131953325  7.862141255182139e11  \\
            0.03244153481320546  5.680397085048728e11  \\
            0.0325398424944576  4.166944267765992e11  \\
            0.032638150175709736  3.0990806063380493e11  \\
            0.03273645785696187  2.3339292567944724e11  \\
            0.03283476553821402  4.276767927503482e17  \\
            0.032933073219466154  3.886287459360114e14  \\
            0.03303138090071829  1.8269433498693094e13  \\
            0.03312968858197043  2.554551020236907e12  \\
            0.033227996263222564  6.001721114535927e11  \\
            0.0333263039444747  1.9108985920062653e11  \\
            0.03342461162572684  7.430053107800029e10  \\
            0.033522919306978975  3.414240916165258e10  \\
            0.03362122698823112  2.778108764967572e10  \\
            0.033719534669483256  2.274190234999395e10  \\
            0.03381784235073539  1.87232306365978e10  \\
            0.03391615003198753  1.5498160649635115e10  \\
            0.03401445771323967  3.4549048064578094e10  \\
            0.034112765394491804  1.5040693894760043e11  \\
            0.03421107307574394  1.1418057739815361e12  \\
            0.034309380756996084  3.004217181785387e13  \\
            0.03440768843824822  2.567256920927091e17  \\
            0.03450599611950036  2.583946411509543e10  \\
            0.034604303800752495  4.1955520958058655e10  \\
            0.03470261148200463  7.070997828061674e10  \\
            0.03480091916325677  1.0e23  \\
            0.034899226844508906  4.459118807866969e16  \\
            0.03499753452576105  1.0e23  \\
            0.035095842207013186  2.0332266071961277e17  \\
            0.03519414988826532  3.6970600903354e15  \\
            0.03529245756951746  3.321990996348948e14  \\
            0.0353907652507696  5.8617351334008195e13  \\
            0.03548907293202173  8.213725197148625e13  \\
            0.03558738061327387  5.201113699112242e14  \\
            0.03568568829452601  7.221003875791261e15  \\
            0.035783995975778145  7.799773703987606e17  \\
            0.03588230365703028  1.4666318270731794e12  \\
            0.03598061133828242  5.833295814754461e12  \\
            0.036078919019534555  1.9997191518304788e15  \\
            0.03617722670078669  3.0421390012891758e13  \\
            0.036275534382038836  8.985574011103178e16  \\
            0.03637384206329097  1.0490186213687922e14  \\
            0.03647214974454311  4.0801046072860056e14  \\
            0.03657045742579525  3.2610498246380455e15  \\
            0.036668765107047384  1.275264661474969e17  \\
            0.03676707278829952  1.0e23  \\
            0.03686538046955166  3.8959795386628726e18  \\
            0.0369636881508038  5.305282293206245e15  \\
            0.03706199583205594  2.633860960450623e14  \\
            0.037160303513308075  1.8669254689241232e15  \\
            0.03725861119456021  5.166967711585438e16  \\
            0.03735691887581235  9.289357826186831e19  \\
            0.037455226557064486  1.2879801461550218e22  \\
            0.03755353423831662  4.907071770282916e11  \\
            0.03765184191956877  2.4435495870679333e11  \\
            0.037750149600820904  1.6929121398454117e11  \\
            0.03784845728207304  3.730730277073617e12  \\
            0.03794676496332518  1.1363512612562628e16  \\
            0.038045072644577314  2.964388485485814e10  \\
            0.03814338032582945  1.966120333763515e10  \\
            0.03824168800708159  1.349587374099759e10  \\
            0.038339995688333725  1.1580006417402634e10  \\
            0.03843830336958587  1.254550274197848e10  \\
            0.038536611050838006  1.360520532849136e10  \\
            0.03863491873209014  1.4769635669441639e10  \\
            0.03873322641334228  1.6050643829821377e10  \\
            0.038831534094594417  1.746159641249947e10  \\
            0.03892984177584655  2.8186514446346786e10  \\
            0.03902814945709869  4.759729476585269e10  \\
            0.039126457138350834  8.385732443125015e10  \\
            0.03922476481960297  1.552883058718664e11  \\
            0.03932307250085511  2.077113404720929e12  \\
            0.039421380182107245  4.4091941816173336e16  \\
            0.03951968786335938  1.494389669449172e12  \\
            0.03961799554461152  3.8887289140061597e12  \\
            0.039716303225863656  1.1822280948515203e13  \\
            0.0398146109071158  4.465418699216239e13  \\
            0.039912918588367936  2.331463174453039e14  \\
            0.04001122626962007  2.0771884381541515e15  \\
            0.04010953395087221  5.297448625114868e16  \\
            0.04020784163212435  3.0024125529904198e19  \\
            0.040306149313376484  5.8393560217263484e13  \\
            0.04040445699462862  3.3017616927059225e14  \\
            0.040502764675880765  3.664804969123984e15  \\
            0.0406010723571329  2.0094304763976502e17  \\
            0.04069938003838504  1.0e23  \\
            0.040797687719637175  1.245724195223783e11  \\
            0.040895995400889305  1.3980385093598145e11  \\
            0.04099430308214144  1.5722202224635062e11  \\
            0.041092610763393586  2.9585203193728974e22  \\
            0.04119091844464572  4.4100795550290664e16  \\
            0.04128922612589786  1.6082901147506652e15  \\
            0.04138753380715  1.8718653588483697e14  \\
            0.041485841488402134  3.775049560744287e13  \\
            0.04158414916965427  1.0508096064879406e13  \\
            0.04168245685090641  3.6163387261969736e12  \\
            0.04178076453215855  1.4465517890744248e12  \\
            0.04187907221341069  6.47362602045106e11  \\
            0.041977379894662825  2.269041460509271e19  \\
            0.04207568757591496  2.5395428933800605e17  \\
            0.0421739952571671  9.434027837376844e15  \\
            0.042272302938419236  5.307353715387197e14  \\
            0.04237061061967137  7.808090363015067e13  \\
            0.04246891830092352  1.7891900472430594e13  \\
            0.042567225982175654  8.213854309303099e12  \\
            0.04266553366342779  2.7845821415959117e13  \\
            0.04276384134467993  1.280061309128972e20  \\
            0.042862149025932064  6.327576457194482e16  \\
            0.0429604567071842  1.8306633083424188e16  \\
            0.04305876438843634  7.5452019044665e18  \\
            0.043157072069688475  4.585032618382154e13  \\
            0.04325537975094062  1.2241012042967582e13  \\
            0.043353687432192756  6.572161852329125e12  \\
            0.04345199511344489  8.140189578128338e12  \\
            0.04355030279469703  1.0157007288024652e13  \\
            0.043648610475949166  1.2774239978678809e13  \\
            0.0437469181572013  1.620325045694557e13  \\
            0.04384522583845344  2.074233241049191e13  \\
            0.043943533519705584  2.6818078389337707e13  \\
            0.04404184120095772  3.5049395311822496e13  \\
            0.04414014888220986  4.634840877755559e13  \\
            0.044238456563461995  6.2081689836676734e13  \\
            0.04433676424471413  8.433674049951233e13  \\
            0.04443507192596627  1.1636569795704511e14  \\
            0.044533379607218405  1.633512754188893e14  \\
            0.04463168728847055  2.3376092481180822e14  \\
            0.044729994969722686  3.4181713963216444e14  \\
            0.04482830265097482  5.121644229219434e14  \\
            0.04492661033222696  7.890339982777026e14  \\
            0.0450249180134791  1.2550584871386342e15  \\
            0.045123225694731234  2.0718947277019985e15  \\
            0.04522153337598337  3.573225267501516e15  \\
            0.045319841057235515  6.492759939173434e15  \\
            0.04541814873848765  1.2570656994183066e16  \\
            0.04551645641973979  2.6332935205828224e16  \\
            0.045614764100991925  6.0992636337865064e16  \\
            0.04571307178224406  1.613011281122985e17  \\
            0.0458113794634962  5.122863251439487e17  \\
            0.045909687144748336  2.130438166226128e18  \\
            0.04600799482600047  1.3719324372929552e19  \\
            0.04610630250725262  2.036186751503583e20  \\
            0.046204610188504754  2.9978052480998203e22  \\
            0.04630291786975689  2.894937810573429e16  \\
            0.04640122555100902  4.820593070836898e15  \\
            0.04649953323226116  1.2579509767916708e15  \\
            0.0465978409135133  1.7689112516170415e15  \\
            0.04669614859476544  4.2317581737251144e16  \\
            0.046794456276017575  7.730841696828455e19  \\
            0.04689276395726971  4.5444250603959e13  \\
            0.04699107163852185  2.6212968720218098e13  \\
            0.047089379319773986  1.6226776465356627e13  \\
            0.04718768700102612  1.0676751326463521e13  \\
            0.047285994682278266  7.420063860998338e12  \\
            0.0473843023635304  9.965997642236e12  \\
            0.04748261004478254  7.068993158455188e13  \\
            0.04758091772603468  1.4777940585117772e15  \\
            0.047679225407286814  6.269701182509518e21  \\
            0.04777753308853895  4.967358922516956e16  \\
            0.04787584076979109  1.4051518299813685e15  \\
            0.047974148451043225  1.7033463406473588e14  \\
            0.04807245613229537  3.940762042151487e13  \\
            0.048170763813547506  1.9643226876305772e14  \\
            0.04826907149479964  1.7069983931890895e15  \\
            0.04836737917605178  4.72482796788019e16  \\
            0.048465686857303916  1.0167556657039214e20  \\
            0.04856399453855605  2.0666569402286587e12  \\
            0.04866230221980819  8.999024478996785e12  \\
            0.048760609901060334  8.588075228518754e11  \\
            0.04885891758231247  2.3845801304402756e11  \\
            0.04895722526356461  9.691786277071999e10  \\
            0.049055532944816745  4.639960117645048e10  \\
        }
        ;
    \addlegendentry { $1.0 \cdot 10^{1}$, $ 0.0 \cdot 10^{0} $, $ 0.0 \cdot 10^{0} $ }
\end{axis}
\end{tikzpicture}

        \caption{Evolution of condition number}
        \label{subfig:cond}
    \end{subfigure}
    \hfill
    \begin{subfigure}{0.49\textwidth}
        \centering
        % Recommended preamble:
% \usetikzlibrary{arrows.meta}
% \usetikzlibrary{backgrounds}
% \usepgfplotslibrary{patchplots}
% \usepgfplotslibrary{fillbetween}
% \pgfplotsset{%
%     layers/standard/.define layer set={%
%         background,axis background,axis grid,axis ticks,axis lines,axis tick labels,pre main,main,axis descriptions,axis foreground%
%     }{
%         grid style={/pgfplots/on layer=axis grid},%
%         tick style={/pgfplots/on layer=axis ticks},%
%         axis line style={/pgfplots/on layer=axis lines},%
%         label style={/pgfplots/on layer=axis descriptions},%
%         legend style={/pgfplots/on layer=axis descriptions},%
%         title style={/pgfplots/on layer=axis descriptions},%
%         colorbar style={/pgfplots/on layer=axis descriptions},%
%         ticklabel style={/pgfplots/on layer=axis tick labels},%
%         axis background@ style={/pgfplots/on layer=axis background},%
%         3d box foreground style={/pgfplots/on layer=axis foreground},%
%     },
% }

\begin{tikzpicture}[/tikz/background rectangle/.style={fill={rgb,1:red,1.0;green,1.0;blue,1.0}, fill opacity={1.0}, draw opacity={1.0}}, show background rectangle]
\begin{axis}[point meta max={nan}, point meta min={nan}, legend cell align={left}, legend columns={1}, title={}, title style={at={{(0.5,1)}}, anchor={south}, font={{\fontsize{14 pt}{18.2 pt}\selectfont}}, color={rgb,1:red,0.0;green,0.0;blue,0.0}, draw opacity={1.0}, rotate={0.0}, align={center}}, legend style={color={rgb,1:red,0.0;green,0.0;blue,0.0}, draw opacity={1.0}, line width={1}, solid, fill={rgb,1:red,1.0;green,1.0;blue,1.0}, fill opacity={1.0}, text opacity={1.0}, font={{\fontsize{8 pt}{10.4 pt}\selectfont}}, text={rgb,1:red,0.0;green,0.0;blue,0.0}, cells={anchor={center}}, at={(1.02, 1)}, anchor={north west}}, axis background/.style={fill={rgb,1:red,1.0;green,1.0;blue,1.0}, opacity={1.0}}, anchor={north west}, xshift={1.0mm}, yshift={-1.0mm}, width={94.6mm}, height={74.2mm}, scaled x ticks={false}, xlabel={$\delta$}, x tick style={color={rgb,1:red,0.0;green,0.0;blue,0.0}, opacity={1.0}}, x tick label style={color={rgb,1:red,0.0;green,0.0;blue,0.0}, opacity={1.0}, rotate={0}}, xlabel style={at={(ticklabel cs:0.5)}, anchor=near ticklabel, at={{(ticklabel cs:0.5)}}, anchor={near ticklabel}, font={{\fontsize{11 pt}{14.3 pt}\selectfont}}, color={rgb,1:red,0.0;green,0.0;blue,0.0}, draw opacity={1.0}, rotate={0.0}}, xmajorgrids={true}, xmin={-0.002943331976689008}, xmax={0.1010543978663225}, xticklabels={{$0.000$,$0.025$,$0.050$,$0.075$,$0.100$}}, xtick={{0.0,0.025000000000000005,0.05000000000000001,0.07500000000000001,0.10000000000000002}}, xtick align={inside}, xticklabel style={font={{\fontsize{8 pt}{10.4 pt}\selectfont}}, color={rgb,1:red,0.0;green,0.0;blue,0.0}, draw opacity={1.0}, rotate={0.0}}, x grid style={color={rgb,1:red,0.0;green,0.0;blue,0.0}, draw opacity={0.1}, line width={0.5}, solid}, axis x line*={left}, x axis line style={color={rgb,1:red,0.0;green,0.0;blue,0.0}, draw opacity={1.0}, line width={1}, solid}, scaled y ticks={false}, ylabel={$\Vert e \Vert_{L^2,solid} $, $\Vert e \Vert_{H^1,dash} $, $\Vert e \Vert_{ah,*,dot}$}, y tick style={color={rgb,1:red,0.0;green,0.0;blue,0.0}, opacity={1.0}}, y tick label style={color={rgb,1:red,0.0;green,0.0;blue,0.0}, opacity={1.0}, rotate={0}}, ylabel style={at={(ticklabel cs:0.5)}, anchor=near ticklabel, at={{(ticklabel cs:0.5)}}, anchor={near ticklabel}, font={{\fontsize{11 pt}{14.3 pt}\selectfont}}, color={rgb,1:red,0.0;green,0.0;blue,0.0}, draw opacity={1.0}, rotate={0.0}}, ymode={log}, log basis y={10}, ymajorgrids={true}, ymin={2.011317062932734}, ymax={3.4299490944969575e6}, yticklabels={{$10^{3}$,$10^{6}$}}, ytick={{1000.0,1.0e6}}, ytick align={inside}, yticklabel style={font={{\fontsize{8 pt}{10.4 pt}\selectfont}}, color={rgb,1:red,0.0;green,0.0;blue,0.0}, draw opacity={1.0}, rotate={0.0}}, y grid style={color={rgb,1:red,0.0;green,0.0;blue,0.0}, draw opacity={0.1}, line width={0.5}, solid}, axis y line*={left}, y axis line style={color={rgb,1:red,0.0;green,0.0;blue,0.0}, draw opacity={1.0}, line width={1}, solid}, colorbar={false}]
    [\addlegendimage{empty legend}] \addlegendentry[font={{\fontsize{11 pt}{14.3 pt}\selectfont}}, text={rgb,1:red,0.0;green,0.0;blue,0.0}] {\hspace{-.6cm}{\textbf{$(\gamma, \gamma_1, \gamma_2)$}}}
    \addplot[color={rgb,1:red,0.0;green,0.0;blue,1.0}, name path={f2ccc329-9100-48c1-ab30-d41b64eca336}, draw opacity={1.0}, line width={1}, solid]
        table[row sep={\\}]
        {
            \\
            0.0  3.6963996705475806  \\
            9.820927516479829e-5  3.696402066720494  \\
            0.00019641855032959658  3.696419816803656  \\
            0.00029462782549439484  3.6964529206353767  \\
            0.00039283710065919316  3.696501381622916  \\
            0.0004910463758239915  3.6965652062797965  \\
            0.0005892556509887897  3.6500125482683368  \\
            0.000687464926153588  3.650166327544067  \\
            0.0007856742013183863  3.6503371920350394  \\
            0.0008838834764831845  3.65052519990309  \\
            0.000982092751647983  3.6507304132139744  \\
            0.0010803020268127812  3.6509529003600787  \\
            0.0011785113019775794  3.6511927344925272  \\
            0.0012767205771423778  3.6514499946744756  \\
            0.001374929852307176  3.651724766407898  \\
            0.0014731391274719742  3.652017139465607  \\
            0.0015713484026367726  3.652327210127063  \\
            0.001669557677801571  3.6526550800521083  \\
            0.001767766952966369  3.653000857288079  \\
            0.0018659762281311677  3.6533646538116455  \\
            0.001964185503295966  3.653746589936994  \\
            0.002062394778460764  3.654146789108943  \\
            0.0021606040536255623  3.6545653825407776  \\
            0.002258813328790361  3.6550025060392497  \\
            0.0023570226039551587  3.6554583021839537  \\
            0.0024552318791199574  3.655932917592506  \\
            0.0025534411542847556  3.6564265070440505  \\
            0.002651650429449554  3.6569392298866616  \\
            0.002749859704614352  3.6574712518251875  \\
            0.0028480689797791506  3.6122753748627954  \\
            0.0029462782549439484  3.6128524567736897  \\
            0.003044487530108747  3.613448514710366  \\
            0.0031426968052735453  3.6140637384603793  \\
            0.0032409060804383435  3.6146983230111873  \\
            0.003339115355603142  3.6153524708790066  \\
            0.0034373246307679403  3.616026387614518  \\
            0.003535533905932738  3.6167202867842145  \\
            0.0036337431810975363  3.617434387672148  \\
            0.0037319524562623354  3.6181689140828586  \\
            0.003830161731427133  3.6189240977533568  \\
            0.003928371006591932  3.6197001742994663  \\
            0.004026580281756729  3.620497386111452  \\
            0.004124789556921528  3.6213159818473395  \\
            0.004222998832086326  3.6221562154419282  \\
            0.004321208107251125  3.6230183464837866  \\
            0.004419417382415923  3.623902641378967  \\
            0.004517626657580722  3.624809372276238  \\
            0.004615835932745519  3.6257388163743123  \\
            0.0047140452079103175  3.626691258396113  \\
            0.004812254483075116  3.627666987309999  \\
            0.004910463758239915  3.628666298773579  \\
            0.005008673033404713  3.629689495477417  \\
            0.005106882308569511  3.6307368844179204  \\
            0.005205091583734309  3.6318087802654855  \\
            0.005303300858899108  3.632905502918161  \\
            0.005401510134063906  3.6340273785694595  \\
            0.005499719409228704  3.6351747401609655  \\
            0.005597928684393502  3.636347925387064  \\
            0.005696137959558301  3.637547280907352  \\
            0.0057943472347230995  3.638773157089658  \\
            0.005892556509887897  3.6400259128682175  \\
            0.005990765785052695  3.6413059130742527  \\
            0.006088975060217494  3.642613527701189  \\
            0.0061871843353822915  3.6439491359211416  \\
            0.0062853936105470905  3.645313122088554  \\
            0.006383602885711889  3.646705878145418  \\
            0.006481812160876687  3.6481278036394955  \\
            0.006580021436041485  3.6495793046070943  \\
            0.006678230711206284  3.651060795317331  \\
            0.006776439986371082  3.652572696382837  \\
            0.006874649261535881  3.654115437269605  \\
            0.006972858536700678  3.655689455274475  \\
            0.007071067811865476  3.657295196007641  \\
            0.007169277087030275  3.658933111629455  \\
            0.007267486362195073  3.6606036662281394  \\
            0.007365695637359872  3.662307330808119  \\
            0.007463904912524671  3.6640445850842407  \\
            0.007562114187689468  3.6658159201843827  \\
            0.007660323462854266  3.667621835136838  \\
            0.0077585327380190645  3.6694628399944866  \\
            0.007856742013183864  3.6713394555864656  \\
            0.007954951288348661  3.6732522125590026  \\
            0.008053160563513458  3.6752016548690216  \\
            0.008151369838678257  3.677188335266767  \\
            0.008249579113843056  3.679212821669108  \\
            0.008347788389007854  3.681275693733284  \\
            0.008445997664172653  3.683377542731266  \\
            0.00854420693933745  3.6855189761255813  \\
            0.00864241621450225  3.687700614726762  \\
            0.008740625489667048  3.6899230936857976  \\
            0.008838834764831846  3.692187064813281  \\
            0.008937044039996645  3.694493195564516  \\
            0.009035253315161444  3.6968421720308453  \\
            0.009133462590326241  3.6992346955596114  \\
            0.009231671865491039  3.7016714893623277  \\
            0.009329881140655836  3.7041532933018133  \\
            0.009428090415820635  3.706680870155783  \\
            0.009526299690985434  3.709255002938411  \\
            0.009624508966150231  3.7118764975187903  \\
            0.00972271824131503  3.714546183805  \\
            0.00982092751647983  3.71726491634325  \\
            0.009919136791644627  3.7200335768185235  \\
            0.010017346066809426  3.7228530727315174  \\
            0.010115555341974223  3.72572434149107  \\
            0.010213764617139022  3.7286483513968633  \\
            0.010311973892303821  3.7316261019852313  \\
            0.010410183167468619  3.7346586262588195  \\
            0.010508392442633416  3.7377469941859736  \\
            0.010606601717798215  3.7408923120059376  \\
            0.010704810992963013  3.744095724494525  \\
            0.010803020268127812  3.747358420712445  \\
            0.010901229543292609  3.7506816316304996  \\
            0.010999438818457408  3.754066634485035  \\
            0.011097648093622207  3.7575147555596984  \\
            0.011195857368787004  3.7610273745479668  \\
            0.011294066643951804  3.7646059245487944  \\
            0.011392275919116603  3.7682518949016006  \\
            0.0114904851942814  3.771966838759085  \\
            0.011588694469446199  3.7757523729547824  \\
            0.011686903744610996  3.6943319884733494  \\
            0.011785113019775794  3.695944921648274  \\
            0.011883322294940593  3.697570097690434  \\
            0.01198153157010539  3.699207331523915  \\
            0.01207974084527019  3.700856436908767  \\
            0.012177950120434988  3.702517228048411  \\
            0.012276159395599787  3.704189516611838  \\
            0.012374368670764583  3.7058731134740306  \\
            0.012472577945929382  3.707567826870776  \\
            0.012570787221094181  3.7092734644797853  \\
            0.01266899649625898  3.7109898296732022  \\
            0.012767205771423778  3.712716728074127  \\
            0.012865415046588575  3.7144539589731926  \\
            0.012963624321753374  3.716201321941983  \\
            0.013061833596918171  3.717958613936086  \\
            0.01316004287208297  3.7197256286806595  \\
            0.01325825214724777  3.7215021574392004  \\
            0.013356461422412568  3.7232879903780165  \\
            0.013454670697577364  3.725082913245391  \\
            0.013552879972742163  3.726886709770136  \\
            0.013651089247906962  3.728699162143357  \\
            0.013749298523071761  3.7305200471472526  \\
            0.01384750779823656  3.732349142522541  \\
            0.013945717073401356  3.7341862185264545  \\
            0.014043926348566155  3.736031044833281  \\
            0.014142135623730952  3.7378833898733066  \\
            0.014240344898895752  3.7397430160826644  \\
            0.01433855417406055  3.74160968405327  \\
            0.014436763449225346  3.74348315173736  \\
            0.014534972724390145  3.7453631726322265  \\
            0.014633181999554944  3.747249499281914  \\
            0.014731391274719743  3.7491418782007107  \\
            0.014829600549884542  3.751040055530052  \\
            0.014927809825049342  3.752943771655598  \\
            0.015026019100214137  3.754852766836473  \\
            0.015124228375378936  3.7567667750840936  \\
            0.015222437650543735  3.7586855292388344  \\
            0.015320646925708533  3.760608758280047  \\
            0.015418856200873332  3.7625361872578074  \\
            0.015517065476038129  3.764467539594937  \\
            0.015615274751202926  3.7664025340009952  \\
            0.015713484026367727  3.768340886893432  \\
            0.015811693301532526  3.770282311989922  \\
            0.015909902576697322  3.77222651856588  \\
            0.01600811185186212  3.7741732125453997  \\
            0.016106321127026917  3.7761220998525697  \\
            0.016204530402191716  3.7780728791571816  \\
            0.016302739677356515  3.780025249162675  \\
            0.016400948952521314  3.7819789028917157  \\
            0.016499158227686113  3.7839335338753703  \\
            0.01659736750285091  3.7858888295350197  \\
            0.016695576778015708  3.787844475773336  \\
            0.016793786053180507  3.7898001560695405  \\
            0.016891995328345306  3.791755548853934  \\
            0.016990204603510105  3.793710332713457  \\
            0.0170884138786749  3.7956641817226306  \\
            0.0171866231538397  3.797616767486519  \\
            0.0172848324290045  3.799567758989769  \\
            0.017383041704169298  3.801516822402146  \\
            0.017481250979334097  3.8034636218927713  \\
            0.017579460254498892  3.8054078182724256  \\
            0.01767766952966369  3.807349071139333  \\
            0.01777587880482849  3.8092870369221874  \\
            0.01787408807999329  3.811221370547687  \\
            0.01797229735515809  3.8131517217333952  \\
            0.018070506630322888  3.8150777422715865  \\
            0.018168715905487683  3.8169990793723905  \\
            0.018266925180652482  3.818915376968884  \\
            0.01836513445581728  3.820826280944185  \\
            0.018463343730982077  3.822731430716029  \\
            0.018561553006146876  3.8246304664506665  \\
            0.018659762281311672  3.8265230266980854  \\
            0.01875797155647647  3.8284087461329546  \\
            0.01885618083164127  3.8302872608841727  \\
            0.01895439010680607  3.758207134419766  \\
            0.019052599381970868  3.7591582488909716  \\
            0.019150808657135664  3.7600980796065184  \\
            0.019249017932300463  3.7610264104304902  \\
            0.019347227207465262  3.761943026646227  \\
            0.01944543648263006  3.7628477144673584  \\
            0.01954364575779486  3.7637402602571948  \\
            0.01964185503295966  3.7646204543462995  \\
            0.019740064308124455  3.7654880850177594  \\
            0.019838273583289254  3.766342945796821  \\
            0.019936482858454053  3.767184827574292  \\
            0.020034692133618852  3.7680135255953005  \\
            0.02013290140878365  3.7688288351214276  \\
            0.020231110683948447  3.7696305542135664  \\
            0.020329319959113246  3.7704184812489983  \\
            0.020427529234278045  3.771192417594245  \\
            0.020525738509442844  3.7719521642637917  \\
            0.020623947784607643  3.7726975257589364  \\
            0.02072215705977244  3.773428308606842  \\
            0.020820366334937238  3.774144319777458  \\
            0.020918575610102033  3.7748453680124743  \\
            0.021016784885266832  3.7755312642567023  \\
            0.02111499416043163  3.776201821983514  \\
            0.02121320343559643  3.7136107199455854  \\
            0.021311412710761226  3.7137170477171404  \\
            0.021409621985926025  3.713806709961095  \\
            0.021507831261090824  3.713879556220752  \\
            0.021606040536255623  3.713935437613979  \\
            0.021704249811420422  3.71397420761706  \\
            0.021802459086585218  3.713995720725129  \\
            0.021900668361750017  3.713999832834659  \\
            0.021998877636914816  3.713986401001501  \\
            0.022097086912079615  3.713955285144616  \\
            0.022195296187244414  3.7139063423113208  \\
            0.022293505462409213  3.7138394344445587  \\
            0.02239171473757401  3.7137544201242996  \\
            0.022489924012738808  3.713651162115179  \\
            0.022588133287903607  3.7135295192399624  \\
            0.022686342563068406  3.7133893530424302  \\
            0.022784551838233205  3.713230522381515  \\
            0.022882761113398  3.713052884252429  \\
            0.0229809703885628  3.7128562961977716  \\
            0.0230791796637276  3.7126406117942814  \\
            0.023177388938892398  3.7124056829730927  \\
            0.023275598214057194  3.7121513565229867  \\
            0.023373807489221993  3.711877476065852  \\
            0.02347201676438679  3.7115838802842465  \\
            0.023570226039551587  3.926290218495534  \\
            0.023668435314716386  3.91235614239628  \\
            0.023766644589881186  3.8998994179111706  \\
            0.023864853865045985  3.888740885638077  \\
            0.02396306314021078  3.8787288484092497  \\
            0.02406127241537558  3.869734071452655  \\
            0.02415948169054038  3.86164583109151  \\
            0.024257690965705177  3.854368754712935  \\
            0.024355900240869977  3.847820276050744  \\
            0.024454109516034772  3.8419285852138225  \\
            0.024552318791199575  3.836630947449338  \\
            0.02465052806636437  3.831872323515013  \\
            0.024748737341529166  3.8276042398788817  \\
            0.02484694661669397  3.8237838411861578  \\
            0.024945155891858764  3.820373109454585  \\
            0.025043365167023567  3.817338201047044  \\
            0.025141574442188362  3.8146488945549817  \\
            0.025239783717353158  3.8122781196411144  \\
            0.02533799299251796  3.8102015627687353  \\
            0.025436202267682756  3.808397324445677  \\
            0.025534411542847555  3.8068456276425806  \\
            0.025632620818012354  3.805528572425791  \\
            0.02573083009317715  3.804429919702291  \\
            0.02582903936834195  3.803534901731979  \\
            0.025927248643506748  3.8028300687614265  \\
            0.026025457918671547  3.8023031435227606  \\
            0.026123667193836343  3.8019429021357296  \\
            0.026221876469001145  3.801739065468979  \\
            0.02632008574416594  3.8016822101554357  \\
            0.026418295019330736  3.801763682319355  \\
            0.02651650429449554  3.801975528897124  \\
            0.026614713569660334  3.802310434523212  \\
            0.026712922844825137  3.8027616610729473  \\
            0.026811132119989933  3.8033230034705254  \\
            0.026909341395154728  3.803988742053342  \\
            0.02700755067031953  3.8047536042687686  \\
            0.027105759945484326  3.805612728841403  \\
            0.02720396922064913  3.8065616336064503  \\
            0.027302178495813925  3.807596188223934  \\
            0.02740038777097872  3.808712586445895  \\
            0.027498597046143523  3.8099073239332495  \\
            0.027596806321308318  3.811177175216312  \\
            0.02769501559647312  3.812519174819228  \\
            0.027793224871637916  3.813930597258985  \\
            0.027891434146802712  3.8154089424893116  \\
            0.027989643421967515  3.8169519178075224  \\
            0.02808785269713231  3.81855742215614  \\
            0.02818606197229711  3.820223533804529  \\
            0.028284271247461905  3.821948495292395  \\
            0.028382480522626704  3.823730701470247  \\
            0.028480689797791503  3.8255686860390132  \\
            0.0285788990729563  3.8274611083396235  \\
            0.0286771083481211  3.8294067461059424  \\
            0.028775317623285897  3.8314044776887832  \\
            0.028873526898450692  3.8334532748536754  \\
            0.028971736173615495  3.8355521909942634  \\
            0.02906994544878029  3.837700349648897  \\
            0.029168154723945093  3.8398969325089984  \\
            0.02926636399910989  3.842141170290413  \\
            0.02936457327427469  3.844432331703373  \\
            0.029462782549439487  3.8467697098832305  \\
            0.029560991824604282  3.8491526124586506  \\
            0.029659201099769085  3.8515803529357147  \\
            0.02975741037493388  3.8540522372455617  \\
            0.029855619650098683  3.856567549217313  \\
            0.02995382892526348  3.8591255443600447  \\
            0.030052038200428274  3.8617254324862893  \\
            0.030150247475593077  3.864366368807636  \\
            0.030248456750757872  3.867047440697857  \\
            0.03034666602592267  3.869767653581353  \\
            0.03044487530108747  3.872525917796676  \\
            0.030543084576252266  3.875321035980671  \\
            0.030641293851417065  3.8781516886054193  \\
            0.030739503126581864  3.8810164177091013  \\
            0.030837712401746663  3.8839136160017853  \\
            0.03093592167691146  3.8868415066949176  \\
            0.031034130952076258  4.03915607626577  \\
            0.031132340227241057  4.0003105873355125  \\
            0.031230549502405853  3.968600283075027  \\
            0.03132875877757066  3.9423617532171575  \\
            0.031426968052735454  3.9204026412027932  \\
            0.03152517732790025  3.9018488521827686  \\
            0.03162338660306505  3.8860469666991158  \\
            0.03172159587822985  3.872500062461709  \\
            0.031819805153394644  3.8608244523763657  \\
            0.031918014428559446  3.850719823804491  \\
            0.03201622370372424  3.8419482062496906  \\
            0.03211443297888904  3.8343188904505277  \\
            0.03221264225405383  3.827677411183277  \\
            0.032310851529218636  3.8218973852983376  \\
            0.03240906080438343  3.8168743739905233  \\
            0.032507270079548234  3.812521200372335  \\
            0.03260547935471303  3.8087643498074883  \\
            0.032703688629877825  3.8055411575076517  \\
            0.03280189790504263  3.8027976073456986  \\
            0.03290010718020742  3.8004865862966772  \\
            0.032998316455372226  3.798566499303183  \\
            0.03309652573053702  3.7970001710826313  \\
            0.03319473500570182  3.795753968349051  \\
            0.03329294428086662  3.79479710742893  \\
            0.033391153556031415  3.794101114289763  \\
            0.03348936283119622  3.7936394031707397  \\
            0.03358757210636101  3.7933869583598407  \\
            0.03368578138152581  3.7933201019861444  \\
            0.03378399065669061  3.793416329566809  \\
            0.03388219993185541  3.793654202879379  \\
            0.03398040920702021  3.7940132885794147  \\
            0.034078618482185005  3.794474133700345  \\
            0.0341768277573498  3.795018263981175  \\
            0.0342750370325146  3.7956282078219816  \\
            0.0343732463076794  3.796287530620438  \\
            0.0344714555828442  3.7969808715718347  \\
            0.034569664858009  3.797693987641682  \\
            0.03466787413317379  3.7984137935432494  \\
            0.034766083408338595  3.7991283932529543  \\
            0.03486429268350339  3.799827106355086  \\
            0.03496250195866819  3.800500479074991  \\
            0.03506071123383299  3.8011402915488723  \\
            0.035158920508997785  3.8017395434669794  \\
            0.03525712978416259  3.802292437495044  \\
            0.03535533905932738  3.80279434538066  \\
            0.035453548334492185  3.803241766599839  \\
            0.03555175760965698  3.8036322766079333  \\
            0.035649966884821783  3.8039644742534136  \\
            0.03574817615998658  3.8042379145013987  \\
            0.035846385435151375  3.804453045458148  \\
            0.03594459471031618  3.804611142042193  \\
            0.03604280398548097  3.804714233621284  \\
            0.036141013260645775  3.8047650447620467  \\
            0.03623922253581057  3.804766924774754  \\
            0.03633743181097537  3.8047237915722527  \\
            0.03643564108614017  3.8046400691526583  \\
            0.036533850361304965  3.8045206395350086  \\
            0.03663205963646976  3.8043707922346948  \\
            0.03673026891163456  3.804196177801033  \\
            0.03682847818679936  3.8040027687818383  \\
            0.036926687461964154  3.8037968188649374  \\
            0.03702489673712895  3.803584824517596  \\
            0.03712310601229375  3.8033734906279753  \\
            0.03722131528745855  3.803169684009987  \\
            0.037319524562623343  3.8029803945561227  \\
            0.037417733837788146  3.802812686963141  \\
            0.03751594311295294  3.8026736350214123  \\
            0.037614152388117744  3.802570262954169  \\
            0.03771236166328254  3.8025094582643457  \\
            0.037810570938447335  3.8024978819354742  \\
            0.03790878021361214  3.8025418593178144  \\
            0.038006989488776934  3.8026472591431455  \\
            0.038105198763941736  3.802819358888086  \\
            0.03820340803910653  3.8030627075576144  \\
            0.03830161731427133  3.8033809788318935  \\
            0.03839982658943613  3.8037768325986914  \\
            0.038498035864600925  3.8042517934116034  \\
            0.03859624513976573  3.8048061528442148  \\
            0.038694454414930524  3.8054389062426464  \\
            0.038792663690095326  3.806147747052462  \\
            0.03889087296526012  3.8069291105444116  \\
            0.03898908224042492  3.807778289704194  \\
            0.03908729151558972  3.8086896110296853  \\
            0.039185500790754516  3.809656676320281  \\
            0.03928371006591932  3.810672652462106  \\
            0.039381919341084114  3.8117305980429053  \\
            0.03948012861624891  3.8128238107731076  \\
            0.03957833789141371  3.8139461585508907  \\
            0.03967654716657851  3.8150924159306854  \\
            0.03977475644174331  3.8162585692804494  \\
            0.039872965716908106  3.8174421583625593  \\
            0.0399711749920729  3.8186427555778137  \\
            0.040069384267237704  3.819862959077649  \\
            0.0401675935424025  3.8076952913239444  \\
            0.0402658028175673  3.807889619193471  \\
            0.0403640120927321  3.808080333423024  \\
            0.04046222136789689  3.808267423626999  \\
            0.040560430643061696  3.808450878178023  \\
            0.04065863991822649  3.808630687837657  \\
            0.040756849193391294  3.8088068403937196  \\
            0.04085505846855609  3.8089793307999065  \\
            0.040953267743720885  3.809148145947486  \\
            0.04105147701888569  3.8093132800035936  \\
            0.04114968629405048  3.809474725186581  \\
            0.041247895569215286  3.809632474534882  \\
            0.04134610484438008  3.809786522576015  \\
            0.04144431411954488  3.809936863045869  \\
            0.04154252339470968  3.810083491272029  \\
            0.041640732669874475  3.810226402581344  \\
            0.04173894194503927  3.8103655925313054  \\
            0.041837151220204066  3.810501059526128  \\
            0.04193536049536887  3.8106328005930714  \\
            0.042033569770533664  3.8107608141696385  \\
            0.04213177904569846  3.8108850991548766  \\
            0.04222998832086326  3.8110056553087657  \\
            0.04232819759602806  3.8111224826579884  \\
            0.04242640687119286  3.8112355824988633  \\
            0.042524616146357656  3.8113449566904376  \\
            0.04262282542152245  3.8114506079556434  \\
            0.042721034696687255  3.8115525382175286  \\
            0.04281924397185205  3.8116507529919788  \\
            0.04291745324701685  3.8117452560215743  \\
            0.04301566252218165  3.811836052068038  \\
            0.043113871797346444  3.811923148412692  \\
            0.043212081072511246  3.812006550720991  \\
            0.04331029034767604  3.8120862674913147  \\
            0.043408499622840845  3.8121623054526856  \\
            0.04350670889800564  3.8122346744059175  \\
            0.043604918173170436  3.8123033841264884  \\
            0.04370312744833524  3.812368445267129  \\
            0.043801336723500034  3.8124298675392256  \\
            0.043899545998664836  3.81248766363959  \\
            0.04399775527382963  3.8125418465597782  \\
            0.04409596454899443  3.8125924294378746  \\
            0.04419417382415923  3.8126394245157282  \\
            0.044292383099324026  3.812682848420405  \\
            0.04439059237448883  3.8127227149105885  \\
            0.044488801649653624  3.8127590414393446  \\
            0.04458701092481843  3.8127918434545274  \\
            0.04468522019998322  3.8128211390713638  \\
            0.04478342947514802  3.812846946843113  \\
            0.04488163875031282  3.8128692835332996  \\
            0.044979848025477616  3.8128881704154702  \\
            0.04507805730064242  3.812903626815264  \\
            0.045176266575807214  3.8129156728770175  \\
            0.04527447585097201  3.8129243303698486  \\
            0.04537268512613681  3.8129296208022048  \\
            0.04547089440130161  3.8129315671628263  \\
            0.04556910367646641  3.8129301911984985  \\
            0.045667312951631206  3.8129255190249594  \\
            0.045765522226796  3.8129175717522563  \\
            0.045863731501960804  3.812906375899057  \\
            0.0459619407771256  3.8128919565149983  \\
            0.0460601500522904  3.8128743383749746  \\
            0.0461583593274552  3.8128535479259575  \\
            0.046256568602619993  3.81282961223281  \\
            0.046354777877784796  3.812802558152178  \\
            0.04645298715294959  3.812772413134874  \\
            0.04655119642811439  3.8127392057906753  \\
            0.04664940570327918  3.812702963154424  \\
            0.046747614978443985  3.812663715352052  \\
            0.04684582425360878  3.812621489338953  \\
            0.04694403352877358  3.812576315817581  \\
            0.04704224280393838  3.812528224705617  \\
            0.047140452079103175  3.812477244671422  \\
            0.04723866135426797  3.812423406926914  \\
            0.04733687062943277  3.8123667414851075  \\
            0.04743507990459757  3.812307278194637  \\
            0.04753328917976237  3.812245048553583  \\
            0.04763149845492717  3.8121800830527572  \\
            0.04772970773009197  3.8121124122411465  \\
            0.047827917005256765  3.812042067824863  \\
            0.04792612628042156  3.811969081292005  \\
            0.04802433555558636  3.811893481381781  \\
            0.04812254483075116  3.8118153016441636  \\
            0.04822075410591596  3.811734571978374  \\
            0.04831896338108076  3.811651324115274  \\
            0.04841717265624555  3.8115655884354105  \\
            0.048515381931410355  3.811477394571019  \\
            0.04861359120657515  3.8113867752999857  \\
            0.04871180048173995  3.8112937588142355  \\
            0.04881000975690475  3.811198377963975  \\
            0.048908219032069544  3.811100659777515  \\
            0.04900642830723435  3.811000635267051  \\
            0.04910463758239915  3.8108983348479497  \\
            0.049202846857563945  3.8107937856316085  \\
            0.04930105613272874  3.8106870174882888  \\
            0.049399265407893536  3.8105780573446677  \\
            0.04949747468305833  3.8104669344192095  \\
            0.04959568395822314  3.81035367451887  \\
            0.04969389323338794  3.810238306136128  \\
            0.04979210250855273  3.810120853605453  \\
            0.04989031178371753  3.8100013439833953  \\
            0.049988521058882324  3.80987980271604  \\
            0.05008673033404713  3.8097562518488832  \\
            0.05018493960921193  3.809630717862221  \\
            0.050283148884376724  3.8095032223223733  \\
            0.05038135815954152  3.809373788522554  \\
            0.050479567434706316  3.8092424383182104  \\
            0.050577776709871125  3.8091091909338335  \\
            0.05067598598503592  3.808974068688148  \\
            0.050774195260200716  3.8088370901107855  \\
            0.05087240453536551  3.808698271580189  \\
            0.05097061381053031  3.8085576337474083  \\
            0.05106882308569511  3.808415191972846  \\
            0.05116703236085991  3.808270961922823  \\
            0.05126524163602471  3.808124958100967  \\
            0.051363450911189504  3.807977195568831  \\
            0.0514616601863543  3.807827686288143  \\
            0.0515598694615191  3.80767644291523  \\
            0.0516580787366839  3.8075234758217498  \\
            0.05175628801184869  3.807368795220429  \\
            0.051854497287013496  3.8072124096466653  \\
            0.05195270656217829  3.807054326403066  \\
            0.052050915837343094  3.806894553843744  \\
            0.05214912511250789  3.8067330948010163  \\
            0.052247334387672685  3.8065699575267296  \\
            0.05234554366283748  3.8064051419149094  \\
            0.05244375293800229  3.806238652133148  \\
            0.052541962213167086  3.8060704891401134  \\
            0.05264017148833188  3.8059006520804477  \\
            0.05273838076349668  3.805729141466257  \\
            0.05283659003866147  3.8055559531811576  \\
            0.05293479931382628  3.8053810850637215  \\
            0.05303300858899108  3.805204532202107  \\
            0.05313121786415587  3.805026289412584  \\
            0.05322942713932067  3.8048463484961705  \\
            0.053327636414485464  3.8046647029155776  \\
            0.053425845689650274  3.8044813428515365  \\
            0.05352405496481507  3.8042962583008713  \\
            0.053622264239979865  3.804109436872626  \\
            0.05372047351514466  3.803920866990636  \\
            0.053818682790309456  3.803730534995634  \\
            0.053916892065474266  3.803538426434829  \\
            0.05401510134063906  3.803344524541021  \\
            0.05411331061580386  3.8031488132870335  \\
            0.05421151989096865  3.802951273285316  \\
            0.05430972916613345  3.8027518866577017  \\
            0.05440793844129826  3.802550633533956  \\
            0.05450614771646305  3.8023474911356394  \\
            0.05460435699162785  3.802142438443598  \\
            0.054702566266792645  3.8019354520348982  \\
            0.05480077554195744  3.801726507513307  \\
            0.05489898481712225  3.801515579926755  \\
            0.054997194092287045  3.801302643260522  \\
            0.05509540336745184  3.8010876705054852  \\
            0.055193612642616637  3.8008706327331714  \\
            0.05529182191778143  3.800651502513849  \\
            0.05539003119294624  3.8004302502566714  \\
            0.05548824046811104  3.8002068434917806  \\
            0.05558644974327583  3.79998125255966  \\
            0.05568465901844063  3.799753445144535  \\
            0.055782868293605424  3.7995233880416848  \\
            0.05588107756877023  3.7992910475221007  \\
            0.05597928684393503  3.7990563891212434  \\
            0.056077496119099825  3.7988193780060744  \\
            0.05617570539426462  3.7985799779362983  \\
            0.056273914669429416  3.798338154271999  \\
            0.05637212394459422  3.7980938672803117  \\
            0.056470333219759014  3.7978470811589222  \\
            0.05656854249492381  3.7975977577324405  \\
            0.05666675177008861  3.797345857318023  \\
            0.05676496104525341  3.7970913412666967  \\
            0.05686317032041821  3.796834170385612  \\
            0.056961379595583006  3.796574304021125  \\
            0.0570595888707478  3.79631170065402  \\
            0.0571577981459126  3.796046320670017  \\
            0.0572560074210774  3.795778122065863  \\
            0.0573542166962422  3.7955070620462066  \\
            0.057452425971407  3.7952330998516923  \\
            0.057550635246571794  3.794956193370587  \\
            0.05764884452173659  3.79467629871032  \\
            0.057747053796901385  3.7943933728917814  \\
            0.057845263072066194  3.7941073738059146  \\
            0.05794347234723099  3.793818257107937  \\
            0.058041681622395785  3.793525978224091  \\
            0.05813989089756058  3.793230494767113  \\
            0.05823810017272538  3.7929317624926733  \\
            0.058336309447890186  3.792629736608944  \\
            0.05843451872305498  3.7923243729961906  \\
            0.05853272799821978  3.7920156266711396  \\
            0.05863093727338457  3.7917034541178865  \\
            0.05872914654854938  3.791387809833502  \\
            0.05882735582371418  3.7910686499150383  \\
            0.058925565098878974  3.7907459290519565  \\
            0.05902377437404377  3.7904196027438117  \\
            0.059121983649208565  3.7900896266660262  \\
            0.059220192924373374  3.789755955112405  \\
            0.05931840219953817  3.789418543359302  \\
            0.059416611474702966  3.7890773470078845  \\
            0.05951482074986776  3.788732320503345  \\
            0.05961303002503256  3.7883834200247692  \\
            0.059711239300197366  3.7880305997571693  \\
            0.05980944857536216  3.787673816440659  \\
            0.05990765785052696  3.787313023480347  \\
            0.06000586712569175  3.7869481767355566  \\
            0.06010407640085655  3.786579232058567  \\
            0.06020228567602136  3.7862061436581578  \\
            0.060300494951186154  3.7858288680111998  \\
            0.06039870422635095  3.7854473594588973  \\
            0.060496913501515745  3.7850615746611322  \\
            0.06059512277668054  3.784671466913016  \\
            0.06069333205184534  3.784276993056702  \\
            0.060791541327010146  3.783878107360286  \\
            0.06088975060217494  3.7834747661175894  \\
            0.06098795987733974  3.7830669230485805  \\
            0.06108616915250453  3.7826545333650627  \\
            0.061184378427669335  3.782237553793634  \\
            0.06128258770283413  3.781815937462697  \\
            0.061380796977998926  3.7673503424470267  \\
            0.06147900625316373  3.767496448276464  \\
            0.061577215528328524  3.7676376540302385  \\
            0.06167542480349333  3.7677738429808687  \\
            0.06177363407865812  3.7679049009782286  \\
            0.06187184335382292  3.768030718317033  \\
            0.061970052628987714  3.768151189530651  \\
            0.062068261904152516  3.768266213531391  \\
            0.06216647117931732  3.768375690134781  \\
            0.062264680454482114  3.768479522807864  \\
            0.06236288972964691  3.768577619937593  \\
            0.062461099004811706  3.7686698928134943  \\
            0.06255930827997651  3.7687562545737223  \\
            0.06265751755514132  3.7688366213488753  \\
            0.06275572683030611  3.768910913438366  \\
            0.06285393610547091  3.76897905134153  \\
            0.0629521453806357  3.7690409622742407  \\
            0.0630503546558005  3.769096572025594  \\
            0.0631485639309653  3.7691458115778893  \\
            0.0632467732061301  3.7691886136222013  \\
            0.0633449824812949  3.769224912640364  \\
            0.0634431917564597  3.76925464696253  \\
            0.06354140103162449  3.769277755806685  \\
            0.06363961030678929  3.7692941816341143  \\
            0.06373781958195408  3.769303869673816  \\
            0.06383602885711889  3.769306767226407  \\
            0.06393423813228369  3.76930282147919  \\
            0.06403244740744848  3.769291984792609  \\
            0.06413065668261328  3.769274210654148  \\
            0.06422886595777808  3.7692494541579564  \\
            0.06432707523294287  3.769217673453552  \\
            0.06442528450810767  3.7691788280040566  \\
            0.06452349378327248  3.7691328788458907  \\
            0.06462170305843727  3.769079791425068  \\
            0.06471991233360207  3.769019529389477  \\
            0.06481812160876686  3.7689520618169987  \\
            0.06491633088393166  3.768877358705657  \\
            0.06501454015909647  3.7687953901653817  \\
            0.06511274943426126  3.768706129425032  \\
            0.06521095870942606  3.7686095537662485  \\
            0.06530916798459085  3.768505638545897  \\
            0.06540737725975565  3.7683943648593026  \\
            0.06550558653492046  3.768275711700508  \\
            0.06560379581008526  3.7681496618147183  \\
            0.06570200508525005  3.768016199961507  \\
            0.06580021436041485  3.767875313569922  \\
            0.06589842363557964  3.7677269887240414  \\
            0.06599663291074445  3.767571216984864  \\
            0.06609484218590925  3.7674079888534986  \\
            0.06619305146107404  3.767237296790251  \\
            0.06629126073623884  3.767059137141834  \\
            0.06638947001140363  3.766873505909637  \\
            0.06648767928656844  3.766680401452658  \\
            0.06658588856173324  3.7664798232615815  \\
            0.06668409783689803  3.766271773526867  \\
            0.06678230711206283  3.766056255658009  \\
            0.06688051638722763  3.7658332736842364  \\
            0.06697872566239244  3.7656028339605863  \\
            0.06707693493755723  3.7653649449991926  \\
            0.06717514421272203  3.76511961571345  \\
            0.06727335348788682  3.7648668578079425  \\
            0.06737156276305162  3.7646066843746713  \\
            0.06746977203821643  3.7643391088788194  \\
            0.06756798131338122  3.764064147304599  \\
            0.06766619058854602  3.763781816575629  \\
            0.06776439986371081  3.7634921361258473  \\
            0.06786260913887561  3.7631951254169804  \\
            0.06796081841404042  3.7628908057485293  \\
            0.06805902768920521  3.7625792021182414  \\
            0.06815723696437001  3.7622603368309777  \\
            0.0682554462395348  3.7619342363223867  \\
            0.0683536555146996  3.761600927975571  \\
            0.06845186478986441  3.761260440976077  \\
            0.0685500740650292  3.760912805297341  \\
            0.068648283340194  3.7605580510664995  \\
            0.0687464926153588  3.7601962124225174  \\
            0.0688447018905236  3.7598273230221118  \\
            0.0689429111656884  3.7594514175743394  \\
            0.0690411204408532  3.759068533134323  \\
            0.069139329716018  3.7586787070005183  \\
            0.06923753899118279  3.758281977964696  \\
            0.06933574826634759  3.7578783876485975  \\
            0.0694339575415124  3.7574679754978706  \\
            0.06953216681667719  3.7570507859301747  \\
            0.06963037609184199  3.7566268608952336  \\
            0.06972858536700678  3.756196246377058  \\
            0.06982679464217158  3.755758988599863  \\
            0.06992500391733639  3.755315133867299  \\
            0.07002321319250118  3.7548647310947296  \\
            0.07012142246766598  3.7544078275641892  \\
            0.07021963174283077  3.753944475939866  \\
            0.07031784101799557  3.7534747257233794  \\
            0.07041605029316038  3.7529986292431743  \\
            0.07051425956832517  3.7525162400674343  \\
            0.07061246884348997  3.752027611929614  \\
            0.07071067811865477  3.7515328003282304  \\
            0.07080888739381958  3.751031859566326  \\
            0.07090709666898437  3.7505248479680713  \\
            0.07100530594414917  3.7500118223810692  \\
            0.07110351521931396  3.7494928406996166  \\
            0.07120172449447876  3.7489679620039484  \\
            0.07129993376964357  3.748437246765904  \\
            0.07139814304480836  3.7479007549272976  \\
            0.07149635231997316  3.7473585479684544  \\
            0.07159456159513795  3.7468106878515455  \\
            0.07169277087030275  3.7462572371460916  \\
            0.07179098014546756  3.745698258449732  \\
            0.07188918942063235  3.745133816229216  \\
            0.07198739869579715  3.7445639740649987  \\
            0.07208560797096195  3.743988797341134  \\
            0.07218381724612674  3.7434083510526603  \\
            0.07228202652129155  3.7428227011738433  \\
            0.07238023579645635  3.7422319136106346  \\
            0.07247844507162114  3.7416360552216106  \\
            0.07257665434678594  3.741035193561798  \\
            0.07267486362195073  3.740429394728972  \\
            0.07277307289711553  3.793824903635854  \\
            0.07287128217228034  3.794018319042166  \\
            0.07296949144744513  3.7942044727385738  \\
            0.07306770072260993  3.7943834205342624  \\
            0.07316590999777473  3.794555218987438  \\
            0.07326411927293952  3.7947199231178517  \\
            0.07336232854810432  3.794877589170649  \\
            0.07346053782326913  3.795028271657944  \\
            0.07355874709843392  3.7951720253848538  \\
            0.07365695637359872  3.7953089046008355  \\
            0.07375516564876351  3.7954389626388743  \\
            0.07385337492392831  3.795562255001585  \\
            0.0739515841990931  3.795678832987957  \\
            0.0740497934742579  3.7957887514536033  \\
            0.07414800274942271  3.7958920630976944  \\
            0.0742462120245875  3.7959888205515413  \\
            0.0743444212997523  3.796079076942057  \\
            0.0744426305749171  3.7961628842808737  \\
            0.07454083985008189  3.796240295598943  \\
            0.07463904912524669  3.796311362897589  \\
            0.0747372584004115  3.7963761385415884  \\
            0.07483546767557629  3.796434674112022  \\
            0.07493367695074109  3.7964870218518154  \\
            0.07503188622590588  3.796533234109267  \\
            0.07513009550107068  3.7965733621489397  \\
            0.07522830477623549  3.7966074564951695  \\
            0.07532651405140028  3.796635571014406  \\
            0.07542472332656508  3.796657756063311  \\
            0.07552293260172988  3.7966740626661077  \\
            0.07562114187689467  3.7966845435682015  \\
            0.07571935115205948  3.7966892495764992  \\
            0.07581756042722428  3.7966882324841444  \\
            0.07591576970238907  3.796681542867375  \\
            0.07601397897755387  3.796669232522585  \\
            0.07611218825271866  3.7966513533978397  \\
            0.07621039752788347  3.7966279550794777  \\
            0.07630860680304827  3.7965990897871538  \\
            0.07640681607821306  3.7965648088252872  \\
            0.07650502535337786  3.796525163642442  \\
            0.07660323462854265  3.796480204695497  \\
            0.07670144390370746  3.79642998245013  \\
            0.07679965317887226  3.796374549730651  \\
            0.07689786245403706  3.796313955995556  \\
            0.07699607172920185  3.7962482528313553  \\
            0.07709428100436666  3.7961774910884247  \\
            0.07719249027953146  3.7961017209088377  \\
            0.07729069955469625  3.7960209950723107  \\
            0.07738890882986105  3.795935361721527  \\
            0.07748711810502584  3.7958448727731984  \\
            0.07758532738019065  3.7957495783185404  \\
            0.07768353665535545  3.795649529760825  \\
            0.07778174593052024  3.7955447763704537  \\
            0.07787995520568504  3.795435368874672  \\
            0.07797816448084983  3.79532135733064  \\
            0.07807637375601464  3.795202790878274  \\
            0.07817458303117944  3.795079720515538  \\
            0.07827279230634424  3.7949521960461796  \\
            0.07837100158150903  3.7948202665875264  \\
            0.07846921085667383  3.7946839812412585  \\
            0.07856742013183864  3.794543389511869  \\
            0.07866562940700343  3.7943985402170464  \\
            0.07876383868216823  3.794249481444009  \\
            0.07886204795733302  3.794096264221334  \\
            0.07896025723249782  3.7939389345693413  \\
            0.07905846650766263  3.7937775409395424  \\
            0.07915667578282742  3.7936121321786413  \\
            0.07925488505799222  3.793442755964539  \\
            0.07935309433315701  3.793269457577273  \\
            0.07945130360832181  3.7930922871577804  \\
            0.07954951288348662  3.7929112885749316  \\
            0.07964772215865142  3.792726509914103  \\
            0.07974593143381621  3.7925379967861352  \\
            0.079844140708981  3.7923457951567205  \\
            0.0799423499841458  3.7921499493750064  \\
            0.08004055925931061  3.7919505061767644  \\
            0.08013876853447541  3.7917475077848644  \\
            0.0802369778096402  3.791541000924404  \\
            0.080335187084805  3.7913310279755144  \\
            0.0804333963599698  3.791117632095645  \\
            0.0805316056351346  3.7909008565446207  \\
            0.0806298149102994  3.790680743416127  \\
            0.0807280241854642  3.7904573343177934  \\
            0.08082623346062899  3.7902306714145744  \\
            0.08092444273579379  3.7900007955589534  \\
            0.0810226520109586  3.7897677462424095  \\
            0.08112086128612339  3.78953156398516  \\
            0.08121907056128819  3.789292287423725  \\
            0.08131727983645298  3.7890499552088843  \\
            0.08141548911161778  3.788804605276021  \\
            0.08151369838678259  3.7885562758899964  \\
            0.08161190766194738  3.7883050027323186  \\
            0.08171011693711218  3.788050822838811  \\
            0.08180832621227697  3.7877937708100604  \\
            0.08190653548744177  3.787533880951287  \\
            0.08200474476260658  3.7872711885794397  \\
            0.08210295403777138  3.787005726631898  \\
            0.08220116331293617  3.7867375272960504  \\
            0.08229937258810097  3.786466623125978  \\
            0.08239758186326576  3.7861930450116086  \\
            0.08249579113843057  3.785916823484699  \\
            0.08259400041359537  3.7856379870804546  \\
            0.08269220968876016  3.7853565653775787  \\
            0.08279041896392496  3.785072586232086  \\
            0.08288862823908975  3.78478607682623  \\
            0.08298683751425455  3.784497062729437  \\
            0.08308504678941936  3.784205570576193  \\
            0.08318325606458415  3.7839116243438835  \\
            0.08328146533974895  3.783615246485922  \\
            0.08337967461491375  3.7833164612150263  \\
            0.08347788389007854  3.7830152902371075  \\
            0.08357609316524334  3.7827117528315957  \\
            0.08367430244040813  3.782405870950318  \\
            0.08377251171557294  3.782097662907053  \\
            0.08387072099073774  3.781787145774999  \\
            0.08396893026590253  3.7814743375156445  \\
            0.08406713954106733  3.81334928660635  \\
            0.08416534881623212  3.81204621742392  \\
            0.08426355809139692  3.8107528357372575  \\
            0.08436176736656173  3.8094691814739603  \\
            0.08445997664172653  3.8081952906640035  \\
            0.08455818591689132  3.8069311995757564  \\
            0.08465639519205612  3.805676941066306  \\
            0.08475460446722091  3.8044325472792506  \\
            0.08485281374238572  3.8031980475459397  \\
            0.08495102301755052  3.801973470796147  \\
            0.08504923229271531  3.800758843489245  \\
            0.08514744156788011  3.7995541904740215  \\
            0.0852456508430449  3.7983595341769445  \\
            0.08534386011820971  3.7971748974762938  \\
            0.08544206939337451  3.79600029884249  \\
            0.0855402786685393  3.794835757668287  \\
            0.0856384879437041  3.793681289793013  \\
            0.0857366972188689  3.7925369116930603  \\
            0.0858349064940337  3.79140263622475  \\
            0.0859331157691985  3.7902784745487437  \\
            0.0860313250443633  3.7891644389629624  \\
            0.08612953431952809  3.7880605375254386  \\
            0.08622774359469289  3.7869667784610095  \\
            0.0863259528698577  3.785883167360995  \\
            0.08642416214502249  3.7848097097571674  \\
            0.08652237142018729  3.783746409584992  \\
            0.08662058069535208  3.782693267346157  \\
            0.08671878997051688  3.7816502855493788  \\
            0.08681699924568169  3.780617463381456  \\
            0.08691520852084648  3.7795947991411247  \\
            0.08701341779601128  3.7785822896396883  \\
            0.08711162707117608  3.77757993039854  \\
            0.08720983634634087  3.7765877172817293  \\
            0.08730804562150568  3.775605642995424  \\
            0.08740625489667048  3.7746336996741223  \\
            0.08750446417183527  3.7736718798203186  \\
            0.08760267344700007  3.772720173011014  \\
            0.08770088272216486  3.7717785685997134  \\
            0.08779909199732967  3.770847054952381  \\
            0.08789730127249447  3.7699256186043293  \\
            0.08799551054765926  3.76901424705101  \\
            0.08809371982282406  3.7681129245486473  \\
            0.08819192909798886  3.767221637101127  \\
            0.08829013837315366  3.766340366381488  \\
            0.08838834764831846  3.765469096202652  \\
            0.08848655692348326  3.7646078090693385  \\
            0.08858476619864805  3.7637564854208394  \\
            0.08868297547381285  3.762915105608194  \\
            0.08878118474897766  3.762083650374647  \\
            0.08887939402414245  3.761262097795945  \\
            0.08897760329930725  3.760450426614965  \\
            0.08907581257447204  3.7596486147783046  \\
            0.08917402184963685  3.7588566385336555  \\
            0.08927223112480165  3.758074475424334  \\
            0.08937044039996644  3.7573021012879204  \\
            0.08946864967513124  3.7565394921087467  \\
            0.08956685895029604  3.755786620968961  \\
            0.08966506822546084  3.7550434638823194  \\
            0.08976327750062564  3.7543099944785268  \\
            0.08986148677579044  3.7535861867601277  \\
            0.08995969605095523  3.7528720140223335  \\
            0.09005790532612003  3.7521674481711225  \\
            0.09015611460128484  3.7514724624806064  \\
            0.09025432387644963  3.750787028956151  \\
            0.09035253315161443  3.750111119320012  \\
            0.09045074242677922  3.7494447060858094  \\
            0.09054895170194402  3.7487877601193174  \\
            0.09064716097710883  3.748140251570756  \\
            0.09074537025227362  3.7475021525533445  \\
            0.09084357952743842  3.7468734336036706  \\
            0.09094178880260322  3.746254066008638  \\
            0.09103999807776801  3.745644019380591  \\
            0.09113820735293282  3.7450432642428004  \\
            0.09123641662809762  3.744451771318119  \\
            0.09133462590326241  3.7438695103081114  \\
            0.09143283517842721  3.7432964526459607  \\
            0.091531044453592  3.7427325664769246  \\
            0.09162925372875681  3.7421778244270203  \\
            0.09172746300392161  3.74163219500799  \\
            0.0918256722790864  3.741095648716926  \\
            0.0919238815542512  3.7405681571268707  \\
            0.092022090829416  3.7400496896539153  \\
            0.0921203001045808  3.739540217153786  \\
            0.0922185093797456  3.739039709044294  \\
            0.0923167186549104  3.7385481378009  \\
            0.09241492793007519  3.7380654739477626  \\
            0.09251313720523999  3.7375916874542257  \\
            0.09261134648040478  3.7371267512858264  \\
            0.09270955575556959  3.736670636388649  \\
            0.09280776503073439  3.736223313256372  \\
            0.09290597430589918  3.7357847555935346  \\
            0.09300418358106398  3.735354934919422  \\
            0.09310239285622877  3.7349338236045933  \\
            0.09320060213139357  3.7345213945659905  \\
            0.09329881140655837  3.734117620997536  \\
            0.09339702068172318  3.733722475624126  \\
            0.09349522995688797  3.7333359326350606  \\
            0.09359343923205277  3.7329579667728847  \\
            0.09369164850721756  3.7325885517340276  \\
            0.09378985778238236  3.7322276623234947  \\
            0.09388806705754715  3.7318752730867124  \\
            0.09398627633271196  3.7315313609777507  \\
            0.09408448560787676  3.7311959009609295  \\
            0.09418269488304155  3.7308688696020798  \\
            0.09428090415820635  3.730550243759775  \\
            0.09437911343337115  3.730240001447479  \\
            0.09447732270853594  3.729938118456667  \\
            0.09457553198370075  3.7296445741573785  \\
            0.09467374125886555  3.729359346995985  \\
            0.09477195053403034  3.7290824167249292  \\
            0.09487015980919514  3.72881376191761  \\
            0.09496836908435993  3.7285533634274484  \\
            0.09506657835952474  3.7283012008112766  \\
            0.09516478763468954  3.728057256321962  \\
            0.09526299690985433  3.7278215110931145  \\
            0.09536120618501913  3.72759394627364  \\
            0.09545941546018394  3.72737454741237  \\
            0.09555762473534873  3.727163294185624  \\
            0.09565583401051353  3.7269601723056263  \\
            0.09575404328567833  3.726765166023858  \\
            0.09585225256084312  3.726578259507901  \\
            0.09595046183600793  3.7263994399574067  \\
            0.09604867111117273  3.726228691022239  \\
            0.09614688038633752  3.726066000820974  \\
            0.09624508966150232  3.725911356828658  \\
            0.09634329893666711  3.7257647455811744  \\
            0.09644150821183192  3.7256261562444877  \\
            0.09653971748699672  3.7254955785457384  \\
            0.09663792676216151  3.7253730006748453  \\
            0.09673613603732631  3.725258413866294  \\
            0.0968343453124911  3.725151808416682  \\
            0.09693255458765591  3.725053175754069  \\
            0.09703076386282071  3.7249625085491336  \\
            0.0971289731379855  3.724879798875885  \\
            0.0972271824131503  3.7248050413637666  \\
            0.0973253916883151  3.724738227427261  \\
            0.0974236009634799  3.724679354188627  \\
            0.0975218102386447  3.724628415692569  \\
            0.0976200195138095  3.7245854077476945  \\
            0.09771822878897429  3.724550327342576  \\
            0.09781643806413909  3.7245231707997206  \\
            0.0979146473393039  3.7245039373587354  \\
            0.0980128566144687  3.724492623499154  \\
            0.09811106588963349  3.724489229310241  \\
        }
        ;
    \addlegendentry { $1.0 \cdot 10^{1}$, $0.5 \cdot 10^{1}$, $1.0 \cdot 10^{-1}$ }
    \addplot[color={rgb,1:red,0.0;green,0.0;blue,1.0}, name path={848a2180-fa84-4d11-8982-0a9431e480bd}, draw opacity={1.0}, line width={1}, dashed, forget plot]
        table[row sep={\\}]
        {
            \\
            0.0  31.49495679323927  \\
            9.820927516479829e-5  31.49497647957422  \\
            0.00019641855032959658  31.495002268301416  \\
            0.00029462782549439484  31.495034183134806  \\
            0.00039283710065919316  31.495072248730157  \\
            0.0004910463758239915  31.495116490658393  \\
            0.0005892556509887897  31.40596609834023  \\
            0.000687464926153588  31.406023065561275  \\
            0.0007856742013183863  31.4060880491607  \\
            0.0008838834764831845  31.40616113647205  \\
            0.000982092751647983  31.40624241587568  \\
            0.0010803020268127812  31.40633197888742  \\
            0.0011785113019775794  31.406429919267225  \\
            0.0012767205771423778  31.40653633310171  \\
            0.001374929852307176  31.406651319643817  \\
            0.0014731391274719742  31.406774979983524  \\
            0.0015713484026367726  31.406907418236564  \\
            0.001669557677801571  31.40704874071024  \\
            0.001767766952966369  31.407199056715424  \\
            0.0018659762281311677  31.407358477135745  \\
            0.001964185503295966  31.407527116573675  \\
            0.002062394778460764  31.407705090765784  \\
            0.0021606040536255623  31.407892519396995  \\
            0.002258813328790361  31.408089523618504  \\
            0.0023570226039551587  31.408296227444147  \\
            0.0024552318791199574  31.408512756769554  \\
            0.0025534411542847556  31.408739240836518  \\
            0.002651650429449554  31.408975810831738  \\
            0.002749859704614352  31.40922260051854  \\
            0.0028480689797791506  31.319346729593406  \\
            0.0029462782549439484  31.319599123277246  \\
            0.003044487530108747  31.319861660305598  \\
            0.0031426968052735453  31.320134484156217  \\
            0.0032409060804383435  31.32041774027622  \\
            0.003339115355603142  31.320711577153233  \\
            0.0034373246307679403  31.321016143885718  \\
            0.003535533905932738  31.32133159327013  \\
            0.0036337431810975363  31.32165807951261  \\
            0.0037319524562623354  31.321995758842053  \\
            0.003830161731427133  31.32234479039253  \\
            0.003928371006591932  31.32270533425934  \\
            0.004026580281756729  31.32307755333278  \\
            0.004124789556921528  31.32346161207933  \\
            0.004222998832086326  31.3238576775292  \\
            0.004321208107251125  31.324265918018106  \\
            0.004419417382415923  31.32468650440982  \\
            0.004517626657580722  31.32511960970199  \\
            0.004615835932745519  31.325565407922237  \\
            0.0047140452079103175  31.326024076622577  \\
            0.004812254483075116  31.326495793659497  \\
            0.004910463758239915  31.326980739584958  \\
            0.005008673033404713  31.32747909760929  \\
            0.005106882308569511  31.32799105152214  \\
            0.005205091583734309  31.32851678843051  \\
            0.005303300858899108  31.329056496875296  \\
            0.005401510134063906  31.329610367096848  \\
            0.005499719409228704  31.33017859216475  \\
            0.005597928684393502  31.330761366484275  \\
            0.005696137959558301  31.331358887356284  \\
            0.0057943472347230995  31.331971353658133  \\
            0.005892556509887897  31.332598966986108  \\
            0.005990765785052695  31.33324193109107  \\
            0.006088975060217494  31.333900451438282  \\
            0.0061871843353822915  31.334574736794657  \\
            0.0062853936105470905  31.335264997876372  \\
            0.006383602885711889  31.335971448101635  \\
            0.006481812160876687  31.336694304076723  \\
            0.006580021436041485  31.337433784314616  \\
            0.006678230711206284  31.33819011076701  \\
            0.006776439986371082  31.338963508272506  \\
            0.006874649261535881  31.33975420439715  \\
            0.006972858536700678  31.340562430396012  \\
            0.007071067811865476  31.341388421151155  \\
            0.007169277087030275  31.34223241350285  \\
            0.007267486362195073  31.343094650334685  \\
            0.007365695637359872  31.343975376903472  \\
            0.007463904912524671  31.34487484216382  \\
            0.007562114187689468  31.34579330062011  \\
            0.007660323462854266  31.346731010243804  \\
            0.0077585327380190645  31.347688233568807  \\
            0.007856742013183864  31.348665238731492  \\
            0.007954951288348661  31.34966229805576  \\
            0.008053160563513458  31.350679690349423  \\
            0.008151369838678257  31.35171769845384  \\
            0.008249579113843056  31.352776612747824  \\
            0.008347788389007854  31.353856729246267  \\
            0.008445997664172653  31.354958349645205  \\
            0.00854420693933745  31.356081783847102  \\
            0.00864241621450225  31.357227348326408  \\
            0.008740625489667048  31.358395366683457  \\
            0.008838834764831846  31.359586171080526  \\
            0.008937044039996645  31.360800101812956  \\
            0.009035253315161444  31.3620375085746  \\
            0.009133462590326241  31.363298749330053  \\
            0.009231671865491039  31.364584192820693  \\
            0.009329881140655836  31.365894217261438  \\
            0.009428090415820635  31.36722921222729  \\
            0.009526299690985434  31.368589578456717  \\
            0.009624508966150231  31.369975728669782  \\
            0.00972271824131503  31.37138808851585  \\
            0.00982092751647983  31.372827096607132  \\
            0.009919136791644627  31.374293206597624  \\
            0.010017346066809426  31.3757868854847  \\
            0.010115555341974223  31.377308617189495  \\
            0.010213764617139022  31.37885890175841  \\
            0.010311973892303821  31.380438256578806  \\
            0.010410183167468619  31.38204721753322  \\
            0.010508392442633416  31.383686340603465  \\
            0.010606601717798215  31.385356201581132  \\
            0.010704810992963013  31.3870573979563  \\
            0.010803020268127812  31.388790551601453  \\
            0.010901229543292609  31.39055630757457  \\
            0.010999438818457408  31.392355337007  \\
            0.011097648093622207  31.394188338313292  \\
            0.011195857368787004  31.396056039760126  \\
            0.011294066643951804  31.397959200219034  \\
            0.011392275919116603  31.39989860979055  \\
            0.0114904851942814  31.401875094951823  \\
            0.011588694469446199  31.4038895186964  \\
            0.011686903744610996  31.152181691083427  \\
            0.011785113019775794  31.15283957680162  \\
            0.011883322294940593  31.15350657055391  \\
            0.01198153157010539  31.15418260843344  \\
            0.01207974084527019  31.15486762165748  \\
            0.012177950120434988  31.155561538246296  \\
            0.012276159395599787  31.156264281509685  \\
            0.012374368670764583  31.156975771105717  \\
            0.012472577945929382  31.157695921996858  \\
            0.012570787221094181  31.15842464555594  \\
            0.01266899649625898  31.159161848335923  \\
            0.012767205771423778  31.15990743484428  \\
            0.012865415046588575  31.160661303696035  \\
            0.012963624321753374  31.16142335103356  \\
            0.013061833596918171  31.16219346865424  \\
            0.01316004287208297  31.162971544600072  \\
            0.01325825214724777  31.163757462489794  \\
            0.013356461422412568  31.16455110379932  \\
            0.013454670697577364  31.165352344515494  \\
            0.013552879972742163  31.16616105790169  \\
            0.013651089247906962  31.166977113942337  \\
            0.013749298523071761  31.167800377337656  \\
            0.01384750779823656  31.16863071168346  \\
            0.013945717073401356  31.169467974211585  \\
            0.014043926348566155  31.170312019908504  \\
            0.014142135623730952  31.171162700960043  \\
            0.014240344898895752  31.17201986456601  \\
            0.01433855417406055  31.172883354897905  \\
            0.014436763449225346  31.17375301308772  \\
            0.014534972724390145  31.174628675887526  \\
            0.014633181999554944  31.175510177511235  \\
            0.014731391274719743  31.176397347438154  \\
            0.014829600549884542  31.177290013390248  \\
            0.014927809825049342  31.178187997792037  \\
            0.015026019100214137  31.17909112153211  \\
            0.015124228375378936  31.179999200655335  \\
            0.015222437650543735  31.18091204873874  \\
            0.015320646925708533  31.18182947566736  \\
            0.015418856200873332  31.182751287849584  \\
            0.015517065476038129  31.183677289089285  \\
            0.015615274751202926  31.184607279092837  \\
            0.015713484026367727  31.185541054774994  \\
            0.015811693301532526  31.186478410532654  \\
            0.015909902576697322  31.18741913630782  \\
            0.01600811185186212  31.1883630193751  \\
            0.016106321127026917  31.189309845382304  \\
            0.016204530402191716  31.190259394627706  \\
            0.016302739677356515  31.19121144606839  \\
            0.016400948952521314  31.19216577460142  \\
            0.016499158227686113  31.193122153705048  \\
            0.01659736750285091  31.194080352726555  \\
            0.016695576778015708  31.195040138523126  \\
            0.016793786053180507  31.196001275892755  \\
            0.016891995328345306  31.196963525670547  \\
            0.016990204603510105  31.19792664733534  \\
            0.0170884138786749  31.19889039731407  \\
            0.0171866231538397  31.199854529204227  \\
            0.0172848324290045  31.20081879444591  \\
            0.017383041704169298  31.2017829421797  \\
            0.017481250979334097  31.20274671928849  \\
            0.017579460254498892  31.20370986983781  \\
            0.01767766952966369  31.20467213678209  \\
            0.01777587880482849  31.205633260085715  \\
            0.01787408807999329  31.20659297871852  \\
            0.01797229735515809  31.207551027927302  \\
            0.018070506630322888  31.20850714305382  \\
            0.018168715905487683  31.209461056954826  \\
            0.018266925180652482  31.210412500034984  \\
            0.01836513445581728  31.21136120276822  \\
            0.018463343730982077  31.21230689247435  \\
            0.018561553006146876  31.213249296080683  \\
            0.018659762281311672  31.21418813926904  \\
            0.01875797155647647  31.215123145730846  \\
            0.01885618083164127  31.216054039380356  \\
            0.01895439010680607  31.171347651471482  \\
            0.019052599381970868  31.171548431572905  \\
            0.019150808657135664  31.171742280073598  \\
            0.019249017932300463  31.17192906686897  \\
            0.019347227207465262  31.17210866374439  \\
            0.01944543648263006  31.172280944694652  \\
            0.01954364575779486  31.17244578467238  \\
            0.01964185503295966  31.172603062492605  \\
            0.019740064308124455  31.172752657275414  \\
            0.019838273583289254  31.17289445295249  \\
            0.019936482858454053  31.173028333224067  \\
            0.020034692133618852  31.173154186605522  \\
            0.02013290140878365  31.173271902923194  \\
            0.020231110683948447  31.1733813761076  \\
            0.020329319959113246  31.173482502305973  \\
            0.020427529234278045  31.173575181200622  \\
            0.020525738509442844  31.173659314880688  \\
            0.020623947784607643  31.173734809955214  \\
            0.02072215705977244  31.1738015763561  \\
            0.020820366334937238  31.17385952708314  \\
            0.020918575610102033  31.173908578714638  \\
            0.021016784885266832  31.173948652253138  \\
            0.02111499416043163  31.173979672515834  \\
            0.02121320343559643  30.910660477506166  \\
            0.021311412710761226  30.910212508380326  \\
            0.021409621985926025  30.909755544891325  \\
            0.021507831261090824  30.909289558196615  \\
            0.021606040536255623  30.908814523680448  \\
            0.021704249811420422  30.9083304211973  \\
            0.021802459086585218  30.907837235059354  \\
            0.021900668361750017  30.90733495360023  \\
            0.021998877636914816  30.906823568976396  \\
            0.022097086912079615  30.906303079309726  \\
            0.022195296187244414  30.905773484001095  \\
            0.022293505462409213  30.90523478909316  \\
            0.02239171473757401  30.90468700223358  \\
            0.022489924012738808  30.90413013733294  \\
            0.022588133287903607  30.903564208776537  \\
            0.022686342563068406  30.902989236467754  \\
            0.022784551838233205  30.902405241838267  \\
            0.022882761113398  30.90181224822284  \\
            0.0229809703885628  30.901210282424177  \\
            0.0230791796637276  30.900599371167218  \\
            0.023177388938892398  30.899979542815295  \\
            0.023275598214057194  30.89935082469391  \\
            0.023373807489221993  30.898713243667448  \\
            0.02347201676438679  30.898066824909225  \\
            0.023570226039551587  31.23150287041385  \\
            0.023668435314716386  31.21841143360939  \\
            0.023766644589881186  31.206629829810815  \\
            0.023864853865045985  31.195990446258797  \\
            0.02396306314021078  31.186352615461235  \\
            0.02406127241537558  31.177597489607106  \\
            0.02415948169054038  31.169624032546814  \\
            0.024257690965705177  31.162345853058373  \\
            0.024355900240869977  31.155688678419075  \\
            0.024454109516034772  31.149588337001273  \\
            0.024552318791199575  31.14398911945224  \\
            0.02465052806636437  31.13884244808132  \\
            0.024748737341529166  31.13410579024975  \\
            0.02484694661669397  31.129741759407953  \\
            0.024945155891858764  31.125717374670366  \\
            0.025043365167023567  31.122003441053092  \\
            0.025141574442188362  31.11857403344426  \\
            0.025239783717353158  31.115406060279316  \\
            0.02533799299251796  31.112478899971162  \\
            0.025436202267682756  31.10977408650143  \\
            0.025534411542847555  31.107275045777982  \\
            0.025632620818012354  31.104966870209477  \\
            0.02573083009317715  31.10283612473113  \\
            0.02582903936834195  31.10087067804462  \\
            0.025927248643506748  31.09905956291346  \\
            0.026025457918671547  31.09739284795645  \\
            0.026123667193836343  31.095861531596814  \\
            0.026221876469001145  31.094457447755932  \\
            0.02632008574416594  31.09317318443852  \\
            0.026418295019330736  31.092002011890266  \\
            0.02651650429449554  31.09093781986947  \\
            0.026614713569660334  31.089975064329284  \\
            0.026712922844825137  31.089108716164734  \\
            0.026811132119989933  31.08833422148958  \\
            0.026909341395154728  31.08764746229883  \\
            0.02700755067031953  31.087044724005967  \\
            0.027105759945484326  31.086522665147793  \\
            0.02720396922064913  31.086078291082355  \\
            0.027302178495813925  31.085708931027185  \\
            0.02740038777097872  31.08541221555213  \\
            0.027498597046143523  31.085186059108473  \\
            0.027596806321308318  31.08502864028768  \\
            0.02769501559647312  31.084938388527384  \\
            0.027793224871637916  31.084913966674684  \\
            0.027891434146802712  31.084954260649344  \\
            0.027989643421967515  31.08505836440647  \\
            0.02808785269713231  31.085225568801253  \\
            0.02818606197229711  31.085455350992117  \\
            0.028284271247461905  31.085747362860346  \\
            0.028382480522626704  31.08610142170363  \\
            0.028480689797791503  31.086517499885247  \\
            0.0285788990729563  31.086995714079297  \\
            0.0286771083481211  31.087536318129096  \\
            0.028775317623285897  31.08813969081765  \\
            0.028873526898450692  31.088806327372964  \\
            0.028971736173615495  31.08953683102283  \\
            0.02906994544878029  31.090331901285662  \\
            0.029168154723945093  31.091192325400403  \\
            0.02926636399910989  31.092118967707844  \\
            0.02936457327427469  31.093112759828337  \\
            0.029462782549439487  31.094174688904655  \\
            0.029560991824604282  31.095305786361873  \\
            0.029659201099769085  31.096507118557906  \\
            0.02975741037493388  31.097779773832954  \\
            0.029855619650098683  31.099124847953963  \\
            0.02995382892526348  31.10054343478775  \\
            0.030052038200428274  31.102036609410156  \\
            0.030150247475593077  31.103605417040576  \\
            0.030248456750757872  31.105250856395394  \\
            0.03034666602592267  31.106973865656865  \\
            0.03044487530108747  31.108775305268992  \\
            0.030543084576252266  31.11065594262811  \\
            0.030641293851417065  31.112616433698957  \\
            0.030739503126581864  31.114657305546846  \\
            0.030837712401746663  31.116778937859998  \\
            0.03093592167691146  31.118981542678316  \\
            0.031034130952076258  31.272506613164833  \\
            0.031132340227241057  31.236251369553166  \\
            0.031230549502405853  31.20690205216008  \\
            0.03132875877757066  31.18279319443841  \\
            0.031426968052735454  31.162752022018  \\
            0.03152517732790025  31.145932229780065  \\
            0.03162338660306505  31.131710076125362  \\
            0.03172159587822985  31.119617329846108  \\
            0.031819805153394644  31.10929680629643  \\
            0.031918014428559446  31.100472129918593  \\
            0.03201622370372424  31.09292672185823  \\
            0.03211443297888904  31.08648890884856  \\
            0.03221264225405383  31.08102117038379  \\
            0.032310851529218636  31.07641225119668  \\
            0.03240906080438343  31.07257128568466  \\
            0.032507270079548234  31.069423357093054  \\
            0.03260547935471303  31.066906104639724  \\
            0.032703688629877825  31.064967093728672  \\
            0.03280189790504263  31.06356176286763  \\
            0.03290010718020742  31.06265180076937  \\
            0.032998316455372226  31.062203855322206  \\
            0.03309652573053702  31.062188502087785  \\
            0.03319473500570182  31.06257941012973  \\
            0.03329294428086662  31.063352669649777  \\
            0.033391153556031415  31.064486248403536  \\
            0.03348936283119622  31.06595954892578  \\
            0.03358757210636101  31.067753052663303  \\
            0.03368578138152581  31.069848033659554  \\
            0.03378399065669061  31.072226331305576  \\
            0.03388219993185541  31.0748701718886  \\
            0.03398040920702021  31.077762033061727  \\
            0.034078618482185005  31.08088454675593  \\
            0.0341768277573498  31.08422043021047  \\
            0.0342750370325146  31.08775244935484  \\
            0.0343732463076794  31.09146340485974  \\
            0.0344714555828442  31.09533613831095  \\
            0.034569664858009  31.099353559382585  \\
            0.03466787413317379  31.103498688814998  \\
            0.034766083408338595  31.107754711379005  \\
            0.03486429268350339  31.112105043160277  \\
            0.03496250195866819  31.116533401798353  \\
            0.03506071123383299  31.121023886445613  \\
            0.035158920508997785  31.125561055041842  \\
            0.03525712978416259  31.13013000547342  \\
            0.03535533905932738  31.13471645130323  \\
            0.035453548334492185  31.13930679440457  \\
            0.03555175760965698  31.143888191529502  \\
            0.035649966884821783  31.148448614301845  \\
            0.03574817615998658  31.152976898519928  \\
            0.035846385435151375  31.157462787357204  \\
            0.03594459471031618  31.161896965006225  \\
            0.03604280398548097  31.166271077934947  \\
            0.036141013260645775  31.170577754151637  \\
            0.03623922253581057  31.174810606106977  \\
            0.03633743181097537  31.178964232476496  \\
            0.03643564108614017  31.183034207172003  \\
            0.036533850361304965  31.187017067212043  \\
            0.03663205963646976  31.190910291760154  \\
            0.03673026891163456  31.19471227523863  \\
            0.03682847818679936  31.198422298494673  \\
            0.036926687461964154  31.202040492507756  \\
            0.03702489673712895  31.205567798216258  \\
            0.03712310601229375  31.209005923392613  \\
            0.03722131528745855  31.2123572889877  \\
            0.037319524562623343  31.215624973248588  \\
            0.037417733837788146  31.21881264846876  \\
            0.03751594311295294  31.221924502083034  \\
            0.037614152388117744  31.224965157086668  \\
            0.03771236166328254  31.227939570641297  \\
            0.037810570938447335  31.230852926245895  \\
            0.03790878021361214  31.23371050798699  \\
            0.038006989488776934  31.236517562111057  \\
            0.038105198763941736  31.23927914449303  \\
            0.03820340803910653  31.24199996030767  \\
            0.03830161731427133  31.244684192886957  \\
            0.03839982658943613  31.247335335284152  \\
            0.038498035864600925  31.24995603004332  \\
            0.03859624513976573  31.25254792348487  \\
            0.038694454414930524  31.25511154694615  \\
            0.038792663690095326  31.25764623885562  \\
            0.03889087296526012  31.26015010637837  \\
            0.03898908224042492  31.262620041672395  \\
            0.03908729151558972  31.265051785737825  \\
            0.039185500790754516  31.267440036970505  \\
            0.03928371006591932  31.269778586218454  \\
            0.039381919341084114  31.272060457631643  \\
            0.03948012861624891  31.27427801826151  \\
            0.03957833789141371  31.276422996251153  \\
            0.03967654716657851  31.27848635383518  \\
            0.03977475644174331  31.28045788195024  \\
            0.039872965716908106  31.28232534679203  \\
            0.0399711749920729  31.284072837944258  \\
            0.040069384267237704  31.285677651504084  \\
            0.0401675935424025  31.212057631879937  \\
            0.0402658028175673  31.212562607606323  \\
            0.0403640120927321  31.213071106495853  \\
            0.04046222136789689  31.213583062539342  \\
            0.040560430643061696  31.214098409482407  \\
            0.04065863991822649  31.214617082240103  \\
            0.040756849193391294  31.21513901455499  \\
            0.04085505846855609  31.215664143700923  \\
            0.040953267743720885  31.216192402925905  \\
            0.04105147701888569  31.216723729382867  \\
            0.04114968629405048  31.21725805978495  \\
            0.041247895569215286  31.21779533052826  \\
            0.04134610484438008  31.218335479788607  \\
            0.04144431411954488  31.218878444659214  \\
            0.04154252339470968  31.219424163961897  \\
            0.041640732669874475  31.2199725761412  \\
            0.04173894194503927  31.22052361990799  \\
            0.041837151220204066  31.22107723577404  \\
            0.04193536049536887  31.221633363448326  \\
            0.042033569770533664  31.222191943743553  \\
            0.04213177904569846  31.22275291779083  \\
            0.04222998832086326  31.22331622724781  \\
            0.04232819759602806  31.223881813919398  \\
            0.04242640687119286  31.224449620763856  \\
            0.042524616146357656  31.225019590915803  \\
            0.04262282542152245  31.22559166816725  \\
            0.042721034696687255  31.22616579574955  \\
            0.04281924397185205  31.226741919612213  \\
            0.04291745324701685  31.227319984223517  \\
            0.04301566252218165  31.22789993492825  \\
            0.043113871797346444  31.228481719155443  \\
            0.043212081072511246  31.229065282446182  \\
            0.04331029034767604  31.229650573152504  \\
            0.043408499622840845  31.230237538276207  \\
            0.04350670889800564  31.23082612648734  \\
            0.043604918173170436  31.231416286828654  \\
            0.04370312744833524  31.232007969157547  \\
            0.043801336723500034  31.232601122561334  \\
            0.043899545998664836  31.233195698129872  \\
            0.04399775527382963  31.233791647431318  \\
            0.04409596454899443  31.23438892180374  \\
            0.04419417382415923  31.234987472616538  \\
            0.044292383099324026  31.235587254074144  \\
            0.04439059237448883  31.236188218217443  \\
            0.044488801649653624  31.236790320072767  \\
            0.04458701092481843  31.237393513092563  \\
            0.04468522019998322  31.23799775298299  \\
            0.04478342947514802  31.238602995218773  \\
            0.04488163875031282  31.23920919500932  \\
            0.044979848025477616  31.239816310265557  \\
            0.04507805730064242  31.240424297426003  \\
            0.045176266575807214  31.24103311424782  \\
            0.04527447585097201  31.241642719155017  \\
            0.04537268512613681  31.242253071037034  \\
            0.04547089440130161  31.242864129599003  \\
            0.04556910367646641  31.243475853702144  \\
            0.045667312951631206  31.24408820557187  \\
            0.045765522226796  31.24470114440444  \\
            0.045863731501960804  31.245314632929265  \\
            0.0459619407771256  31.245928633193955  \\
            0.0460601500522904  31.246543107158846  \\
            0.0461583593274552  31.2471580181048  \\
            0.046256568602619993  31.24777333021514  \\
            0.046354777877784796  31.248389007391534  \\
            0.04645298715294959  31.249005014540213  \\
            0.04655119642811439  31.249621317165985  \\
            0.04664940570327918  31.25023788027024  \\
            0.046747614978443985  31.250854671003854  \\
            0.04684582425360878  31.25147165515886  \\
            0.04694403352877358  31.252088800646444  \\
            0.04704224280393838  31.25270607553181  \\
            0.047140452079103175  31.25332344734935  \\
            0.04723866135426797  31.25394088549845  \\
            0.04733687062943277  31.25455835923628  \\
            0.04743507990459757  31.255175837554923  \\
            0.04753328917976237  31.255793291680163  \\
            0.04763149845492717  31.256410691485875  \\
            0.04772970773009197  31.257028008444166  \\
            0.047827917005256765  31.257645214185874  \\
            0.04792612628042156  31.258262281323336  \\
            0.04802433555558636  31.258879180847934  \\
            0.04812254483075116  31.259495886915897  \\
            0.04822075410591596  31.260112372771406  \\
            0.04831896338108076  31.260728612105215  \\
            0.04841717265624555  31.261344579298452  \\
            0.048515381931410355  31.261960248057985  \\
            0.04861359120657515  31.262575595012013  \\
            0.04871180048173995  31.26319059403639  \\
            0.04881000975690475  31.26380522284372  \\
            0.048908219032069544  31.264419455682823  \\
            0.04900642830723435  31.265033270330346  \\
            0.04910463758239915  31.26564664415627  \\
            0.049202846857563945  31.266259553573136  \\
            0.04930105613272874  31.2668719767768  \\
            0.049399265407893536  31.267483891579882  \\
            0.04949747468305833  31.268095277177967  \\
            0.04959568395822314  31.268706111127972  \\
            0.04969389323338794  31.269316373832194  \\
            0.04979210250855273  31.269926043616827  \\
            0.04989031178371753  31.270535100835676  \\
            0.049988521058882324  31.271143525836553  \\
            0.05008673033404713  31.271751297480414  \\
            0.05018493960921193  31.272358398046567  \\
            0.050283148884376724  31.272964807518935  \\
            0.05038135815954152  31.273570507432503  \\
            0.050479567434706316  31.274175479517027  \\
            0.050577776709871125  31.274779704408484  \\
            0.05067598598503592  31.275383165135672  \\
            0.050774195260200716  31.27598584349845  \\
            0.05087240453536551  31.276587720814415  \\
            0.05097061381053031  31.277188781054654  \\
            0.05106882308569511  31.27778900662972  \\
            0.05116703236085991  31.278388380417248  \\
            0.05126524163602471  31.278986884972714  \\
            0.051363450911189504  31.279584504811663  \\
            0.0514616601863543  31.280181222434845  \\
            0.0515598694615191  31.2807770220752  \\
            0.0516580787366839  31.28137188746403  \\
            0.05175628801184869  31.281965802188168  \\
            0.051854497287013496  31.28255875054947  \\
            0.05195270656217829  31.283150716124304  \\
            0.052050915837343094  31.283741684259734  \\
            0.05214912511250789  31.28433163760534  \\
            0.052247334387672685  31.284920562729432  \\
            0.05234554366283748  31.285508442005  \\
            0.05244375293800229  31.286095261250992  \\
            0.052541962213167086  31.286681004625486  \\
            0.05264017148833188  31.28726565613013  \\
            0.05273838076349668  31.28784920162875  \\
            0.05283659003866147  31.288431624569302  \\
            0.05293479931382628  31.289012909947903  \\
            0.05303300858899108  31.289593042473864  \\
            0.05313121786415587  31.290172006739873  \\
            0.05322942713932067  31.290749786658512  \\
            0.053327636414485464  31.291326367231616  \\
            0.053425845689650274  31.29190173271904  \\
            0.05352405496481507  31.29247586760563  \\
            0.053622264239979865  31.293048755264998  \\
            0.05372047351514466  31.29362038049929  \\
            0.053818682790309456  31.294190727285997  \\
            0.053916892065474266  31.294759779534203  \\
            0.05401510134063906  31.29532752077981  \\
            0.05411331061580386  31.29589393467766  \\
            0.05421151989096865  31.296459004038898  \\
            0.05430972916613345  31.297022712926452  \\
            0.05440793844129826  31.29758504437141  \\
            0.05450614771646305  31.29814598035168  \\
            0.05460435699162785  31.298705504167327  \\
            0.054702566266792645  31.29926359822377  \\
            0.05480077554195744  31.29982024419053  \\
            0.05489898481712225  31.300375424509962  \\
            0.054997194092287045  31.300929120649705  \\
            0.05509540336745184  31.30148131389396  \\
            0.055193612642616637  31.302031985134754  \\
            0.05529182191778143  31.302581115842163  \\
            0.05539003119294624  31.30312868630822  \\
            0.05548824046811104  31.30367467608624  \\
            0.05558644974327583  31.304219065456962  \\
            0.05568465901844063  31.304761834118942  \\
            0.055782868293605424  31.305302961102832  \\
            0.05588107756877023  31.305842425089793  \\
            0.05597928684393503  31.30638020455459  \\
            0.056077496119099825  31.306916277448362  \\
            0.05617570539426462  31.307450621171554  \\
            0.056273914669429416  31.30798321405778  \\
            0.05637212394459422  31.308514031510132  \\
            0.056470333219759014  31.309043050560305  \\
            0.05656854249492381  31.309570247592415  \\
            0.05666675177008861  31.310095597216478  \\
            0.05676496104525341  31.310619075036808  \\
            0.05686317032041821  31.311140655833757  \\
            0.056961379595583006  31.311660313386916  \\
            0.0570595888707478  31.312178020953798  \\
            0.0571577981459126  31.312693752259747  \\
            0.0572560074210774  31.31320747995598  \\
            0.0573542166962422  31.31371917525506  \\
            0.057452425971407  31.314228810672258  \\
            0.057550635246571794  31.314736357414326  \\
            0.05764884452173659  31.315241785208055  \\
            0.057747053796901385  31.31574506405148  \\
            0.057845263072066194  31.31624616422953  \\
            0.05794347234723099  31.316745053713554  \\
            0.058041681622395785  31.317241700428685  \\
            0.05813989089756058  31.31773607293345  \\
            0.05823810017272538  31.318228137817837  \\
            0.058336309447890186  31.31871786159839  \\
            0.05843451872305498  31.31920521003595  \\
            0.05853272799821978  31.31969014784776  \\
            0.05863093727338457  31.320172640281925  \\
            0.05872914654854938  31.320652650893663  \\
            0.05882735582371418  31.321130143040314  \\
            0.058925565098878974  31.321605079089718  \\
            0.05902377437404377  31.32207742097378  \\
            0.059121983649208565  31.32254713042189  \\
            0.059220192924373374  31.323014167142485  \\
            0.05931840219953817  31.323478491005783  \\
            0.059416611474702966  31.323940061614252  \\
            0.05951482074986776  31.32439883661772  \\
            0.05961303002503256  31.324854774080453  \\
            0.059711239300197366  31.325307830439122  \\
            0.05980944857536216  31.325757962455707  \\
            0.05990765785052696  31.326205124587055  \\
            0.06000586712569175  31.326649271282275  \\
            0.06010407640085655  31.327090356865586  \\
            0.06020228567602136  31.32752833324401  \\
            0.060300494951186154  31.327963153308247  \\
            0.06039870422635095  31.32839476759208  \\
            0.060496913501515745  31.32882312694832  \\
            0.06059512277668054  31.32924817973076  \\
            0.06069333205184534  31.3296698751843  \\
            0.060791541327010146  31.330088160488174  \\
            0.06088975060217494  31.330502982558006  \\
            0.06098795987733974  31.33091428641255  \\
            0.06108616915250453  31.33132201675708  \\
            0.061184378427669335  31.331726118154013  \\
            0.06128258770283413  31.332126532003898  \\
            0.061380796977998926  31.074716479351054  \\
            0.06147900625316373  31.075316921880074  \\
            0.061577215528328524  31.07591433276372  \\
            0.06167542480349333  31.076508679593967  \\
            0.06177363407865812  31.077099929933915  \\
            0.06187184335382292  31.07768805289654  \\
            0.061970052628987714  31.078273018992203  \\
            0.062068261904152516  31.078854799816046  \\
            0.06216647117931732  31.079433366264684  \\
            0.062264680454482114  31.08000869083873  \\
            0.06236288972964691  31.08058074719867  \\
            0.062461099004811706  31.08114951024773  \\
            0.06255930827997651  31.08171495443963  \\
            0.06265751755514132  31.082277055531684  \\
            0.06275572683030611  31.082835790477  \\
            0.06285393610547091  31.083391135638855  \\
            0.0629521453806357  31.083943070002345  \\
            0.0630503546558005  31.084491571608325  \\
            0.0631485639309653  31.085036619839244  \\
            0.0632467732061301  31.085578194731415  \\
            0.0633449824812949  31.08611627694927  \\
            0.0634431917564597  31.08665084770955  \\
            0.06354140103162449  31.087181888815078  \\
            0.06363961030678929  31.087709382852662  \\
            0.06373781958195408  31.088233313336175  \\
            0.06383602885711889  31.088753664209936  \\
            0.06393423813228369  31.089270418996126  \\
            0.06403244740744848  31.089783563359827  \\
            0.06413065668261328  31.09029308262049  \\
            0.06422886595777808  31.09079896269403  \\
            0.06432707523294287  31.091301190898193  \\
            0.06442528450810767  31.09179975443474  \\
            0.06452349378327248  31.09229464039858  \\
            0.06462170305843727  31.092785838453466  \\
            0.06471991233360207  31.09327333660885  \\
            0.06481812160876686  31.093757124623014  \\
            0.06491633088393166  31.094237193370958  \\
            0.06501454015909647  31.094713532284082  \\
            0.06511274943426126  31.095186132476435  \\
            0.06521095870942606  31.095654986723755  \\
            0.06530916798459085  31.096120086224524  \\
            0.06540737725975565  31.096581424755385  \\
            0.06550558653492046  31.097038994498995  \\
            0.06560379581008526  31.097492789204743  \\
            0.06570200508525005  31.09794280334485  \\
            0.06580021436041485  31.098389032355715  \\
            0.06589842363557964  31.09883147013932  \\
            0.06599663291074445  31.099270113478504  \\
            0.06609484218590925  31.099704958073563  \\
            0.06619305146107404  31.100136000112382  \\
            0.06629126073623884  31.100563237543433  \\
            0.06638947001140363  31.100986667495313  \\
            0.06648767928656844  31.10140628807839  \\
            0.06658588856173324  31.101822097789  \\
            0.06668409783689803  31.10223409569711  \\
            0.06678230711206283  31.102642281110246  \\
            0.06688051638722763  31.103046653818588  \\
            0.06697872566239244  31.10344721359127  \\
            0.06707693493755723  31.103843961750663  \\
            0.06717514421272203  31.1042368986307  \\
            0.06727335348788682  31.104626026302746  \\
            0.06737156276305162  31.105011346874605  \\
            0.06746977203821643  31.10539286179623  \\
            0.06756798131338122  31.105770574438388  \\
            0.06766619058854602  31.106144487130624  \\
            0.06776439986371081  31.106514604035887  \\
            0.06786260913887561  31.106880928263493  \\
            0.06796081841404042  31.10724346408743  \\
            0.06805902768920521  31.10760221685408  \\
            0.06815723696437001  31.107957190016585  \\
            0.0682554462395348  31.108308389229748  \\
            0.0683536555146996  31.108655820065273  \\
            0.06845186478986441  31.10899948839012  \\
            0.0685500740650292  31.10933940050903  \\
            0.068648283340194  31.109675562174132  \\
            0.0687464926153588  31.110007980501894  \\
            0.0688447018905236  31.110336662818366  \\
            0.0689429111656884  31.110661615616383  \\
            0.0690411204408532  31.110982847128714  \\
            0.069139329716018  31.111300364520975  \\
            0.06923753899118279  31.111614175956156  \\
            0.06933574826634759  31.111924290519095  \\
            0.0694339575415124  31.11223071550841  \\
            0.06953216681667719  31.11253346058129  \\
            0.06963037609184199  31.11283253399197  \\
            0.06972858536700678  31.11312794530474  \\
            0.06982679464217158  31.113419704080105  \\
            0.06992500391733639  31.11370781971852  \\
            0.07002321319250118  31.113992301973784  \\
            0.07012142246766598  31.114273160034717  \\
            0.07021963174283077  31.11455040510735  \\
            0.07031784101799557  31.114824046671288  \\
            0.07041605029316038  31.115094095226976  \\
            0.07051425956832517  31.115360561230467  \\
            0.07061246884348997  31.11562345556114  \\
            0.07071067811865477  31.115882789044374  \\
            0.07080888739381958  31.116138571704187  \\
            0.07090709666898437  31.116390815659244  \\
            0.07100530594414917  31.11663953125482  \\
            0.07110351521931396  31.11688472973781  \\
            0.07120172449447876  31.11712642211023  \\
            0.07129993376964357  31.117364620291777  \\
            0.07139814304480836  31.117599334959802  \\
            0.07149635231997316  31.117830577921104  \\
            0.07159456159513795  31.118058360522415  \\
            0.07169277087030275  31.11828269454866  \\
            0.07179098014546756  31.1185035907407  \\
            0.07188918942063235  31.118721061627987  \\
            0.07198739869579715  31.118935117824325  \\
            0.07208560797096195  31.11914577177535  \\
            0.07218381724612674  31.119353034513068  \\
            0.07228202652129155  31.119556917760082  \\
            0.07238023579645635  31.119757433214286  \\
            0.07247844507162114  31.119954592618324  \\
            0.07257665434678594  31.120148407608074  \\
            0.07267486362195073  31.120338889227813  \\
            0.07277307289711553  31.290592545438574  \\
            0.07287128217228034  31.291187489859876  \\
            0.07296949144744513  31.29177705950981  \\
            0.07306770072260993  31.292361303322664  \\
            0.07316590999777473  31.29294026961571  \\
            0.07326411927293952  31.293514005661063  \\
            0.07336232854810432  31.294082558127602  \\
            0.07346053782326913  31.29464597207069  \\
            0.07355874709843392  31.29520429197042  \\
            0.07365695637359872  31.295757561124496  \\
            0.07375516564876351  31.296305821883674  \\
            0.07385337492392831  31.29684911648013  \\
            0.0739515841990931  31.297387484335577  \\
            0.0740497934742579  31.297920966020783  \\
            0.07414800274942271  31.298449600289647  \\
            0.0742462120245875  31.29897342444443  \\
            0.0743444212997523  31.299492475938756  \\
            0.0744426305749171  31.300006790783822  \\
            0.07454083985008189  31.300516404146254  \\
            0.07463904912524669  31.30102135054928  \\
            0.0747372584004115  31.301521663410284  \\
            0.07483546767557629  31.302017375196225  \\
            0.07493367695074109  31.302508517737493  \\
            0.07503188622590588  31.30299512248519  \\
            0.07513009550107068  31.303477219248695  \\
            0.07522830477623549  31.30395483649202  \\
            0.07532651405140028  31.304428003758197  \\
            0.07542472332656508  31.30489674845606  \\
            0.07552293260172988  31.305361096518087  \\
            0.07562114187689467  31.305821075152235  \\
            0.07571935115205948  31.30627670881462  \\
            0.07581756042722428  31.306728022096532  \\
            0.07591576970238907  31.307175038063548  \\
            0.07601397897755387  31.30761777981979  \\
            0.07611218825271866  31.30805626963917  \\
            0.07621039752788347  31.308490527738343  \\
            0.07630860680304827  31.308920574551188  \\
            0.07640681607821306  31.309346430102405  \\
            0.07650502535337786  31.30976811315639  \\
            0.07660323462854265  31.310185640954654  \\
            0.07670144390370746  31.31059903041803  \\
            0.07679965317887226  31.31100829882367  \\
            0.07689786245403706  31.311413460968197  \\
            0.07699607172920185  31.311814531644405  \\
            0.07709428100436666  31.31221152465779  \\
            0.07719249027953146  31.312604453113337  \\
            0.07729069955469625  31.312993330120733  \\
            0.07738890882986105  31.313378165879296  \\
            0.07748711810502584  31.313758971709902  \\
            0.07758532738019065  31.31413575762976  \\
            0.07768353665535545  31.314508532769974  \\
            0.07778174593052024  31.314877305487894  \\
            0.07787995520568504  31.315242082945193  \\
            0.07797816448084983  31.315602872418506  \\
            0.07807637375601464  31.31595967930093  \\
            0.07817458303117944  31.316312509104105  \\
            0.07827279230634424  31.316661366640187  \\
            0.07837100158150903  31.317006254891556  \\
            0.07846921085667383  31.31734717691813  \\
            0.07856742013183864  31.31768413483056  \\
            0.07866562940700343  31.31801712954195  \\
            0.07876383868216823  31.318346161386202  \\
            0.07886204795733302  31.31867123105251  \\
            0.07896025723249782  31.318992336494944  \\
            0.07905846650766263  31.31930947604827  \\
            0.07915667578282742  31.31962264727039  \\
            0.07925488505799222  31.319931846866503  \\
            0.07935309433315701  31.320237069702774  \\
            0.07945130360832181  31.320538312292904  \\
            0.07954951288348662  31.320835567486974  \\
            0.07964772215865142  31.321128829767442  \\
            0.07974593143381621  31.321418091157838  \\
            0.079844140708981  31.321703344215337  \\
            0.0799423499841458  31.32198457947245  \\
            0.08004055925931061  31.322261788219244  \\
            0.08013876853447541  31.322534959051538  \\
            0.0802369778096402  31.32280408218982  \\
            0.080335187084805  31.32306914497175  \\
            0.0804333963599698  31.323330134844216  \\
            0.0805316056351346  31.32358703899994  \\
            0.0806298149102994  31.32383984273962  \\
            0.0807280241854642  31.32408853162729  \\
            0.08082623346062899  31.324333090344865  \\
            0.08092444273579379  31.32457350278811  \\
            0.0810226520109586  31.324809751299043  \\
            0.08112086128612339  31.325041818826847  \\
            0.08121907056128819  31.325269686866925  \\
            0.08131727983645298  31.325493335837322  \\
            0.08141548911161778  31.325712746383804  \\
            0.08151369838678259  31.325927898205993  \\
            0.08161190766194738  31.326138769656595  \\
            0.08171011693711218  31.32634533921608  \\
            0.08180832621227697  31.326547583894605  \\
            0.08190653548744177  31.32674548018797  \\
            0.08200474476260658  31.326939004763936  \\
            0.08210295403777138  31.327128132645534  \\
            0.08220116331293617  31.327312838641777  \\
            0.08229937258810097  31.327493096856433  \\
            0.08239758186326576  31.327668880682673  \\
            0.08249579113843057  31.32784016267956  \\
            0.08259400041359537  31.32800691476756  \\
            0.08269220968876016  31.328169108416095  \\
            0.08279041896392496  31.32832671445325  \\
            0.08288862823908975  31.32847970302677  \\
            0.08298683751425455  31.328628043382533  \\
            0.08308504678941936  31.328771704818553  \\
            0.08318325606458415  31.328910655460767  \\
            0.08328146533974895  31.329044862120366  \\
            0.08337967461491375  31.329174292651437  \\
            0.08347788389007854  31.32929891332673  \\
            0.08357609316524334  31.329418689049117  \\
            0.08367430244040813  31.329533585898186  \\
            0.08377251171557294  31.32964356808406  \\
            0.08387072099073774  31.32974859927138  \\
            0.08396893026590253  31.32984864278372  \\
            0.08406713954106733  31.28899872866034  \\
            0.08416534881623212  31.28841963396441  \\
            0.08426355809139692  31.28784528163601  \\
            0.08436176736656173  31.28727567686592  \\
            0.08445997664172653  31.286710822309544  \\
            0.08455818591689132  31.2861507212457  \\
            0.08465639519205612  31.2855953752553  \\
            0.08475460446722091  31.285044785241798  \\
            0.08485281374238572  31.28449895068614  \\
            0.08495102301755052  31.28395787117074  \\
            0.08504923229271531  31.283421545151235  \\
            0.08514744156788011  31.282889970142154  \\
            0.0852456508430449  31.282363142565803  \\
            0.08534386011820971  31.281841059079515  \\
            0.08544206939337451  31.281323714439832  \\
            0.0855402786685393  31.280811104002513  \\
            0.0856384879437041  31.280303221286424  \\
            0.0857366972188689  31.279800060334562  \\
            0.0858349064940337  31.279301613575935  \\
            0.0859331157691985  31.2788078729651  \\
            0.0860313250443633  31.2783188309988  \\
            0.08612953431952809  31.277834477997  \\
            0.08622774359469289  31.277354805244524  \\
            0.0863259528698577  31.276879802386638  \\
            0.08642416214502249  31.276409459469807  \\
            0.08652237142018729  31.275943765966797  \\
            0.08662058069535208  31.27548270969755  \\
            0.08671878997051688  31.27502628009066  \\
            0.08681699924568169  31.27457446482664  \\
            0.08691520852084648  31.274127251771777  \\
            0.08701341779601128  31.273684628058184  \\
            0.08711162707117608  31.273246580452227  \\
            0.08720983634634087  31.27281309643983  \\
            0.08730804562150568  31.272384161969583  \\
            0.08740625489667048  31.271959762996303  \\
            0.08750446417183527  31.271539886237793  \\
            0.08760267344700007  31.2711245168092  \\
            0.08770088272216486  31.270713640722015  \\
            0.08779909199732967  31.27030724300793  \\
            0.08789730127249447  31.269905308763757  \\
            0.08799551054765926  31.26950782330829  \\
            0.08809371982282406  31.269114771410496  \\
            0.08819192909798886  31.268726138269844  \\
            0.08829013837315366  31.268341907675133  \\
            0.08838834764831846  31.26796206468989  \\
            0.08848655692348326  31.267586594252165  \\
            0.08858476619864805  31.267215480452652  \\
            0.08868297547381285  31.266848707377704  \\
            0.08878118474897766  31.26648625997563  \\
            0.08887939402414245  31.26612812231722  \\
            0.08897760329930725  31.26577427884724  \\
            0.08907581257447204  31.26542471405842  \\
            0.08917402184963685  31.265079411909877  \\
            0.08927223112480165  31.264738357117196  \\
            0.08937044039996644  31.264401534241795  \\
            0.08946864967513124  31.264068927936798  \\
            0.08956685895029604  31.263740521716947  \\
            0.08966506822546084  31.26341630134259  \\
            0.08976327750062564  31.263096250916036  \\
            0.08986148677579044  31.26278035552204  \\
            0.08995969605095523  31.262468600233106  \\
            0.09005790532612003  31.262160969342982  \\
            0.09015611460128484  31.261857448486293  \\
            0.09025432387644963  31.26155802268837  \\
            0.09035253315161443  31.2612626773112  \\
            0.09045074242677922  31.26097139808269  \\
            0.09054895170194402  31.26068417061153  \\
            0.09064716097710883  31.260400979904155  \\
            0.09074537025227362  31.26012181244989  \\
            0.09084357952743842  31.259846654258588  \\
            0.09094178880260322  31.259575491942446  \\
            0.09103999807776801  31.259308311301012  \\
            0.09113820735293282  31.259045098937168  \\
            0.09123641662809762  31.25878584166455  \\
            0.09133462590326241  31.258530526520822  \\
            0.09143283517842721  31.25827914070469  \\
            0.091531044453592  31.258031670732418  \\
            0.09162925372875681  31.25778810506444  \\
            0.09172746300392161  31.25754843054425  \\
            0.0918256722790864  31.257312635265503  \\
            0.0919238815542512  31.257080707715108  \\
            0.092022090829416  31.256852635535182  \\
            0.0921203001045808  31.25662840752065  \\
            0.0922185093797456  31.25640801148274  \\
            0.0923167186549104  31.256191437132888  \\
            0.09241492793007519  31.25597867309502  \\
            0.09251313720523999  31.255769708402852  \\
            0.09261134648040478  31.255564533082  \\
            0.09270955575556959  31.255363136415085  \\
            0.09280776503073439  31.255165507764648  \\
            0.09290597430589918  31.254971637887476  \\
            0.09300418358106398  31.25478151662785  \\
            0.09310239285622877  31.25459513455301  \\
            0.09320060213139357  31.254412482350727  \\
            0.09329881140655837  31.254233550976025  \\
            0.09339702068172318  31.254058330812214  \\
            0.09349522995688797  31.253886813788842  \\
            0.09359343923205277  31.25371899134494  \\
            0.09369164850721756  31.253554855061854  \\
            0.09378985778238236  31.253394396677578  \\
            0.09388806705754715  31.253237608094434  \\
            0.09398627633271196  31.253084481955614  \\
            0.09408448560787676  31.25293501067571  \\
            0.09418269488304155  31.252789186713496  \\
            0.09428090415820635  31.25264700303097  \\
            0.09437911343337115  31.252508453016677  \\
            0.09447732270853594  31.25237352919273  \\
            0.09457553198370075  31.252242225281186  \\
            0.09467374125886555  31.25211453527615  \\
            0.09477195053403034  31.25199045305385  \\
            0.09487015980919514  31.251869972329974  \\
            0.09496836908435993  31.251753087492638  \\
            0.09506657835952474  31.251639792760397  \\
            0.09516478763468954  31.251530082858515  \\
            0.09526299690985433  31.251423952760003  \\
            0.09536120618501913  31.251321396735634  \\
            0.09545941546018394  31.251222411522576  \\
            0.09555762473534873  31.251126990476546  \\
            0.09565583401051353  31.251035129999995  \\
            0.09575404328567833  31.250946826084032  \\
            0.09585225256084312  31.250862074182255  \\
            0.09595046183600793  31.250780871126597  \\
            0.09604867111117273  31.250703211818042  \\
            0.09614688038633752  31.25062909356253  \\
            0.09624508966150232  31.250558512932663  \\
            0.09634329893666711  31.250491466454722  \\
            0.09644150821183192  31.250427951017475  \\
            0.09653971748699672  31.250367964226154  \\
            0.09663792676216151  31.250311502544623  \\
            0.09673613603732631  31.250258564169467  \\
            0.0968343453124911  31.250209146110386  \\
            0.09693255458765591  31.250163246406768  \\
            0.09703076386282071  31.250120863249343  \\
            0.0971289731379855  31.25008199447257  \\
            0.0972271824131503  31.250046638930908  \\
            0.0973253916883151  31.25001479374119  \\
            0.0974236009634799  31.249986458993106  \\
            0.0975218102386447  31.249961632585222  \\
            0.0976200195138095  31.249940313564345  \\
            0.09771822878897429  31.249922501519055  \\
            0.09781643806413909  31.249908195074646  \\
            0.0979146473393039  31.249897394300106  \\
            0.0980128566144687  31.24989009803416  \\
            0.09811106588963349  31.24988630649165  \\
        }
        ;
    \addplot[color={rgb,1:red,0.0;green,0.0;blue,1.0}, name path={9b0fb34e-5d2b-42ec-8ff8-f1e5ff140438}, draw opacity={1.0}, line width={1}, dotted, forget plot]
        table[row sep={\\}]
        {
            \\
            0.0  1033.878856495483  \\
            9.820927516479829e-5  1033.879183902441  \\
            0.00019641855032959658  1033.880171782528  \\
            0.00029462782549439484  1033.8818200448063  \\
            0.00039283710065919316  1033.8841285973942  \\
            0.0004910463758239915  1033.8870973482474  \\
            0.0005892556509887897  1032.6038080145865  \\
            0.000687464926153588  1032.6065171080475  \\
            0.0007856742013183863  1032.6098012239966  \\
            0.0008838834764831845  1032.6136601902322  \\
            0.000982092751647983  1032.6180938246903  \\
            0.0010803020268127812  1032.6231019348068  \\
            0.0011785113019775794  1032.628684318894  \\
            0.0012767205771423778  1032.6348407652288  \\
            0.001374929852307176  1032.6415710531319  \\
            0.0014731391274719742  1032.648874954006  \\
            0.0015713484026367726  1032.6567522311634  \\
            0.001669557677801571  1032.665202641344  \\
            0.001767766952966369  1032.6742259354178  \\
            0.0018659762281311677  1032.6838218599833  \\
            0.001964185503295966  1032.6939901579126  \\
            0.002062394778460764  1032.7047305707956  \\
            0.0021606040536255623  1032.7160428398913  \\
            0.002258813328790361  1032.7279267083077  \\
            0.0023570226039551587  1032.740381922472  \\
            0.0024552318791199574  1032.7534082354132  \\
            0.0025534411542847556  1032.7670054075068  \\
            0.002651650429449554  1032.7811732103926  \\
            0.002749859704614352  1032.7959114289438  \\
            0.0028480689797791506  1031.5015963083179  \\
            0.0029462782549439484  1031.5151666793547  \\
            0.003044487530108747  1031.5293122371447  \\
            0.0031426968052735453  1031.5440328710233  \\
            0.0032409060804383435  1031.55932850064  \\
            0.003339115355603142  1031.575199079256  \\
            0.0034373246307679403  1031.5916445993098  \\
            0.003535533905932738  1031.6086650953546  \\
            0.0036337431810975363  1031.6262606489531  \\
            0.0037319524562623354  1031.644431394358  \\
            0.003830161731427133  1031.663177522141  \\
            0.003928371006591932  1031.6824992855654  \\
            0.004026580281756729  1031.7023970055668  \\
            0.004124789556921528  1031.72287107569  \\
            0.004222998832086326  1031.7439219693606  \\
            0.004321208107251125  1031.7655502448983  \\
            0.004419417382415923  1031.787756551565  \\
            0.004517626657580722  1031.810541636702  \\
            0.004615835932745519  1031.833906351954  \\
            0.0047140452079103175  1031.8578516600223  \\
            0.004812254483075116  1031.8823786418588  \\
            0.004910463758239915  1031.907488503509  \\
            0.005008673033404713  1031.9331825834922  \\
            0.005106882308569511  1031.959462360519  \\
            0.005205091583734309  1031.9863294607765  \\
            0.005303300858899108  1032.0137856661074  \\
            0.005401510134063906  1032.041832921374  \\
            0.005499719409228704  1032.0704733433531  \\
            0.005597928684393502  1032.0997092285395  \\
            0.005696137959558301  1032.1295430608745  \\
            0.0057943472347230995  1032.1599775219004  \\
            0.005892556509887897  1032.1910154976467  \\
            0.005990765785052695  1032.2226600885385  \\
            0.006088975060217494  1032.2549146184306  \\
            0.0061871843353822915  1032.287782642693  \\
            0.0062853936105470905  1032.321267958599  \\
            0.006383602885711889  1032.3553746139858  \\
            0.006481812160876687  1032.390106917093  \\
            0.006580021436041485  1032.425469446061  \\
            0.006678230711206284  1032.4614670590797  \\
            0.006776439986371082  1032.4981049047526  \\
            0.006874649261535881  1032.5353884313681  \\
            0.006972858536700678  1032.5733233983706  \\
            0.007071067811865476  1032.6119158868232  \\
            0.007169277087030275  1032.651172310335  \\
            0.007267486362195073  1032.691099425967  \\
            0.007365695637359872  1032.7317043462444  \\
            0.007463904912524671  1032.7729945510614  \\
            0.007562114187689468  1032.8149778993152  \\
            0.007660323462854266  1032.857662642531  \\
            0.0077585327380190645  1032.9010574365673  \\
            0.007856742013183864  1032.9451713561987  \\
            0.007954951288348661  1032.9900139085355  \\
            0.008053160563513458  1033.0355950473568  \\
            0.008151369838678257  1033.0819251889961  \\
            0.008249579113843056  1033.129015226887  \\
            0.008347788389007854  1033.176876548774  \\
            0.008445997664172653  1033.2255210539488  \\
            0.00854420693933745  1033.2749611704303  \\
            0.00864241621450225  1033.3252098742985  \\
            0.008740625489667048  1033.3762807096045  \\
            0.008838834764831846  1033.4281878083575  \\
            0.008937044039996645  1033.480945913427  \\
            0.009035253315161444  1033.534570400294  \\
            0.009133462590326241  1033.5890773029375  \\
            0.009231671865491039  1033.6444833375836  \\
            0.009329881140655836  1033.7008059316906  \\
            0.009428090415820635  1033.7580632511442  \\
            0.009526299690985434  1033.8162742315951  \\
            0.009624508966150231  1033.8754586103157  \\
            0.00972271824131503  1033.9356369603513  \\
            0.00982092751647983  1033.9968307267904  \\
            0.009919136791644627  1034.0590622653417  \\
            0.010017346066809426  1034.1223548834075  \\
            0.010115555341974223  1034.1867328836368  \\
            0.010213764617139022  1034.2522216097586  \\
            0.010311973892303821  1034.3188474965643  \\
            0.010410183167468619  1034.3866381225862  \\
            0.010508392442633416  1034.4556222651363  \\
            0.010606601717798215  1034.5258299612765  \\
            0.010704810992963013  1034.5972925712276  \\
            0.010803020268127812  1034.6700428457889  \\
            0.010901229543292609  1034.7441150000375  \\
            0.010999438818457408  1034.8195447906423  \\
            0.011097648093622207  1034.896369598981  \\
            0.011195857368787004  1034.974628519656  \\
            0.011294066643951804  1035.054362456625  \\
            0.011392275919116603  1035.1356142245093  \\
            0.0114904851942814  1035.2184286576658  \\
            0.011588694469446199  1035.3028527278252  \\
            0.011686903744610996  1031.3267870743946  \\
            0.011785113019775794  1031.3573094979342  \\
            0.011883322294940593  1031.3882436635415  \\
            0.01198153157010539  1031.4195928494862  \\
            0.01207974084527019  1031.4513604030294  \\
            0.012177950120434988  1031.483549740804  \\
            0.012276159395599787  1031.5161643490976  \\
            0.012374368670764583  1031.5492077839547  \\
            0.012472577945929382  1031.5826836714266  \\
            0.012570787221094181  1031.6165957076923  \\
            0.01266899649625898  1031.6509476601223  \\
            0.012767205771423778  1031.6857433662228  \\
            0.012865415046588575  1031.7209867360905  \\
            0.012963624321753374  1031.7566817515124  \\
            0.013061833596918171  1031.792832466819  \\
            0.01316004287208297  1031.8294430104022  \\
            0.01325825214724777  1031.8665175844233  \\
            0.013356461422412568  1031.9040604662841  \\
            0.013454670697577364  1031.9420760092567  \\
            0.013552879972742163  1031.9805686437612  \\
            0.013651089247906962  1032.0195428776888  \\
            0.013749298523071761  1032.0590032983173  \\
            0.01384750779823656  1032.0989545727034  \\
            0.013945717073401356  1032.139401449968  \\
            0.014043926348566155  1032.1803487616592  \\
            0.014142135623730952  1032.221801422841  \\
            0.014240344898895752  1032.2637644349447  \\
            0.01433855417406055  1032.306242886186  \\
            0.014436763449225346  1032.3492419535148  \\
            0.014534972724390145  1032.3927669044988  \\
            0.014633181999554944  1032.4368230983648  \\
            0.014731391274719743  1032.481415989075  \\
            0.014829600549884542  1032.5265511261582  \\
            0.014927809825049342  1032.5722341574522  \\
            0.015026019100214137  1032.6184708304581  \\
            0.015124228375378936  1032.6652669958562  \\
            0.015222437650543735  1032.7126286082362  \\
            0.015320646925708533  1032.7605617296313  \\
            0.015418856200873332  1032.8090725318266  \\
            0.015517065476038129  1032.8581672983835  \\
            0.015615274751202926  1032.9078524280344  \\
            0.015713484026367727  1032.9581344368423  \\
            0.015811693301532526  1033.0090199616538  \\
            0.015909902576697322  1033.0605157627365  \\
            0.01600811185186212  1033.1126287274146  \\
            0.016106321127026917  1033.1653658719702  \\
            0.016204530402191716  1033.218734346956  \\
            0.016302739677356515  1033.2727414386068  \\
            0.016400948952521314  1033.3273945741798  \\
            0.016499158227686113  1033.3827013240511  \\
            0.01659736750285091  1033.4386694066568  \\
            0.016695576778015708  1033.4953066919904  \\
            0.016793786053180507  1033.5526212053676  \\
            0.016891995328345306  1033.6106211324031  \\
            0.016990204603510105  1033.6693148221907  \\
            0.0170884138786749  1033.7287107925192  \\
            0.0171866231538397  1033.7888177342936  \\
            0.0172848324290045  1033.8496445160758  \\
            0.017383041704169298  1033.9112001891774  \\
            0.017481250979334097  1033.97349399213  \\
            0.017579460254498892  1034.036535356517  \\
            0.01767766952966369  1034.1003339114407  \\
            0.01777587880482849  1034.1648994896245  \\
            0.01787408807999329  1034.230242132602  \\
            0.01797229735515809  1034.2963720970145  \\
            0.018070506630322888  1034.3632998594114  \\
            0.018168715905487683  1034.4310361232854  \\
            0.018266925180652482  1034.4995918253462  \\
            0.01836513445581728  1034.5689781407866  \\
            0.018463343730982077  1034.6392064914814  \\
            0.018561553006146876  1034.7102885515174  \\
            0.018659762281311672  1034.7822362544161  \\
            0.01875797155647647  1034.8550618008956  \\
            0.01885618083164127  1034.9287776650967  \\
            0.01895439010680607  1035.3251339088908  \\
            0.019052599381970868  1035.4010049553085  \\
            0.019150808657135664  1035.4778017758024  \\
            0.019249017932300463  1035.5555378127851  \\
            0.019347227207465262  1035.6342268436345  \\
            0.01944543648263006  1035.7138829901937  \\
            0.01954364575779486  1035.7945207273485  \\
            0.01964185503295966  1035.8761548921389  \\
            0.019740064308124455  1035.9588006941426  \\
            0.019838273583289254  1036.0424737245553  \\
            0.019936482858454053  1036.1271899671985  \\
            0.020034692133618852  1036.2129658084152  \\
            0.02013290140878365  1036.2998180481054  \\
            0.020231110683948447  1036.3877639109892  \\
            0.020329319959113246  1036.476821058085  \\
            0.020427529234278045  1036.5670075981252  \\
            0.020525738509442844  1036.6583421006928  \\
            0.020623947784607643  1036.7508436078742  \\
            0.02072215705977244  1036.8445316474567  \\
            0.020820366334937238  1036.9394262469514  \\
            0.020918575610102033  1037.0355479465702  \\
            0.021016784885266832  1037.1329178141145  \\
            0.02111499416043163  1037.2315574590516  \\
            0.02121320343559643  1034.3598974704134  \\
            0.021311412710761226  1034.451041415309  \\
            0.021409621985926025  1034.5435074994184  \\
            0.021507831261090824  1034.637319309058  \\
            0.021606040536255623  1034.7325010460972  \\
            0.021704249811420422  1034.8290775456492  \\
            0.021802459086585218  1034.927074294881  \\
            0.021900668361750017  1035.0265174515957  \\
            0.021998877636914816  1035.1274338642038  \\
            0.022097086912079615  1035.2298510921023  \\
            0.022195296187244414  1035.333797427658  \\
            0.022293505462409213  1035.439301917269  \\
            0.02239171473757401  1035.5463943853322  \\
            0.022489924012738808  1035.6551054577308  \\
            0.022588133287903607  1035.7654665867574  \\
            0.022686342563068406  1035.8775100773803  \\
            0.022784551838233205  1035.9912691143995  \\
            0.022882761113398  1036.1067777908452  \\
            0.0229809703885628  1036.2240711374939  \\
            0.0230791796637276  1036.3431851540183  \\
            0.023177388938892398  1036.464156841799  \\
            0.023275598214057194  1036.5870242384933  \\
            0.023373807489221993  1036.7118264535325  \\
            0.02347201676438679  1036.8386037064931  \\
            0.023570226039551587  1042.9437227242013  \\
            0.023668435314716386  1042.4757061263124  \\
            0.023766644589881186  1042.0742013085626  \\
            0.023864853865045985  1041.7300636973616  \\
            0.02396306314021078  1041.4357092090465  \\
            0.02406127241537558  1041.1848064760861  \\
            0.02415948169054038  1040.9720376188473  \\
            0.024257690965705177  1040.792910671507  \\
            0.024355900240869977  1040.643611303343  \\
            0.024454109516034772  1040.5208847073998  \\
            0.024552318791199575  1040.421940841398  \\
            0.02465052806636437  1040.3443778905598  \\
            0.024748737341529166  1040.2861200503592  \\
            0.02484694661669397  1040.245366647031  \\
            0.024945155891858764  1040.2205502880863  \\
            0.025043365167023567  1040.2103022543301  \\
            0.025141574442188362  1040.2134237294038  \\
            0.025239783717353158  1040.2288617630768  \\
            0.02533799299251796  1040.255689092776  \\
            0.025436202267682756  1040.2930871258664  \\
            0.025534411542847555  1040.3403315235084  \\
            0.025632620818012354  1040.396779933428  \\
            0.02573083009317715  1040.461861508337  \\
            0.02582903936834195  1040.535067913643  \\
            0.025927248643506748  1040.615945577071  \\
            0.026025457918671547  1040.704088986185  \\
            0.026123667193836343  1040.799134865197  \\
            0.026221876469001145  1040.900757097516  \\
            0.02632008574416594  1041.0086622771903  \\
            0.026418295019330736  1041.1225857998243  \\
            0.02651650429449554  1041.242288408124  \\
            0.026614713569660334  1041.3675531303745  \\
            0.026712922844825137  1041.4981825528964  \\
            0.026811132119989933  1041.6339963790251  \\
            0.026909341395154728  1041.7748292356644  \\
            0.02700755067031953  1041.9205286927581  \\
            0.027105759945484326  1042.070953464907  \\
            0.02720396922064913  1042.2259717714608  \\
            0.027302178495813925  1042.385459833302  \\
            0.02740038777097872  1042.5493004866046  \\
            0.027498597046143523  1042.7173818991016  \\
            0.027596806321308318  1042.8895963731475  \\
            0.02769501559647312  1043.0658392253424  \\
            0.027793224871637916  1043.2460077303342  \\
            0.027891434146802712  1043.4300001209615  \\
            0.027989643421967515  1043.617714635943  \\
            0.02808785269713231  1043.8090486096983  \\
            0.02818606197229711  1044.003897595306  \\
            0.028284271247461905  1044.202154518798  \\
            0.028382480522626704  1044.4037088580287  \\
            0.028480689797791503  1044.60844584372  \\
            0.0285788990729563  1044.8162456777113  \\
            0.0286771083481211  1045.026982767614  \\
            0.028775317623285897  1045.240524975789  \\
            0.028873526898450692  1045.456732879067  \\
            0.028971736173615495  1045.6754590399235  \\
            0.02906994544878029  1045.8965472884095  \\
            0.029168154723945093  1046.1198320135782  \\
            0.02926636399910989  1046.3451374661618  \\
            0.02936457327427469  1046.572277070965  \\
            0.029462782549439487  1046.801052754156  \\
            0.029560991824604282  1047.031254281467  \\
            0.029659201099769085  1047.2626586149715  \\
            0.02975741037493388  1047.495029287555  \\
            0.029855619650098683  1047.728115798869  \\
            0.02995382892526348  1047.9616530368544  \\
            0.030052038200428274  1048.1953607281432  \\
            0.030150247475593077  1048.4289429217758  \\
            0.030248456750757872  1048.6620875111364  \\
            0.03034666602592267  1048.8944658005737  \\
            0.03044487530108747  1049.1257321198543  \\
            0.030543084576252266  1049.3555234959274  \\
            0.030641293851417065  1049.5834593851323  \\
            0.030739503126581864  1049.8091414779904  \\
            0.030837712401746663  1050.0321535766368  \\
            0.03093592167691146  1050.2520615621102  \\
            0.031034130952076258  1077.6127849080149  \\
            0.031132340227241057  1075.189902951735  \\
            0.031230549502405853  1073.246975742579  \\
            0.03132875877757066  1071.6543359405814  \\
            0.031426968052735454  1070.3203611238353  \\
            0.03152517732790025  1069.1785138536231  \\
            0.03162338660306505  1068.1792772061087  \\
            0.03172159587822985  1067.2849829079278  \\
            0.031819805153394644  1066.4664072975042  \\
            0.031918014428559446  1065.7004803530212  \\
            0.03201622370372424  1064.9687143271153  \\
            0.03211443297888904  1064.2561087046752  \\
            0.03221264225405383  1063.55037710804  \\
            0.032310851529218636  1062.8413958131175  \\
            0.03240906080438343  1062.1208071830295  \\
            0.032507270079548234  1061.3817327014724  \\
            0.03260547935471303  1060.6185641537502  \\
            0.032703688629877825  1059.8268106595694  \\
            0.03280189790504263  1059.002985409376  \\
            0.03290010718020742  1058.1445201765987  \\
            0.032998316455372226  1057.2496986375038  \\
            0.03309652573053702  1056.3176016548116  \\
            0.03319473500570182  1055.348059272596  \\
            0.03329294428086662  1054.3416054065515  \\
            0.033391153556031415  1053.2994322044103  \\
            0.03348936283119622  1052.223341904874  \\
            0.03358757210636101  1051.115694755735  \\
            0.03368578138152581  1049.9793522038342  \\
            0.03378399065669061  1048.8176151578127  \\
            0.03388219993185541  1047.6341576301093  \\
            0.03398040920702021  1046.4329565061169  \\
            0.034078618482185005  1045.2182185385677  \\
            0.0341768277573498  1043.9943059264667  \\
            0.0342750370325146  1042.7656620102257  \\
            0.0343732463076794  1041.5367386786306  \\
            0.0344714555828442  1040.3119270713862  \\
            0.034569664858009  1039.095493045001  \\
            0.03466787413317379  1037.8915187022633  \\
            0.034766083408338595  1036.703851044081  \\
            0.03486429268350339  1035.5360585347603  \\
            0.03496250195866819  1034.3913960787743  \\
            0.03506071123383299  1033.2727786077721  \\
            0.035158920508997785  1032.182763202956  \\
            0.03525712978416259  1031.123539414319  \\
            0.03535533905932738  1030.0969272278571  \\
            0.035453548334492185  1029.1043819543422  \\
            0.03555175760965698  1028.1470051882593  \\
            0.035649966884821783  1027.2255609010508  \\
            0.03574817615998658  1026.3404957063149  \\
            0.035846385435151375  1025.4919623274304  \\
            0.03594459471031618  1024.6798453455333  \\
            0.03604280398548097  1023.9037883651359  \\
            0.036141013260645775  1023.1632218203376  \\
            0.03623922253581057  1022.4573907445981  \\
            0.03633743181097537  1021.785381926965  \\
            0.03643564108614017  1021.1461499900618  \\
            0.036533850361304965  1020.5385420203144  \\
            0.03663205963646976  1019.9613204852184  \\
            0.03673026891163456  1019.413184256522  \\
            0.03682847818679936  1018.8927876365501  \\
            0.036926687461964154  1018.3987573534544  \\
            0.03702489673712895  1017.9297075468746  \\
            0.03712310601229375  1017.4842528094167  \\
            0.03722131528745855  1017.0610193890493  \\
            0.037319524562623343  1016.6586546779988  \\
            0.037417733837788146  1016.2758351377341  \\
            0.03751594311295294  1015.911272820033  \\
            0.037614152388117744  1015.5637206430629  \\
            0.03771236166328254  1015.231976591257  \\
            0.037810570938447335  1014.9148869928047  \\
            0.03790878021361214  1014.6113490246713  \\
            0.038006989488776934  1014.3203125798076  \\
            0.038105198763941736  1014.0407816142343  \\
            0.03820340803910653  1013.7718150737848  \\
            0.03830161731427133  1013.5125274826963  \\
            0.03839982658943613  1013.2620892572264  \\
            0.038498035864600925  1013.0197267936632  \\
            0.03859624513976573  1012.7847223762521  \\
            0.038694454414930524  1012.55641395592  \\
            0.038792663690095326  1012.3341948668383  \\
            0.03889087296526012  1012.117513589432  \\
            0.03898908224042492  1011.9058737236958  \\
            0.03908729151558972  1011.6988344191781  \\
            0.039185500790754516  1011.496011614356  \\
            0.03928371006591932  1011.297080580992  \\
            0.039381919341084114  1011.1017804581115  \\
            0.03948012861624891  1010.9099217110214  \\
            0.03957833789141371  1010.7213977985829  \\
            0.03967654716657851  1010.5362027499093  \\
            0.03977475644174331  1010.3544567510306  \\
            0.039872965716908106  1010.1764415347996  \\
            0.0399711749920729  1010.0026438226286  \\
            0.040069384267237704  1009.8337880741702  \\
            0.0401675935424025  1009.5106172015741  \\
            0.0402658028175673  1009.4653162817228  \\
            0.0403640120927321  1009.4205288466604  \\
            0.04046222136789689  1009.3762533431914  \\
            0.040560430643061696  1009.3324882100101  \\
            0.04065863991822649  1009.289231876793  \\
            0.040756849193391294  1009.2464827653197  \\
            0.04085505846855609  1009.2042392872441  \\
            0.040953267743720885  1009.1624998470257  \\
            0.04105147701888569  1009.1212628392176  \\
            0.04114968629405048  1009.0805266499003  \\
            0.041247895569215286  1009.040289656055  \\
            0.04134610484438008  1009.0005502258733  \\
            0.04144431411954488  1008.9613067185123  \\
            0.04154252339470968  1008.9225574843863  \\
            0.041640732669874475  1008.8843008647232  \\
            0.04173894194503927  1008.8465351919446  \\
            0.041837151220204066  1008.8092587890751  \\
            0.04193536049536887  1008.772469970601  \\
            0.042033569770533664  1008.7361670418506  \\
            0.04213177904569846  1008.7003482992837  \\
            0.04222998832086326  1008.6650120304455  \\
            0.04232819759602806  1008.6301565140685  \\
            0.04242640687119286  1008.5957800199479  \\
            0.042524616146357656  1008.561880809185  \\
            0.04262282542152245  1008.5284571340211  \\
            0.042721034696687255  1008.4955072383098  \\
            0.04281924397185205  1008.4630293570557  \\
            0.04291745324701685  1008.4310217169436  \\
            0.04301566252218165  1008.3994825360907  \\
            0.043113871797346444  1008.3684100246129  \\
            0.043212081072511246  1008.3378023840279  \\
            0.04331029034767604  1008.307657808004  \\
            0.043408499622840845  1008.2779744824048  \\
            0.04350670889800564  1008.2487505847813  \\
            0.043604918173170436  1008.2199842853512  \\
            0.04370312744833524  1008.1916737465859  \\
            0.043801336723500034  1008.1638171239795  \\
            0.043899545998664836  1008.1364125652228  \\
            0.04399775527382963  1008.1094582112572  \\
            0.04409596454899443  1008.0829521960446  \\
            0.04419417382415923  1008.0568926472829  \\
            0.044292383099324026  1008.0312776856205  \\
            0.04439059237448883  1008.006105425765  \\
            0.044488801649653624  1007.9813739762759  \\
            0.04458701092481843  1007.9570814399488  \\
            0.04468522019998322  1007.9332259139672  \\
            0.04478342947514802  1007.9098054899  \\
            0.04488163875031282  1007.8868182549247  \\
            0.044979848025477616  1007.8642622906957  \\
            0.04507805730064242  1007.8421356744374  \\
            0.045176266575807214  1007.820436479461  \\
            0.04527447585097201  1007.7991627746331  \\
            0.04537268512613681  1007.7783126253521  \\
            0.04547089440130161  1007.7578840934884  \\
            0.04556910367646641  1007.7378752378627  \\
            0.045667312951631206  1007.7182841141573  \\
            0.045765522226796  1007.6991087763034  \\
            0.045863731501960804  1007.6803472753584  \\
            0.0459619407771256  1007.6619976608837  \\
            0.0460601500522904  1007.6440579809307  \\
            0.0461583593274552  1007.6265262823335  \\
            0.046256568602619993  1007.6094006110044  \\
            0.046354777877784796  1007.5926790125583  \\
            0.04645298715294959  1007.5763595324642  \\
            0.04655119642811439  1007.5604402163007  \\
            0.04664940570327918  1007.5449191104992  \\
            0.046747614978443985  1007.5297942620867  \\
            0.04684582425360878  1007.5150637201147  \\
            0.04694403352877358  1007.5007255344575  \\
            0.04704224280393838  1007.4867777575313  \\
            0.047140452079103175  1007.4732184442206  \\
            0.04723866135426797  1007.4600456520184  \\
            0.04733687062943277  1007.4472574417222  \\
            0.04743507990459757  1007.4348518775015  \\
            0.04753328917976237  1007.4228270277038  \\
            0.04763149845492717  1007.4111809644421  \\
            0.04772970773009197  1007.3999117650941  \\
            0.047827917005256765  1007.38901751127  \\
            0.04792612628042156  1007.3784962904086  \\
            0.04802433555558636  1007.3683461956401  \\
            0.04812254483075116  1007.3585653254903  \\
            0.04822075410591596  1007.3491517855869  \\
            0.04831896338108076  1007.3401036874543  \\
            0.04841717265624555  1007.3314191502051  \\
            0.048515381931410355  1007.3230963000684  \\
            0.04861359120657515  1007.3151332704642  \\
            0.04871180048173995  1007.3075282031897  \\
            0.04881000975690475  1007.3002792476767  \\
            0.048908219032069544  1007.2933845623307  \\
            0.04900642830723435  1007.2868423138679  \\
            0.04910463758239915  1007.2806506777588  \\
            0.049202846857563945  1007.2748078390462  \\
            0.04930105613272874  1007.2693119917241  \\
            0.049399265407893536  1007.2641613398747  \\
            0.04949747468305833  1007.2593540968156  \\
            0.04959568395822314  1007.2548884860079  \\
            0.04969389323338794  1007.2507627409196  \\
            0.04979210250855273  1007.2469751056426  \\
            0.04989031178371753  1007.243523834038  \\
            0.049988521058882324  1007.2404071908144  \\
            0.05008673033404713  1007.2376234513711  \\
            0.05018493960921193  1007.2351709013897  \\
            0.050283148884376724  1007.2330478376542  \\
            0.05038135815954152  1007.2312525673996  \\
            0.050479567434706316  1007.2297834089132  \\
            0.050577776709871125  1007.2286386912133  \\
            0.05067598598503592  1007.2278167539157  \\
            0.050774195260200716  1007.2273159476383  \\
            0.05087240453536551  1007.2271346340513  \\
            0.05097061381053031  1007.2272711845344  \\
            0.05106882308569511  1007.2277239819374  \\
            0.05116703236085991  1007.2284914194476  \\
            0.05126524163602471  1007.2295719001916  \\
            0.051363450911189504  1007.2309638380916  \\
            0.0514616601863543  1007.2326656568041  \\
            0.0515598694615191  1007.2346757900369  \\
            0.0516580787366839  1007.2369926815154  \\
            0.05175628801184869  1007.2396147840242  \\
            0.051854497287013496  1007.2425405602753  \\
            0.05195270656217829  1007.2457684818143  \\
            0.052050915837343094  1007.2492970288783  \\
            0.05214912511250789  1007.2531246910446  \\
            0.052247334387672685  1007.2572499653646  \\
            0.05234554366283748  1007.2616713579588  \\
            0.05244375293800229  1007.2663873821558  \\
            0.052541962213167086  1007.2713965588586  \\
            0.05264017148833188  1007.2766974164464  \\
            0.05273838076349668  1007.282288490148  \\
            0.05283659003866147  1007.2881683218849  \\
            0.05293479931382628  1007.294335459496  \\
            0.05303300858899108  1007.3007884571308  \\
            0.05313121786415587  1007.3075258742477  \\
            0.05322942713932067  1007.3145462761445  \\
            0.053327636414485464  1007.321848232177  \\
            0.053425845689650274  1007.3294303169816  \\
            0.05352405496481507  1007.3372911090705  \\
            0.053622264239979865  1007.3454291910409  \\
            0.05372047351514466  1007.3538431487701  \\
            0.053818682790309456  1007.3625315714547  \\
            0.053916892065474266  1007.3714930510564  \\
            0.05401510134063906  1007.3807261822776  \\
            0.05411331061580386  1007.3902295616308  \\
            0.05421151989096865  1007.4000017878565  \\
            0.05430972916613345  1007.4100414609192  \\
            0.05440793844129826  1007.4203471820352  \\
            0.05450614771646305  1007.4309175535844  \\
            0.05460435699162785  1007.4417511782499  \\
            0.054702566266792645  1007.4528466592614  \\
            0.05480077554195744  1007.4642025996706  \\
            0.05489898481712225  1007.4758176027638  \\
            0.054997194092287045  1007.4876902708821  \\
            0.05509540336745184  1007.4998192059279  \\
            0.055193612642616637  1007.5122030092026  \\
            0.05529182191778143  1007.5248402804466  \\
            0.05539003119294624  1007.5377296182062  \\
            0.05548824046811104  1007.5508696201482  \\
            0.05558644974327583  1007.564258881577  \\
            0.05568465901844063  1007.5778959964429  \\
            0.055782868293605424  1007.5917795570181  \\
            0.05588107756877023  1007.605908153472  \\
            0.05597928684393503  1007.6202803740475  \\
            0.056077496119099825  1007.6348948047137  \\
            0.05617570539426462  1007.6497500296972  \\
            0.056273914669429416  1007.6648446307299  \\
            0.05637212394459422  1007.6801771880702  \\
            0.056470333219759014  1007.6957462792119  \\
            0.05656854249492381  1007.7115504801098  \\
            0.05666675177008861  1007.7275883646179  \\
            0.05676496104525341  1007.7438585047863  \\
            0.05686317032041821  1007.7603594708528  \\
            0.056961379595583006  1007.7770898314752  \\
            0.0570595888707478  1007.7940481543213  \\
            0.0571577981459126  1007.8112330050214  \\
            0.0572560074210774  1007.8286429488525  \\
            0.0573542166962422  1007.8462765501687  \\
            0.057452425971407  1007.8641323723189  \\
            0.057550635246571794  1007.8822089787434  \\
            0.05764884452173659  1007.9005049328874  \\
            0.057747053796901385  1007.91901879847  \\
            0.057845263072066194  1007.9377491397667  \\
            0.05794347234723099  1007.9566945220764  \\
            0.058041681622395785  1007.9758535125129  \\
            0.05813989089756058  1007.9952246792398  \\
            0.05823810017272538  1008.0148065929708  \\
            0.058336309447890186  1008.034597827382  \\
            0.05843451872305498  1008.0545969587828  \\
            0.05853272799821978  1008.0748025673442  \\
            0.05863093727338457  1008.0952132371292  \\
            0.05872914654854938  1008.1158275570191  \\
            0.05882735582371418  1008.1366441206692  \\
            0.058925565098878974  1008.1576615277605  \\
            0.05902377437404377  1008.17887838411  \\
            0.059121983649208565  1008.2002933022433  \\
            0.059220192924373374  1008.221904902426  \\
            0.05931840219953817  1008.243711812715  \\
            0.059416611474702966  1008.2657126700274  \\
            0.05951482074986776  1008.2879061207021  \\
            0.05961303002503256  1008.3102908207738  \\
            0.059711239300197366  1008.3328654373639  \\
            0.05980944857536216  1008.355628648711  \\
            0.05990765785052696  1008.3785791455964  \\
            0.06000586712569175  1008.4017156313568  \\
            0.06010407640085655  1008.4250368229314  \\
            0.06020228567602136  1008.4485414518219  \\
            0.060300494951186154  1008.4722282645073  \\
            0.06039870422635095  1008.4960960235695  \\
            0.060496913501515745  1008.520143507903  \\
            0.06059512277668054  1008.5443695147919  \\
            0.06069333205184534  1008.568772858924  \\
            0.060791541327010146  1008.5933523750904  \\
            0.06088975060217494  1008.6181069174061  \\
            0.06098795987733974  1008.6430353615874  \\
            0.06108616915250453  1008.668136604859  \\
            0.061184378427669335  1008.6934095669927  \\
            0.06128258770283413  1008.7188531917744  \\
            0.061380796977998926  1008.7458841837445  \\
            0.06147900625316373  1008.7704291063883  \\
            0.061577215528328524  1008.795190014055  \\
            0.06167542480349333  1008.8201652443568  \\
            0.06177363407865812  1008.8453531377929  \\
            0.06187184335382292  1008.8707520377943  \\
            0.061970052628987714  1008.8963602909157  \\
            0.062068261904152516  1008.9221762467776  \\
            0.06216647117931732  1008.9481982587002  \\
            0.062264680454482114  1008.9744246838425  \\
            0.06236288972964691  1009.0008538825103  \\
            0.062461099004811706  1009.0274842191479  \\
            0.06255930827997651  1009.0543140620795  \\
            0.06265751755514132  1009.0813417837976  \\
            0.06275572683030611  1009.108565760619  \\
            0.06285393610547091  1009.1359843735022  \\
            0.0629521453806357  1009.1635960069401  \\
            0.0630503546558005  1009.1913990505686  \\
            0.0631485639309653  1009.2193918978222  \\
            0.0632467732061301  1009.2475729467379  \\
            0.0633449824812949  1009.2759406000123  \\
            0.0634431917564597  1009.304493264437  \\
            0.06354140103162449  1009.3332293519225  \\
            0.06363961030678929  1009.3621472783221  \\
            0.06373781958195408  1009.3912454644718  \\
            0.06383602885711889  1009.4205223355262  \\
            0.06393423813228369  1009.4499763217776  \\
            0.06403244740744848  1009.4796058577552  \\
            0.06413065668261328  1009.5094093825537  \\
            0.06422886595777808  1009.539385340071  \\
            0.06432707523294287  1009.5695321790098  \\
            0.06442528450810767  1009.5998483524703  \\
            0.06452349378327248  1009.6303323183088  \\
            0.06462170305843727  1009.660982538791  \\
            0.06471991233360207  1009.6917974813757  \\
            0.06481812160876686  1009.7227756174076  \\
            0.06491633088393166  1009.7539154233631  \\
            0.06501454015909647  1009.7852153801587  \\
            0.06511274943426126  1009.8166739733672  \\
            0.06521095870942606  1009.8482896925733  \\
            0.06530916798459085  1009.8800610327969  \\
            0.06540737725975565  1009.9119864926321  \\
            0.06550558653492046  1009.9440645758913  \\
            0.06560379581008526  1009.9762937906064  \\
            0.06570200508525005  1010.0086726491295  \\
            0.06580021436041485  1010.0411996682249  \\
            0.06589842363557964  1010.0738733692352  \\
            0.06599663291074445  1010.1066922776055  \\
            0.06609484218590925  1010.1396549234428  \\
            0.06619305146107404  1010.1727598409522  \\
            0.06629126073623884  1010.2060055685738  \\
            0.06638947001140363  1010.2393906490522  \\
            0.06648767928656844  1010.2729136294485  \\
            0.06658588856173324  1010.3065730610275  \\
            0.06668409783689803  1010.3403674990045  \\
            0.06678230711206283  1010.374295502838  \\
            0.06688051638722763  1010.4083556363911  \\
            0.06697872566239244  1010.442546467285  \\
            0.06707693493755723  1010.476866567313  \\
            0.06717514421272203  1010.5113145123258  \\
            0.06727335348788682  1010.5458888821773  \\
            0.06737156276305162  1010.5805882607377  \\
            0.06746977203821643  1010.6154112357519  \\
            0.06756798131338122  1010.6503563992045  \\
            0.06766619058854602  1010.6854223467476  \\
            0.06776439986371081  1010.7206076779391  \\
            0.06786260913887561  1010.7559109965084  \\
            0.06796081841404042  1010.7913309100143  \\
            0.06805902768920521  1010.8268660295028  \\
            0.06815723696437001  1010.8625149703756  \\
            0.0682554462395348  1010.8982763518779  \\
            0.0683536555146996  1010.9341487968479  \\
            0.06845186478986441  1010.9701309319547  \\
            0.0685500740650292  1011.0062213880109  \\
            0.068648283340194  1011.0424187996401  \\
            0.0687464926153588  1011.078721804967  \\
            0.0688447018905236  1011.1151290464683  \\
            0.0689429111656884  1011.1516391700882  \\
            0.0690411204408532  1011.1882508260262  \\
            0.069139329716018  1011.2249626681901  \\
            0.06923753899118279  1011.2617733544662  \\
            0.06933574826634759  1011.2986815465528  \\
            0.0694339575415124  1011.3356859104208  \\
            0.06953216681667719  1011.3727851156908  \\
            0.06963037609184199  1011.4099778365127  \\
            0.06972858536700678  1011.447262750656  \\
            0.06982679464217158  1011.4846385401202  \\
            0.06992500391733639  1011.5221038915588  \\
            0.07002321319250118  1011.5596574951057  \\
            0.07012142246766598  1011.5972980460284  \\
            0.07021963174283077  1011.6350242431708  \\
            0.07031784101799557  1011.6728347904394  \\
            0.07041605029316038  1011.7107283961142  \\
            0.07051425956832517  1011.7487037727544  \\
            0.07061246884348997  1011.7867596381765  \\
            0.07071067811865477  1011.8248947144522  \\
            0.07080888739381958  1011.863107729163  \\
            0.07090709666898437  1011.9013974144473  \\
            0.07100530594414917  1011.9397625078357  \\
            0.07110351521931396  1011.9782017521434  \\
            0.07120172449447876  1012.0167138956867  \\
            0.07129993376964357  1012.0552976922371  \\
            0.07139814304480836  1012.0939519014163  \\
            0.07149635231997316  1012.1326752888972  \\
            0.07159456159513795  1012.1714666263139  \\
            0.07169277087030275  1012.2103246919578  \\
            0.07179098014546756  1012.2492482703251  \\
            0.07188918942063235  1012.2882361530442  \\
            0.07198739869579715  1012.3272871384901  \\
            0.07208560797096195  1012.3664000326663  \\
            0.07218381724612674  1012.4055736489339  \\
            0.07228202652129155  1012.4448068086484  \\
            0.07238023579645635  1012.4840983414931  \\
            0.07247844507162114  1012.5234470856428  \\
            0.07257665434678594  1012.5628518880031  \\
            0.07267486362195073  1012.6023116051501  \\
            0.07277307289711553  1011.4741395004393  \\
            0.07287128217228034  1011.5071094690352  \\
            0.07296949144744513  1011.5401645930965  \\
            0.07306770072260993  1011.5733027263128  \\
            0.07316590999777473  1011.6065217457713  \\
            0.07326411927293952  1011.6398195524571  \\
            0.07336232854810432  1011.6731940696629  \\
            0.07346053782326913  1011.7066432433345  \\
            0.07355874709843392  1011.7401650412975  \\
            0.07365695637359872  1011.7737574529117  \\
            0.07375516564876351  1011.8074184887681  \\
            0.07385337492392831  1011.841146179614  \\
            0.0739515841990931  1011.8749385771796  \\
            0.0740497934742579  1011.9087937525322  \\
            0.07414800274942271  1011.9427097965539  \\
            0.0742462120245875  1011.9766848189552  \\
            0.0743444212997523  1012.0107169485309  \\
            0.0744426305749171  1012.0448043324635  \\
            0.07454083985008189  1012.0789451356588  \\
            0.07463904912524669  1012.1131375411304  \\
            0.0747372584004115  1012.1473797490718  \\
            0.07483546767557629  1012.1816699769291  \\
            0.07493367695074109  1012.2160064586275  \\
            0.07503188622590588  1012.2503874447169  \\
            0.07513009550107068  1012.2848112018722  \\
            0.07522830477623549  1012.3192760126229  \\
            0.07532651405140028  1012.3537801745618  \\
            0.07542472332656508  1012.3883220011074  \\
            0.07552293260172988  1012.422899820147  \\
            0.07562114187689467  1012.457511974459  \\
            0.07571935115205948  1012.4921568209002  \\
            0.07581756042722428  1012.5268327305309  \\
            0.07591576970238907  1012.5615380881591  \\
            0.07601397897755387  1012.5962712920448  \\
            0.07611218825271866  1012.6310307535745  \\
            0.07621039752788347  1012.6658148973441  \\
            0.07630860680304827  1012.7006221600695  \\
            0.07640681607821306  1012.7354509913165  \\
            0.07650502535337786  1012.7702998525909  \\
            0.07660323462854265  1012.8051672169572  \\
            0.07670144390370746  1012.8400515696154  \\
            0.07679965317887226  1012.8749514063655  \\
            0.07689786245403706  1012.9098652346665  \\
            0.07699607172920185  1012.9447915722847  \\
            0.07709428100436666  1012.9797289474959  \\
            0.07719249027953146  1013.0146758992428  \\
            0.07729069955469625  1013.0496309756804  \\
            0.07738890882986105  1013.0845927354318  \\
            0.07748711810502584  1013.1195597460578  \\
            0.07758532738019065  1013.1545305845624  \\
            0.07768353665535545  1013.189503836389  \\
            0.07778174593052024  1013.2244780964403  \\
            0.07787995520568504  1013.2594519672076  \\
            0.07797816448084983  1013.2944240599277  \\
            0.07807637375601464  1013.3293929935788  \\
            0.07817458303117944  1013.3643573945233  \\
            0.07827279230634424  1013.3993158967859  \\
            0.07837100158150903  1013.4342671414927  \\
            0.07846921085667383  1013.469209776854  \\
            0.07856742013183864  1013.5041424575013  \\
            0.07866562940700343  1013.5390638446835  \\
            0.07876383868216823  1013.5739726059538  \\
            0.07886204795733302  1013.6088674142891  \\
            0.07896025723249782  1013.6437469492755  \\
            0.07905846650766263  1013.6786098956871  \\
            0.07915667578282742  1013.7134549433075  \\
            0.07925488505799222  1013.748280787437  \\
            0.07935309433315701  1013.7830861284763  \\
            0.07945130360832181  1013.8178696709281  \\
            0.07954951288348662  1013.852630124618  \\
            0.07964772215865142  1013.8873662031536  \\
            0.07974593143381621  1013.9220766244861  \\
            0.079844140708981  1013.9567601107844  \\
            0.0799423499841458  1013.9914153880256  \\
            0.08004055925931061  1014.0260411855539  \\
            0.08013876853447541  1014.0606362370194  \\
            0.0802369778096402  1014.0951992786745  \\
            0.080335187084805  1014.1297290506891  \\
            0.0804333963599698  1014.1642242963972  \\
            0.0805316056351346  1014.19868376212  \\
            0.0806298149102994  1014.2331061971539  \\
            0.0807280241854642  1014.2674903541053  \\
            0.08082623346062899  1014.301834988056  \\
            0.08092444273579379  1014.3361388571378  \\
            0.0810226520109586  1014.3704007224875  \\
            0.08112086128612339  1014.4046193476878  \\
            0.08121907056128819  1014.4387934996768  \\
            0.08131727983645298  1014.4729219475086  \\
            0.08141548911161778  1014.5070034637055  \\
            0.08151369838678259  1014.5410368231035  \\
            0.08161190766194738  1014.5750208041056  \\
            0.08171011693711218  1014.6089541875076  \\
            0.08180832621227697  1014.642835757756  \\
            0.08190653548744177  1014.6766643023486  \\
            0.08200474476260658  1014.7104386118958  \\
            0.08210295403777138  1014.7441574807619  \\
            0.08220116331293617  1014.7778197071889  \\
            0.08229937258810097  1014.8114240927388  \\
            0.08239758186326576  1014.8449694436681  \\
            0.08249579113843057  1014.8784545701012  \\
            0.08259400041359537  1014.9118782872661  \\
            0.08269220968876016  1014.945239414626  \\
            0.08279041896392496  1014.978536777249  \\
            0.08288862823908975  1015.0117692054616  \\
            0.08298683751425455  1015.044935535863  \\
            0.08308504678941936  1015.0780346106021  \\
            0.08318325606458415  1015.1110652787785  \\
            0.08328146533974895  1015.1440263968641  \\
            0.08337967461491375  1015.1769168281587  \\
            0.08347788389007854  1015.2097354442163  \\
            0.08357609316524334  1015.2424811253801  \\
            0.08367430244040813  1015.2751527603086  \\
            0.08377251171557294  1015.3077492475253  \\
            0.08387072099073774  1015.3402694960425  \\
            0.08396893026590253  1015.3727124248561  \\
            0.08406713954106733  1013.754314971553  \\
            0.08416534881623212  1013.7901862758362  \\
            0.08426355809139692  1013.8259036306019  \\
            0.08436176736656173  1013.8614654753073  \\
            0.08445997664172653  1013.8968702472324  \\
            0.08455818591689132  1013.9321163810077  \\
            0.08465639519205612  1013.9672023088389  \\
            0.08475460446722091  1014.0021264598652  \\
            0.08485281374238572  1014.0368872603956  \\
            0.08495102301755052  1014.0714831337981  \\
            0.08504923229271531  1014.105912500637  \\
            0.08514744156788011  1014.1401737783788  \\
            0.0852456508430449  1014.1742653816835  \\
            0.08534386011820971  1014.2081857219073  \\
            0.08544206939337451  1014.2419332082405  \\
            0.0855402786685393  1014.2755062466681  \\
            0.0856384879437041  1014.3089032405976  \\
            0.0857366972188689  1014.3421225909285  \\
            0.0858349064940337  1014.375162696314  \\
            0.0859331157691985  1014.4080219533772  \\
            0.0860313250443633  1014.4406987561462  \\
            0.08612953431952809  1014.4731914972318  \\
            0.08622774359469289  1014.505498567606  \\
            0.0863259528698577  1014.5376183567511  \\
            0.08642416214502249  1014.5695492528398  \\
            0.08652237142018729  1014.6012896431971  \\
            0.08662058069535208  1014.6328379146642  \\
            0.08671878997051688  1014.6641924532871  \\
            0.08681699924568169  1014.6953516451039  \\
            0.08691520852084648  1014.7263138764471  \\
            0.08701341779601128  1014.7570775339549  \\
            0.08711162707117608  1014.7876410050629  \\
            0.08720983634634087  1014.8180026780926  \\
            0.08730804562150568  1014.8481609428518  \\
            0.08740625489667048  1014.8781141908772  \\
            0.08750446417183527  1014.9078608154466  \\
            0.08760267344700007  1014.9373992123351  \\
            0.08770088272216486  1014.9667277801879  \\
            0.08779909199732967  1014.9958449202411  \\
            0.08789730127249447  1015.0247490375929  \\
            0.08799551054765926  1015.0534385403055  \\
            0.08809371982282406  1015.0819118413606  \\
            0.08819192909798886  1015.1101673572216  \\
            0.08829013837315366  1015.1382035099591  \\
            0.08838834764831846  1015.1660187261144  \\
            0.08848655692348326  1015.1936114378586  \\
            0.08858476619864805  1015.2209800832487  \\
            0.08868297547381285  1015.2481231063033  \\
            0.08878118474897766  1015.2750389574426  \\
            0.08887939402414245  1015.3017260943548  \\
            0.08897760329930725  1015.3281829815094  \\
            0.08907581257447204  1015.3544080909817  \\
            0.08917402184963685  1015.3803999028946  \\
            0.08927223112480165  1015.4061569052586  \\
            0.08937044039996644  1015.4316775946685  \\
            0.08946864967513124  1015.4569604766934  \\
            0.08956685895029604  1015.4820040663155  \\
            0.08966506822546084  1015.5068068876536  \\
            0.08976327750062564  1015.5313674747526  \\
            0.08986148677579044  1015.5556843717347  \\
            0.08995969605095523  1015.5797561333655  \\
            0.09005790532612003  1015.6035813251456  \\
            0.09015611460128484  1015.6271585234296  \\
            0.09025432387644963  1015.6504863161645  \\
            0.09035253315161443  1015.6735633027963  \\
            0.09045074242677922  1015.69638809453  \\
            0.09054895170194402  1015.718959315153  \\
            0.09064716097710883  1015.7412756007936  \\
            0.09074537025227362  1015.7633356000954  \\
            0.09084357952743842  1015.7851379748464  \\
            0.09094178880260322  1015.8066814000995  \\
            0.09103999807776801  1015.8279645644741  \\
            0.09113820735293282  1015.8489861701802  \\
            0.09123641662809762  1015.8697449333736  \\
            0.09133462590326241  1015.8902395848057  \\
            0.09143283517842721  1015.9104688689164  \\
            0.091531044453592  1015.9304315456516  \\
            0.09162925372875681  1015.9501263889395  \\
            0.09172746300392161  1015.969552188401  \\
            0.0918256722790864  1015.9887077487796  \\
            0.0919238815542512  1016.0075918898607  \\
            0.092022090829416  1016.0262034474697  \\
            0.0921203001045808  1016.0445412731814  \\
            0.0922185093797456  1016.0626042345198  \\
            0.0923167186549104  1016.0803912148901  \\
            0.09241492793007519  1016.0979011141503  \\
            0.09251313720523999  1016.1151328489492  \\
            0.09261134648040478  1016.1320853519211  \\
            0.09270955575556959  1016.1487575727979  \\
            0.09280776503073439  1016.165148478414  \\
            0.09290597430589918  1016.1812570519418  \\
            0.09300418358106398  1016.197082294182  \\
            0.09310239285622877  1016.2126232231926  \\
            0.09320060213139357  1016.2278788742636  \\
            0.09329881140655837  1016.2428483001139  \\
            0.09339702068172318  1016.2575305711799  \\
            0.09349522995688797  1016.2719247755409  \\
            0.09359343923205277  1016.2860300187351  \\
            0.09369164850721756  1016.2998454247727  \\
            0.09378985778238236  1016.3133701352273  \\
            0.09388806705754715  1016.3266033098649  \\
            0.09398627633271196  1016.3395441261393  \\
            0.09408448560787676  1016.3521917804434  \\
            0.09418269488304155  1016.3645454867708  \\
            0.09428090415820635  1016.3766044777158  \\
            0.09437911343337115  1016.3883680040786  \\
            0.09447732270853594  1016.3998353356316  \\
            0.09457553198370075  1016.4110057598564  \\
            0.09467374125886555  1016.4218785835297  \\
            0.09477195053403034  1016.4324531312324  \\
            0.09487015980919514  1016.4427287470934  \\
            0.09496836908435993  1016.452704793069  \\
            0.09506657835952474  1016.4623806505849  \\
            0.09516478763468954  1016.4717557190922  \\
            0.09526299690985433  1016.4808294175331  \\
            0.09536120618501913  1016.4896011833653  \\
            0.09545941546018394  1016.4980704724724  \\
            0.09555762473534873  1016.5062367607037  \\
            0.09565583401051353  1016.5140995415909  \\
            0.09575404328567833  1016.5216583285452  \\
            0.09585225256084312  1016.5289126535953  \\
            0.09595046183600793  1016.5358620673736  \\
            0.09604867111117273  1016.5425061402161  \\
            0.09614688038633752  1016.5488444609363  \\
            0.09624508966150232  1016.5548766374399  \\
            0.09634329893666711  1016.5606022972144  \\
            0.09644150821183192  1016.5660210859776  \\
            0.09653971748699672  1016.5711326689936  \\
            0.09663792676216151  1016.5759367306488  \\
            0.09673613603732631  1016.5804329741578  \\
            0.0968343453124911  1016.5846211220395  \\
            0.09693255458765591  1016.5885009158828  \\
            0.09703076386282071  1016.5920721163009  \\
            0.0971289731379855  1016.5953345031786  \\
            0.0972271824131503  1016.5982878751113  \\
            0.0973253916883151  1016.6009320506018  \\
            0.0974236009634799  1016.6032668664129  \\
            0.0975218102386447  1016.6052921789945  \\
            0.0976200195138095  1016.6070078637442  \\
            0.09771822878897429  1016.6084138152985  \\
            0.09781643806413909  1016.6095099475303  \\
            0.0979146473393039  1016.610296192869  \\
            0.0980128566144687  1016.610772503813  \\
            0.09811106588963349  1016.6109388512815  \\
        }
        ;
    \addplot[color={rgb,1:red,1.0;green,0.0;blue,0.0}, name path={cf810885-7c2d-4fe4-ad8a-dc5e4ea278b7}, draw opacity={1.0}, line width={1}, solid]
        table[row sep={\\}]
        {
            \\
            0.0  3.031007597951104  \\
            9.820927516479829e-5  3.031004198482724  \\
            0.00019641855032959658  3.03099783696499  \\
            0.00029462782549439484  3.0309885102632577  \\
            0.00039283710065919316  3.0309762137297582  \\
            0.0004910463758239915  3.0309609417658083  \\
            0.0005892556509887897  3.0316375887978935  \\
            0.000687464926153588  3.031963664122244  \\
            0.0007856742013183863  3.032304134238337  \\
            0.0008838834764831845  3.0326305241033746  \\
            0.000982092751647983  3.0329442678834035  \\
            0.0010803020268127812  3.033246146560827  \\
            0.0011785113019775794  3.033536794706182  \\
            0.0012767205771423778  3.033816878077495  \\
            0.001374929852307176  3.034086923382811  \\
            0.0014731391274719742  3.0343473995216117  \\
            0.0015713484026367726  3.034598730805687  \\
            0.001669557677801571  3.0348412896229715  \\
            0.001767766952966369  3.035075408764521  \\
            0.0018659762281311677  3.035301378982894  \\
            0.001964185503295966  3.035519460429534  \\
            0.002062394778460764  3.035729881472976  \\
            0.0021606040536255623  3.0359328451323195  \\
            0.002258813328790361  3.0361285303485763  \\
            0.0023570226039551587  3.0363170923303837  \\
            0.0024552318791199574  3.0364986662087583  \\
            0.0025534411542847556  3.0366733693221333  \\
            0.002651650429449554  3.036841299392644  \\
            0.002749859704614352  3.0370025389402224  \\
            0.0028480689797791506  3.038104907952734  \\
            0.0029462782549439484  3.038368403127107  \\
            0.003044487530108747  3.038605796469933  \\
            0.0031426968052735453  3.0388101701493295  \\
            0.0032409060804383435  3.0390010710953197  \\
            0.003339115355603142  3.0391883495707503  \\
            0.0034373246307679403  3.039370053238496  \\
            0.003535533905932738  3.0395441364727085  \\
            0.0036337431810975363  3.0397102602317387  \\
            0.0037319524562623354  3.039868643397295  \\
            0.003830161731427133  3.040020127062537  \\
            0.003928371006591932  3.0401636771164977  \\
            0.004026580281756729  3.040299614154317  \\
            0.004124789556921528  3.040427355181731  \\
            0.004222998832086326  3.040547085048053  \\
            0.004321208107251125  3.0406583950752952  \\
            0.004419417382415923  3.040761010449667  \\
            0.004517626657580722  3.040854690084123  \\
            0.004615835932745519  3.040939068020058  \\
            0.0047140452079103175  3.0410137300301683  \\
            0.004812254483075116  3.0410782355220882  \\
            0.004910463758239915  3.0411321384586163  \\
            0.005008673033404713  3.0411748595069836  \\
            0.005106882308569511  3.041205765390401  \\
            0.005205091583734309  3.0412241072243082  \\
            0.005303300858899108  3.041229050728529  \\
            0.005401510134063906  3.0412196687408697  \\
            0.005499719409228704  3.0411948490820775  \\
            0.005597928684393502  3.04115333953084  \\
            0.005696137959558301  3.041093645542457  \\
            0.0057943472347230995  3.0410140920118076  \\
            0.005892556509887897  3.04091267587693  \\
            0.005990765785052695  3.0407870421903307  \\
            0.006088975060217494  3.0406344262838694  \\
            0.0061871843353822915  3.0404514978584127  \\
            0.0062853936105470905  3.040234277419542  \\
            0.006383602885711889  3.0399779387694146  \\
            0.006481812160876687  3.0396765685765508  \\
            0.006580021436041485  3.0393228601379696  \\
            0.006678230711206284  3.038907692348891  \\
            0.006776439986371082  3.0384195021075433  \\
            0.006874649261535881  3.0378434441003552  \\
            0.006972858536700678  3.0371601970264392  \\
            0.007071067811865476  3.0363441312419313  \\
            0.007169277087030275  3.035360683938954  \\
            0.007267486362195073  3.0341623776260933  \\
            0.007365695637359872  3.0326829098040986  \\
            0.007463904912524671  3.030828965574167  \\
            0.007562114187689468  3.0284720619435315  \\
            0.007660323462854266  3.025459980216394  \\
            0.0077585327380190645  3.021778719029684  \\
            0.007856742013183864  3.018925793666584  \\
            0.007954951288348661  3.037291948085294  \\
            0.008053160563513458  4.692340672042638  \\
            0.008151369838678257  3.425675982720836  \\
            0.008249579113843056  3.1618998773180578  \\
            0.008347788389007854  3.1110018905352286  \\
            0.008445997664172653  3.090769085017331  \\
            0.00854420693933745  3.0801029829132247  \\
            0.00864241621450225  3.0735527352232896  \\
            0.008740625489667048  3.0691228052979205  \\
            0.008838834764831846  3.0659188315411283  \\
            0.008937044039996645  3.0634825958333822  \\
            0.009035253315161444  3.061555783838992  \\
            0.009133462590326241  3.0599818938227044  \\
            0.009231671865491039  3.058660598079828  \\
            0.009329881140655836  3.057524582660986  \\
            0.009428090415820635  3.0565269483616357  \\
            0.009526299690985434  3.055633979102563  \\
            0.009624508966150231  3.054820791882807  \\
            0.00972271824131503  3.054068641134182  \\
            0.00982092751647983  3.053363196527462  \\
            0.009919136791644627  3.0526934517106743  \\
            0.010017346066809426  3.0520510464208503  \\
            0.010115555341974223  3.051429902172876  \\
            0.010213764617139022  3.0508261275333393  \\
            0.010311973892303821  3.050238199169987  \\
            0.010410183167468619  3.0496674811749522  \\
            0.010508392442633416  3.0491192365147373  \\
            0.010606601717798215  3.0486044001253925  \\
            0.010704810992963013  3.0481426425066234  \\
            0.010803020268127812  3.0477677145637534  \\
            0.010901229543292609  3.047537058066413  \\
            0.010999438818457408  3.047549821102672  \\
            0.011097648093622207  3.0479825613909717  \\
            0.011195857368787004  3.0491652360974055  \\
            0.011294066643951804  3.051758700005692  \\
            0.011392275919116603  3.057224692537131  \\
            0.0114904851942814  3.0693135343366786  \\
            0.011588694469446199  3.1003023232774796  \\
            0.011686903744610996  3.218164141103442  \\
            0.011785113019775794  4.0612544452879416  \\
            0.011883322294940593  4.880284474361792  \\
            0.01198153157010539  3.1572798335924457  \\
            0.01207974084527019  3.070800680558604  \\
            0.012177950120434988  3.236194919524907  \\
            0.012276159395599787  10.312758619461293  \\
            0.012374368670764583  4.408310326054701  \\
            0.012472577945929382  3.4902399546854657  \\
            0.012570787221094181  3.299776974551504  \\
            0.01266899649625898  3.2246097426171563  \\
            0.012767205771423778  3.1857764371864943  \\
            0.012865415046588575  3.1624788995357447  \\
            0.012963624321753374  3.147082105932371  \\
            0.013061833596918171  3.136167163736236  \\
            0.01316004287208297  3.1279815601995846  \\
            0.01325825214724777  3.1215314850948466  \\
            0.013356461422412568  3.1162048735511623  \\
            0.013454670697577364  3.111595700893622  \\
            0.013552879972742163  3.1074140848859546  \\
            0.013651089247906962  3.1034363189415175  \\
            0.013749298523071761  3.09947469652741  \\
            0.01384750779823656  3.095357556580507  \\
            0.013945717073401356  3.090914785011504  \\
            0.014043926348566155  3.0859666829426393  \\
            0.014142135623730952  3.0803165814523896  \\
            0.014240344898895752  3.073752275187886  \\
            0.01433855417406055  3.0660744603552454  \\
            0.014436763449225346  3.057213204121067  \\
            0.014534972724390145  3.04765142806401  \\
            0.014633181999554944  3.0400536416576744  \\
            0.014731391274719743  3.046638263215332  \\
            0.014829600549884542  3.134201403825188  \\
            0.014927809825049342  3.9104921937218284  \\
            0.015026019100214137  281.6248573590398  \\
            0.015124228375378936  5.231216200422359  \\
            0.015222437650543735  4.015473588935146  \\
            0.015320646925708533  3.6849477757497264  \\
            0.015418856200873332  3.5421977878165536  \\
            0.015517065476038129  3.4661839564207346  \\
            0.015615274751202926  3.4208877018189123  \\
            0.015713484026367727  3.392217172960918  \\
            0.015811693301532526  3.373658017477619  \\
            0.015909902576697322  3.361857659034739  \\
            0.01600811185186212  3.3549784036860153  \\
            0.016106321127026917  3.3519987588020044  \\
            0.016204530402191716  3.3523938238377897  \\
            0.016302739677356515  3.3559904354781196  \\
            0.016400948952521314  3.362919348930137  \\
            0.016499158227686113  3.37364058297828  \\
            0.01659736750285091  3.3890520977035603  \\
            0.016695576778015708  3.4107331687955087  \\
            0.016793786053180507  3.4414573878100603  \\
            0.016891995328345306  3.486327998371776  \\
            0.016990204603510105  3.555558405050941  \\
            0.0170884138786749  3.6723916030668127  \\
            0.0171866231538397  3.9013394366310785  \\
            0.0172848324290045  4.4940685588371805  \\
            0.017383041704169298  7.605927157905653  \\
            0.017481250979334097  9.258404325077874  \\
            0.017579460254498892  3.4730335150521237  \\
            0.01767766952966369  3.0969079466595226  \\
            0.01777587880482849  3.0517644475133783  \\
            0.01787408807999329  3.0563590580099818  \\
            0.01797229735515809  3.0699652651223666  \\
            0.018070506630322888  3.084253350086279  \\
            0.018168715905487683  3.0974028081559193  \\
            0.018266925180652482  3.109198046309597  \\
            0.01836513445581728  3.119861545122821  \\
            0.018463343730982077  3.129757769591147  \\
            0.018561553006146876  3.1393560352450054  \\
            0.018659762281311672  3.1493293814024863  \\
            0.01875797155647647  3.1608665585876485  \\
            0.01885618083164127  3.176661011272613  \\
            0.01895439010680607  3.221685562977504  \\
            0.019052599381970868  3.3125581391666508  \\
            0.019150808657135664  17.661082748828846  \\
            0.019249017932300463  3.0587800431041905  \\
            0.019347227207465262  3.0930288948874813  \\
            0.01944543648263006  3.1070014452146584  \\
            0.01954364575779486  3.1114452262118824  \\
            0.01964185503295966  3.1094909667613027  \\
            0.019740064308124455  3.101742317728295  \\
            0.019838273583289254  3.096290404121744  \\
            0.019936482858454053  3.5116744175059424  \\
            0.020034692133618852  4.310704796947368  \\
            0.02013290140878365  3.44826211068305  \\
            0.020231110683948447  3.3302226145525298  \\
            0.020329319959113246  3.288228102297607  \\
            0.020427529234278045  3.26788756646681  \\
            0.020525738509442844  3.2564555755998823  \\
            0.020623947784607643  3.249482798640844  \\
            0.02072215705977244  3.2450343322062523  \\
            0.020820366334937238  3.242141737283442  \\
            0.020918575610102033  3.240270679353734  \\
            0.021016784885266832  3.239104745427302  \\
            0.02111499416043163  3.238446548795991  \\
            0.02121320343559643  3.2396401073763883  \\
            0.021311412710761226  3.2402134244210066  \\
            0.021409621985926025  3.2410302126269745  \\
            0.021507831261090824  3.2420445715967636  \\
            0.021606040536255623  3.243230589285841  \\
            0.021704249811420422  3.2445744352653416  \\
            0.021802459086585218  3.2460737107586817  \\
            0.021900668361750017  3.2477387876331014  \\
            0.021998877636914816  3.2495972000969906  \\
            0.022097086912079615  3.2517039645443  \\
            0.022195296187244414  3.2541649349752015  \\
            0.022293505462409213  3.257194929953257  \\
            0.02239171473757401  3.261286252368333  \\
            0.022489924012738808  3.267835894187426  \\
            0.022588133287903607  3.2827750354137715  \\
            0.022686342563068406  3.383900431202305  \\
            0.022784551838233205  3.27138158466976  \\
            0.022882761113398  3.2466835211254135  \\
            0.0229809703885628  3.2518383278297467  \\
            0.0230791796637276  3.2561504339165  \\
            0.023177388938892398  3.259878287065088  \\
            0.023275598214057194  3.2633843053491325  \\
            0.023373807489221993  3.266901686909212  \\
            0.02347201676438679  3.270648777101659  \\
            0.023570226039551587  3.2761692671399647  \\
            0.023668435314716386  3.280485819840805  \\
            0.023766644589881186  3.2853616437708983  \\
            0.023864853865045985  3.291125126520493  \\
            0.02396306314021078  3.2983286489463417  \\
            0.02406127241537558  3.308007083770069  \\
            0.02415948169054038  3.3223712515311337  \\
            0.024257690965705177  3.347107235128839  \\
            0.024355900240869977  3.4020174955829496  \\
            0.024454109516034772  3.6166018636635484  \\
            0.024552318791199575  195.18310718635433  \\
            0.02465052806636437  3.2870360854489347  \\
            0.024748737341529166  3.2370260331742053  \\
            0.02484694661669397  3.2418877323400537  \\
            0.024945155891858764  3.2490596595839847  \\
            0.025043365167023567  3.2553243186211893  \\
            0.025141574442188362  3.2606541566629414  \\
            0.025239783717353158  3.2653010609615807  \\
            0.02533799299251796  3.2694846697904816  \\
            0.025436202267682756  3.2733713985729924  \\
            0.025534411542847555  3.2770889651456105  \\
            0.025632620818012354  3.280741179814008  \\
            0.02573083009317715  3.2844192568622907  \\
            0.02582903936834195  3.2882105774442834  \\
            0.025927248643506748  3.292206300566001  \\
            0.026025457918671547  3.2965090955124214  \\
            0.026123667193836343  3.301242341131835  \\
            0.026221876469001145  3.3065625281131696  \\
            0.02632008574416594  3.3126775426213824  \\
            0.026418295019330736  3.3198754929471685  \\
            0.02651650429449554  3.3285728199110074  \\
            0.026614713569660334  3.3393993309067493  \\
            0.026712922844825137  3.3533583647071405  \\
            0.026811132119989933  3.372152419887142  \\
            0.026909341395154728  3.3989118203805444  \\
            0.02700755067031953  3.440044370731569  \\
            0.027105759945484326  3.510828932999649  \\
            0.02720396922064913  3.657435298846132  \\
            0.027302178495813925  4.09424995667618  \\
            0.02740038777097872  8.121077552181246  \\
            0.027498597046143523  4.352963897224766  \\
            0.027596806321308318  3.211272528257487  \\
            0.02769501559647312  3.145159578624853  \\
            0.027793224871637916  3.148337860972893  \\
            0.027891434146802712  3.159239380545878  \\
            0.027989643421967515  3.1696827520525837  \\
            0.02808785269713231  3.178503539316263  \\
            0.02818606197229711  3.185770966334531  \\
            0.028284271247461905  3.191752483877366  \\
            0.028382480522626704  3.196702965911447  \\
            0.028480689797791503  3.200828299041833  \\
            0.0285788990729563  3.204287815418535  \\
            0.0286771083481211  3.2072037497254797  \\
            0.028775317623285897  3.2096701963810346  \\
            0.028873526898450692  3.211760153956325  \\
            0.028971736173615495  3.213530763659186  \\
            0.02906994544878029  3.215027131158816  \\
            0.029168154723945093  3.216285091056539  \\
            0.02926636399910989  3.2173332046851804  \\
            0.02936457327427469  3.2181941650338346  \\
            0.029462782549439487  3.2188857514090143  \\
            0.029560991824604282  3.2194213660674973  \\
            0.029659201099769085  3.2198101592765322  \\
            0.02975741037493388  3.2200566121970566  \\
            0.029855619650098683  3.220159283103057  \\
            0.02995382892526348  3.220108031160726  \\
            0.030052038200428274  3.2198780897842325  \\
            0.030150247475593077  3.2194167980093327  \\
            0.030248456750757872  3.2186108037506354  \\
            0.03034666602592267  3.2171907769513632  \\
            0.03044487530108747  3.214374265403403  \\
            0.030543084576252266  3.2067168916177256  \\
            0.030641293851417065  3.157959898347176  \\
            0.030739503126581864  3.2482588194369355  \\
            0.030837712401746663  3.2350644764162406  \\
            0.03093592167691146  3.2314599636654933  \\
            0.031034130952076258  3.221972997979749  \\
            0.031132340227241057  3.2257482407012126  \\
            0.031230549502405853  3.2303731042658357  \\
            0.03132875877757066  3.2358571138954164  \\
            0.031426968052735454  3.242362957170309  \\
            0.03152517732790025  3.2502151441565186  \\
            0.03162338660306505  3.259983513865889  \\
            0.03172159587822985  3.272686230951223  \\
            0.031819805153394644  3.2902778123330623  \\
            0.031918014428559446  3.316976485204704  \\
            0.03201622370372424  3.3637053511341803  \\
            0.03211443297888904  3.4690307830173466  \\
            0.03221264225405383  3.8998728220543226  \\
            0.032310851529218636  220.0299525616486  \\
            0.03240906080438343  3.1256722627903804  \\
            0.032507270079548234  3.115834368559513  \\
            0.03260547935471303  3.1386840842080073  \\
            0.032703688629877825  3.15644138742575  \\
            0.03280189790504263  3.169685498385239  \\
            0.03290010718020742  3.179938508099962  \\
            0.032998316455372226  3.1881983097586124  \\
            0.03309652573053702  3.195080765755045  \\
            0.03319473500570182  3.2009750102001635  \\
            0.03329294428086662  3.2061358737529515  \\
            0.033391153556031415  3.2107361762440396  \\
            0.03348936283119622  3.2148968676340193  \\
            0.03358757210636101  3.2187049831776013  \\
            0.03368578138152581  3.2222247119437544  \\
            0.03378399065669061  3.225504442369399  \\
            0.03388219993185541  3.2285813795693645  \\
            0.03398040920702021  3.231484650282883  \\
            0.034078618482185005  3.234237440263199  \\
            0.0341768277573498  3.2368584964700147  \\
            0.0342750370325146  3.2393632014817286  \\
            0.0343732463076794  3.2417643546466084  \\
            0.0344714555828442  3.244072748781052  \\
            0.034569664858009  3.2462976011663054  \\
            0.03466787413317379  3.2484468770470065  \\
            0.034766083408338595  3.2505275411514054  \\
            0.03486429268350339  3.2525457426578543  \\
            0.03496250195866819  3.2545069698580362  \\
            0.03506071123383299  3.256416157907413  \\
            0.035158920508997785  3.258277782260524  \\
            0.03525712978416259  3.260095926793423  \\
            0.03535533905932738  3.2618743371087553  \\
            0.035453548334492185  3.2636164627945403  \\
            0.03555175760965698  3.2653254885818828  \\
            0.035649966884821783  3.2670043577126773  \\
            0.03574817615998658  3.268655786526727  \\
            0.035846385435151375  3.270282276710208  \\
            0.03594459471031618  3.271886117725429  \\
            0.03604280398548097  3.2734693889341853  \\
            0.036141013260645775  3.2750339543541687  \\
            0.03623922253581057  3.2765814534334274  \\
            0.03633743181097537  3.2781132900672296  \\
            0.03643564108614017  3.279630615667323  \\
            0.036533850361304965  3.281134308086767  \\
            0.03663205963646976  3.2826249471960027  \\
            0.03673026891163456  3.284102787087592  \\
            0.03682847818679936  3.2855677212401853  \\
            0.036926687461964154  3.2870192435844996  \\
            0.03702489673712895  3.2884564069172604  \\
            0.03712310601229375  3.289877769375983  \\
            0.03722131528745855  3.291281341020339  \\
            0.037319524562623343  3.2926645146796854  \\
            0.037417733837788146  3.294023991455598  \\
            0.03751594311295294  3.2953556901035723  \\
            0.037614152388117744  3.2966546358353046  \\
            0.03771236166328254  3.2979148254292947  \\
            0.037810570938447335  3.2991290513645044  \\
            0.03790878021361214  3.3002886700638214  \\
            0.038006989488776934  3.3013832809498522  \\
            0.038105198763941736  3.302400277114105  \\
            0.03820340803910653  3.303324186225528  \\
            0.03830161731427133  3.3041356960293284  \\
            0.03839982658943613  3.3048101922133846  \\
            0.038498035864600925  3.305315601951231  \\
            0.03859624513976573  3.305609515608303  \\
            0.038694454414930524  3.3056370355919675  \\
            0.038792663690095326  3.3053399303911486  \\
            0.03889087296526012  3.3047497287631575  \\
            0.03898908224042492  3.304813565179441  \\
            0.03908729151558972  3.320793760857489  \\
            0.039185500790754516  19.96976619786416  \\
            0.03928371006591932  3.3987456550024984  \\
            0.039381919341084114  3.3468286304975288  \\
            0.03948012861624891  3.333526945049303  \\
            0.03957833789141371  3.3273349788484423  \\
            0.03967654716657851  3.3233964821520354  \\
            0.03977475644174331  3.3202807588000156  \\
            0.039872965716908106  3.3173991297583454  \\
            0.0399711749920729  3.3144651832122376  \\
            0.040069384267237704  3.3114111117803913  \\
            0.0401675935424025  3.3117977804604917  \\
            0.0402658028175673  3.3112846534412554  \\
            0.0403640120927321  3.3108628329701117  \\
            0.04046222136789689  3.310508569221747  \\
            0.040560430643061696  3.310205232443549  \\
            0.04065863991822649  3.309940841790004  \\
            0.040756849193391294  3.3097065569942408  \\
            0.04085505846855609  3.3094957191129275  \\
            0.040953267743720885  3.3093032229156196  \\
            0.04105147701888569  3.3091250932274767  \\
            0.04114968629405048  3.308958190648411  \\
            0.041247895569215286  3.3088000104890183  \\
            0.04134610484438008  3.308648528347639  \\
            0.04144431411954488  3.3085020959450895  \\
            0.04154252339470968  3.308359357598831  \\
            0.041640732669874475  3.308219192641596  \\
            0.04173894194503927  3.3080806663614006  \\
            0.041837151220204066  3.3079429949192813  \\
            0.04193536049536887  3.3078055175544328  \\
            0.042033569770533664  3.3076676752731777  \\
            0.04213177904569846  3.307528993275626  \\
            0.04222998832086326  3.307389065408903  \\
            0.04232819759602806  3.307247544532957  \\
            0.04242640687119286  3.3071041326782966  \\
            0.042524616146357656  3.3069585729429134  \\
            0.04262282542152245  3.306810644670135  \\
            0.042721034696687255  3.3066601575463404  \\
            0.04281924397185205  3.3065069459920156  \\
            0.04291745324701685  3.3063508679356555  \\
            0.04301566252218165  3.3061918000147745  \\
            0.043113871797346444  3.306029635968816  \\
            0.043212081072511246  3.305864284524005  \\
            0.04331029034767604  3.305695665385945  \\
            0.043408499622840845  3.30552371117298  \\
            0.04350670889800564  3.305348362864453  \\
            0.043604918173170436  3.3051695714339666  \\
            0.04370312744833524  3.3049872927889266  \\
            0.043801336723500034  3.3048014922761886  \\
            0.043899545998664836  3.3046121388170646  \\
            0.04399775527382963  3.3044192076915437  \\
            0.04409596454899443  3.304222677922728  \\
            0.04419417382415923  3.304022533033776  \\
            0.044292383099324026  3.3038187593440664  \\
            0.04439059237448883  3.3036113463919645  \\
            0.044488801649653624  3.3034002873122468  \\
            0.04458701092481843  3.303185575837246  \\
            0.04468522019998322  3.3029672088652395  \\
            0.04478342947514802  3.3027451850517626  \\
            0.04488163875031282  3.302519504414788  \\
            0.044979848025477616  3.3022901672271816  \\
            0.04507805730064242  3.3020571764084434  \\
            0.045176266575807214  3.3018205353608594  \\
            0.04527447585097201  3.3015802469414233  \\
            0.04537268512613681  3.3013363167789587  \\
            0.04547089440130161  3.3010887488266953  \\
            0.04556910367646641  3.3008375489392794  \\
            0.045667312951631206  3.300582721766387  \\
            0.045765522226796  3.3003242723600232  \\
            0.045863731501960804  3.300062206675358  \\
            0.0459619407771256  3.2997965292778133  \\
            0.0460601500522904  3.2995272449667614  \\
            0.0461583593274552  3.2992543583107876  \\
            0.046256568602619993  3.2989778735636426  \\
            0.046354777877784796  3.298697792966461  \\
            0.04645298715294959  3.2984141201717434  \\
            0.04655119642811439  3.2981268565419777  \\
            0.04664940570327918  3.2978360031656373  \\
            0.046747614978443985  3.2975415606047145  \\
            0.04684582425360878  3.2972435279541426  \\
            0.04694403352877358  3.296941903385088  \\
            0.04704224280393838  3.296636682936271  \\
            0.047140452079103175  3.296327862881116  \\
            0.04723866135426797  3.2960154372525885  \\
            0.04733687062943277  3.2956993994399464  \\
            0.04743507990459757  3.295379739680622  \\
            0.04753328917976237  3.2950564478381055  \\
            0.04763149845492717  3.2947295118757034  \\
            0.04772970773009197  3.2943989177945885  \\
            0.047827917005256765  3.294064648777804  \\
            0.04792612628042156  3.2937266878291505  \\
            0.04802433555558636  3.293385014226466  \\
            0.04812254483075116  3.2930396044090138  \\
            0.04822075410591596  3.292690434202608  \\
            0.04831896338108076  3.292337475294961  \\
            0.04841717265624555  3.29198069773347  \\
            0.048515381931410355  3.2916200670101476  \\
            0.04861359120657515  3.2912555476385643  \\
            0.04871180048173995  3.290887099052784  \\
            0.04881000975690475  3.29051467817523  \\
            0.048908219032069544  3.2901382382456803  \\
            0.04900642830723435  3.2897577286975173  \\
            0.04910463758239915  3.289373094460237  \\
            0.049202846857563945  3.2889842761472106  \\
            0.04930105613272874  3.288591210119141  \\
            0.049399265407893536  3.2881938283617935  \\
            0.04949747468305833  3.2877920564829495  \\
            0.04959568395822314  3.287385815782084  \\
            0.04969389323338794  3.2869750213880433  \\
            0.04979210250855273  3.2865595816749544  \\
            0.04989031178371753  3.286139399228608  \\
            0.049988521058882324  3.2857143699432187  \\
            0.05008673033404713  3.2852843811520414  \\
            0.05018493960921193  3.2848493140391284  \\
            0.050283148884376724  3.2844090386928877  \\
            0.05038135815954152  3.2839634192607305  \\
            0.050479567434706316  3.2835123088004243  \\
            0.050577776709871125  3.2830555491449247  \\
            0.05067598598503592  3.2825929738127897  \\
            0.050774195260200716  3.2821244013876254  \\
            0.05087240453536551  3.2816496402295128  \\
            0.05097061381053031  3.2811684848270146  \\
            0.05106882308569511  3.2806807133261966  \\
            0.05116703236085991  3.280186090014078  \\
            0.05126524163602471  3.2796843627538026  \\
            0.051363450911189504  3.279175259949038  \\
            0.0514616601863543  3.2786584893913107  \\
            0.0515598694615191  3.2781337409378897  \\
            0.0516580787366839  3.277600678391786  \\
            0.05175628801184869  3.277058942754632  \\
            0.051854497287013496  3.2765081471311457  \\
            0.05195270656217829  3.275947875096415  \\
            0.052050915837343094  3.275377679393156  \\
            0.05214912511250789  3.2747970778483437  \\
            0.052247334387672685  3.2742055497762044  \\
            0.05234554366283748  3.273602534728081  \\
            0.05244375293800229  3.272987425537052  \\
            0.052541962213167086  3.272359567426307  \\
            0.05264017148833188  3.2717182505646853  \\
            0.05273838076349668  3.2710627066681517  \\
            0.05283659003866147  3.270392100952532  \\
            0.05293479931382628  3.26970552837898  \\
            0.05303300858899108  3.2690020037983043  \\
            0.05313121786415587  3.26828045641192  \\
            0.05322942713932067  3.2675397177632908  \\
            0.053327636414485464  3.266778512677167  \\
            0.053425845689650274  3.2659954500739325  \\
            0.05352405496481507  3.2651890064947566  \\
            0.053622264239979865  3.264357513908787  \\
            0.05372047351514466  3.2634991449838817  \\
            0.053818682790309456  3.2626118937487663  \\
            0.053916892065474266  3.2616935572600796  \\
            0.05401510134063906  3.2607417152171627  \\
            0.05411331061580386  3.259753707470965  \\
            0.05421151989096865  3.2587266098566916  \\
            0.05430972916613345  3.257657211442247  \\
            0.05440793844129826  3.256541991485226  \\
            0.05450614771646305  3.255377099669394  \\
            0.05460435699162785  3.254158345828012  \\
            0.054702566266792645  3.2528812004626517  \\
            0.05480077554195744  3.25154082426621  \\
            0.05489898481712225  3.250132135625593  \\
            0.054997194092287045  3.2486499545570955  \\
            0.05509540336745184  3.2470892649945813  \\
            0.055193612642616637  3.24544567892176  \\
            0.05529182191778143  3.243716248024059  \\
            0.05539003119294624  3.241900861394281  \\
            0.05548824046811104  3.240004677253375  \\
            0.05558644974327583  3.2380423892457433  \\
            0.05568465901844063  3.2360458706649826  \\
            0.055782868293605424  3.234078208289796  \\
            0.05588107756877023  3.2322603557256118  \\
            0.05597928684393503  3.23082389117398  \\
            0.056077496119099825  3.23022106371738  \\
            0.05617570539426462  3.2313698803147863  \\
            0.056273914669429416  3.2362475057708973  \\
            0.05637212394459422  3.249492055360602  \\
            0.056470333219759014  3.2834082112932355  \\
            0.05656854249492381  3.377210297693933  \\
            0.05666675177008861  3.698613478594086  \\
            0.05676496104525341  5.505342196286939  \\
            0.05686317032041821  38.95989834583419  \\
            0.056961379595583006  5.5440807666166565  \\
            0.0570595888707478  4.164915422850994  \\
            0.0571577981459126  3.7761485308951075  \\
            0.0572560074210774  3.608579900295238  \\
            0.0573542166962422  3.5190823114441976  \\
            0.057452425971407  3.464619943681292  \\
            0.057550635246571794  3.4284390338549953  \\
            0.05764884452173659  3.4028412597655606  \\
            0.057747053796901385  3.383853039885003  \\
            0.057845263072066194  3.369238513024125  \\
            0.05794347234723099  3.357652902714554  \\
            0.058041681622395785  3.3482433338788793  \\
            0.05813989089756058  3.3404450222032485  \\
            0.05823810017272538  3.33387030187093  \\
            0.058336309447890186  3.328244907936234  \\
            0.05843451872305498  3.32336972265853  \\
            0.05853272799821978  3.319096920420626  \\
            0.05863093727338457  3.3153145912766666  \\
            0.05872914654854938  3.31193655175331  \\
            0.05882735582371418  3.3088954073341394  \\
            0.058925565098878974  3.3061377394746456  \\
            0.05902377437404377  3.303620683619933  \\
            0.059121983649208565  3.3013094632780624  \\
            0.059220192924373374  3.2991755794658952  \\
            0.05931840219953817  3.297195462484915  \\
            0.059416611474702966  3.2953494557722247  \\
            0.05951482074986776  3.293621038389709  \\
            0.05961303002503256  3.2919962263029268  \\
            0.059711239300197366  3.290463103413023  \\
            0.05980944857536216  3.2890114547916642  \\
            0.05990765785052696  3.2876324698916344  \\
            0.06000586712569175  3.2863185135720685  \\
            0.06010407640085655  3.285062930428453  \\
            0.06020228567602136  3.283859895592632  \\
            0.060300494951186154  3.2827042847785215  \\
            0.06039870422635095  3.2815915734741314  \\
            0.060496913501515745  3.2805177481842347  \\
            0.06059512277668054  3.2794792367085512  \\
            0.06069333205184534  3.2784728453812604  \\
            0.060791541327010146  3.277495711635925  \\
            0.06088975060217494  3.276545258265459  \\
            0.06098795987733974  3.2756191593005854  \\
            0.06108616915250453  3.274715306593206  \\
            0.061184378427669335  3.2738317856052683  \\
            0.06128258770283413  3.2729668498613727  \\
            0.061380796977998926  3.2486936235077746  \\
            0.06147900625316373  3.2456392730005073  \\
            0.061577215528328524  3.2437202994372445  \\
            0.06167542480349333  3.242163966229522  \\
            0.06177363407865812  3.2408754709275946  \\
            0.06187184335382292  3.2397608562990525  \\
            0.061970052628987714  3.238745735173782  \\
            0.062068261904152516  3.2377539895622203  \\
            0.06216647117931732  3.2367094415185704  \\
            0.062264680454482114  3.2355256888784094  \\
            0.06236288972964691  3.2340925133636156  \\
            0.062461099004811706  3.23226144806074  \\
            0.06255930827997651  3.2298133623270835  \\
            0.06265751755514132  3.2263987659822995  \\
            0.06275572683030611  3.221412355879347  \\
            0.06285393610547091  3.213700719106919  \\
            0.0629521453806357  3.20081432662105  \\
            0.0630503546558005  3.1775152036555125  \\
            0.0631485639309653  3.2001054251339487  \\
            0.0632467732061301  3.9870257939565152  \\
            0.0633449824812949  3.4194281870543985  \\
            0.0634431917564597  3.3478559810282356  \\
            0.06354140103162449  3.3217226715271315  \\
            0.06363961030678929  3.3086561912473322  \\
            0.06373781958195408  3.3010740261400002  \\
            0.06383602885711889  3.2962971511475856  \\
            0.06393423813228369  3.2931423676338114  \\
            0.06403244740744848  3.291007031895965  \\
            0.06413065668261328  3.289551963339041  \\
            0.06422886595777808  3.288571560151145  \\
            0.06432707523294287  3.2879338489082817  \\
            0.06442528450810767  3.287550142622103  \\
            0.06452349378327248  3.287358560802875  \\
            0.06462170305843727  3.287314551031316  \\
            0.06471991233360207  3.2873851903863396  \\
            0.06481812160876686  3.2875455744764968  \\
            0.06491633088393166  3.2877764996204952  \\
            0.06501454015909647  3.2880628945608636  \\
            0.06511274943426126  3.288392752188418  \\
            0.06521095870942606  3.288756375741652  \\
            0.06530916798459085  3.289145840623247  \\
            0.06540737725975565  3.289554598465837  \\
            0.06550558653492046  3.2899771892508363  \\
            0.06560379581008526  3.2904090163321627  \\
            0.06570200508525005  3.290846184280118  \\
            0.06580021436041485  3.291285364826538  \\
            0.06589842363557964  3.291723700143011  \\
            0.06599663291074445  3.292158719646489  \\
            0.06609484218590925  3.292588278694475  \\
            0.06619305146107404  3.293010505586454  \\
            0.06629126073623884  3.2934237635514885  \\
            0.06638947001140363  3.2938266144244137  \\
            0.06648767928656844  3.294217790059287  \\
            0.06658588856173324  3.2945961733448574  \\
            0.06668409783689803  3.294960775701745  \\
            0.06678230711206283  3.2953107226009526  \\
            0.06688051638722763  3.2956452402672634  \\
            0.06697872566239244  3.2959636432162562  \\
            0.06707693493755723  3.2962653264595994  \\
            0.06717514421272203  3.2965497564860406  \\
            0.06727335348788682  3.29681646428207  \\
            0.06737156276305162  3.2970650402032233  \\
            0.06746977203821643  3.297295127151514  \\
            0.06756798131338122  3.297506418856655  \\
            0.06766619058854602  3.2976986545798037  \\
            0.06776439986371081  3.297871616508253  \\
            0.06786260913887561  3.2980251265226883  \\
            0.06796081841404042  3.2981590459709325  \\
            0.06805902768920521  3.2982732710431204  \\
            0.06815723696437001  3.2983677332250396  \\
            0.0682554462395348  3.2984423995915066  \\
            0.0683536555146996  3.298497268109445  \\
            0.06845186478986441  3.298532370419709  \\
            0.0685500740650292  3.2985477699366066  \\
            0.068648283340194  3.2985435634564233  \\
            0.0687464926153588  3.29851988011636  \\
            0.0688447018905236  3.2984768822561668  \\
            0.0689429111656884  3.298414765726038  \\
            0.0690411204408532  3.2983337637766676  \\
            0.069139329716018  3.2982341455354787  \\
            0.06923753899118279  3.298116218315509  \\
            0.06933574826634759  3.2979803317115666  \\
            0.0694339575415124  3.297826878955435  \\
            0.06953216681667719  3.297656300841301  \\
            0.06963037609184199  3.2974690874308306  \\
            0.06972858536700678  3.297265786686036  \\
            0.06982679464217158  3.297047005735628  \\
            0.06992500391733639  3.2968134196906442  \\
            0.07002321319250118  3.296565776451214  \\
            0.07012142246766598  3.296304907649002  \\
            0.07021963174283077  3.2960317382061906  \\
            0.07031784101799557  3.295747296543538  \\
            0.07041605029316038  3.295452731782128  \\
            0.07051425956832517  3.2951493260570652  \\
            0.07061246884348997  3.294838521552246  \\
            0.07071067811865477  3.2945219373243075  \\
            0.07080888739381958  3.294201403265639  \\
            0.07090709666898437  3.293878997765103  \\
            0.07100530594414917  3.293557085331792  \\
            0.07110351521931396  3.293238376951006  \\
            0.07120172449447876  3.29292599794  \\
            0.07129993376964357  3.292623575464532  \\
            0.07139814304480836  3.2923353272637206  \\
            0.07149635231997316  3.292066238842161  \\
            0.07159456159513795  3.2918221938616328  \\
            0.07169277087030275  3.29161023882626  \\
            0.07179098014546756  3.291438903190351  \\
            0.07188918942063235  3.291318556061063  \\
            0.07198739869579715  3.291262071187849  \\
            0.07208560797096195  3.2912853585731567  \\
            0.07218381724612674  3.2914089826128405  \\
            0.07228202652129155  3.2916594218832427  \\
            0.07238023579645635  3.292069323771017  \\
            0.07247844507162114  3.2926860831174305  \\
            0.07257665434678594  3.293582244643305  \\
            0.07267486362195073  3.2964754046523312  \\
            0.07277307289711553  3.266640825322108  \\
            0.07287128217228034  3.2674011486521315  \\
            0.07296949144744513  3.2681543548036376  \\
            0.07306770072260993  3.268900475915722  \\
            0.07316590999777473  3.2696395443663175  \\
            0.07326411927293952  3.270371589450324  \\
            0.07336232854810432  3.2710966397534307  \\
            0.07346053782326913  3.271814721283439  \\
            0.07355874709843392  3.272525858511756  \\
            0.07365695637359872  3.2732300750872176  \\
            0.07375516564876351  3.2739273906115205  \\
            0.07385337492392831  3.2746178241358828  \\
            0.0739515841990931  3.275301392058671  \\
            0.0740497934742579  3.275978108408675  \\
            0.07414800274942271  3.2766479847387857  \\
            0.0742462120245875  3.2773110309365956  \\
            0.0743444212997523  3.2779672528796553  \\
            0.0744426305749171  3.278616654923991  \\
            0.07454083985008189  3.279259237741068  \\
            0.07463904912524669  3.2798949995059123  \\
            0.0747372584004115  3.2805239348829303  \\
            0.07483546767557629  3.281146035259084  \\
            0.07493367695074109  3.2817612882029215  \\
            0.07503188622590588  3.2823696775796947  \\
            0.07513009550107068  3.282971183514309  \\
            0.07522830477623549  3.2835657819672517  \\
            0.07532651405140028  3.2841534430789556  \\
            0.07542472332656508  3.284734133299287  \\
            0.07552293260172988  3.2853078138378353  \\
            0.07562114187689467  3.2858744396127038  \\
            0.07571935115205948  3.2864339598944468  \\
            0.07581756042722428  3.2869863171923765  \\
            0.07591576970238907  3.2875314475947737  \\
            0.07601397897755387  3.2880692791051533  \\
            0.07611218825271866  3.2885997311054505  \\
            0.07621039752788347  3.2891227151391305  \\
            0.07630860680304827  3.289638132009622  \\
            0.07640681607821306  3.290145872172483  \\
            0.07650502535337786  3.290645814688721  \\
            0.07660323462854265  3.2911378254642836  \\
            0.07670144390370746  3.291621757035829  \\
            0.07679965317887226  3.2920974458241097  \\
            0.07689786245403706  3.292564711364255  \\
            0.07699607172920185  3.2930233543438963  \\
            0.07709428100436666  3.2934731540706492  \\
            0.07719249027953146  3.293913866369209  \\
            0.07729069955469625  3.2943452214844973  \\
            0.07738890882986105  3.2947669197379175  \\
            0.07748711810502584  3.295178627854077  \\
            0.07758532738019065  3.2955799763102878  \\
            0.07768353665535545  3.295970551900234  \\
            0.07778174593052024  3.2963498950259797  \\
            0.07787995520568504  3.296717489331653  \\
            0.07797816448084983  3.2970727573853185  \\
            0.07807637375601464  3.29741504839344  \\
            0.07817458303117944  3.2977436285211166  \\
            0.07827279230634424  3.2980576677748887  \\
            0.07837100158150903  3.298356222969464  \\
            0.07846921085667383  3.298638218453696  \\
            0.07856742013183864  3.2989024248636696  \\
            0.07866562940700343  3.299147427686618  \\
            0.07876383868216823  3.2993715933283774  \\
            0.07886204795733302  3.2995730265733347  \\
            0.07896025723249782  3.2997495138533606  \\
            0.07905846650766263  3.2998984566813725  \\
            0.07915667578282742  3.300016783146064  \\
            0.07925488505799222  3.3001008382248505  \\
            0.07935309433315701  3.3001462340902803  \\
            0.07945130360832181  3.3001476624181265  \\
            0.07954951288348662  3.300098639958344  \\
            0.07964772215865142  3.2999911637159878  \\
            0.07974593143381621  3.2998152402688783  \\
            0.079844140708981  3.2995582297599104  \\
            0.0799423499841458  3.299203909630406  \\
            0.08004055925931061  3.2987311149044647  \\
            0.08013876853447541  3.2981117115379304  \\
            0.0802369778096402  3.2973074892585283  \\
            0.080335187084805  3.296265245954528  \\
            0.0804333963599698  3.2949087243512407  \\
            0.0805316056351346  3.2931248045245853  \\
            0.0806298149102994  3.2907387280906724  \\
            0.0807280241854642  3.287467196380428  \\
            0.08082623346062899  3.282825058365637  \\
            0.08092444273579379  3.275938636196535  \\
            0.0810226520109586  3.265305026580918  \\
            0.08112086128612339  3.252125258148686  \\
            0.08121907056128819  3.6861737518526714  \\
            0.08131727983645298  3.838347164740823  \\
            0.08141548911161778  3.445521733224243  \\
            0.08151369838678259  3.3882755664606075  \\
            0.08161190766194738  3.3660090948983066  \\
            0.08171011693711218  3.35437402838792  \\
            0.08180832621227697  3.347313056337073  \\
            0.08190653548744177  3.3426235496152525  \\
            0.08200474476260658  3.339317117252149  \\
            0.08210295403777138  3.336885578882245  \\
            0.08220116331293617  3.335041353594258  \\
            0.08229937258810097  3.3336096424784922  \\
            0.08239758186326576  3.332478144299015  \\
            0.08249579113843057  3.3315714482402767  \\
            0.08259400041359537  3.3308370484700376  \\
            0.08269220968876016  3.3302372689510826  \\
            0.08279041896392496  3.3297443749335542  \\
            0.08288862823908975  3.329337499803468  \\
            0.08298683751425455  3.3290006422631704  \\
            0.08308504678941936  3.32872132516139  \\
            0.08318325606458415  3.3284896768652934  \\
            0.08328146533974895  3.32829778301281  \\
            0.08337967461491375  3.3281392248889587  \\
            0.08347788389007854  3.3280087392473248  \\
            0.08357609316524334  3.327901969797455  \\
            0.08367430244040813  3.3278152793236417  \\
            0.08377251171557294  3.3277456054375185  \\
            0.08387072099073774  3.3276903508003204  \\
            0.08396893026590253  3.327647298106215  \\
            0.08406713954106733  3.306041597836663  \\
            0.08416534881623212  3.306064296083234  \\
            0.08426355809139692  3.3059553821574674  \\
            0.08436176736656173  3.305797290430002  \\
            0.08445997664172653  3.3055942462295453  \\
            0.08455818591689132  3.3053502640769046  \\
            0.08465639519205612  3.3050689858562663  \\
            0.08475460446722091  3.3047537212809304  \\
            0.08485281374238572  3.3044075023717654  \\
            0.08495102301755052  3.304033102593696  \\
            0.08504923229271531  3.3036330470421986  \\
            0.08514744156788011  3.3032096545804124  \\
            0.0852456508430449  3.3027650488118514  \\
            0.08534386011820971  3.3023011814794803  \\
            0.08544206939337451  3.3018198482948375  \\
            0.0855402786685393  3.301322699964932  \\
            0.0856384879437041  3.300811261512747  \\
            0.0857366972188689  3.300286939284231  \\
            0.0858349064940337  3.2997510323293793  \\
            0.0859331157691985  3.2992047422972233  \\
            0.0860313250443633  3.298649181392593  \\
            0.08612953431952809  3.2980853803766417  \\
            0.08622774359469289  3.2975142951370255  \\
            0.0863259528698577  3.2969368114535333  \\
            0.08642416214502249  3.296353753460608  \\
            0.08652237142018729  3.2957658851112615  \\
            0.08662058069535208  3.2951739177701787  \\
            0.08671878997051688  3.294578511803414  \\
            0.08681699924568169  3.2939802826519915  \\
            0.08691520852084648  3.2933798026184435  \\
            0.08701341779601128  3.2927776032138047  \\
            0.08711162707117608  3.2921741811605667  \\
            0.08720983634634087  3.291569997699978  \\
            0.08730804562150568  3.2909654822273358  \\
            0.08740625489667048  3.2903610350872583  \\
            0.08750446417183527  3.289757028257327  \\
            0.08760267344700007  3.289153807868035  \\
            0.08770088272216486  3.288551696097617  \\
            0.08779909199732967  3.2879509917336383  \\
            0.08789730127249447  3.287351973281434  \\
            0.08799551054765926  3.286754898460506  \\
            0.08809371982282406  3.28616000580304  \\
            0.08819192909798886  3.285567516520769  \\
            0.08829013837315366  3.2849776348869413  \\
            0.08838834764831846  3.2843905495560697  \\
            0.08848655692348326  3.2838064337702786  \\
            0.08858476619864805  3.283225446379036  \\
            0.08868297547381285  3.282647732974382  \\
            0.08878118474897766  3.2820734261093927  \\
            0.08887939402414245  3.2815026457022136  \\
            0.08897760329930725  3.280935499931637  \\
            0.08907581257447204  3.2803720855862872  \\
            0.08917402184963685  3.2798124892018063  \\
            0.08927223112480165  3.2792567854455004  \\
            0.08937044039996644  3.278705039657313  \\
            0.08946864967513124  3.278157306615501  \\
            0.08956685895029604  3.277613631821582  \\
            0.08966506822546084  3.2770740508328076  \\
            0.08976327750062564  3.276538590032416  \\
            0.08986148677579044  3.276007266434996  \\
            0.08995969605095523  3.275480088141256  \\
            0.09005790532612003  3.274957053076759  \\
            0.09015611460128484  3.274438151087743  \\
            0.09025432387644963  3.2739233619901484  \\
            0.09035253315161443  3.273412656515177  \\
            0.09045074242677922  3.272905995046402  \\
            0.09054895170194402  3.272403329026035  \\
            0.09064716097710883  3.2719045992557243  \\
            0.09074537025227362  3.2714097354080915  \\
            0.09084357952743842  3.2709186573998794  \\
            0.09094178880260322  3.2704312729475293  \\
            0.09103999807776801  3.2699474766996497  \\
            0.09113820735293282  3.2694671524966936  \\
            0.09123641662809762  3.2689901691150225  \\
            0.09133462590326241  3.2685163812196443  \\
            0.09143283517842721  3.2680456275757295  \\
            0.091531044453592  3.267577731393761  \\
            0.09162925372875681  3.2671124959742026  \\
            0.09172746300392161  3.2666497066385674  \\
            0.0918256722790864  3.266189126626257  \\
            0.0919238815542512  3.2657304949130004  \\
            0.092022090829416  3.2652735261247807  \\
            0.0921203001045808  3.264817907627635  \\
            0.0922185093797456  3.2643632938216793  \\
            0.0923167186549104  3.263909306042569  \\
            0.09241492793007519  3.2634555272780927  \\
            0.09251313720523999  3.263001499368123  \\
            0.09261134648040478  3.262546715827798  \\
            0.09270955575556959  3.262090619591242  \\
            0.09280776503073439  3.261632594107168  \\
            0.09290597430589918  3.2611719571692577  \\
            0.09300418358106398  3.2607079539079566  \\
            0.09310239285622877  3.260239746904829  \\
            0.09320060213139357  3.259766405400044  \\
            0.09329881140655837  3.259286895298893  \\
            0.09339702068172318  3.258800063695454  \\
            0.09349522995688797  3.2583046275786445  \\
            0.09359343923205277  3.25779915520852  \\
            0.09369164850721756  3.257282049019179  \\
            0.09378985778238236  3.2567515316047504  \\
            0.09388806705754715  3.256205622830204  \\
            0.09398627633271196  3.2556421286045403  \\
            0.09408448560787676  3.255058630935268  \\
            0.09418269488304155  3.2544524902561665  \\
            0.09428090415820635  3.2538208731152185  \\
            0.09437911343337115  3.253160820110749  \\
            0.09447732270853594  3.2524693913814584  \\
            0.09457553198370075  3.2517439499837613  \\
            0.09467374125886555  3.2509826946696583  \\
            0.09477195053403034  3.2501856471930095  \\
            0.09487015980919514  3.249356498799276  \\
            0.09496836908435993  3.248506109275251  \\
            0.09506657835952474  3.247659308283363  \\
            0.09516478763468954  3.2468686055835674  \\
            0.09526299690985433  3.2462431667828633  \\
            0.09536120618501913  3.2460139355179565  \\
            0.09545941546018394  3.2466922010240276  \\
            0.09555762473534873  3.249498802269394  \\
            0.09565583401051353  3.2577066494569684  \\
            0.09575404328567833  3.2818137071030127  \\
            0.09585225256084312  3.367246550556222  \\
            0.09595046183600793  3.870325525385179  \\
            0.09604867111117273  86.87166819420528  \\
            0.09614688038633752  4.393170925503983  \\
            0.09624508966150232  3.635874071631451  \\
            0.09634329893666711  3.462946492010219  \\
            0.09644150821183192  3.394439152847593  \\
            0.09653971748699672  3.3594817843039184  \\
            0.09663792676216151  3.3388696778880154  \\
            0.09673613603732631  3.3255500475851987  \\
            0.0968343453124911  3.31640476339253  \\
            0.09693255458765591  3.30987581116879  \\
            0.09703076386282071  3.305122765505068  \\
            0.0971289731379855  3.301684054830773  \\
            0.0972271824131503  3.299342349975067  \\
            0.0973253916883151  3.2981130856672385  \\
            0.0974236009634799  3.2984109880650703  \\
            0.0975218102386447  3.3018187470089635  \\
            0.0976200195138095  3.3153632176043972  \\
            0.09771822878897429  3.3994465343014095  \\
            0.09781643806413909  49.331209474279866  \\
            0.0979146473393039  3.4279501658473985  \\
            0.0980128566144687  3.337815564642224  \\
            0.09811106588963349  3.3257219901926747  \\
        }
        ;
    \addlegendentry { $1.0 \cdot 10^{1}$, $ 0.0 \cdot 10^{0} $, $ 0.0 \cdot 10^{0} $ }
    \addplot[color={rgb,1:red,1.0;green,0.0;blue,0.0}, name path={c02a20ba-16c8-4c86-9958-f83cacd60d2a}, draw opacity={1.0}, line width={1}, dashed, forget plot]
        table[row sep={\\}]
        {
            \\
            0.0  27.42554413146028  \\
            9.820927516479829e-5  27.42556265737348  \\
            0.00019641855032959658  27.425577156237683  \\
            0.00029462782549439484  27.42558764796911  \\
            0.00039283710065919316  27.425594148789504  \\
            0.0004910463758239915  27.425596671754338  \\
            0.0005892556509887897  27.425842477225515  \\
            0.000687464926153588  27.42637865352698  \\
            0.0007856742013183863  27.426916077702305  \\
            0.0008838834764831845  27.427437224061954  \\
            0.000982092751647983  27.427943505870374  \\
            0.0010803020268127812  27.42843587481621  \\
            0.0011785113019775794  27.428915147239383  \\
            0.0012767205771423778  27.429382113627323  \\
            0.001374929852307176  27.429837444433346  \\
            0.0014731391274719742  27.430281743587912  \\
            0.0015713484026367726  27.430715558052185  \\
            0.001669557677801571  27.431139378783264  \\
            0.001767766952966369  27.431553649634026  \\
            0.0018659762281311677  27.431958769332386  \\
            0.001964185503295966  27.43235509948816  \\
            0.002062394778460764  27.43274296613126  \\
            0.0021606040536255623  27.43312266493595  \\
            0.002258813328790361  27.43349446331384  \\
            0.0023570226039551587  27.43385860166152  \\
            0.0024552318791199574  27.43421529676686  \\
            0.0025534411542847556  27.434564743445982  \\
            0.002651650429449554  27.434907115301595  \\
            0.002749859704614352  27.43524256629  \\
            0.0028480689797791506  27.436081038841728  \\
            0.0029462782549439484  27.436528231339878  \\
            0.003044487530108747  27.4369570469095  \\
            0.0031426968052735453  27.437363721530065  \\
            0.0032409060804383435  27.43775972624743  \\
            0.003339115355603142  27.438150891208565  \\
            0.0034373246307679403  27.43853613845318  \\
            0.003535533905932738  27.438914298891046  \\
            0.0036337431810975363  27.43928523824204  \\
            0.0037319524562623354  27.439649103935793  \\
            0.003830161731427133  27.44000642220847  \\
            0.003928371006591932  27.44035660184418  \\
            0.004026580281756729  27.440699845186458  \\
            0.004124789556921528  27.441035810445648  \\
            0.004222998832086326  27.441364600119083  \\
            0.004321208107251125  27.44168596202703  \\
            0.004419417382415923  27.441999711892198  \\
            0.004517626657580722  27.44230567491518  \\
            0.004615835932745519  27.442603593720545  \\
            0.0047140452079103175  27.442893171349542  \\
            0.004812254483075116  27.44317408487187  \\
            0.004910463758239915  27.443445993419644  \\
            0.005008673033404713  27.443708464374737  \\
            0.005106882308569511  27.443961019011002  \\
            0.005205091583734309  27.444203092955533  \\
            0.005303300858899108  27.444434053453197  \\
            0.005401510134063906  27.444653190254662  \\
            0.005499719409228704  27.444859661843264  \\
            0.005597928684393502  27.445052515671378  \\
            0.005696137959558301  27.445230627068174  \\
            0.0057943472347230995  27.445392727275006  \\
            0.005892556509887897  27.445537313336796  \\
            0.005990765785052695  27.445662626582177  \\
            0.006088975060217494  27.44576660963198  \\
            0.0061871843353822915  27.4458468048357  \\
            0.0062853936105470905  27.445900291592377  \\
            0.006383602885711889  27.445923556005322  \\
            0.006481812160876687  27.44591233051098  \\
            0.006580021436041485  27.445861392849768  \\
            0.006678230711206284  27.445764293516042  \\
            0.006776439986371082  27.445612963418743  \\
            0.006874649261535881  27.44539720172153  \\
            0.006972858536700678  27.44510397682225  \\
            0.007071067811865476  27.444716415401867  \\
            0.007169277087030275  27.444212480553297  \\
            0.007267486362195073  27.44356325262771  \\
            0.007365695637359872  27.4427311220849  \\
            0.007463904912524671  27.441669554854414  \\
            0.007562114187689468  27.440331757421575  \\
            0.007660323462854266  27.438721811218432  \\
            0.0077585327380190645  27.437165850272812  \\
            0.007856742013183864  27.43806224937665  \\
            0.007954951288348661  27.46523817695699  \\
            0.008053160563513458  29.52930128123155  \\
            0.008151369838678257  27.813314981187762  \\
            0.008249579113843056  27.555091233845445  \\
            0.008347788389007854  27.509258707227733  \\
            0.008445997664172653  27.491726683726867  \\
            0.00854420693933745  27.482748384477663  \\
            0.00864241621450225  27.47737826191663  \\
            0.008740625489667048  27.4738414260472  \\
            0.008838834764831846  27.471354234592784  \\
            0.008937044039996645  27.469520278748725  \\
            0.009035253315161444  27.468118856042587  \\
            0.009133462590326241  27.467018170067472  \\
            0.009231671865491039  27.466135384817097  \\
            0.009329881140655836  27.46541659712497  \\
            0.009428090415820635  27.46482612966123  \\
            0.009526299690985434  27.46434056609121  \\
            0.009624508966150231  27.46394537808738  \\
            0.00972271824131503  27.46363310334415  \\
            0.00982092751647983  27.46340254837093  \\
            0.009919136791644627  27.46325879479374  \\
            0.010017346066809426  27.46321395741658  \\
            0.010115555341974223  27.46328880897051  \\
            0.010213764617139022  27.463515553636583  \\
            0.010311973892303821  27.463942289624374  \\
            0.010410183167468619  27.464640105578905  \\
            0.010508392442633416  27.465714497127458  \\
            0.010606601717798215  27.467324141152105  \\
            0.010704810992963013  27.469712747044884  \\
            0.010803020268127812  27.473265210949478  \\
            0.010901229543292609  27.478611373889724  \\
            0.010999438818457408  27.486828992179607  \\
            0.011097648093622207  27.49986975065333  \\
            0.011195857368787004  27.521536534066833  \\
            0.011294066643951804  27.560000669104546  \\
            0.011392275919116603  27.63539674156933  \\
            0.0114904851942814  27.808703925905373  \\
            0.011588694469446199  28.34262333288196  \\
            0.011686903744610996  31.739440637017314  \\
            0.011785113019775794  62.44833900996132  \\
            0.011883322294940593  35.270626678589856  \\
            0.01198153157010539  28.542470044261737  \\
            0.01207974084527019  27.895052394675186  \\
            0.012177950120434988  27.848534595087067  \\
            0.012276159395599787  40.61188053250721  \\
            0.012374368670764583  29.380164310990313  \\
            0.012472577945929382  28.101089156947495  \\
            0.012570787221094181  27.84452443492166  \\
            0.01266899649625898  27.737856456009023  \\
            0.012767205771423778  27.67928899457119  \\
            0.012865415046588575  27.642071386352654  \\
            0.012963624321753374  27.616170368174107  \\
            0.013061833596918171  27.596960959571312  \\
            0.01316004287208297  27.582000328192922  \\
            0.01325825214724777  27.569865082151985  \\
            0.013356461422412568  27.559659772978197  \\
            0.013454670697577364  27.55078333514964  \\
            0.013552879972742163  27.542807526722292  \\
            0.013651089247906962  27.535408767641943  \\
            0.013749298523071761  27.528327449530718  \\
            0.01384750779823656  27.521342496951473  \\
            0.013945717073401356  27.514255539148156  \\
            0.014043926348566155  27.506883225194482  \\
            0.014142135623730952  27.499060837782018  \\
            0.014240344898895752  27.49066952633831  \\
            0.01433855417406055  27.48172378875886  \\
            0.014436763449225346  27.472630560151735  \\
            0.014534972724390145  27.464993375858253  \\
            0.014633181999554944  27.464420394252603  \\
            0.014731391274719743  27.492482923016485  \\
            0.014829600549884542  27.65821195890458  \\
            0.014927809825049342  29.01767446803049  \\
            0.015026019100214137  1008.0198793674591  \\
            0.015124228375378936  31.122388327480884  \\
            0.015222437650543735  28.83035011446365  \\
            0.015320646925708533  28.296569340744522  \\
            0.015418856200873332  28.079326750367063  \\
            0.015517065476038129  27.966534505201267  \\
            0.015615274751202926  27.899703804894635  \\
            0.015713484026367727  27.856942321809335  \\
            0.015811693301532526  27.82840931693342  \\
            0.015909902576697322  27.80912332852246  \\
            0.01600811185186212  27.79636611582503  \\
            0.016106321127026917  27.78860455618675  \\
            0.016204530402191716  27.785001380924616  \\
            0.016302739677356515  27.78518861210788  \\
            0.016400948952521314  27.789179770448627  \\
            0.016499158227686113  27.797378933228998  \\
            0.01659736750285091  27.810692151892667  \\
            0.016695576778015708  27.83079917604738  \\
            0.016793786053180507  27.86074741435358  \\
            0.016891995328345306  27.906306208648864  \\
            0.016990204603510105  27.979399255304276  \\
            0.0170884138786749  28.108340801878764  \\
            0.0171866231538397  28.376933556504984  \\
            0.0172848324290045  29.152068512377852  \\
            0.017383041704169298  34.61422199796023  \\
            0.017481250979334097  39.09205638330908  \\
            0.017579460254498892  28.101841755342278  \\
            0.01767766952966369  27.59736995283299  \\
            0.01777587880482849  27.516267171802216  \\
            0.01787408807999329  27.502662886567027  \\
            0.01797229735515809  27.504906556872264  \\
            0.018070506630322888  27.51113772688371  \\
            0.018168715905487683  27.518000497066804  \\
            0.018266925180652482  27.524483330886717  \\
            0.01836513445581728  27.530326115287654  \\
            0.018463343730982077  27.53553971361442  \\
            0.018561553006146876  27.540260336643918  \\
            0.018659762281311672  27.54474064259798  \\
            0.01875797155647647  27.549467933109465  \\
            0.01885618083164127  27.555623894943356  \\
            0.01895439010680607  27.581609413165076  \\
            0.019052599381970868  27.625823685067505  \\
            0.019150808657135664  53.03537710392368  \\
            0.019249017932300463  27.541421303049496  \\
            0.019347227207465262  27.53209397773156  \\
            0.01944543648263006  27.529565586298606  \\
            0.01954364575779486  27.52534549642013  \\
            0.01964185503295966  27.520361046953404  \\
            0.019740064308124455  27.522160033709604  \\
            0.019838273583289254  27.591725226066337  \\
            0.019936482858454053  29.38526401617728  \\
            0.020034692133618852  30.845071562307165  \\
            0.02013290140878365  28.19811074257014  \\
            0.020231110683948447  27.897713371952438  \\
            0.020329319959113246  27.7961069374605  \\
            0.020427529234278045  27.747142868738038  \\
            0.020525738509442844  27.718938586565702  \\
            0.020623947784607643  27.70084378974228  \\
            0.02072215705977244  27.68836264738837  \\
            0.020820366334937238  27.679289505655504  \\
            0.020918575610102033  27.672422023693585  \\
            0.021016784885266832  27.667051432973743  \\
            0.02111499416043163  27.662733822877268  \\
            0.02121320343559643  27.659347442239554  \\
            0.021311412710761226  27.656293241791523  \\
            0.021409621985926025  27.65367390852488  \\
            0.021507831261090824  27.651381664992428  \\
            0.021606040536255623  27.64933995434515  \\
            0.021704249811420422  27.64749168675963  \\
            0.021802459086585218  27.64579512904877  \\
            0.021900668361750017  27.64422252765525  \\
            0.021998877636914816  27.642762357514922  \\
            0.022097086912079615  27.641428425006165  \\
            0.022195296187244414  27.640285607073395  \\
            0.022293505462409213  27.639525739838874  \\
            0.02239171473757401  27.63972777508331  \\
            0.022489924012738808  27.643020104446144  \\
            0.022588133287903607  27.6614507308758  \\
            0.022686342563068406  27.933250237220253  \\
            0.022784551838233205  28.048958311685492  \\
            0.022882761113398  27.682576373624975  \\
            0.0229809703885628  27.649383630885364  \\
            0.0230791796637276  27.63795813487258  \\
            0.023177388938892398  27.631556557472837  \\
            0.023275598214057194  27.626882419718743  \\
            0.023373807489221993  27.622922416788796  \\
            0.02347201676438679  27.619376553148367  \\
            0.023570226039551587  27.613633341758  \\
            0.023668435314716386  27.612364651450772  \\
            0.023766644589881186  27.61160212601141  \\
            0.023864853865045985  27.611993709927507  \\
            0.02396306314021078  27.61504691642684  \\
            0.02406127241537558  27.62445419959335  \\
            0.02415948169054038  27.650311371216585  \\
            0.024257690965705177  27.725732963752247  \\
            0.024355900240869977  27.999431779439522  \\
            0.024454109516034772  29.76627734351807  \\
            0.024552318791199575  2306.2009752537097  \\
            0.02465052806636437  31.303581458243396  \\
            0.024748737341529166  28.85681114129109  \\
            0.02484694661669397  28.316817151603885  \\
            0.024945155891858764  28.10537334102035  \\
            0.025043365167023567  27.99872057870711  \\
            0.025141574442188362  27.936863259248415  \\
            0.025239783717353158  27.89784270480299  \\
            0.02533799299251796  27.871957172301645  \\
            0.025436202267682756  27.85434760599386  \\
            0.025534411542847555  27.842360792531988  \\
            0.025632620818012354  27.834463311766246  \\
            0.02573083009317715  27.829744478841047  \\
            0.02582903936834195  27.827671734354645  \\
            0.025927248643506748  27.827966137295554  \\
            0.026025457918671547  27.83054168330469  \\
            0.026123667193836343  27.83548421044399  \\
            0.026221876469001145  27.843061534638828  \\
            0.02632008574416594  27.85376714981117  \\
            0.026418295019330736  27.868411396146943  \\
            0.02651650429449554  27.88829303035857  \\
            0.026614713569660334  27.91552343453807  \\
            0.026712922844825137  27.953667812466996  \\
            0.026811132119989933  28.009107738257875  \\
            0.026909341395154728  28.094233383001555  \\
            0.02700755067031953  28.235972503111597  \\
            0.027105759945484326  28.503179143540596  \\
            0.02720396922064913  29.12279961391746  \\
            0.027302178495813925  31.282430018906197  \\
            0.02740038777097872  57.201440247723475  \\
            0.027498597046143523  38.22369713412942  \\
            0.027596806321308318  29.147229010534442  \\
            0.02769501559647312  28.11282725556719  \\
            0.027793224871637916  27.834938517310558  \\
            0.027891434146802712  27.730841023909814  \\
            0.027989643421967515  27.684576179594576  \\
            0.02808785269713231  27.661951399620936  \\
            0.02818606197229711  27.650328604386345  \\
            0.028284271247461905  27.644293777045277  \\
            0.028382480522626704  27.641274349719904  \\
            0.028480689797791503  27.639951350082796  \\
            0.0285788990729563  27.639608770157267  \\
            0.0286771083481211  27.639840789970528  \\
            0.028775317623285897  27.64040996156448  \\
            0.028873526898450692  27.64117431501975  \\
            0.028971736173615495  27.642048024518417  \\
            0.02906994544878029  27.642979305943918  \\
            0.029168154723945093  27.64393756364834  \\
            0.02926636399910989  27.644905689750008  \\
            0.02936457327427469  27.645875306820557  \\
            0.029462782549439487  27.646843722148148  \\
            0.029560991824604282  27.647811835921168  \\
            0.029659201099769085  27.64878250833528  \\
            0.02975741037493388  27.6497589324562  \\
            0.029855619650098683  27.65074246182451  \\
            0.02995382892526348  27.65172891437885  \\
            0.030052038200428274  27.652701249816197  \\
            0.030150247475593077  27.653613278317575  \\
            0.030248456750757872  27.65434893963324  \\
            0.03034666602592267  27.65460265808109  \\
            0.03044487530108747  27.653428080612663  \\
            0.030543084576252266  27.646526392547607  \\
            0.030641293851417065  27.595034813073738  \\
            0.030739503126581864  27.707185971191013  \\
            0.030837712401746663  27.69481651052701  \\
            0.03093592167691146  27.69584350696108  \\
            0.031034130952076258  27.69347533708721  \\
            0.031132340227241057  27.705180640526898  \\
            0.031230549502405853  27.720148783345007  \\
            0.03132875877757066  27.73925724567804  \\
            0.031426968052735454  27.76409005204723  \\
            0.03152517732790025  27.79733955308506  \\
            0.03162338660306505  27.84372167907601  \\
            0.03172159587822985  27.91204955518625  \\
            0.031819805153394644  28.02040669684608  \\
            0.031918014428559446  28.2111089945299  \\
            0.03201622370372424  28.604382038453767  \\
            0.03211443297888904  29.66956511891186  \\
            0.03221264225405383  34.90071207146762  \\
            0.032310851529218636  7470.114857090897  \\
            0.03240906080438343  32.865062476923704  \\
            0.032507270079548234  28.90868800045238  \\
            0.03260547935471303  28.135511231759658  \\
            0.032703688629877825  27.872852271779365  \\
            0.03280189790504263  27.757473310720272  \\
            0.03290010718020742  27.698741570347313  \\
            0.032998316455372226  27.665909294674616  \\
            0.03309652573053702  27.646367218975882  \\
            0.03319473500570182  27.63422823393821  \\
            0.03329294428086662  27.626470954932103  \\
            0.033391153556031415  27.621428814122766  \\
            0.03348936283119622  27.61812892121012  \\
            0.03358757210636101  27.61597682108926  \\
            0.03368578138152581  27.61459547463813  \\
            0.03378399065669061  27.613738238257632  \\
            0.03388219993185541  27.613239565958278  \\
            0.03398040920702021  27.612985938579126  \\
            0.034078618482185005  27.61289812347339  \\
            0.0341768277573498  27.61292001747281  \\
            0.0342750370325146  27.6130114455005  \\
            0.0343732463076794  27.61314340196499  \\
            0.0344714555828442  27.613294841410045  \\
            0.034569664858009  27.613450472897455  \\
            0.03466787413317379  27.613599219389123  \\
            0.034766083408338595  27.61373312552355  \\
            0.03486429268350339  27.61384657083252  \\
            0.03496250195866819  27.613935700820107  \\
            0.03506071123383299  27.613998004832244  \\
            0.035158920508997785  27.61403200627915  \\
            0.03525712978416259  27.614037029582615  \\
            0.03535533905932738  27.614013026999725  \\
            0.035453548334492185  27.613960447937213  \\
            0.03555175760965698  27.613880143145302  \\
            0.035649966884821783  27.613773293918335  \\
            0.03574817615998658  27.613641362984637  \\
            0.035846385435151375  27.613486064056328  \\
            0.03594459471031618  27.61330934522066  \\
            0.03604280398548097  27.613113389128372  \\
            0.036141013260645775  27.612900626466033  \\
            0.03623922253581057  27.612673766176993  \\
            0.03633743181097537  27.612435843633225  \\
            0.03643564108614017  27.612190290160473  \\
            0.036533850361304965  27.611941029973032  \\
            0.03663205963646976  27.611692610632087  \\
            0.03673026891163456  27.611450380518363  \\
            0.03682847818679936  27.6112207246116  \\
            0.036926687461964154  27.61101138403589  \\
            0.03702489673712895  27.610831888574495  \\
            0.03712310601229375  27.610694146246214  \\
            0.03722131528745855  27.610613260149087  \\
            0.037319524562623343  27.610608667198235  \\
            0.037417733837788146  27.610705754887746  \\
            0.03751594311295294  27.610938188393174  \\
            0.037614152388117744  27.61135132482359  \\
            0.03771236166328254  27.612007323733653  \\
            0.037810570938447335  27.61299297919511  \\
            0.03790878021361214  27.61443203895474  \\
            0.038006989488776934  27.616505181079017  \\
            0.038105198763941736  27.619483557023624  \\
            0.03820340803910653  27.623787451462558  \\
            0.03830161731427133  27.63009391148131  \\
            0.03839982658943613  27.639545913270222  \\
            0.038498035864600925  27.65418841403831  \\
            0.03859624513976573  27.677960809486898  \\
            0.038694454414930524  27.719226756591613  \\
            0.038792663690095326  27.798286626021635  \\
            0.03889087296526012  27.975155936570435  \\
            0.03898908224042492  28.495275025185514  \\
            0.03908729151558972  31.312928017543722  \\
            0.039185500790754516  474.39455957458637  \\
            0.03928371006591932  31.17394409126724  \\
            0.039381919341084114  28.590525776983473  \\
            0.03948012861624891  28.055344451598053  \\
            0.03957833789141371  27.857364522073645  \\
            0.03967654716657851  27.76093792502601  \\
            0.03977475644174331  27.705302037805893  \\
            0.039872965716908106  27.668999258317797  \\
            0.0399711749920729  27.642804072173618  \\
            0.040069384267237704  27.622214133897863  \\
            0.0401675935424025  27.60859636687772  \\
            0.0402658028175673  27.60009774849186  \\
            0.0403640120927321  27.593459353609934  \\
            0.04046222136789689  27.58815094814919  \\
            0.040560430643061696  27.58382056692017  \\
            0.04065863991822649  27.58022682361442  \\
            0.040756849193391294  27.577199491459314  \\
            0.04085505846855609  27.57461560138749  \\
            0.040953267743720885  27.572384438208488  \\
            0.04105147701888569  27.57043783319458  \\
            0.04114968629405048  27.568723713943932  \\
            0.041247895569215286  27.567201719465785  \\
            0.04134610484438008  27.56584015140992  \\
            0.04144431411954488  27.564613819730916  \\
            0.04154252339470968  27.563502488760726  \\
            0.041640732669874475  27.56248974484112  \\
            0.04173894194503927  27.561562153979754  \\
            0.041837151220204066  27.560708630009252  \\
            0.04193536049536887  27.559919956052017  \\
            0.042033569770533664  27.55918841537344  \\
            0.04213177904569846  27.558507506206688  \\
            0.04222998832086326  27.557871716610975  \\
            0.04232819759602806  27.55727634893712  \\
            0.04242640687119286  27.55671737887622  \\
            0.042524616146357656  27.5561913418421  \\
            0.04262282542152245  27.555695242377652  \\
            0.042721034696687255  27.555226480432182  \\
            0.04281924397185205  27.554782788976034  \\
            0.04291745324701685  27.554362185839878  \\
            0.04301566252218165  27.553962931388263  \\
            0.043113871797346444  27.5535834941726  \\
            0.043212081072511246  27.55322252280859  \\
            0.04331029034767604  27.55287882057953  \\
            0.043408499622840845  27.55255132631297  \\
            0.04350670889800564  27.552239096204534  \\
            0.043604918173170436  27.551941290116176  \\
            0.04370312744833524  27.551657157365344  \\
            0.043801336723500034  27.55138602807359  \\
            0.043899545998664836  27.551127302385716  \\
            0.04399775527382963  27.550880442973522  \\
            0.04409596454899443  27.550644968556757  \\
            0.04419417382415923  27.55042044725092  \\
            0.044292383099324026  27.550206491944838  \\
            0.04439059237448883  27.55000275521363  \\
            0.044488801649653624  27.54980892630985  \\
            0.04458701092481843  27.54962472636536  \\
            0.04468522019998322  27.549449906652256  \\
            0.04478342947514802  27.549284245429035  \\
            0.04488163875031282  27.549127545237507  \\
            0.044979848025477616  27.548979631028363  \\
            0.04507805730064242  27.54884034921755  \\
            0.045176266575807214  27.54870956485658  \\
            0.04527447585097201  27.54858716064183  \\
            0.04537268512613681  27.548473036497803  \\
            0.04547089440130161  27.548367106908053  \\
            0.04556910367646641  27.548269301641866  \\
            0.045667312951631206  27.548179563571527  \\
            0.045765522226796  27.548097848631084  \\
            0.045863731501960804  27.548024125331782  \\
            0.0459619407771256  27.547958373679386  \\
            0.0460601500522904  27.54790058497535  \\
            0.0461583593274552  27.547850761953253  \\
            0.046256568602619993  27.54780891751108  \\
            0.046354777877784796  27.547775074450765  \\
            0.04645298715294959  27.547749266717915  \\
            0.04655119642811439  27.547731537561457  \\
            0.04664940570327918  27.547721940235242  \\
            0.046747614978443985  27.547720538216378  \\
            0.04684582425360878  27.547727404535884  \\
            0.04694403352877358  27.547742622204595  \\
            0.04704224280393838  27.547766284002034  \\
            0.047140452079103175  27.54779849330975  \\
            0.04723866135426797  27.547839363533587  \\
            0.04733687062943277  27.547889019038827  \\
            0.04743507990459757  27.547947594521865  \\
            0.04753328917976237  27.548015235993617  \\
            0.04763149845492717  27.548092101299016  \\
            0.04772970773009197  27.548178360014198  \\
            0.047827917005256765  27.548274194012837  \\
            0.04792612628042156  27.548379798758027  \\
            0.04802433555558636  27.54849538263388  \\
            0.04812254483075116  27.548621168191033  \\
            0.04822075410591596  27.54875739349215  \\
            0.04831896338108076  27.54890431171774  \\
            0.04841717265624555  27.549062193087394  \\
            0.048515381931410355  27.54923132475329  \\
            0.04861359120657515  27.549412012886958  \\
            0.04871180048173995  27.549604583003088  \\
            0.04881000975690475  27.549809381508997  \\
            0.048908219032069544  27.55002677718939  \\
            0.04900642830723435  27.55025716226496  \\
            0.04910463758239915  27.55050095399665  \\
            0.049202846857563945  27.550758596802638  \\
            0.04930105613272874  27.551030563808652  \\
            0.049399265407893536  27.551317359421613  \\
            0.04949747468305833  27.551619520289567  \\
            0.04959568395822314  27.55193761925369  \\
            0.04969389323338794  27.552272267520742  \\
            0.04979210250855273  27.552624116859544  \\
            0.04989031178371753  27.552993863949204  \\
            0.049988521058882324  27.5533822533554  \\
            0.05008673033404713  27.553790081248557  \\
            0.05018493960921193  27.554218200554228  \\
            0.050283148884376724  27.554667523975027  \\
            0.05038135815954152  27.555139031392645  \\
            0.050479567434706316  27.555633774279876  \\
            0.050577776709871125  27.556152881450792  \\
            0.05067598598503592  27.55669756827086  \\
            0.050774195260200716  27.557269141924312  \\
            0.05087240453536551  27.557869012108593  \\
            0.05097061381053031  27.558498700033933  \\
            0.05106882308569511  27.559159848392817  \\
            0.05116703236085991  27.55985423481226  \\
            0.05126524163602471  27.560583784947042  \\
            0.051363450911189504  27.561350586818385  \\
            0.0514616601863543  27.562156908137535  \\
            0.0515598694615191  27.563005216809213  \\
            0.0516580787366839  27.563898200570783  \\
            0.05175628801184869  27.564838792803684  \\
            0.051854497287013496  27.565830199534684  \\
            0.05195270656217829  27.566875931257933  \\
            0.052050915837343094  27.567979839184762  \\
            0.05214912511250789  27.569146155412298  \\
            0.052247334387672685  27.57037954061838  \\
            0.05234554366283748  27.571685137813237  \\
            0.05244375293800229  27.57306863411465  \\
            0.052541962213167086  27.574536333657733  \\
            0.05264017148833188  27.576095239658443  \\
            0.05273838076349668  27.57775315230634  \\
            0.05283659003866147  27.57951878112429  \\
            0.05293479931382628  27.581401878539577  \\
            0.05303300858899108  27.583413394913155  \\
            0.05313121786415587  27.585565663783168  \\
            0.05322942713932067  27.587872619495652  \\
            0.053327636414485464  27.590350058247044  \\
            0.053425845689650274  27.59301595028464  \\
            0.05352405496481507  27.595890814351552  \\
            0.053622264239979865  27.598998171760762  \\
            0.05372047351514466  27.602365099576108  \\
            0.053818682790309456  27.606022904675452  \\
            0.053916892065474266  27.610007956739775  \\
            0.05401510134063906  27.614362719105138  \\
            0.05411331061580386  27.619137036370788  \\
            0.05421151989096865  27.62438975316929  \\
            0.05430972916613345  27.630190768674517  \\
            0.05440793844129826  27.636623663129928  \\
            0.05450614771646305  27.64378909086933  \\
            0.05460435699162785  27.651809205515484  \\
            0.054702566266792645  27.660833494916137  \\
            0.05480077554195744  27.671046569007622  \\
            0.05489898481712225  27.68267868451373  \\
            0.054997194092287045  27.696020174072956  \\
            0.05509540336745184  27.71144153336787  \\
            0.055193612642616637  27.72942186418111  \\
            0.05529182191778143  27.750589917230716  \\
            0.05539003119294624  27.77578456794176  \\
            0.05548824046811104  27.80614604710064  \\
            0.05558644974327583  27.843257259928702  \\
            0.05568465901844063  27.88936939978013  \\
            0.055782868293605424  27.94777480737831  \\
            0.05588107756877023  28.02344841049657  \\
            0.05597928684393503  28.124204423183258  \\
            0.056077496119099825  28.26290260135467  \\
            0.05617570539426462  28.46195262927227  \\
            0.056273914669429416  28.76331938572499  \\
            0.05637212394459422  29.253267622311174  \\
            0.056470333219759014  30.132909410164295  \\
            0.05656854249492381  31.96356133221794  \\
            0.05666675177008861  36.82457851730194  \\
            0.05676496104525341  58.16268356914382  \\
            0.05686317032041821  386.26523318816055  \\
            0.056961379595583006  47.891551790682314  \\
            0.0570595888707478  34.189990134778085  \\
            0.0571577981459126  30.660295923545377  \\
            0.0572560074210774  29.29252543544926  \\
            0.0573542166962422  28.635232506880463  \\
            0.057452425971407  28.273850757675046  \\
            0.057550635246571794  28.056015388580025  \\
            0.05764884452173659  27.915651589524625  \\
            0.057747053796901385  27.820526682280075  \\
            0.057845263072066194  27.753468061416047  \\
            0.05794347234723099  27.70467776242938  \\
            0.058041681622395785  27.668250502631174  \\
            0.05813989089756058  27.640463889324284  \\
            0.05823810017272538  27.61888216556489  \\
            0.058336309447890186  27.601860284595453  \\
            0.05843451872305498  27.58825670494004  \\
            0.05853272799821978  27.577260460354836  \\
            0.05863093727338457  27.568283472214016  \\
            0.05872914654854938  27.56089149581682  \\
            0.05882735582371418  27.554758673564  \\
            0.058925565098878974  27.549636938112744  \\
            0.05902377437404377  27.54533497942704  \\
            0.059121983649208565  27.541703521739635  \\
            0.059220192924373374  27.538624842213153  \\
            0.05931840219953817  27.536005195979968  \\
            0.059416611474702966  27.53376926589786  \\
            0.05951482074986776  27.531856043229222  \\
            0.05961303002503256  27.53021573584819  \\
            0.059711239300197366  27.528807421135834  \\
            0.05980944857536216  27.527597248287066  \\
            0.05990765785052696  27.526557045693416  \\
            0.06000586712569175  27.525663238099053  \\
            0.06010407640085655  27.524895990194292  \\
            0.06020228567602136  27.524238532975318  \\
            0.060300494951186154  27.523676622295337  \\
            0.06039870422635095  27.523198106713743  \\
            0.060496913501515745  27.522792576990607  \\
            0.06059512277668054  27.522451081359034  \\
            0.06069333205184534  27.522165892312422  \\
            0.060791541327010146  27.52193031697788  \\
            0.06088975060217494  27.521738537952118  \\
            0.06098795987733974  27.52158548307968  \\
            0.06108616915250453  27.521466715782513  \\
            0.061184378427669335  27.521378345056753  \\
            0.06128258770283413  27.521316946921896  \\
            0.061380796977998926  27.520629173084316  \\
            0.06147900625316373  27.520375107734775  \\
            0.061577215528328524  27.520427315530217  \\
            0.06167542480349333  27.52063365492194  \\
            0.06177363407865812  27.520956641281774  \\
            0.06187184335382292  27.521369423901774  \\
            0.061970052628987714  27.5218569379314  \\
            0.062068261904152516  27.522413070407037  \\
            0.06216647117931732  27.52303894414058  \\
            0.062264680454482114  27.523744194011936  \\
            0.06236288972964691  27.52454952885835  \\
            0.062461099004811706  27.525492886801256  \\
            0.06255930827997651  27.526642393951203  \\
            0.06265751755514132  27.528128882533622  \\
            0.06275572683030611  27.530226509677213  \\
            0.06285393610547091  27.533603075915895  \\
            0.0629521453806357  27.540316665542186  \\
            0.0630503546558005  27.560090740747547  \\
            0.0631485639309653  27.738427984162083  \\
            0.0632467732061301  27.761765899999034  \\
            0.0633449824812949  27.53896506525033  \\
            0.0634431917564597  27.527719356910797  \\
            0.06354140103162449  27.525933253101382  \\
            0.06363961030678929  27.525837651483414  \\
            0.06373781958195408  27.526202878465828  \\
            0.06383602885711889  27.52671981910876  \\
            0.06393423813228369  27.527290147741628  \\
            0.06403244740744848  27.527878134089935  \\
            0.06413065668261328  27.528469901423545  \\
            0.06422886595777808  27.52906002943677  \\
            0.06432707523294287  27.529646588027678  \\
            0.06442528450810767  27.530229127575527  \\
            0.06452349378327248  27.530807810596922  \\
            0.06462170305843727  27.531383034597827  \\
            0.06471991233360207  27.531955258930576  \\
            0.06481812160876686  27.53252492306995  \\
            0.06491633088393166  27.533092421162237  \\
            0.06501454015909647  27.533658088029796  \\
            0.06511274943426126  27.53422220122839  \\
            0.06521095870942606  27.534784984460792  \\
            0.06530916798459085  27.53534661377048  \\
            0.06540737725975565  27.535907222868794  \\
            0.06550558653492046  27.536466910757404  \\
            0.06560379581008526  27.53702574457933  \\
            0.06570200508525005  27.537583766587506  \\
            0.06580021436041485  27.538140996534825  \\
            0.06589842363557964  27.53869743638459  \\
            0.06599663291074445  27.539253072043536  \\
            0.06609484218590925  27.539807877002957  \\
            0.06619305146107404  27.540361813494965  \\
            0.06629126073623884  27.540914835476556  \\
            0.06638947001140363  27.54146688965804  \\
            0.06648767928656844  27.542017915651638  \\
            0.06658588856173324  27.54256784963133  \\
            0.06668409783689803  27.543116622809958  \\
            0.06678230711206283  27.543664163957207  \\
            0.06688051638722763  27.54421039902009  \\
            0.06697872566239244  27.544755251988953  \\
            0.06707693493755723  27.545298645686994  \\
            0.06717514421272203  27.545840501841578  \\
            0.06727335348788682  27.546380741403073  \\
            0.06737156276305162  27.546919284818756  \\
            0.06746977203821643  27.5474560519297  \\
            0.06756798131338122  27.547990962723855  \\
            0.06766619058854602  27.548523936725747  \\
            0.06776439986371081  27.549054893233382  \\
            0.06786260913887561  27.54958375135712  \\
            0.06796081841404042  27.550110429933973  \\
            0.06805902768920521  27.550634846803952  \\
            0.06815723696437001  27.55115691931224  \\
            0.0682554462395348  27.551676564227943  \\
            0.0683536555146996  27.552193696263053  \\
            0.06845186478986441  27.55270822871753  \\
            0.0685500740650292  27.5532200724475  \\
            0.068648283340194  27.553729136323017  \\
            0.0687464926153588  27.554235325290634  \\
            0.0688447018905236  27.554738540700388  \\
            0.0689429111656884  27.555238678675597  \\
            0.0690411204408532  27.555735630543797  \\
            0.069139329716018  27.556229280682263  \\
            0.06923753899118279  27.55671950556478  \\
            0.06933574826634759  27.55720617309536  \\
            0.0694339575415124  27.557689140676892  \\
            0.06953216681667719  27.558168253964247  \\
            0.06963037609184199  27.5586433439108  \\
            0.06972858536700678  27.559114226597085  \\
            0.06982679464217158  27.55958069896509  \\
            0.06992500391733639  27.560042537493807  \\
            0.07002321319250118  27.560499493811307  \\
            0.07012142246766598  27.560951291865063  \\
            0.07021963174283077  27.561397623970606  \\
            0.07031784101799557  27.5618381454137  \\
            0.07041605029316038  27.562272469181355  \\
            0.07051425956832517  27.562700159532522  \\
            0.07061246884348997  27.563120725077574  \\
            0.07071067811865477  27.563533609472426  \\
            0.07080888739381958  27.56393818042172  \\
            0.07090709666898437  27.564333721865733  \\
            0.07100530594414917  27.56471941269201  \\
            0.07110351521931396  27.565094318599673  \\
            0.07120172449447876  27.56545736519605  \\
            0.07129993376964357  27.565807321764215  \\
            0.07139814304480836  27.56614276539887  \\
            0.07149635231997316  27.566462059416292  \\
            0.07159456159513795  27.56676329435801  \\
            0.07169277087030275  27.567044261188734  \\
            0.07179098014546756  27.56730239890659  \\
            0.07188918942063235  27.567534685688212  \\
            0.07198739869579715  27.567737614983606  \\
            0.07208560797096195  27.567907049726752  \\
            0.07218381724612674  27.568038214690077  \\
            0.07228202652129155  27.568125357092484  \\
            0.07238023579645635  27.56816163100228  \\
            0.07247844507162114  27.568140284101155  \\
            0.07257665434678594  27.568047405564563  \\
            0.07267486362195073  27.568421100045725  \\
            0.07277307289711553  27.57298039681567  \\
            0.07287128217228034  27.573876837455366  \\
            0.07296949144744513  27.574767906900977  \\
            0.07306770072260993  27.575653544313113  \\
            0.07316590999777473  27.5765336871873  \\
            0.07326411927293952  27.577408270988006  \\
            0.07336232854810432  27.578277229187023  \\
            0.07346053782326913  27.579140492974947  \\
            0.07355874709843392  27.579997991079054  \\
            0.07365695637359872  27.58084965080067  \\
            0.07375516564876351  27.58169539587903  \\
            0.07385337492392831  27.58253514813904  \\
            0.0739515841990931  27.583368826705364  \\
            0.0740497934742579  27.5841963476708  \\
            0.07414800274942271  27.58501762424794  \\
            0.0742462120245875  27.585832566917876  \\
            0.0743444212997523  27.586641082557506  \\
            0.0744426305749171  27.587443074773528  \\
            0.07454083985008189  27.58823844387353  \\
            0.07463904912524669  27.58902708644023  \\
            0.0747372584004115  27.589808894931416  \\
            0.07483546767557629  27.59058375825164  \\
            0.07493367695074109  27.591351560894072  \\
            0.07503188622590588  27.592112182971313  \\
            0.07513009550107068  27.592865500145482  \\
            0.07522830477623549  27.593611383597686  \\
            0.07532651405140028  27.594349698963562  \\
            0.07542472332656508  27.595080307372037  \\
            0.07552293260172988  27.595803064666587  \\
            0.07562114187689467  27.596517820881218  \\
            0.07571935115205948  27.597224420803283  \\
            0.07581756042722428  27.597922703033394  \\
            0.07591576970238907  27.598612500942796  \\
            0.07601397897755387  27.599293641547117  \\
            0.07611218825271866  27.599965945573874  \\
            0.07621039752788347  27.60062922856663  \\
            0.07630860680304827  27.601283299366106  \\
            0.07640681607821306  27.60192796150276  \\
            0.07650502535337786  27.602563013123685  \\
            0.07660323462854265  27.603188246804113  \\
            0.07670144390370746  27.603803451766762  \\
            0.07679965317887226  27.604408412797824  \\
            0.07689786245403706  27.605002912822556  \\
            0.07699607172920185  27.605586733883726  \\
            0.07709428100436666  27.606159659398735  \\
            0.07719249027953146  27.60672147659614  \\
            0.07729069955469625  27.607271980181448  \\
            0.07738890882986105  27.607810976625878  \\
            0.07748711810502584  27.608338289561598  \\
            0.07758532738019065  27.608853767698335  \\
            0.07768353665535545  27.609357292759388  \\
            0.07778174593052024  27.609848793055274  \\
            0.07787995520568504  27.610328256465703  \\
            0.07797816448084983  27.610795751837962  \\
            0.07807637375601464  27.61125145219897  \\
            0.07817458303117944  27.611695667661323  \\
            0.07827279230634424  27.612128886637855  \\
            0.07837100158150903  27.612551829935846  \\
            0.07846921085667383  27.61296552033302  \\
            0.07856742013183864  27.613371376141615  \\
            0.07866562940700343  27.61377133188677  \\
            0.07876383868216823  27.61416800125407  \\
            0.07886204795733302  27.61456489422644  \\
            0.07896025723249782  27.614966710353038  \\
            0.07905846650766263  27.615379739386608  \\
            0.07915667578282742  27.61581241232803  \\
            0.07925488505799222  27.616276070649175  \\
            0.07935309433315701  27.616786047489615  \\
            0.07945130360832181  27.617363218120577  \\
            0.07954951288348662  27.618036248005915  \\
            0.07964772215865142  27.61884491229163  \\
            0.07974593143381621  27.619845091727473  \\
            0.079844140708981  27.621116453332153  \\
            0.0799423499841458  27.622774545673042  \\
            0.08004055925931061  27.624990380089592  \\
            0.08013876853447541  27.628023149416485  \\
            0.0802369778096402  27.632276934249564  \\
            0.080335187084805  27.638403306095434  \\
            0.0804333963599698  27.64749670132199  \\
            0.0805316056351346  27.66149014542602  \\
            0.0806298149102994  27.684020276060743  \\
            0.0807280241854642  27.72251007157803  \\
            0.08082623346062899  27.793862678362302  \\
            0.08092444273579379  27.943017172942657  \\
            0.0810226520109586  28.321623649533805  \\
            0.08112086128612339  29.71472902255918  \\
            0.08121907056128819  45.91868694116537  \\
            0.08131727983645298  61.02667714295901  \\
            0.08141548911161778  31.221316688891868  \\
            0.08151369838678259  29.013188782660663  \\
            0.08161190766194738  28.39497043210791  \\
            0.08171011693711218  28.133177776467658  \\
            0.08180832621227697  27.996314875766934  \\
            0.08190653548744177  27.914967824872544  \\
            0.08200474476260658  27.86226046426543  \\
            0.08210295403777138  27.825930922720655  \\
            0.08220116331293617  27.79970206073515  \\
            0.08229937258810097  27.78007328884228  \\
            0.08239758186326576  27.764958710090813  \\
            0.08249579113843057  27.753047943742043  \\
            0.08259400041359537  27.743481478792432  \\
            0.08269220968876016  27.735674913475496  \\
            0.08279041896392496  27.729218640206277  \\
            0.08288862823908975  27.72381799569415  \\
            0.08298683751425455  27.71925617219326  \\
            0.08308504678941936  27.715370470910234  \\
            0.08318325606458415  27.71203665804598  \\
            0.08328146533974895  27.70915839511227  \\
            0.08337967461491375  27.706659938103087  \\
            0.08347788389007854  27.70448099151751  \\
            0.08357609316524334  27.702573016730703  \\
            0.08367430244040813  27.70089654083235  \\
            0.08377251171557294  27.699419163881345  \\
            0.08387072099073774  27.69811406588587  \\
            0.08396893026590253  27.696958872862684  \\
            0.08406713954106733  27.677044015645738  \\
            0.08416534881623212  27.676404232574704  \\
            0.08426355809139692  27.675755404241098  \\
            0.08436176736656173  27.675150083071763  \\
            0.08445997664172653  27.674582757842032  \\
            0.08455818591689132  27.6740490090943  \\
            0.08465639519205612  27.673545142486258  \\
            0.08475460446722091  27.67306808039098  \\
            0.08485281374238572  27.672615258054503  \\
            0.08495102301755052  27.672184532871004  \\
            0.08504923229271531  27.671774102762758  \\
            0.08514744156788011  27.67138246292938  \\
            0.0852456508430449  27.671008351772816  \\
            0.08534386011820971  27.670650716683706  \\
            0.08544206939337451  27.67030868251181  \\
            0.0855402786685393  27.669981522129834  \\
            0.0856384879437041  27.669668639461715  \\
            0.0857366972188689  27.66936954825308  \\
            0.0858349064940337  27.669083857573053  \\
            0.0859331157691985  27.668811259855335  \\
            0.0860313250443633  27.668551519616024  \\
            0.08612953431952809  27.668304464899446  \\
            0.08622774359469289  27.66806998006071  \\
            0.0863259528698577  27.667847998754308  \\
            0.08642416214502249  27.667638499964742  \\
            0.08652237142018729  27.66744150214908  \\
            0.08662058069535208  27.66725706109234  \\
            0.08671878997051688  27.667085265753006  \\
            0.08681699924568169  27.666926236788534  \\
            0.08691520852084648  27.666780124387074  \\
            0.08701341779601128  27.666647105795203  \\
            0.08711162707117608  27.66652738611311  \\
            0.08720983634634087  27.666421195709216  \\
            0.08730804562150568  27.666328790435912  \\
            0.08740625489667048  27.666250451711495  \\
            0.08750446417183527  27.66618648591757  \\
            0.08760267344700007  27.666137224763858  \\
            0.08770088272216486  27.666103026328347  \\
            0.08779909199732967  27.66608427492592  \\
            0.08789730127249447  27.666081382748207  \\
            0.08799551054765926  27.66609479035071  \\
            0.08809371982282406  27.66612496822894  \\
            0.08819192909798886  27.6661724183461  \\
            0.08829013837315366  27.66623767588573  \\
            0.08838834764831846  27.66632131082095  \\
            0.08848655692348326  27.66642393093312  \\
            0.08858476619864805  27.666546182827723  \\
            0.08868297547381285  27.66668875635139  \\
            0.08878118474897766  27.666852386288795  \\
            0.08887939402414245  27.667037855691806  \\
            0.08897760329930725  27.66724600038632  \\
            0.08907581257447204  27.66747771211986  \\
            0.08917402184963685  27.667733943662085  \\
            0.08927223112480165  27.668015712341674  \\
            0.08937044039996644  27.668324106907228  \\
            0.08946864967513124  27.668660292241356  \\
            0.08956685895029604  27.66902551664109  \\
            0.08966506822546084  27.66942111823056  \\
            0.08976327750062564  27.6698485333401  \\
            0.08986148677579044  27.670309304646292  \\
            0.08995969605095523  27.670805091904786  \\
            0.09005790532612003  27.67133768105898  \\
            0.09015611460128484  27.671908997799648  \\
            0.09025432387644963  27.672521119587355  \\
            0.09035253315161443  27.67317629088164  \\
            0.09045074242677922  27.67387693964598  \\
            0.09054895170194402  27.67462569569925  \\
            0.09064716097710883  27.675425411504587  \\
            0.09074537025227362  27.67627918482451  \\
            0.09084357952743842  27.67719038639752  \\
            0.09094178880260322  27.678162687387335  \\
            0.09103999807776801  27.67920009391349  \\
            0.09113820735293282  27.68030698558814  \\
            0.09123641662809762  27.68148815655335  \\
            0.09133462590326241  27.68274886635281  \\
            0.09143283517842721  27.68409489475504  \\
            0.091531044453592  27.685532606651705  \\
            0.09162925372875681  27.68706902474733  \\
            0.09172746300392161  27.688711915787238  \\
            0.0918256722790864  27.690469887694565  \\
            0.0919238815542512  27.69235250356106  \\
            0.092022090829416  27.69437041512555  \\
            0.0921203001045808  27.69653551739076  \\
            0.0922185093797456  27.69886112968108  \\
            0.0923167186549104  27.701362210734867  \\
            0.09241492793007519  27.704055611067414  \\
            0.09251313720523999  27.706960374996914  \\
            0.09261134648040478  27.71009809816689  \\
            0.09270955575556959  27.713493359700244  \\
            0.09280776503073439  27.717174240546942  \\
            0.09290597430589918  27.721172953055007  \\
            0.09300418358106398  27.725526609619433  \\
            0.09310239285622877  27.73027816454524  \\
            0.09320060213139357  27.735477577650233  \\
            0.09329881140655837  27.7411832632881  \\
            0.09339702068172318  27.74746390449949  \\
            0.09349522995688797  27.754400747159632  \\
            0.09359343923205277  27.762090522243422  \\
            0.09369164850721756  27.770649204685913  \\
            0.09378985778238236  27.780216899953782  \\
            0.09388806705754715  27.79096425533629  \\
            0.09398627633271196  27.803100984794142  \\
            0.09408448560787676  27.81688733723661  \\
            0.09418269488304155  27.832649744934454  \\
            0.09428090415820635  27.850802504400257  \\
            0.09437911343337115  27.87187831795225  \\
            0.09447732270853594  27.896572124717032  \\
            0.09457553198370075  27.92580531295602  \\
            0.09467374125886555  27.96082198222471  \\
            0.09477195053403034  28.003337053200994  \\
            0.09487015980919514  28.055770971950523  \\
            0.09496836908435993  28.121634422558852  \\
            0.09506657835952474  28.206184071284436  \\
            0.09516478763468954  28.31759268322559  \\
            0.09526299690985433  28.46915383975559  \\
            0.09536120618501913  28.683718310747327  \\
            0.09545941546018394  29.003375364420336  \\
            0.09555762473534873  29.51286782882781  \\
            0.09565583401051353  30.404414330425933  \\
            0.09575404328567833  32.19404399947244  \\
            0.09585225256084312  36.678842091534776  \\
            0.09595046183600793  54.05761081703904  \\
            0.09604867111117273  1607.1918712246152  \\
            0.09614688038633752  54.37902358054147  \\
            0.09624508966150232  35.616527509461314  \\
            0.09634329893666711  31.21234731416597  \\
            0.09644150821183192  29.587116361130168  \\
            0.09653971748699672  28.83512869957161  \\
            0.09663792676216151  28.437070725340188  \\
            0.09673613603732631  28.20840786843591  \\
            0.0968343453124911  28.071674838584187  \\
            0.09693255458765591  27.99087994678528  \\
            0.09703076386282071  27.94893215913132  \\
            0.0971289731379855  27.9393335805698  \\
            0.0972271824131503  27.964180359257718  \\
            0.0973253916883151  28.037548292255718  \\
            0.0974236009634799  28.20033142279092  \\
            0.0975218102386447  28.57775634153795  \\
            0.0976200195138095  29.67700001398612  \\
            0.09771822878897429  35.283888339636356  \\
            0.09781643806413909  1212.6881666556617  \\
            0.0979146473393039  36.03849307062275  \\
            0.0980128566144687  30.700333899109317  \\
            0.09811106588963349  29.9400396664686  \\
        }
        ;
    \addplot[color={rgb,1:red,1.0;green,0.0;blue,0.0}, name path={0e0fdbe2-ce04-4e27-84e1-442de55e62a6}, draw opacity={1.0}, line width={1}, dotted, forget plot]
        table[row sep={\\}]
        {
            \\
            0.0  1001.3809339451059  \\
            9.820927516479829e-5  1001.3816535954381  \\
            0.00019641855032959658  1001.3838628706625  \\
            0.00029462782549439484  1001.387563509338  \\
            0.00039283710065919316  1001.3927584852978  \\
            0.0004910463758239915  1001.3994520196796  \\
            0.0005892556509887897  1039.6701482469307  \\
            0.000687464926153588  1012.8389950146326  \\
            0.0007856742013183863  1008.4774005545846  \\
            0.0008838834764831845  1006.6850704463204  \\
            0.000982092751647983  1005.7094589564655  \\
            0.0010803020268127812  1005.0969885161269  \\
            0.0011785113019775794  1004.6777888158707  \\
            0.0012767205771423778  1004.374012422379  \\
            0.001374929852307176  1004.1448801045368  \\
            0.0014731391274719742  1003.96700046813  \\
            0.0015713484026367726  1003.8260086119885  \\
            0.001669557677801571  1003.7125966166943  \\
            0.001767766952966369  1003.6204766835324  \\
            0.0018659762281311677  1003.5452495222022  \\
            0.001964185503295966  1003.4837527192071  \\
            0.002062394778460764  1003.4336568033261  \\
            0.0021606040536255623  1003.3932133446593  \\
            0.002258813328790361  1003.3610884273756  \\
            0.0023570226039551587  1003.3362490485899  \\
            0.0024552318791199574  1003.3178878913671  \\
            0.0025534411542847556  1003.3053686286642  \\
            0.002651650429449554  1003.2981860495062  \\
            0.002749859704614352  1003.2959387673384  \\
            0.0028480689797791506  1111.0641328709264  \\
            0.0029462782549439484  1051.9376017494092  \\
            0.003044487530108747  1035.1274010812176  \\
            0.0031426968052735453  1027.0240608392103  \\
            0.0032409060804383435  1022.2786582520682  \\
            0.003339115355603142  1019.1995458399496  \\
            0.0034373246307679403  1017.0443393359311  \\
            0.003535533905932738  1015.4512799149755  \\
            0.0036337431810975363  1014.2289150600441  \\
            0.0037319524562623354  1013.2655922340953  \\
            0.003830161731427133  1012.493012777579  \\
            0.003928371006591932  1011.8613713784259  \\
            0.004026580281756729  1011.3397294132895  \\
            0.004124789556921528  1010.9042076317714  \\
            0.004222998832086326  1010.5391027422879  \\
            0.004321208107251125  1010.2316484131126  \\
            0.004419417382415923  1009.9724852651839  \\
            0.004517626657580722  1009.7545537201971  \\
            0.004615835932745519  1009.5722353797648  \\
            0.0047140452079103175  1009.4211250979334  \\
            0.004812254483075116  1009.2978036261405  \\
            0.004910463758239915  1009.1996674552985  \\
            0.005008673033404713  1009.1245932090152  \\
            0.005106882308569511  1009.070986723691  \\
            0.005205091583734309  1009.0376276818512  \\
            0.005303300858899108  1009.0236899444202  \\
            0.005401510134063906  1009.028712097416  \\
            0.005499719409228704  1009.052488172089  \\
            0.005597928684393502  1009.0951592335597  \\
            0.005696137959558301  1009.1571345638029  \\
            0.0057943472347230995  1009.239248629966  \\
            0.005892556509887897  1009.342680828867  \\
            0.005990765785052695  1009.4690776429394  \\
            0.006088975060217494  1009.6206777492455  \\
            0.0061871843353822915  1009.8003915006474  \\
            0.0062853936105470905  1010.0120636550108  \\
            0.006383602885711889  1010.2607380229441  \\
            0.006481812160876687  1010.5530693386883  \\
            0.006580021436041485  1010.8979550098934  \\
            0.006678230711206284  1011.3074751027377  \\
            0.006776439986371082  1011.7982994240883  \\
            0.006874649261535881  1012.3939737545137  \\
            0.006972858536700678  1013.1286834851564  \\
            0.007071067811865476  1014.0535714045278  \\
            0.007169277087030275  1015.2481353199809  \\
            0.007267486362195073  1016.8416401171779  \\
            0.007365695637359872  1019.0562186419791  \\
            0.007463904912524671  1022.3010032507327  \\
            0.007562114187689468  1027.4000471918623  \\
            0.007660323462854266  1036.225337708358  \\
            0.0077585327380190645  1053.8244679021748  \\
            0.007856742013183864  1097.9674748599675  \\
            0.007954951288348661  1270.3639146046853  \\
            0.008053160563513458  4184.670205790675  \\
            0.008151369838678257  1571.5646809355287  \\
            0.008249579113843056  1136.3855639965377  \\
            0.008347788389007854  1063.7262274069913  \\
            0.008445997664172653  1040.2079579368651  \\
            0.00854420693933745  1030.1032107320468  \\
            0.00864241621450225  1025.0861210832807  \\
            0.008740625489667048  1022.4245063698518  \\
            0.008838834764831846  1021.0175306640728  \\
            0.008937044039996645  1020.35747786069  \\
            0.009035253315161444  1020.1879231954476  \\
            0.009133462590326241  1020.3719646405882  \\
            0.009231671865491039  1020.8352703966619  \\
            0.009329881140655836  1021.5393267953639  \\
            0.009428090415820635  1022.4681598361595  \\
            0.009526299690985434  1023.621648549899  \\
            0.009624508966150231  1025.0123588456208  \\
            0.00972271824131503  1026.6645193714035  \\
            0.00982092751647983  1028.614506215604  \\
            0.009919136791644627  1030.9126725253218  \\
            0.010017346066809426  1033.626617677251  \\
            0.010115555341974223  1036.8462899300978  \\
            0.010213764617139022  1040.6916402334377  \\
            0.010311973892303821  1045.3240632962802  \\
            0.010410183167468619  1050.9636785016594  \\
            0.010508392442633416  1057.9159638691758  \\
            0.010606601717798215  1066.6138280137484  \\
            0.010704810992963013  1077.686130500028  \\
            0.010803020268127812  1092.0732906708117  \\
            0.010901229543292609  1111.2306428524882  \\
            0.010999438818457408  1137.5041489066984  \\
            0.011097648093622207  1174.8668523070123  \\
            0.011195857368787004  1230.4715697112058  \\
            0.011294066643951804  1318.2443108224581  \\
            0.011392275919116603  1468.3179699827806  \\
            0.0114904851942814  1756.852471304471  \\
            0.011588694469446199  2434.8943773342808  \\
            0.011686903744610996  5031.67342988068  \\
            0.011785113019775794  17767.12194377187  \\
            0.011883322294940593  6196.83731567103  \\
            0.01198153157010539  2599.0573776769324  \\
            0.01207974084527019  2084.918196060023  \\
            0.012177950120434988  2377.0665555848086  \\
            0.012276159395599787  10536.38386583832  \\
            0.012374368670764583  2783.7185043250975  \\
            0.012472577945929382  1593.9565473123423  \\
            0.012570787221094181  1326.4720593592  \\
            0.01266899649625898  1219.3745188990938  \\
            0.012767205771423778  1163.6230457779243  \\
            0.012865415046588575  1130.0441288825164  \\
            0.012963624321753374  1107.9592142464091  \\
            0.013061833596918171  1092.6102958399395  \\
            0.01316004287208297  1081.5765326167618  \\
            0.01325825214724777  1073.508422748227  \\
            0.013356461422412568  1067.607189425807  \\
            0.013454670697577364  1063.385812549492  \\
            0.013552879972742163  1060.5510798362434  \\
            0.013651089247906962  1058.9440758227001  \\
            0.013749298523071761  1058.5130284005675  \\
            0.01384750779823656  1059.3082116147102  \\
            0.013945717073401356  1061.4973319713422  \\
            0.014043926348566155  1065.4074470099577  \\
            0.014142135623730952  1071.6106221021978  \\
            0.014240344898895752  1081.0929451117413  \\
            0.01433855417406055  1095.59913727472  \\
            0.014436763449225346  1118.382840056787  \\
            0.014534972724390145  1155.99741862781  \\
            0.014633181999554944  1223.131418706611  \\
            0.014731391274719743  1358.113265514299  \\
            0.014829600549884542  1687.8261821146775  \\
            0.014927809825049342  2890.7968296934882  \\
            0.015026019100214137  224284.67468681076  \\
            0.015124228375378936  2840.2105750792443  \\
            0.015222437650543735  1682.6097414433475  \\
            0.015320646925708533  1359.2969656344198  \\
            0.015418856200873332  1225.4343748006363  \\
            0.015517065476038129  1158.2005644181568  \\
            0.015615274751202926  1120.162919730524  \\
            0.015713484026367727  1096.885823337828  \\
            0.015811693301532526  1081.8998163454712  \\
            0.015909902576697322  1071.9867655936398  \\
            0.01600811185186212  1065.427129705891  \\
            0.016106321127026917  1061.2656543907958  \\
            0.016204530402191716  1058.9778835212264  \\
            0.016302739677356515  1058.3144418405104  \\
            0.016400948952521314  1059.2365204120222  \\
            0.016499158227686113  1061.9115319907305  \\
            0.01659736750285091  1066.768739771317  \\
            0.016695576778015708  1074.6464537854172  \\
            0.016793786053180507  1087.1229571567585  \\
            0.016891995328345306  1107.2809715576861  \\
            0.016990204603510105  1141.643622629304  \\
            0.0170884138786749  1205.8131801800173  \\
            0.0171866231538397  1344.6164378630335  \\
            0.0172848324290045  1732.5164292090028  \\
            0.017383041704169298  3759.6594763851476  \\
            0.017481250979334097  6000.826106400968  \\
            0.017579460254498892  1907.943424306588  \\
            0.01767766952966369  1380.943184569514  \\
            0.01777587880482849  1211.9969226056091  \\
            0.01787408807999329  1138.7708394382694  \\
            0.01797229735515809  1101.3056649794842  \\
            0.018070506630322888  1080.1022553180835  \\
            0.018168715905487683  1067.4029194993893  \\
            0.018266925180652482  1059.7031102522546  \\
            0.01836513445581728  1055.3199139385313  \\
            0.018463343730982077  1053.5037702119878  \\
            0.018561553006146876  1054.157803877143  \\
            0.018659762281311672  1058.0003441862075  \\
            0.01875797155647647  1067.8880920370768  \\
            0.01885618083164127  1106.596571895197  \\
            0.01895439010680607  1113.3029281824038  \\
            0.019052599381970868  1294.6136846946245  \\
            0.019150808657135664  25042.78878735999  \\
            0.019249017932300463  1323.39680660813  \\
            0.019347227207465262  1134.882063830107  \\
            0.01944543648263006  1112.6153464451056  \\
            0.01954364575779486  1127.7040470088157  \\
            0.01964185503295966  1176.4245875283757  \\
            0.019740064308124455  1297.3926968200642  \\
            0.019838273583289254  1672.5312806445115  \\
            0.019936482858454053  4148.910324800596  \\
            0.020034692133618852  3950.857886480887  \\
            0.02013290140878365  1645.017373946492  \\
            0.020231110683948447  1281.4366365116405  \\
            0.020329319959113246  1160.7580787618213  \\
            0.020427529234278045  1107.2252907417517  \\
            0.020525738509442844  1079.1655040124  \\
            0.020623947784607643  1062.7765757317318  \\
            0.02072215705977244  1052.4662864484155  \\
            0.020820366334937238  1045.63340862037  \\
            0.020918575610102033  1040.9398117239164  \\
            0.021016784885266832  1037.6427388443474  \\
            0.02111499416043163  1035.3052414352285  \\
            0.02121320343559643  1045.613809299608  \\
            0.021311412710761226  1035.4881767315078  \\
            0.021409621985926025  1033.5774736505136  \\
            0.021507831261090824  1032.7753561153518  \\
            0.021606040536255623  1032.5202563938594  \\
            0.021704249811420422  1032.671692752209  \\
            0.021802459086585218  1033.1985108102594  \\
            0.021900668361750017  1034.130511086242  \\
            0.021998877636914816  1035.5630860853003  \\
            0.022097086912079615  1037.6979202912687  \\
            0.022195296187244414  1040.9528106036641  \\
            0.022293505462409213  1046.264701164483  \\
            0.02239171473757401  1056.0682725218535  \\
            0.022489924012738808  1078.3821111903515  \\
            0.022588133287903607  1153.2825359869241  \\
            0.022686342563068406  1826.834861611501  \\
            0.022784551838233205  1931.6648512230763  \\
            0.022882761113398  1176.955489610351  \\
            0.0229809703885628  1099.3276123025146  \\
            0.0230791796637276  1079.885800908107  \\
            0.023177388938892398  1075.0243963923094  \\
            0.023275598214057194  1076.4117132051176  \\
            0.023373807489221993  1082.6367306279378  \\
            0.02347201676438679  1118.1587913270957  \\
            0.023570226039551587  1095.3744076832795  \\
            0.023668435314716386  1109.3693640160986  \\
            0.023766644589881186  1128.9557897412355  \\
            0.023864853865045985  1157.0171619613386  \\
            0.02396306314021078  1198.827672717303  \\
            0.02406127241537558  1264.7000426293632  \\
            0.02415948169054038  1376.8916710191966  \\
            0.024257690965705177  1590.885573851501  \\
            0.024355900240869977  2081.415648682662  \\
            0.024454109516034772  3745.0218819162133  \\
            0.024552318791199575  634320.1969215311  \\
            0.02465052806636437  3648.017203531113  \\
            0.024748737341529166  2021.0199267364944  \\
            0.02484694661669397  1544.0753222596977  \\
            0.024945155891858764  1339.7445277181678  \\
            0.025043365167023567  1235.3227145166786  \\
            0.025141574442188362  1175.954856577954  \\
            0.025239783717353158  1139.753416893255  \\
            0.02533799299251796  1116.6630448305493  \\
            0.025436202267682756  1101.5885575930308  \\
            0.025534411542847555  1091.7593767558087  \\
            0.025632620818012354  1085.5949254037969  \\
            0.02573083009317715  1082.1728169769206  \\
            0.02582903936834195  1080.9642049532545  \\
            0.025927248643506748  1081.6983798340923  \\
            0.026025457918671547  1084.2961988550826  \\
            0.026123667193836343  1088.845608825527  \\
            0.026221876469001145  1095.6093487242408  \\
            0.02632008574416594  1105.065869020462  \\
            0.026418295019330736  1117.9955213253395  \\
            0.02651650429449554  1135.6409149288188  \\
            0.026614713569660334  1160.0030652573937  \\
            0.026712922844825137  1194.4070919020457  \\
            0.026811132119989933  1244.6460054537076  \\
            0.026909341395154728  1321.4799408914173  \\
            0.02700755067031953  1446.6983787911333  \\
            0.027105759945484326  1670.1656245614545  \\
            0.02720396922064913  2130.1020246442513  \\
            0.027302178495813925  3376.415751729258  \\
            0.02740038777097872  11967.450988386188  \\
            0.027498597046143523  6930.636946329745  \\
            0.027596806321308318  2830.289932962407  \\
            0.02769501559647312  1919.2798056824297  \\
            0.027793224871637916  1551.8423159576553  \\
            0.027891434146802712  1366.9667572464748  \\
            0.027989643421967515  1261.7096431906593  \\
            0.02808785269713231  1196.6411079907987  \\
            0.02818606197229711  1153.9377850569754  \\
            0.028284271247461905  1124.6090121970967  \\
            0.028382480522626704  1103.7410460649437  \\
            0.028480689797791503  1088.4760782810774  \\
            0.0285788990729563  1077.0668938735046  \\
            0.0286771083481211  1068.4005979732567  \\
            0.028775317623285897  1061.7438571142447  \\
            0.028873526898450692  1056.599696332886  \\
            0.028971736173615495  1052.6235696892397  \\
            0.02906994544878029  1049.5724020556383  \\
            0.029168154723945093  1047.2727253550336  \\
            0.02926636399910989  1045.6002875029533  \\
            0.02936457327427469  1044.4668099025926  \\
            0.029462782549439487  1043.8113859483035  \\
            0.029560991824604282  1043.5950761636184  \\
            0.029659201099769085  1043.7979340517722  \\
            0.02975741037493388  1044.418247912725  \\
            0.029855619650098683  1045.4744362922847  \\
            0.02995382892526348  1047.0112415478823  \\
            0.030052038200428274  1049.1148684966715  \\
            0.030150247475593077  1051.9507869931724  \\
            0.030248456750757872  1055.8708724935977  \\
            0.03034666602592267  1061.7880463468907  \\
            0.03044487530108747  1073.0071926744904  \\
            0.030543084576252266  1112.1596978286934  \\
            0.030641293851417065  3345.8140550060525  \\
            0.030739503126581864  1151.567559762515  \\
            0.030837712401746663  1103.963336690719  \\
            0.03093592167691146  1115.1192986914161  \\
            0.031034130952076258  1109.651296408389  \\
            0.031132340227241057  1123.6778708610545  \\
            0.031230549502405853  1142.6132857857838  \\
            0.03132875877757066  1167.7023170196949  \\
            0.031426968052735454  1201.2375930799726  \\
            0.03152517732790025  1246.994814124334  \\
            0.03162338660306505  1311.2678360114794  \\
            0.03172159587822985  1405.0427631014932  \\
            0.031819805153394644  1548.8565473968206  \\
            0.031918014428559446  1785.0260116813658  \\
            0.03201622370372424  2214.555458744257  \\
            0.03211443297888904  3146.835779738173  \\
            0.03221264225405383  6196.814646005837  \\
            0.032310851529218636  2.2851555852168505e6  \\
            0.03240906080438343  5761.321550735716  \\
            0.032507270079548234  3052.0577334409013  \\
            0.03260547935471303  2179.977627020841  \\
            0.032703688629877825  1770.2838116503901  \\
            0.03280189790504263  1542.3699525802454  \\
            0.03290010718020742  1402.331170942741  \\
            0.032998316455372226  1310.279911238735  \\
            0.03309652573053702  1246.6760504640156  \\
            0.03319473500570182  1200.9833962804264  \\
            0.03329294428086662  1167.1090192598142  \\
            0.033391153556031415  1141.3337469063754  \\
            0.03348936283119622  1121.2863814870977  \\
            0.03358757210636101  1105.3996027657583  \\
            0.03368578138152581  1092.6053060442925  \\
            0.03378399065669061  1082.156219026974  \\
            0.03388219993185541  1073.5174413408288  \\
            0.03398040920702021  1066.2983234773506  \\
            0.034078618482185005  1060.2084531791704  \\
            0.0341768277573498  1055.0284988625779  \\
            0.0342750370325146  1050.5904611711703  \\
            0.0343732463076794  1046.7640280711587  \\
            0.0344714555828442  1043.446976791231  \\
            0.034569664858009  1040.5583122785495  \\
            0.03466787413317379  1038.033289662996  \\
            0.034766083408338595  1035.8197553623907  \\
            0.03486429268350339  1033.8754253294544  \\
            0.03496250195866819  1032.1658388527758  \\
            0.03506071123383299  1030.6628060197004  \\
            0.035158920508997785  1029.3432206348004  \\
            0.03525712978416259  1028.1881472222553  \\
            0.03535533905932738  1027.1821163790073  \\
            0.035453548334492185  1026.312580761622  \\
            0.03555175760965698  1025.5694970079176  \\
            0.035649966884821783  1024.945008297545  \\
            0.03574817615998658  1024.4332093235125  \\
            0.035846385435151375  1024.0299808343432  \\
            0.03594459471031618  1023.7328851860879  \\
            0.03604280398548097  1023.5411179272419  \\
            0.036141013260645775  1023.455513685232  \\
            0.03623922253581057  1023.4786076964315  \\
            0.03633743181097537  1023.614757651685  \\
            0.03643564108614017  1023.8703343183893  \\
            0.036533850361304965  1024.2539940335828  \\
            0.03663205963646976  1024.777052172772  \\
            0.03673026891163456  1025.453984759757  \\
            0.03682847818679936  1026.3030964936527  \\
            0.036926687461964154  1027.3474091866499  \\
            0.03702489673712895  1028.6158471965934  \\
            0.03712310601229375  1030.1448295606958  \\
            0.03722131528745855  1031.9804279235393  \\
            0.037319524562623343  1034.181324281031  \\
            0.037417733837788146  1036.822918399902  \\
            0.03751594311295294  1040.0031176815755  \\
            0.037614152388117744  1043.850637005864  \\
            0.03771236166328254  1048.5371231981285  \\
            0.037810570938447335  1054.2952447789419  \\
            0.03790878021361214  1061.4463302769714  \\
            0.038006989488776934  1070.4437420440115  \\
            0.038105198763941736  1081.9430485248952  \\
            0.03820340803910653  1096.9195831065133  \\
            0.03830161731427133  1116.8735123903923  \\
            0.03839982658943613  1144.2049493253833  \\
            0.038498035864600925  1182.940228853176  \\
            0.03859624513976573  1240.2396589830298  \\
            0.038694454414930524  1329.8182547593763  \\
            0.038792663690095326  1480.6896986427898  \\
            0.03889087296526012  1763.748896383869  \\
            0.03898908224042492  2399.063903578591  \\
            0.03908729151558972  4523.630241902142  \\
            0.039185500790754516  137266.73679927067  \\
            0.03928371006591932  4225.151568541164  \\
            0.039381919341084114  2302.4822364854394  \\
            0.03948012861624891  1706.3161376152284  \\
            0.03957833789141371  1438.3133606891108  \\
            0.03967654716657851  1295.2987445250437  \\
            0.03977475644174331  1210.5552537258661  \\
            0.039872965716908106  1156.6544832814877  \\
            0.0399711749920729  1121.363680334358  \\
            0.040069384267237704  1118.7630239608247  \\
            0.0401675935424025  1071.8571174518452  \\
            0.0402658028175673  1057.8725287982695  \\
            0.0403640120927321  1047.1391321889296  \\
            0.04046222136789689  1038.7309325353344  \\
            0.040560430643061696  1032.0273963833663  \\
            0.04065863991822649  1026.6006910792564  \\
            0.040756849193391294  1022.1484612734454  \\
            0.04085505846855609  1018.4523453405659  \\
            0.040953267743720885  1015.3515907716496  \\
            0.04105147701888569  1012.7258194081251  \\
            0.04114968629405048  1010.4834991842581  \\
            0.041247895569215286  1008.5540674003396  \\
            0.04134610484438008  1006.8824449906446  \\
            0.04144431411954488  1005.4251492331965  \\
            0.04154252339470968  1004.147495458932  \\
            0.041640732669874475  1003.0215534745997  \\
            0.04173894194503927  1002.0246353512978  \\
            0.041837151220204066  1001.138162763302  \\
            0.04193536049536887  1000.3468090450015  \\
            0.042033569770533664  999.6378425740381  \\
            0.04213177904569846  999.0006193741979  \\
            0.04222998832086326  998.4261874997505  \\
            0.04232819759602806  997.9069759805419  \\
            0.04242640687119286  997.4365483435306  \\
            0.042524616146357656  997.009405860491  \\
            0.04262282542152245  996.6208294088487  \\
            0.042721034696687255  996.2667515462156  \\
            0.04281924397185205  995.9436523848128  \\
            0.04291745324701685  995.6484743602552  \\
            0.04301566252218165  995.3785520755981  \\
            0.043113871797346444  995.1315542639667  \\
            0.043212081072511246  994.9054355380183  \\
            0.04331029034767604  994.698396090647  \\
            0.043408499622840845  994.5088478906745  \\
            0.04350670889800564  994.3353862059862  \\
            0.043604918173170436  994.1767655213084  \\
            0.04370312744833524  994.0318790937004  \\
            0.043801336723500034  993.8997415364331  \\
            0.043899545998664836  993.7794739309369  \\
            0.04399775527382963  993.6702910576478  \\
            0.04409596454899443  993.5714904142214  \\
            0.04419417382415923  993.4824427369202  \\
            0.044292383099324026  993.4025838057535  \\
            0.04439059237448883  993.3314073329824  \\
            0.044488801649653624  993.2684587818401  \\
            0.04458701092481843  993.2133299802724  \\
            0.04468522019998322  993.1656544153368  \\
            0.04478342947514802  993.1251031190457  \\
            0.04488163875031282  993.091381061252  \\
            0.044979848025477616  993.0642239823854  \\
            0.04507805730064242  993.0433956103531  \\
            0.045176266575807214  993.028685212844  \\
            0.04527447585097201  993.0199054389574  \\
            0.04537268512613681  993.0168904207563  \\
            0.04547089440130161  993.0194940974999  \\
            0.04556910367646641  993.0275887399724  \\
            0.045667312951631206  993.0410636547349  \\
            0.045765522226796  993.0598240370542  \\
            0.045863731501960804  993.0837899727784  \\
            0.0459619407771256  993.1128955597449  \\
            0.0460601500522904  993.1470881443176  \\
            0.0461583593274552  993.1863276588003  \\
            0.046256568602619993  993.2305860508486  \\
            0.046354777877784796  993.2798467995309  \\
            0.04645298715294959  993.3341045044832  \\
            0.04655119642811439  993.3933645535734  \\
            0.04664940570327918  993.4576428466557  \\
            0.046747614978443985  993.5269655926838  \\
            0.04684582425360878  993.6013691585314  \\
            0.04694403352877358  993.6808999751048  \\
            0.04704224280393838  993.7656144984479  \\
            0.047140452079103175  993.8555792203932  \\
            0.04723866135426797  993.9508707321611  \\
            0.04733687062943277  994.0515758363142  \\
            0.04743507990459757  994.157791709794  \\
            0.04753328917976237  994.2696261133785  \\
            0.04763149845492717  994.3871976559922  \\
            0.04772970773009197  994.5106361063606  \\
            0.047827917005256765  994.6400827596465  \\
            0.04792612628042156  994.7756908581681  \\
            0.04802433555558636  994.9176260688743  \\
            0.04812254483075116  995.0660670197982  \\
            0.04822075410591596  995.2212059022061  \\
            0.04831896338108076  995.3832491361715  \\
            0.04841717265624555  995.5524181097094  \\
            0.048515381931410355  995.7289499950728  \\
            0.04861359120657515  995.9130986435083  \\
            0.04871180048173995  996.1051355740839  \\
            0.04881000975690475  996.305351052293  \\
            0.048908219032069544  996.5140552772525  \\
            0.04900642830723435  996.7315796812317  \\
            0.04910463758239915  996.9582783513665  \\
            0.049202846857563945  997.1945295876185  \\
            0.04930105613272874  997.4407376098242  \\
            0.049399265407893536  997.697334424877  \\
            0.04949747468305833  997.9647818760309  \\
            0.04959568395822314  998.2435738881819  \\
            0.04969389323338794  998.5342389318535  \\
            0.04979210250855273  998.8373427268758  \\
            0.04989031178371753  999.1534912175266  \\
            0.049988521058882324  999.4833338411876  \\
            0.05008673033404713  999.8275671287423  \\
            0.05018493960921193  1000.1869386746981  \\
            0.050283148884376724  1000.5622515164519  \\
            0.05038135815954152  1000.9543689735805  \\
            0.050479567434706316  1001.3642200036137  \\
            0.050577776709871125  1001.7928051328141  \\
            0.05067598598503592  1002.2412030379355  \\
            0.050774195260200716  1002.7105778584211  \\
            0.05087240453536551  1003.2021873331155  \\
            0.05097061381053031  1003.717391869824  \\
            0.05106882308569511  1004.2576646706526  \\
            0.05116703236085991  1004.8246030538223  \\
            0.05126524163602471  1005.4199411415688  \\
            0.051363450911189504  1006.0455640972717  \\
            0.0514616601863543  1006.7035241408468  \\
            0.0515598694615191  1007.3960585886474  \\
            0.0516580787366839  1008.1256102310459  \\
            0.05175628801184869  1008.8948503837348  \\
            0.051854497287013496  1009.7067050345113  \\
            0.05195270656217829  1010.564384561436  \\
            0.052050915837343094  1011.471417594209  \\
            0.05214912511250789  1012.4316896903565  \\
            0.052247334387672685  1013.4494876304987  \\
            0.05234554366283748  1014.5295502722537  \\
            0.05244375293800229  1015.6771271124608  \\
            0.052541962213167086  1016.8980459076607  \\
            0.05264017148833188  1018.1987909990662  \\
            0.05273838076349668  1019.5865943175271  \\
            0.05283659003866147  1021.0695414673829  \\
            0.05293479931382628  1022.6566958146078  \\
            0.05303300858899108  1024.3582441510985  \\
            0.05313121786415587  1026.185668327445  \\
            0.05322942713932067  1028.1519482746617  \\
            0.053327636414485464  1030.2718031436973  \\
            0.053425845689650274  1032.5619789469604  \\
            0.05352405496481507  1035.041593224989  \\
            0.053622264239979865  1037.7325500005682  \\
            0.05372047351514466  1040.6600418579999  \\
            0.053818682790309456  1043.8531606459192  \\
            0.053916892065474266  1047.3456444270178  \\
            0.05401510134063906  1051.1767964657572  \\
            0.05411331061580386  1055.3926229251174  \\
            0.05421151989096865  1060.0472506332137  \\
            0.05430972916613345  1065.2047063081434  \\
            0.05440793844129826  1070.9411661275408  \\
            0.05450614771646305  1077.3478227579512  \\
            0.05460435699162785  1084.5345706493767  \\
            0.054702566266792645  1092.6347867469194  \\
            0.05480077554195744  1101.811593729767  \\
            0.05489898481712225  1112.2661535455377  \\
            0.054997194092287045  1124.2487772169213  \\
            0.05509540336745184  1138.0739963055657  \\
            0.055193612642616637  1154.1412934887903  \\
            0.05529182191778143  1172.9640555097358  \\
            0.05539003119294624  1195.2107005706152  \\
            0.05548824046811104  1221.7642184061087  \\
            0.05558644974327583  1253.8102349973924  \\
            0.05568465901844063  1292.9705002194123  \\
            0.055782868293605424  1341.5110506059182  \\
            0.05588107756877023  1402.6778158201485  \\
            0.05597928684393503  1481.2596191513983  \\
            0.056077496119099825  1584.5792895904679  \\
            0.05617570539426462  1724.3456774939991  \\
            0.056273914669429416  1920.3858235440453  \\
            0.05637212394459422  2208.943125184744  \\
            0.056470333219759014  2663.7558481383667  \\
            0.05656854249492381  3460.9127430067037  \\
            0.05666675177008861  5150.086188917671  \\
            0.05676496104525341  10765.58161945802  \\
            0.05686317032041821  81667.86142015102  \\
            0.056961379595583006  8476.781318619796  \\
            0.0570595888707478  4526.878663617011  \\
            0.0571577981459126  3144.064079681527  \\
            0.0572560074210774  2455.5415200936636  \\
            0.0573542166962422  2052.560099258032  \\
            0.057452425971407  1793.5768043061573  \\
            0.057550635246571794  1616.6010906230692  \\
            0.05764884452173659  1490.2502282485746  \\
            0.057747053796901385  1396.9963396751161  \\
            0.057845263072066194  1326.3325275689244  \\
            0.05794347234723099  1271.6166185127927  \\
            0.058041681622395785  1228.4718676069606  \\
            0.05813989089756058  1193.9164903258174  \\
            0.05823810017272538  1165.8623056964916  \\
            0.058336309447890186  1142.8124829464725  \\
            0.05843451872305498  1123.6723489245035  \\
            0.05853272799821978  1107.627182760648  \\
            0.05863093727338457  1094.0611384034762  \\
            0.05872914654854938  1082.5021945732046  \\
            0.05882735582371418  1072.5840112707506  \\
            0.058925565098878974  1064.019021237182  \\
            0.05902377437404377  1056.579138999661  \\
            0.059121983649208565  1050.0817283670815  \\
            0.059220192924373374  1044.3792589553761  \\
            0.05931840219953817  1039.3515889426226  \\
            0.059416611474702966  1034.9001426323239  \\
            0.05951482074986776  1030.943472058476  \\
            0.05961303002503256  1027.4138410846392  \\
            0.059711239300197366  1024.2545728960456  \\
            0.05980944857536216  1021.4179730615764  \\
            0.05990765785052696  1018.8636905440774  \\
            0.06000586712569175  1016.5574148148029  \\
            0.06010407640085655  1014.4698329929412  \\
            0.06020228567602136  1012.5757896873229  \\
            0.060300494951186154  1010.8536059787981  \\
            0.06039870422635095  1009.2845242010512  \\
            0.060496913501515745  1007.8522527875463  \\
            0.06059512277668054  1006.5425912054527  \\
            0.06069333205184534  1005.3431193593063  \\
            0.060791541327010146  1004.2429391665103  \\
            0.06088975060217494  1003.2324585819819  \\
            0.06098795987733974  1002.3032103267487  \\
            0.06108616915250453  1001.4476991261852  \\
            0.061184378427669335  1000.6592724719666  \\
            0.06128258770283413  999.9320108819151  \\
            0.061380796977998926  1374.258877890558  \\
            0.06147900625316373  1129.2827304029795  \\
            0.061577215528328524  1075.164752955544  \\
            0.06167542480349333  1051.995141609578  \\
            0.06177363407865812  1039.5585004070153  \\
            0.06187184335382292  1032.2230934605568  \\
            0.061970052628987714  1027.8104070607812  \\
            0.062068261904152516  1025.3584684567656  \\
            0.06216647117931732  1024.4272548640495  \\
            0.062264680454482114  1024.8763205515304  \\
            0.06236288972964691  1026.8041050722957  \\
            0.062461099004811706  1030.5757894798037  \\
            0.06255930827997651  1036.9799286224197  \\
            0.06265751755514132  1047.6339564617792  \\
            0.06275572683030611  1066.0465647381698  \\
            0.06285393610547091  1100.8488892363835  \\
            0.0629521453806357  1178.0756836584837  \\
            0.0630503546558005  1408.782177028701  \\
            0.0631485639309653  2859.1986637454634  \\
            0.0632467732061301  3370.514214738126  \\
            0.0633449824812949  1369.8234724972301  \\
            0.0634431917564597  1124.8234578736658  \\
            0.06354140103162449  1053.6591524702585  \\
            0.06363961030678929  1024.836952625038  \\
            0.06373781958195408  1010.7509622392198  \\
            0.06383602885711889  1002.9828250828654  \\
            0.06393423813228369  998.323963298623  \\
            0.06403244740744848  995.3532718219537  \\
            0.06413065668261328  993.3691991881723  \\
            0.06422886595777808  991.9957582545358  \\
            0.06432707523294287  991.0180475158924  \\
            0.06442528450810767  990.3066861401753  \\
            0.06452349378327248  989.7803738973793  \\
            0.06462170305843727  989.3861528643051  \\
            0.06471991233360207  989.0884628018271  \\
            0.06481812160876686  988.8627819841066  \\
            0.06491633088393166  988.691810349854  \\
            0.06501454015909647  988.5630918261393  \\
            0.06511274943426126  988.467496500246  \\
            0.06521095870942606  988.3982237532301  \\
            0.06530916798459085  988.3501349113063  \\
            0.06540737725975565  988.3192961362471  \\
            0.06550558653492046  988.3026604608201  \\
            0.06560379581008526  988.2978422598444  \\
            0.06570200508525005  988.3029555882367  \\
            0.06580021436041485  988.3164961060862  \\
            0.06589842363557964  988.3372543206508  \\
            0.06599663291074445  988.3642503673374  \\
            0.06609484218590925  988.3966852998497  \\
            0.06619305146107404  988.4339037077685  \\
            0.06629126073623884  988.4753650482535  \\
            0.06638947001140363  988.5206214638837  \\
            0.06648767928656844  988.5693003717543  \\
            0.06658588856173324  988.6210908642523  \\
            0.06668409783689803  988.6757328298652  \\
            0.06678230711206283  988.733008357442  \\
            0.06688051638722763  988.7927348017714  \\
            0.06697872566239244  988.8547592124985  \\
            0.06707693493755723  988.9189538480542  \\
            0.06717514421272203  988.9852124985738  \\
            0.06727335348788682  989.0534475417032  \\
            0.06737156276305162  989.1235874988058  \\
            0.06746977203821643  989.19557509577  \\
            0.06756798131338122  989.2693656643742  \\
            0.06766619058854602  989.344925856998  \\
            0.06776439986371081  989.4222326677682  \\
            0.06786260913887561  989.5012726048952  \\
            0.06796081841404042  989.5820411338674  \\
            0.06805902768920521  989.664542256269  \\
            0.06815723696437001  989.7487882578477  \\
            0.0682554462395348  989.834799637774  \\
            0.0683536555146996  989.9226051181299  \\
            0.06845186478986441  990.0122418918783  \\
            0.0685500740650292  990.1037559352781  \\
            0.068648283340194  990.1972025798576  \\
            0.0687464926153588  990.2926471350139  \\
            0.0688447018905236  990.3901658731003  \\
            0.0689429111656884  990.4898470431475  \\
            0.0690411204408532  990.5917924009024  \\
            0.069139329716018  990.6961187794634  \\
            0.06923753899118279  990.802960276577  \\
            0.06933574826634759  990.9124707114704  \\
            0.0694339575415124  991.0248268144236  \\
            0.06953216681667719  991.1402318925517  \\
            0.06963037609184199  991.2589205221105  \\
            0.06972858536700678  991.3811642399066  \\
            0.06982679464217158  991.5072786351928  \\
            0.06992500391733639  991.6376318625117  \\
            0.07002321319250118  991.7726557385523  \\
            0.07012142246766598  991.9128595302338  \\
            0.07021963174283077  992.0588472435937  \\
            0.07031784101799557  992.2113396063127  \\
            0.07041605029316038  992.3712034176137  \\
            0.07051425956832517  992.5394877088786  \\
            0.07061246884348997  992.7174739997018  \\
            0.07071067811865477  992.9067396280661  \\
            0.07080888739381958  993.109245921756  \\
            0.07090709666898437  993.3274567697802  \\
            0.07100530594414917  993.5645014362298  \\
            0.07110351521931396  993.8244037316429  \\
            0.07120172449447876  994.1124100353564  \\
            0.07129993376964357  994.435462945495  \\
            0.07139814304480836  994.8028994148393  \\
            0.07149635231997316  995.2275317605656  \\
            0.07159456159513795  995.7273044568477  \\
            0.07169277087030275  996.327994999661  \\
            0.07179098014546756  997.0677273699433  \\
            0.07188918942063235  998.0048508052931  \\
            0.07198739869579715  999.2325488624648  \\
            0.07208560797096195  1000.9074432505666  \\
            0.07218381724612674  1003.3116883001318  \\
            0.07228202652129155  1007.0016880770091  \\
            0.07238023579645635  1013.2234578379192  \\
            0.07247844507162114  1025.4133492757944  \\
            0.07257665434678594  1057.5102996335759  \\
            0.07267486362195073  1306.676000149856  \\
            0.07277307289711553  993.7325953346298  \\
            0.07287128217228034  993.8470332428824  \\
            0.07296949144744513  993.9640581649309  \\
            0.07306770072260993  994.0837703209992  \\
            0.07316590999777473  994.2062761369027  \\
            0.07326411927293952  994.3316886841964  \\
            0.07336232854810432  994.4601281559657  \\
            0.07346053782326913  994.5917223872132  \\
            0.07355874709843392  994.7266074195659  \\
            0.07365695637359872  994.8649281196718  \\
            0.07375516564876351  995.0068388521795  \\
            0.07385337492392831  995.152504217108  \\
            0.0739515841990931  995.3020998536804  \\
            0.0740497934742579  995.4558133242173  \\
            0.07414800274942271  995.6138450807097  \\
            0.0742462120245875  995.7764095266549  \\
            0.0743444212997523  995.943736184598  \\
            0.0744426305749171  996.1160709797288  \\
            0.07454083985008189  996.2936776563338  \\
            0.07463904912524669  996.4768393388034  \\
            0.0747372584004115  996.6658602568575  \\
            0.07483546767557629  996.8610676558836  \\
            0.07493367695074109  997.0628139124066  \\
            0.07503188622590588  997.271478882464  \\
            0.07513009550107068  997.4874725119007  \\
            0.07522830477623549  997.7112377437604  \\
            0.07532651405140028  997.9432537592617  \\
            0.07542472332656508  998.1840395995351  \\
            0.07552293260172988  998.4341582189603  \\
            0.07562114187689467  998.6942210299651  \\
            0.07571935115205948  998.9648930087318  \\
            0.07581756042722428  999.2468984422375  \\
            0.07591576970238907  999.5410274104195  \\
            0.07601397897755387  999.8481431130949  \\
            0.07611218825271866  1000.16919016952  \\
            0.07621039752788347  1000.5052040407537  \\
            0.07630860680304827  1000.8573217528867  \\
            0.07640681607821306  1001.2267941305807  \\
            0.07650502535337786  1001.6149997843874  \\
            0.07660323462854265  1002.0234611545412  \\
            0.07670144390370746  1002.4538629536199  \\
            0.07679965317887226  1002.9080734337032  \\
            0.07689786245403706  1003.388168979686  \\
            0.07699607172920185  1003.896462640336  \\
            0.07709428100436666  1004.4355373276834  \\
            0.07719249027953146  1005.0082845826502  \\
            0.07729069955469625  1005.6179500000649  \\
            0.07738890882986105  1006.2681866456154  \\
            0.07748711810502584  1006.9631181161279  \\
            0.07758532738019065  1007.707413284704  \\
            0.07768353665535545  1008.5063752586448  \\
            0.07778174593052024  1009.3660477355235  \\
            0.07787995520568504  1010.2933427370921  \\
            0.07797816448084983  1011.2961947778317  \\
            0.07807637375601464  1012.383747892895  \\
            0.07817458303117944  1013.5665837624858  \\
            0.07827279230634424  1014.8570015701598  \\
            0.07837100158150903  1016.2693634108239  \\
            0.07846921085667383  1017.8205233563566  \\
            0.07856742013183864  1019.5303640961616  \\
            0.07866562940700343  1021.4224730238515  \\
            0.07876383868216823  1023.52500062898  \\
            0.07886204795733302  1025.8717594243851  \\
            0.07896025723249782  1028.5036433210664  \\
            0.07905846650766263  1031.4704784339265  \\
            0.07915667578282742  1034.833461262292  \\
            0.07925488505799222  1038.668406337186  \\
            0.07935309433315701  1043.070124007286  \\
            0.07945130360832181  1048.1583986960163  \\
            0.07954951288348662  1054.08626882908  \\
            0.07964772215865142  1061.051673144362  \\
            0.07974593143381621  1069.314112411991  \\
            0.079844140708981  1079.2189373634133  \\
            0.0799423499841458  1091.2334981555673  \\
            0.08004055925931061  1106.0022156484752  \\
            0.08013876853447541  1124.432711423135  \\
            0.0802369778096402  1147.8346055580323  \\
            0.080335187084805  1178.1510520034067  \\
            0.0804333963599698  1218.3609618509265  \\
            0.0805316056351346  1273.2125768680914  \\
            0.0806298149102994  1350.6440815895166  \\
            0.0807280241854642  1464.7539857563077  \\
            0.08082623346062899  1642.682519959409  \\
            0.08092444273579379  1943.0411684111807  \\
            0.0810226520109586  2517.5072351118147  \\
            0.08112086128612339  3914.6004019428588  \\
            0.08121907056128819  11604.974785451723  \\
            0.08131727983645298  17429.405892815514  \\
            0.08141548911161778  4459.858492590917  \\
            0.08151369838678259  2695.931588331046  \\
            0.08161190766194738  2023.4345402051817  \\
            0.08171011693711218  1684.4482993422164  \\
            0.08180832621227697  1488.1986336741836  \\
            0.08190653548744177  1364.4408480916263  \\
            0.08200474476260658  1281.5972838651242  \\
            0.08210295403777138  1223.5952345893886  \\
            0.08220116331293617  1181.5220749927846  \\
            0.08229937258810097  1150.1112199224158  \\
            0.08239758186326576  1126.0919113234668  \\
            0.08249579113843057  1107.3491234704316  \\
            0.08259400041359537  1092.468246716808  \\
            0.08269220968876016  1080.4755448347864  \\
            0.08279041896392496  1070.6838828140178  \\
            0.08288862823908975  1062.5977090772394  \\
            0.08298683751425455  1055.852707603779  \\
            0.08308504678941936  1050.1764245611912  \\
            0.08318325606458415  1045.361960321832  \\
            0.08328146533974895  1041.2500134580298  \\
            0.08337967461491375  1037.716388506039  \\
            0.08347788389007854  1034.6631532060974  \\
            0.08357609316524334  1032.0122797216013  \\
            0.08367430244040813  1029.7010058297503  \\
            0.08377251171557294  1027.6784058954204  \\
            0.08387072099073774  1025.9028251489449  \\
            0.08396893026590253  1024.3399382458083  \\
            0.08406713954106733  1117.5490074686068  \\
            0.08416534881623212  1039.7210170418969  \\
            0.08426355809139692  1030.8846170942957  \\
            0.08436176736656173  1026.990404691928  \\
            0.08445997664172653  1024.6000457586251  \\
            0.08455818591689132  1022.8996329198222  \\
            0.08465639519205612  1021.5924562505627  \\
            0.08475460446722091  1020.5417784403161  \\
            0.08485281374238572  1019.6742806370812  \\
            0.08495102301755052  1018.9461350575834  \\
            0.08504923229271531  1018.328926311162  \\
            0.08514744156788011  1017.8030510824304  \\
            0.0852456508430449  1017.3543098657017  \\
            0.08534386011820971  1016.9720156579197  \\
            0.08544206939337451  1016.6478781462488  \\
            0.0855402786685393  1016.3753088681306  \\
            0.0856384879437041  1016.148970819578  \\
            0.0857366972188689  1015.9644737082677  \\
            0.0858349064940337  1015.8181620663776  \\
            0.0859331157691985  1015.7069629897829  \\
            0.0860313250443633  1015.6282739963251  \\
            0.08612953431952809  1015.579878914214  \\
            0.08622774359469289  1015.5598830907759  \\
            0.0863259528698577  1015.5666633230649  \\
            0.08642416214502249  1015.5988278106795  \\
            0.08652237142018729  1015.6551845751009  \\
            0.08662058069535208  1015.7347154271212  \\
            0.08671878997051688  1015.8365551871005  \\
            0.08681699924568169  1015.9599743584687  \\
            0.08691520852084648  1016.1043648119668  \\
            0.08701341779601128  1016.269228067454  \\
            0.08711162707117608  1016.4541653698001  \\
            0.08720983634634087  1016.6588696181328  \\
            0.08730804562150568  1016.8831185272716  \\
            0.08740625489667048  1017.1267690496932  \\
            0.08750446417183527  1017.3897528493908  \\
            0.08760267344700007  1017.6720725517565  \\
            0.08770088272216486  1017.9737988625818  \\
            0.08779909199732967  1018.2950683473875  \\
            0.08789730127249447  1018.6360818175535  \\
            0.08799551054765926  1018.9971033255015  \\
            0.08809371982282406  1019.3784596558291  \\
            0.08819192909798886  1019.7805403431823  \\
            0.08829013837315366  1020.2037981440529  \\
            0.08838834764831846  1020.6487500071473  \\
            0.08848655692348326  1021.1159784718022  \\
            0.08858476619864805  1021.606133542822  \\
            0.08868297547381285  1022.1199350334095  \\
            0.08878118474897766  1022.6581753756849  \\
            0.08887939402414245  1023.2217229518993  \\
            0.08897760329930725  1023.8115259232931  \\
            0.08907581257447204  1024.4286166341926  \\
            0.08917402184963685  1025.074116617389  \\
            0.08927223112480165  1025.7492422142009  \\
            0.08937044039996644  1026.4553109302642  \\
            0.08946864967513124  1027.1937485184242  \\
            0.08956685895029604  1027.9660969219917  \\
            0.08966506822546084  1028.7740231259106  \\
            0.08976327750062564  1029.6193290305725  \\
            0.08986148677579044  1030.5039624641156  \\
            0.08995969605095523  1031.4300294530285  \\
            0.09005790532612003  1032.399807904473  \\
            0.09015611460128484  1033.4157628787057  \\
            0.09025432387644963  1034.4805636359229  \\
            0.09035253315161443  1035.597102702622  \\
            0.09045074242677922  1036.7685171977546  \\
            0.09054895170194402  1037.9982127426706  \\
            0.09064716097710883  1039.2898902973918  \\
            0.09074537025227362  1040.6475763243488  \\
            0.09084357952743842  1042.0756567653466  \\
            0.09094178880260322  1043.5789153762025  \\
            0.09103999807776801  1045.16257706402  \\
            0.09113820735293282  1046.8323569859658  \\
            0.09123641662809762  1048.5945162995959  \\
            0.09133462590326241  1050.4559256020746  \\
            0.09143283517842721  1052.4241373052953  \\
            0.091531044453592  1054.507468402324  \\
            0.09162925372875681  1056.7150953811067  \\
            0.09172746300392161  1059.057163355792  \\
            0.0918256722790864  1061.5449119125942  \\
            0.0919238815542512  1064.190820661572  \\
            0.092022090829416  1067.0087781277127  \\
            0.0921203001045808  1070.0142783356898  \\
            0.0922185093797456  1073.2246504459058  \\
            0.0923167186549104  1076.6593279429803  \\
            0.09241492793007519  1080.3401653894537  \\
            0.09251313720523999  1084.2918126034153  \\
            0.09261134648040478  1088.5421585216386  \\
            0.09270955575556959  1093.122859995873  \\
            0.09280776503073439  1098.069974646032  \\
            0.09290597430589918  1103.4247219141046  \\
            0.09300418358106398  1109.234402914003  \\
            0.09310239285622877  1115.5535181481648  \\
            0.09320060213139357  1122.4451333717948  \\
            0.09329881140655837  1129.982558706618  \\
            0.09339702068172318  1138.2514259512077  \\
            0.09349522995688797  1147.3522760410383  \\
            0.09359343923205277  1157.4038051210985  \\
            0.09369164850721756  1168.5469685267942  \\
            0.09378985778238236  1180.9502123735426  \\
            0.09388806705754715  1194.8162022277434  \\
            0.09398627633271196  1210.390560840656  \\
            0.09408448560787676  1227.9733337231564  \\
            0.09418269488304155  1247.9342060026188  \\
            0.09428090415820635  1270.7329507185975  \\
            0.09437911343337115  1296.9472863102146  \\
            0.09447732270853594  1327.3114087786207  \\
            0.09457553198370075  1362.770200706874  \\
            0.09467374125886555  1404.556965310987  \\
            0.09477195053403034  1454.3073357545443  \\
            0.09487015980919514  1514.230391495854  \\
            0.09496836908435993  1587.3732085257564  \\
            0.09506657835952474  1678.0438662476429  \\
            0.09516478763468954  1792.515376973318  \\
            0.09526299690985433  1940.254770129027  \\
            0.09536120618501913  2136.199163687192  \\
            0.09545941546018394  2405.2926163256125  \\
            0.09555762473534873  2792.4279640784175  \\
            0.09565583401051353  3387.1808469329953  \\
            0.09575404328567833  4397.516959897335  \\
            0.09585225256084312  6441.426212028898  \\
            0.09595046183600793  12538.950397094579  \\
            0.09604867111117273  441798.11601570615  \\
            0.09614688038633752  13229.345798197306  \\
            0.09624508966150232  6578.276523426708  \\
            0.09634329893666711  4440.2098525724405  \\
            0.09644150821183192  3406.8100938986963  \\
            0.09653971748699672  2812.498953837029  \\
            0.09663792676216151  2437.9414536722534  \\
            0.09673613603732631  2190.2715822489895  \\
            0.0968343453124911  2024.1645351137445  \\
            0.09693255458765591  1915.925063811711  \\
            0.09703076386282071  1853.5658532164546  \\
            0.0971289731379855  1833.063747377885  \\
            0.0972271824131503  1858.0361675201098  \\
            0.0973253916883151  1943.0805627820741  \\
            0.0974236009634799  2124.837680083  \\
            0.0975218102386447  2497.5366266944156  \\
            0.0976200195138095  3358.5845908195265  \\
            0.09771822878897429  6248.174880485139  \\
            0.09781643806413909  326323.0285999885  \\
            0.0979146473393039  6119.613457004757  \\
            0.0980128566144687  3568.2989373246223  \\
            0.09811106588963349  3098.7944032218247  \\
        }
        ;
\end{axis}
\end{tikzpicture}

        \caption{Evolution of error number}
        \label{subfig:error}
    \end{subfigure}
    \caption{Comparison of translation test results with and without ghost penalty.}
    \label{fig:combined}
\end{figure}




\begin{figure}[h!]
    \centering
    \begin{subfigure}{0.49\textwidth}
        \centering
        
% Recommended preamble:
% \usetikzlibrary{arrows.meta}
% \usetikzlibrary{backgrounds}
% \usepgfplotslibrary{patchplots}
% \usepgfplotslibrary{fillbetween}
% \pgfplotsset{%
%     layers/standard/.define layer set={%
%         background,axis background,axis grid,axis ticks,axis lines,axis tick labels,pre main,main,axis descriptions,axis foreground%
%     }{
%         grid style={/pgfplots/on layer=axis grid},%
%         tick style={/pgfplots/on layer=axis ticks},%
%         axis line style={/pgfplots/on layer=axis lines},%
%         label style={/pgfplots/on layer=axis descriptions},%
%         legend style={/pgfplots/on layer=axis descriptions},%
%         title style={/pgfplots/on layer=axis descriptions},%
%         colorbar style={/pgfplots/on layer=axis descriptions},%
%         ticklabel style={/pgfplots/on layer=axis tick labels},%
%         axis background@ style={/pgfplots/on layer=axis background},%
%         3d box foreground style={/pgfplots/on layer=axis foreground},%
%     },
% }

\begin{tikzpicture}[/tikz/background rectangle/.style={fill={rgb,1:red,1.0;green,1.0;blue,1.0}, fill opacity={1.0}, draw opacity={1.0}}, show background rectangle]
\begin{axis}[point meta max={nan}, point meta min={nan}, legend cell align={left}, legend columns={1}, title={}, title style={at={{(0.5,1)}}, anchor={south}, font={{\fontsize{14 pt}{18.2 pt}\selectfont}}, color={rgb,1:red,0.0;green,0.0;blue,0.0}, draw opacity={1.0}, rotate={0.0}, align={center}}, legend style={color={rgb,1:red,0.0;green,0.0;blue,0.0}, draw opacity={1.0}, line width={1}, solid, fill={rgb,1:red,1.0;green,1.0;blue,1.0}, fill opacity={1.0}, text opacity={1.0}, font={{\fontsize{8 pt}{10.4 pt}\selectfont}}, text={rgb,1:red,0.0;green,0.0;blue,0.0}, cells={anchor={center}}, at={(1.02, 1)}, anchor={north west}}, axis background/.style={fill={rgb,1:red,1.0;green,1.0;blue,1.0}, opacity={1.0}}, anchor={north west}, xshift={1.0mm}, yshift={-1.0mm}, width={145.4mm}, height={99.6mm}, scaled x ticks={false}, xlabel={$\delta$}, x tick style={color={rgb,1:red,0.0;green,0.0;blue,0.0}, opacity={1.0}}, x tick label style={color={rgb,1:red,0.0;green,0.0;blue,0.0}, opacity={1.0}, rotate={0}}, xlabel style={at={(ticklabel cs:0.5)}, anchor=near ticklabel, at={{(ticklabel cs:0.5)}}, anchor={near ticklabel}, font={{\fontsize{11 pt}{14.3 pt}\selectfont}}, color={rgb,1:red,0.0;green,0.0;blue,0.0}, draw opacity={1.0}, rotate={0.0}}, xmajorgrids={true}, xmin={-0.001471665988344504}, xmax={0.05052719893316125}, xticklabels={{$0.00$,$0.01$,$0.02$,$0.03$,$0.04$,$0.05$}}, xtick={{0.0,0.010000000000000002,0.020000000000000004,0.030000000000000006,0.04000000000000001,0.05000000000000001}}, xtick align={inside}, xticklabel style={font={{\fontsize{8 pt}{10.4 pt}\selectfont}}, color={rgb,1:red,0.0;green,0.0;blue,0.0}, draw opacity={1.0}, rotate={0.0}}, x grid style={color={rgb,1:red,0.0;green,0.0;blue,0.0}, draw opacity={0.1}, line width={0.5}, solid}, axis x line*={left}, x axis line style={color={rgb,1:red,0.0;green,0.0;blue,0.0}, draw opacity={1.0}, line width={1}, solid}, scaled y ticks={false}, ylabel={$\kappa(A)$}, y tick style={color={rgb,1:red,0.0;green,0.0;blue,0.0}, opacity={1.0}}, y tick label style={color={rgb,1:red,0.0;green,0.0;blue,0.0}, opacity={1.0}, rotate={0}}, ylabel style={at={(ticklabel cs:0.5)}, anchor=near ticklabel, at={{(ticklabel cs:0.5)}}, anchor={near ticklabel}, font={{\fontsize{11 pt}{14.3 pt}\selectfont}}, color={rgb,1:red,0.0;green,0.0;blue,0.0}, draw opacity={1.0}, rotate={0.0}}, ymode={log}, log basis y={10}, ymajorgrids={true}, ymin={100000.0}, ymax={1.0e25}, yticklabels={{$10^{5}$,$10^{10}$,$10^{15}$,$10^{20}$,$10^{25}$}}, ytick={{100000.0,1.0e10,1.0e15,1.0e20,1.0e25}}, ytick align={inside}, yticklabel style={font={{\fontsize{8 pt}{10.4 pt}\selectfont}}, color={rgb,1:red,0.0;green,0.0;blue,0.0}, draw opacity={1.0}, rotate={0.0}}, y grid style={color={rgb,1:red,0.0;green,0.0;blue,0.0}, draw opacity={0.1}, line width={0.5}, solid}, axis y line*={left}, y axis line style={color={rgb,1:red,0.0;green,0.0;blue,0.0}, draw opacity={1.0}, line width={1}, solid}, colorbar={false}]
    [\addlegendimage{empty legend}] \addlegendentry[font={{\fontsize{11 pt}{14.3 pt}\selectfont}}, text={rgb,1:red,0.0;green,0.0;blue,0.0}] {\hspace{-.6cm}{\textbf{$(\gamma, \gamma_1, \gamma_2)$}}}
    \addplot[color={rgb,1:red,0.0;green,0.0;blue,1.0}, name path={d224c932-a57b-4b2d-b00b-6e21c1ac3de3}, draw opacity={1.0}, line width={1}, solid]
        table[row sep={\\}]
        {
            \\
            0.0  1.2679162841962957e8  \\
            0.012263883236204186  1.2718633259851372e8  \\
            0.024527766472408372  1.2693944203837477e8  \\
            0.03679164970861256  1.271863190038903e8  \\
            0.049055532944816745  1.2679162950937325e8  \\
        }
        ;
    \addlegendentry { $1.0 \cdot 10^{1}$, $0.5 \cdot 10^{1}$, $1.0 \cdot 10^{-1}$ }
    \addplot[color={rgb,1:red,1.0;green,0.0;blue,0.0}, name path={7adea5c0-5395-462d-ae7d-265324150fe4}, draw opacity={1.0}, line width={1}, solid]
        table[row sep={\\}]
        {
            \\
            0.0  4.6399601211603455e10  \\
            0.012263883236204186  4.93225235137952e12  \\
            0.024527766472408372  4.09594540313553e12  \\
            0.03679164970861256  4.93225240700152e12  \\
            0.049055532944816745  4.639960117645048e10  \\
        }
        ;
    \addlegendentry { $1.0 \cdot 10^{1}$, $ 0.0 \cdot 10^{0} $, $ 0.0 \cdot 10^{0} $ }
\end{axis}
\end{tikzpicture}

        \caption{Evolution of condition number}
        \label{subfig:cond}
    \end{subfigure}
    \hfill
    \begin{subfigure}{0.49\textwidth}
        \centering
        % Recommended preamble:
% \usetikzlibrary{arrows.meta}
% \usetikzlibrary{backgrounds}
% \usepgfplotslibrary{patchplots}
% \usepgfplotslibrary{fillbetween}
% \pgfplotsset{%
%     layers/standard/.define layer set={%
%         background,axis background,axis grid,axis ticks,axis lines,axis tick labels,pre main,main,axis descriptions,axis foreground%
%     }{
%         grid style={/pgfplots/on layer=axis grid},%
%         tick style={/pgfplots/on layer=axis ticks},%
%         axis line style={/pgfplots/on layer=axis lines},%
%         label style={/pgfplots/on layer=axis descriptions},%
%         legend style={/pgfplots/on layer=axis descriptions},%
%         title style={/pgfplots/on layer=axis descriptions},%
%         colorbar style={/pgfplots/on layer=axis descriptions},%
%         ticklabel style={/pgfplots/on layer=axis tick labels},%
%         axis background@ style={/pgfplots/on layer=axis background},%
%         3d box foreground style={/pgfplots/on layer=axis foreground},%
%     },
% }

\begin{tikzpicture}[/tikz/background rectangle/.style={fill={rgb,1:red,1.0;green,1.0;blue,1.0}, fill opacity={1.0}, draw opacity={1.0}}, show background rectangle]
\begin{axis}[point meta max={nan}, point meta min={nan}, legend cell align={left}, legend columns={1}, title={}, title style={at={{(0.5,1)}}, anchor={south}, font={{\fontsize{14 pt}{18.2 pt}\selectfont}}, color={rgb,1:red,0.0;green,0.0;blue,0.0}, draw opacity={1.0}, rotate={0.0}, align={center}}, legend style={color={rgb,1:red,0.0;green,0.0;blue,0.0}, draw opacity={1.0}, line width={1}, solid, fill={rgb,1:red,1.0;green,1.0;blue,1.0}, fill opacity={1.0}, text opacity={1.0}, font={{\fontsize{8 pt}{10.4 pt}\selectfont}}, text={rgb,1:red,0.0;green,0.0;blue,0.0}, cells={anchor={center}}, at={(1.02, 1)}, anchor={north west}}, axis background/.style={fill={rgb,1:red,1.0;green,1.0;blue,1.0}, opacity={1.0}}, anchor={north west}, xshift={1.0mm}, yshift={-1.0mm}, width={145.4mm}, height={99.6mm}, scaled x ticks={false}, xlabel={$h$}, x tick style={color={rgb,1:red,0.0;green,0.0;blue,0.0}, opacity={1.0}}, x tick label style={color={rgb,1:red,0.0;green,0.0;blue,0.0}, opacity={1.0}, rotate={0}}, xlabel style={at={(ticklabel cs:0.5)}, anchor=near ticklabel, at={{(ticklabel cs:0.5)}}, anchor={near ticklabel}, font={{\fontsize{11 pt}{14.3 pt}\selectfont}}, color={rgb,1:red,0.0;green,0.0;blue,0.0}, draw opacity={1.0}, rotate={0.0}}, xmode={log}, log basis x={2}, xmajorgrids={true}, xmin={0.0071889660205068364}, xmax={0.13584185781575728}, xticklabels={{$2^{-6}$,$2^{-4}$}}, xtick={{0.015625,0.0625}}, xtick align={inside}, xticklabel style={font={{\fontsize{8 pt}{10.4 pt}\selectfont}}, color={rgb,1:red,0.0;green,0.0;blue,0.0}, draw opacity={1.0}, rotate={0.0}}, x grid style={color={rgb,1:red,0.0;green,0.0;blue,0.0}, draw opacity={0.1}, line width={0.5}, solid}, axis x line*={left}, x axis line style={color={rgb,1:red,0.0;green,0.0;blue,0.0}, draw opacity={1.0}, line width={1}, solid}, scaled y ticks={false}, ylabel={$\Vert e \Vert_{L^2,solid} $, $\Vert e \Vert_{H^1,dash} $, $\Vert e \Vert_{ah,*,dot}$}, y tick style={color={rgb,1:red,0.0;green,0.0;blue,0.0}, opacity={1.0}}, y tick label style={color={rgb,1:red,0.0;green,0.0;blue,0.0}, opacity={1.0}, rotate={0}}, ylabel style={at={(ticklabel cs:0.5)}, anchor=near ticklabel, at={{(ticklabel cs:0.5)}}, anchor={near ticklabel}, font={{\fontsize{11 pt}{14.3 pt}\selectfont}}, color={rgb,1:red,0.0;green,0.0;blue,0.0}, draw opacity={1.0}, rotate={0.0}}, ymode={log}, log basis y={2}, ymajorgrids={true}, ymin={0.035222707312254}, ymax={4966.784325040246}, yticklabels={{$2^{0}$,$2^{8}$}}, ytick={{1.0,256.0}}, ytick align={inside}, yticklabel style={font={{\fontsize{8 pt}{10.4 pt}\selectfont}}, color={rgb,1:red,0.0;green,0.0;blue,0.0}, draw opacity={1.0}, rotate={0.0}}, y grid style={color={rgb,1:red,0.0;green,0.0;blue,0.0}, draw opacity={0.1}, line width={0.5}, solid}, axis y line*={left}, y axis line style={color={rgb,1:red,0.0;green,0.0;blue,0.0}, draw opacity={1.0}, line width={1}, solid}, colorbar={false}]
    [\addlegendimage{empty legend}] \addlegendentry[font={{\fontsize{11 pt}{14.3 pt}\selectfont}}, text={rgb,1:red,0.0;green,0.0;blue,0.0}] {\hspace{-.6cm}{\textbf{$(\gamma, \gamma_1, \gamma_2)$}}}
    \addplot[color={rgb,1:red,0.0;green,0.0;blue,1.0}, name path={d75fd04b-fa7b-40d5-925e-5d389c673476}, draw opacity={1.0}, line width={1}, solid]
        table[row sep={\\}]
        {
            \\
            0.125  16.34842746444951  \\
            0.0625  3.6963996701551864  \\
            0.03125  0.832182235270678  \\
            0.015625  0.22689359881799284  \\
            0.0078125  0.049893466171725895  \\
        }
        ;
    \addlegendentry { $1.0 \cdot 10^{1}$, $0.5 \cdot 10^{1}$, $1.0 \cdot 10^{-1}$ }
    \addplot[color={rgb,1:red,0.0;green,0.0;blue,1.0}, name path={83510fc8-71b9-43a7-8e29-7858ff508099}, draw opacity={1.0}, line width={1}, dashed, forget plot]
        table[row sep={\\}]
        {
            \\
            0.125  133.63863186431385  \\
            0.0625  31.494956792924192  \\
            0.03125  7.391091604884702  \\
            0.015625  1.8115670140035576  \\
            0.0078125  0.43974840251135394  \\
        }
        ;
    \addplot[color={rgb,1:red,0.0;green,0.0;blue,1.0}, name path={7e270904-0631-4169-8e55-281998d4631e}, draw opacity={1.0}, line width={1}, dotted, forget plot]
        table[row sep={\\}]
        {
            \\
            0.125  3550.926954471599  \\
            0.0625  1033.8788564955153  \\
            0.03125  291.62518463147273  \\
            0.015625  80.25236932142877  \\
            0.0078125  25.95463804360027  \\
        }
        ;
    \addplot[color={rgb,1:red,1.0;green,0.0;blue,0.0}, name path={b4b941b2-6b03-4713-93b7-c4a35b58c048}, draw opacity={1.0}, line width={1}, solid]
        table[row sep={\\}]
        {
            \\
            0.125  11.675160426868874  \\
            0.0625  3.0310075980218545  \\
            0.03125  0.785117037888616  \\
            0.015625  0.22179524786759885  \\
            0.0078125  0.04926702036032627  \\
        }
        ;
    \addlegendentry { $1.0 \cdot 10^{1}$, $ 0.0 \cdot 10^{0} $, $ 0.0 \cdot 10^{0} $ }
    \addplot[color={rgb,1:red,1.0;green,0.0;blue,0.0}, name path={289ccf37-65a0-435a-bd96-db6115e0f889}, draw opacity={1.0}, line width={1}, dashed, forget plot]
        table[row sep={\\}]
        {
            \\
            0.125  110.97495251333412  \\
            0.0625  27.425544131416416  \\
            0.03125  6.816491784603148  \\
            0.015625  1.7161738801520496  \\
            0.0078125  0.42689977409580526  \\
        }
        ;
    \addplot[color={rgb,1:red,1.0;green,0.0;blue,0.0}, name path={a3ee6f8b-19ef-4836-a893-cfb3e47b573f}, draw opacity={1.0}, line width={1}, dotted, forget plot]
        table[row sep={\\}]
        {
            \\
            0.125  3502.409729589457  \\
            0.0625  1001.3809339450683  \\
            0.03125  330.88972441337853  \\
            0.015625  83.26742393508408  \\
            0.0078125  29.738439170431377  \\
        }
        ;
\end{axis}
\end{tikzpicture}

        \caption{Evolution of error number}
        \label{subfig:error}
    \end{subfigure}
    \caption{Comparison of convergence test results with and without ghost penalty.}
    \label{fig:combined}
\end{figure}

In this section, we present the numerical results of the convergence analysis for a numerical method applied to a square grid with a length of 1. The grid is discretized with $n=2^6$ grid points. The analysis is based on a manufactured solution, which is given by the expression:

\[
u_{\text{ex}}(x) = 100 \sin\left(\frac{m(2\pi)}{L}x_1\right)\cos\left(\frac{r(2\pi)}{L}x_2\right)
\]

Here, $L$, $m$, and $r$ are constants representing the length of the square grid, the spatial frequency in the $x_1$ direction, and the spatial frequency in the $x_2$ direction, respectively.

The convergence analysis involves evaluating the errors in different norms as well as studying the condition number of the system. Table \ref{table:CutFEM_error1} presents the results of the convergence analysis for the numerical method. The column labeled "$n$" corresponds to the number of grid points, while the columns labeled "$\Vert e \Vert_{L^2}$," "$\Vert e \Vert_{H^1}$," and "$\Vert e \Vert_{a_h,}$" represent the errors in the $L^2$, $H^1$, and energy ($\Vert e \Vert_{a_h,}$) norms, respectively. The column labeled "EOC" denotes the experimental order of convergence. The last two columns provide the condition number of the system and the number of degrees of freedom (ndofs), respectively.

Figure \ref{fig:CutFEM_error1} displays a plot illustrating the error convergence in the $L^2$, $H^1$, and energy norms for the CutCIP method (Laplace) with second-order accuracy.

Additionally, we compare the results with and without the ghost penalty in two different tests: a translation test and a convergence test. The translation test examines the evolution of the condition number and error number, as shown in Figure \ref{fig:combined}. The convergence test also investigates the condition number and error number, and its results are presented in Figure \ref{fig:combined}.

Overall, these numerical results provide insights into the performance and accuracy of the numerical method, showcasing the convergence behavior and the impact of the ghost penalty on the solution.
