
\newpage
\section{Numerical results}%
\label{sec:numerical_results}


\subsection{Numerical Results for CutCIP Biharmonic Equation  }%
\label{sub:numerical_results_for_cutcip_biharmonic_equation_}

\begin{itemize}
    \item EOC Test on circle and its configuration
        \begin{enumerate}[label=\arabic*)]
            \item Penalty parameters $\gamma , \gamma _{g1}, \gamma _{g2} = 20,10, 0.1$
            \item Background mesh $L = 3.11$ and circle $R=1$, flower $r0, r1 = L0.3, L0.1$
        \end{enumerate}
    \item Translation
        \begin{enumerate}[label=\arabic*)]
            \item Background mesh $L = 3.11$ and $R=1$
        \end{enumerate}
\end{itemize}


In this section will we provide a numerical study of the methods studied in Section \ref{sec:cutcip_biharmonic_problem}.

\subsubsection{EOC test}%
\label{ssub:eoc_test}

Here we consider the manufactured solution $l,r,m = (2, 1, 1) $ s.t.
\[
u_{\text{ex}}(x,y) = (x^2- y^2 -1) \sin\left(\frac{2\pi m}{l}x_1\right)\cos\left(\frac{2\pi r}{l}y\right)
\]
On a square background mesh $L=3.11$ with a circle domain of radius $R=1$.







% In this section, we present the numerical results of the convergence analysis for a numerical method applied to a backrgound square grid with a length of 1. The grid is discretized with $n=2^6$ grid points. The analysis is based on a manufactured solution, which is given by the expression:

% \[
% u_{\text{ex}}(x) = 100 \sin\left(\frac{m(2\pi)}{L}x_1\right)\cos\left(\frac{r(2\pi)}{L}x_2\right)
% \]

% Here, $L$, $m$, and $r$ are constants representing the length of the square grid, the spatial frequency in the $x_1$ direction, and the spatial frequency in the $x_2$ direction, respectively.

% The convergence analysis involves evaluating the errors in different norms as well as studying the condition number of the system. Table \ref{table:CutFEM_error1} presents the results of the convergence analysis for the numerical method. The column labeled "$n$" corresponds to the number of grid points, while the columns labeled "$\Vert e \Vert_{L^2}$," "$\Vert e \Vert_{H^1}$," and "$\Vert e \Vert_{a_h,}$" represent the errors in the $L^2$, $H^1$, and energy ($\Vert e \Vert_{a_h,}$) norms, respectively. The column labeled "EOC" denotes the experimental order of convergence. The last two columns provide the condition number of the system and the number of degrees of freedom (ndofs), respectively.

% Figure \ref{fig:CutFEM_error1} displays a plot illustrating the error convergence in the $L^2$, $H^1$, and energy norms for the CutCIP method (Laplace) with second-order accuracy.

% Additionally, we compare the results with and without the ghost penalty in two different tests: a translation test and a convergence test. The translation test examines the evolution of the condition number and error number, as shown in Figure \ref{fig:combined}. The convergence test also investigates the condition number and error number, and its results are presented in Figure \ref{fig:combined}.

% Overall, these numerical results provide insights into the performance and accuracy of the numerical method, showcasing the convergence behavior and the impact of the ghost penalty on the solution.





