
\newpage
\section{Numerical results}%
\label{sec:numerical_results}


\begin{table}[h!]
    \caption{Convergence analysis of the numerical method}
    \label{table:CutFEM_error1}
  \begin{tabular}{rrrrrrrrr}
    \hline\hline
    \textbf{$n$} & \textbf{$\Vert e \Vert_{L^2}$} & \textbf{EOC} & \textbf{$ \Vert e \Vert_{H^1}$} & \textbf{EOC} & \textbf{$\Vert e \Vert_{ a_h,* }$} & \textbf{EOC} & \textbf{Cond number} & \textbf{ndofs} \\\hline
    8 & 1.6E+01 &  & 1.3E+02 &  & 3.6E+03 &  & 3.0E+05 & 2.4E+02 \\
    16 & 3.7E+00 & 2.14 & 3.1E+01 & 2.09 & 1.0E+03 & 1.78 & 2.2E+06 & 8.3E+02 \\
    32 & 8.3E-01 & 2.15 & 7.4E+00 & 2.09 & 2.9E+02 & 1.83 & 1.6E+07 & 3.0E+03 \\
    64 & 2.3E-01 & 1.87 & 1.8E+00 & 2.03 & 8.0E+01 & 1.86 & 1.3E+08 & 1.1E+04 \\
    128 & 5.0E-02 & 2.19 & 4.4E-01 & 2.04 & 2.6E+01 & 1.63 & 1.0E+09 & 4.3E+04 \\
    256 & 1.3E-02 & 1.92 & 1.1E-01 & 2.01 & 8.4E+00 & 1.63 & 7.9E+09 & 1.7E+05 \\\hline\hline
  \end{tabular}

\end{table}

\begin{figure}[h!]
    \centering
    % Recommended preamble:
% \usetikzlibrary{arrows.meta}
% \usetikzlibrary{backgrounds}
% \usepgfplotslibrary{patchplots}
% \usepgfplotslibrary{fillbetween}
% \pgfplotsset{%
%     layers/standard/.define layer set={%
%         background,axis background,axis grid,axis ticks,axis lines,axis tick labels,pre main,main,axis descriptions,axis foreground%
%     }{
%         grid style={/pgfplots/on layer=axis grid},%
%         tick style={/pgfplots/on layer=axis ticks},%
%         axis line style={/pgfplots/on layer=axis lines},%
%         label style={/pgfplots/on layer=axis descriptions},%
%         legend style={/pgfplots/on layer=axis descriptions},%
%         title style={/pgfplots/on layer=axis descriptions},%
%         colorbar style={/pgfplots/on layer=axis descriptions},%
%         ticklabel style={/pgfplots/on layer=axis tick labels},%
%         axis background@ style={/pgfplots/on layer=axis background},%
%         3d box foreground style={/pgfplots/on layer=axis foreground},%
%     },
% }

\begin{tikzpicture}[/tikz/background rectangle/.style={fill={rgb,1:red,1.0;green,1.0;blue,1.0}, fill opacity={1.0}, draw opacity={1.0}}, show background rectangle]
\begin{axis}[point meta max={nan}, point meta min={nan}, legend cell align={left}, legend columns={1}, title={}, title style={at={{(0.5,1)}}, anchor={south}, font={{\fontsize{14 pt}{18.2 pt}\selectfont}}, color={rgb,1:red,0.0;green,0.0;blue,0.0}, draw opacity={1.0}, rotate={0.0}, align={center}}, legend style={color={rgb,1:red,0.0;green,0.0;blue,0.0}, draw opacity={1.0}, line width={1}, solid, fill={rgb,1:red,1.0;green,1.0;blue,1.0}, fill opacity={1.0}, text opacity={1.0}, font={{\fontsize{8 pt}{10.4 pt}\selectfont}}, text={rgb,1:red,0.0;green,0.0;blue,0.0}, cells={anchor={center}}, at={(1.02, 1)}, anchor={north west}}, axis background/.style={fill={rgb,1:red,1.0;green,1.0;blue,1.0}, opacity={1.0}}, anchor={north west}, xshift={1.0mm}, yshift={-1.0mm}, width={94.6mm}, height={74.2mm}, scaled x ticks={false}, xlabel={$\delta$}, x tick style={color={rgb,1:red,0.0;green,0.0;blue,0.0}, opacity={1.0}}, x tick label style={color={rgb,1:red,0.0;green,0.0;blue,0.0}, opacity={1.0}, rotate={0}}, xlabel style={at={(ticklabel cs:0.5)}, anchor=near ticklabel, at={{(ticklabel cs:0.5)}}, anchor={near ticklabel}, font={{\fontsize{11 pt}{14.3 pt}\selectfont}}, color={rgb,1:red,0.0;green,0.0;blue,0.0}, draw opacity={1.0}, rotate={0.0}}, xmajorgrids={true}, xmin={-0.001471665988344504}, xmax={0.05052719893316125}, xticklabels={{$0.00$,$0.01$,$0.02$,$0.03$,$0.04$,$0.05$}}, xtick={{0.0,0.010000000000000002,0.020000000000000004,0.030000000000000006,0.04000000000000001,0.05000000000000001}}, xtick align={inside}, xticklabel style={font={{\fontsize{8 pt}{10.4 pt}\selectfont}}, color={rgb,1:red,0.0;green,0.0;blue,0.0}, draw opacity={1.0}, rotate={0.0}}, x grid style={color={rgb,1:red,0.0;green,0.0;blue,0.0}, draw opacity={0.1}, line width={0.5}, solid}, axis x line*={left}, x axis line style={color={rgb,1:red,0.0;green,0.0;blue,0.0}, draw opacity={1.0}, line width={1}, solid}, scaled y ticks={false}, ylabel={$\kappa(A)$}, y tick style={color={rgb,1:red,0.0;green,0.0;blue,0.0}, opacity={1.0}}, y tick label style={color={rgb,1:red,0.0;green,0.0;blue,0.0}, opacity={1.0}, rotate={0}}, ylabel style={at={(ticklabel cs:0.5)}, anchor=near ticklabel, at={{(ticklabel cs:0.5)}}, anchor={near ticklabel}, font={{\fontsize{11 pt}{14.3 pt}\selectfont}}, color={rgb,1:red,0.0;green,0.0;blue,0.0}, draw opacity={1.0}, rotate={0.0}}, ymode={log}, log basis y={10}, ymajorgrids={true}, ymin={100000.0}, ymax={1.0e25}, yticklabels={{$10^{5}$,$10^{10}$,$10^{15}$,$10^{20}$,$10^{25}$}}, ytick={{100000.0,1.0e10,1.0e15,1.0e20,1.0e25}}, ytick align={inside}, yticklabel style={font={{\fontsize{8 pt}{10.4 pt}\selectfont}}, color={rgb,1:red,0.0;green,0.0;blue,0.0}, draw opacity={1.0}, rotate={0.0}}, y grid style={color={rgb,1:red,0.0;green,0.0;blue,0.0}, draw opacity={0.1}, line width={0.5}, solid}, axis y line*={left}, y axis line style={color={rgb,1:red,0.0;green,0.0;blue,0.0}, draw opacity={1.0}, line width={1}, solid}, colorbar={false}]
    [\addlegendimage{empty legend}] \addlegendentry[font={{\fontsize{11 pt}{14.3 pt}\selectfont}}, text={rgb,1:red,0.0;green,0.0;blue,0.0}] {\hspace{-.6cm}{\textbf{$(\gamma, \gamma_1, \gamma_2)$}}}
    \addplot[color={rgb,1:red,0.0;green,0.0;blue,1.0}, name path={e73f70a5-bbbd-4a24-8407-eef9d08bc09d}, draw opacity={1.0}, line width={1}, solid]
        table[row sep={\\}]
        {
            \\
            0.0  1.2679164207221107e8  \\
            0.012263883236204186  1.2718633484176141e8  \\
            0.024527766472408372  1.2693944128424127e8  \\
            0.03679164970861256  1.2718633338545807e8  \\
            0.049055532944816745  1.2679164340263365e8  \\
        }
        ;
    \addlegendentry { $1.0 \cdot 10^{1}$, $0.5 \cdot 10^{1}$, $1.0 \cdot 10^{-1}$ }
    \addplot[color={rgb,1:red,1.0;green,0.0;blue,0.0}, name path={8dbacc37-3690-4b1f-9d5d-1b6f1fc900ac}, draw opacity={1.0}, line width={1}, solid]
        table[row sep={\\}]
        {
            \\
            0.0  4.639960110962716e10  \\
            0.012263883236204186  4.932252358128007e12  \\
            0.024527766472408372  4.0959531093983247e12  \\
            0.03679164970861256  4.932252223523245e12  \\
            0.049055532944816745  4.639960117645048e10  \\
        }
        ;
    \addlegendentry { $1.0 \cdot 10^{1}$, $ 0.0 \cdot 10^{0} $, $ 0.0 \cdot 10^{0} $ }
\end{axis}
\end{tikzpicture}

    \caption{The plot presents the error in $L^2$, $H^1$ the energy norm ($\Vert e \Vert_{a_h,*}$) with order 2 of the CutCIP method (Laplace).}
    \label{fig:CutFEM_error1}
\end{figure}

\begin{figure}[h!]
    \centering
    \begin{subfigure}{\textwidth}
        \centering
        
% Recommended preamble:
% \usetikzlibrary{arrows.meta}
% \usetikzlibrary{backgrounds}
% \usepgfplotslibrary{patchplots}
% \usepgfplotslibrary{fillbetween}
% \pgfplotsset{%
%     layers/standard/.define layer set={%
%         background,axis background,axis grid,axis ticks,axis lines,axis tick labels,pre main,main,axis descriptions,axis foreground%
%     }{
%         grid style={/pgfplots/on layer=axis grid},%
%         tick style={/pgfplots/on layer=axis ticks},%
%         axis line style={/pgfplots/on layer=axis lines},%
%         label style={/pgfplots/on layer=axis descriptions},%
%         legend style={/pgfplots/on layer=axis descriptions},%
%         title style={/pgfplots/on layer=axis descriptions},%
%         colorbar style={/pgfplots/on layer=axis descriptions},%
%         ticklabel style={/pgfplots/on layer=axis tick labels},%
%         axis background@ style={/pgfplots/on layer=axis background},%
%         3d box foreground style={/pgfplots/on layer=axis foreground},%
%     },
% }

\begin{tikzpicture}[/tikz/background rectangle/.style={fill={rgb,1:red,1.0;green,1.0;blue,1.0}, fill opacity={1.0}, draw opacity={1.0}}, show background rectangle]
\begin{axis}[point meta max={nan}, point meta min={nan}, legend cell align={left}, legend columns={1}, title={}, title style={at={{(0.5,1)}}, anchor={south}, font={{\fontsize{14 pt}{18.2 pt}\selectfont}}, color={rgb,1:red,0.0;green,0.0;blue,0.0}, draw opacity={1.0}, rotate={0.0}, align={center}}, legend style={color={rgb,1:red,0.0;green,0.0;blue,0.0}, draw opacity={1.0}, line width={1}, solid, fill={rgb,1:red,1.0;green,1.0;blue,1.0}, fill opacity={1.0}, text opacity={1.0}, font={{\fontsize{8 pt}{10.4 pt}\selectfont}}, text={rgb,1:red,0.0;green,0.0;blue,0.0}, cells={anchor={center}}, at={(1.02, 1)}, anchor={north west}}, axis background/.style={fill={rgb,1:red,1.0;green,1.0;blue,1.0}, opacity={1.0}}, anchor={north west}, xshift={1.0mm}, yshift={-1.0mm}, width={145.4mm}, height={99.6mm}, scaled x ticks={false}, xlabel={$\delta$}, x tick style={color={rgb,1:red,0.0;green,0.0;blue,0.0}, opacity={1.0}}, x tick label style={color={rgb,1:red,0.0;green,0.0;blue,0.0}, opacity={1.0}, rotate={0}}, xlabel style={at={(ticklabel cs:0.5)}, anchor=near ticklabel, at={{(ticklabel cs:0.5)}}, anchor={near ticklabel}, font={{\fontsize{11 pt}{14.3 pt}\selectfont}}, color={rgb,1:red,0.0;green,0.0;blue,0.0}, draw opacity={1.0}, rotate={0.0}}, xmajorgrids={true}, xmin={-0.001471665988344504}, xmax={0.05052719893316125}, xticklabels={{$0.00$,$0.01$,$0.02$,$0.03$,$0.04$,$0.05$}}, xtick={{0.0,0.010000000000000002,0.020000000000000004,0.030000000000000006,0.04000000000000001,0.05000000000000001}}, xtick align={inside}, xticklabel style={font={{\fontsize{8 pt}{10.4 pt}\selectfont}}, color={rgb,1:red,0.0;green,0.0;blue,0.0}, draw opacity={1.0}, rotate={0.0}}, x grid style={color={rgb,1:red,0.0;green,0.0;blue,0.0}, draw opacity={0.1}, line width={0.5}, solid}, axis x line*={left}, x axis line style={color={rgb,1:red,0.0;green,0.0;blue,0.0}, draw opacity={1.0}, line width={1}, solid}, scaled y ticks={false}, ylabel={$\kappa(A)$}, y tick style={color={rgb,1:red,0.0;green,0.0;blue,0.0}, opacity={1.0}}, y tick label style={color={rgb,1:red,0.0;green,0.0;blue,0.0}, opacity={1.0}, rotate={0}}, ylabel style={at={(ticklabel cs:0.5)}, anchor=near ticklabel, at={{(ticklabel cs:0.5)}}, anchor={near ticklabel}, font={{\fontsize{11 pt}{14.3 pt}\selectfont}}, color={rgb,1:red,0.0;green,0.0;blue,0.0}, draw opacity={1.0}, rotate={0.0}}, ymode={log}, log basis y={10}, ymajorgrids={true}, ymin={100000.0}, ymax={1.0e25}, yticklabels={{$10^{5}$,$10^{10}$,$10^{15}$,$10^{20}$,$10^{25}$}}, ytick={{100000.0,1.0e10,1.0e15,1.0e20,1.0e25}}, ytick align={inside}, yticklabel style={font={{\fontsize{8 pt}{10.4 pt}\selectfont}}, color={rgb,1:red,0.0;green,0.0;blue,0.0}, draw opacity={1.0}, rotate={0.0}}, y grid style={color={rgb,1:red,0.0;green,0.0;blue,0.0}, draw opacity={0.1}, line width={0.5}, solid}, axis y line*={left}, y axis line style={color={rgb,1:red,0.0;green,0.0;blue,0.0}, draw opacity={1.0}, line width={1}, solid}, colorbar={false}]
    [\addlegendimage{empty legend}] \addlegendentry[font={{\fontsize{11 pt}{14.3 pt}\selectfont}}, text={rgb,1:red,0.0;green,0.0;blue,0.0}] {\hspace{-.6cm}{\textbf{$(\gamma, \gamma_1, \gamma_2)$}}}
    \addplot[color={rgb,1:red,0.0;green,0.0;blue,1.0}, name path={d224c932-a57b-4b2d-b00b-6e21c1ac3de3}, draw opacity={1.0}, line width={1}, solid]
        table[row sep={\\}]
        {
            \\
            0.0  1.2679162841962957e8  \\
            0.012263883236204186  1.2718633259851372e8  \\
            0.024527766472408372  1.2693944203837477e8  \\
            0.03679164970861256  1.271863190038903e8  \\
            0.049055532944816745  1.2679162950937325e8  \\
        }
        ;
    \addlegendentry { $1.0 \cdot 10^{1}$, $0.5 \cdot 10^{1}$, $1.0 \cdot 10^{-1}$ }
    \addplot[color={rgb,1:red,1.0;green,0.0;blue,0.0}, name path={7adea5c0-5395-462d-ae7d-265324150fe4}, draw opacity={1.0}, line width={1}, solid]
        table[row sep={\\}]
        {
            \\
            0.0  4.6399601211603455e10  \\
            0.012263883236204186  4.93225235137952e12  \\
            0.024527766472408372  4.09594540313553e12  \\
            0.03679164970861256  4.93225240700152e12  \\
            0.049055532944816745  4.639960117645048e10  \\
        }
        ;
    \addlegendentry { $1.0 \cdot 10^{1}$, $ 0.0 \cdot 10^{0} $, $ 0.0 \cdot 10^{0} $ }
\end{axis}
\end{tikzpicture}

        \caption{Evolution of condition number}
        \label{subfig:cond}
    \end{subfigure}
    \hfill
    \begin{subfigure}{\textwidth}
        \centering
        
% Recommended preamble:
% \usetikzlibrary{arrows.meta}
% \usetikzlibrary{backgrounds}
% \usepgfplotslibrary{patchplots}
% \usepgfplotslibrary{fillbetween}
% \pgfplotsset{%
%     layers/standard/.define layer set={%
%         background,axis background,axis grid,axis ticks,axis lines,axis tick labels,pre main,main,axis descriptions,axis foreground%
%     }{
%         grid style={/pgfplots/on layer=axis grid},%
%         tick style={/pgfplots/on layer=axis ticks},%
%         axis line style={/pgfplots/on layer=axis lines},%
%         label style={/pgfplots/on layer=axis descriptions},%
%         legend style={/pgfplots/on layer=axis descriptions},%
%         title style={/pgfplots/on layer=axis descriptions},%
%         colorbar style={/pgfplots/on layer=axis descriptions},%
%         ticklabel style={/pgfplots/on layer=axis tick labels},%
%         axis background@ style={/pgfplots/on layer=axis background},%
%         3d box foreground style={/pgfplots/on layer=axis foreground},%
%     },
% }

\begin{tikzpicture}[/tikz/background rectangle/.style={fill={rgb,1:red,1.0;green,1.0;blue,1.0}, fill opacity={1.0}, draw opacity={1.0}}, show background rectangle]
\begin{axis}[point meta max={nan}, point meta min={nan}, legend cell align={left}, legend columns={1}, title={}, title style={at={{(0.5,1)}}, anchor={south}, font={{\fontsize{14 pt}{18.2 pt}\selectfont}}, color={rgb,1:red,0.0;green,0.0;blue,0.0}, draw opacity={1.0}, rotate={0.0}, align={center}}, legend style={color={rgb,1:red,0.0;green,0.0;blue,0.0}, draw opacity={1.0}, line width={1}, solid, fill={rgb,1:red,1.0;green,1.0;blue,1.0}, fill opacity={1.0}, text opacity={1.0}, font={{\fontsize{8 pt}{10.4 pt}\selectfont}}, text={rgb,1:red,0.0;green,0.0;blue,0.0}, cells={anchor={center}}, at={(1.02, 1)}, anchor={north west}}, axis background/.style={fill={rgb,1:red,1.0;green,1.0;blue,1.0}, opacity={1.0}}, anchor={north west}, xshift={1.0mm}, yshift={-1.0mm}, width={145.4mm}, height={99.6mm}, scaled x ticks={false}, xlabel={$\delta$}, x tick style={color={rgb,1:red,0.0;green,0.0;blue,0.0}, opacity={1.0}}, x tick label style={color={rgb,1:red,0.0;green,0.0;blue,0.0}, opacity={1.0}, rotate={0}}, xlabel style={at={(ticklabel cs:0.5)}, anchor=near ticklabel, at={{(ticklabel cs:0.5)}}, anchor={near ticklabel}, font={{\fontsize{11 pt}{14.3 pt}\selectfont}}, color={rgb,1:red,0.0;green,0.0;blue,0.0}, draw opacity={1.0}, rotate={0.0}}, xmajorgrids={true}, xmin={-0.001471665988344504}, xmax={0.05052719893316125}, xticklabels={{$0.00$,$0.01$,$0.02$,$0.03$,$0.04$,$0.05$}}, xtick={{0.0,0.010000000000000002,0.020000000000000004,0.030000000000000006,0.04000000000000001,0.05000000000000001}}, xtick align={inside}, xticklabel style={font={{\fontsize{8 pt}{10.4 pt}\selectfont}}, color={rgb,1:red,0.0;green,0.0;blue,0.0}, draw opacity={1.0}, rotate={0.0}}, x grid style={color={rgb,1:red,0.0;green,0.0;blue,0.0}, draw opacity={0.1}, line width={0.5}, solid}, axis x line*={left}, x axis line style={color={rgb,1:red,0.0;green,0.0;blue,0.0}, draw opacity={1.0}, line width={1}, solid}, scaled y ticks={false}, ylabel={$\kappa(A)$}, y tick style={color={rgb,1:red,0.0;green,0.0;blue,0.0}, opacity={1.0}}, y tick label style={color={rgb,1:red,0.0;green,0.0;blue,0.0}, opacity={1.0}, rotate={0}}, ylabel style={at={(ticklabel cs:0.5)}, anchor=near ticklabel, at={{(ticklabel cs:0.5)}}, anchor={near ticklabel}, font={{\fontsize{11 pt}{14.3 pt}\selectfont}}, color={rgb,1:red,0.0;green,0.0;blue,0.0}, draw opacity={1.0}, rotate={0.0}}, ymode={log}, log basis y={10}, ymajorgrids={true}, ymin={100000.0}, ymax={1.0e25}, yticklabels={{$10^{5}$,$10^{10}$,$10^{15}$,$10^{20}$,$10^{25}$}}, ytick={{100000.0,1.0e10,1.0e15,1.0e20,1.0e25}}, ytick align={inside}, yticklabel style={font={{\fontsize{8 pt}{10.4 pt}\selectfont}}, color={rgb,1:red,0.0;green,0.0;blue,0.0}, draw opacity={1.0}, rotate={0.0}}, y grid style={color={rgb,1:red,0.0;green,0.0;blue,0.0}, draw opacity={0.1}, line width={0.5}, solid}, axis y line*={left}, y axis line style={color={rgb,1:red,0.0;green,0.0;blue,0.0}, draw opacity={1.0}, line width={1}, solid}, colorbar={false}]
    [\addlegendimage{empty legend}] \addlegendentry[font={{\fontsize{11 pt}{14.3 pt}\selectfont}}, text={rgb,1:red,0.0;green,0.0;blue,0.0}] {\hspace{-.6cm}{\textbf{$(\gamma, \gamma_1, \gamma_2)$}}}
    \addplot[color={rgb,1:red,0.0;green,0.0;blue,1.0}, name path={d224c932-a57b-4b2d-b00b-6e21c1ac3de3}, draw opacity={1.0}, line width={1}, solid]
        table[row sep={\\}]
        {
            \\
            0.0  1.2679162841962957e8  \\
            0.012263883236204186  1.2718633259851372e8  \\
            0.024527766472408372  1.2693944203837477e8  \\
            0.03679164970861256  1.271863190038903e8  \\
            0.049055532944816745  1.2679162950937325e8  \\
        }
        ;
    \addlegendentry { $1.0 \cdot 10^{1}$, $0.5 \cdot 10^{1}$, $1.0 \cdot 10^{-1}$ }
    \addplot[color={rgb,1:red,1.0;green,0.0;blue,0.0}, name path={7adea5c0-5395-462d-ae7d-265324150fe4}, draw opacity={1.0}, line width={1}, solid]
        table[row sep={\\}]
        {
            \\
            0.0  4.6399601211603455e10  \\
            0.012263883236204186  4.93225235137952e12  \\
            0.024527766472408372  4.09594540313553e12  \\
            0.03679164970861256  4.93225240700152e12  \\
            0.049055532944816745  4.639960117645048e10  \\
        }
        ;
    \addlegendentry { $1.0 \cdot 10^{1}$, $ 0.0 \cdot 10^{0} $, $ 0.0 \cdot 10^{0} $ }
\end{axis}
\end{tikzpicture}

        \caption{Evolution of error number}
        \label{subfig:error}
    \end{subfigure}
    \caption{Comparison of translation from with $500$ steps from test results with and without ghost penalty.}





    \label{fig:combined}
\end{figure}




\begin{figure}[h!]
    \centering
    \begin{subfigure}{0.49\textwidth}
        \centering
        % Recommended preamble:
% \usetikzlibrary{arrows.meta}
% \usetikzlibrary{backgrounds}
% \usepgfplotslibrary{patchplots}
% \usepgfplotslibrary{fillbetween}
% \pgfplotsset{%
%     layers/standard/.define layer set={%
%         background,axis background,axis grid,axis ticks,axis lines,axis tick labels,pre main,main,axis descriptions,axis foreground%
%     }{
%         grid style={/pgfplots/on layer=axis grid},%
%         tick style={/pgfplots/on layer=axis ticks},%
%         axis line style={/pgfplots/on layer=axis lines},%
%         label style={/pgfplots/on layer=axis descriptions},%
%         legend style={/pgfplots/on layer=axis descriptions},%
%         title style={/pgfplots/on layer=axis descriptions},%
%         colorbar style={/pgfplots/on layer=axis descriptions},%
%         ticklabel style={/pgfplots/on layer=axis tick labels},%
%         axis background@ style={/pgfplots/on layer=axis background},%
%         3d box foreground style={/pgfplots/on layer=axis foreground},%
%     },
% }

\begin{tikzpicture}[/tikz/background rectangle/.style={fill={rgb,1:red,1.0;green,1.0;blue,1.0}, fill opacity={1.0}, draw opacity={1.0}}, show background rectangle]
\begin{axis}[point meta max={nan}, point meta min={nan}, legend cell align={left}, legend columns={1}, title={}, title style={at={{(0.5,1)}}, anchor={south}, font={{\fontsize{14 pt}{18.2 pt}\selectfont}}, color={rgb,1:red,0.0;green,0.0;blue,0.0}, draw opacity={1.0}, rotate={0.0}, align={center}}, legend style={color={rgb,1:red,0.0;green,0.0;blue,0.0}, draw opacity={1.0}, line width={1}, solid, fill={rgb,1:red,1.0;green,1.0;blue,1.0}, fill opacity={1.0}, text opacity={1.0}, font={{\fontsize{8 pt}{10.4 pt}\selectfont}}, text={rgb,1:red,0.0;green,0.0;blue,0.0}, cells={anchor={center}}, at={(1.02, 1)}, anchor={north west}}, axis background/.style={fill={rgb,1:red,1.0;green,1.0;blue,1.0}, opacity={1.0}}, anchor={north west}, xshift={1.0mm}, yshift={-1.0mm}, width={145.4mm}, height={99.6mm}, scaled x ticks={false}, xlabel={$h$}, x tick style={color={rgb,1:red,0.0;green,0.0;blue,0.0}, opacity={1.0}}, x tick label style={color={rgb,1:red,0.0;green,0.0;blue,0.0}, opacity={1.0}, rotate={0}}, xlabel style={at={(ticklabel cs:0.5)}, anchor=near ticklabel, at={{(ticklabel cs:0.5)}}, anchor={near ticklabel}, font={{\fontsize{11 pt}{14.3 pt}\selectfont}}, color={rgb,1:red,0.0;green,0.0;blue,0.0}, draw opacity={1.0}, rotate={0.0}}, xmode={log}, log basis x={2}, xmajorgrids={true}, xmin={0.0071889660205068364}, xmax={0.13584185781575728}, xticklabels={{$2^{-6}$,$2^{-4}$}}, xtick={{0.015625,0.0625}}, xtick align={inside}, xticklabel style={font={{\fontsize{8 pt}{10.4 pt}\selectfont}}, color={rgb,1:red,0.0;green,0.0;blue,0.0}, draw opacity={1.0}, rotate={0.0}}, x grid style={color={rgb,1:red,0.0;green,0.0;blue,0.0}, draw opacity={0.1}, line width={0.5}, solid}, axis x line*={left}, x axis line style={color={rgb,1:red,0.0;green,0.0;blue,0.0}, draw opacity={1.0}, line width={1}, solid}, scaled y ticks={false}, ylabel={$\kappa(A)$}, y tick style={color={rgb,1:red,0.0;green,0.0;blue,0.0}, opacity={1.0}}, y tick label style={color={rgb,1:red,0.0;green,0.0;blue,0.0}, opacity={1.0}, rotate={0}}, ylabel style={at={(ticklabel cs:0.5)}, anchor=near ticklabel, at={{(ticklabel cs:0.5)}}, anchor={near ticklabel}, font={{\fontsize{11 pt}{14.3 pt}\selectfont}}, color={rgb,1:red,0.0;green,0.0;blue,0.0}, draw opacity={1.0}, rotate={0.0}}, ymode={log}, log basis y={10}, ymajorgrids={true}, ymin={100000.0}, ymax={1.0e25}, yticklabels={{$10^{5}$,$10^{10}$,$10^{15}$,$10^{20}$,$10^{25}$}}, ytick={{100000.0,1.0e10,1.0e15,1.0e20,1.0e25}}, ytick align={inside}, yticklabel style={font={{\fontsize{8 pt}{10.4 pt}\selectfont}}, color={rgb,1:red,0.0;green,0.0;blue,0.0}, draw opacity={1.0}, rotate={0.0}}, y grid style={color={rgb,1:red,0.0;green,0.0;blue,0.0}, draw opacity={0.1}, line width={0.5}, solid}, axis y line*={left}, y axis line style={color={rgb,1:red,0.0;green,0.0;blue,0.0}, draw opacity={1.0}, line width={1}, solid}, colorbar={false}]
    [\addlegendimage{empty legend}] \addlegendentry[font={{\fontsize{11 pt}{14.3 pt}\selectfont}}, text={rgb,1:red,0.0;green,0.0;blue,0.0}] {\hspace{-.6cm}{\textbf{$(\gamma, \gamma_1, \gamma_2)$}}}
    \addplot[color={rgb,1:red,0.0;green,0.0;blue,1.0}, name path={cd886131-6e89-4aef-9cd9-09573d95afaf}, draw opacity={1.0}, line width={1}, solid]
        table[row sep={\\}]
        {
            \\
            0.125  303341.95674632216  \\
            0.0625  2.1898888697866597e6  \\
            0.03125  1.6470284999248588e7  \\
            0.015625  1.2679164353299382e8  \\
            0.0078125  9.985977585452957e8  \\
        }
        ;
    \addlegendentry { $1.0 \cdot 10^{1}$, $0.5 \cdot 10^{1}$, $1.0 \cdot 10^{-1}$ }
    \addplot[color={rgb,1:red,1.0;green,0.0;blue,0.0}, name path={df6879df-cee6-4335-863a-ae6ea998e383}, draw opacity={1.0}, line width={1}, solid]
        table[row sep={\\}]
        {
            \\
            0.125  207446.0080439886  \\
            0.0625  2.4729631192660045e7  \\
            0.03125  6.453262740089494e7  \\
            0.015625  4.639960110962716e10  \\
            0.0078125  7.409810205046619e10  \\
        }
        ;
    \addlegendentry { $1.0 \cdot 10^{1}$, $ 0.0 \cdot 10^{0} $, $ 0.0 \cdot 10^{0} $ }
\end{axis}
\end{tikzpicture}

        \caption{Evolution of condition number}
        \label{subfig:cond}
    \end{subfigure}
    \hfill
    \begin{subfigure}{0.49\textwidth}
        \centering
        % Recommended preamble:
% \usetikzlibrary{arrows.meta}
% \usetikzlibrary{backgrounds}
% \usepgfplotslibrary{patchplots}
% \usepgfplotslibrary{fillbetween}
% \pgfplotsset{%
%     layers/standard/.define layer set={%
%         background,axis background,axis grid,axis ticks,axis lines,axis tick labels,pre main,main,axis descriptions,axis foreground%
%     }{
%         grid style={/pgfplots/on layer=axis grid},%
%         tick style={/pgfplots/on layer=axis ticks},%
%         axis line style={/pgfplots/on layer=axis lines},%
%         label style={/pgfplots/on layer=axis descriptions},%
%         legend style={/pgfplots/on layer=axis descriptions},%
%         title style={/pgfplots/on layer=axis descriptions},%
%         colorbar style={/pgfplots/on layer=axis descriptions},%
%         ticklabel style={/pgfplots/on layer=axis tick labels},%
%         axis background@ style={/pgfplots/on layer=axis background},%
%         3d box foreground style={/pgfplots/on layer=axis foreground},%
%     },
% }

\begin{tikzpicture}[/tikz/background rectangle/.style={fill={rgb,1:red,1.0;green,1.0;blue,1.0}, fill opacity={1.0}, draw opacity={1.0}}, show background rectangle]
\begin{axis}[point meta max={nan}, point meta min={nan}, legend cell align={left}, legend columns={1}, title={}, title style={at={{(0.5,1)}}, anchor={south}, font={{\fontsize{14 pt}{18.2 pt}\selectfont}}, color={rgb,1:red,0.0;green,0.0;blue,0.0}, draw opacity={1.0}, rotate={0.0}, align={center}}, legend style={color={rgb,1:red,0.0;green,0.0;blue,0.0}, draw opacity={1.0}, line width={1}, solid, fill={rgb,1:red,1.0;green,1.0;blue,1.0}, fill opacity={1.0}, text opacity={1.0}, font={{\fontsize{8 pt}{10.4 pt}\selectfont}}, text={rgb,1:red,0.0;green,0.0;blue,0.0}, cells={anchor={center}}, at={(1.02, 1)}, anchor={north west}}, axis background/.style={fill={rgb,1:red,1.0;green,1.0;blue,1.0}, opacity={1.0}}, anchor={north west}, xshift={1.0mm}, yshift={-1.0mm}, width={145.4mm}, height={99.6mm}, scaled x ticks={false}, xlabel={$h$}, x tick style={color={rgb,1:red,0.0;green,0.0;blue,0.0}, opacity={1.0}}, x tick label style={color={rgb,1:red,0.0;green,0.0;blue,0.0}, opacity={1.0}, rotate={0}}, xlabel style={at={(ticklabel cs:0.5)}, anchor=near ticklabel, at={{(ticklabel cs:0.5)}}, anchor={near ticklabel}, font={{\fontsize{11 pt}{14.3 pt}\selectfont}}, color={rgb,1:red,0.0;green,0.0;blue,0.0}, draw opacity={1.0}, rotate={0.0}}, xmode={log}, log basis x={2}, xmajorgrids={true}, xmin={0.0071889660205068364}, xmax={0.13584185781575728}, xticklabels={{$2^{-6}$,$2^{-4}$}}, xtick={{0.015625,0.0625}}, xtick align={inside}, xticklabel style={font={{\fontsize{8 pt}{10.4 pt}\selectfont}}, color={rgb,1:red,0.0;green,0.0;blue,0.0}, draw opacity={1.0}, rotate={0.0}}, x grid style={color={rgb,1:red,0.0;green,0.0;blue,0.0}, draw opacity={0.1}, line width={0.5}, solid}, axis x line*={left}, x axis line style={color={rgb,1:red,0.0;green,0.0;blue,0.0}, draw opacity={1.0}, line width={1}, solid}, scaled y ticks={false}, ylabel={$\Vert e \Vert_{L^2,solid} $, $\Vert e \Vert_{H^1,dash} $, $\Vert e \Vert_{ah,*,dot}$}, y tick style={color={rgb,1:red,0.0;green,0.0;blue,0.0}, opacity={1.0}}, y tick label style={color={rgb,1:red,0.0;green,0.0;blue,0.0}, opacity={1.0}, rotate={0}}, ylabel style={at={(ticklabel cs:0.5)}, anchor=near ticklabel, at={{(ticklabel cs:0.5)}}, anchor={near ticklabel}, font={{\fontsize{11 pt}{14.3 pt}\selectfont}}, color={rgb,1:red,0.0;green,0.0;blue,0.0}, draw opacity={1.0}, rotate={0.0}}, ymode={log}, log basis y={2}, ymajorgrids={true}, ymin={0.035222707312254}, ymax={4966.784325040246}, yticklabels={{$2^{0}$,$2^{8}$}}, ytick={{1.0,256.0}}, ytick align={inside}, yticklabel style={font={{\fontsize{8 pt}{10.4 pt}\selectfont}}, color={rgb,1:red,0.0;green,0.0;blue,0.0}, draw opacity={1.0}, rotate={0.0}}, y grid style={color={rgb,1:red,0.0;green,0.0;blue,0.0}, draw opacity={0.1}, line width={0.5}, solid}, axis y line*={left}, y axis line style={color={rgb,1:red,0.0;green,0.0;blue,0.0}, draw opacity={1.0}, line width={1}, solid}, colorbar={false}]
    [\addlegendimage{empty legend}] \addlegendentry[font={{\fontsize{11 pt}{14.3 pt}\selectfont}}, text={rgb,1:red,0.0;green,0.0;blue,0.0}] {\hspace{-.6cm}{\textbf{$(\gamma, \gamma_1, \gamma_2)$}}}
    \addplot[color={rgb,1:red,0.0;green,0.0;blue,1.0}, name path={d75fd04b-fa7b-40d5-925e-5d389c673476}, draw opacity={1.0}, line width={1}, solid]
        table[row sep={\\}]
        {
            \\
            0.125  16.34842746444951  \\
            0.0625  3.6963996701551864  \\
            0.03125  0.832182235270678  \\
            0.015625  0.22689359881799284  \\
            0.0078125  0.049893466171725895  \\
        }
        ;
    \addlegendentry { $1.0 \cdot 10^{1}$, $0.5 \cdot 10^{1}$, $1.0 \cdot 10^{-1}$ }
    \addplot[color={rgb,1:red,0.0;green,0.0;blue,1.0}, name path={83510fc8-71b9-43a7-8e29-7858ff508099}, draw opacity={1.0}, line width={1}, dashed, forget plot]
        table[row sep={\\}]
        {
            \\
            0.125  133.63863186431385  \\
            0.0625  31.494956792924192  \\
            0.03125  7.391091604884702  \\
            0.015625  1.8115670140035576  \\
            0.0078125  0.43974840251135394  \\
        }
        ;
    \addplot[color={rgb,1:red,0.0;green,0.0;blue,1.0}, name path={7e270904-0631-4169-8e55-281998d4631e}, draw opacity={1.0}, line width={1}, dotted, forget plot]
        table[row sep={\\}]
        {
            \\
            0.125  3550.926954471599  \\
            0.0625  1033.8788564955153  \\
            0.03125  291.62518463147273  \\
            0.015625  80.25236932142877  \\
            0.0078125  25.95463804360027  \\
        }
        ;
    \addplot[color={rgb,1:red,1.0;green,0.0;blue,0.0}, name path={b4b941b2-6b03-4713-93b7-c4a35b58c048}, draw opacity={1.0}, line width={1}, solid]
        table[row sep={\\}]
        {
            \\
            0.125  11.675160426868874  \\
            0.0625  3.0310075980218545  \\
            0.03125  0.785117037888616  \\
            0.015625  0.22179524786759885  \\
            0.0078125  0.04926702036032627  \\
        }
        ;
    \addlegendentry { $1.0 \cdot 10^{1}$, $ 0.0 \cdot 10^{0} $, $ 0.0 \cdot 10^{0} $ }
    \addplot[color={rgb,1:red,1.0;green,0.0;blue,0.0}, name path={289ccf37-65a0-435a-bd96-db6115e0f889}, draw opacity={1.0}, line width={1}, dashed, forget plot]
        table[row sep={\\}]
        {
            \\
            0.125  110.97495251333412  \\
            0.0625  27.425544131416416  \\
            0.03125  6.816491784603148  \\
            0.015625  1.7161738801520496  \\
            0.0078125  0.42689977409580526  \\
        }
        ;
    \addplot[color={rgb,1:red,1.0;green,0.0;blue,0.0}, name path={a3ee6f8b-19ef-4836-a893-cfb3e47b573f}, draw opacity={1.0}, line width={1}, dotted, forget plot]
        table[row sep={\\}]
        {
            \\
            0.125  3502.409729589457  \\
            0.0625  1001.3809339450683  \\
            0.03125  330.88972441337853  \\
            0.015625  83.26742393508408  \\
            0.0078125  29.738439170431377  \\
        }
        ;
\end{axis}
\end{tikzpicture}

        \caption{Evolution of error number}
        \label{subfig:error}
    \end{subfigure}
    \caption{Comparison of convergence test results with and without ghost penalty.}
    \label{fig:combined}
\end{figure}

In this section, we present the numerical results of the convergence analysis for a numerical method applied to a square grid with a length of 1. The grid is discretized with $n=2^6$ grid points. The analysis is based on a manufactured solution, which is given by the expression:

\[
u_{\text{ex}}(x) = 100 \sin\left(\frac{m(2\pi)}{L}x_1\right)\cos\left(\frac{r(2\pi)}{L}x_2\right)
\]

Here, $L$, $m$, and $r$ are constants representing the length of the square grid, the spatial frequency in the $x_1$ direction, and the spatial frequency in the $x_2$ direction, respectively.

The convergence analysis involves evaluating the errors in different norms as well as studying the condition number of the system. Table \ref{table:CutFEM_error1} presents the results of the convergence analysis for the numerical method. The column labeled "$n$" corresponds to the number of grid points, while the columns labeled "$\Vert e \Vert_{L^2}$," "$\Vert e \Vert_{H^1}$," and "$\Vert e \Vert_{a_h,}$" represent the errors in the $L^2$, $H^1$, and energy ($\Vert e \Vert_{a_h,}$) norms, respectively. The column labeled "EOC" denotes the experimental order of convergence. The last two columns provide the condition number of the system and the number of degrees of freedom (ndofs), respectively.

Figure \ref{fig:CutFEM_error1} displays a plot illustrating the error convergence in the $L^2$, $H^1$, and energy norms for the CutCIP method (Laplace) with second-order accuracy.

Additionally, we compare the results with and without the ghost penalty in two different tests: a translation test and a convergence test. The translation test examines the evolution of the condition number and error number, as shown in Figure \ref{fig:combined}. The convergence test also investigates the condition number and error number, and its results are presented in Figure \ref{fig:combined}.

Overall, these numerical results provide insights into the performance and accuracy of the numerical method, showcasing the convergence behavior and the impact of the ghost penalty on the solution.





