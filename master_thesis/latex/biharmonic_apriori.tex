
\newpage
\subsection{A priori estimates}%
\label{sec:a_priori_estimates}


In this section will we do only a brief overview of the important assumptions and definitions required to prove optimal convergence.
Recall that for $v \in H^{1}( \mathcal{T } _{h}) $ these inequalities holds $\forall T \in \mathcal{T} _{h}$ s.t. \[
\begin{split}
    \| v \|_{ \partial T }^{  } &\lesssim h^{-\frac{1}{2}}_{T}\|  v \|_{ T }^{  }+ h^{\frac{1}{2}} \| \nabla v \|_{T  }^{   }  , \\
    \| v \|_{ \Gamma \cap T }^{  } &\lesssim  h^{-\frac{1}{2}} \| v \|_{T  }^{  }   + h^{\frac{1}{2}}_{T} \| \nabla v \|_{ T }^{  }.
\end{split}
\]
For proof, see \cite[Lemma 4.2]{hansbo2003finite}.

\subsubsection{Cléments interpolation}%
\label{ssub:clement_operator}

We want to compute the expected convergence rate of the energy norm \eqref{eq:bi_Ah_norm}. An important tool in the process is the Cléments interpolation operator, $C_{h}$.
It is used for interpolation on non smooth functions by applying an regularization on so-called macroelements. Let us denote the $\mathcal{P}_{c}^{k}( \Omega )  $ to be an $H^{1}$ conformal polynomial space. We denote $\left\{ a_{1}, \ldots, a_{N}
\right\} $ to be the Lagrange nodes. Associated with each node $a_{i}$ we denote the macroelement $A_{i}$ to consist of all simplices containing $a_{i}$. Let $n_{cf}$ be the number of configurations for the macroelement, then we define the index $j:
\left\{ 1,\ldots,N \right\} \to \left\{ 1, \ldots, n_{cf} \right\}  $ s.t. $j( i) $ is the index associated with the reference configuration $\widehat{A}_{j(i) }$ for corresponding macroelement $A_{i}$. Let us define a $C^{0}$-diffeomorphism $G_{A_{i}}:
\widehat{A}_{j( i) } \to A_{i}$ s.t. for all $\widehat{T} \in \widehat{A}_{j( i) } $ is the restriction $G_{A_{i}  \mid \widehat{T}}$ affine. The Cléments interpolation operator $C_{h}$ is defined as a $L^2$-projection onto the macroelements. That is, given
a reference macroelement $\widehat{A}_{j( i) }$ and a function $\hat{v} \in L^{1}( \widehat{A}_{j( i) })  $, then $\widehat{C}_{j( i) } \hat{v}$  is the unique polynomial in $\mathcal{P}^{k} ( \widehat{A}_{j( i) })  $ s.t. \[
\int_{  \widehat{A}_{j( i) }}^{} ( \widehat{C}_{j( i) } \hat{v} - \hat{v}) p \ dx  = 0 \quad  \forall p \in \mathcal{P}^{k} ( \widehat{A}_{j( i) })
\]
Finally, we define the Cléments interpolator $C_{h} : L^{1}( \Omega )  \to \mathcal{P} ^{k}_{c}(\Omega  ) $ s.t.
\[
C_{h} v = \sum_{i=1}^{N} \widehat{C}_{j( i) } ( v (G_{A_{i}}) (G^{-1}_{A_{i}}(a_{i})) )\phi _{i},
\]
where $\phi _{i}$ is the corresponding polynomial basis at node $a_{i}$.

Recall the integral norm notation,
\[
\| u \|_{ m,p,T }^{  } = \left( \sum_{ \left\lvert \alpha  \right\rvert \le m}^{} \int_{T}^{}  \left\lvert  \partial ^{\alpha } u \right\rvert^{p} dx   \right)^{\frac{1}{2}}
\]
where we use that $\| u \|_{ T  }^{  } = \| u \|_{ m,0,T  }^{  } $ and similarly $\| u \|_{ 2,T  }^{  } = \| u \|_{ m,2,T  }^{  }  $.

We denote a patch, $\omega \left( T \right) $, as the set of elements in $\mathcal{T} _{h}$  sharing at least one vertex with $T \in \mathcal{T} _{h}$ . And similarly we denote a another patch, $\omega \left( F \right) $, as the set of all elements in $\mathcal{T}_{h} $
sharing at least one vertex with $F \in  \mathcal{F} _{h}$. Finally, we have the following lemma

\begin{lemma}
    \label{lemma:clements}

We define the Clement interpolation as the projection
$C_{h}: H^{m} \left( \Omega  \right) \mapsto V_{h}$, where $V_{h}$ has the order $k$. Then does the following stability estimate hold,
\[
 \| C_{h} v \|_{H^{m}\left( \Omega  \right)   }^{  } \lesssim \| v \|_{ H^{m}\left( \Omega  \right)  }^{  } \quad \forall v \in H^{m}\left( \Omega  \right),
\]
and if the following conditions for an parameter $l$ is satisfied, it exists error estimates s.t.,
\[
    \begin{split}
      m\le l \le k+1  \implies \| v - C_{h} v \|_{ m,p,T   }^{  }  &  \lesssim h^{l-m}_{T} \| v \|_{l,p,\omega \left( T \right)  }^{  } \quad  \forall T \in \mathcal{T} _{h}, \forall v \in H^{l}( \omega \left( T \right)
      ), \\
      m +\frac{1}{2}\le l \le k+1  \implies \| v - C_{h} v \|_{ m,p,F }^{  } & \lesssim h^{l-m- \frac{1}{2}}_{T} \| v \|_{l,p,\omega \left( F \right)  }^{  } \quad  \forall \partial T \in \mathcal{T} _{h}, \forall v \in H^{l}( \omega \left( F
      \right)).
    \end{split}
\]

\end{lemma}

\todo[inline]{ Why first $C_{h}: L^{1} \to \mathcal{P} ^{k}$ and now
    $C_{h}: H^{m} \to \mathcal{P} ^{k}$?
}

\begin{corollary}
    \label{cor:celement_apriori}
    Let $0 \le l \le k+1$ and let $0\le m \le \min_{} ( 1,l )$.
    Given Lemma \ref{lemma:clements}  then there exists an $C > 0$ s.t.
    \[
    \inf_{v_{h} \in \mathcal{P} ^{k}_{c}( \Omega ) } \| v - v_{h} \|_{  m,p,\Omega }^{  } \le C h^{l-m}  \| v \|_{ l,p,\Omega  }^{  }    \forall v \in W_{l,p}( \Omega ).
    \]
\end{corollary}


This result is very useful since it is now sufficient to show that a priori estimates holds given to prove convergence rate.

For further detailed information, please investigate \cite[Chapter 1.6]{ern04}.
We will use these estimates to compute convergence rate given that Ceas' Lemma holds.


\subsubsection{Energy a priori estimates.  }%
\label{ssub:extension}

A key idea is to distinguish between the physical space $\Omega $ and the polyhedra consisting of the active mesh $\Omega ^{e}_{h} = \mathcal{T}_{h}$. Thus, to do an
a priori estimate we find it necessary to define an bounded extension operator satisfying, \[
( \cdot ) ^{e}: W^{m,q}( \Omega )  \to W^{m,q} ( \Omega ^{e}), \quad \| v^{e} \|_{ m,q,\Omega ^{e}  }^{  } \lesssim \| v \|_{ m,q, \Omega  }^{  }.
\]
where $0< m \le \infty$ and $1 \le q \le \infty$.
\todo[inline]{ The physical space $\Omega $ is a subset of $\Omega ^{e}$, so I do not understand why $\| v^{e} \|_{ m,q,\Omega ^{e}  }^{  } \lesssim \| v \|_{ m,q, \Omega  }^{  } $ should hold.}

Now assume that $\Omega _{h}^{e} \subset  \Omega^{e} $. We define an unfitted Cléments interpolator $C_{h}^{e}: H^{r}( \Omega ^{e}_{h}) \to V_{h}$
s.t.  $C ^{e} _{h} v := C _{h} v^{e} $.
We can immediately observe that the interpolation satisfies the global error estimates, that is,
\begin{align}
    \label{eq:bi_projection_estimates_1}
    \| v - C _{h}^{e} v \|_{  r, \mathcal{T} _{h} }^{  } & \lesssim h^{s-r}\sum_{T \in \mathcal{T}_h} \| v \|_{ s, \omega(T) }^{  }, \quad 0\le r\le s \\
    \label{eq:bi_projection_estimates_2}
\| v - C ^{e}_{h}v \|_{ r,\mathcal{F} _{h} }^{  } & \lesssim h^{s-r-\frac{1}{2}}\sum_{T \in \mathcal{T}_h} \| v \|_{ s, \omega(F)  }^{  }, \quad 0  \le  r \le   s- \frac{1}{2} \\
    \label{eq:bi_projection_estimates_3}
\| v - C ^{e}_{h}v \|_{ r, \Gamma }^{  } & \lesssim h^{s-r-\frac{1}{2}} \sum_{T \in \mathcal{T}_h}  \| v \|_{ s,  \omega(T)  }^{  }, \quad 0  \le  r \le   s- \frac{1}{2}
\end{align}

Naturally can we see this is the tools we need to construct an estimate for the energy norm.

\begin{lemma}
    \label{lemma:astar_estimate}
    Let $u \in H^{s}( \Omega ) $ for $s\ge 3$ be a solution of \eqref{eq:Bi_strong}. Then we have  \[
    \|  u - C_{h}u \|_{ a_{h},*  }^{  } \lesssim h^{s(s-2)} \| u \|_{ H^{s}( \Omega )  }^{  }
    \]

\end{lemma}
\begin{proof}
    By definition is
    \[
        \begin{split}
            \| u - C_{h}^{e}u \|_{ a_{h}, * }^{  2}  =& \ \| |\alpha |^{\frac{1}{2}} ( u - C_{h}^{e}u) \|_{ \mathcal{T} _{h} \cap \Omega  }^{ 2}  + \| D^2 ( u - C_{h}^{e}u ) \|_{\mathcal{T} _{h} \cap \Omega   }^{ 2 } \\  &  + \gamma \| h^{-\frac{1}{2}} \jump{ \partial _{n} (u -
        C_{h}^{e} u) }   \|_{ \mathcal{F}_{h}^{}\cap \Omega    }^{ 2
        } + \gamma \| h^{-\frac{1}{2}}  \partial _{n} (u - C_{h}^{e}u)    \|_{ \Gamma   }^{ 2 } \\
          & + \| h^{\frac{1}{2}} \mean{ \partial _{nn} (u - C_{h}^{e}u) }   \|_{\mathcal{F} _{h}^{} \cap \Omega   }^{  2} +  \| h^{\frac{1}{2}} \partial _{nn}(u - C_{h}^{e}u)     \|_{ \Gamma }^{  2}.
        \end{split}
    \]

    The strategy is to bound each term individually by applying the estimates \eqref{eq:bi_projection_estimates_1}, \eqref{eq:bi_projection_estimates_2} and \eqref{eq:bi_projection_estimates_2}.
    \begin{enumerate}[label=\arabic*)]
        \item     Starting with the first term we get
    \[
            \| |\alpha |^{\frac{1}{2}} ( u - C_{h}^{e}u) \|_{ \mathcal{T} _{h} \cap \Omega  }^{ 2}  \le  \| \alpha  \|_{L^\infty ( \Omega )    }^{2  }   \|  ( u - C_{h}^{e}u) \|_{0,\mathcal{T} _{h}  }^{ 2} \lesssim  h^{2s} \sum_{T \in \mathcal{T}
            _{h}}^{}   \| v \|_{s,w( T)  }^{  2}
    \]
    Here we simply used \eqref{eq:bi_projection_estimates_1}.
\item
    Similarly can we use  for the second term, \[
    \| D^2 ( u - C_{h}^{e}u ) \|_{\mathcal{T} _{h} \cap \Omega   }^{ 2 } \lesssim  \|  u - C_{h}^{e}u  \|_{2,\mathcal{T} _{h}   }^{ 2 } \lesssim \sum_{T \in \mathcal{T} _{h}}^{} h^{2(s-2)} \| u \|_{ s, \omega ( T)  }^{ 2 }.
    \]
    Again, this is via the estimate \eqref{eq:bi_projection_estimates_1}.
\item
        Recall $\| \jump{ u }   \|_{ \mathcal{F} _{h} }^{  } \le \| v^{+}   \|_{ \mathcal{F} _{h} }^{  } +   \| v^{-}   \|_{ \mathcal{F} _{h} }^{  } \lesssim  \| u \|_{ \partial\mathcal{T }_{h}  }^{2  }  $ and the inverse estimate $\| \partial _{n} u \|_{ F  }^{} \lesssim h^{-\frac{1}{2}} \| \nabla u \|_{ T }^{  }  $.
     Hence, by \eqref{eq:bi_projection_estimates_2}  \[
        \begin{split}
            \gamma \| h^{-\frac{1}{2}} \jump{ \partial _{n} ( u - C_{h}^{e}u ) }   \|_{ \mathcal{F}_{h} \cap \Omega   }^{  2}   &\lesssim \| h^{-\frac{1}{2}}  \partial _{n} ( u - C_{h}^{e}u )    \|_{  \partial \mathcal{T} _{h} }^{2  }
       \lesssim h^{-2} \|   \nabla  ( u - C_{h}^{e}u )    \|_{  \mathcal{T} _{h} }^{2  }     \\
                                                                                                                 &\lesssim h^{-2} \|      u - C_{h}^{e}u     \|_{1,  \mathcal{T}_{h} }^{2  }
                                                                                                                 \lesssim \sum_{T \in \mathcal{T} _{h}}^{} h^{2(s-1)} \| u \|_{ s,T  }^{ 2}
        \end{split}
    \]
    \todo[inline]{
        Not sure if \eqref{eq:bi_projection_estimates_2} holds here!
        \[
     \| \partial _{n}(u - C_{h}^{e}u)     \|_{1, \partial \mathcal{T}_{h} }^{2  }
                                                                                                                 \lesssim \sum_{T \in \mathcal{T}_{h} }^{}
                                                                                                                 h^{ 2(l - m - \frac{1}{2})  } \| u      \|_{ s, \omega (  F)  }^{  2}
    \]  }
\item
    And for the boundary term we simply apply \eqref{eq:bi_projection_estimates_3}, \[
        \begin{split}
            \gamma \| h^{-\frac{1}{2}}  \partial _{n} ( u - C_{h}^{e}u ) \|_{ \Gamma    }^{  2} \lesssim h^{-2} \|   \nabla  ( u - C_{h}^{e}u )    \|_{ \Gamma  }^{2  } \lesssim  h^{-2} \|    u - C_{h}^{e}u     \|_{1, \Gamma  }^{2  }\lesssim h^{2(s- r-
            \frac{1}{2}) -2}
              \sum_{T \in \mathcal{T} _{h}}^{}  \| u \|_{s,T  }^{  2}
        \end{split}
    \]
    \todo[inline]{ Does it makes sense to integrate $\| \nabla u \|_{ \Gamma  }^{  } $, i.e. no normal vectors? I guess not. Hence, \eqref{eq:bi_projection_estimates_3} is kinda strange.}
\item

            We also know that $\| \mean{ u }   \|_{ \mathcal{F} _{h} }^{  } \le \| u^{+} \|_{ \mathcal{F} _{h}  }^{  } + \| u^{-} \|_{ \mathcal{F} _{h}  }^{  }   \lesssim  \| u \|_{ \partial\mathcal{T }_{h}  }^{2  }  $ and the $\| \partial _{nn} u \|_{ F  }^{} \lesssim h^{-\frac{1}{2}} \| D^2 u \|_{ T }^{  }  $. It is clear
            by using \eqref{eq:bi_projection_estimates_2} that this must holds
            \[
                \begin{split}
 \| h^{\frac{1}{2}} \mean{ \partial _{nn} (u - C_{h}^{e}u) }   \|_{\mathcal{F} _{h}^{} \cap \Omega   }^{  2} &  \lesssim h^{} \|   \partial _{nn} (u - C_{h}^{e}u)    \|_{\partial \mathcal{T} _{h}   }^{  2}  \lesssim  \|   D^2 (u - C_{h}^{e}u)    \|_{ \mathcal{T} _{h}   }^{  2} \\
                                                                                                                &  = \|   u - C_{h}^{e}u    \|_{ 2, \mathcal{T} _{h}   }^{  2} \lesssim h^{2s} \sum_{T \in \mathcal{T}_h}^{} \| u \|_{s, \omega ( T)   }^{  }
                \end{split}
            \]
            \item
                Similarly we can easily see by using \eqref{eq:bi_projection_estimates_3} that this must hold,
                \[
              \| h^{\frac{1}{2}} \partial _{nn}(u - C_{h}^{e}u)     \|_{ \Gamma }^{  2} \lesssim   \|  D^2(u - C_{h}^{e}u)     \|_{ \Gamma  }^{  2} \lesssim \|  u - C_{h}^{e}u \|_{ 2, \Gamma}^{2  } \lesssim h^{2(s-r -\frac{1}{2})} \sum_{T \in
              \mathcal{T}_h }^{} \| u \|_{ s, \omega ( T)  }^{2  }
            \]
    \end{enumerate}

    Finally, combining all the estimates we have
    \[
        \begin{split}
            \| u - C_{h}^{e}u \|_{ a_{h}, * }^{  2}  \lesssim  & \   \sum_{T \in \mathcal{T}_h}^{}  \bigg(    h^{2s}\| u \|_{s, \omega ( T)   }^{  } +  h^{2(s-2)} \| u \|_{ s, \omega ( T)  }^{ 2 } \\
             & +  h^{2(s-1)} \| u \|_{s,T }^{ 2} + h^{2(s- r- \frac{1}{2}) -2}   \| u \|_{s,T  }^{  2} \\
            & + h^{2s}  \| u \|_{s, \omega ( T)   }^{  } + h^{2(s-r -\frac{1}{2})}  \| u \|_{ s, \omega ( T)  }^{2  } \bigg) \\
            \lesssim & \ h^{2s ( s-2) } \| u \|_{H^{s}( \Omega )   }^{2  }
        \end{split}
    \]
    Thus, the proof is complete.
    \todo[inline]{ Cleanup proof }
\end{proof}

\begin{lemma}[Weak galerkin orthogonality]
Let $u \in H^{s}( \Omega )  $, $ s\ge 3 $  be solution to \eqref{eq:Bi_strong} and $u_{h} \in V_{h}$ is a discrete solution to \eqref{eq:Bi_a_h}. Then is \[
    a_{h}( u - u_{h}, v) = g_{h} ( u_{h}, v) \quad \forall v \in V_{h}.
    \]
\end{lemma}

\begin{assumption}[EP2]
    \label{as:bi_EP2}
    For $v \in H^{s}( \Omega ) $ and $r = \min \{s,k+1 \} $, the semi-norm $\abs{ \cdot  }_{g_{h}} $ satisfies the following estimate, \[
    \abs{ C _{h}^{e} v } _{g_{h}} \lesssim  h^{r-1} \| v \|_{ r,\Omega  }^{  }.
    \]
\end{assumption}


\begin{theorem}
    Let $u \in H^{s}( \Omega ) $ , $s\ge 3$ be the solution to \eqref{eq:Bi_strong} and let $u \in V_{h}$ of order $k$ be the solution to \eqref{eq:Bi_a_h}. Then for $r = \min_{}\{s, k+1\} $ the error $e = u - u_{h}$ satisfies
    \begin{align}
        \label{eq:bi_apriori_1}
            \| e \|_{ A_{h},* }^{  } &\lesssim   h^{r-1} \| u \|_{ r,\Omega  }^{  }\\
        \label{eq:bi_apriori_2}
            \| e \|_{ \Omega  }^{  } &\lesssim   h^{r} \| u \|_{ r,\Omega  }^{  }
    \end{align}

\end{theorem}

\begin{proof}
    We will divide the proof into two steps.
    \begin{enumerate}[label=\arabic*)]
        \item We want to prove that $\| e \|_{ a_{h},* }^{  } \lesssim   h^{r-1} \| u \|_{ r,\Omega  }^{  }$.
    Let $e = u - u_{h}$ consist of $e = e_{h} + e_{\pi }$, where the discrete error has the form $e_{h} = C _{h}^{e} u - u_{h}$ and the projection error $e_{\pi } = u - C _{h} ^{e}u$. We can then observe that
    \[
    \| u - u_{h} \|_{ a_{h} }^{  }  \le \| e_{\pi } \|_{a_{h},*}^{  } + \| e_{h} \|_{A_{h}  }^{  }
    \]
Using Lemma \ref{lemma:astar_estimate}, can we see that $\| e_{\pi } \|_{a_{h},*}^{  } \lesssim h^{r-1} \| u \|_{ u,\Omega  }^{  }  $ is already fulfilled. So it remains to check the discrete part. From Lemma \ref{lemma:bi_Ah_coercive}, \ref{lemma:bi_Ah_bounded}, the weak Galerkin orthogonality and Assumption \ref{as:bi_EP2} is it natural to see that, \[
    \begin{split}
\| e_{h} \|_{ A_{h} }^{ 2 } & \lesssim a_{h}( C _{h}^{e} u - u_{h}, e_{h}) + g_{h}( C _{h}^{e}u - u_{h}, e_{h}) \\
 & = a_{h}( C _{h}^{e} u - u, e_{h}) + a_{h}( u - u_{h}, e_{h}) + g_{h}( C _{h}^{e}u - u_{h}, e_{h}) \\
 & = a_{h}( C _{h}^{e} u - u, e_{h}) + g_{h}( C _{h}^{e}u, e_{h}) \\
 & \lesssim h^{r-1} \| u \|_{ r, \Omega  }^{  } \| e_{h} \|_{ A_{h} }^{  }.
    \end{split}
\]
The last line of the calculations above comes from the fact that,
\[
    \begin{split}
        a_{h}( C _{h}^{e} u - u, e_{h}) + g_{h}( C _{h}^{e}u, e_{h}) &\lesssim \| C _{h}^{e} u - u \|_{a_{h},*  }^{  } \| e_{h} \|_{a_{h},*  }^{  }
        + \abs{ C _{h}^{e}u }_{g_{h}} \abs{e_{h}  }_{g_{h}} \\
         &\lesssim \| C _{h}^{e} u - u \|_{a_{h},*  }^{  } \| e_{h} \|_{a_{h},*  }^{  } + h^{r-1} \| e_{h} \|_{r, \Omega   }^{  }\abs{e_{h}  }_{g_{h}} \\
         &\lesssim (\| C _{h}^{e} u - u \|_{a_{h},*  }^{  } + h^{r-1} \| e_{h} \|_{r, \Omega   }^{  }) \|e_{h}\|_{A_{h}} \\
         &\lesssim  h^{r-1} \| u \|_{r, \Omega   }^{  } \|e_{h}\|_{A_{h}}.
    \end{split}
\]
Here we noticed that $\| e_{h} \|_{a_{h},*  }^{  } + \abs{e_{h}  }_{g_{h}} \lesssim \| e_{h} \|_{ A_{h} }^{  }  $. We also argued that $\| C _{h}^{e} u - u \|_{a_{h},*  }^{  } \lesssim h^{r-1}\| u \|_{ r,\Omega  }^{  }  $ from Lemma
\ref{lemma:astar_estimate}.
\todo[inline]{ TODO: Need to show $ \| e_{h} \|_{a_{h},*  }^{  } + |e_{h}  |_{g_{h}} \le  \| e_{h} \|_{ A_{h} }^{  }  $ }
Hence, the first part of the proof is complete.

\item Here is the second part.
    \end{enumerate}
\end{proof}



