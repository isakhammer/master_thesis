

\documentclass[11pt]{article}

%%%% DEPENDENCIES v1.8 %%%%%%

\usepackage{graphicx}
\usepackage[backend=bibtex,style=numeric,natbib=true,citestyle=numeric,sorting=none]{biblatex}
\synctex=1


\usepackage{amsthm}
\usepackage{amsfonts}
\usepackage{mathtools}
%\usepackage{enumerate}
\usepackage{enumitem}
\usepackage[section]{placeins}

\setlength {\marginparwidth }{2cm}
\usepackage{todonotes}
\usepackage{esint}
\usepackage{float}
%
%\usepackage{mathrsfs}

\usepackage{hyperref}
\hypersetup{
    colorlinks=true, %set true if you want colored links
    linktoc=all,     %set to all if you want both sections and subsections linked
    linkcolor=blue,  %choose some color if you want links to stand out
}
\hypersetup{linktocpage}

% inscape-figures
%\usepackage{import}
%\usepackage{pdfpages}
%\usepackage{transparent}
% \usepackage{xcolor}
%\newcommand{\incfig}[2][1]{%
%\def\svgwidth{#1\columnwidth}
%\import{./figures/}{#2.pdf_tex} } \pdfsuppresswarningpagegroup=1

% Box environment
\usepackage{tcolorbox}
\usepackage{mdframed}

% \newmdtheoremenv{definition}{Definition}[section]
\newmdtheoremenv{theorem}{Theorem}[section]
\newtheorem{definition}[theorem]{Definition}
\newtheorem{problem}[theorem]{Problem}
\newtheorem{assumption}{Assumption}
\newtheorem*{assumption*}{Assumption} % unnumbered
\newtheorem{corollary}[theorem]{Corollary}
\newtheorem{proposition}[theorem]{Proposition}
\newtheorem{lemma}[theorem]{Lemma}

\DeclareMathOperator{\atantwo}{atan2}
\DeclareMathOperator{\arctantwo}{arctan2}

\theoremstyle{remark}
\newtheorem*{remark}{Remark}
%\newtheorem{example}{Example}

\newcommand{\newpara}
    {
    \vskip 0.4cm
    }

\newcommand\Item[1][]{%
  \ifx\relax#1\relax  \item \else \item[#1] \fi
  \abovedisplayskip=0pt\abovedisplayshortskip=0pt~\vspace*{-\baselineskip}}

\newcommand{\mean}[1]{\left\{\!\!\left\{#1\right\}\!\!\right\}}
\newcommand{\jump}[1]{\left[\!\left[ #1 \right]\!\right]}
\newcommand{\abs}[1]{\left\lvert #1 \right\rvert}
\newcommand{\red}[1]{\textcolor{red}{ #1 }}
\let\oldleq\le
\renewcommand{\le}{\leqslant}
\let\oldgeq\ge
\renewcommand{\ge}{\geqslant}

\usepackage{geometry}
%\usepackage{showframe} %This line can be used to clearly show the new margins

\newgeometry{vmargin={15mm}, hmargin={15mm,37mm}}

\usepackage{amsmath}
\numberwithin{equation}{section}
\usepackage{amssymb}
\usepackage{subfig}

% \usepackage{subcaption} % cannot be used with subfig
%\subcaption -> \subfloat[Laplace]{\input{ex.tex}}
% \usepackage{newtxmath}
% \usepackage{mwe}


\addbibresource{bibliography.bib} % Makes the bibliography file available to biblatex.

%% PGF PLOTS DEPENDECIES
\usepackage{pgfplots}
\usetikzlibrary{calc}
\usepgfplotslibrary{patchplots}
\usepgfplotslibrary{fillbetween}
\usetikzlibrary{arrows.meta}
\usetikzlibrary{backgrounds}
\pgfplotsset{compat=newest} % Set the compatibility mode for pgfplots

\pgfplotsset{%
    layers/standard/.define layer set={%
        background,axis background,axis grid,axis ticks,axis lines,axis tick labels,pre main,main,axis descriptions,axis foreground%
    }{
        grid style={/pgfplots/on layer=axis grid},%
        tick style={/pgfplots/on layer=axis ticks},%
        axis line style={/pgfplots/on layer=axis lines},%
        label style={/pgfplots/on layer=axis descriptions},%
        legend style={/pgfplots/on layer=axis descriptions},%
        title style={/pgfplots/on layer=axis descriptions},%
        colorbar style={/pgfplots/on layer=axis descriptions},%
        ticklabel style={/pgfplots/on layer=axis tick labels},%
        axis background@ style={/pgfplots/on layer=axis background},%
        3d box foreground style={/pgfplots/on layer=axis foreground},%
    },
}
%%%

% \usepackage{amsmath} % Required for the \Vert command%%%%%%%%%%%%%%%%%%%%%%%%%%%%%%%%%%%%%%%%%%%%%%%%%%%%%%%%%%%%

%
\newgeometry{vmargin={15mm}, hmargin={25mm,27mm}}
\usepackage{pdfpages}

\begin{document}

    \includepdf[pages=-]{results/illustration/front_page.pdf}

    \newpage


    \tableofcontents
    % \section{Introduction}


\begin{frame}{Introducing Myself}
    \begin{columns}
        % Column 1
        \begin{column}{0.5\textwidth}
            \begin{itemize}
                \item Isak Hammer, 27 year old, Lofoten
                \item Graduate student in Industrial Mathematics
                \item Research Focus: Numerical methods for Partial Differential Equations (PDEs).
            \end{itemize}
        \end{column}

        % Column 2
        \begin{column}{0.5\textwidth}
            \begin{figure}
                \centering
                \includegraphics[width=0.7\textwidth]{figures/isak.jpg}
            \end{figure}
        \end{column}
    \end{columns}
\end{frame}

\begin{frame}{Importance and Motivation of the Cahn Hilliard Equation}
    \begin{columns}
        % Column 1
        \begin{column}{0.5\textwidth}
            \begin{itemize}
                \item Thermodynamically modelling of a two-component liquid separation\footnotemark[1].
            \item Modelling of so-called lipid rafts in biological membrane dynamics \footnotemark[2].
            \end{itemize}
        \end{column}
        \begin{column}{0.5\textwidth}
            \begin{itemize}
                \item Droplet dynamics, i.e., coalescence, breakup and movement by coupling with Navier-Stokes \footnotemark[3].
            \end{itemize}
        \end{column}
    \end{columns}
    \footnotetext[1]{\fullcite{cahn1959free}}
    \footnotetext[2]{\fullcite{yushutin2019computational}}
    \footnotetext[3]{\fullcite{zimmermann2019calculation}}
\end{frame}

\begin{frame}
    \begin{block}{The Cahn Hilliard Equation}
        The general Cahn Hilliard Equation  has the form $u( x, t): \Omega \times [0,T] \mapsto [-1,1]   $ s.t.
            \[
            \begin{split}
                 u_t+\Delta\left(\varepsilon \Delta u-\frac{1}{\varepsilon} f(u)\right)&=0 \quad \text{in } \Omega \\
\partial_n u=\partial_n \Delta u& =0 \quad \text{on } \Gamma  \\
 u & =u_0 \quad \text{on } \Omega
            \end{split}
            \]
where $f(s)=F^{\prime}(s)$ and $F(s)=\frac{1}{4}\left(s^2-1\right)^2$ and $\Omega \subset \mathbf{R}^d, d=2,3$, is a bounded domain.
\end{block}

\begin{block}{Challenges}
    \begin{enumerate}
        \item Highly nonlinear and stiff. Often practical applications require $\varepsilon \ll 1$.
        \item 4th order system.
        % \item Conservation of mass and the Neumann conditions conditions.
    \end{enumerate}
\end{block}

\end{frame}

\begin{frame}
    \begin{block}{Why Finite Element Method (FEM)}
        \begin{enumerate}
            \item \textbf{Robust mathematical framework}
            \item \textbf{Can easily handle complex geometries}
            \item \textbf{High flexibility of basis functions}
            \item \textbf{Other: } Supports adaptive refinements, easily adaptable to multi-physics problems ++ .
        \end{enumerate}
    \end{block}
\end{frame}

% \begin{frame}
%     \begin{block}{Strategy to solve the Cahn-Hilliard problem on smooth domains}
%         \begin{enumerate}
%             \item Solve the biharmonic problem $\Delta ^2 u = f$ for polygonal domains.
%             \item Modify the method to handle smooth domains.
%             \item Utilize the time integration to handle non-linearity.
%         \end{enumerate}
%     \end{block}
% \end{frame}

% \begin{frame}
%     \begin{block}{Strategy to solve the Cahn-Hilliard problem on smooth domains}
%         \begin{enumerate}
%             \item First find a suitable method to solve an 4th order PDE for polygonal domains.
%             \item Modify the problem formulation to take account for smooth domains.
%             \item Utilize the time integration to handle nonlinearity.
% \item \textcolor{red}{Make the time-steps small enough or implement adaptivity time schemes.}
%         \end{enumerate}
%     \end{block}
% \end{frame}


% \begin{frame}
% \begin{columns}
% \column{0.55\textwidth}
% \begin{block}{The Biharmonic Problem}
% Let $\Omega \subseteq \mathbb{R} ^d$ be a bounded domain with boundary $\Gamma $ . Let the biharmonic problem have the form s.t. $u:\Omega \mapsto \mathbb{R} $,
% \begin{equation}
% \label{eq:bi_problem}
% \begin{split}
% \Delta^2 u + \alpha u & = f( x) \quad \text{in } \Omega, \\
% \partial_{n} u & = g_{1} \quad \text{on } \Gamma , \\
% \partial_{n} \Delta u & = g_{2} \quad \text{on } \Gamma . \\
% \end{split}
% \end{equation}
% Here is $\Delta ^2 = \Delta \left( \Delta \right) $ the biharmonic operator. The functions $g_{1},g_{2}: \Omega \to \mathbb{R}$ are denoted as boundary conditions.
% \end{block}

% \column{0.45\textwidth}
% \begin{figure}[htpb!]
%     \centering
%     \begin{tikzpicture}
%         % Circle
%         \draw (0.0,0.0) circle (1.5cm);
%         \fill[blue!30] (0.0,0.0) circle (1.5cm);

%         \draw[->, line width=1.0pt] ({1.5*cos(65)}, { 1.5*sin(65) }) -- ({ 2.5*cos(65)  }, { 2.5*sin(65)  }) node[ above left] {$n $};
%         % \draw[->, line width=1.0pt] ({1.5*cos(45)}, { 1.5*sin(45) }) -- ({ 1.5*cos(45) - 1.5*sin(45) }, { 1.5*sin(45) + 1.5*cos(45) }) node[ above right] {$t $};

%         % Labels
%         \node[below right] at (0.5,0.5) {$\Omega$};
%         \node[below right] at (-1.5,-1.2) {$\Gamma$};
%     \end{tikzpicture}
%     \caption{ Illustration of the domain $\Omega $, the boundary $\Gamma $ and the normal vector $n$. }
%     \label{fig:domain_construction}
% \end{figure}
% \end{columns}

% \end{frame}

\begin{frame}
    \frametitle{The Biharmonic Problem (on a polygon)}
    \begin{columns}
        \column{0.55\textwidth}
        \begin{block}{}
            Let $\Omega \approx \Omega_{h} = \mathcal{T}_{h}$ be a bounded \textcolor{red}{polygonal} domain with boundary $\Gamma $ . Let the biharmonic problem have the form s.t. $u:\Omega \mapsto \mathbb{R} $,
            \begin{equation}
                \label{eq:bi_problem}
                \begin{split}
                    \Delta^2 u + \alpha u & = f( x) \quad \text{in } \Omega, \\
                    \partial_{n} u & = 0 \quad \text{on } \Gamma , \\
                    \partial_{n} \Delta u & = 0 \quad \text{on } \Gamma . \\
                \end{split}
            \end{equation}
            Here is $\Delta ^2 = \Delta \left( \Delta \right) $ the biharmonic operator.
        \end{block}

        \column{0.45\textwidth}
        \begin{figure}[htpb!]
            \centering
            \begin{tikzpicture}
                % FIGURE OF FITTED MESH
                % Boundary points
                \foreach \i in {0, 45, ..., 315} {
                    \coordinate (boundary-\i) at (\i:1.5cm);
                }
                % Interior points
                \coordinate (interior-1) at (0.75, 0);
                \coordinate (interior-2) at (-0.75, 0);
                \coordinate (interior-3) at (0, 0.75);
                \coordinate (interior-4) at (0, -0.75);

                % Create a cycle connecting all the boundary points
                \fill[blue!30] (boundary-0) -- (boundary-45) -- (boundary-90) -- (boundary-135) -- (boundary-180) -- (boundary-225) -- (boundary-270) -- (boundary-315) -- cycle;

                % Labels
                \node[below right] at (0.4,0.7) {$\Omega $};
                \node[below right] at (-1.5,1.7) {$\Gamma $};

                % Triangulation (manually specified)
                \draw[line width=0.1pt] (boundary-0) -- (boundary-45) -- (interior-1) -- cycle;
                \draw[line width=0.1pt] (boundary-45) -- (boundary-90) -- (interior-3) -- cycle;
                \draw[line width=0.1pt] (boundary-90) -- (boundary-135) -- (interior-3) -- cycle;
                \draw[line width=0.1pt] (boundary-135) -- (boundary-180) -- (interior-2) -- cycle;
                \draw[line width=0.1pt] (boundary-180) -- (boundary-225) -- (interior-2) -- cycle;
                \draw[line width=0.1pt] (boundary-225) -- (boundary-270) -- (interior-4) -- cycle;
                \draw[line width=0.1pt] (boundary-270) -- (boundary-315) -- (interior-4) -- cycle;
                \draw[line width=0.1pt] (boundary-315) -- (boundary-0) -- (interior-1) -- cycle;

                % Triangulation between interior points
                \draw[line width=0.1pt] (interior-1) -- (interior-2) -- (interior-3) -- cycle;
                \draw[line width=0.1pt] (interior-1) -- (interior-2) -- (interior-4) -- cycle;

                \draw[->, line width=1.0pt] ({1.4*cos(65)}, { 1.4*sin(65) }) -- ({ 2.4*cos(65)  }, { 2.4*sin(65)  }) node[ above left] {$n $};

            \end{tikzpicture}
            \caption{ Illustration of the mesh $\Omega_{h} $, the boundary $\Gamma $ and the normal vector $n$. }
            \label{fig:domain_construction}
        \end{figure}
    \end{columns}
\end{frame}



\begin{frame}
\frametitle{ $C^0$ Interior Penalty Method (CIP) for the Biharmonic Problem }

\begin{block}{}
The proposed numerical scheme is to find an  $w \in V_{h}$ .t.
\begin{equation*}
\label{eq:CP_A_F}
a_{h}( w, v )   = l_{h}( v) = ( f,v)_{\Omega } , \quad \forall v \in V_{h}  .
\end{equation*}
where
\begin{equation*}
\begin{split}
a_{h} \left( w, v \right)   =&
    \left( \alpha  w, v \right) _{\Omega }   +  \left( \Delta  w, \Delta v \right) _{\Omega } \\
 & +
  \left( \mean{  \Delta  w }, \jump{ \partial _{n }v} \right)_{\mathcal{F}_{h}}  +
 \left( \mean{ \Delta  v }, \jump{ \partial _{n}w }      \right)_{\mathcal{F}_{h}}  + \frac{\gamma }{h}  \left( \jump{ \partial _{n} w}, \jump{ \partial _{n} v   }   \right)_{\mathcal{F}_{h}}
 % l_{h}( v_{h}) & =  \left( f, v \right) _{\Omega }
\end{split}
\end{equation*}

Which is inspired from Brenner2012 \footnotemark[1]
\end{block}

\footnotetext[1]{\fullcite{brenner2012}}

\end{frame}


\begin{frame}
\frametitle{Cut Finite Element Method (CutFEM)}

\begin{block}{Unfitted mesh vs fitted mesh}
    CutFEM is a numerical method for solving partial differential equations (PDEs) using an unfitted mesh.

\begin{figure}
    \centering
    % First TikZ picture
    \begin{minipage}{0.45\textwidth}
        \centering
        \begin{tikzpicture}[scale=0.80]
            \draw[fill=blue!30] (0.2, 0.2) circle (1.5cm);
            % Background mesh
            \foreach \i in {-2.5, -1.5, ..., 2.5} {
                \draw[line width=0.1pt, shift={(-2.5,\i)}] (0,0) -- (5,0);
                \draw[line width=0.1pt, shift={(\i,-2.5)}] (0,0) -- (0,5);
            }
            % Labels
            \node[below right] at (0.4,0.5) {$\Omega $};
            \node[below right] at (-1.5,1.5) {$\Gamma $};
            % \draw[blue, thick] (-2.5, -2.5) rectangle (2.5, 2.5);
        \end{tikzpicture}
    \end{minipage}
    \hfill
    % Second TikZ picture
    \begin{minipage}{0.45\textwidth}
        \centering
        \begin{tikzpicture}[scale=1.0]
            % FIGURE OF UNFITTED MESH
            % Boundary points
            \foreach \i in {0, 45, ..., 315} {
                \coordinate (boundary-\i) at (\i:1.5cm);
            }
            % Interior points
            \coordinate (interior-1) at (0.75, 0);
            \coordinate (interior-2) at (-0.75, 0);
            \coordinate (interior-3) at (0, 0.75);
            \coordinate (interior-4) at (0, -0.75);

            % Create a cycle connecting all the boundary points
            \fill[blue!30] (boundary-0) -- (boundary-45) -- (boundary-90) -- (boundary-135) -- (boundary-180) -- (boundary-225) -- (boundary-270) -- (boundary-315) -- cycle;

            % Labels
            \node[below right] at (0.4,0.7) {$\Omega $};
            \node[below right] at (-1.5,1.7) {$\Gamma $};

            % Triangulation (manually specified)
            \draw[line width=0.1pt] (boundary-0) -- (boundary-45) -- (interior-1) -- cycle;
            \draw[line width=0.1pt] (boundary-45) -- (boundary-90) -- (interior-3) -- cycle;
            \draw[line width=0.1pt] (boundary-90) -- (boundary-135) -- (interior-3) -- cycle;
            \draw[line width=0.1pt] (boundary-135) -- (boundary-180) -- (interior-2) -- cycle;
            \draw[line width=0.1pt] (boundary-180) -- (boundary-225) -- (interior-2) -- cycle;
            \draw[line width=0.1pt] (boundary-225) -- (boundary-270) -- (interior-4) -- cycle;
            \draw[line width=0.1pt] (boundary-270) -- (boundary-315) -- (interior-4) -- cycle;
            \draw[line width=0.1pt] (boundary-315) -- (boundary-0) -- (interior-1) -- cycle;

            % Triangulation between interior points
            \draw[line width=0.1pt] (interior-1) -- (interior-2) -- (interior-3) -- cycle;
            \draw[line width=0.1pt] (interior-1) -- (interior-2) -- (interior-4) -- cycle;

            % \draw[blue, thick] (-2.5, -2.5) rectangle (2.5, 2.5);

        \end{tikzpicture}
    \end{minipage}


    % \caption{Mesh comparison: unfitted mesh (left) adheres to domain and boundary, while fitted mesh (right) employs a triangular mesh for polygonal approximation of the circular domain.}
    \label{fig:domain_mesh}
    \end{figure}
\end{block}

\end{frame}

\begin{frame}
    \frametitle{Cut Finite Element Method}
Background Mesh
    \begin{block}{}
        \begin{tikzpicture}[scale=1.0]

            \fill[yellow!30] (-2.5,2.5) -- (2.5,2.5) -- (2.5,-2.5) -- (-2.5,-2.5) -- cycle;

            \draw (0.1, 0.1) circle (1.5cm);
            % Background mesh
            \foreach \i in {-2.5, -2, ..., 2.5} {
                \draw[line width=0.1pt, shift={(-2.5,\i)}] (0,0) -- (5,0);
                \draw[line width=0.1pt, shift={(\i,-2.5)}] (0,0) -- (0,5);
            }


            % Labels
            \node[below right] at (2.5,2.5) {$\widetilde{\mathcal{T}}_{h}$};
        \end{tikzpicture}
    \end{block}
\end{frame}

\begin{frame}
    \frametitle{Cut Finite Element Method}
    Active Mesh
    \begin{block}{}
        \begin{tikzpicture}[scale=1.0]

            % POTENTIAL ACTIVE MESH
            \fill[orange!30] (2,2) -- (2,-1.5) --(-1.5,-1.5) -- (-1.5,2) -- cycle;

            % ELEMENTS WITH NO INTERSECTION
            % lower left
            \fill[white] (-1.5,-1.5) rectangle (-1.0,-1.0);
            \fill[white] (-1.5,2.0) rectangle (-1.0,1.5);
            \fill[white] (-1.0,2.0) rectangle (-0.5,1.5);
            \fill[white] (2,2) rectangle (1.5,1.5);
            \fill[white] (1.5,2) rectangle (1.0,1.5);
            \fill[white] (2,1.5) rectangle (1.5,1.0);
            \fill[white] (1.5,-1) rectangle (2,-1.5);
            \fill[white] (1.5,-0.5) rectangle (2,-1.0);

            % CUT ELEMENTS
            \fill[orange!30] (-0.5,2.0) rectangle (1.0,1.5);
            \fill[orange!30] (-1.5,1.5) rectangle (0.0,1.0);
            \fill[orange!30] (0.5,1.5) rectangle (1.5,1.0);
            \fill[orange!30] (-1.5,1.0) rectangle (-1.0,-1.0);
            \fill[orange!30] (-1.0,-0.5) rectangle (-0.5,-1.5);
            \fill[orange!30] (-0.5,-1.5) rectangle (1.5,-1.0);
            \fill[orange!30] (1.5,-1) rectangle (1.0,-0.0);
            \fill[orange!30] (1.5,-0.5) rectangle (2.0,1.0);
            \fill[orange!30] (1.0,0.5) rectangle (1.5,1.0);

            \draw (0.1, 0.1) circle (1.5cm);
            % Background mesh
            \foreach \i in {-2.5, -2, ..., 2.5} {
                \draw[line width=0.1pt, shift={(-2.5,\i)}] (0,0) -- (5,0);
                \draw[line width=0.1pt, shift={(\i,-2.5)}] (0,0) -- (0,5);
            }


            % Labels
            % \node[below right] at (2.5,2.5) {$\widetilde{\mathcal{T}}_{h}$};
            % \node[below right] at (0.4,0.5) {$\mathcal{T}_{int}$};
            \node[below right] at (-0.5,0.5) {$\mathcal{T}_{h }$};
        \end{tikzpicture}
    \end{block}
\end{frame}

\begin{frame}
    \frametitle{Cut Finite Element Method}
    Interior Mesh and Cut Cells
    \begin{block}{}
        \begin{tikzpicture}[scale=1]

            % POTENTIAL ACTIVE MESH
            \fill[blue!30] (2,2) -- (2,-1.5) --(-1.5,-1.5) -- (-1.5,2) -- cycle;

            % ELEMENTS WITH NO INTERSECTION
            % lower left
            \fill[white] (-1.5,-1.5) rectangle (-1.0,-1.0);
            \fill[white] (-1.5,2.0) rectangle (-1.0,1.5);
            \fill[white] (-1.0,2.0) rectangle (-0.5,1.5);
            \fill[white] (2,2) rectangle (1.5,1.5);
            \fill[white] (1.5,2) rectangle (1.0,1.5);
            \fill[white] (2,1.5) rectangle (1.5,1.0);
            \fill[white] (1.5,-1) rectangle (2,-1.5);
            \fill[white] (1.5,-0.5) rectangle (2,-1.0);

            % CUT ELEMENTS
            \fill[green!40] (-0.5,2.0) rectangle (1.0,1.5);
            \fill[green!40] (-1.5,1.5) rectangle (0.0,1.0);
            \fill[green!40] (0.5,1.5) rectangle (1.5,1.0);
            \fill[green!40] (-1.5,1.0) rectangle (-1.0,-1.0);
            \fill[green!40] (-1.0,-0.5) rectangle (-0.5,-1.5);
            \fill[green!40] (-0.5,-1.5) rectangle (1.5,-1.0);
            \fill[green!40] (1.5,-1) rectangle (1.0,-0.0);
            \fill[green!40] (1.5,-0.5) rectangle (2.0,1.0);
            \fill[green!40] (1.0,0.5) rectangle (1.5,1.0);

            \draw (0.1, 0.1) circle (1.5cm);
            % Background mesh
            \foreach \i in {-2.5, -2, ..., 2.5} {
                \draw[line width=0.1pt, shift={(-2.5,\i)}] (0,0) -- (5,0);
                \draw[line width=0.1pt, shift={(\i,-2.5)}] (0,0) -- (0,5);
            }


            % Labels
            \node[below right] at (0.4,0.5) {$\mathcal{T}_{int}$};
            \node[below right] at (-1.5,0.5) {$\mathcal{T}_{\Gamma }$};
        \end{tikzpicture}
    \end{block}
\end{frame}

\begin{frame}
\frametitle{Cut Finite Element Method (CutFEM)}

A recent and promising numerical technique for PDEs, has gained significant momentum in the past decade \footnotemark[1]\footnotemark[2].

\begin{block}{}
    \begin{itemize}
        \item  Complex domains and moving domains efficiently.
        \item Utilizing so-called ghost penalties to ensure well-posedness.
        % \item Dividing the computational domains into \textbf{background} , \textbf{active}, \textbf{cut}  and \textbf{interior} mesh.
    \end{itemize}
\end{block}
\footnotetext[1]{\fullcite{burman2015cutfem}}
\footnotetext[2]{\fullcite{gurkan2019stabilized}}

\end{frame}




% \begin{frame}
% \frametitle{ Cut $C^0$ Interior penalty method (CutCIP) }

% \begin{block}{}
% The discretized numerical problem is to solve $w \in V_{h}$ such that
% \begin{equation*}
% \label{eq:CP_A_F}
% A( w, v )  = a_{h}( w,v) + \textcolor{red}{g_{h}( w,v)}  = l_{h}( v), \quad \forall v \in V_{h}  .
% \end{equation*}

% Where the additional bilinear term $g_{h}( w,v) : V_{h} \times V_{h} \to  \mathbb{R} $ is the so-called \textbf{ghost penalty}, which does the numerical regularization to ensure stability on cut cells.

% \end{block}

% \end{frame}

\begin{frame}
\frametitle{ CutCIP Method}

My master's thesis is dedicated to demonstrating that the relevant properties remain valid for CutCIP formulation still holds.

    \begin{block}{Well-posedness }
         The discrete bilinear form $a_{h}$ is wellposed on $V_{h}$ if this holds; \[
             \begin{split}
                 (Coercivity) \quad  A( v,v) &  \gtrsim  \| v \|_{A }^{ 2 } \quad  \forall v \in  V_{h} \\
            (Boundedness) \quad A( v,w) & \lesssim  \| v \|_{A }^{  }\| w \|_{a_{h} }^{  } \quad  \forall v,w \in  V_{h}
             \end{split}
        \]
    \end{block}

    \begin{block}{Apriori Estimates }

         Let $u$ be the solution to the strong problem with the corresponding discrete solution $u_{h}$ with polynomical order $k\ge 2$ .
        Then does it exist  $l = \min_{} ( 2, k)  $ s.t.
\[
        \| u - u_{h} \|_{a_{h}  }^{  } \lesssim  h^{l-1} \| u \|_{ H^{l} ( \Omega ) }^{  }
\]
    \end{block}
    % \footnotetext[1]{\fullcite{Gu2012}}
\end{frame}



\begin{frame}
\frametitle{ Cut $C^0$ Interior Penalty Method (CutCIP) Results }
% \frametitle{Manufactured Solution}


\begin{block}{Manufactured solution}
    In the experiments will we only consider polynomial order $k=2$.
We consider the manufactured solution:
$$
u_{ex}(\mathbf{x}) = \left(x_1^2 + x_2^2 - 1\right)^2 \cos(2\pi x_1) \cos(2\pi x_2)
$$
where $\mathbf{x}=(x_1,x_2)$ and $\Omega=\{(x_1,x_2): x_1^2 + x_2^2 \le  1\}$.
This manufactured solution can be used to test the accuracy of numerical methods for solving the above differential equation.
\end{block}
\end{frame}


\begin{frame}
\frametitle{ Cut $C^0$ Interior penalty method (CutCIP) Results }
\resizebox{\textwidth}{!}{
\begin{tabular}{rrrrrrrrr}
    \hline\hline
    \textbf{$n$} & \textbf{$\Vert e \Vert_{L^2}$} & \textbf{EOC} & \textbf{$ \Vert e \Vert_{H^1}$} & \textbf{EOC} & \textbf{$\Vert e \Vert_{ a_h,* }$} & \textbf{EOC} & \textbf{Cond number} & \textbf{ndofs} \\\hline
    4 & 2.4E+00 &  & 3.3E+00 &  & 6.2E+01 &  & 8.7E+04 & 8.1E+01 \\
    8 & 3.6E-01 & 2.72 & 1.1E+00 & 1.60 & 3.9E+01 & 0.68 & 5.1E+05 & 2.4E+02 \\
    16 & 2.2E-02 & 4.06 & 2.5E-01 & 2.12 & 1.4E+01 & 1.51 & 3.7E+06 & 8.3E+02 \\
    32 & 5.6E-03 & 1.97 & 6.0E-02 & 2.04 & 3.6E+00 & 1.93 & 2.8E+07 & 3.0E+03 \\
    64 & 1.4E-03 & 2.00 & 1.5E-02 & 2.02 & 9.2E-01 & 1.96 & 2.1E+08 & 1.1E+04 \\
    128 & 3.5E-04 & 2.00 & 3.7E-03 & 2.01 & 2.4E-01 & 1.94 & 1.7E+09 & 4.3E+04 \\\hline\hline
  \end{tabular}
}
\end{frame}






\begin{frame}
\frametitle{ Cut $C^0$ Interior penalty method (CutCIP) Results }

\begin{figure}[h!]
    \centering
    \resizebox{0.6\textwidth}{!}{% Recommended preamble:
% \usetikzlibrary{arrows.meta}
% \usetikzlibrary{backgrounds}
% \usepgfplotslibrary{patchplots}
% \usepgfplotslibrary{fillbetween}
% \pgfplotsset{%
%     layers/standard/.define layer set={%
%         background,axis background,axis grid,axis ticks,axis lines,axis tick labels,pre main,main,axis descriptions,axis foreground%
%     }{
%         grid style={/pgfplots/on layer=axis grid},%
%         tick style={/pgfplots/on layer=axis ticks},%
%         axis line style={/pgfplots/on layer=axis lines},%
%         label style={/pgfplots/on layer=axis descriptions},%
%         legend style={/pgfplots/on layer=axis descriptions},%
%         title style={/pgfplots/on layer=axis descriptions},%
%         colorbar style={/pgfplots/on layer=axis descriptions},%
%         ticklabel style={/pgfplots/on layer=axis tick labels},%
%         axis background@ style={/pgfplots/on layer=axis background},%
%         3d box foreground style={/pgfplots/on layer=axis foreground},%
%     },
% }

\begin{tikzpicture}[/tikz/background rectangle/.style={fill={rgb,1:red,1.0;green,1.0;blue,1.0}, fill opacity={1.0}, draw opacity={1.0}}, show background rectangle]
\begin{axis}[point meta max={nan}, point meta min={nan}, legend cell align={left}, legend columns={1}, title={}, title style={at={{(0.5,1)}}, anchor={south}, font={{\fontsize{14 pt}{18.2 pt}\selectfont}}, color={rgb,1:red,0.0;green,0.0;blue,0.0}, draw opacity={1.0}, rotate={0.0}, align={center}}, legend style={color={rgb,1:red,0.0;green,0.0;blue,0.0}, draw opacity={1.0}, line width={1}, solid, fill={rgb,1:red,1.0;green,1.0;blue,1.0}, fill opacity={1.0}, text opacity={1.0}, font={{\fontsize{8 pt}{10.4 pt}\selectfont}}, text={rgb,1:red,0.0;green,0.0;blue,0.0}, cells={anchor={center}}, at={(1.02, 1)}, anchor={north west}}, axis background/.style={fill={rgb,1:red,1.0;green,1.0;blue,1.0}, opacity={1.0}}, anchor={north west}, xshift={1.0mm}, yshift={-1.0mm}, width={94.6mm}, height={74.2mm}, scaled x ticks={false}, xlabel={$\delta$}, x tick style={color={rgb,1:red,0.0;green,0.0;blue,0.0}, opacity={1.0}}, x tick label style={color={rgb,1:red,0.0;green,0.0;blue,0.0}, opacity={1.0}, rotate={0}}, xlabel style={at={(ticklabel cs:0.5)}, anchor=near ticklabel, at={{(ticklabel cs:0.5)}}, anchor={near ticklabel}, font={{\fontsize{11 pt}{14.3 pt}\selectfont}}, color={rgb,1:red,0.0;green,0.0;blue,0.0}, draw opacity={1.0}, rotate={0.0}}, xmajorgrids={true}, xmin={-0.001471665988344504}, xmax={0.05052719893316125}, xticklabels={{$0.00$,$0.01$,$0.02$,$0.03$,$0.04$,$0.05$}}, xtick={{0.0,0.010000000000000002,0.020000000000000004,0.030000000000000006,0.04000000000000001,0.05000000000000001}}, xtick align={inside}, xticklabel style={font={{\fontsize{8 pt}{10.4 pt}\selectfont}}, color={rgb,1:red,0.0;green,0.0;blue,0.0}, draw opacity={1.0}, rotate={0.0}}, x grid style={color={rgb,1:red,0.0;green,0.0;blue,0.0}, draw opacity={0.1}, line width={0.5}, solid}, axis x line*={left}, x axis line style={color={rgb,1:red,0.0;green,0.0;blue,0.0}, draw opacity={1.0}, line width={1}, solid}, scaled y ticks={false}, ylabel={$\kappa(A)$}, y tick style={color={rgb,1:red,0.0;green,0.0;blue,0.0}, opacity={1.0}}, y tick label style={color={rgb,1:red,0.0;green,0.0;blue,0.0}, opacity={1.0}, rotate={0}}, ylabel style={at={(ticklabel cs:0.5)}, anchor=near ticklabel, at={{(ticklabel cs:0.5)}}, anchor={near ticklabel}, font={{\fontsize{11 pt}{14.3 pt}\selectfont}}, color={rgb,1:red,0.0;green,0.0;blue,0.0}, draw opacity={1.0}, rotate={0.0}}, ymode={log}, log basis y={10}, ymajorgrids={true}, ymin={100000.0}, ymax={1.0e25}, yticklabels={{$10^{5}$,$10^{10}$,$10^{15}$,$10^{20}$,$10^{25}$}}, ytick={{100000.0,1.0e10,1.0e15,1.0e20,1.0e25}}, ytick align={inside}, yticklabel style={font={{\fontsize{8 pt}{10.4 pt}\selectfont}}, color={rgb,1:red,0.0;green,0.0;blue,0.0}, draw opacity={1.0}, rotate={0.0}}, y grid style={color={rgb,1:red,0.0;green,0.0;blue,0.0}, draw opacity={0.1}, line width={0.5}, solid}, axis y line*={left}, y axis line style={color={rgb,1:red,0.0;green,0.0;blue,0.0}, draw opacity={1.0}, line width={1}, solid}, colorbar={false}]
    [\addlegendimage{empty legend}] \addlegendentry[font={{\fontsize{11 pt}{14.3 pt}\selectfont}}, text={rgb,1:red,0.0;green,0.0;blue,0.0}] {\hspace{-.6cm}{\textbf{$(\gamma, \gamma_1, \gamma_2)$}}}
    \addplot[color={rgb,1:red,0.0;green,0.0;blue,1.0}, name path={e73f70a5-bbbd-4a24-8407-eef9d08bc09d}, draw opacity={1.0}, line width={1}, solid]
        table[row sep={\\}]
        {
            \\
            0.0  1.2679164207221107e8  \\
            0.012263883236204186  1.2718633484176141e8  \\
            0.024527766472408372  1.2693944128424127e8  \\
            0.03679164970861256  1.2718633338545807e8  \\
            0.049055532944816745  1.2679164340263365e8  \\
        }
        ;
    \addlegendentry { $1.0 \cdot 10^{1}$, $0.5 \cdot 10^{1}$, $1.0 \cdot 10^{-1}$ }
    \addplot[color={rgb,1:red,1.0;green,0.0;blue,0.0}, name path={8dbacc37-3690-4b1f-9d5d-1b6f1fc900ac}, draw opacity={1.0}, line width={1}, solid]
        table[row sep={\\}]
        {
            \\
            0.0  4.639960110962716e10  \\
            0.012263883236204186  4.932252358128007e12  \\
            0.024527766472408372  4.0959531093983247e12  \\
            0.03679164970861256  4.932252223523245e12  \\
            0.049055532944816745  4.639960117645048e10  \\
        }
        ;
    \addlegendentry { $1.0 \cdot 10^{1}$, $ 0.0 \cdot 10^{0} $, $ 0.0 \cdot 10^{0} $ }
\end{axis}
\end{tikzpicture}
}
    \caption{The plot presents the $L^2$ and $H^1$ error norms and the error in the energy norm ($\Vert e \Vert_{a_h,*}$).}
    \label{fig:CutFEM_error1}
\end{figure}

\end{frame}

\begin{frame}
\frametitle{The Cahn Hilliard Equation}
    \begin{columns}
        % Left column
        \begin{column}{0.5\textwidth}
            \begin{block}{Recall}
                The problem has the form $u( x, t): \Omega \times [0,T] \mapsto [-1,1]$ s.t.
                \[
                \begin{split}
                    u_t+\Delta\left(\varepsilon \Delta u-\frac{1}{\varepsilon} f(u)\right)&=0 \quad \text{in } \Omega \\
                    \partial_n u=\partial_n \Delta u& =0 \quad \text{on } \Gamma  \\
                    u & =u_0 \quad \text{on } \Omega
                \end{split}
                \]
                where $f(u)$ is a nonlinear function.
            \end{block}
        \end{column}
        % Right column
        \begin{column}{0.5\textwidth}
            \begin{block}{Plan forward}
                \begin{enumerate}
                    \item We have now a tool to solve the $\Delta ( \Delta u) $ operator
                    \item Will utilize the time-iteration scheme to solve non-linearity
                \end{enumerate}
            \end{block}
        \end{column}
    \end{columns}
\end{frame}


\begin{frame}
\frametitle{The CutCIP Cahn-Hilliard Formulation}


Drawing upon the concepts delineated in Feng\footnotemark[1], the most efficient approach to address the nonlinear term is by employing an implicit-explicit (IMEX) scheme.

\begin{block}{IMEX method on the CutCIP formulation}
    Let $u^{m}_{h} \in V_{h}$ for the timesteps $m=0,1,\ldots,M$. Let $u_{h}^{0} = u_{0}$ be the initial timestep, then is.
\[
(\overline{\partial}  _{t} u_{h}^m, v_h ) + \varepsilon A (u_{h}^{m}, v_h )+\frac{1}{\varepsilon} c_h ( u_{h}^{m-1}, v_h)=0 \quad \forall v_h \in V_h^m .
\]
Here is $c_{h}( . , .) $ an the nonlinear terms handled in a implicit fashion. The $ \overline{\partial}  _{t}$ operator is simply a finite difference scheme in time-dimension.

\end{block}

\footnotetext[1]{\fullcite{feng2007fully}}
\end{frame}

\begin{frame}
\frametitle{The CutCIP Cahn-Hilliard Experiments}
Implemented using the Gridap FEM framework written in Julia \footnotemark[1].


\begin{block}{Simulation parameters}
    \begin{itemize}
        \item Physical domain $\Omega$  is a 4 discs of radius $R=1$ with distance $d=0.999$, i.e. they are touching!
        \item Inital data is $u_0 = random(-1,1)$ in physical domain $\Omega $.
        % \item Backgroundmesh with size ($L\times L$ ) for $L=2$ and $n=128$.
        % \item Polynomial order $k=2$ .
        % \item Physical parameter $\varepsilon  = \frac{1}{30}$.
        % \item Time-step $\tau = \varepsilon ^{2} \frac{1}{10} $ for the interval $ 0\le t \le 10^3 \tau $.
    \end{itemize}
\end{block}

\footnotetext[1]{\fullcite{badia2020gridap}}


\end{frame}

\begin{frame}
\frametitle{The CutCIP Cahn-Hilliard Experiments}

\begin{figure}[h]
    \centering
    \includegraphics[width=0.5\textwidth]{CH-example/0.png}
    \caption{Iteration 0}
    \label{fig:your_image_label}
\end{figure}
\end{frame}

\begin{frame}
\frametitle{The CutCIP Cahn-Hilliard Experiments}

\begin{figure}[h]
    \centering
    \includegraphics[width=0.5\textwidth]{CH-example/1.png}
    \caption{Iteration 1}
\end{figure}
\end{frame}

\begin{frame}
\frametitle{The CutCIP Cahn-Hilliard Experiments}
\begin{figure}[h]
    \centering
    \includegraphics[width=0.5\textwidth]{CH-example/10.png}
    \caption{Iteration 10}
\end{figure}
\end{frame}

\begin{frame}
\frametitle{The CutCIP Cahn-Hilliard Experiments}
\begin{figure}[h]
    \centering
    \includegraphics[width=0.5\textwidth]{CH-example/50.png}
    \caption{Iteration 50}
\end{figure}
\end{frame}

\begin{frame}
\frametitle{The CutCIP Cahn-Hilliard Experiments}
\begin{figure}[h]
    \centering
    \includegraphics[width=0.5\textwidth]{CH-example/200.png}
    \caption{Iteration 200}
\end{figure}
\end{frame}

\begin{frame}
\frametitle{The CutCIP Cahn-Hilliard Experiments}
\begin{figure}[h]
    \centering
    \includegraphics[width=0.5\textwidth]{CH-example/1000.png}
    \caption{Iteration 1000}
\end{figure}
\end{frame}


\begin{frame}
\frametitle{Further work}
\begin{enumerate}
    \item Adaptive time steps.
    \item Further numerical validation.
    \item Extend the method to handle moving domains.

\end{enumerate}
\end{frame}


\begin{frame}
\frametitle{Questions?}
\end{frame}




\newpage

\section{Introduction}\label{sec:introduction}


The first application of the Cahn-Hilliard (CH) problem appeared when modelling phase separation of two-component incompressible fluids \cite{cahn1958free, cahn1959free, cahn1961spinodal}, but was quickly generalized to handle multi-component system
as well \cite{bosch2015fractional, eyre1993systems, toth2016phase, miranville2017cahn}. In engineering, CH is the critical component in
the phase-field model, a mathematical framework to model transitions and interface dynamics in materials and fluid dynamics \cite{steinbach2009phase, chen2002phase}.
From this, the equation found many interesting applications for a wide variety of problems. To mention a few, we have
multiphase fluid dynamical problems \cite{badalassi2003computation, li2016lattice, kim2012phase, shen2010phase}, solidification of binary or multi-component alloys \cite{kim1999phase, echebarria2004quantitative}, and continuum modelling of fracture dynamics in
materials \cite{kuhn2010continuum, li2015phase}. Perhaps an unexpected application is that CH can be used for in painting when recovering damaged parts of an image \cite{bertozzi2006inpainting, burger2009cahn, bosch2015fractional, brkic2020image}
and modelling the origin of the irregular structure in Saturn's rings \cite{tremaine2003origin}.
CH is also essential in many areas of biology and medicine. For example, from a macroscopic viewpoint, CH has been used to model tumour growth, wound healing and brain tumours \cite{agosti2017cahn, cristini2009nonlinear}.
On the microscopic level on the biomembrane, there is an ongoing debate about the existence of the accumulation of lipids into so-called lipid rafts, which serve as a rigid platform for proteins with
special properties such as signalling and intercellular trafficking \cite{ levental2020lipid, hancock2006lipid, munro2003lipid, simons1997functional}. It turns out that the hypothesis can be tested by modelling the problem as a separation problem using
CH \cite{miller2020divide, garcke2016coupled, yushutin2019computational}.

\subsection{The physical Cahn Hilliard problem}%
\label{sub:the_equations}

The CH problem comes in many variants depending on its application, but we will in this report focus on the binary mixture version \cite{miranville2017cahn}. Let $\Omega \subset  \mathbb{R} ^{d} $ be a compact domain for $d=2,3$ with a sufficiently smooth boundary
$\Gamma $,  see Figure \ref{fig:domain_construction}. We define the time duration parameter $T \in  [0,\infty) $ and the
so-called unknown phase-field function as the
mapping $u: \left[ 0, T \right] \times \Omega  \to \left[ -1,1 \right]  $, which denotes the local difference of a binary mixture of two concentrations $c_{A}, c_{B} \in \left[ 0,1\right] $, i.e. $u = c_{A} -c_{B}$ and $c_{A} + c_{B} = 1$. Note that
if a local point exists so $u$ attains the extreme value $+1$, then it implies that the particular point has $100\%$ concentration $c_{A}$ and vice-versa for $c_{B}$. On the other hand, if $u$ is zero, it implies that the mixture is $50\% - 50\%$.

\begin{figure}[htpb!]
    \centering
    \begin{tikzpicture}
        % Define points for square
        \coordinate (A) at (-1.5,-1.5);
        \coordinate (B) at (1.0,0.0);
        \coordinate (C) at (1.5,1.5);
        \coordinate (D) at (-1.5,1.5);

        \filldraw[fill=blue!30,draw=black] (A) to[out=0,in=-90] (B) to[out=90,in=0] (C) to[out=180,in=90] (D) to[out=-90,in=180] cycle;
\draw[->, line width=1.0pt] ({1.8*cos(90)}, { 1.8*sin(90) }) -- ({ 2.5*cos(85)  }, { 2.5*sin(85)  }) node[below left, xshift=-0.2cm] {$n $};

        \node at (0,0) {$\Omega$};
        \node at (2,1.7) {$\Gamma$};

    \end{tikzpicture}
    \caption{Illustration of the physical domain $\Omega$ for $d=2$ , the boundary $\Gamma$ and the corresponding normal vector $n$.}
    \label{fig:domain_construction}
\end{figure}

Let $\varepsilon \ll 1 $ be a small parameter. For an isotropic
binary mixture non-uniform, the standard Ginzburg-Landay free energy functional is given by
\begin{equation}
E( u)  = \int_{\Omega }^{} \frac{\varepsilon^2 }{2} \abs{ \nabla u } ^2 +  F( u) dx.
\end{equation}
The nonlinear function $F( u) $ denotes the (Helmholtz) free energy density associated with the interaction dynamics between the components and thus comes in many forms depending on the thermodynamic properties, see \cite{miranville2017cahn}.
However, we will in this article assume that $F( u) = ( 1 / 4 ) ( u^2 -1 ) ^{2} $.
The chemical potential $\mu $ is defined as the variational derivative,
\begin{equation}
\mu = \frac{ \delta E( u) }{ \delta  u} = f( u)  - \varepsilon^2  \Delta u ,
\end{equation}
where we used the notation $f( u) = F'( u) =u( u^2 -1)  $.
First of all, to guarantee local mass conservation, we may enforce the continuity equation; that is,
\begin{equation}
    \partial _{t} u + \nabla \cdot \mathcal{J}  = 0,
\end{equation}
where $\mathcal{J} $ denotes the flux governed by the physical dynamics. Hence, this naturally leads to the no-flux and the Neumann boundary conditions,
\begin{align}
\label{eq:bc1}
\mathcal{J}  \cdot n & = 0 \text{ on } \Gamma, \\
\label{eq:bc2}
\partial _{n} u & = 0 \text{ on } \Gamma.
\end{align}
 A well-accepted law for the flux is that it is proportional to the chemical energy gradient, $\mathcal{J} = - M  \nabla \mu  $ for a parameter $M$. For the simplicity we assume $M=1$. This implies that we can rewrite the boundary condition
 \eqref{eq:bc1} such that
 \begin{equation}
     \mathcal{J}  \cdot n  = \partial _{n} \left(  \varepsilon^2 \Delta u - f( u)   \right)  = \varepsilon^2  \partial _{n} \Delta u -  f' ( u) \partial _{n}u =  \varepsilon^2  \partial _{n} \Delta u    ,
 \end{equation}
 for $u$ evaluated on $\Gamma $. Here we used that $\partial _{n} f( u) = f' ( u) \partial _{n}u  = 0$ from the boundary condition \eqref{eq:bc2}.
 Hence, we now have an equivalent set of boundary conditions and can finally write the strong form of the Cahn-Hilliard equation. Let $ u( x,0) =  u_{0}$, then is the dynamics of $u$ governed by,

\begin{equation}
\label{eq:strongch}
    \begin{split}
\partial _{t} u + \Delta  \left(  \varepsilon^2  \Delta u - f( u) \right)   &=0  \quad \text{ in } \Omega  \\
\partial _{n} u &= 0 \quad \text{ on } \Gamma  \\
\partial _{n}    \Delta u       &= 0 \quad \text{ on } \Gamma  \\
    \end{split}
\end{equation}

In this report, we will refer to this formulation as the Physical CH formulation due to the physical characteristics it embodies.
Based on these laws and the boundary conditions, it becomes evident that the energy functional serves as a Lyapunov function in the sense that its time derivative is monotonically decreasing and that the global mass concentration is conserved, i.e.
\begin{equation}
\label{eq:mass_cons_energy_decrease}
\frac{d}{dt} E( u) \le   0 \text{ and }\frac{d}{dt} \int_{\Omega }^{}  u dx = 0.
\end{equation}
Note that the inequality computation utilizes the assumption of $M$ to be constant, and both equations require the no-flux boundary condition $\mathcal{J} \cdot n = 0$.
For details, see \cite[Equation 17]{lee2014physical} and \cite[Equation 1.7]{garcke2020weak}.
This is useful since we expect $E( u( \cdot , t_{2}) ) \le  E( u( \cdot , t_{1}) ) $ for $0 < t_{1} < t_{2} $ and that the global mass is conserved,
\begin{equation}
    \int_{\Omega }^{} u ( x,t)  dx = \int_{\Omega }^{} u_{0}(x)  dx.
\end{equation}
These properties serve as a theoretical foundation for establishing the existence, uniqueness, and long-term behaviour of the CH problem. Consequently, these properties are well-comprehended from a mathematical standpoint. For
references, see \cite{abels2007convergence, cherfils2011cahn, elliott1986cahn}.

\subsection{Numerical methods}%
\label{sub:numerical_methods}

One of the key challenges with the CH problem is that it involves fourth-order spatial derivatives. It has, for simple domains, successfully been implemented using Finite Difference Methods \cite{furihata2001stable,
cheng2019energy} and Spectral Methods \cite{liu2003phase, he2009class}. However, these methods are generally constrained to simple domains (with some notable exceptions \cite{li2013conservative, shen2009efficient, feng2009fourier}).

As a further evolution to address the CH problem, it is common to consider a corresponding biharmonic (BH) problem as a numerical testbed in the spatial discretization schemes. This problem is defined as follows: Find $u:\Omega \to \mathbb{R}   $
such that
\begin{equation}
\label{eq:BH-problem}
\begin{split}
    \alpha u + \Delta ^2 u  & = f( x) \quad \text{in }  \Omega, \\
    \partial _{n} u & = g_{1}( x)  \quad \text{on } \Gamma,   \\
    \partial _{n} \Delta  u & = g_{2}( x)  \quad \text{on } \Gamma,   \\
\end{split}
\end{equation}
for given $f,g_{1} ,g_{2}: \Omega  \to \mathbb{R} $ and constant $\alpha >0  $.
The BH problem holds relevance since it is provide a proper spatial-discretization test framework prior to moving on solving the nonlinearities and timeintegration.

The early Finite Element Methods (FEM) for CH were proposed in \cite{elliott1987numerical, elliott1986cahn} utilizing global $C^{1}$ and $C^{2}$ in one spatial dimension, but later it has been shown that making $C^{1}$ (or higher order) elements in
multiple space dimensions is far from being trivial. For
reference, see \cite{kapl2021family, percell1976cubic, argyris1968tuba}.

% Describe the isogeometric analysis and virtual elements methods as alternatives.
There exist several promising alternative methods that guarantee $C^{1}$ continuity, and these have shown potential for solving the CH problem. A notable mention is isogeometric analysis (IGA), a technique that leverages Non-Uniform
Rational B-Splines (NURBS) to efficiently handle complex geometries and smooth boundaries without the need for translating CAD-based geometry description into classical FEM meshes. Thus, IGA presents a desirable alternative for problems dealing with intricate and smooth domains
\cite{hughes2005isogeometric}. Specifically for the CH problem, has IGA successfully been implemented \cite{kastner2016isogeometric, gomez2008isogeometric}.
Another rising method is the virtual finite element method (VFEM), which has applied so-called virtual $C^{1}$ elements to handle the $C^{1}$ continuity requirement \cite{antonietti2016c}.

\begin{figure}[h!]
    \centering
    % First TikZ picture
    \begin{minipage}{0.45\textwidth}
        \centering
        \begin{tikzpicture}
            % Draw x-axis
            \draw[->, thick] (0,0) -- (4,0) node[right] {$x$};
            % Draw y-axis
            \draw[->, thick] (0,0) -- (0,3) node[above] {$y$};

            % Draw nodes at center, A, B, and C with arbitrary y-values
            \foreach \x/\y in {0/2.5, 1/1.7, 2/2.5, 3/1.2} {
                \fill (\x,\y) circle (2pt); % Filled nodes with y-values
                \fill (\x,0) circle (2pt); % Filled circles on x-axis
            }

            % Exact solution
            % \draw[red, thick] (0,2.5) .. controls (0.5,0.5) and (0.5,3) .. (1,1.7) .. controls (1.5,0) and (1.5,6) .. (2,2.5) .. controls (2.5,1) and (2.5,3.5) .. (3,1.2);
            \node at (1,0.6) [above] {$\jump{ \partial _{n} u_{h} } \neq 0$};
            \node at (1,0.7) [below] {$\jump{ u_{h} }   = 0$};

            %% Second
            \draw[blue, thick] (0,2.5) .. controls (0.5,1.2) .. (1,1.7)
            .. controls (1.7,3.9) .. (2,2.5)
            .. controls (2.7,1.9) .. (3,1.2); Second

            \draw (3.2,3.2) rectangle (4.8,3.8); % Legend box
            \draw[blue, thick] (3.4,3.5) -- (3.9,3.5); % Blue line for u_h
            % \draw[red, thick] (3.4,3.1) -- (3.9,3.1); % Red line for u
            \node[anchor=west] at (4.0,3.5) {$u_h$}; % Label for u_h
            % \node[anchor=west] at (4.0,3.1) {$u$}; % Label for u

        \end{tikzpicture}
    \end{minipage}
    \hfill
    % Second TikZ picture
    \begin{minipage}{0.45\textwidth}
        \centering
    \begin{tikzpicture}
        % Draw x-axis
        \draw[->, thick] (0,0) -- (4,0) node[right] {$x$};
        % Draw y-axis
        \draw[->, thick] (0,0) -- (0,3) node[above] {$y$};

        % Draw nodes at center, A, B, and C with arbitrary y-values
        \foreach \x/\y in {0/2.5, 1/1.7, 2/2.5, 3/1.2} {
            \fill (\x,\y) circle (2pt); % Filled nodes with y-values
            \fill (\x,0) circle (2pt); % Filled circles on x-axis
        }

        % Exact solution
        % \draw[red, thick] (0,2.5) .. controls (0.5,0.5) and (0.5,3) .. (1,1.7) .. controls (1.5,0) and (1.5,6) .. (2,2.5) .. controls (2.5,1) and (2.5,3.5) .. (3,1.2);
        \node at (1,0.6) [above] {$\jump{ \partial _{n} u_{h} } = 0$};
        \node at (1,0.7) [below] {$\jump{ u_{h} }   = 0$};

        % Second
        \draw[blue, thick]
        (0,2.5) .. controls (0.25,2.1) .. (1,1.7)
        (1,1.7) .. controls (1.5,1.7) .. (2,2.5)
        (2,2.5) .. controls (2.5,3.3) .. (3,1.2);

        \draw (3.2,3.2) rectangle (4.8,3.8); % Legend box
        \draw[blue, thick] (3.4,3.5) -- (3.9,3.5); % Blue line for u_h
        % \draw[red, thick] (3.4,3.1) -- (3.9,3.1); % Red line for u
        \node[anchor=west] at (4.0,3.5) {$u_h$}; % Label for u_h
        % \node[anchor=west] at (4.0,3.1) {$u$}; % Label for u
    \end{tikzpicture}
    \end{minipage}
    \caption{ Illustration of global $C^{0}$ continuous elements (left) and global $C^{1}$ elements (right) in 1 dimension. Here, $u_{h}$ is a discrete solution
    % in $\mathcal{P}^2$
, where the jump between the elements is denoted as $ \jump{ u_{h} } = u_{+} - u_{-}$. }
\label{fig:global_C0}
\end{figure}

An alternative approach is to avoid global $C^1$ continuity and weakend it to global $C^{0}$ continuity, see Figure \ref{fig:global_C0} . As a result, this strategy has led to the development of two distinct families of methods for solving the CH problem.
% Paragraph of CIP
The first involves the Continuous Interior Penalty (CIP) methods, which uses the standard weak formulation but penalizes the discontinuity of the derivative between elements as a form of regularization. The method has been designed for several interesting
stable variants for the BH problem, that is \cite{brenner2012, brenner2012quadratic, brenner2012quadratic_kirk, mu2014weak, georgoulis2009discontinuous}, and recently also for tri-harmonic problems \cite{diegel2023c0}.
This method is advantageous due to its symmetry and relationship with discontinuous Galerkin (DG) methods \cite{di2011mathematical}, renowned for their natural way of handling inhomogeneous boundary conditions, flexibility with unstructured meshes, efficient parallelization, and strong stability. This connection lends robust stability analysis tools, making the method highly suitable for intricate computational problems.
The CIP formulation has also then been adapted to solve CH by applying the Newton-Raphson scheme to handle the non-linearities \cite{wells2006discontinuous} or utilizing an implicit-explicit (IMEX) time integration scheme, where the stiff part is treated implicitly (such as backward Euler) and the nonlinear part explicitly (such as the forward
Euler or explicit Runge-Kutta) \cite{ feng2007fully}.

Another popular variant is to rewrite the BH problem as a system of second-order problems in a mixed formulation. This strategy not only broadens the problem's flexibility but also provides a more natural means to incorporate boundary conditions,
see \cite{falk1978approximation,
ciarlet1974mixed, gudi2008mixed, cheng2000some}. This approach also leverages the general saddle point theory for mixed
FEM methods, which provide a mathematical framework to ensure numerical stability \cite{john2016finite}.
Moreover, this approach adapts well to the CH problem \cite{wells2006discontinuous,feng2004error, valseth2021stable}, and some methods even apply a so-called convex splitting scheme approach in a way that preserves the convexity of the energy functional, making the
system easier to solve \cite{diegel2015analysis, brenner2018robust}.
A combination of these methods, the DG and mixed formulation for the CH problem, has recently been considered \cite{chave2016hybrid, medina2022stabilized}.

\begin{figure}[h!]
    \centering
    % First TikZ picture
    \begin{minipage}{0.45\textwidth}
        \centering
        \begin{tikzpicture}
            \draw[fill=blue!30] (0.2, 0.2) circle (1.5cm);
            % Background mesh
            \foreach \i in {-2.5, -1.5, ..., 2.5} {
                \draw[line width=0.1pt, shift={(-2.5,\i)}] (0,0) -- (5,0);
                \draw[line width=0.1pt, shift={(\i,-2.5)}] (0,0) -- (0,5);
            }
            % Labels
            \node[below right] at (0.4,0.5) {$\Omega $};
            \node[below right] at (-1.5,1.5) {$\Gamma $};
            % \draw[blue, thick] (-2.5, -2.5) rectangle (2.5, 2.5);
        \end{tikzpicture}
    \end{minipage}
    \hfill
    % Second TikZ picture
    \begin{minipage}{0.45\textwidth}
        \centering
        \begin{tikzpicture}
            % FIGURE OF FITTED MESH
            % Boundary points
            \foreach \i in {0, 45, ..., 315} {
                \coordinate (boundary-\i) at (\i:1.5cm);
            }
            % Interior points
            \coordinate (interior-1) at (0.75, 0);
            \coordinate (interior-2) at (-0.75, 0);
            \coordinate (interior-3) at (0, 0.75);
            \coordinate (interior-4) at (0, -0.75);

            % Create a cycle connecting all the boundary points
            \fill[blue!30] (boundary-0) -- (boundary-45) -- (boundary-90) -- (boundary-135) -- (boundary-180) -- (boundary-225) -- (boundary-270) -- (boundary-315) -- cycle;

            % Labels
            \node[below right] at (0.4,0.7) {$\Omega $};
            \node[below right] at (-1.5,1.7) {$\Gamma $};

            % Triangulation (manually specified)
            \draw[line width=0.1pt] (boundary-0) -- (boundary-45) -- (interior-1) -- cycle;
            \draw[line width=0.1pt] (boundary-45) -- (boundary-90) -- (interior-3) -- cycle;
            \draw[line width=0.1pt] (boundary-90) -- (boundary-135) -- (interior-3) -- cycle;
            \draw[line width=0.1pt] (boundary-135) -- (boundary-180) -- (interior-2) -- cycle;
            \draw[line width=0.1pt] (boundary-180) -- (boundary-225) -- (interior-2) -- cycle;
            \draw[line width=0.1pt] (boundary-225) -- (boundary-270) -- (interior-4) -- cycle;
            \draw[line width=0.1pt] (boundary-270) -- (boundary-315) -- (interior-4) -- cycle;
            \draw[line width=0.1pt] (boundary-315) -- (boundary-0) -- (interior-1) -- cycle;

            % Triangulation between interior points
            \draw[line width=0.1pt] (interior-1) -- (interior-2) -- (interior-3) -- cycle;
            \draw[line width=0.1pt] (interior-1) -- (interior-2) -- (interior-4) -- cycle;

            % \draw[blue, thick] (-2.5, -2.5) rectangle (2.5, 2.5);

        \end{tikzpicture}
    \end{minipage}
\caption{Mesh comparison: unfitted mesh (left) adheres to domain and boundary, while fitted mesh (right) employs a triangular mesh for polygonal approximation of the circular domain.}
\label{fig:domain_mesh}
\end{figure}

Creating a high-quality mesh in two and three dimensions for realistic problems is a challenging task that can consume a significant amount of time in the simulation workflow. It is also difficult to scale properly on distributed platforms, making it less suitable for moving domains, highly complex meshes, or smooth boundaries. An interesting class to approach the problem is the so-called unfitted finite element method, which utilizes a background mesh and does not align with the physical boundary. For an illustration
see Figure \ref{fig:domain_mesh}.
This greatly reduces the need to generate an unstructured mesh and makes it very applicable for parallelization and moving domains
since it avoids the need of remeshing entirely. However, without paying attention to the so-called cut cells, which are the elements intersecting with the boundary, the method quickly leads to instability and ill-conditioning.
One of the methods to counter this is the cut finite element method (CutFEM), where the focus is to penalize the cut-cells weakly by adding an additional ghost penalty term to ensure well-posedness and optimal convergence properties \cite{burman2015cutfem}. Notably, there exist related methods, sometimes considered equivalent, named Extended FEM and Trace FEM \cite{cai2021nitsche,zonca2018unfitted}.
This method has been successfully implemented for the BH problem for the
mixed formulation \cite{cai2023nitsche}, the CIP formulation \cite{chen2023arbitrary, cai2021nitsche} or using so-called $C^{1}$ continuous Bogner-Fox-Schmit elements \cite{burman2020cut}.
However, both implementations are considering an interface problem between two domains. Specifically for the CH problem, the mixed formulation \cite{karatzas2021reduced} has been shown to be successful.
 Aggregated unfitted finite element method (AgFEM) is a close relative to CutFEM and has also shown to
be promising \cite{badia2018aggregated, badia2022linking}. The method is an alternative way to the ghost penalty, which instead applies a so-called cell aggregation with respect to a cut cell ( assuming each cell has enough support with interior elements) and,
thus, the badly cut cells are removed, ensuring robustness and well-posedness.
Recent results have shown that investigations on unfitted versions of IGA
\cite{zhao2017variational} and its applicability to moving surfaces \cite{zimmermann2019isogeometric} is also possible.



\subsection{Outline of the report}%
\label{sub:outline_of_the_report}
In this thesis, we propose an novel stabilized unfitted cut continuous interior penalty method (CutCIP) specifically for the BH problem with Cahn-Hilliard type boundary conditions, which incorporates the CutFEM methodology in combination with a CIP
formulation. We then extend this method to a CIP CutFEM solver for CH, giving the first CIP CutFEM formulation applied for the problem.
Our approach is inspired by the theoretical
procedure as presented in the DG Poisson formulation proposed by \cite{gurkan2019stabilized}, but instead apply the CIP BH formulation while taking into account and extending the to analytical results provided in \cite{feng2007fully, brenner2012quadratic}.

The structure of this thesis is as follows. Firstly, in Section \ref{sec:mathematical_background}, we aim to establish notation by reviewing necessary mathematical tools. In Section \ref{sec:CIP_biharmonic_problem} the basic construction of the CIP
BH formulations and the related properties is reviewed. Subsequently, in Section \ref{sec:cutcip_biharmonic_problem} we propose the corresponding CutCIP BH method and provide a theoretical analysis showing that the stability and convergence
properties from the original CIP method are conserved. To verify these theoretical properties, we also presented numerical experiments as validation. Finally, in Section \ref{sec:cahn_hilliar_applications} we demonstrate how to extend the method to handle the CH problem.




\section{Mathematical Background}%
\label{sec:mathematical_background}

In this section, we revisit the established definitions and provide a brief overview of Sobolev spaces. Afterward, we briefly review the finite element method and discuss the necessary tools required for calculating a priori estimates. We will
generally follow the formulations presented in \cite{pietro2012, ErnGuermond2021}.


\subsection{Sobolev spaces}%
\label{sub:notation}

We will in this report assume $\Omega $ to be a compact and open set in $\mathbb{R} ^{d}$. Let $p \in \mathbb{R} $, $ 1 \le  p \le  \infty$, and  define the space $L^{p}\left( \Omega  \right) $ to be the set of all measurable functions $u: \Omega  \mapsto \mathbb{R} $ such that
$\left\lvert f \right\rvert ^{p}$ is Lebesgue integrable, i.e,

\begin{equation*}
    L^{p}\left( \Omega  \right) = \left\{ u: \Omega \mapsto \mathbb{R}  \mid \int_{\Omega }^{} \left\lvert u \right\rvert ^{p} d \Omega  < \infty  \right\}
.\end{equation*}
Let $u \in L^{p}\left( \Omega  \right) $. We define the integral norm of order $p$ to be \[
\| u \|_{ L^{p}\left( \Omega  \right)  }^{  }  = \left( \int_{\Omega }^{} \left\lvert u \right\rvert ^{p} dx  \right) ^{\frac{1}{p}}.
\]

The following definition for derivatives is employed.
For $d$ dimensions of order $k$ we define the multi-index $\alpha  = ( \alpha _{1}, \ldots, \alpha _{d})  $ with the absolute value $\abs{ \alpha  } = \sum_{i=1}^{d}  \alpha _{i} = k $ such that
\begin{equation}
    \label{eq:der}
\partial ^{\alpha} u = \frac{\partial ^{ \alpha_{1}  }  } {\partial^{} x_{1}^{\alpha _{1}}  } \ldots \frac{\partial ^{ \alpha_{d}  }  } {\partial^{} x_{d}^{\alpha _{d}}  } u \quad  \text{for }u \in C^{\left\lvert \alpha  \right\rvert }( \Omega )
\end{equation}

  Let $k\ge 0$ be an integer and let $1 \le  p <  \infty$ be a real number, then the Sobolev space $W^{k,p}( \Omega ) $ is defined by
  \begin{equation}
W^{k,p}\left( \Omega  \right) = \left\{ u \in L^{p}\left( \Omega  \right)  \mid  \partial ^{\alpha } u \in L^{p}\left( \Omega  \right)  \forall \alpha : \left\lvert \alpha  \right\rvert  \le k \right\}.
  \end{equation}
with the corresponding norm
\begin{equation}
\| u \|_{ W^{k,p}\left( \Omega  \right)  }^{  }  = \left(   \sum_{j = 0}^{k}  \left\lvert u \right\rvert ^{p} _{  W^{j,p}\left( \Omega  \right) } \right)^{\frac{1}{p}} .
\end{equation}
Here the seminorm is defined such that $ \left\lvert u \right\rvert _{W^{k,p}( \Omega  ) }^{} =  ( \sum_{\left\lvert \alpha  \right\rvert  = k}^{} \| \partial ^{\alpha }u \|_{ L^{p}( \Omega )   }^{ p } )^{\frac{1}{p}} $.

For shorthand notation, we denote
\begin{equation}
    \begin{split}
\| \cdot \|_{k,p}^{  } & = \| \cdot  \|_{k,p,\Omega   }^{  } = \| \cdot  \|_{ W^{k,p}( \Omega )   }^{  }, \\
   \abs{ \  \cdot \ }_{k,p}^{  } &= \abs{ \   \cdot \ }_{k,p,\Omega   }^{  } = \abs{ \  \cdot \ }_{ W^{k,p}( \Omega )   }^{  },
    \end{split}
\end{equation}

Since $p=2$ is frequently used in this report, we also define for convenience a compact notation $\| u \|_{ \Omega  }^{  }  = \| u \|_{ L^{2}\left( \Omega  \right)  }^{  } $ and $H^{m}( \Omega ) = W^{m,2}( \Omega )  $ such that,
\begin{equation}
\begin{split}
\| \cdot  \|_{H^{m}( \Omega )   }^{  } & = \|   \cdot  \|_{k,\Omega   }^{  } = \|   \cdot  \|_{k,2,\Omega   }^{  },  \\
\abs{ \   \cdot \  }_{H^{m}( \Omega )   }^{  } & = \abs{ \   \cdot \  } _{k,\Omega   }^{  } =  \abs{ \ \cdot \   } _{k,2,\Omega   }^{  }.
\end{split}
\end{equation}

Recall that $L^{2}( \Omega ) $ and $H^{m}( \Omega ) $ are Hilbert spaces equipped with the inner products
\begin{equation}
    \begin{split}
\left( u,v \right) _{\Omega } & = \left( u,v \right) _{L^2\left( \Omega  \right) } = \int_{\Omega }^{} u  v dx, \quad \forall u,v \in L^{2}( \Omega )  \\
    \left( u,v \right) _{m,\Omega } & = \sum_{\left\lvert \alpha  \right\rvert  \le  m}^{}  ( \partial ^{\alpha } u ,  \partial ^{\alpha } v )_{\Omega }, \quad  \forall u,v \in H^{m}( \Omega  ).
    \end{split}
\end{equation}

 let $u,v \in H^{2}(\Omega)$ be scalar functions. The following are the definition for their corresponding operator inner products.

\begin{equation}
\begin{split}
( \nabla v, \nabla u)_{\Omega } & = \int_{\Omega } \nabla v \cdot \nabla u \ dx \\
( \Delta v, \Delta u)_{\Omega } & = \int_{\Omega } \Delta v \Delta u \ dx \\
( D^2 v, D^2 u)_{\Omega } & = \int_{\Omega } D^2 v : D^2 u \ dx \\
\end{split}
\end{equation}
Also, $ \|  \nabla v\|_{  \Omega}^{2} = ( \nabla v, \nabla v)_{\Omega}$, $ \|   \Delta v \|_{\Omega}^{2} = (\Delta v, \Delta v)_{\Omega}$, and $ \| D^2 v \|_{ \Omega}^{2} = (D^2 v, D^2 v)_{\Omega}$.
Here, $\nabla v \cdot \nabla u$ and $D^2 v : D^2 u$ represent the inner product of the gradients and the Frobenius inner product \footnote{
The Frobenius inner product for for two tensors $A, B \in \mathbb{R} ^{\overbrace{n \times n \times \ldots \times n}^{r \text{ times }} } $ is defined such that $ A:B = \sum_{1\le  i_{1}, \ldots, i_{r} \le  d}^{} A_{i_{1} \ldots i_{r}} B_{i_{1}
    \ldots
i_{r}}  $.
} of Hessian matrices, respectively.

Let $v \in H^{r}( \Omega ) $, we define $D^r v$ to be a tensor of order $r$ such that
\begin{equation}
    \label{eq:tensor}
\left[ D^r v \right] _{i_{1} \ldots i_{r}} = \frac{\partial^r v}{\partial x_{ i_{1} } \ldots \partial x_{ i_{r} } } \quad  \forall i_{1}, \ldots, i_{r} \in \left\{ 1,\ldots,d \right\}
\end{equation}
where the norm $\| D^r v\|_{\Omega   }^{ 2 } =
\int_{\Omega }^{} D^r v : D^r v \ dx$ is defined via the standard Frobenius inner product. Observe that this notations holds such that $D^{0} v = v$, $ D^{1} v = \nabla v $ and $D ^{2} v  = J(\nabla v) = \mathrm{Hess}(v)$, where $J$ is the Jacobian operator .

Given the context of Sobolev spaces, we consider the functions $ u \in H^{2}( \Omega )$ and $ v \in H^{1}( \Omega )$. We can denote Greens theorem, which links integrals over a volume and its boundary, as follows,
\[
( \Delta u, v) _{\Omega } = -( \nabla u, \nabla v)_{\Omega } + ( u, \partial _{n}v)_{\Gamma }
\]
This identity serves as an essential tool for the calculations done.



\subsection{Computational Domains}%
\label{sub:computational_domain}
Assume that $\Omega \subset \mathbb{R} ^{d} $ is an open and bounded domain with a boundary $\Gamma $. In standard FEM methods a key assumption is that the set $\Omega $ is a polyhedra. This is useful since a polyhedra can be fully covered by a collection of polyhedra and, hence, motivating us to define a fitted mesh.
We define a fitted mesh $\mathcal{T} $ of the domain $\Omega $ to be a collection of closed polyhedra $\left\{ T \right\}  $ with disjoint interior forming a partition of $\Omega $ such that $\overline{\Omega } = \bigcup _{T \in \mathcal{T} } T $, for illustration see Figure
\ref{fig:domain_mesh}.
Here we say that each $T \in  \mathcal{T} $ is a mesh element or an element.
The mesh size is defined as the maximum diameter $h := h_{max} $ of any polyhedra in the mesh $\mathcal{T} = \left\{ T \right\}  $, that is, $ h_{max} = \max_{T \in \mathcal{T} }  h_{T}$ s.t.
$h _{T}  = \mathrm{diam}\left( T \right)   = \mathrm{max}_{x_1, x_{2} \in T} \ \mathrm{ dist }(x_{1}, x_{2})$
Hence, motivating us to use the notation $\mathcal{T} _{h}$ for a mesh $\mathcal{T} $ with size $h$.

For simplicity we restrict ourself to simplicial and quadrilateral elements.
A mesh $\mathcal{T}_{h}$ in $ \mathbb{R} ^{d}$ is said to be matching if
for all neighbouring elements $T_{1}, T_{2} \in \mathcal{T} _{h}$ such that the intersection is non-empty, $T_{1} \cap T_{2} \neq \emptyset  $, then $T_{1}\cap T_{2}$ is for $d=2$ either a common vertex, edge, and for $d=3$ a common vertex, edge or a face.

Let the chunkiness parameter $c_{T} := h_{T}/r_{T}$, where $r_{T}$  is the largest ball that be inscribed inside a element $T \in \mathcal{T}_{h} $.
A mesh is said to be shape regular if $c_{T}\le  c$ is independent of $T$  and $h$. We also say that the mesh is quasi-uniform only if it is shape regular and $h_{\mathrm{ max }} \le  c h_{\mathrm{ min }}$.
For a more complete description of meshes, see \cite[Chapter 8]{ErnGuermond2021}.

In this thesis will we assume that a mesh $\mathcal{T}_{h} $ is matching, shape regular and quasi-uniform unless specified.
 The fact that the mesh is conform makes is a useful property since the interface between mesh elements has come into contact in the sense
that it is either a vertex or a facet. This with the combination of shape regularity and quasi-uniformity is a major key to prove important inequalities in broken Sobolev spaces \cite[Chapter 1.4.1]{pietro2012}. Hence, the assumptions are very handy when proving convergence.


Let $\mathcal{T}_{h}  = \left\{ T \right\} $ be a mesh of $\Omega \subset  \mathbb{R} ^d $ consisting of polygons $T \in \mathbb{R} ^{d}$.
The set of all facets is the union of external and internal facets, $\mathcal{F} _{h} = \mathcal{F} ^{ext}_{h} \cup \mathcal{F} _{h}^{int} $, where each are defined by
\[
            \mathcal{F}^{int} _{h}  = \left\{ F=T^{+}\cap T^{-}  \mid  T^{+}, T^{-} \in \mathcal{T}_{h}  \right\} \text{ and }
            \mathcal{F}^{ext} _{h}  = \left\{ F= \partial T \cap \partial \Omega    \mid  T  \in \mathcal{T}_{h}  \right\}.
\]
Assume $T^{+} \neq T^{-}$ . Next, we define the following normal vectors.
\begin{enumerate}[label=\arabic*)]
    \item We define $ n= n  _{\partial T}$ to be unit outward normal on $\partial T$ for each $T \in \mathcal{T}_{h} $
\item Let $F \in \mathcal{F }^{int} _{h}$. We define $n$ to be the facet normal $ n =  n _F = n | _{\partial T^{+}} $  from $T^{+}$ to $T^{-}$, illustrated in Figure \ref{fig:normal}.
 \item Let $F \in \mathcal{F} ^{ext}_{h}$. Then we define the facet normal $n | _{F} = n | _{\partial T} $ to be the unit outward normal.
\end{enumerate}
Please note that we for convenience employ the notation $n$ when it is clear what entity the normal is associated with.

    Let $v\in L^2( \Omega ) $ be a scalar function on $\Omega$ with a corresponding shape regular and quasi-uniform mesh $\mathcal{T}_{h} $. We will use the following definitions.
    \begin{enumerate}[label=\arabic*)]
        \item Let $F \in \mathcal{F}^{int} _{h}$ and $v^{\pm}| _{F} = \lim_{t\to 0^{+}} v( x \mp tn)   $ for $x \in F$. We define the mean as $\mean{ v} |_{F} = \frac{1}{2} (v^{+}_{F} + v^{-}_{F})   $ and the jump as $\jump{v}|_{F} =  v^{+}_{F} - v^{-}_{F} $.
        \item Let $F \in \mathcal{F}^{ext} _{h}$ and let $ v( x) =  v(x)|_{F} $ for  $x \in F$.
We define the mean as $\mean{ v} |_{F} = v    $ and the jump as $\jump{v}|_{F} = v$.

    \end{enumerate}
    To simplify will we use the notation $\mean{ v } = \mean{ v }|_{F}    $ and $\jump{ v } = \jump{ v }| _{F}    $ for all $F \in \mathcal{F} _{h}$.
    Remark that if we have two functions $u,v$, for which $u^{\pm}( x) $ and $v^{\pm}( x) $ are defined, then the following identity holds $  \jump{ uv }    = \jump{ u }   \mean{ v }    + \mean{ u }  \jump{ v }$ along all facets $ \mathcal{F}_{h} $ associated with the
    triangulation $\mathcal{T} _{h}$.


\begin{figure}[h!]
\centering
\begin{tikzpicture}[scale=1]
\coordinate (A) at (-1.5, -0);
\coordinate (C) at (0,3);
\coordinate (B) at (4,0);
\coordinate (D) at (4,3.5);

\draw (A) -- (B) -- (C) -- cycle;
\draw (B) -- (C) -- (D) -- cycle;
\fill[red, opacity=0.5] (A) -- (B) -- (C);
\fill[blue, opacity=0.5] (B) -- (C) -- (D);
\draw[ultra  thick] (C) -- (B);

\coordinate (Tm) at (3.6,1.5);
\coordinate (Tp) at (2.0, 0.5);
\coordinate (e) at (0.5, 2.2);
\node[below left] at (Tm) {$T^{-} $ };
\node[above right] at (Tp) {$T^{+}$ };
\node[below right] at (e) {$F$ };

\coordinate (start) at (1.7, 1.7);
\coordinate (endPlus) at (2.25, 2.4);
\coordinate (endMinus) at (1.15, 1.0);

\draw [->, thick] (start) -- (endPlus);
\node[above right] at (endPlus) {$n^{+}$};

\draw [->, thick] (start) -- (endMinus);
\node[below left] at (endMinus) {$n^{-}$};

\end{tikzpicture}

\caption{Facet $F \in \mathcal{F}_h^{int} $ shared by the triangles $T^{+}, T^{-} \in \mathcal{T}_{h} $ and the normal unit vector $n^{+}$ and $n^{-}$. If we pick $T=T^{+}$ and want to evaluate the normal vector $n$ along a facet $F$, then we define $n = n  \mid _{F} = n^{+}$.}
    \label{fig:normal}
\end{figure}



\subsection{Broken Sobolev spaces}%
\label{sub:broken_sobolev_spaces}

In this work will we compute norms on discontinuous elements, thus, it will be necessary to define broken Sobolev spaces.
Let $\mathcal{T}_{h} $ be a mesh and some integer $m\le n$. Then we define the broken Sobolev space to be \[
    \begin{split}
H^{m}( \mathcal{T}_{h} ) & := \left\{ v \in L^2( \Omega )  \mid \ v|_{T} \in H^{m}( T) \quad     \forall T \in  \mathcal{T}_{h} \right\}\\
        L^{2}( \mathcal{F}_{h} ) &:= \left\{ v \in L^2( \mathcal{T}_{h}  )  \mid   \ v|_{F} \in L^{2}( F)  \quad  \forall F \in  \mathcal{F}_{h}   \right\}.
    \end{split}
\]
This motivates us to define broken Sobolev norms and inner products using summation over mesh elements,
\[
 \| v \|_{H^{m}( \mathcal{T}_{h} ) }^{2} = \sum_{T \in  \mathcal{T}_{h} }^{} \| v  \|_{ H^{m}( T ) }^{2  } \quad \text{ and } \quad
 (v ,w )_{ m, \mathcal{T} _{h} }^{} = \sum_{T \in \mathcal{T} _{h}}^{} (v ,w )_{ m,T }^{  } .
\]
As before, we use the shorthand notation,  $\| v \|_{\mathcal{T}_{h}} =  \| v \|_{L^{2}( \mathcal{T}_{h} ) }$ and  $(v ,w )_{ \mathcal{T}_{h} }^{} = (v ,w )_{L^2( \mathcal{T}_{h} ) }^{} $.
That is,
\[
 \| v \|_{L^{2}( \mathcal{F}_{h} ) }^{2} = \sum_{F \in  \mathcal{F}_{h} }^{} \| v  \|_{ L^{2}( F ) }^{2  } \quad \text{ and } \quad
 (v ,w )_{L^{2}( \mathcal{F}_{h} ) }^{} = \sum_{T \in \mathcal{F} _{h}}^{} (v ,w )_{ L^{2}( F ) }^{  } .
\]
Again, we often use the more compact notation $\| v \|_{\mathcal{F}_{h}} =  \| v \|_{L^{2}( \mathcal{F}_{h} ) }$ and  $(v ,w )_{ \mathcal{F}_{h} }^{} = (v ,w )_{L^2( \mathcal{F}_{h} ) }^{} $.
Often  it is needed to integrate over boundaries of elements $\partial \mathcal{T} _{h}  = \left\{ \partial T \  \mid \  \forall T \in \mathcal{T}_{h} \right\}  $, hence, we also denote the notation $\| \cdot  \|_{ \partial \mathcal{T} _{h} }^{  } = \sum_{T \in \mathcal{T} _{h} }^{} \| \cdot  \|_{ \partial T }^{  }
=\sum_{T \in \mathcal{T} _{h} }^{} \sum_{F \in \partial T}^{}  \| \cdot  \|_{ F }^{  }  $.

A very useful lemma when working with estimates on broken Sobolev spaces is that a if a function is continuous, then the jump between the mesh elements is zero. Keep in mind a function $ v \in  H^{1}( \mathcal{T}_{h} ) $ belongs to $ H^{1}( \Omega )  $ if and only
if $ \jump{ v }   = 0 \  \forall F \in \mathcal{F}^{int}_{h}$, see \cite[Lemma 1.23]{pietro2012}.

Recall, the "$\lesssim$" symbol denotes an inequality up to a constant factor. That is, given $a,b >0 $, the statement $a \lesssim b$ is true if there exists a constant $C>0$ such that $a \leq Cb$. Generally, this constant will contain
information related to the properties of the mesh, such as shape regularity and quasi-uniformity, but it often also includes the maximum finite number or measure of a quantity. For instance, let  $w , v_{i} \in L^{2}( \Omega ), \ a_{i} \in \mathbb{R}  \text{ for } i =
1,\ldots,N  $   and  $ \| w \|_{\Omega   }^{  }  =  \| \sum_{i}^{N} a_{i}  v_{i} \|_{\Omega   }^{  } $, then is $ \| w \|_{\Omega   }^{  }
\lesssim \sum_{i}^{N}  \  \| v_{i} \|_{\Omega   }^{  }  $. To maintain clarity and avoid unnecessary complexity, we will not delve into this particular detail related to the constant.

We can express several general useful
basic inequalities and estimates.
\begin{enumerate}[label=(\roman*)]
    \item A fundamental property of the inner-product the so-called Cauchy-Schwarz inequality
        \begin{equation}
            \label{eq:cauchy-schwartz}
     ( u,v)_{m,\Omega   }   \le \| u \|_{m,\Omega    }^{  } \| v \|_{m,\Omega     }^{  }\quad  \forall u,v \in H^{m}( \Omega ).
        \end{equation}

    \item Let $v \in  H^{m}( \Omega ) $, then is
        \begin{equation}
            \| D^{m}v \|_{\Omega   }^{  }  \lesssim \| v \|_{m,\Omega   }^{  }.
        \end{equation}
        Similarly is $\| \Delta  v \|_{\Omega   }^{  } \lesssim \| v \|_{2,\Omega   }^{  }  $.


    \item For all  $u \in L ^{2}( \mathcal{T} _{h})$ we have,
\begin{equation}
    \label{eq:mean_jump_estimate}
    \begin{split}
        \| \jump{  u }   \|_{ \mathcal{F} _{h} }^{  } & \le \|  u^{+}   \|_{ \mathcal{F} _{h} }^{  } +
        \| u^{-}   \|_{ \mathcal{F} _{h} }^{  }  \lesssim  \| u \|_{ \partial\mathcal{T }_{h}  }^{2  },\\
        \| \mean{ u }   \|_{ \mathcal{F} _{h} }^{  } & \le \| u^{+} \|_{ \mathcal{F} _{h}  }^{  } + \| u^{-} \|_{ \mathcal{F} _{h}  }^{  }    \lesssim  \| u \|_{ \partial\mathcal{T }_{h}  }^{2  }.
    \end{split}
\end{equation}

    \item For any $a,b > 0 $ the well known Young's $\varepsilon $-inequality is on the form,
        \begin{equation}
            \label{eq:young-epsilon}
            2ab \le \varepsilon a^2+ \frac{1}{\varepsilon } b^2.
        \end{equation}


\end{enumerate}



% \begin{enumerate}[label=(\roman*)]
%     \item General inverse inequality.
%     Let $\alpha = ( \alpha _{1}, \ldots, \alpha _{N}) $ and $\beta = ( \beta _{1}, \ldots, \beta _{N} ) $.
%     Assume $u \in H^{\abs{ \beta  }  }( T) $. The following inverse inequalities hold.
%     \[
%         \begin{split}
%         \| \partial ^{\alpha } u \|_{T  }^{  } & \lesssim h^{ - \abs{ \beta  }  } \| \partial ^{\alpha - \beta } u \|_{T }^{  } \\
%         \| \partial ^{\alpha }_{n} u  \|_{F  }^{  } &\lesssim h^{-\frac{1}{2}  } \| \partial ^{\alpha } u \|_{T }^{  }
%         \end{split}
%     \]
% \end{enumerate}
    % For a full triangle we have \[
    % \| v \|_{ \partial T }^{  } \lesssim h_{T}^{-\frac{1}{2}} \| v \|_{ T }^{  } + h_{T}^{\frac{1}{2}} \| \nabla v \|_{ T }^{  }
    % \]




\subsection{Lax-Milgram lemma}%
\label{sub:lax_milgram_lemma}

The intention is to introduce a abstract framework which can handle various types of partial differential equations (PDE). Let  $\mathcal{A} : X \to Y $ be a abstract linear operator encoding the structure of any linear PDE, including boundary
conditions and $X,Y$ are spaces of functions. Then we denote the abstract strong formulation as the equation
\begin{equation}
\label{eq:strong_abs}
\mathcal{A} u = f
\end{equation}
for a given function $f: \Omega \subset \mathbb{R}^d \mapsto \mathbb{R}$. We assume that the function $u: \Omega \rightarrow \mathbb{R}$ satisfies the relation \eqref{eq:strong_abs} pointwise so that $\mathcal{A} u(x)=f(x) \forall x \in \Omega$.
We will discover that Sobolev spaces are specifically engineered to study these kinds of problems.

\begin{definition}[Linear bounded functional]
    \label{def:linear_function}
Let $V$ be a Hilbert space. Furthermore, we define the dual space $V' $ to be the space of linear and bounded functionals $F: V  \mapsto \mathbb{R} $, i.e.,
\begin{equation}
V'  = \left\{ F: V \to \mathbb{R}   \mid  F \text{ is linear and bounded} \right\}.
\end{equation}
\end{definition}

\begin{problem}[Abstract linear problem]
    \label{def:abstract_linear_problem}
    Assume $X$ and $Y$  to be two Hilbert spaces. Let the vector space $\mathcal{L}( X,Y)  $ be all linear bounded operators spanned from $V$ to $Y$. We define the abstract linear problem as follows; find $u \in V$ such that
    \begin{equation}
    a( u,v)  = l(v ) := \left<f,v \right>_{V' , V}  \quad  \forall v \in V
    \end{equation}
    Where $a \in  \mathcal{L} ( V, V,\mathbb{R} ) $ is a bounded bilinear form and $f \in V':= \mathcal{L} ( V,\mathbb{R} )  $ is a bounded linear form associated with the abstract strong formulation \eqref{eq:strong_abs}. Here we denote by $\left<\cdot ,\cdot  \right>_{V',V} $ the duality pairing between $V'$ and $V$.

\end{problem}


\begin{definition}[Coercivity and Boundedness]
    \label{def:coercivity}
    Let $V$ be a Hilbert space and let $a( \cdot ,\cdot )  \in  \mathcal{L} ( V, V,\mathbb{R} )  $. Recall that the bilinear form $a( \cdot ,\cdot ) $ is coercive if \[
     a( v,v) \gtrsim  \| v \|_{ V }^{  } \quad  \forall v \in  V.
    \]
     The bilinear form $a( \cdot ,\cdot ) $ is said to bounded if   \[
    a( w,v)  \lesssim  \| w \|_{ V }^{  }  \| v \|_{V }^{  }\quad  \forall w,v \in V.
    \]
\end{definition}


\begin{lemma}[Lax-Milgram]
    \label{def:lax-milgram}
    The abstract linear problem \ref{def:abstract_linear_problem} is well-posed if $a(\cdot , \cdot  ) $ is bounded and coercive. Moreover, the following a priori estimate holds true.\[
    \| u \|_{ V }^{  } \lesssim  \| f \|_{ V'  }^{  }
    \]
\end{lemma}
\begin{proof}
    The problem can easily be proved using a special case of the Banach–Nečas–Babuška theorem. See \cite[Lemma 1.4]{pietro2012}
\end{proof}


\subsection{Finite element methods}%
\label{sub:finite_element_method}

The finite element method (FEM) is a numerical method to solve partial differential equation by finding an approximation of the Problem \ref{def:abstract_linear_problem}.  Let $V_{h}$ be a finite-dimensional (polynomial) approximation space on the mesh
$\mathcal{T} _{h}$. We say that a method is conform if $V_{h}\subset V $ and non-conform if $V _{h} \not\subset V$. In this thesis, our primary focus will be on the non-conforming methods. We define the approximate problem as follows.
\begin{problem}[The approximate problem]
    \label{def:approx_problem}
    Let $V_{h} \not\subset V$ a non-conform finite-dimensional space. Find  $u_{h} \in V_{h}$ such that, \[
    a_{h}(u_{h},v_{h} ) = l_{h}( v_{h}) :=  \left<f,v_{h} \right> \quad   \forall v_{h} \in V_{h}.
    \]
\end{problem}

We denote the bilinear form $a_{h}: V_{h} \times V_{h} \to \mathbb{R} $ as approximation of $a: V \times V \to \mathbb{R} $, and similarly for the right-hand side $l_{h} : V _{h} \to \mathbb{R} $ as an approximation of $l: V \to \mathbb{R} $.
Not that since the discrete is non-conformal, $V_{h} \not \subset  V_{}$, implies that boundedness and coercivity is not inherited from the continuous bilinear form $a( \cdot ,\cdot ) $. Hence, the discrete formulation $a_{h}:V_{h} \times V_{h}
\to \mathbb{R}  $ well-posed if it is coercive and bounded, \begin{equation}
    \begin{split}
   a_{h}( v_{h}, w_{h}) & \lesssim \| v_{h} \|_{ V_{h} }^{  } \| w_{h} \|_{ V_{h} }^{  } \quad    \forall w_{h},v_{h} \in V_{h},\\
   a_{h}( v_{h}, v_{h}) & \gtrsim  \| v_{h} \|_{ V_{h} }^{  2} \quad  \forall v_{h} \in V_{h}.
    \end{split}
\end{equation}
Note that we have relaxed the boundedness of $l_{h}( \cdot ) $, which arises of the finite dimensional property of $V_{h}$.

Strictly speaking, if $a_{h}( \cdot ,\cdot ) $ is well-posed does not necessary imply that it is consistent. Hence, a critical assumption about $a_{h}( \cdot ,\cdot )$ is its consistency. To be more precise, the discrete bilinear form can extend into a mapping $a_{h} : V \times V_{h} \to \mathbb{R}$
\footnote{
Technically speaking, we must assume there is a subspace $V^{'} \subset V$  such that the exact solution $u$  belongs to $ V'  $ such that $a_{h}: V' \times V_{h} \to \mathbb{R} $ (we can not generalize to $a_{h} : V \times
V_{h} \to \mathbb{R} $ ), but this is outside the scope of the thesis. For more information, see \cite[Definition 1.31]{pietro2012}}. Within this framework, there exists an exact solution $u \in V$, for which it holds that $a_{h}( u,v_{h}) = l_{h}( v_{h})$. This inherently implies the so-called Galerkin orthogonality, where
\begin{equation}
    \label{eq:galerin_orth}
a_{h}( u - u_{h}, v_{h}) = 0.
\end{equation}
Let $v + v_{h} \in V_{h} \oplus V$, which is the direct sum of the spaces, with a corresponding norm $\| \cdot  \|_{V_{h},*  }^{  } $. Note that $V$ must sufficiently regular enough for the $\| \cdot  \|_{V_{h},*  }^{  } $ to be well-defined. For the forthcoming a priori analysis outlined in Section \ref{sub:ceas_lemma}, a necessary assumption is that the discrete bilinear form is bounded in $V \otimes V_{h}$, i.e.
\begin{equation}
    \label{eq:ah_star}
a_{h}( v + v_{h}, w_{h}) \lesssim  \| v + v_{h} \|_{ V_{h},* }^{  } \| w_{h} \|_{V_{h}  }^{  }.
\end{equation}


Following on, we will concentrate on establishing a theoretical framework for finite element analysis.
\begin{definition}[Local polynomial space]
    \label{def:local_space}
    Let $T$ be a element in a mesh $\mathcal{T}_{h} $,  $x = \left[ x_{1}, \ldots, x_{d} \right] $ be a vector, and $\alpha  = \left[ \alpha _{1}, \ldots, \alpha _{d} \right] \in \mathbb{N} ^{d} $ be a multi index.
    The local polynomial space $\mathcal{P} ^{k}( T) $ for a simplex is denoted as
    \begin{equation}
    \label{eq:pol_space}
        \mathcal{P}^{k}( T) =  \mathrm{span}\left\{ x^{\alpha } \ \text{for } x \in T \text{ and } 0 \le  \alpha _{i} \le k \right\}.
    \end{equation}
    where  $x^{\alpha }$ is a monomial such that $x^{\alpha } = x_{1}^{\alpha _{1}} \ldots x_{d}^{\alpha _{d}}$.

    Let $T$ be a cuboid, i.e.,  $T = \prod_{i=1}^{d} [z_{i}^{-},z_{i}^{+}]$ where $z_{i}^{-}< z_{i}^{+}$ for $z_{i}^{\pm} \in \mathbb{R} $. Then the polynomial space $\mathcal{Q}^{k}( T)$  in $\mathbb{R} ^{d}$ is defined as the tensor product of 1-dimensional
    finite elements, i.e.,
      \[
    \mathcal{Q} ^{k}(T)  := \mathcal{P}^{k}( [z_{1}^{-},z_{1}^{+}] ) \otimes \ldots \otimes \mathcal{P}^{k}( [z_{d}^{-},z_{d}^{+}] )
    \]
\end{definition}
For more information about the local polynomial spaces, see \cite[Chapter 6.4, 7.3]{ErnGuermond2021}

Following Ciarlet \cite[pp.93]{ciarlet1991basic}, the abstract definition of a finite element is defined as the triplet $( T, \mathcal{P}, \Sigma ) $.
In our case, $T$ represents either a simplex or a quadrilateral geometry, and $\mathcal{P}$ denotes a finite-dimensional polynomial space consisting of $N$ shape functions $\left\{ \phi_{i} \right\}_{i\in \mathcal{I} } $, where $\mathcal{I} = \left\{
1, \ldots, N \right\} $, as depicted in Definition \ref{def:local_space}.
On the other hand, $\Sigma $ is the so-called dual of $\mathcal{P}$, that is, the set of linear forms $\left\{ \sigma _{i} \right\}_{i \in \mathcal{I} } $ such that. $ \sigma_{j} ( \phi_{i} ) = \delta _{ij}$ and $p( x) = \sum_{i\in \mathcal{I} }^{} \sigma_{i} ( p) p_{i} $.
If there is a set of points $\left\{ a_{i} \right\}_{i \in \mathcal{I} } $  in $T$ such that
$\sigma_{i}( p) = p( a_{i}) \  \forall p \in \mathcal{P}$,  then the triple $( T, \mathcal{P}, \Sigma  ) $ is called a Lagrangian finite element. The set of points $\left\{ a_{i} \right\}_{i \in \mathcal{I} }  $ is called nodes and is associated with the
so-called nodal basis of $\mathcal{P} $ such that  $\phi ( a_{i}) = \delta _{ij} \ \forall i,j  \in \mathcal{I} $

As anticipated, the local node configuration of the polynomial space is influenced by the form of $T$. For the purpose of our discussion, let us represent the polynomial basis for a simplicial element and a quadrilateral
element as $\mathcal{P} ^{k}(T )$ and $\mathcal{Q} ^{k}( T)$, both of polynomial order $k$. In Figure \ref{fig:ill_nodes} is it illustrated for  $k=1,2,3$ in dimension $d=2$ on how the node configuration evolve.

\begin{figure}[h!]
    \centering
    \hfill
    \subfloat[$\mathcal{P}^{1}( T)  $]{
        \begin{tikzpicture}[scale=1.0]
            % Define the coordinates
            \coordinate (A) at (0,0);
            \coordinate (B) at (2,0);
            \coordinate (C) at (1,{sqrt(3)});

            % Draw the triangle
            \fill[blue!30] (A) -- (B) -- (C) -- cycle;
            \draw (A) -- (B) -- (C) -- cycle;

            % Draw the nodes
            \fill (A) circle (2pt);
            \fill (B) circle (2pt);
            \fill (C) circle (2pt);
        \end{tikzpicture}
    }
    \hfill
    \subfloat[ $\mathcal{P}^{2}( T)  $]{
        \begin{tikzpicture}[scale=1.0]
            % Define the coordinates
            \coordinate (A) at (0,0);
            \coordinate (B) at (2,0);
            \coordinate (C) at (1,{sqrt(3)});

            % Define the midpoints
            \coordinate (D) at ($(A)!0.5!(B)$);
            \coordinate (E) at ($(B)!0.5!(C)$);
            \coordinate (F) at ($(A)!0.5!(C)$);

            % Draw the triangle
            \fill[blue!30] (A) -- (B) -- (C) -- cycle;
            \draw (A) -- (B) -- (C) -- cycle;

            % Draw the nodes
            \fill (A) circle (2pt);
            \fill (B) circle (2pt);
            \fill (C) circle (2pt);
            \fill (D) circle (2pt);
            \fill (E) circle (2pt);
            \fill (F) circle (2pt);
        \end{tikzpicture}
    }
    \hfill
    \subfloat[$\mathcal{P} ^{3}( T)  $ ]{
        \begin{tikzpicture}[scale=1.0]
            % Define the coordinates
            \coordinate (A) at (0,0);
            \coordinate (B) at (2,0);
            \coordinate (C) at (1,{sqrt(3)});

            % Define the additional points along the edges
            \coordinate (D) at ($(A)!0.3333!(B)$);
            \coordinate (E) at ($(B)!0.3333!(C)$);
            \coordinate (F) at ($(C)!0.3333!(A)$);
            \coordinate (G) at ($(A)!0.6667!(B)$);
            \coordinate (H) at ($(B)!0.6667!(C)$);
            \coordinate (I) at ($(C)!0.6667!(A)$);

            % Define the centroid
            \coordinate (J) at (1, {sqrt(3)/3});

            % Draw the triangle
            \fill[blue!30] (A) -- (B) -- (C) -- cycle;
            \draw (A) -- (B) -- (C) -- cycle;

            % Draw the nodes
            \fill (A) circle (2pt);
            \fill (B) circle (2pt);
            \fill (C) circle (2pt);
            \fill (D) circle (2pt);
            \fill (E) circle (2pt);
            \fill (F) circle (2pt);
            \fill (G) circle (2pt);
            \fill (H) circle (2pt);
            \fill (I) circle (2pt);
            \fill (J) circle (2pt);
        \end{tikzpicture}
    }
    \\
    \hfill
    \subfloat[$\mathcal{Q}^{1}( T)  $]{
        \begin{tikzpicture}[scale=1.0]
            % Define the coordinates
            \coordinate (A) at (0,0);
            \coordinate (B) at (2,0);
            \coordinate (C) at (2,2);
            \coordinate (D) at (0,2);

            % Draw the square
            \fill[blue!30] (A) -- (B) -- (C) -- (D) -- cycle;
            \draw (A) -- (B) -- (C) -- (D) -- cycle;

            % Draw the nodes
            \fill (A) circle (2pt);
            \fill (B) circle (2pt);
            \fill (C) circle (2pt);
            \fill (D) circle (2pt);
        \end{tikzpicture}
    }
    \hfill
    \subfloat[ $\mathcal{Q} ^{2}( T)  $]{
        \begin{tikzpicture}[scale=1.0]
            % Define the coordinates
            \coordinate (A) at (0,0);
            \coordinate (B) at (2,0);
            \coordinate (C) at (2,2);
            \coordinate (D) at (0,2);

            % Define the midpoints
            \coordinate (E) at ($(A)!0.5!(B)$);
            \coordinate (F) at ($(B)!0.5!(C)$);
            \coordinate (G) at ($(C)!0.5!(D)$);
            \coordinate (H) at ($(D)!0.5!(A)$);
            % Define the centroid
            \coordinate (M) at (1, 1);

            % Draw the square
            \fill[blue!30] (A) -- (B) -- (C) -- (D) -- cycle;
            \draw (A) -- (B) -- (C) -- (D) -- cycle;

            % Draw the nodes
            \fill (A) circle (2pt);
            \fill (B) circle (2pt);
            \fill (C) circle (2pt);
            \fill (D) circle (2pt);
            \fill (E) circle (2pt);
            \fill (F) circle (2pt);
            \fill (G) circle (2pt);
            \fill (H) circle (2pt);
            \fill (M) circle (2pt);
        \end{tikzpicture}
    }
    % And so on for $\mathcal{P}^{3}( S)  $, $\mathcal{P}^{4}( S)  $, etc.
    \hfill
    \subfloat[ $\mathcal{Q} ^{3}( T)  $]{
        \begin{tikzpicture}[scale=1.0]
            % Define the coordinates
            \coordinate (A) at (0,0);
            \coordinate (B) at (2,0);
            \coordinate (C) at (2,2);
            \coordinate (D) at (0,2);

            % Define the additional points along the edges
            \coordinate (E) at ($(A)!0.3333!(B)$);
            \coordinate (F) at ($(B)!0.3333!(C)$);
            \coordinate (G) at ($(C)!0.3333!(D)$);
            \coordinate (H) at ($(D)!0.3333!(A)$);
            \coordinate (I) at ($(A)!0.6667!(B)$);
            \coordinate (J) at ($(B)!0.6667!(C)$);
            \coordinate (K) at ($(C)!0.6667!(D)$);
            \coordinate (L) at ($(D)!0.6667!(A)$);

            % Define the internal nodes
            \coordinate (M) at ($(E)+(0,0.6667)$);
            \coordinate (N) at ($(E)+(0,1.332)$);
            \coordinate (O) at ($(I)+(0,0.6667)$);
            \coordinate (P) at ($(I)+(0,1.332)$);

            % Draw the square
            \fill[blue!30] (A) -- (B) -- (C) -- (D) -- cycle;
            \draw (A) -- (B) -- (C) -- (D) -- cycle;

            % Draw the nodes
            \fill (A) circle (2pt);
            \fill (B) circle (2pt);
            \fill (C) circle (2pt);
            \fill (D) circle (2pt);
            \fill (E) circle (2pt);
            \fill (F) circle (2pt);
            \fill (G) circle (2pt);
            \fill (H) circle (2pt);
            \fill (I) circle (2pt);
            \fill (J) circle (2pt);
            \fill (K) circle (2pt);
            \fill (L) circle (2pt);
            \fill (M) circle (2pt);
            \fill (N) circle (2pt);
            \fill (O) circle (2pt);
            \fill (P) circle (2pt);
        \end{tikzpicture}
        }
        \caption{Illustration of the nodes for the element of a simplex a quadrilateral for dimension $d=2$ for polynomial orders $k=1,2,3$.}
        \label{fig:ill_nodes}
\end{figure}


We may introduce the reference element $\hat{T} $ in $d$ dimensions. The reference for a quadrilateral is denoted as $\hat{T} = [0,1]^{d}$. The reference for a simplex in  is defined by the convex hull spanned by the points $( z_{0}, e_{1}, \ldots, e_{d}) $ where $z_{0}:=0$ is the origin and $ \left\{ e_{i}
\right\}_{i=1}^{d} $ is the standard Cartesian  unit basis in $ \mathbb{R} ^{d} $. A corresponding reference finite element is defined as $( \hat{T}, \hat{\mathcal{P} }, \hat{\Sigma}   ) $.



\begin{figure}[th!]
    \centering
    \hspace{-2.2cm}  % Adjust the space as necessary
    \subfloat[]{
        \begin{tikzpicture}[scale=1.0]
            % Define the Reference triange
            \coordinate (A) at (0,0);
            \coordinate (B) at (1,0);
            \coordinate (C) at (0,1);
            \coordinate (G1) at (1/3,1/3);

            \fill[red!30] (A) -- (B) -- (C) -- cycle;
            \draw (A) -- (B) -- (C) -- cycle;
            \fill (A) circle (2pt);
            \fill (B) circle (2pt);
            \fill (C) circle (2pt);

            % Define the arbitary triange
            \coordinate (D) at (2,0);
            \coordinate (E) at (4,0);
            \coordinate (F) at (3,{sqrt(2)});
            \coordinate (G2) at (3, {sqrt(3)/3});

            \fill[blue!30] (D) -- (E) -- (F) -- cycle;
            \draw (D) -- (E) -- (F) -- cycle;
            \fill (D) circle (2pt);
            \fill (E) circle (2pt);
            \fill (F) circle (2pt);

            % Draw arrow to signify the mapping G
            \draw[->, thick, >=stealth] ($(G1)+(0.2,0.2)$) to[bend left] node[midway, above, yshift=0.1cm] {$\mathcal{G}$} ($(G2)+(-0.2,0.2)$);
        \end{tikzpicture}
    }
    \hspace{2.0cm}  % Adjust the space as necessary
    \subfloat[]{
        \begin{tikzpicture}[scale=1.0]
            % Define the Reference quadrilateral
            \coordinate (A) at (0,0);
            \coordinate (B) at (1,0);
            \coordinate (C) at (1,1);
            \coordinate (D) at (0,1);
            \coordinate (G1) at (0.5,0.5);

            \fill[red!30] (A) -- (B) -- (C) -- (D) -- cycle;
            \draw (A) -- (B) -- (C) -- (D) -- cycle;
            \fill (A) circle (2pt);
            \fill (B) circle (2pt);
            \fill (C) circle (2pt);
            \fill (D) circle (2pt);

            % Define the arbitrary quadrilateral
            \coordinate (E) at (2,0);
            \coordinate (F) at (3,1);
            \coordinate (G) at (3,2);
            \coordinate (H) at (2,1);
            \coordinate (G2) at (2.5,0.5);

            \fill[blue!30] (E) -- (F) -- (G) -- (H) -- cycle;
            \draw (E) -- (F) -- (G) -- (H) -- cycle;
            \fill (E) circle (2pt);
            \fill (F) circle (2pt);
            \fill (G) circle (2pt);
            \fill (H) circle (2pt);

            % Draw arrow to signify the mapping G
            \draw[->, thick, >=stealth] ($(G1)+(0.3,0.1)$) to[bend left] node[midway, above, yshift=0.1cm] {$\mathcal{G}$} ($(G2)+(-0.2,0.2)$);
        \end{tikzpicture}
    }
    \\
    \subfloat[]{
        \begin{tikzpicture}[scale=1.0]
            % Define the Reference tetra
            \coordinate (A) at (0,0);
            \coordinate (B) at (1,0);
            \coordinate (C) at (0,1);
            \coordinate (D) at (0.7,0.7);
            \coordinate (G1) at (1/3,1/3);

            \fill[red!65] (A) -- (B) -- (C) -- cycle;
            \fill[red!30] (D) -- (B) -- (C) -- cycle;
            \draw (A) -- (B) -- (C) -- cycle;
            \draw (D) -- (B) -- (C) -- cycle;
            \draw[dashed] (A) -- (D);
            \fill (A) circle (2pt);
            \fill (B) circle (2pt);
            \fill (C) circle (2pt);
            \fill (D) circle (2pt);


            % Define the Translated tetra
            \coordinate (A') at ($(A) + (2,0)$);
            \coordinate (B') at ($(B) + (2,0)$);
            \coordinate (C') at ($(C) + (2.6, 0.6)$);
            \coordinate (D') at ($(D) + (2.5, 0.2)$);

            \fill[blue!65] (A') -- (B') -- (C') -- cycle;
            \fill[blue!30] (D') -- (B') -- (C') -- cycle;
            \draw (A') -- (B') -- (C') -- cycle;
            \draw (D') -- (B') -- (C') -- cycle;
            \draw[dashed] (A') -- (D');
            \fill (A') circle (2pt);
            \fill (B') circle (2pt);
            \fill (C') circle (2pt);
            \fill (D') circle (2pt);

            % Draw arrow to signify the mapping G
            \draw[->, thick, >=stealth] ($(G1)+(0.6,0.2)$) to[bend left] node[midway, above, yshift=0.1cm] {$\mathcal{G}$} ($(D')+(-0.6,0.1)$);
        \end{tikzpicture}
    }
    \hspace{2cm}  % Adjust the space as necessary
    \subfloat[]{
        \begin{tikzpicture}[scale=1.0]
    % Define the Reference cube
    \coordinate (A) at (0,0);
    \coordinate (B) at (1,0);
    \coordinate (C) at (1,1);
    \coordinate (D) at (0,1);
    \coordinate (E) at (0.3,0.3);
    \coordinate (F) at (1.3,0.3);
    \coordinate (G) at (1.3,1.3);
    \coordinate (H) at (0.3,1.3);

    \fill[red!65] (A) -- (B) -- (C) -- (D) -- cycle;
    \fill[red!45] (D) -- (H) -- (G) -- (C) -- cycle;
    \fill[red!30] (B) -- (C) -- (G) -- (F) -- cycle;
    \draw (A) -- (B) -- (C) -- (D) -- cycle;
    \draw (A) -- (E) -- (F) -- (B);
    \draw (D) -- (H) -- (G) -- (C);
    \draw[dashed] (E) -- (H);
    \draw[dashed] (F) -- (G);

    % Define the Translated cube
    \coordinate (A') at ($(A) + (3.3, 0.0)$);
    \coordinate (B') at ($(B) + (3.3, 0.0)$);
    \coordinate (C') at ($(C) + (3.3, 0.0)$);
    \coordinate (D') at ($(D) + (3.3, 0.0)$);
    \coordinate (E') at ($(E) + (2.2, 0.3)$);
    \coordinate (F') at ($(F) + (2.2, 0.3)$);
    \coordinate (G') at ($(G) + (2.2, 0.3)$);
    \coordinate (H') at ($(H) + (2.2, 0.3)$);

    \fill[blue!65] (A') -- (B') -- (C') -- (D') -- cycle;
    \fill[blue!30] (D') -- (H') -- (G') -- (C') -- cycle;
    \fill[blue!45] (D') -- (A') -- (E') -- (H') -- cycle;
    \draw (A') -- (B') -- (C') -- (D') -- cycle;
    \draw (A') -- (E') -- (F') -- (B');
    \draw (D') -- (H') -- (G') -- (C');
    \draw[dashed] (E') -- (H');
    \draw[dashed] (F') -- (G');

    \fill (A') circle (2pt);
    \fill (B') circle (2pt);
    \fill (C') circle (2pt);
    \fill (D') circle (2pt);
    \fill (E') circle (2pt);
    \fill [opacity=0.5] (F') circle (2pt);
    \fill (G') circle (2pt);
    \fill (H') circle (2pt);

    \fill (A) circle (2pt);
    \fill (B) circle (2pt);
    \fill (C) circle (2pt);
    \fill (D) circle (2pt); % D has half opacity now
    \fill [opacity=0.5](E) circle (2pt);
    \fill (F) circle (2pt);
    \fill (G) circle (2pt);
    \fill (H) circle (2pt);

    % Draw arrow to signify the mapping G
    \draw[->, thick, >=stealth] ($(A)+(0.8,0.5)$) to[bend left] node[midway, above, yshift=0.1cm] {$\mathcal{G}$} ($(A')+(-0.5,0.8)$);
\end{tikzpicture}
    }
    \caption{Illustration of affine mapping $\mathcal{G} : \hat{T} \to T $ in dimensions $d=2,3$ from a reference element $\hat{T}$  to element $T$ for simplexes and and quadrilaterals.  } \label{fig:affine_mapping}
\end{figure}

Let the mapping $\mathcal{G} : \hat{T} \to  T$ an affine
mapping, i.e. $\mathcal{G}(x) = Ax +
b$.
The important property of affine transformations is the preservation of parallelism. Hence,  for any two vectors $x,y \in  \hat{T}$ that are parallel in the reference element, their images $\mathcal{G}(x)  $  and $\mathcal{G}( y)  $ will
also be parallel. Generally speaking, an affine transformation of the reference simplex is a transformation to any another simplex of the same dimension. However, for any quadrilateral, an affine transformation preserves the parallelism of opposite
sides, for an illustration see Figure \ref{fig:affine_mapping} and for a counter example see Figure \ref{fig:nonaffine}.

\begin{figure}[h!]
    \centering
    \hspace{-2.2cm}  % Adjust the space as necessary
    \subfloat[]{
        \begin{tikzpicture}[scale=1.0]
            % Define the Reference quadrilateral
            \coordinate (A) at (0,0);
            \coordinate (B) at (1,0);
            \coordinate (C) at (1,1);
            \coordinate (D) at (0,1);
            \coordinate (G1) at (0.5,0.5);

            \fill[red!30] (A) -- (B) -- (C) -- (D) -- cycle;
            \draw (A) -- (B) -- (C) -- (D) -- cycle;
            \fill (A) circle (2pt);
            \fill (B) circle (2pt);
            \fill (C) circle (2pt);
            \fill (D) circle (2pt);

            % Define the arbitrary quadrilateral
            \coordinate (E) at (2,0);
            \coordinate (F) at (3,0);
            \coordinate (G) at (3.5,1);
            \coordinate (H) at (2.5,1);
            \coordinate (G2) at (2.5,0.5);

            \fill[blue!30] (E) -- (F) -- (G) -- (H) -- cycle;
            \draw (E) -- (F) -- (G) -- (H) -- cycle;
            \fill (E) circle (2pt);
            \fill (F) circle (2pt);
            \fill (G) circle (2pt);
            \fill (H) circle (2pt);

            % Draw arrow to signify the mapping G
            \draw[->, thick, >=stealth] ($(G1)+(0.3,0.1)$) to[bend left] node[midway, above, yshift=0.1cm] {$\mathcal{G}$} ($(G2)+(-0.2,0.2)$);
        \end{tikzpicture}
    }
    \hspace{2.0cm}  % Adjust the space as necessary
    \subfloat[]{
        \begin{tikzpicture}[scale=1.0]
            % Define the Reference quadrilateral
            \coordinate (A) at (0,0);
            \coordinate (B) at (1,0);
            \coordinate (C) at (1,1);
            \coordinate (D) at (0,1);
            \coordinate (G1) at (0.5,0.5);

            \coordinate (R1) at (1.5,0.7);
            \coordinate (R2) at (1.4,1.0);

            \fill[red!30] (A) -- (B) -- (C) -- (D) -- cycle;
            \draw (A) -- (B) -- (C) -- (D) -- cycle;
            \draw[very thick] (R1) -- (R2);
            \fill (A) circle (2pt);
            \fill (B) circle (2pt);
            \fill (C) circle (2pt);
            \fill (D) circle (2pt);

            % Define the arbitrary quadrilateral
            \coordinate (E) at (2,0);
            \coordinate (F) at (3.5,0);
            \coordinate (G) at (3.0,1);
            \coordinate (H) at (2.5,1);
            \coordinate (G2) at (2.5,0.5);

            \fill[blue!30] (E) -- (F) -- (G) -- (H) -- cycle;
            \draw (E) -- (F) -- (G) -- (H) -- cycle;
            \fill (E) circle (2pt);
            \fill (F) circle (2pt);
            \fill (G) circle (2pt);
            \fill (H) circle (2pt);

            % Draw arrow to signify the mapping G
            \draw[->, thick, >=stealth] ($(G1)+(0.3,0.1)$) to[bend left] node[midway, above, yshift=0.1cm] {$\mathcal{G}$} ($(G2)+(-0.2,0.2)$);

        \end{tikzpicture}
    }
    \caption{Illustration of an affine mapping $\mathcal{G}: \hat{T} \mapsto T$  versus a non-affine transformation. The left figure preserves parallel lines before and after the transformation, indicating an affine transformation. However, the right figure does not maintain parallelism, making it a non-affine transformation.}
    \label{fig:nonaffine}
\end{figure}

Following \cite[Example 9.4]{ErnGuermond2021}, the $ ( \hat{T}, \hat{\mathcal{P} }, \hat{\Sigma} ) $ is denoted as the reference finite element associated with the nodes $\left\{ \hat{a}_{i} \right\}_{i\in N} $. Let the mapping $\psi $ be function in $\mathcal{L}( \mathcal{P} ^{k}( T), \mathcal{P} ^{k}(\hat{T}
)  )  $ such that is an isomorphic from  $ \psi : \mathcal{P} ^{k}( T) \to \mathcal{P} ^{k}( \hat{T})   $.  Then $\sigma ( p) = \hat{\sigma }( \psi ( p) ) (a_{i} )  = ( p \circ \mathcal{G} )( \hat{a}_{i})      $ for all $p \in \mathcal{P}^{k}( T)  $.
Then the Lagrange interpolation follows,  \[
    p (x) = \sum_{i \in N }^{} \sigma ( a_{i}) \phi_{i}( x) \quad \text{for } a_{i} = \mathcal{G}( \hat{a}_{i}) \quad  \forall i \in N     .
\]
Hence, the finite element $( T, \mathcal{P}, \Sigma  ) $, associated with the notes $\left\{ a_{i} \right\}_{i\in N} $, is reconstructed via the reference finite element $( \hat{T}, \hat{\mathcal{P} }, \hat{\Sigma} ) $.
Thus, using the affine transformation can the definition of the local polynomial space be extended to dependent on the reference element. That is, \[
    \begin{split}
\mathcal{P}^{k}( T) & = \left\{ \hat{v} \circ  \mathcal{G}^{-1}( T)   \mid  \hat{v} \in \mathcal{P}^{k}( \hat{T})      \right\} \\
\mathcal{Q}^{k}( T) & = \left\{ \hat{v} \circ  \mathcal{G}^{-1}( T)     \mid  \hat{v} \in \mathcal{Q}^{k}( \hat{T})      \right\} \\
    \end{split}
\]
Working on shape-regular and affine geometries has shown to greatly simplify and generalise local interpolation estimates, see \cite[Theorem 11.12]{ErnGuermond2021}, and thus is very useful for deriving a priori estimates.
Please note that workarounds exist for proving nonaffine local interpolation estimates, but they require key assumptions on the relationship between the nodes $a_{i}$ and the regularity of the mapping $\mathcal{G} $ \cite[Chapter 13]{ErnGuermond2021}.
Hence, affine meshes is essential for the error analysis which utilize the interpolation estimates, but it limits us to work on structure mesh if we specifically choose on quadrilateral meshes.

\begin{definition}[Broken polynomial spaces]
    Let $\mathcal{T}_{h} $ be a mesh of $\Omega \in \mathbb{R} ^{d} $ and $\Omega _{h} = \bigcup_{T \in \mathcal{T}_{h} } T$ . Let $\mathcal{P}^{k}(T) $ be the space of all polynomials of order $k$ in the mesh element $T$ in $\mathcal{T}_{h}$ . We define the broken polynomial space and the global $C^{0}$ continuous polynomial space as
    \begin{equation}
        \begin{split}
    \mathcal{P}^{k} ( \mathcal{T}_{h} ) & := \left\{ v \in L^{2}( \Omega _{h} )    \mid  v|_{T} \in \mathcal{P}^k( T) \quad  \forall T \in  \mathcal{T}_{h}   \right\}. \\
    \mathcal{P}^{k}_{c} ( \mathcal{T}_{h} ) & := \left\{ v \in C^{0}( \Omega _{h}  )   \mid  v|_{T} \in \mathcal{P}^k( T) \quad  \forall T \in  \mathcal{T}_{h}   \right\}.
        \end{split}
    \end{equation}
    Similarly, for quadrilatural elements is the polynomial spaces defined as,
    \begin{equation}
        \begin{split}
    \mathcal{Q}^{k} ( \mathcal{T}_{h} ) & := \left\{ v \in L^{2}( \Omega _{h} )    \mid  v|_{T} \in \mathcal{Q}^k( T) \quad  \forall T \in  \mathcal{T}_{h}   \right\}. \\
    \mathcal{Q}^{k}_{c} ( \mathcal{T}_{h} ) & := \left\{ v \in C^{0}( \Omega _{h} )   \mid  v|_{T} \in \mathcal{Q}^k( T) \quad  \forall T \in  \mathcal{T}_{h}   \right\}.
        \end{split}
    \end{equation}

\end{definition}


In this thesis will we generally utilize the global $C^{0}$ continuity, hence, for the rest of the thesis do we define
\begin{equation}
    \label{def:Vh_background}
V_{h} =  \left\{ P_{c}^{k}( \mathcal{T}_{h} ) \text{ or }  Q_{c}^{k}( \mathcal{T}_{h} )
 \right\} \end{equation}
Hence, all results holds for both polynomial spaces.

\begin{figure}[h!]
\centering
\begin{tikzpicture}[node distance=2cm,
                    box1/.style={rectangle, minimum width=3cm, minimum height=1cm, text centered, draw=black, fill=red!30},
                    box2/.style={rectangle, minimum width=3cm, minimum height=1cm, text centered, draw=black, fill=orange!30},
                    box3/.style={rectangle, minimum width=3cm, minimum height=1cm, text centered, draw=black, fill=green!30},
                    box4/.style={rectangle, minimum width=3cm, minimum height=1cm, text centered, draw=black, fill=blue!30},
                    arrow/.style={thick,->,>=stealth}
                    ]
\node (sf) [box1] { Find $u:\Omega \to \mathbb{R} $ such that $\mathcal{A} u = f$};
\node[left=0.5cm of sf,font=\bfseries, align=right] {Strong formulation:};

\node (alp) [box2, below of=sf] {Find $u \in V$ such that $a(u,v) = l(v) \  \forall v \in V$};
\node[left=0.5cm of alp,font=\bfseries, align=right] {Abstract weak problem:};

\node (dlp) [box3, below of=alp] {Find $u_h \in V_{h}$ such that $a_{h}(u_h,v_h) = l_h(v) \  \forall v_h \in V_{h}$};
\node[left=0.5cm of dlp,font=\bfseries, align=right] {Discrete weak problem:};

\node (lse) [box4, below of=dlp] {Solve $A U = F$};
\node[left=0.5cm of lse,font=\bfseries, align=right] {Linear system of equations:};

\draw [arrow] (sf) -- (alp);
\draw [arrow] (alp) -- (dlp);
\draw [arrow] (dlp) -- (lse);
\end{tikzpicture}
\caption{Workflow of solving linear PDEs using the FEM method.}
\label{fig:fem_workflow}
\end{figure}

We now have a well-defined discrete global space  $V_{h} $ consisting of the finite set of basis functions $\left\{ \phi _{i} \right\}_{i=1}^{N} $ associated with the Lagriangian nodes $\left\{ a_{i} \right\}_{i=1}^{N}  $. The degree of freedoms,
also known as ndofs, is denoted as $dim(V_{h}) = N$. Let $U_{j} = u_{h}\left(
a_{j} \right) $, so that $u_{h} = \sum_{j=1}^{N} U_{j} \phi _{j}  $. Then the Problem \ref{def:approx_problem} is equivalent to
\begin{equation}
    \label{eq:discretized_system}
\sum_{j = 1}^{N} u_{j} a_{h}\left( \phi _{j}, \phi _{i} \right)  = l_{h}\left( \phi _{i} \right)
\end{equation}
Hence, by letting $U = \left[ U_{j} \right] $ , $F = \left[ \left( f, \phi _{i}  \right) _{\Omega } \right] $  and $A = \left[ a_{h}\left( \phi _{j}, \phi _{i} \right)  \right] $ can we construct a linear system,
\begin{equation}
\label{eq:linear_system}
    A U =F.
\end{equation}
Ultimately, the matrix $A$ is shown to be symmetric positive definite only if $a_{h}( \cdot ,\cdot ) $ is well-posed.


To summarize the workflow of solving linear PDEs using the FEM method, see Figure \ref{fig:fem_workflow}.

\subsection{Condition number}%
\label{sub:note_on_condition_number}

Recall the discrete $l^{p}$ norm for a vector,

 \begin{equation}
     \forall U \in \mathbb{R} ^{N}, \quad
\| U  \|_{ p }^{  } =
     \begin{cases}
     &  \left( \sum_{i=1}^{N}  \abs{ U_{i}  }^{p} \right)^{\frac{1}{p}} , \quad  1\le p < \infty \\
   &  \max_{i}  \abs{ U_i },  \quad  p= \infty
     \end{cases}
 \end{equation}
Also recall the definition of the matrix norm,
\begin{equation}
 \forall A \in \mathbb{R} ^{N\times N}, 1\le p \le  \infty, \quad  \| A  \|_{p  }^{  }  = \max_{U \in \mathbb{R} ^{N} \setminus 0} \frac{\| AU \|_{ p }^{  } }{\| U \|_{p}^{  } }.
\end{equation}
Remark that this notation is not to be confused with Sobolev norms.
Assume that $A $ is invertible, then we define the condition number for a matrix in $l^{p}  $ norm defined such that
\begin{equation}
    \label{eq:condition_num}
 \forall A \in \mathbb{R} ^{N\times N}, 1\le p \le  \infty, \quad  \kappa_{p} ( A) = \| A  \|_{ p }^{  } \| A ^{-1} \|_{ p }.
\end{equation}

From basic theory is it known that $\| A \|_{ 2 }^{  } $   is equal to the maximum singular value of $A$ , where singular values of $ A$  are the square roots of the eigenvalues of $A^TA$ \cite[Theorem 2.9]{ suli2003introduction}. Because of the connection
between $\| A \|_{ 2 }^{  } $ norms and its singular values, is $k_{2}( A) $ is often in preferred numerical analysis.
A challenge is that the computations generally involves performing Singular Value Decomposition (SVD)
 or power iteration, which can be quite expensive operations particularly for large sparse matrices.
 However,  $\| A \|_{ \infty }^{  } $ is computed as the maximum absolute row sum and ,hence, only necessary to compute the sum for the non-zero elements in each row. Thus, we seek to estimate
 $\kappa_{2}( A) $ using $ \kappa _{\infty}( A) $.

It is well established that $ \frac{1}{\sqrt{N} }  \| A \|_{ \infty }^{  }  \le \| A \|_{2  }^{  } \le
\sqrt{N}  \| A \|_{\infty  }^{  }  $ for any matrix $A \in R^{N\times N}$  \footnote{
    The identity naturally appears from the standard inequality $\| v \|_{ \infty }^{  } \le  \| v \|_{2  }^{  } \le \sqrt{N}  \| v \|_{ \infty }^{  }   $ for $v \in R^{N}$, which simply comes from the fact that  $ \| v \|_{\infty  }^{ 2 } = \max_{i} \abs{ v_{i} }^{2} \le \sum_{i}^{n}   \abs{ v_{i} }^{2} = \| v \|_{ 2 }^{2  } \le N      \max_{i} \abs{ v_{i} }^{2} = N \| v \|_{
    \infty}^{ 2 }$. Now let $A\in \mathbb{R} ^{N \times N}$ be any matrix. We can then deduce that $ \| A \|_{ 2 }^{  } = \max_{v \in \mathbb{R} ^{N}} \frac{\| Av \|_{2  }^{  }}{\| v \|_{ 2 }^{  } } \ge \max_{v \in \mathbb{R} ^{N}}
    \frac{\| Av \|_{ \infty  }^{  }}{ \sqrt{N} \| v \|_{ \infty}^{  } } = \frac{1}{ \sqrt{N} } \| A \|_{ \infty }^{  }    $ and $ \| A \|_{ 2 }^{  } = \max_{v \in \mathbb{R} ^{N}} \frac{\| Av \|_{2  }^{  }}{\| v \|_{ 2 }^{  } } \le  \max_{v \in \mathbb{R} ^{N}}
    \frac{\sqrt{N} \| Av \|_{ \infty  }^{  }}{  \| v \|_{ \infty}^{  } } =  \sqrt{N} \| A \|_{ \infty }^{  }    $. Hence, the identity $\frac{1}{\sqrt{N} } \| A \|_{\infty  }^{  } \le \| A \|_{ 2 }^{  }\le \sqrt{N} \| A \|_{ \infty  }^{  }   $  is proven.

}. Applying this identity, we obtain the upper and lower bounds for $\kappa_{2}(A)$. Specifically, we have
\begin{equation}
\frac{1}{N} \kappa_{\infty} ( A) \le  \kappa_{2} ( A) \le N \kappa _{\infty}(A)
\end{equation}
Thus, since these norms are equivalent, will we in this thesis focus on $ \kappa_{\infty}( A)  $ because of the efficiency of computing $\| A \|_{ \infty }^{  } $  .

\subsection{Cléments interpolation}%
\label{ssub:clement_operator}
Our goal is to to utilize interpolation estimates to compute convergence rates. An important tool in the process is the so-called Cléments interpolation operator, $C_{h}$.
It is used for interpolation on non smooth functions by applying an regularization on so-called macroelements. However, we need to define affine operations on so-called macroelements before we can proceed with the error estimates.

A patch for a element $\omega \left( T \right) $ is denoted as the set of elements in $\mathcal{T} _{h}$  sharing at least one vertex with $T \in \mathcal{T} _{h}$. Similarly,  a patch of a facet $\omega \left( F \right) $ is defined as the set of all elements in $\mathcal{T}_{h} $
sharing at least one vertex with $F \in  \mathcal{F} _{h}$. For an illustrative example of patches in a two-dimensional triangular mesh, please refer to Figure \ref{fig:example_patch}.

\begin{figure}[ht!]
    \centering

  \centering
    \subfloat{{
    \begin{minipage}[b]{0.45\textwidth}
        \begin{tikzpicture}
            % Define the coordinates
            \coordinate (A) at (0,0);
            \coordinate (B) at (2,0);
            \coordinate (C) at (1,2);
            \coordinate (D) at (2.5,2);
            \coordinate (F) at (1.5,3);
            \coordinate (E) at (-1,2);
            \coordinate (G) at (1,-2);
            \coordinate (H) at (2.5,-2);
            \coordinate (I) at (4,-1);

            \fill[blue!30] (A) -- (G)-- (H) -- (I)  -- (D) -- (F)-- (E)   -- cycle;

            % Draw the triangle
            \draw (A) -- (G)-- (H) -- (I)  -- (D) -- (F)-- (E)   -- cycle;
            \draw (A) -- (C);
            \draw (A) -- (B);
            \draw (G) -- (B);
            \draw (I) -- (B);
            \draw (F) -- (C);
            \draw (H) -- (B);
            \draw (D) -- (C);
            \draw (D) -- (B);
            \draw (E) -- (C);
            \draw[line width=2pt] (B) -- (C);
            \node at (1.6,1.3) {$F$};

        \end{tikzpicture}
    \end{minipage}\hfill

}}%
    \qquad
    \subfloat{{

    \begin{minipage}[b]{0.45\textwidth}
        \begin{tikzpicture}
                    % Define the coordinates
        \coordinate (A) at (0,0);
        \coordinate (B) at (2,0);
        \coordinate (C) at (1,2);
        \coordinate (D) at (2.5,2);
        \coordinate (F) at (1.5,3);
        \coordinate (E) at (-1,2);
        \coordinate (G) at (1,-2);
        \coordinate (H) at (2.5,-2);
        \coordinate (I) at (4,-1);
        \coordinate (K) at (-2,-0.3);

        \fill[blue!30] (K) -- (G)-- (H) -- (I)  -- (D) -- (F)-- (E)   -- cycle;
        % \fill[red!30] (B) -- (C) -- (A);

        % Draw the triangle
        \draw (A) -- (G)-- (H) -- (I)  -- (D) -- (F)-- (E)   -- cycle;
        \draw (A) -- (C);
        \draw (A) -- (B);
        \draw (G) -- (B);
        \draw (I) -- (B);
        \draw (F) -- (C);
        \draw (H) -- (B);
        \draw (D) -- (C);
        \draw (D) -- (B);
        \draw (E) -- (C);
        \draw (K) -- (A);
        \draw (K) -- (G);
        \draw (K) -- (E);
        \draw[line width=2pt] (B) -- (C) -- (A) -- cycle;
        \node at (1.0,1.0) {$T$};

        \end{tikzpicture}
    \end{minipage}

    }}%
\caption{Illustration of the patch $\omega ( F) $ on the left-hand side and $\omega(T)$ on the right-hand side.}
    \label{fig:example_patch}%
\end{figure}

\begin{figure}[]
\begin{minipage}{.5\linewidth}
\centering
\subfloat[]{
    \label{fig:macroelements:a}
    \begin{tikzpicture}[scale=0.5]
        % Arbitrary triangle
        \coordinate (A1) at (0,0);
        \coordinate (B1) at (4,1);
        \coordinate (C1) at (1,3);
        \fill [blue!30] (A1) -- (B1) -- (C1) -- cycle;
        \draw (A1) -- (B1) -- (C1) -- cycle;

        % Centroid
        \coordinate (ai) at (barycentric cs:A1=1,B1=1,C1=1);
        \fill (ai) circle (2pt);
        \node[anchor=north east] at (ai) {$a_i$};

        % Reference triangle
        \coordinate (A2) at ($(A1) + (6, 0)$);
        \coordinate (B2) at ($(A2) + (0, 3)$);
        \coordinate (C2) at ($(A2) + (3, 0)$);
        \fill [red!30] (A2) -- (B2) -- (C2) -- cycle;
        \draw (A2) -- (B2) -- (C2) -- cycle;

        % Centroid of the reference triangle
        \coordinate (ahi) at (barycentric cs:A2=1,B2=1,C2=1);
        \fill (ahi) circle (2pt);
        \node[anchor=north west, xshift=-0.1cm, yshift=0.1cm] at (ahi) {$\hat{a}_{j( i)} $};

        \draw[->, thick, >=stealth] ($(ahi)+(0.2,+0.2)$) to[bend right] node[midway, above] {$\mathcal{G}_{A_i}$} ($(ai)+(0.2,+0.3)$);

    \end{tikzpicture}
}
\end{minipage}%
\begin{minipage}{.5\linewidth}
\centering
\subfloat[]{
    \label{fig:macroelements:b}
    \begin{tikzpicture}[scale=0.5]
    % Arbitrary square
    \coordinate (A1) at (0,0);
    \coordinate (B1) at (3,1);
    \coordinate (C1) at (4,4);
    \coordinate (D1) at (1,3);
    \fill[blue!30] (A1) -- (B1) -- (C1) -- (D1) -- cycle;
    \draw (A1) -- (B1) -- (C1) -- (D1) -- cycle;

    % Draw edge from D1 to B1
    \draw (D1) -- (B1);

    % Pick a point on the edge and label it as a_i
    \coordinate (a_i) at ($(D1)!.6!(B1)$);
    \fill (a_i) circle (2pt);
    \node[anchor=north] at (a_i) {$a_i$};

    % Reference equilateral triangle
    \coordinate (A2) at ($(A1) + (6, 0)$);
    \coordinate (B2) at ($(A2) + (2, 3.464)$); % 3.464 = 2 * sqrt(3)
    \coordinate (C2) at ($(A2) + (4, 0)$);
    \fill[red!30] (A2) -- (B2) -- (C2) -- cycle;
    \draw (A2) -- (B2) -- (C2) -- cycle;

    % Divide the equilateral triangle into two right triangles
    \coordinate (M) at ($(A2)!.5!(C2)$);
    \draw (B2) -- (M);

    % Pick a point on the shared edge and label it as ahat_i
    \coordinate (ahat_i) at ($(B2)!.7!(M)$);
    \fill (ahat_i) circle (2pt);
    \node[anchor=north west] at (ahat_i) {$\hat{a}_{j( i)} $};

    % Draw the mapping G_{A_i} from ahat_i to a_i
    \draw[->, thick, >=stealth] ($(ahat_i)+(0.2,0.2)$) to[bend right] node[midway, above] {$\mathcal{G}_{A_i}$} ($(a_i)+(0.2,0.2)$);
\end{tikzpicture}
}
\end{minipage}\par\medskip
\centering
\subfloat[]{
    \label{fig:macroelements:c}
 \begin{tikzpicture}[scale=0.5]

    % Central vertex a_i
    \coordinate (ah_i) at (0, 0);

    % Reference hexagon
    \foreach \angle in {0, 60, ..., 300} {
        \coordinate (A) at (\angle:2.5);
        \coordinate (B) at (\angle + 60:2.5);
        \fill[red!30] (ah_i) -- (A) -- (B) -- cycle;
        \draw (ah_i) -- (A) -- (B) -- cycle;
    }

    \fill (ah_i) circle (2pt);
    \node[anchor=east, yshift=0.25cm] at (ah_i) {$\hat{a}_{j( i) }$};

    \coordinate (A1) at (5, 0);
    \coordinate (B1) at (8, 0);
    \coordinate (C1) at (8, 3);
    \coordinate (D1) at (7, 4);
    \coordinate (E1) at (3.5, 4);
    \coordinate (F1) at (3, 2);
    \fill[blue!30] (A1) -- (B1) -- (C1) -- (D1) -- (E1) -- (F1) -- cycle;
    \draw (A1) -- (B1) -- (C1) -- (D1) -- (E1) -- (F1) -- cycle;

    \coordinate (ai) at (barycentric cs:A1=1,B1=1,C1=1,D1=1,E1=1,F1=1);
    % Draw lines from vertices to a_i
    \draw (A1) -- (ai);
    \draw (B1) -- (ai);
    \draw (C1) -- (ai);
    \draw (D1) -- (ai);
    \draw (E1) -- (ai);
    \draw (F1) -- (ai);
    % Centroid
    \fill (ai) circle (2pt);
    \node[anchor=south, yshift=0.2cm] at (ai) {$a_i$};

    \draw[->, thick, >=stealth] ($(ah_i)+(0.2,- 0.2)$) to[bend right] node[midway, above, yshift=0.1cm] {$\mathcal{G}_{A_i}$} ($(ai)+(-0.2,-0.2)$);

\end{tikzpicture}
}

\caption{Illustration of the different cases when mapping from the reference macroelement $\widehat{A}_{j( i) }$  to the domain $A_{i}$,  $\mathcal{G} _{A_{i}}: \widehat{A}_{j( i) } \to A_{i}$. Here we have defined $\hat{a}_{j(i)} \in
    \widehat{A}_{j(i)}$ s.t. $\mathcal{G}_{A_{i}} ( \hat{a}_{j( i) }) = a_i$. }

\label{fig:macroelements}
\end{figure}


 Let the set  $\left\{ a_{i}\right\}_{i\in N}$ be all Lagrange nodes on the mesh $\mathcal{T}_{h}$. Associated with each node $a_{i}$ we denote the macroelement $A_{i}$ to consist of all elements containing $a_{i}$. Let $n_{cf}$ be the number of configurations for the macroelement, then we define the index $j:
\left\{ 1,\ldots,N \right\} \to \left\{ 1, \ldots, n_{cf} \right\}  $ s.t. $j( i) $ is the index associated with the reference configuration $\widehat{A}_{j(i) }$ for corresponding macroelement $A_{i}$.
Let us define a $C^{0}$-diffeomorphism $\mathcal{G}_{A_{i}}:
\widehat{A}_{j( i) } \to A_{i}$ on the reference macroelements such that for all $\hat{T} \in \widehat{A}_{j( i) } $ is the restriction $\mathcal{G} _{A_{i}}|_{ \hat{T} }$ affine. For an illustration of the reference
macroelement $\widehat{A}_{j( i) }$ and how it related to $a_{i}$ , see
Figure \ref{fig:macroelements}.

The Cléments interpolation operator $C_{h}$ is the $L^2$-projection onto the macroelements. That is, given
a reference macroelement $\widehat{A}_{j( i) }$ and a function $\hat{v} \in L^{1}( \widehat{A}_{j( i) })  $, then $\widehat{C}_{j( i) } \hat{v}$  is the unique polynomial in $\mathcal{P}^{k} ( \widehat{A}_{j( i) })  $ s.t. \[
\int_{  \widehat{A}_{j( i) }}^{} ( \widehat{C}_{j( i) } \hat{v} - \hat{v}) p \ dx  = 0 \quad  \forall p \in \mathcal{P}^{k} ( \widehat{A}_{j( i) })
\]
Finally, the Cléments interpolator is defined as the mapping $C_{h} : L^{1}( \Omega )  \to \mathcal{P} ^{k}_{c}(\mathcal{T}_{h}   ) $ such that
\[
C_{h} v = \sum_{i=1}^{N} \widehat{C}_{j( i) } ( v (\mathcal{G} _{A_{i}}) (\mathcal{G}^{-1}_{A_{i}}(a_{i})) )\phi _{i},
\]
where $\phi _{i}$ is the corresponding polynomial basis associated with the node $a_{i}$.

% Recall the general Sobolev norm notation,
% \[
% \| u \|_{ m,p,T }^{  } = \left( \sum_{ \left\lvert \alpha  \right\rvert \le m}^{} \int_{T}^{}  \left\lvert  \partial ^{\alpha } u \right\rvert^{p} dx   \right)^{\frac{1}{2}}
% \]
% where we use the convenient notation $\| u \|_{L^2(T) }^{  } = \| u \|_{ T  }^{  } = \| u \|_{ 0,2,T  }^{  } $ and similarly $\| u \|_{ H^r( T )  }^{  } = \| u \|_{ r,T  }^{  } = \| u \|_{ r,2,T  }^{  }  $.


Finally, we have the following a priori estimate.

\begin{lemma}
    \label{lemma:clements}

    Let $v \in H^{s}( \Omega ) $. We define the Clement interpolation as the mapping
$C_{h}: L^{2}( \Omega )   \to  V_{h}$, where $V_{h}$ has the order $k$. Then does the following stability estimate hold,
\[
 \| C_{h} v \|_{ s, \Omega     }^{  } \lesssim \| v \|_{ s, \Omega   }^{  } \quad \forall v \in H^{s}( \Omega ) ,
\]
Let $r = \mathrm{min} ( s, k+1) $. If the following conditions for an parameter $l$ is satisfied, it exists error estimates such that
\begin{equation}
    \begin{split}
      0\le l \le r  \implies \| v - C_{h} v \|_{ m,T   }^{  }  &  \lesssim h^{r-l}_{T} \| v \|_{l,\omega \left( T \right)  }^{  } \quad  \forall T \in \mathcal{T} _{h}, \forall v \in H^{l}( \omega \left( T \right)
      ), \\
      0\le l \le r-\frac{1}{2}  \implies \| v - C_{h} v \|_{ m,F }^{  } & \lesssim h^{r - l - \frac{1}{2}}_{T} \| v \|_{l,\omega \left( F \right)  }^{  } \quad  \forall \partial T \in \mathcal{T} _{h}, \forall v \in H^{l}( \omega \left( F
      \right)).
    \end{split}
\end{equation}

\end{lemma}


\begin{corollary}
    \label{cor:celement_apriori}
    Let $0 \le l \le k+1$ and let $0\le m \le \min_{} ( 1,l )$.
    Given Lemma \ref{lemma:clements}  then does the following estimate hold
    \[
    \inf_{v_{h} \in V_{h} } \| v - v_{h} \|_{  m,\Omega }^{  } \lesssim   h^{l-m}  \| v \|_{ l,\Omega  }^{  } \quad     \forall v \in H^{l}( \Omega ).
    \]
\end{corollary}
This result is very useful since it is now sufficient to show that a priori estimates holds to prove convergence rate. For further detailed information about the Cléments interpolation, please investigate \cite[Chapter 1.6]{ern04}.

\subsection{Useful local inverse estimates}%
\label{sub:some_general_inequalities}


    Choose any element $T \in \mathcal{T}_{h} $ and let $v_{h} \in \mathcal{P} ^{m}( T)   $. Then does the local inverse estimate hold,
\begin{equation}
\label{eq:inv1}
\abs{ v_{h} }_{H^{l}( T) }  \lesssim h^{m-l} \abs{ v_{h} }_{H^{m}( T  ) }
\text{ for } l \le m.
\end{equation}
 For proof, see \cite[Lemma 12.1]{ErnGuermond2021}. An essential example is the following inequality.
 \begin{equation}
     \label{eq:degrade}
\| D^2v_{h} \|_{T  }^{  } \lesssim h^{-1} \| \nabla v_{h}  \|_{ T  }^{  } \lesssim h^{-2} \| v_{h} \|_{T  }^{  }
 \end{equation}

Another very useful inequality is the so-called trace inequality which connect the relationship of evaluating the norm on element $ T $ and with any of the corresponding facets $F \in \partial T$. The general form is
\begin{equation}
    \label{eq:inv2}
\| v_{h} \|_{F   }^{  }  \lesssim
h^{-\frac{1}{2}} \| v_{h} \|_{ T  }^{  },
\end{equation}
For proof, see \cite[Lemma 12.8]{ErnGuermond2021}.

Let $\partial _{n} v = \nabla v \ n$ and $\partial_{nn} v = n^{T} D^2 v \ n $. Keeping in mind that the normal vector has a unit length and, thus, evidently applying the
trace inverse inequality we have two present useful examples,
\begin{equation}
    \label{eq:fund_inv_est}
\begin{split}
    \| \partial _{n} v_{h} \|_{F  }^{  }  & \le \| \nabla v_{h} \|_{F  }^{  }  \le h^{-\frac{1}{2}} \| \nabla  v_{h} \|_{T  }^{  },  \\
    \| \partial _{nn} v_{h} \|_{ F }^{  } & \le  \| D^2 v_{h} \|_{ F }^{  }   \le  h^{-\frac{1}{2}} \| D^2 v_{h} \|_{ T }^{  }.
\end{split}
\end{equation}
Combining \eqref{eq:inv1} and \eqref{eq:inv2}, we establish that
\begin{equation}
    \label{eq:general}
\abs{ v_{h} }_{l,F}  \lesssim h^{m-l - \frac{1}{2}} \abs{ v_{h} }_{m, T}
\text{ for } l \le m.
\end{equation}

\subsection{Abstract error estimate}%
\label{sub:ceas_lemma}

Since $V_{h} \not \subset V_{h}$, it is clear that we cannot directly employ the standard Cea's lemma \cite[p. 66]{quartdiff}. Yet, given the discrete stability \eqref{eq:ah_star} and discrete Galerkin orthogonality \eqref{eq:galerin_orth}, coupled with an additional
requirement of on $u$, we are indeed equipped to formulate an effective workaround.

\begin{equation}
\label{eq:cealemma_proof}
    \begin{split}
 \| v_{h} -u_{h} \|_{ V_{h}  }^{ 2 } & \lesssim   a_{h}( v_{h} - u_{h}, v_{h} - u_{h} )    \\
 &= a_{h}( v_{h} - u, v_{h}- u_{h}) + a_{h}( u - u_{h}, v_{h} - u_{h})   \\
 & \lesssim \| v_{h}- v \|_{ V_{h},* }^{  } \| v_{h}- u_{h} \|_{V_{h} }^{  }   \\
    \end{split}
.\end{equation}
Hence, we now have abstract error estimate
\begin{equation}
\| u - u_{h} \|_{ V_{h},*  }^{  }  \lesssim  \inf_{v_{h} \in  V_{h}}  \| v - v_{h} \|_{ V_{h},*  }^{  } = 0 \quad  \forall v \in V.
\end{equation}
For more information, see \cite[Theorem 1.35]{pietro2012}.
A useful property is that for a conformal numerical method to converge can we now simply require
\begin{equation}
\lim_{h \to 0}  \inf_{v_{h} \in  V_{h}}  \| v - v_{h} \|_{ V_{h},*  }^{  } = 0 \quad  \forall v \in V.
\end{equation}
In that case will $\| u - u_{h} \|_{ V_{h},*  }^{  }  \to  0$, $h \to  0$. Hence, if this requirement is fulfilled, the numerical methods will converge towards the unique solution.
Note that in combination with Corollary \ref{cor:celement_apriori} is this very useful for estimating a priori estimates.



\section{Continuous interior penalty methods for the biharmonic problem with Cahn-Hilliard type boundary conditions}%
\label{sec:CIP_biharmonic_problem}

One of the objective of this section is to discuss the strong formulation for the biharmonic problem.
Following this, we will present both the continuous weak formulation and the derivation of the two proposed discrete weak formulations, specifically the continuous interior penalty methods.
We then present a short discussion of the current status of the properties of the methods.

\subsection{The biharmonic equation}
\label{sub:the_biharmonic_problem_with_c_h_boundary_conditions}

Let $\Omega \subseteq    \mathbb{R} ^d$ be a bounded polygonal domain and $\Gamma $ be its corresponding boundary. Also let $\mathcal{T}_{h} = \left\{ T \right\} $ be a shape-regular fitted mesh s.t. $\mathcal{T}_{h} = \Omega $. Let the biharmonic problem have the form,
\begin{subequations}
\begin{align}
    \Delta^2  u  + \alpha  u  & = f( x)  \quad \text{in } \Omega,   \label{eq:bi_problem_a}\\
    \partial _{n} u & = g_{1}(x)   \quad \text{on } \Gamma ,  \label{eq:bi_problem_b}\\
    \partial _{n} \Delta  u & = g_{2}( x)   \quad \text{on } \Gamma .  \label{eq:bi_problem_c}
\end{align}
\label{eq:bi_problem}
\end{subequations}
Here is $\Delta ^2 = \Delta  \left( \Delta  \right) $ the biharmonic operator, also known as the bilaplacian. We will assume for the strong form that $u \in H^{4}\left( \Omega  \right) $, $\alpha  >0 $ and $f \in L^{2}\left( \Omega  \right)
$. The functions $g_{1},g_{2}: \Omega  \to \mathbb{R}$ are denoted as boundary conditions similar to the CH problem.

\begin{remark}
It is worth noting that the problem is closely related to the Kirchhoff's plate problem by changing the boundary conditions such that $u = \partial _{n } u = 0$ on $\Gamma $, which is in the literature known as so-called clamped boundary conditions.
Many of the papers we refer to may consider clamped boundary condition and not the CH boundary conditions. The main difference relies on if the problem is treated with homogeneous or non-homogeneous boundary conditions and if the discrete space is
imposing the Dirichlet and Neumann conditions strongly in the discrete solution space or weakly using the Nitsche's method \cite{nitsche1971variationsprinzip}.
\end{remark}

We want to construct a weak form for the strong biharmonic problem \eqref{eq:bi_problem}. Let $v \in H^{2}( \Omega ) $  Using Greens Theorem is it obvious that \(
\left( \Delta ^2 u,v \right) _{ \Omega  }   = ( \partial _{n} \Delta u, v ) _{\Gamma   } - ( \nabla \left( \Delta  u \right) , \nabla v ) _{ \Omega }
\).
Next, applying a new iteration of the Greens theorem we get
$ -( \nabla ( \Delta u ) , \nabla v ) _{\Omega  }  =   ( \Delta u, \Delta v ) _{\Omega } - ( \Delta u, \partial _{n}v )_{\Gamma } $.
Hence, we obtain the identity
\begin{equation}
\label{eq:iden_bi}
( \Delta ^2 u, v ) _{\Omega } = ( \Delta u, \Delta v)_{\Omega } +  ( \partial _{n} \Delta u, v)_{\Gamma } - ( \partial _{n} v, \Delta u) _{\Gamma }
\end{equation}
Taking into account the boundary conditions, we end up with the following corresponding weak formulation of the biharmonic problem \eqref{eq:bi_problem}.

\begin{equation}
\label{eq:V_deg}
V := \left\{ v \in  H^s( \Omega ), \  s \ge  5  /2 + \varepsilon   \mid \partial _{n} v  = g_{1}     \right\}
\end{equation}
Consider the bilinear form $a:V\times V \to \mathbb{R}$ and the linear form $l: V \to \mathbb{R}$. We define the continuous weak formulation problem formulation as follows.


\begin{equation}
\label{eq:cont_weak_problem}
\text{Find } u \in V   \text{ such that } a( u,v) = l( v)  \forall v \in V
\end{equation}
where
\begin{equation}
    \begin{split}
a( u,v) & =  ( \alpha u,v)_{\Omega } + ( \Delta u, \Delta v)_{\Omega } - (  \Delta u, \partial _{n} v) _{\Gamma }\\
l( v)  &= ( f,v)_{\Omega } - ( g_{2} , v)_\Omega
    \end{split}
\end{equation}
Note that the Neumann boundary condition $g_{1}$ is strongly imposed in $V$, and $g_{2}$ is naturally incorporated in the weak formulation.


\subsection{Detailed construction of Hessian and Laplacian Formulations }%
\label{sub:construction_of_laplacian_cip}
The goal is to construct two CIP formulations for the problem \eqref{eq:cont_weak_problem}, that is: Find $u_{h} \in V_{h} \not\in V$ such that $a_{h}( u_{h}, v_{h}) = l_{h}( v_{h}) $ for all $v_{h} \in V_{h}$. We will follow the ideas presented in \cite{brenner2012} and \cite{feng2007fully}.
A property that is necessary is that when we have a bilinear form and replace the exact solution with $u_{h} \in V_{h}$, the system still remains consistent in $V$. Hence, we guarantee a consistency of the discrete weak formulation by assuming during
the construction that $u \in H^{4}( \Omega ) $ and $v_{h} \in  V_{h}$ .
 However, due to the nonconformal nature of $V_{h}$, it becomes necessary to introduce penalty terms to ensure the discrete system is well-posed when we replace with $u_{h} \in V_{h}$. Keep in mind that the $C^{1}$
continuity is imposed weakly and in the same way is the Neumann conditions also imposed weakly.

To achieve the objective of constructing the Hessian and Laplacian formulations, the following lemmas will be the primary components.

\subsubsection{Construction of the Hessian formulation}%
\label{ssub:construction_of_the_hessian_formulation}



\begin{lemma}
    \label{lemma:hessian}
    Assume the homogeneous Neumann conditions $g_1 = 0 $.
Let $u \in H^{4}( \Omega ) $ be the solution to \eqref{eq:bi_problem}, let $ v_{h} \in V_{h}$ and a constant $\gamma >0$. Then does the following identity hold.
\begin{equation}
\label{eq:bi_basic_dg_full}
\begin{split}
    \left( \Delta  ^{2} u, v_h \right) _{\Omega }  =&   \left( D^2u, D^2v_h \right)_{\Omega } +  \left(g_{2}, v_h  \right) _{ \Gamma  }\\
    &    -  ( \mean{ \partial _{nn} u }   , \jump{ \partial_{n} v_h } )_{\mathcal{F}_{h}^{int} } -  (  \jump{ \partial_{n} u
    },\mean{ \partial _{nn} v_h } )_{\mathcal{F}_{h}^{int} } + \frac{\gamma }{h} (  \jump{ \partial_{n} u
    },\mean{ \partial _{nn} v_h } )_{\mathcal{F}_{h}^{int} } \\
    & - ( \partial _{nn} u , \partial _{n} v_h)_{\Gamma  }- ( \partial _{n} u , \partial _{nn} v_h)_{\Gamma  } + \frac{\gamma }{h}  \left(  \partial _{n} u,  \partial _{n} v_{h}      \right)_{\Gamma }
\end{split}
\end{equation}
\end{lemma}


\begin{figure}[h]
\centering
    \begin{tikzpicture}
        % Define points for square
        \coordinate (A) at (-0.5,-0.5);
        \coordinate (B) at (1.0,0.0);
        \coordinate (C) at (1.5,1.5);
        \coordinate (D) at (-1.5,1.5);
        \filldraw[fill=blue!30,draw=black] (A) to[out=0,in=-90] (B) to[out=90,in=0] (C) to[out=180,in=90] (D) to[out=-90,in=180] cycle;


        \def\lenfactorn{4.5} % change this value to easily adjust the length of n
        \def\lenfactort{4.5} % change this value to easily adjust the length of t
        \draw[->, line width=0.5pt] ({1.8*cos(90)}, {1.8*sin(90)}) -- ({1.8*cos(90) + \lenfactorn*(2.5*cos(85) - 1.8*cos(90))}, {1.8*sin(90) + \lenfactorn*(2.5*sin(85) - 1.8*sin(90))}) node[below right, xshift=-0.0cm] {$n $};
        \draw[->, line width=0.5pt, rotate around={90:({1.8*cos(90)}, {1.8*sin(90)})}] ({1.8*cos(90)}, {1.8*sin(90)}) -- ({1.8*cos(90) + \lenfactort*(2.5*cos(85) - 1.8*cos(90))}, {1.8*sin(90) + \lenfactort*(2.5*sin(85) - 1.8*sin(90))}) node[below left, xshift=-0.2cm] {$t $};

        \def\lenfactorPf{2.5} % change this value to easily adjust the length of Pf
        \def\lenfactorQf{2.1} % change this value to easily adjust the length of Qf
        \draw[->, line width=1.0pt] ({1.8*cos(90)}, {1.8*sin(90)}) -- ({1.8*cos(90) + \lenfactorPf*(2.5*cos(85) - 1.8*cos(90))}, {1.8*sin(90) + \lenfactorPf*(2.5*sin(85) - 1.8*sin(90))}) node[below right, xshift=-0.0cm] {$Q_F\nabla u$};
        \draw[->, line width=1.0pt, rotate around={90:({1.8*cos(90)}, {1.8*sin(90)})}] ({1.8*cos(90)}, {1.8*sin(90)}) -- ({1.8*cos(90) + \lenfactorQf*(2.5*cos(85) - 1.8*cos(90))}, {1.8*sin(90) + \lenfactorQf*(2.5*sin(85) - 1.8*sin(90))}) node[below left, xshift=-0.2cm] {$P_F\nabla u$};
        \def\lenfactorGradu{1.3} % change this value to easily adjust the length of \nabla u
        \draw[->, line width=1.0pt] ({1.8*cos(90)}, {1.8*sin(90)}) -- ({1.8*cos(90) + \lenfactorGradu*(3.5*cos(100) - 1.8*cos(90))}, {1.8*sin(90) + \lenfactorGradu*(3.5*sin(100) - 1.8*sin(90))}) node[below right, xshift=-0.0cm] {$\nabla u$};

        \node[circle,fill,inner sep=1pt,label={below:$x_0$}] at ({1.8*cos(90)}, {1.8*sin(90)}) {};
        \node at (0,0) {$\Omega$};
        \node at (2,1.7) {$\Gamma$};

    \end{tikzpicture}
\caption{Let $x_{0}$ be a point at the boundary $\Gamma $ for dimension $d=2$. Here is a illustration of the gradient $\nabla u$ with the corresponding normal and tangential decomposition, $Q_{F} \nabla  u$ and $P_{F} \nabla  u$.    }
\label{fig:projection_QF_PF}
\end{figure}
\begin{proof}

 We will start constructing a local theory for a element $T$ and then extend it to the full mesh
$\mathcal{T}_{h} $. Using Greens Theorem is it obvious that
\begin{equation}
    \label{eq:id1}
\left( \Delta ^2 u,v_h \right) _{T }   = \left( \partial _{n} \Delta u, v_h \right) _{\partial T  } - \left( \nabla \left( \Delta  u \right) , \nabla v_h \right) _{T }
\end{equation}

We can expand the second term in the following way.
\begin{equation}
    \label{eq:id2}
    \begin{split}
( \nabla ( \Delta u ) , \nabla v_h ) _{T } & = \sum_{i = 1}^{ d}  ( \Delta  \partial _{x_{i}} u, \partial _{x_{i}}v_h ) _{T }
                                           = \sum_{i = 1}^{d}  ( \nabla \cdot ( \nabla \partial _{x_{i}} u ) , \partial _{x_{i}} v_h )_{T }  \\
&= \sum_{i = 1}^{d}  \left( ( \partial_n  \partial _{x_{i}} u,  \partial _{x_{i}} v_h ) _{\partial T } -   ( \nabla \partial _{x_{i}} u, \nabla \partial _{x_{i}} v_h )_{T } \right) \\
& = (  \partial_n\nabla u, \nabla v_h ) _{\partial_{} T  } - ( D^2 u, D^2v_h ) _{T }
    \end{split}
.\end{equation}
Hence, the normal flux of $\Delta u$ appears naturally into the formulation. It can be denoted that $D^2$ is the Hessian matrix operator. Also remark that we apply the notation
$( D^2u, D^2v_h )_{\Omega } = \int_{\Omega }^{} D^{2}u : D^2v_h  dx$ for the inner product $D^2u:D^2v_h$.

Next, we want to decompose the evaluation of $\nabla  u $ on the boundary $\partial T$ in the tangential and normal direction. Pick a facet  $F \in \partial T$, then we define the following decomposition of linear transformation $\nabla u = P_{F}\nabla u  + Q_{F}  \nabla u  $ s.t. the
orthogonality, $
P_{F} \nabla u  \cdot Q_{F}  \nabla u = 0$, holds. Here, the normal projection matrix is defined as $Q_{F} = n \otimes n $ and the tangential decomposition follows from $ P_{F} = I - Q_{F} = I - n \otimes n  =  \sum_{i=1}^{d-1} t_{i} \otimes t_i$,
where we defined a orthonormal basis $t_{i}$, $i = 1, \ldots, d-1$ for the space orthogonal to the outer normal vector $n$ on a facet $F$. For demonstration in $d=2$, see Figure \ref{fig:projection_QF_PF}. Let $ a_{1}, a_{2}, a_{3} \in \mathbb{R} ^{d}$ be any vectors, then it is well known that the following identity holds $ ( a_{1}
\otimes a_{2}  ) a_{3} = ( a_{2}^{T}  a_{3}) a_{1} $. Hence, we have
\begin{equation}
\label{eq:projection}
    \begin{split}
   Q_{F} \nabla u & = ( n \otimes n ) \nabla u =  (n^{T} \nabla u)n \\
   P_{F} \nabla u & =( I - n \otimes n ) \nabla u =   \nabla u  - (n^{T}  \nabla u)n =  \sum_{ i =1 }^{d-1} ( t_{i}^{T}  \nabla u ) t_{i}
    \end{split}
\end{equation}

Given that $u$ is evaluated only on $\partial T$ can we write
$\nabla u = \left( n^{T} \nabla u   \right) n + \sum_i^{d-1} \left( t_i^{T} \nabla u   \right) t_i$ such that,
\begin{equation}
\label{eq:id3}
    \begin{split}
(  \partial_n\nabla u, \nabla v_h ) _{\partial_{} T  } & =  ( \partial _{n} ( \partial_{n}u \ n), \partial _{n} v_h \ n )_{\partial T}   +\sum_{i,j=1}^{d-1} ( \partial _{n} ( \partial_{t_{i}}u \ t_{i}), \partial _{t_{j}} v_h \ t_{j} )_{\partial T} \\
& =  ( \partial _{nn} u, \partial _{n} v_h  )_{\partial T}+\sum_{i=1}^{d-1} ( \partial _{n t_{i}}u , \partial _{t_{i}} v_h  )_{\partial T}
    \end{split}
\end{equation}
Here we used that $n^{T} n = 1$ and $t_{i}^{T} t_{j} = \delta_{ij}$.
Remark that simple relation was applied,
    \begin{align*}
\partial_n (\partial_n u)  & = n^T \nabla (\partial_n u)  = n ^T (D^2 u \ n)  = n^{T} D^2 u \ n = \partial _{nn} u, \\
\partial_n (\partial_{t_{i}} u)  & = t_{i}^T \nabla (\partial_n u)  = t_i^T (D^2 u \ n )   = n^{T} D^2 u \ t_{i} = \partial _{n t_{i}} u.
    \end{align*}
We may also deduce the relationship $\partial _{nt_{i}} u = \partial _{t_{i}n}u$ which arise from the fact that $n^{T} D^2u \ t_{i} = ( t_{i}^{T} D^2u \  n)^T = t_{i}^{T}  D^2u \  n$, where we utilized the symmetry $D^2u = ( D^2u) ^{T} $ and that the
product is a scalar.
Combining \eqref{eq:id1}, \eqref{eq:id2} and \eqref{eq:id3} we see that,
\[
    ( \Delta ^2 u, v_h) _{T}   = ( D^2 u, D^2v_h)_{T } + ( \partial _{n}  \Delta u, v_h )_{\partial T} -( \partial _{nn}u , \partial _{n} v_h  )_{\partial T}-\sum_{i=1}^{d-1} ( \partial _{n t_{i}}u, \partial _{t_{i}} v_h  )_{\partial T}
\]
Since we aim construct a identity for the full mesh $\mathcal{T} _{h}$, we sum over the elements.
\begin{equation}
\label{eq:bi_basic_dg2}
\left( \Delta  ^{2} u,v_h \right) _{\Omega } = \sum_{T \in  \mathcal{T} _{h}}^{}  ( D^2 u, D^2v_h)_{T } + ( \partial _{n}  \Delta u, v_h )_{\partial T} -( \partial _{nn}u , \partial _{n} v_h  )_{\partial T}-\sum_{i=1}^{d-1} ( \partial _{n t_{i}} u , \partial _{t_{i}} v_h  )_{\partial T}
\end{equation}
Our goal is to simplify the equation above so we can take account for discontinuities of the derivatives.
By integrating over exterior facets $\mathcal{F} _{h}^{ext}$ and interior facets $\mathcal{F} _{h}^{int}$ we will get e more suitable formulation which makes it easier to control the jumps between the elements, hence makes it possible to penalize discontinuities.

\begin{equation*}
    \begin{split}
 ( \Delta  ^{2} u,v_h ) _{\Omega }  =&\sum_{T\in \mathcal{T} _{h}}^{} ( D^2u,D^2v_h ) _{T }  + (\partial _{n} \Delta  u,v_h)_{\partial T} - (\partial _{nn} u, \partial _{n}v_h )_{\partial T}  - \sum_{i=1}^{d-1} ( \partial _{t_{i}n}u , \partial _{t_{i}} v_h  )_{\partial T}   \\
= &\sum_{T\in \mathcal{T} _{h}}^{} ( D^2u,D^2v_h ) _{T }  + \sum_{F \in \mathcal{F}_{h}^{ext} }^{}  (\partial _{n} \Delta  u,v_h)_{F} - (\partial _{nn} u, \partial _{n}v_h )_{F}  - \sum_{i=1}^{d-1} ( \partial _{ t_{i}n} u , \partial _{t_{i}} v_h
)_{F}     \\
   &  + \sum_{F \in \mathcal{F} _{h}^{int}}^{} \underbrace{\left( (\partial _{n^{+}} \Delta  u^{+}
        ,v_h^{+} )_{F}
+ \left(\partial _{n^{-}} \Delta  u^{+} ,v_h^{-}\right)_{F}  \right)}_{(I)}    \\
    &\quad \quad  -
\underbrace{\left( \left(\partial _{n^{+}n^{+}} u^{+}, \partial _{n^{+}} v_h^{+} \right) _{F} + \left(\partial _{n^{-}n^{-}} u^{-}, \partial _{n^{-}} v_h^{-}
\right) _{F} \right) }_{(II)} \\
   &  \quad \quad - \sum_{i=1}^{d-1}\underbrace{( (\partial _{n^{+}t_{i}} u^{+}, \partial_{t_{i}} v_h^{+} )_{F} +  \left(\partial _{n^{-}t_{i}} u^{-},
        \partial_{t_{i}} v_h^{-}
\right)_{F} ) }_{(III)} \\
    \end{split}
.\end{equation*}

Where integration of the interior facets is computed in the following fashion.
\begin{equation}
    \label{eq:dg2_facets}
    \begin{split}
        (I) &  =    \left(\partial _{n^{+}} \Delta  u^{+} ,v_h^{+}\right)_{F} +
        \left(\partial _{n^{-}} \Delta  u^{-} ,v_h^{-}\right)_{F}  \\
            & =   \int_{F}^{}
            \jump{ \partial _{n} \Delta  u \cdot v_h } =
            \int_{F}^{}
            \mean{ \partial _{n} \Delta  u } \underbrace{\jump{ v_h }}_{= 0}    + \underbrace{\jump{ \partial _{n} \Delta  u
            }}_{= 0}    \mean{ v_h } = 0 \\
            (II) &  =     \left(\partial _{n^{+}n^{+}} u^{+}, \partial_{n^{+}} v_h^{+} \right)_{F} +  \left(\partial _{n^{-}n^{-}} u^{-}, \partial_{n^{-}} v_h^{-} \right)_{F}    \\
                 &= \int_{F}^{} \jump{ \partial _{nn} u \cdot  \partial_{n} v_h }   = \int_{F}^{}
                       \mean{ \partial _{nn} u    } \underbrace{\jump{ \partial_{n} v_h }  }_{\neq 0}    + \underbrace{\jump{ \partial
                               _{nn}  u
                       }}_{= 0}    \mean{ \partial _{n}v_h } \\
            (III) &  =     \left(\partial _{n^{+}t_{i}} u^{+}, \partial_{t_{i}} v_h^{+}
                \right)_{F} +  \left(\partial _{n^{-}t_{i}} u^{-}, \partial_{t_{i}} v_h^{-}
                \right)_{F}   \\
                 &  =   \int_{F}^{}
                 \jump{ \partial _{nt_{i}} u \cdot  \partial_{t_{i}} v_h } =
                 \int_{F}^{}
                 \mean{ \partial _{nt_{i}} u    } \underbrace{\jump{ \partial_{t_{i}} v_h }  }_{= 0}    + \underbrace{\jump{ \partial
                         _{nt_{i}}  u
                 }}_{= 0}    \mean{ \partial _{t_{i}}v_h }  = 0
                   \end{split}
.\end{equation}
Observe that the cancellations in the term $(I)$ and term $(III)$  appears of the continuity of $v_h\in V_{h} $ and $u\in H^{4}( \Omega ) $ which makes the jumps and derivative jumps zero. On the other hand, the second term $(II)$  does not vanish
since the derivative of $v_h \in V_{h}$ has a nonzero jump. It can also be raised that $\mean{
\partial _{nn} u } = \partial _{nn} u  $ holds of $H^{4}( \Omega  ) $.

Combining \eqref{eq:dg2_facets} and inserting the boundary condition $g_{2} = \partial _{n} \Delta u $ is it clear that the formulation presented in \eqref{eq:bi_basic_dg2} is equivalent to the following formulation.
\begin{equation}
\label{eq:bi_basic_dg_full_1}
\begin{split}
    \left( \Delta  ^{2} u, v_h \right) _{\Omega }  =&   \left( D^2u, D^2v_h \right)_{\mathcal{T} _{h}} +  \left(g_{2}, v_h  \right) _{\Gamma  }  -  ( \mean{ \partial _{nn} u }   , \jump{ \partial_{n} v_h } )_{\mathcal{F}_{h}^{int} } \\
                                                  &  - ( \partial _{nn} u , \partial _{n} v_h)_{\mathcal{F}^{ext}_{h} } - \sum_{i =1  }^{d-1} ( \partial   _{t_{i}n} u  ,  \partial   _{t_{i}}  v_h  )_{ \mathcal{F}^{ext} _{h}  }
\end{split}
\end{equation}
Under the assumption that $g_{1} = 0$ on $\Gamma$, and given that the tangential decomposition is orthogonal to $n$, we can assert that $ \partial_{t_{i} n} u = \partial_{t_{i}} ( \partial _{n} u )= \partial_{t_{i}} ( g_{1} ) = 0 $ holds for any $i =
1,\ldots, d -1$. This implies that the
last term of the equation vanish.

We also note that we add consistent symmetry terms $( \mean{ \partial _{nn} v_h } ,\jump{ \partial _{n} u }    )_{\mathcal{F}^{int}_{h} } $ and $( \partial _{nn} v_h  , \partial _{n} u     )_{\Gamma  } $ in addition to the penalty terms $\frac{\gamma
}{h} ( \partial _{n} u, \partial _{n}v)_{\Gamma }  $ and $\frac{\gamma
}{h} ( \jump{ \partial _{n} u }  , \jump{    \partial _{n}v})_{\mathcal{F}^{int}_{h} }  $ . Since $u\in H^{4}( \Omega ) $ and the boundary condition,  $\partial _{n} u = g_{1}=0$ on $\Gamma $, is each of these terms effectively zero, but does provide symmetry and will later
be proven to be essential for well-posedness for the discrete problem. Finally, we have
\begin{equation}
\begin{split}
    \left( \Delta  ^{2} u, v_h \right) _{\Omega }  =&   \left( D^2u, D^2v_h \right)_{\Omega } + \left(g_{2}, v_h  \right) _{ \Gamma  }\\
    &    -  ( \mean{ \partial _{nn} u }   , \jump{ \partial_{n} v_h } )_{\mathcal{F}_{h}^{int} } -  (  \jump{ \partial_{n} u
    },\mean{ \partial _{nn} v_h } )_{\mathcal{F}_{h}^{int} } + \frac{\gamma }{h}  (  \jump{ \partial_{n} u },\jump{ \partial _{n} v_h } )_{\mathcal{F}_{h}^{int} } \\
    & - ( \partial _{nn} u , \partial _{n} v_h)_{\Gamma  } - (\partial _{n} u, \partial _{nn} v_h )_{\Gamma  }  + \frac{\gamma }{h}  \left(  \partial _{n} u,  \partial _{n} v      \right)_{\Gamma }
\end{split}
\end{equation}

The proof is complete.

\end{proof}

Note that since $V_{h} \not\subset V $ is it necessary to define the space $V \oplus V_{h}$, which essentially is the direct sum of these two spaces. This new space includes all elements from $V$ and $V_h$ and all possible linear combinations of these elements. i.e., let $u \in V$ and $u_{h} \in V_{h}$, then $u + u_{h} \in V \oplus V_{h} $.

We will now assemble the Hessian CIP formulation.
Assume that the homogeneous boundary condition $g_{1}=0$.
The discrete problem is as follows:
 \begin{equation}
    \label{eq:hessian_prob}
    \text{Find } u_{h} \in V_{h} \text{ such that } a^{H}( u_{h}, v_{h})  = l_{h}^{H}( v_{h} )  \quad \forall v_{h} \in  V_{h}.
\end{equation}
Here is the corresponding bilinear and linear form  defined as,
\begin{equation}
    \label{eq:hessian_form}
\begin{split}
a_{h}^{H} \left( u_{h}, v_{h} \right)   =&
    \left( \alpha  u_{h}, v_{h} \right) _{\Omega }   +  \left( D^2 u_{h}, D^2v_{h} \right) _{\Omega } \\
 & - \left( \mean{  \partial _{n n} u_{h} }, \jump{ \partial _{n }v_{h}} \right)_{\mathcal{F}_{h}^{int}}  -
 \left( \jump{ \partial _{n}u_{h} }, \mean{ \partial _{n n} v_{h} } \right)_{\mathcal{F}_{h}^{int}}  + \frac{\gamma }{h}  \left( \jump{ \partial _{n} u_{h}}, \jump{ \partial _{n} v_{h}   }   \right)_{\mathcal{F}_{h}^{int}} \\
 & - \left(   \partial _{n n} u_{h} ,  \partial _{n }v_{h} \right)_{\Gamma }  -
 \left( \partial _{n}u_{h} , \partial _{n n} v_{h}       \right)_{\Gamma }  + \frac{\gamma }{h}  \left(  \partial _{n} u_{h},  \partial _{n} v_{h}      \right)_{\Gamma }   \\
 l_{h}^{H}( v_{h})  =&  \left( f, v_{h} \right) _{\Omega }  - \left(g_{2}, v_{h}  \right) _{\Gamma }
\end{split}
.
\end{equation}

With the corresponding energy norms,
\begin{equation}
\label{eq:a_cip_energy_norm_hes}
    \begin{split}
 \| v_{h} \|_{ a_{h}^{H} }^{ 2 }& = \alpha  \| v_{h}\|_{ \Omega  }^{2  }  +  \| D ^2 v_{h} \|_{ \Omega   }^{ 2 }  + \|  h^{-\frac{1}{2}} \jump{ \partial _{n} v_{h}    }\|_{  \mathcal{F} _{h}^{int} }^{2  }+ \|  h^{-\frac{1}{2}}  \partial _{n} v_{h}
 \|_{  \Gamma  }^{2  },  \quad v_{h} \in V_{h}  \\
   \| v \|_{ a_{h}^{H},* }^{ 2 } &= \| v \|_{ a_{h} }^{ 2 }  + \| h^{\frac{1}{2}}  \mean{     \partial _{nn } v}  \|_{ \mathcal{F}_{h}^{int}   }^{  2}+ \| h^{\frac{1}{2}} \partial _{nn } v  \|_{ \Gamma    }^{  2}, \quad  v\in V \oplus V_{h}.
    \end{split}
\end{equation}

\begin{remark}
    This formulation accommodates the nonconformity of $V_{h}$ by factoring in the discontinuities among the facets, yet it preserves consistency when $u$ exhibits sufficient regularity, specifically when $u\in H^{s}( \Omega ), s\ge \frac{5}{2} +
\varepsilon $. This implies that the solution $u$  is continuous across the boundaries of interior elements, i.e.,  $\jump{ \partial _{n} u }   = 0 $ and  $\mean{ \partial _{nn} u }   = \partial _{nn} u $ on any $ F \in \mathcal{F} ^{int}_{h} $ .

It is noteworthy that we have the consistent terms $\left( \mean{  \partial _{n n} u_{h} }, \jump{ \partial _{n }v_{h}} \right)_{\mathcal{F}_{h}^{int}} $ and $ \left(   \partial _{n n} u_{h} ,  \partial _{n }v_h \right)_{\Gamma }$ naturally appear in the
derivation. However, we also added two symmetry terms,  $\left( \mean{  \partial _{n n} u_{h} }, \jump{ \partial _{n }v_{h}}
\right)_{\mathcal{F}_{h}^{int}} $ and $ \left(   \partial _{n n} u_{h} ,  \partial _{n }v_{h} \right)_{\Gamma }$, and the so-called penalty terms, $ \frac{\gamma }{h}  \left( \jump{ \partial _{n} u_{h}}, \jump{ \partial _{n} v_{h}   }   \right)_{\mathcal{F}_{h}^{int}}$ and $ \frac{\gamma }{h}  \left(  \partial _{n} u_{h},  \partial _{n} v_{h}      \right)_{\Gamma }$.
     These terms is essential for making the problem well-posed, hence, the name interior penalty method or symmetric interior penalty method. For more information of  nonconformal CIP error analysis, see \cite[Chapter 1.3]{pietro2012}.
\end{remark}

\subsubsection{Construction of the Laplacian formulation}%
\label{ssub:construction_of_the_laplacian_formulation}


\begin{lemma}

 Let $u \in H^{4}( \Omega ) $ the solution of \eqref{eq:bi_problem}, $v_{h} \in  V_{h}$ and a constant $\gamma >0$. Then we have the following identity.
\[
    \begin{split}
( \Delta ^2 u, v_{h} ) _{\Omega }  =& ( \Delta u, \Delta v_{h})_{\mathcal{T} _{h} }  + ( g_{2} , v_{h} )_{\Gamma }  - ( g_{1} , \Delta v_{h})_{\Gamma } + \frac{\gamma }{h} ( g_{1} ,  \partial _{n}v_{h})_{\Gamma }\\
& -  ( \jump{ \partial _{n} u} , \mean{ \Delta v_{h} })_{\mathcal{F}_{h}^{int} }-  (  \mean{ \Delta u }, \jump{ \partial _{n} v_{h}} )_{\mathcal{F}_{h}^{int} } + \frac{\gamma }{h}  \left( \jump{ \partial _{n} u}, \jump{ \partial _{n} v_{h}   }
\right)_{\mathcal{F}_{h}^{int} } \\
& - (  \Delta u, \partial _{n} v_{h})_{\Gamma }- ( \partial _{n} u, \Delta v_{h})_{\Gamma } + \frac{\gamma }{h}( \partial _{n} u, \partial _{n} v_{h})_{\Gamma }   .
    \end{split}
\]
\end{lemma}

\begin{proof}

  Similarly, we start by constructing integration by parts identities locally for a element $T$ and then extend it to the full mesh $\mathcal{T}_{h} $.
Utilizing \eqref{eq:iden_bi} can we see that \[
( \Delta ^2 u, v_{h} ) _{T} = ( \Delta u, \Delta v_{h}) +  ( \partial _{n} \Delta u, v_{h})_{\partial T} - ( \partial _{n} v_{h}, \Delta u) _{\partial T}
\]
Now, summing over all elements we get \[
    \begin{split}
( \Delta ^2 u, v_{h} ) _{\Omega } & = \sum_{T \in \mathcal{T}_{h} }^{}  \left( ( \Delta u, \Delta v_{h})_{T}
+  ( \partial _{n} \Delta u, v_{h})_{\partial T} - ( \partial _{n} v_{h}, \Delta u) _{\partial T} \right)  \\
 & =   ( \Delta u, \Delta v_{h})_{\mathcal{T} _{h}} +  \sum_{F \in \mathcal{F}_{h}^{ext} }^{}
 (\overbrace{( \partial _{n} \Delta u, v_{h})_{ F}}^{=( g_{2},v_{h})_{F} }  - ( \partial _{n} v_{h}, \Delta u) _{F}) \\
  &   \quad + \sum_{F \in \mathcal{F}_{h}^{int} }^{} \underbrace{( ( \partial _{n^{+}} \Delta u, v_{h})_{ F} + ( \partial _{n^{-}} \Delta u, v_{h})_{ F} )}_{(I)}  - \underbrace{( ( \partial _{n^{+}} v_{h}, \Delta u) _{F} + ( \partial _{n^{-}} v_{h}, \Delta u) _{F}
  )}_{(II)}   \\
    \end{split}
\]
Decomposing the terms and utilizing the regularity of $u \in H^{4}( \Omega ) $ and the $C^{0}$ continuity of $v_{h}\in V_{h}$ is it easy to see that,    \[
\begin{split}
    (I) & = ( \partial _{n^{+}} \Delta u, v_{h})_{ F} + ( \partial _{n^{-}} \Delta u, v_{h})_{ F}  = \int_{F}^{} \left[ \partial _{n} \Delta u \  v_{h} \right] =  (  \mean{ \partial _{n^{+}} \Delta u } , \underbrace{\jump{v_{h}  }}_{ = 0}      )_{ F} + (  \underbrace{\jump{ \partial _{n}
    \Delta u }}_{=0}  , \mean{v_{h}  }     )_{ F} \\
    (II) &=  ( \partial _{n^{+}} v_{h}, \Delta u) _{F} + ( \partial _{n^{-}} v_{h}, \Delta u) _{F} = \int_{F}^{} \jump{ \partial _{n} v_{h} \ \Delta u } =  ( \underbrace{\jump{ \partial _{n} v_{h}}}_{ \neq 0 } , \mean{ \Delta u })_{F}  + ( \mean{
    \partial _{n} v_{h}}, \underbrace{\jump{ \Delta u
    }}_{=0} )_{F}
\end{split} .
\]
Hence, we end up with the identity,
\[
( \Delta ^2 u, v_{h} ) _{\Omega } = ( \Delta u, \Delta v_{h})_{\mathcal{T} _{h} }  +  ( \jump{ \partial _{n} v_{h}} , \mean{ \Delta u })_{\mathcal{F}_{h} }  + ( g_{2} , v_{h} )_{\Gamma } - ( \partial _{n} v_{h}, \Delta u)_{\Gamma }.
\]

Similarly as for Lemma \ref{lemma:hessian}, we add consistent symmetry terms $(  \jump{ \partial _{n} u }, \mean{ \Delta  v_{h} })_{\mathcal{F}^{int}_{h} } $ and $(  \partial _{n} u, \Delta  v_{h}       )_{\Gamma  } -(  g_{1}, \Delta  v_{h}  )_{\Gamma  }  $ and
the penalty terms $ \frac{\gamma }{h} (  \partial _{n} u, \partial _{n}   v_{h}       )_{\Gamma  } -\frac{\gamma }{h}(  g_{1}, \partial _{n} v_{h}  )_{\Gamma  }$ and $\frac{\gamma }{h}( \jump{ \partial _{n} u } , \jump{ \partial _{n} v_{h} }
)_{\mathcal{F}^{int}_{h} }  $  . Effectively is the terms adding zero
because of the regularity $ u \in H^{4}( \Omega ) $ and the boundary condition $\partial _{n} u = g_{1}$.
Finally we have
\[
    \begin{split}
( \Delta ^2 u, v_{h} ) _{\Omega }  =& ( \Delta u, \Delta v_{h})_{\mathcal{T} _{h} }  + ( g_{2} , v_{h} )_{\Gamma }  - ( g_{1} , \Delta v_{h})_{\Gamma } + \frac{\gamma }{h} ( g_{1} ,  \partial _{n}v_{h})_{\Gamma }\\
& -  ( \jump{ \partial _{n} u} , \mean{ \Delta v_{h} })_{\mathcal{F}_{h}^{int} }-  (  \mean{ \Delta u }, \jump{ \partial _{n} v_{h}} )_{\mathcal{F}_{h}^{int} } + \frac{\gamma }{h}  \left( \jump{ \partial _{n} u}, \jump{ \partial _{n} v_{h}   }
\right)_{\mathcal{F}_{h}^{int} } \\
& - (  \Delta u , \partial _{n} v_{h})_{\Gamma }- ( \partial _{n} u, \Delta v_{h})_{\Gamma } + \frac{\gamma }{h}( \partial _{n} u, \partial _{n} v_{h})_{\Gamma }   .
    \end{split}
\]
and the proof is complete.
\end{proof}

We will now assemble the Laplace CIP formulation.
The discrete problem is as follows:
 \begin{equation}
     \label{eq:laplace_prob}
    \text{Find } u_{h} \in V_{h} \text{ such that } a^{L}( u_{h}, v_{h})  = l_{h}^{L}( v_{h} )  \quad \forall v_{h} \in  V_{h}.
\end{equation}
The corresponding bilinear and linear form is defined as,
    \begin{equation}
        \label{eq:cip_laplace_form}
        \begin{split}
            a_{h}^{L} \left( u_{h}, v_{h} \right)   =&
            \left( \alpha  u_{h}, v_{h} \right) _{\Omega }   +  \left( \Delta  u_{h}, \Delta v_{h} \right) _{ \Omega } \\
                                             & - \left( \mean{  \Delta  u_{h} }, \jump{ \partial _{n }v_{h}} \right)_{\mathcal{F}_{h}^{int}  }  - \left( \mean{ \Delta  v_{h} }, \jump{ \partial _{n}u_{h} }      \right)_{\mathcal{F}_{h}^{int}  }  + \frac{\gamma }{h}
                                             \left( \jump{ \partial _{n} u_{h}}, \jump{ \partial _{n} v_{h}   }   \right)_{\mathcal{F}_{h}^{int} } \\
                                             & - \left(   \Delta  u_{h} ,  \partial _{n }v_{h} \right)_{\Gamma   }  - \left(  \partial _{n}u_{h},  \Delta  v_{h} \right)_{\Gamma  }  + \frac{\gamma }{h}  \left(  \partial _{n} u_{h},  \partial _{n} v_{h}      \right)_{ \Gamma } \\
                                             l^{L}_{h}( v_{h})  =&  \left( f, v_{h} \right) _{\Omega } - ( g_{2},  v_{h} )_{\Gamma } -  ( g_{1}, \Delta  v_{h}  )_{\Gamma }  + \frac{\gamma }{h} ( g_{1}, \partial _{n} v_{h}  )_{\Gamma }
                                         \end{split}
                                     \end{equation}
                                     With the corresponding energy norms
                                     \begin{equation}
                                         \label{eq:a_cip_energy_norm_lap}
                                         \begin{split}
                                             \| v \|_{ a_{h}^{L} }^{ 2 }& = \alpha \| v\|_{ \Omega  }^{2  }  +  \| \Delta   v \|_{ \Omega   }^{ 2 }  + \|  h^{-\frac{1}{2}} \jump{ \partial _{n} v    }\|_{  \mathcal{F} _{h}^{int} }^{2  } +  \|  h^{-\frac{1}{2}}
                                             \partial _{n} v  \|_{  \Gamma  }^{2  },  \quad v \in V_{h}  \\
                                             \| v \|_{ a_{h}^{L},* }^{ 2 } &= \| v \|_{ a_{h} }^{ 2 }  + \| h^{\frac{1}{2}}  \mean{     \partial _{nn } v}  \|_{ \mathcal{F}_{h}^{int}   }^{  2}+ \| h^{\frac{1}{2}}       \partial _{nn } v  \|_{ \Gamma    }^{  2}, \quad  v\in V \oplus V_{h}.
                                         \end{split}
                                         .
                                     \end{equation}



                                     \begin{remark}
Again, note that we have the consistent terms $\left( \mean{  \Delta  u_{h} }, \jump{ \partial _{n }v_{h}} \right)_{\mathcal{F}_{h}^{int}} $ and $ \left(   \Delta  u_{h} ,  \partial _{n }v_{h} \right)_{\Gamma }$ naturally appearing in the
derivation. Also recall the symmetry terms, $(  \jump{ \partial _{n} u_{h} }, \mean{ \Delta  v_{h} })_{\mathcal{F}^{int}_{h} } $ and $ (  \partial _{n} u_{h} -  g_{1}, \Delta  v_{h}  )_{\Gamma  } $, with corresponding Nitsche penalty terms, $
\frac{\gamma }{h}(  \partial _{n} u_{h} - g_{1}, \partial _{n} v_{h}  )_{\Gamma  }$ and $ \\ \frac{\gamma }{h}( \jump{ \partial _{n} u_{h} } , \jump{ \partial _{n} v_{h} }
_{\mathcal{F}^{int}_{h} }  $, thus making the bilinear form $a^{L}( \cdot,\cdot ) $ symmetric and well-posed.
                                     \end{remark}

\subsubsection{Comments and earlier work}%
\label{ssub:remarks}

It should be noted that the Hessian formulation has a substantial limitation in that it is only valid for homogeneous Neumann conditions. This constraint arises from the challenges associated with imposing $g_{1}$ via the tangential derivative terms
in Equation \eqref{eq:bi_basic_dg_full_1} during the proof of Lemma \ref{lemma:hessian}.
From a physical perspective, this is not problematic as it aligns with the boundary conditions of the original CH problem \eqref{eq:strongch}. However, from the standpoint of numerical validation, the homogeneous Neumann condition enforces strict
rules on the design of manufactured solutions on arbitrary domains. One way to fix this is to enforce tangential derivatives of $g_{1}$, i.e., inserting $\partial _{n}u = g_{1}$ for $ ( \partial _{t_{i}}( \partial_{n}  u ) , \partial _{n} v)_\Gamma  $ into
\eqref{eq:bi_basic_dg_full_1}. A downside with this method is that we must require $g_{1}$ in $ H^{\frac{3}{2}}( \Gamma ) $.
Consequently, the examples illustrated in section \ref{sec:numerical_results} are only demonstrated on simple domains. This particular constraint does not apply to the Laplace formulation.

The Hessian formulation is well investigated by Susanne Brenner in several papers for \cite{brenner2012, brenner2012quadratic, brenner2012quadratic_kirk} with a corresponding analysis and numerical validation. Similarly, variants of the Laplace formulation can be found here
\cite{feng2007fully, georgoulis2009discontinuous}. In these article there also is good theoretical and experimental evidence that both formulation have the following expected a priori estimates. Let  $u \in H^{s}( \Omega ) $ for $s\ge  \frac{5}{2} + \varepsilon$, and $u_{h}\in
V_{h}  $ of order $k\ge 2$. Then with $r = \min\left\{ s,
k+2 \right\}$ the a priori estimates are   \[
    \begin{split}
\| u - u_{h} \|_{ a_{h},*  }^{  }  & \lesssim  h^{r-2} \| u \|_{ r, \Omega  }^{  } \\
\| u - u_{h} \|_{ \Omega   }^{  }  & \lesssim  h^{r- \max_{}\left\{ 0, 3-k \right\}  } \| u \|_{ r,\Omega  }^{  }
    \end{split}
\]

Be aware that the $\| \cdot  \|_{\Omega   }^{  } $ norm estimates is suboptimal for $k=2$.
It is worth noting that technically is the interior regularisation equivalent to do a Nitsche's method in all interior boundaries of the elements with boundary conditions of each element weakly
imposed to zero. Thus, we expect the penalty parameter $\gamma$ to be the same interior and exterior elements. Let where $k\ge 2$ is the polynomial order, then for the Hessian formulation is it theoretically proven that $\gamma = 2k ( k-1 ) $
\cite{brenner2012quadratic, brenner2012}. However, we still prefer to experimentally verify the best parameter.


\subsection{Note on the biharmonic mixed formulation}%
\label{subsec:biharmonic_mixed_formulation}

It is easy to see that the biharmonic problem can be rewritten into an equivalent mixed formulation , that is, to find $\sigma, \tau  \in H^2( \Omega ) $ s.t. \[
    \begin{split}
\Delta \sigma  & = f \quad  \text{in } \Omega \\
\sigma   & = \Delta u  \text{ in } \Omega \\
\partial _{n} \sigma  & = g_{1} \text{ on } \Gamma  \\
\partial _{n} u   & = g_{2} \text{ on } \Gamma
    \end{split}
\]
The goal is to obtain an useful weak formulation. Using Greens theorem on the first equation we get,
\[
( \sigma, v)_{\Omega } = ( \nabla  u , \nabla v  )_{\Omega } - ( \nabla _{n} u , v) _{\Gamma }.
\]
Similarly for the second equation we obtain
\[
( \nabla \sigma , \nabla \varphi  )_{\Omega} - ( \partial _{n} \sigma ,  \varphi )_{\Gamma } = ( f,\varphi ) _{\Omega}
\]
Putting it all together we have the following mixed weak formulation; Find $( u, \sigma ) \in H^{1}( \Omega ) \times H^{1}( \Omega )  $ s.t. \[
    \begin{split}
     ( \nabla  u , \nabla v  )_{\Omega } -( \sigma, v)_{\Omega }  & =   ( g_{1} , v) _{\Gamma } \quad  \forall v \in H^{1}( \Omega ) \\
( \nabla \sigma , \nabla \varphi  )_{\Omega}  & = ( f,\varphi ) _{\Omega} + ( g_{2} ,  \varphi )_{\Gamma } \quad  \forall \varphi \in H^{1}( \Omega )
    \end{split}
\]
Now we want to relate this formulation to the abstract saddle point problem (SPP).
Let $V = H^{1}( \Omega ) $  and $W=H^{1}( \Omega ) $ be  Hilbert spaces and define the bilinear form $a: V\times V \to \mathbb{R}  $ and $b: V \times W \to \mathbb{R} $ s.t. $a( \sigma,v ) = - ( \sigma , v) _{\Omega }  $ and $b( u,v) = ( \nabla u,
\nabla v)_{\Omega  }  $. We also may define the linear forms, $G,F: V \to \mathbb{R} $ s.t. $ G( v)  = ( g_{1}, v) _{\Gamma } $ and $F( \varphi ) = ( f, \varphi )_{\Omega } + ( g_{2}, \varphi )_{\Gamma } $.

Hence, we obtain the following SPP. Find $( u,\sigma ) \in V \times W$ s.t.  \[
    \begin{cases}
       a( \sigma ,v) + b ( u, v )  & = G( v)   \quad  \forall v \in V \\
       b( u, \varphi  )  & = F( \varphi )     \quad \forall \phi \in W
    \end{cases}
\]
This is useful since we can now apply standard saddle point theory to do an analysis for the problem. We will see that it is now easier to handle the boundary constraints naturally, but with the cost of a more challenging time discretization
procedure.
For more information about the biharmonic mixed formulation, see \cite{babuvska1980analysis,cai2023nitsche}.
However, in this thesis is the focus on solving the biharmonic equation avoiding the mixed formulation using the CIP formulation, which does in fact handle the downsides with the SPP problem.


% A well known and mature application for SPP is the well known Stokes equation, hence, a good place to start \cite{john2016finite, knabner2003numerical}.



    % \newpage
\section{Cut continuous interior penalty methods for the biharmonic problem}%
\label{sec:cutcip_biharmonic_problem}


Questions arise when we want to allow for complex geometries where some physical domain $\Omega $ has a smooth boundary $\Gamma $.
A method would be to generate a structured background mesh which fully covers $\Omega $, but does not necessarily fit to the boundary perfectly.
For consistency, integration of the contribution is performed on the physical domain, but the involved discrete test and trial functions in $V_{h}$ are defined on the background mesh.
However, we will run into geometrical problems when
so-called cut finite elements has a very small intersection of the physical domain, i.e, $\abs{ T \cap \Omega  }_{d} \ll \abs{ T }_{d}  \lesssim h^{d}$ and $\abs{ \Gamma \cap T }_{d-1} \ll  h^{d-1}$, where $\abs{\ \cdot \  }_{d} $ is the
measure of the volume in dimension $d$.
The cut elements are identified as a crucial geometric issue since the inverse inequalities outlined in Section \ref{sub:some_general_inequalities} do not generalize well to these elements. This identifies the challenges in establishing theoretical stability and a priori estimates for any geometric arrangement.

One way do handle this issue is to introduce the so-called cut finite element method (CutFEM).
The method involves adding a stabilization term, commonly referred to as the ghost penalty term.
This term serves the purpose of controlling the energy norm associated with the bilinear form, utilizing the entire background mesh to ensure stabilization and geometric robustness.
For more information, see
\cite{burman2015cutfem, burman2010ghost, burman2022cutfem, burman2012fictitious}.
Inspired by the corresponding CutFEM DG elliptic framework outlined in \cite{gurkan2019stabilized}, the objective of this section is to engineer suitable stabilized ghost penalty terms for the biharmonic problem, starting from the CIP methods introduced in Section \ref{sec:CIP_biharmonic_problem}.
We will show what assumptions are needed for the ghost-penalty method for the discrete problem to be stable in Section \ref{sub:stability_estimate} and derive optimal convergence in Section \ref{sec:a_priori_estimates}.  Once
this is fulfilled we propose a ghost penalty which fulfills these assumptions in Subsection \ref{sec:constructing_ghost_penalties}.


\subsection{Computational domain}%
\label{sub:unfitted_mesh}

\begin{figure}[h!]
    \centering
% \subfloat[]{\label{a1}
%         \begin{tikzpicture}[scale=0.7]

%             \draw[fill=blue!30] (0.1, 0.1) circle (1.5cm);
%             % Background mesh
%             \foreach \i in {-2.5, -2, ..., 2.5} {
%                 \draw[line width=0.1pt, shift={(-2.5,\i)}, opacity=0.2] (0,0) -- (5,0);
%                 \draw[line width=0.1pt, shift={(\i,-2.5)}, opacity=0.2] (0,0) -- (0,5);
%             }


%             % Labels
%             \node[below right] at (-0.2,0.4) {$\Omega$};
%             \node[below right] at (1.2,1.5) {$\Gamma$};
%         \end{tikzpicture}

% }\hfill
\subfloat[]{\label{a}
        \begin{tikzpicture}[scale=1.0]

            \fill[yellow!30] (-2.5,2.5) -- (2.5,2.5) -- (2.5,-2.5) -- (-2.5,-2.5) -- cycle;
            \draw[fill=blue!30, opacity=0.4] (0.1, 0.1) circle (1.5cm);

            \draw (0.1, 0.1) circle (1.5cm);
            % Background mesh
            \foreach \i in {-2.5, -2, ..., 2.5} {
                \draw[line width=0.1pt, shift={(-2.5,\i)}, opacity=0.2] (0,0) -- (5,0);
                \draw[line width=0.1pt, shift={(\i,-2.5)}, opacity=0.2] (0,0) -- (0,5);
            }

            \node[below right] at (-0.2,0.4) {$\Omega$};
            \node[below right] at (-1.6,1.6) {$\Gamma$};

            % Labels
            \node[below right] at (1.6,2.0) {$\widetilde{\mathcal{T}}_{h}$};
        \end{tikzpicture}

}\hfill
\subfloat[]{\label{c}
        \begin{tikzpicture}[scale=1.0]

            % POTENTIAL ACTIVE MESH
            \fill[blue!30] (2,2) -- (2,-1.5) --(-1.5,-1.5) -- (-1.5,2) -- cycle;

            % ELEMENTS WITH NO INTERSECTION
            % lower left
            \fill[white] (-1.5,-1.5) rectangle (-1.0,-1.0);
            \fill[white] (-1.5,2.0) rectangle (-1.0,1.5);
            \fill[white] (-1.0,2.0) rectangle (-0.5,1.5);
            \fill[white] (2,2) rectangle (1.5,1.5);
            \fill[white] (1.5,2) rectangle (1.0,1.5);
            \fill[white] (2,1.5) rectangle (1.5,1.0);
            \fill[white] (1.5,-1) rectangle (2,-1.5);
            \fill[white] (1.5,-0.5) rectangle (2,-1.0);

            % CUT ELEMENTS
            \fill[green!40] (-0.5,2.0) rectangle (1.0,1.5);
            \fill[green!40] (-1.5,1.5) rectangle (0.0,1.0);
            \fill[green!40] (0.5,1.5) rectangle (1.5,1.0);
            \fill[green!40] (-1.5,1.0) rectangle (-1.0,-1.0);
            \fill[green!40] (-1.0,-0.5) rectangle (-0.5,-1.5);
            \fill[green!40] (-0.5,-1.5) rectangle (1.5,-1.0);
            \fill[green!40] (1.5,-1) rectangle (1.0,-0.0);
            \fill[green!40] (1.5,-0.5) rectangle (2.0,1.0);
            \fill[green!40] (1.0,0.5) rectangle (1.5,1.0);

            \draw (0.1, 0.1) circle (1.5cm);
            % Background mesh
            \foreach \i in {-2.5, -2, ..., 2.5} {
                \draw[line width=0.1pt, shift={(-2.5,\i)}, opacity=0.2] (0,0) -- (5,0);
                \draw[line width=0.1pt, shift={(\i,-2.5)}, opacity=0.2] (0,0) -- (0,5);
            }


            % Labels
            \node[below right] at (0.0,0.5) {$\mathcal{T}^{\mathrm{int} }_{h}$};
            \node[below right] at (-1.9,1.6) {$\mathcal{T}^{\Gamma }_{h}$};
        \end{tikzpicture}
}
\hfill
\subfloat[]{\label{b}
        \begin{tikzpicture}[scale=1.0]

            % POTENTIAL ACTIVE MESH
            \fill[orange!30] (2,2) -- (2,-1.5) --(-1.5,-1.5) -- (-1.5,2) -- cycle;

            % ELEMENTS WITH NO INTERSECTION
            % lower left
            \fill[white] (-1.5,-1.5) rectangle (-1.0,-1.0);
            \fill[white] (-1.5,2.0) rectangle (-1.0,1.5);
            \fill[white] (-1.0,2.0) rectangle (-0.5,1.5);
            \fill[white] (2,2) rectangle (1.5,1.5);
            \fill[white] (1.5,2) rectangle (1.0,1.5);
            \fill[white] (2,1.5) rectangle (1.5,1.0);
            \fill[white] (1.5,-1) rectangle (2,-1.5);
            \fill[white] (1.5,-0.5) rectangle (2,-1.0);

            % CUT ELEMENTS
            \fill[orange!30] (-0.5,2.0) rectangle (1.0,1.5);
            \fill[orange!30] (-1.5,1.5) rectangle (0.0,1.0);
            \fill[orange!30] (0.5,1.5) rectangle (1.5,1.0);
            \fill[orange!30] (-1.5,1.0) rectangle (-1.0,-1.0);
            \fill[orange!30] (-1.0,-0.5) rectangle (-0.5,-1.5);
            \fill[orange!30] (-0.5,-1.5) rectangle (1.5,-1.0);
            \fill[orange!30] (1.5,-1) rectangle (1.0,-0.0);
            \fill[orange!30] (1.5,-0.5) rectangle (2.0,1.0);
            \fill[orange!30] (1.0,0.5) rectangle (1.5,1.0);

            \draw (0.1, 0.1) circle (1.5cm);
            % Background mesh
            \foreach \i in {-2.5, -2, ..., 2.5} {
                \draw[line width=0.1pt, shift={(-2.5,\i)}, opacity=0.2] (0,0) -- (5,0);
                \draw[line width=0.1pt, shift={(\i,-2.5)}, opacity=0.2] (0,0) -- (0,5);
            }


            % Labels
            % \node[below right] at (2.5,2.5) {$\widetilde{\mathcal{T}}_{h}$};
            % \node[below right] at (0.4,0.5) {$\mathcal{T}_{int}$};
            \node[below right] at (-0.5,0.5) {$\mathcal{T}_{h }$};
        \end{tikzpicture}


}


\caption{Illustration of the domain $\Omega$ with the corresponding boundary $\Gamma$, the background mesh $\widetilde{\mathcal{T}}_{h} $,  the cut cells $\mathcal{T} _{\Gamma }$, the interior cells $\mathcal{T} _{int}$ and the active set $\mathcal{T} _{h} =
\mathcal{T}^{ \mathrm{int}  }_{h} \cup \mathcal{T}_{h }^{ \Gamma  }  $. }
\label{fig:background_mesh}
\end{figure}


We want to devise a CutFEM based on the CIP formulation for the biharmonic problem. Assume that the physical domain $\Omega \subseteq    \mathbb{R} ^d$ to be open and bounded with a corresponding a sufficiently smooth boundary $\Gamma  $.
 Let $\widetilde{\mathcal{T}_{h} } $ be a shape-regular and quasi-uniform mesh which covers $\Omega $, but does not need to fit the
domain. Let us denote the active set $\mathcal{T} _{h} \subseteq \widetilde{\mathcal{T}_{h}}$ which intersects the interior of the active domain $\Omega $, that is
\begin{equation}
\label{eq:active_set}
\mathcal{T} _{h} = \left\{ T \in \widetilde{\mathcal{T} }_{h}  \mid  T \cap \Omega   \neq \emptyset    \right\}.
\end{equation}
We define the corresponding set of interior facets, \[
    \mathcal{F} _{h}^{\mathrm{int} } = \left\{ F = T^{+} \cap T^{-}  \mid  T^{+}, T^{-} \in \mathcal{T} _{h} \text{ and } T^{+} \neq T^{-} \right\},
\]
and the set of elements cut by the boundary \[
\mathcal{T}_{h} ^{\Gamma } = \left\{ T \in \mathcal{T} _{h}   \mid  T \cap \Gamma \neq \emptyset  \right\}.
\]
For convenience, will we define also the interior of the active set as $\mathcal{T} _{int}$.
\[
\mathcal{T} ^{\mathrm{int} }_{h} = \left\{ T \in \mathcal{T} _{h}   \mid  T \cap  \mathrm{Int}(\Omega ) \neq \emptyset  \right\}.
\]
Hence, we have that the active set is the union of the interior and cut elements, $\mathcal{T} _{h} = \mathcal{T}_{h} ^{\mathrm{int} } \cup  \mathcal{T} ^{\Gamma }_{h}$. For an illustration, see Figure \ref{fig:background_mesh}.



\subsection{Cut continuous interior penalty methods }%
\label{sub:cut_cip_method}

As $\Omega $ is static is it easy to observe that having a polynomial basis on the full mesh, $\widetilde{\mathcal{T}}_{h}$, is not necessary. Restricting us to the active set, we define the domain $\Omega _{h} = \bigcup _{T \in \mathcal{T} _{h}} T$.
Hence, we define the polynomial space only on the active set $\mathcal{T}_{h} $ from \eqref{eq:active_set},
\begin{equation}
V_{h} = \left\{ v \in C^{0}( \Omega _{h}  )  \mid  v_{T} = v | _{T} \in \mathcal{P} ^{k}\left( T \right) \text{ or } Q^{k}\left( T \right) \ \ \forall T \in
\mathcal{T}_{h}    \right\}.
\end{equation}
Here is $k$ the polynomial order.
Furthermore, drawing on the principles outlined in Section \ref{sec:CIP_biharmonic_problem}, we can indeed recall two CIP formulations for the biharmonic equation: the Hessian formulation \eqref{eq:hessian_prob} and the Laplace formulation \eqref{eq:laplace_prob}.

To make sure the problem is stabilized will we add a consistent symmetric positive semi-definite bilinear ghost-penalty term  $g_{h}: V_{h} \times  V_{h} \to \mathbb{R} $ to our bilinear form. That is, we define the discrete problem to be:
\begin{equation}
\label{eq:discrete_CutCIP_prob}
\text{ Find } u_{h} \in V_{h}  \text{ such that }    A_{h}( u_{h} ,v ) := a_{h}( u_{h}, v)  + g_{h}( u_{h},v) = l_{h} ( v) \quad  \forall v \in  V_{h}.
\end{equation}
Here $a_{h}( \cdot,\cdot  ) $ stand for either $a_{h}^{L}( \cdot ,\cdot ) $ or $a_{h}^{H}( \cdot ,\cdot ) $.

In this section, we provide a full proof for the Hessian formulation, however, the proof of the Laplace formulation does not differ too much. For simplification will we use the notation $a_{h}(u,v ) = a_{h}^{H}( u,v)$ and $l_{h}(v) =l_{h}^{H}(v)$ for the rest of the stability and convergence analysis.

Keep in mind that in contrast to the standard CIP methods our proposed method is defined on an unfitted mesh. As we will see in the analysis, the ghost penalty is a method to ensure numerical stability on cut cell $\mathcal{T} ^{\Gamma }_{h}$. The main reason why
this numerical instability is happening for a unfitted mesh is when a cell is badly cut, see examples in Figure \ref{fig:intersection-example}.
In other words, when a cell is "badly cut," it means that it is intersected by the boundary $\Gamma$ in such a way that only a very small part of the interior of an element $T$  intersects with the physical domain $\Omega $ , i.e. $\abs{ \Omega \cap
T }_{d} \ll h^{d}$. This can lead to both stability issues and a very poor condition number of the system matrix causing numerical instability.

The ghost penalty stabilization technique is designed to tackle this issue. Essentially, this approach introduces additional terms into the finite element method that penalize jumps in the discrete solution and its gradients across cell interfaces,
typically the cut-cells. This penalty not only improves the conditioning of the system matrix but also enhances the robustness of the method with respect to the location of the boundary inside each cell. However, to make this possible, we assume a
so-called fat-intersection property, which will be relevant in Section \ref{sec:constructing_ghost_penalties}.

Our first assumption as as follows;
for a $T \in \mathcal{T} ^{\Gamma }_{h}$ there always exists a patch $\omega ( T) $ which contains $T$ and an element $T'$ with a so-called fat intersection $
        \abs{ T' \cap \Omega  } _{d} \gtrsim \abs{ T' } _{d}$, where $\abs{ \cdot  }_{d} $ is the measure of an element of dimensions $d =2,3  $ . For an illustration, see Figure \ref{fig:fat_intersection_property}.

\begin{figure}[t]
    \centering
    \begin{tikzpicture}
        \coordinate (center) at (0, 0);

        % Reference hexagon vertices
        \coordinate (A1) at (0:2.5);
        \coordinate (A2) at (55:2.5);
        \coordinate (A3) at (125:2.5);
        \coordinate (A4) at (180:2.5);
        \coordinate (A5) at (235:2.5);
        \coordinate (A6) at (305:2.5);

        \fill[green!40] (center) --(A1) -- (A2) -- (A3) -- (A4) -- (A5)  -- cycle;
        \fill[blue!30] (center) -- (A5)--(A6) -- (A1) -- cycle;

        % Draw the individual edges
        \draw (center) -- (A1);
        \draw (center) -- (A2) --(A1);
        \draw (center) -- (A3);
        \draw (center) -- (A4);
        \draw (A2) -- (A3);
        \draw (A3) -- (A4);
        \draw (A5) -- (A6);
        \draw (A4) -- (A5);
        \draw (center) -- (A5);
        \draw (center) -- (A6) -- (A1);

        \coordinate (Ti) at (-0,-1.5);
        \coordinate (Tg) at (0.5,2.1);
        % \node[below] at (Tg) {$\mathcal{T}_{\Gamma }$};
        % \node[below] at (Ti) {$\mathcal{T}_{int }$};

        \coordinate (T0) at (1.1,1.0);
        \coordinate (T1) at (-1.1,-0.4);
        \node[below] at (T0) {$T$};
        \node[below] at (T1) {$T'$};

        \coordinate (C1) at (-3,-1.0);
        \coordinate (C2) at (2.6,0.4);
        \draw[-, line width=2pt, >=stealth] ($(C2)$) to[bend right=16.9] node[midway,xshift=-2.3cm, yshift=-1.3cm] {$\Gamma $} ($(C1)$);

        % \coordinate (D1) at (1,0.0);
        % \coordinate (D2) at (1.0,0.4);
        % \coordinate (D3) at (1.2,0.2);
        % \node at (D3) {$\varepsilon$};
        % \draw[line width=1.3pt, dotted] (D1) -- (D2);

        % Draw the line with brackets

    % Legend
    \begin{scope}[shift={(2.3,0.3)}]
        \draw (0,0.4) rectangle (1.3,1.5);
        \fill[green!30] (0.2,1.3) rectangle (0.4,1.1);
        \node[right] at (0.4,1.2) {$\mathcal{T}_{\Gamma}$};
        \fill[blue!30] (0.2,0.8) rectangle (0.4,0.6);
        \node[right] at (0.4,0.7) {$\mathcal{T}_{int}$};
    \end{scope}

    \end{tikzpicture}
\caption{Illustration of the fat intersection property. Let $T \in \mathcal{T}_{\Gamma } $. It shows a patch $\omega ( T) $ that contains elements $T \in  \mathcal{T}_{\Gamma } $ and $T'\in \mathcal{T} _{int} \cup \mathcal{T} _{\Gamma } = \mathcal{T} _{h}$, with $T' $ having a sufficiently large intersection with $\Omega$.}
    \label{fig:fat_intersection_property}
\end{figure}


We define the underlying norms for $ v \in V_{h} $ as
    \begin{align}
        \label{eq:bi_ah_norm}
        \| v \|_{ a_{h} }^{ 2 } & =    \alpha \|   v \|_{ \mathcal{T} _{h} \cap \Omega  }^{ 2}  + \| D^2 v \|_{\mathcal{T} _{h} \cap \Omega   }^{ 2 } +  \| h^{-\frac{1}{2}} \jump{ \partial _{n} v }   \|_{ \mathcal{F}_{h}^{}\cap \Omega    }^{ 2
        } +  \| h^{-\frac{1}{2}}  \partial _{n} v    \|_{ \Gamma   }^{ 2 },    \\
        \label{eq:bi_gh_norm}
\abs{ v } _{g_{h}}^{2} & = g( v,v), \\
        \label{eq:bi_Ah_norm}
\| v \|_{A_{h}  }^{  2}  & = \| v \|_{ a_{h} }^{ 2 } + \abs{ v } _{g_{h}}^{2}, \\
    \end{align}
and for $v \in V \oplus V_{h}$ we also introduce,
\begin{equation}
    \label{eq:astarnorm}
\| v \|_{ a_{h}, * }^{  2}  =\| v \|_{ a_{h} }^{ 2 } +  \| h^{\frac{1}{2}} \mean{ \partial _{nn} v }   \|_{\mathcal{F} _{h}^{} \cap \Omega   }^{  2} +  \| h^{\frac{1}{2}} \partial _{nn} v    \|_{ \Gamma }^{  2}.
\end{equation}
\begin{remark}
Note that it holds that $\mathcal{T} _{h} \cap  \Omega   = \Omega  $ and $\mathcal{T} _{h} \cap  \Gamma  = \Gamma $. Depending on context, we choose the best suitable notation.
\end{remark}
\begin{remark}
    The necessity to define the supplementary terms in the $\| \cdot   \|_{a{h},* }^{ } $  may raise certain questions.  The reason is because when $v$ is continuous, i.e. $v \in V$, the local inverse estimates  \ref{eq:fund_inv_est} does not hold for $\| \mean{ \partial
    _{nn} v }  \|_{ \mathcal{F}_{h} \cap \Omega    }^{  }  $ and  $\|  \partial
    _{nn} v   \|_{ \Gamma }^{  }  $ when evaluating $a_{h}( v, v) $. Hence, this leads necessity adding the additional terms into the norm.
\end{remark}


\subsection{Stability estimate}%
\label{sub:stability_estimate}


\begin{figure}[b]
    \centering
    \begin{minipage}{0.4\textwidth}
        \centering
        \begin{tikzpicture}
            \coordinate (center) at (0, 0);

            % Reference hexagon vertices
            \coordinate (A1) at (0:2.5);
            \coordinate (A2) at (55:2.5);
            \coordinate (A3) at (125:2.5);
            \coordinate (A4) at (180:2.5);
            \coordinate (A5) at (235:2.5);
            \coordinate (A6) at (305:2.5);

            \fill[green!40] (center) --(A1) -- (A2) -- (A3) -- (A4) -- (A5)  -- cycle;
            \fill[blue!30] (center) -- (A5)--(A6) -- (A1) -- cycle;

            % Draw the individual edges
            \draw (center) -- (A1);
            \draw (center) -- (A2) --(A1);
            \draw (center) -- (A3);
            \draw (center) -- (A4);
            \draw (A2) -- (A3);
            \draw (A3) -- (A4);
            \draw (A5) -- (A6);
            \draw (A4) -- (A5);
            \draw (center) -- (A5);
            \draw (center) -- (A6) -- (A1);

            \coordinate (Ti) at (-0,-1.5);
            \coordinate (Tg) at (0.5,2.1);
            % \node[below] at (Tg) {$\mathcal{T}_{\Gamma }$};
            % \node[below] at (Ti) {$\mathcal{T}_{int }$};

            \coordinate (T0) at (1.1,1.0);
            \coordinate (T1) at (-1.1,-0.4);
            \node[below] at (T0) {$T$};
            % \node[below] at (T1) {$T'$};

            \coordinate (C1) at (-3,-1.0);
            \coordinate (C2) at (2.6,0.4);
            \draw[-, line width=2pt, >=stealth] ($(C2)$) to[bend right=16.9] node[midway,xshift=-2.3cm, yshift=-1.3cm] {$\Gamma $} ($(C1)$);

            \coordinate (D1) at (1,0.0);
            \coordinate (D2) at (1.0,0.4);
            \coordinate (D3) at (1.2,0.2);
            \node at (D3) {$\varepsilon$};
            \draw[line width=1.3pt, dotted] (D1) -- (D2);

            % Draw the line with brackets

            % Legend
            % \begin{scope}[shift={(2.3,0.3)}]
            %     \draw (0,0.4) rectangle (1.3,1.5);
            %     \fill[green!30] (0.2,1.3) rectangle (0.4,1.1);
            %     \node[right] at (0.4,1.2) {$\mathcal{T}_{\Gamma}$};
            %     \fill[blue!30] (0.2,0.8) rectangle (0.4,0.6);
            %     \node[right] at (0.4,0.7) {$\mathcal{T}_{int}$};
            % \end{scope}

        \end{tikzpicture}
    \end{minipage}
    \hspace{1cm}
    \begin{minipage}{0.4\textwidth}
        \centering
        \begin{tikzpicture}
            \coordinate (center) at (0, 0);

            % Reference hexagon vertices
            \coordinate (A1) at (0:2.5);
            \coordinate (A2) at (60:2.5);
            \coordinate (A3) at (120:2.5);
            \coordinate (A4) at (180:2.5);
            \coordinate (A5) at (240:2.5);
            \coordinate (A6) at (300:2.5);
            \coordinate (A7) at (300:2.5);


            \fill[green!40] (center) --(A1) -- (A2) -- (A3) -- (A4) -- cycle;
            \fill[blue!30] (center) (A4) -- (A5)--(A6)-- (A7) -- (A1) -- cycle;

            % Draw the individual edges
            \draw (center) -- (A1);
            \draw (center) -- (A2) --(A1);
            \draw (center) -- (A3);
            \draw (center) -- (A4);
            \draw (A2) -- (A3);
            \draw (A3) -- (A4);
            \draw (A5) -- (A6);
            \draw (A4) -- (A5);
            \draw (center) -- (A5);
            \draw (center) -- (A6) -- (A1);

            \coordinate (Ti) at (-0,-1.5);
            \coordinate (Tg) at (0.5,2.2);
            % \node[below] at (Tg) {$\mathcal{T}_{\Gamma }$};
            % \node[below] at (Ti) {$\mathcal{T}_{int }$};

            \coordinate (T0) at (-0,1.9);
            \coordinate (T1) at (-1.1,-0.4);
            \node[below] at (T0) {$T$};

            \coordinate (C1) at (-3,0.7);
            \coordinate (C2) at (2.6,0.7);
            \draw[-, line width=2pt, >=stealth] ($(C2)$) to[bend right=0.9] node[midway,xshift=-2.5cm, yshift=-0.5cm] {$\Gamma $} ($(C1)$);
            % Legend
            \begin{scope}[shift={(2.3,0.4)}]
                \draw (0,0.4) rectangle (1.3,1.5);
                \fill[green!30] (0.2,1.3) rectangle (0.4,1.1);
                \node[right] at (0.4,1.2) {$\mathcal{T}^{\Gamma}_{h}$};
                \fill[blue!30] (0.2,0.8) rectangle (0.4,0.6);
                \node[right] at (0.4,0.7) {$\mathcal{T}^{\mathrm{int} }_{h}$};
            \end{scope}

            \coordinate (D1) at (0,0.0);
            \coordinate (D2) at (0.0,0.7);
            \coordinate (D3) at (0.3,0.2);
            \node at (D3) {$\varepsilon$};
            \draw[line width=1.3pt, dotted] (D1) -- (D2);

        \end{tikzpicture}
    \end{minipage}
        \caption{Illustration of two examples of bad cut cells with an arbitrary small length $\varepsilon \ll 1 $. Let $T \in  \mathcal{T}^{\Gamma }_h $ be a cut cell.  On the left example, is it clear that $\abs{ \Gamma \cap T }
            \lesssim  h^{d-1}$ and $  \ \abs{ \Omega  \cap T } \lesssim  \varepsilon  h^{d}$  . However, on the right example is it clear that  $\abs{ \Gamma \cap T }
            \lesssim  \varepsilon h^{d-1}$ and $\abs{ \Omega  \cap T } \lesssim  \varepsilon  h^{d}$.}
        \label{fig:intersection-example}
\end{figure}

% From basic theory we have the following inverse estimate for $ v \in \mathcal{P}^{k}( T)$ s.t. \[
%      \| \partial _{nn}  v \|_{F   }^{ }  \lesssim  \| h_{T}^{-\frac{1}{2}} D ^2 v \|_{ T }^{  },
% \]
% where the hidden constant depend on dimension $d$, order $k$ and the shape regularity.

Recall the Subsection \ref{sub:some_general_inequalities} where we discussed standard local inverse estimates.
Similarly for cut elements is it easy to see that this must hold,
\begin{equation}
     \| \partial _{nn}  v_{h} \|_{F \cap \Omega    }^{  }  \lesssim\| \partial _{nn}  v_{h} \|_{F }^{  }  \lesssim   \| h_{T}^{-\frac{1}{2}} D ^2 v_{h} \|_{ T }^{  }.
\end{equation}
A useful variant is the following inequality that is,
\begin{equation}
    \label{eq:inv_full}
\| \partial _{nn} v_{h} \|_{ \Gamma \cap T  }^{  } \lesssim h^{-\frac{1}{2}} \| D^2 v_{h} \|_{ T }^{  }.
\end{equation}
\begin{remark}
    It may be natural to instead look at $\| \partial _{nn} v_{h} \|_{ \Gamma \cap T  }^{  } \lesssim h^{-\frac{1}{2}} \| D^2 v_{h} \|_{ T\cap \Omega  }^{  }$, however, this cannot hold for a arbitary cut configutration for an unfitted mesh. To demonstrate, let $\varepsilon \ll 1$ be a small length. For the examples provided in
    Figure \ref{fig:intersection-example} we have two cases: i) $\abs{ \Gamma \cap \Omega  }_{d-1} \lesssim \varepsilon h^{d-1}  $ and $\abs{ T \cap \Omega  }_{d-1} \lesssim \varepsilon h^{d-1}  $, and ii)
    $\abs{ \Gamma \cap \Omega  }_{d-1} \lesssim  h^{d-1}  $ and $\abs{ T \cap \Omega  }_{d-1} \lesssim \varepsilon h^{d-1}  $.
    The first first case impacts the condition number since it is introducing almost vanishing entries in the system matrix from \eqref{eq:linear_system}.
    While for the second case is bad for inverse estimates, and, thus, problematic for proving coercivity. To recover, one in fact must incorporate the full element $T$ into the inverse estimate.
\end{remark}


Since the inequalities above holds for all elements locally is it natural to extend it to the full mesh  $\mathcal{T}_{h} $. This implies that
\begin{align}
\label{eq:bi_cut_inverse_1}
\| \partial _{nn} v_h \|_{ \mathcal{T} _{h} \cap \Gamma  }^{  } &\lesssim h^{-\frac{1}{2}} \| D^2 v_h \|_{ \mathcal{T}_h }^{  }, \\
\label{eq:bi_cut_inverse_2}
\| \partial _{nn}  v_h \|_{ \mathcal{F}_h \cap \Omega    }^{  }  &  \lesssim   h^{-\frac{1}{2}} \| D^2 v_h \|_{ \mathcal{T}_h  }^{  }.
\end{align}
    We aware that these inequalities also holds for the first order, that is.
\begin{align}
\label{eq:bi_n_cut_inverse_1}
\| \partial _{n} v_h \|_{ \mathcal{T} _{h} \cap \Gamma  }^{  } &\lesssim h^{-\frac{1}{2}} \| \nabla v \|_{ \mathcal{T}_h }^{  }, \\
\label{eq:bi_n_cut_inverse_2}
\| \partial _{n}  v_h \|_{ \mathcal{F}_h \cap \Omega    }^{  }  &  \lesssim   h^{-\frac{1}{2}} \| \nabla v_h \|_{ \mathcal{T}_h  }^{  }.
\end{align}
In fact, combining the second order inequalities we get the following identity.
\begin{equation}
\label{eq:bi_identity}
h\| \partial _{nn}  v_{h} \|_{ \mathcal{F}_h \cap \Omega    }^{2 } + h\| \partial _{nn} v_{h} \|_{ \mathcal{T} _{h} \cap \Gamma  }^{2  } \lesssim \| D^2 v_{h} \|_{ \mathcal{T} _{h}  }^{2  } \quad  \forall v_{h} \in V_{h}.
\end{equation}
For more information about the derivations of the inequalities, see discussion in \cite[Section 2.4]{gurkan2019stabilized}.

We may introduce our first assumption on the ghost penalty.  Inspired by the expansion for the $H^{1}$-norm as detailed in \cite[Equation 2.23]{gurkan2019stabilized}, we adopt an analogous approach for the $H^{2}$-norm in this scenario.
\begin{assumption*}[EP1]
    \label{as:bi_EP1}
    The ghost penalty $g_{h}$ extends the $H^{2}$ norm such that
    \begin{equation*}
    \| D^2 v \|_{ \mathcal{T} _{h} }^{ 2 } \lesssim  \| D^2 v \|_{ \Omega  }^{ 2 } + \abs{ v } _{g_{h}}^{2}.
    \end{equation*}
\end{assumption*}


Combing the results we get the following convenient corollary.

\begin{corollary}
    \label{cor:bi_inverse_thm}
    Let  $g_{h}$ satisfy Assumption \ref{as:bi_EP1}, then
    \begin{equation}
        \label{eq:inv_est}
            h\| \partial _{nn}  v_{h} \|_{ \mathcal{F}_h^{} \cap \Omega    }^{2 } + h\| \partial _{nn} v_{h} \|_{ \mathcal{T} _{h} \cap \Gamma  }^{2  }   \lesssim  \| D^2 v_{h} \|_{ \Omega  }^{ 2 } + \abs{ v_{h} } _{g_{h}}^{2} \\
              \lesssim \| v_{h} \|_{ A_{h} }^{  2} \quad  \forall v_{h} \in V_{h}
    \end{equation}
    From this is it also clear that \begin{equation}
        \label{eq:asta_Ah}
    \| v_{h} \|_{ a_{h},* }^{  }  \lesssim \| v_{h} \|_{ A_{h} }^{  } \quad  \forall v_{h} \in V_{h}
    \end{equation}
\end{corollary}
\begin{proof}
    The the first result \eqref{eq:inv_est} is a direct result of \eqref{eq:bi_identity}, Assumption \ref{as:bi_EP1} and the definition of $\| \cdot  \|_{ A_{h} }^{  } $.
    The second result \eqref{eq:asta_Ah} is simply a observation that the terms in \eqref{eq:inv_est} appears in $\| \cdot   \|_{a_{h},*  }^{  } $, hence, the inequality follows.
\end{proof}

\begin{lemma}
    \label{lemma:bi_Ah_coercive}
    The discrete form $A_{h}$ is coercive, that is, \[
    \| v_{h} \|_{ A_{h} }^{ 2 }  \lesssim A_{h}( v_{h},v_{h})\quad  \forall v_{h} \in V_{h}.
    \]
\end{lemma}

\begin{proof}
    Let $v_{h} \in V_{h}$.
    Observe that
    \begin{equation}
    A_{h}( v_{h},v_{h}) = a_{h}( v_{h},v_{h})  + \abs{ v_{h} }_{g_{h}}^{2}
    \end{equation}
    Firstly, the ghost penalty term is already a part of the $\| \cdot  \|_{ A_{h} }^{  } $ norm, hence, it only remains to check the $a_{h}( \cdot ,\cdot ) $ term.
    \begin{equation}
        \label{eq:coerciv_ah}
    \begin{split}
       a_{h}( v_{h},v_{h}) &=   \alpha \| v_{h} \|_{   \Omega   }^{2} + \| D^2v_{h} \|_{   \Omega  }^{2  }  + \frac{\gamma }{h}  \|  \jump{ \partial _{n} v_{h} }\|_{\mathcal{F} _{h}^{}  }^{ 2 } + \frac{\gamma }{h}  \| \partial _{n} v_{h} \|_{ \Gamma  }^{ 2 }. \\
                   & + 2 ( \mean{ \partial _{nn} v_{h} }, \jump{ \partial _{n} v_{h} }    )_{\mathcal{F} ^{}_{h} \cap \Omega }  + 2 (  \partial _{nn} v ,
       \partial _{n} v_{h}  )_{\Gamma } \\
    \end{split}
    \end{equation}
    We first focus on the last two terms in \eqref{eq:coerciv_ah}. Using Cauchy-Schwarz \eqref{eq:cauchy-schwartz}, we observe that
    \begin{equation}
        \begin{split}
    ( \mean{ \partial _{nn} v_{h} }  , \jump{ \partial _{n} v_{h} }  )_{\mathcal{F}^{}_{h}\cap \Omega  } & \ge - \| h^{\frac{1}{2}}\mean{ \partial _{nn} v_{h} }   \|_{ \mathcal{F}^{}_{h}\cap \Omega   }^{  }  \|h^{-\frac{1}{2}} \jump{ \partial _{n} v_{h} }   \|_{
    \mathcal{F}^{}_{h}\cap \Omega   }^{  } \\
    (  \partial _{nn} v_{h}   ,  \partial _{n} v_{h}   )_{\Gamma   } & \ge - \| h^{\frac{1}{2}} \partial _{nn} v_{h}    \|_{ \Gamma    }^{  }  \|h^{-\frac{1}{2}}  \partial _{n} v_{h}    \|_{ \Gamma    }^{  }
        \end{split}
    \end{equation}
    Using inverse-inequalities \eqref{eq:bi_cut_inverse_1} and \eqref{eq:bi_cut_inverse_2} and the Corollary \ref{cor:bi_inverse_thm} can we easily observe that
    \begin{equation}
        \begin{split}
     \| h^{\frac{1}{2}} \mean{ \partial _{nn}v_{h} } \|_{ \mathcal{F}_{h} \cap \Omega    }^{  2} & \le C_{1} \| D^2 v_{h} \|_{ \mathcal{T}_{h}   }^{2  } \lesssim   \| D^2 v_{h} \|_{ \Omega  }^{ 2 }  + \abs{ v_{h} } _{ g_{h} }^{2  }   \\
     \|  \partial _{nn}v_{h}  \|_{ \Gamma     }^{ 2 } & \le C_{2} \| D^2 v_{h} \|_{ \mathcal{T} _{h}  }^{2  } \lesssim    \| D^2 v_{h} \|_{ \Omega  }^{ 2 }  + \abs{ v_{h} } _{ g_{h} }^{2  }
        \end{split}
    \end{equation}
    Thus, by applying Youngs $\varepsilon $-inequality \eqref{eq:young-epsilon}, is it natural to see that,
    \begin{equation}
        \begin{split}
- C_{1}^{\frac{1}{2}} \| D^2 v_{h}    \|_{ \mathcal{T} _{h}   }^{  }  \|h^{-\frac{1}{2}} \jump{ \partial _{n} v_{h} }   \|_{ \mathcal{F}^{}_{h}\cap \Omega   }^{  }
& \ge - \frac{1}{\varepsilon } C  (\| D^2 v_{h} \|_{ \Omega  }^{ 2 }  + \abs{ v_{h} } _{ g_{h} }^{2  } ) -  \varepsilon \|h^{-\frac{1}{2}} \jump{ \partial _{n} v_{h} }   \|_{ \mathcal{F}^{}_{h}\cap \Omega   }^{2  } \\
- C_{2}^{\frac{1}{2}}  \| D^2 v_{h} \|_{ \mathcal{T} _{h} }^{  } \| h^{-\frac{1}{2}}  \partial _{n} v_{h}    \|_{ \Gamma    }^{  }
& \ge - \frac{1}{\varepsilon } C  (\| D^2 v_{h} \|_{ \Omega  }^{ 2 }  + \abs{ v_{h} } _{ g_{h} }^{2  } ) -  \varepsilon \|h^{-\frac{1}{2}}  \partial _{n} v_{h}    \|_{ \Gamma    }^{2  } \\
        \end{split}
    \end{equation}
    Combining these ideas do we end up with the following inequality,
    \begin{equation}
    \begin{split}
       a_{h}( v_{h},v_{h})  \ge& \  \alpha     \|\  v  \|_{   \Omega   }^{2} +\| D^2v_{h}  \|_{   \Omega   }^{2} -  \frac{1}{\varepsilon } 4C  (\| D^2 v_{h} \|_{ \Omega  }^{ 2 }  + \abs{ v_{h} } _{ g_{h} }^{2  } )  \\
                       & + (\gamma - 2\varepsilon  )\left( \|h^{-\frac{1}{2}}  \jump{ \partial _{n} v_{h} }\|_{\mathcal{F} _{h}^{} \cap \Omega   }^{ 2 } + \| h^{-\frac{1}{2}} \partial _{n} v_{h} \|_{ \Gamma  }^{ 2} \right)        \\
    \end{split}
    \end{equation}
    This inequality is useful since if we add a ghost penalty on the left hand side we get,
    \begin{equation}
        \begin{split}
     A_{h}( v_{h},v_{h}) & = a( v_{h},v_{h}) + \abs{ v_{h} }_{g_{h}}^{2} \\
     & \gtrsim    \   \| \ |\alpha|^{\frac{1}{2}} \  v_{h}  \|_{   \Omega   }^{2} + (1  - \frac{1}{\varepsilon } 4C)  (\| D^2 v_{h} \|_{ \Omega  }^{ 2 }  + \abs{ v_{h} } _{ g_{h} }^{2  } )  \\
                       & + (\gamma - 2\varepsilon  )\left( \|h^{-\frac{1}{2}}  \jump{ \partial _{n} v_{h} }\|_{\mathcal{F} _{h}^{}\cap \Omega   }^{ 2 } + \| h^{-\frac{1}{2}} \partial _{n} v_{h} \|_{ \Gamma  }^{ 2} \right)        .
        \end{split}
    \end{equation}
    Setting $\varepsilon = 8C$ and $\gamma = 32C $ we arrive at the desired state,
    \begin{equation}
        \begin{split}
           A_{h}( v_{h},v_{h})  \gtrsim & \   \| \ |\alpha|^{\frac{1}{2}} \    v_{h}  \|_{  \Omega   }^{2} + \frac{1}{2}  (\| D^2 v_{h} \|_{ \Omega  }^{ 2 }  + \abs{ v_{h} } _{ g_{h} }^{2  } )  \\
                       & + \frac{\gamma }{2}\left( \|h^{-\frac{1}{2}}  \jump{ \partial _{n} v_{h} }\|_{\mathcal{F} _{h}^{}\cap \Omega   }^{ 2 } + \| h^{-\frac{1}{2}} \partial _{n} v_{h} \|_{ \Gamma  }^{ 2} \right) \\
                        \gtrsim & \  \| v_{h} \|_{ A_{h} }^{2  }
        \end{split}
.
    \end{equation}
\end{proof}


\begin{lemma}
    \label{lemma:bi_Ah_bounded}
    The discrete form $A_{h}$ is bounded, that is,
    \begin{equation}
    \label{eq:bi_A_h_bounded}
     A_{h}( v_{h},w_{h}) \lesssim \| v_{h} \|_{A_{h}  }^{  }\| w_{h} \|_{A_{h}  }^{  } \quad   \forall v_{h},w_{h} \in V_{h}
    \end{equation}
    Moreover, for $v \in V_{h} \oplus V$  and $w_{h} \in V_{h}$ the discrete bilinear form $a_{h}( \cdot ,\cdot  ) $ satisfies,
    \begin{equation}
        \label{eq:bi_a_h_bounded}
        a_{h} ( v,w_{h}) \lesssim \| v \|_{ a_{h},* }^{  } \| w_{h} \|_{ A_{h} }^{  }.
    \end{equation}
\end{lemma}

\begin{proof}
         \textbf{Step 1.} The goal is to prove the inequality \eqref{eq:bi_A_h_bounded}.
         \begin{equation}
                \abs{ A_{h}( v_{h} ,w_{h} ) } \lesssim   \abs{a_{h}( v_{h}, w_{h}) }   + \abs{g_{h}( v_{h},w_{h})  }
         \end{equation}
                By assumption is the ghost penalty $g_{h}$ positive semi-definite, thus, it fulfills the Cauchy-Schwarz inequality \eqref{eq:cauchy-schwartz},
                \begin{equation}
                \abs{ g_{h}(v_{h},w_{h} ) } \lesssim \abs{ v_{h} } _{g_{h}}\abs{ w_{h} }_{g_{h}}.
                \end{equation}
                Hence, by definition of $A_{h}( \cdot ,\cdot ) $, is $\abs{ g_{h}(v_{h},w_{h} ) } \lesssim \| v_{h} \|_{ A_{h} }^{  } \| w_{h} \|_{ A_{h} }^{  } $. It remains to show that the bilinear term $ a_{h}( \cdot ,\cdot ) $ is bounded. We numerate the terms in this fashion.
                \begin{equation}
                    \begin{split}
                         a_{h} \left( v_{h}, w_{h} \right)    \le  &   \left( \alpha  v_{h}, w_{h} \right) _{\mathcal{T} _{h} \cap \Omega }      +  \left( D^2 v_{h}, D^2w_{h} \right) _{\mathcal{T} _{h} \cap \Omega}    \\
                                                     & + \left( \mean{  \partial _{n n} v_{h} }, \jump{ \partial _{n }w_{h}} \right)_{\mathcal{F}_{h}^{} \cap \Omega}     + \left( \mean{ \partial _{nn } w_{h} }, \jump{ \partial _{n}v_{h} }
                                                     \right)_{\mathcal{F}_{h}^{} \cap \Omega}+ \frac{\gamma }{h}  \left( \jump{ \partial _{n} v_{h}}, \jump{ \partial _{n} w_{h}   }   \right)_{\mathcal{F}_{h}^{} \cap \Omega}   \\
                                                     & + \left(  \partial _{n n} v_{h} ,  \partial _{n }w_{h} \right)_{\Gamma } + \left(  \partial _{n n} w_{h} ,  \partial _{n}v_{h}       \right)_{\Gamma }   + \frac{\gamma }{h}  \left(  \partial _{n}
                                                         v_{h},  \partial
                                                     _{n} w_{h} \right)_{\Gamma } \\
                                                     &=  ( \mathrm{I} ) + \ldots+( \mathrm{VIII} )     \\
                    \end{split}
                \end{equation}
                The strategy is to bound each term individually using Cauchy-Schwarz \eqref{eq:cauchy-schwartz}. From this is it easy to see that $\abs{( \mathrm{I} )   } +  \abs{( \mathrm{II} )   }   \lesssim \| v_{h} \|_{a_{h}  }^{  } \| w_{h} \|_{ a_{h}
                }^{  } $. For the terms symmetrical terms $(
                \mathrm{III} ) $ and  $( \mathrm{IV} ) $ we apply the inverse inequality
                \eqref{eq:bi_cut_inverse_2}.
                \begin{equation}
                    \label{eq:invest_1}
                    \abs{(\mathrm{III} )  }  \lesssim  \|h^{\frac{1}{2}} \partial _{n n} v_{h}  \|_{ \mathcal{F}_{h}^{} \cap \Omega}^{  }\| h^{-\frac{1}{2}} \jump{ \partial _{n} w_{h} }     \|_{\mathcal{F}_{h}^{} \cap \Omega}^{  } \lesssim  \| v_{h} \|_{A_{h}
                    }^{  } \|w    \|_{ a_{h}}^{  }.
                \end{equation}
                Here we used that  $\|h^{\frac{1}{2}} \partial _{n n}  v_{h} \|_{\mathcal{F}_{h}^{} \cap \Omega} \lesssim \| v_{h} \|_{ A_{h}  }^{  }  $ thanks to Corollary \ref{cor:bi_inverse_thm}.
              The interior penalty can we easily see that,
              \begin{equation}
              \abs{ ( \mathrm{V} )  }  \lesssim  \|h^{-\frac{1}{2}} \jump{ \partial _{n} v_{h}}  \|_{ \mathcal{F}_{h}^{} \cap \Omega }^{  }
             \|h^{-\frac{1}{2}} \jump{ \partial _{n} w_{h}}  \|_{ \mathcal{F}_{h}^{} \cap \Omega }^{  }  \lesssim  \| v_{h}  \|_{ a_{h} }^{  }
             \| w_{h}  \|_{ a_{h} }^{  }.
              \end{equation}
              It remains to handle the symmetry terms $ ( \mathrm{VI} )$ and $ ( \mathrm{VII} )$.
              \begin{equation}
                    \label{eq:invest_2}
                \abs{ ( \mathrm{VI} )  }  \lesssim \| h^{\frac{1}{2}}\partial _{nn} v_{h} \|_{\Gamma   }^{  } \| h^{-\frac{1}{2}} \partial _{n}w_{h} \|_{\Gamma   }^{  }  \lesssim \|  v_{h} \|_{A_{h}  }^{  } \| w_{h} \|_{ a_{h}   }^{  }
              \end{equation}
             Again, here we used the Corollary \ref{cor:bi_inverse_thm}.
             Finally, using the definition of the norm is it easily to see that,
             \[
\abs{ ( \mathrm{VIII} )  }  \lesssim \| \partial _{n}v_{h} \|_{ \Gamma  }^{  }
\| \partial _{n} w_{h} \|_{ \Gamma  }^{  }  \lesssim \| v_{h} \|_{ a_{h} }^{  }
\| w_{h} \|_{ a_{h} }^{  } .
             \]

             Hence, we can conclude \begin{equation}
                 \label{eq:ah_ahnorm}
                 \abs{ a_{h}( v_{h},w_{h})  } \le \| v_{h} \|_{ a_{h} }^{  } \| w_{h} \|_{ a_{h} }^{  } \forall v_{h},w_{h} \in V_{h}.
             \end{equation}
             Therefore, since $\| \cdot  \|_{a_{h}  }^{  } \lesssim  \| \cdot  \|_{A_{h}  }^{  } $, it has been demonstrated that $a_{h}( \cdot ,\cdot )$ is bounded within the $\|\cdot   \|_{A{h} }^{ }$ norm.

             \textbf{Step 2.} The goal is to prove \eqref{eq:bi_a_h_bounded}. Let $v \in V_{h} \oplus V $ and $w_{h} \in V_{h}$.
             The only difference is that since $v$ is continious we cannot apply to Corollary \ref{cor:bi_inverse_thm} on the estimates \eqref{eq:invest_1} and \eqref{eq:invest_2}. However, this is not a problem since $\| h^{\frac{1}{2}} \mean{ \partial _{nn} v }
             \|_{\mathcal{F} _{h}\cap  \Omega   }^{  } $ and  $\|h^{\frac{1}{2}}  \partial _{nn}v \|_{\Gamma   }^{  } $ are terms in the norm $\|  v \|_{a_{h},*  }^{  } $. Thus, we know that
             \begin{equation}
                  \abs{ a_{h}( v,w_{h}) }  \le \| v \|_{ a_{h},* }^{  } \| w_{h} \|_{ A_{h} }^{  } \ \ \forall v \in V_{h} \oplus V \text{ and } w_{h} \in V_{h}
             \end{equation}

\end{proof}





    % 
\subsection{A priori error estimate}%
\label{sec:a_priori_estimates}


For the proposed method, we want to derive a priori error estimate with respect to both the  $\| \cdot  \|_{a_{h},*   }^{  } $-norm and the  $\| \cdot  \|_{ \Omega  }^{
} $-norm.  These estimates are geometrically robust in that they remain unaffected by specific cut configurations, thanks to the ghost penalty they incorporate.
First, we construct a suitable (quasi-)interpolation operator, here we use the Clement quasi interpolation operator which in contrast to the standard Lagrange nodal interpolation operator is also defined for low regularity function $u \in L^{2}(
\Omega ) $.
In combination with discrete coercivity this allows us to derive an a priori error estimate in the energy norm. Finally, we use a standard duality argument, also known as Aubin-Nitsche trick, to derive the $L^{2}( \Omega ) $-error estimate.

Recall that for $v \in H^{3}( \mathcal{T } _{h}) $ the following inequalities.
\begin{align}
    \label{eq:trace:1}
    \| \nabla v \|_{ \partial T }^{  } &\lesssim h^{-\frac{1}{2}}_{T}\|  \nabla v \|_{ T }^{  }+ h^{\frac{1}{2}} \| D^2 v \|_{T  }^{   }  , \\
    \label{eq:trace:2}
    \| \nabla v \|_{ \Gamma \cap T }^{  } &\lesssim  h^{-\frac{1}{2}} \| \nabla v \|_{T  }^{  }   + h^{\frac{1}{2}}_{T} \| D^2 v \|_{ T }^{  },\\
    \label{eq:trace:3}
    \| D^2 v \|_{ \Gamma \cap T }^{  } &\lesssim  h^{-\frac{1}{2}} \| D^2 v \|_{T  }^{  }   + h^{\frac{1}{2}}_{T} \| D^3 v \|_{ T }^{  },
\end{align}
holds $\forall T \in \mathcal{T} _{h}$, for proof see \cite[Lemma 4.2]{hansbo2003finite}. In this context is $D^3v$ a tensor defined in Equation \eqref{eq:tensor}.

Assume that $\Omega $ has a boundary $\Gamma $ in $C^{1}$, then there exists a bounded extension operator,
\begin{equation}
    ( \cdot ) ^{e}: H^{m}( \Omega )  \to H^{m} ( \mathbb{R} ^{d}),
\end{equation}
for all  $v \in H^{m}( \Omega )$ which satisfies
\begin{equation}
    \begin{split}
 v^{e}| _{\Omega } =   v,  \\
\| v^{e} \|_{ m,\mathbb{R} ^{d}  }^{  } & \lesssim \| v \|_{ m, \Omega  }^{  }.
    \end{split}
\end{equation}
For more information, see \cite[Theorem 9.7]{brezis2011functional} and \cite[p.181, p.185]{stein1970singular}. For the notation we simply write $ v := v^{e}   $ for $v \in \mathbb{R} ^{d} \setminus \Omega $.


Starting from Lemma \ref{lemma:clements}, assume $v \in H^{s}( \Omega ) $ and let $r = \mathrm{min}(s , k+1) $. Revisit the definition of $V_{h}$ from \eqref{eq:vh_energy}, which is a polynomial of degree $k$. We can then employ the combination of the Clément interpolator with the extension operator to create $C_{h}^{e}: H^{m}( \mathbb{R} ^{d}) \to V_{h}$, such that $C ^{e} _{h} v := C _{h} v^{e}$.
 Next, recall that $ \sum_{T\in \mathcal{T}_{h}}^{} \| v \|_{s,\omega ( T)   }^{  } \le C  \| v \|_{s, \mathcal{T}_{h}   }^{  } $ where $C$ is some constant decided by shape regularity of the mesh and the maximal number of different patches a single element can
 belong to. This also holds for the inequality $ \sum_{T \in \mathcal{T}_{h} }^{} \sum_{F \in \partial T}^{}  \| v \|_{s,\omega ( F)   }^{  } \le C  \| v \|_{s, \mathcal{T}_{h}   }^{  } $.
The following estimates are thereby established.

\begin{align}
    \label{eq:bi_projection_estimates_1}
    \| v - C _{h}^{e} v \|_{  l, \mathcal{T} _{h} }^{  } & \lesssim h^{r-l}\sum_{T \in \mathcal{T}_h} \| v \|_{ r, \omega(T) }^{  } \lesssim  h^{r-l-\frac{1}{2}}  \| v \|_{ r, \Omega  }^{  }, \quad 0\le l\le r, \\
    \label{eq:bi_projection_estimates_2}
\| v - C ^{e}_{h}v \|_{ l, \partial \mathcal{T} _{h} }^{  } & \lesssim h^{r-l-\frac{1}{2}}\sum_{T \in \mathcal{T}_h} \sum_{F \in \partial T} \| v \|_{ r, \omega(F)  }^{  } \lesssim h^{r-l-\frac{1}{2}} \| v \|_{ r, \Omega   }^{  }, \quad 0  \le  l \le   r-\frac{1}{2}.
    % \label{eq:bi_projection_estimates_3}
    % \| v - C ^{e}_{h}v \|_{ l, \Gamma }^{  } & \lesssim h^{r-l-\frac{1}{2}} \sum_{T \in \mathcal{T}_h}  \| v \|_{ r,  \omega(T)  }^{  }\lesssim h^{r-l-\frac{1}{2}} \lesssim    \| v \|_{ r,  \Omega   }^{  }, \quad 0  \le  l \le  r-\frac{1}{2}.
\end{align}

% \todo[inline]{ Maybe hard to argue \eqref{eq:bi_projection_estimates_3} to hold on $\Gamma $, but may be related to some generalization of \eqref{eq:bi_n_cut_inverse_1} and \eqref{eq:bi_cut_inverse_1}. Anyhow, \eqref{eq:bi_projection_estimates_2} and
% \eqref{eq:bi_projection_estimates_3} was never used in the proof of Lemma \ref{lemma:astar_estimate} since we used inverse estimates and ended up with \eqref{eq:bi_projection_estimates_1} instead on all of them.}

\begin{lemma}
    \label{lemma:astar_estimate}
    Let $u \in H^{s}( \Omega ) $ for $s\ge 3$ be the exact solution to $\eqref{eq:cont_weak_problem} $ and let $k$ be the polynomial order of $V_{h}$. Set $r = \mathrm{min} ( s, k+1)$, then we have the interpolation estimates
    \begin{equation}
    \|  u - C_{h}u \|_{ a_{h},*  }^{  } \lesssim h^{r-2} \| u \|_{ r, \Omega  }^{  }.
    \end{equation}
\end{lemma}

\begin{proof}
    By definition,
    \begin{equation}
        \begin{split}
            \| u - C_{h}^{e}u \|_{ a_{h}, * }^{  2}  =& \ \alpha  \overbrace{\|  ( u - C_{h}^{e}u) \|_{ \mathcal{T} _{h} \cap \Omega  }^{ 2}}^{\mathrm{I} }   + \overbrace{\| D^2 ( u - C_{h}^{e}u ) \|_{\mathcal{T} _{h} \cap \Omega   }^{ 2
            }}^{\mathrm{II} }  \\  &  +
            \overbrace{\gamma \| h^{-\frac{1}{2}} \jump{ \partial _{n} (u -
        C_{h}^{e} u) }   \|_{ \mathcal{F}_{h}^{}\cap \Omega    }^{ 2
        }}^{\mathrm{III} }  + \overbrace{\gamma \| h^{-\frac{1}{2}}  \partial _{n} (u - C_{h}^{e}u)    \|_{ \Gamma   }^{ 2 }}^{\mathrm{IV} }  \\
          & + \overbrace{\| h^{\frac{1}{2}} \mean{ \partial _{nn} (u - C_{h}^{e}u) }   \|_{\mathcal{F} _{h}^{} \cap \Omega   }^{  2}}^{\mathrm{V} }  +  \overbrace{\| h^{\frac{1}{2}} \partial _{nn}(u - C_{h}^{e}u)     \|_{ \Gamma }^{  2}}^{\mathrm{VI}
          } \\
          =& \  \mathrm{I}  + \ldots + \mathrm{VI}.
        \end{split}
    \end{equation}
    The strategy is to bound each term individually.
    By initially focusing on the first two terms and employing equation \eqref{eq:bi_projection_estimates_1}, we can easily observe
             \begin{equation}
        \begin{split}
            \mathrm{I} +\mathrm{II}  & \lesssim \|   u - C_{h}^{e}u \|_{ \mathcal{T} _{h}  }^{ 2} + \|  D^2( u - C_{h}^{e}u )  \|_{\mathcal{T} _{h} }^{ 2 } \\
                                     & \lesssim  ( h^{2r}  + h^{2(r-2)} )\| u \|_{r,\mathcal{T}_{h} }^{  2} \lesssim h^{2(r -2)} \| u
                                     \|_{r, \mathcal{T} _{h} }^{2  }  .
        \end{split}
             \end{equation}
    From \eqref{eq:mean_jump_estimate} is it clear that $\| \jump{ \partial _{n} u }   \|_{ \mathcal{F}_{h}   }^{  } \lesssim \| \nabla  u \|_{ \partial  \mathcal{T} _{h} }^{  }   $. Hence, first applying the trace inequality \eqref{eq:trace:1}  and then
    \eqref{eq:bi_projection_estimates_1} is it clear that,
    \begin{equation}
        \begin{split}
            \mathrm{III}  & \lesssim h^{-1} \|  \nabla ( u - C_{h}^{e})  \|_{\partial \mathcal{T}_{h}   }^{2  }  \lesssim h^{-2} \| \nabla ( u - C^{e}_{h}u)  \|_{ \mathcal{T} _{h}
                          }^{ 2 } + \|  D^2  ( u - C_{h}^{e})  \|_{\mathcal{T}_{h}   }^{ 2 } \\
                          & \lesssim  ( h ^{2( r-1) -2 } + h^{2( r-2) } ) \| u  \|_{r, \mathcal{T}_{h}   }^{2  }  \lesssim  h^{2( r-2) }  \| u  \|_{r, \mathcal{T}_{h}   }^{ 2 }
    \end{split}
\end{equation}
% And for the boundary term we apply estimate \eqref{eq:bi_projection_estimates_3}
%         \begin{equation}
%             \mathrm{IV}   \lesssim h^{-1} \|  \nabla  ( u - C_{h}^{e}u ) \|_{ \Gamma  }^{2  }  \lesssim  h^{2( r - 2 )} \| u \|_{r, \mathcal{T}_{h}   }^{  }
%         \end{equation}
And for the boundary term we apply \eqref{eq:trace:2} and then \eqref{eq:bi_projection_estimates_1}
        \begin{equation}
            \begin{split}
            \mathrm{IV}   & \lesssim h^{-1} \|  \nabla  ( u - C_{h}^{e}u ) \|_{ \Gamma  }^{2  }    \lesssim h^{-2} \| \nabla ( u - C_{h}^{e}u )  \|_{ \mathcal{T}_{h}   }^{2  } + \| D^2( u - C_{h}^{e}u ) \|_{ \mathcal{T}_{h}   }^{ 2 } \\   & \lesssim  h^{2( r-2) }  \| u  \|_{r, \mathcal{T}_{h}   }^{ 2 }
            \end{split}
        \end{equation}
Again, from \eqref{eq:mean_jump_estimate} is it clear that $\| \mean{ \partial _{nn} u }   \|_{ \mathcal{F}_{h}   }^{  } \lesssim \| D^2  u \|_{ \partial  \mathcal{T} _{h} }^{  }   $, thus we see that,
        \begin{equation}
            \mathrm{V}   \lesssim h \|  D^2  ( u - C_{h}^{e}u ) \|_{\partial \mathcal{T}_{h}}^{2  }  \lesssim  h^{2( r - 2 )} \| u \|_{r, \mathcal{T}_{h}   }^{ 2 }.
        \end{equation}
        The final term we we apply \eqref{eq:trace:3} and then \eqref{eq:bi_projection_estimates_1}

        \begin{equation}
            \begin{split}
            \mathrm{VI} &    \lesssim h \|  D^2  ( u - C_{h}^{e}u ) \|_{\Gamma }^{2  }  \\
            & \lesssim  h^{-2} \| D^2 ( u - C_{h}^{e}u )  \|_{ \mathcal{T}_{h}   }^{2  } + \| D^3( u - C_{h}^{e}u ) \|_{ \mathcal{T}_{h}   }^{ 2 } \\
            % & \lesssim h^{-2}( \| u- C_{h}^{e} \|_{2,\mathcal{T}_{h}   }^{2  } + \| u- C_{h}^{e} \|_{3,\mathcal{T}_{h}   }^{2  }  )\\
            &  \lesssim h^{-2}( h^{2(r - 2 - \frac{1}{2})} + h^{2(r - 3 - \frac{1}{2})})\| u \|_{r,\Omega   }^{  } \\
            &  \lesssim ( h^{2r - 5)} + h^{2r - 6)})\| u \|_{r,\Omega   }^{  } \lesssim h^{2(r-2)} \| u \|_{r,\Omega   }^{  }
            \end{split}
        \end{equation}
    Hence, we have $ \| u - C_{h}^{e} u \|_{a_{h},*  }^{  } \lesssim   h^{r-2}  \| u \|_{ r, \mathcal{T}_{h}   }^{2  } $.
\end{proof}

\begin{lemma}[Weak Galerkin orthogonality]
Let $u \in H^{s}( \Omega )  $, $ s\ge 3 $  be the exact solution to   \eqref{eq:cont_weak_problem} and $u_{h} \in V_{h}$ is a discrete solution to \eqref{eq:discrete_CutCIP_prob}. Then is \[
    a_{h}( u - u_{h}, v_{h}) = g_{h} ( u_{h}, v_{h}) \quad \forall v_{h} \in V_{h}.
    \]
\end{lemma}

\begin{proof}
   From the definition of the problem \eqref{eq:discrete_CutCIP_prob} and utilizing that for $u \in H^{s}( \Omega ) $ we have the identity  $A_{h}( u,v_{h}) = a_{h}( u,v_{h}) = l(v_{h} )  \ \forall v_{h} \in V_{h} $. Consequently, it follows that \[
       \begin{split}
   l(v_{h} ) & =  A_{h}( u_{h},v_{h}) =  a_{h}( u,v_{h})  = a_{h}( u_{h},v_{h})+g_{h}( u_{h},v_{h})  \quad \forall  v_{h} \in  V_{h}.
       \end{split}
   \]
    Hence, we have $a_{h}( u -  u_{h}, v_{h}) = g_{h}( u_{h},v_{h})  $.
\end{proof}

\begin{assumption*}[EP2]
    For $v \in H^{s}( \Omega ) $ and $r = \min \{s,k+1 \} $, the semi-norm $\abs{ \ \cdot \  }_{g_{h}} $ is weakly consistent in the sense that
    \begin{equation}
        \label{as:bi_EP2}
        \abs{ C _{h}^{e} v } _{g_{h}} \lesssim  h^{r-2} \| v \|_{ r,\Omega  }^{  }.
    \end{equation}
\end{assumption*}

\begin{theorem}
    \label{thm:apriori_result}
    Let $u \in H^{s}( \Omega ) $ , $s\ge 3$ be a solution to \eqref{eq:cont_weak_problem} and let $u_{h} \in V_{h}$ of order $k\ge 2$ be the discrete solution to \eqref{eq:discrete_CutCIP_prob}. Then with $r = \min_{}\{s, k+1\} $ the error $e = u - u_{h}$ satisfies
    \begin{align}
        \label{eq:bi_apriori_1}
            \| e \|_{ a_{h},* }^{  } &\lesssim   h^{r-2} \| u \|_{ r,\Omega  }^{  },\\
        \label{eq:bi_apriori_2}
        \| e \|_{ \Omega  }^{  } &\lesssim   h^{r-\mathrm{max}\left\{ 0, 3-k \right\} } \| u \|_{ r,\Omega  }^{  }.
    \end{align}

\end{theorem}
\begin{remark}
    Be aware that for $k=2$ the estimate \eqref{eq:bi_apriori_2} is suboptimal with $1$ order.
\end{remark}

\begin{proof}
    We will divide the proof into two steps.
    \\
        \textbf{Step 1.} We want to prove that $\| e \|_{ a_{h},* }^{  } \lesssim   h^{r-2} \| u \|_{ r,\Omega  }^{  }$.
    Decompose $e = u - u_{h}$ intro $e = e_{h} + e_{\pi }$, where we denote the discrete error $e_{h} = C _{h}^{e} u - u_{h}$ and the interpolation error $e_{\pi } = u - C _{h} ^{e}u$. We can then observe that
    \begin{equation}
        \begin{split}
    \| u - u_{h} \|_{ a_{h} }^{  } & \le   \| u - C_{h}^{e} u + C_{h}^{e}u - u_{h} \|_{ a_{h},* }^{  } \\
    & \le \|  u - C_{h}^{e} u \|_{a_{h},*  }^{  } +  \| C_{h}^{e}u - u_{h} \|_{a_{h},*  }^{  }\\
                                     & \lesssim  \| e_{\pi } \|_{a_{h},*}^{  } + \| e_{h} \|_{A_{h}  }^{  }.
        \end{split}
    \end{equation}
    Using Lemma \ref{lemma:astar_estimate}, is it clear that $\| e_{\pi } \|_{a_{h},*}^{  } \lesssim h^{r-2} \| u \|_{ r,\Omega  }^{  }  $ is already fulfilled, hence, it remains to estimate $e_{h}$. From Lemma \ref{lemma:bi_Ah_coercive} and
    \ref{lemma:bi_Ah_bounded}, the weak Galerkin orthogonality and Assumption EP2 \eqref{as:bi_EP2} is it natural to arrive at,
    \begin{equation}
        \label{eq:apriori_energy1}
    \begin{split}
\| e_{h} \|_{ A_{h} }^{ 2 } & \lesssim a_{h}( C _{h}^{e} u - u_{h}, e_{h}) + g_{h}( C _{h}^{e}u - u_{h}, e_{h}) \\
 & = a_{h}( C _{h}^{e} u - u, e_{h}) + a_{h}( u - u_{h}, e_{h}) + g_{h}( C _{h}^{e}u - u_{h}, e_{h}) \\
 & = a_{h}( C _{h}^{e} u - u, e_{h}) + g_{h}( C _{h}^{e}u, e_{h}).
 % & \lesssim h^{r-2} \| u \|_{ r, \Omega  }^{  } \| e_{h} \|_{ A_{h} }^{  }.
    \end{split}
    \end{equation}
Hence, now utilizing the Assumption EP2 \eqref{as:bi_EP2} is it clear that
\begin{equation}
        \label{eq:apriori_energy2}
    \begin{split}
        a_{h}( C _{h}^{e} u - u, e_{h}) + g_{h}( C _{h}^{e}u, e_{h}) &\lesssim \| C _{h}^{e} u - u \|_{a_{h},*  }^{  } \| e_{h} \|_{a_{h}  }^{  }
        + \abs{ C _{h}^{e}u }_{g_{h}} \abs{e_{h}  }_{g_{h}} \\
         &\lesssim \| C _{h}^{e} u - u \|_{a_{h},*  }^{  } \| e_{h} \|_{a_{h}  }^{  } + h^{r-2} \| u \|_{r, \Omega   }^{  }\abs{e_{h}  }_{g_{h}} \\
         &\lesssim (\| C _{h}^{e} u - u \|_{a_{h},*  }^{  } + h^{r-2} \| u \|_{r, \Omega   }^{  }) \|e_{h}\|_{A_{h}} \\
         &\lesssim  h^{r-2} \| u \|_{r, \Omega   }^{  } \|e_{h}\|_{A_{h}}.
    \end{split}
\end{equation}
Here we noticed that $\| e_{h} \|_{a_{h}  }^{  } + \abs{e_{h}  }_{g_{h}} \lesssim \| e_{h} \|_{ A_{h} }^{  }  $, and used that $\| C _{h}^{e} u - u \|_{a_{h},*  }^{  } \lesssim h^{r-2}\| u \|_{ r,\Omega  }^{  }  $ from Lemma
\ref{lemma:astar_estimate}. Combining \eqref{eq:apriori_energy1} and \eqref{eq:apriori_energy2} and dividing by $\| e_{h} \|_{ A_{h} }^{  } $  is it clear that $\| e_{h} \|_{ A_{h}  }^{  } \lesssim h^{r-2} \| u \|_{r, \Omega   }^{  }  $.
Hence, the first part of the proof is complete.

        \textbf{Step 2.}
        We want to show that $ \| e \|_{ \Omega  }^{  } \lesssim   h^{r- \mathrm{max}(0,3-k)} \| u \|_{ r ,\Omega  }^{  }$. The idea is to apply the so-called Aubin-Nitsche duality trick while being aware of the ghost penalty $g_{h}$. Let us denote the following
        observation.
        Let $e:= u -u_{h} \in L^{2}( \Omega ) $ and $\psi  \in H^{4}( \Omega ) $.
        Then is the corresponding dual problem to \eqref{eq:bi_problem} be
        \begin{equation}
            \begin{split}
            \Delta ^2 \psi &= e  \quad  \text{ in } \Omega  \\
            \partial _{n} \psi &= 0 \quad \text{ on } \Gamma \\
            \partial _{n} \Delta \psi & = 0 \quad  \text{ on } \Gamma   \\
            \end{split}
        \end{equation}
 Here is the regularity of $\psi $ and $e$ a consequence of the assumptions of the regularity of $u$ and $u_{h}$  \cite[pp. 113]{brenner2012}.
        This implies that it exists a $\psi \in H^{4}( \Omega ) $ such that $a_{h}(v, \psi ) = ( e,v)_{\Omega } \ \forall v \in V_{h}  $. Hence, we can easily observe that \begin{equation}
            \label{eq:ni_1}
            \begin{split}
        \| e \|_{ \Omega  }^{ 2 }  & = ( e,e)_\Omega   = ( e, \Delta ^2 \psi )_{\Omega } \\
        &= a_{h}( e, \psi  ) = a_{h}( u-u_h, \psi ) \\
        &= a_{h}( u-u_h, \psi + C^{e}_{h}\psi  - C^{e}_{h}\psi )  \\
        &= a_{h}( u-u_h, \psi   - C^{e}_{h}\psi ) +  a_{h}( u-u_h, C^{e}_{h}\psi )  \\
        &= a_{h}( u-u_h, \psi  - C^{e}_{h}\psi ) +  g_{h}( u_h, C^{e}_{h}\psi ) \\
        &= a_{h}( u-u_h, \psi  - C^{e}_{h}\psi ) +  g_{h}( u_h - C^{e}_{h}u, C^{e}_{h}\psi )+  g_{h}( C^{e}_{h}u, C^{e}_{h}\psi ) \\
        &  = \mathrm{I} + \mathrm{II} + \mathrm{III}
            \end{split}
        \end{equation}


        For the first term we know by Cauchy-Schwartz inequality that
        \begin{equation}
             \mathrm{I}    \lesssim     \|u-u_{h}  \|_{a_{h},*  }^{  }    \| \psi  - C^{e}_{h}\psi \|_{a_{h},*  }^{  } = \mathrm{I} _{a} \mathrm{I} _{b}.
        \end{equation}
        Now, for $\mathrm{I}_{a} $ we simply apply the energy a priori estimate \eqref{eq:bi_apriori_1}
        \begin{equation}
            \label{eq:ni_2}
            \mathrm{I}_{a} \lesssim h^{r-2} \| u \|_{ r,\Omega   }^{  }
        \end{equation}
        However, to estimate $\mathrm{I}_{b} $ we set $\widetilde{r} = \mathrm{min} ( 4, k+1)  $, where $4$ comes from the regularity $\psi \in H^{4}( \Omega ) $ and $\| \psi \|_{4,\Omega   }^{  } \lesssim \| e \|_{ \Omega   }^{  }  $, thus,
        \begin{equation}
            \label{eq:ni_3}
            \mathrm{I}_{b} \lesssim h^{\widetilde{r} -2} \| \psi \|_{ \widetilde{r},\Omega   }^{  } \lesssim h^{\widetilde{r} -2} \| e \|_{ \Omega   }^{  }.
        \end{equation}
        Specifically for the term $\mathrm{II} $ we have \begin{equation}
            \begin{split}
                \mathrm{II}  & \lesssim \abs{ u_{h} - C_{h}^{e}u } _{g_{h}} \abs{ C^{e}_{h} \psi }_{g_{h}}  \\
                & \lesssim  \| u_{h} - C_{h}^{e}u  \|_{A_{h}} \abs{ C^{e}_{h} \psi }_{g_{h}} \\
                             & \lesssim h^{r-2}  \| u  \|_{r,\Omega } h^{\widetilde{r}-2} \| \psi  \|_{\widetilde{r}, \Omega   }^{  } \\
                             & \lesssim h^{r + \widetilde{r} -4}   \| u  \|_{r,\Omega } \| e \|_{ \Omega   }^{  }
            \end{split}
        \end{equation}
        Hence, combining \eqref{eq:ni_1}, \eqref{eq:ni_2} and \eqref{eq:ni_3}  can we conclude \begin{equation}
            \| e \|_{\Omega   }^{  } \lesssim h^{r+ \widetilde{r} -4} \| u \|_{ r,\Omega   }^{  }
        \end{equation}


        Having a clear look at $\widetilde{r}$, wee see that \begin{equation}
            \widetilde{r} = \mathrm{min}(4, k+1) = \begin{cases}
                & 3, \quad  k=2 \\
                & 4, \quad  k\ge 3 \\
            \end{cases}
        \end{equation}

        So we have the following estimate,\begin{equation}
            \| e \|_{ \Omega  }^{  }  \lesssim  \| u \|_{ r,\Omega  }^{  }  \begin{cases}
                h^{r-1}, \quad k=2 \\
                h^{r-2}, \quad k\ge 3
            \end{cases}
        \end{equation}
        or equivalently $\| e \|_{\Omega   }^{  } \lesssim \| u \|_{\Omega   }^{r- \mathrm{max}(0,k-3)   } $.


\end{proof}



    % % 
% \newpage
% \subsection{Condition number}%
% \label{sec:condition_number}

% Here is a nice article \cite{li07}

    % 
\newpage
\subsection{Constructing Ghost Penalties}%
\label{sec:constructing_ghost_penalties}

We have the following assumptions for the ghost penalty.
\begin{enumerate}[label=\textbf{EP\arabic*}]
    \item \label{as:EP1} The ghost penalty $g_{h}$ extends the $H^{1}$ norm s.t. \[
    \| D^2v \|_{ \mathcal{T} _{h} }^{ 2 }  \lesssim \| D^2 v \|_{ \Omega  }^{  2} + \abs{ v } _{g_{h}}^2
    \]
\item \label{as:EP2} For $v \in H^{s}( \Omega ) $ and $r = \min \{s,k+2 \} $, the semi-norm $\abs{ \cdot  }_{g_{h}} $ satisfies the following estimate, \[
    \abs{ \pi _{h}^{e} v } _{g_{h}} \lesssim  h^{r-2} \| v \|_{ r,\Omega  }^{  }.
    \]
\end{enumerate}


Our goal is to construct a variant of face-based ghost penalties such that the fulfilled.
What we will see is the following result is fundamental.

\begin{lemma}
    Let $T_{1},T_{2 } \in  \mathcal{T} _{h}$ be two elements sharing a common face $F$. Then for $v \in V_{h}$  we have \[
    \| v \|_{ T_{1} }^{  }  \lesssim \| v \|_{ T_{2} }^{  } \sum_{0\le j\le k}  {h^{2j +1}}^{} ( \jump{ \partial _{n}^{j} v }, \jump{ \partial ^{j}_{n} v }    )_{F}
    \]

\end{lemma}
\begin{proof}
    See \cite[Lemma 2.19]{gurkan2019stabilized}.
\end{proof}
The goal in this chapter is to engineer an ghost penalty which fulfills these assumptions.
We denote the multi-index $\alpha  = ( \alpha _{1}, \ldots, \alpha _{d})  $ of order $\abs{ \alpha  } = \sum_{i}^{}  \alpha _{i} = k $   and the normal vectors $n^{\alpha } = n_{1}^{\alpha _{1}} \ldots n_{d}^{\alpha _{d}}$.
Recall the notation for the derivates $D^{\alpha } v$
\footnote{
Remark that the operator $D^{\alpha }$  is related to with the derivate operator $\partial ^{\alpha } $ introduced in \eqref{eq:der}. For instance, for $d=2$ we have $\alpha = ( \alpha _{1}, \alpha _{2}) $  such that $ D^{1} v =  \nabla v  = \left[ \partial ^{( 1,0 )} v,\partial ^{( 0,1 )}v
\right]^{T}$ and $   D^2 v  = \begin{bmatrix}
\partial ^{( 2,0 )} v &  \partial ^{( 1,1) }v \\
\partial ^{( 1,1 )} v &  \partial ^{( 0,2) }v
\end{bmatrix}
 $.
},
that is \[
D ^{0} v  = v, \quad   D ^{1}v  = \nabla v \text{ and }  D ^{2} v  = J(\nabla v) = \mathrm{Hess}(v).
\]
where $J$ is the Jacobian operator .

Hence, from this is it natural to introduce the generalization of the normal derivative, \[
\partial _{n}^{j} v = \sum_{\abs{ \alpha  } =j }^{k} \frac{D ^{\alpha }v( x) n^{\alpha }}{\alpha !}, \quad \abs{ \alpha  } = \sum_{i=0}^{d} \alpha_{i}.
\]
An useful result that may help us design ghost penalty is the following estimate.

\begin{lemma}
    \label{lemma:bi_local_facet_estimate}
    Let $T_{1}, T_{2} \in  \mathcal{T} _{h} $ share a common facet $F \in \mathcal{F}_{h} $. Then for $v_{} \in  V_{h}$  does this hold \[
    \| v \|_{ T_{1} }^{  2}  \lesssim  \| v \|_{ T_{2} }^{2  }  + \sum_{j=0}^{k}  h^{2j +1} ( \jump{ \partial _{n}^{j} v}, \jump{ \partial _{n}^{j} v}    )_{F}
    \]
        Here is $k$ the polynomial degree.
\end{lemma}

\begin{proof}
    For a detailed proof, see \cite{gurkan2019stabilized}.
\end{proof}

We will now introduce the so-called ghost penalty faces, that is, \[
\mathcal{F} ^{g}_{h} = \left\{ F\in \mathcal{F} _{h} : T^{+}\cap \Gamma \neq \emptyset  \vee T^{-}\cap \Gamma \neq \emptyset  \right\}.
\]
This set is simply all facets that belong to all elements of the active mesh $\mathcal{T} _{h}$  intersected with $\Gamma $, i.e., all triangles to the cut cells $\mathcal{T} _{\Gamma }$. For an illustration, see Figure \ref{fig:illustration_F_g}.




\begin{proposition}
    \label{prop:hessian_change}
    The following identity holds for $j=0,1,2$.
    $$\partial ^{j}_{n} (D^2v) = D^2 ( \partial ^{j}_{n} v)  $$
\end{proposition}

\begin{proof}
        Recall that $\left[ D^2 v \right]_{i,j} = \partial _{x_{i}x_{j}} v $. We will compute each index individually.
    \begin{enumerate}[label=\arabic*)]
        \item $j = 0$ is trivially true.
        \item Let $j=1$. We can then easily see that, \[
        \partial ^{}_{n} ( \partial _{x_{i} x_{j}} v)  = \nabla  ( \partial _{x_{i} x_{j}} v)   n = \partial _{x_{i} x_{j}} (\nabla  v) n =\partial _{x_{i} x_{j}} (\partial _{n} v)
        \]
        Hence, $\partial _{n} (D^2v) = D^2( \partial _{n}v)$.
        \item Let $j=2$. Similarly can we see that, \[
                \begin{split}
                \partial^{2} _{n} (\partial _{x_{i} x_{j}} v) & = n^{T}  D^2(\partial _{x_{i} x_{j}} v) n = n^{T}  J( \nabla (\partial _{x_{i} x_{j}}v) ) n \\
                & =  n^{T}  J(\partial _{x_{i} x_{j}}(\nabla v) ) n = \partial _{x_{i} x_{j}} n^{T} J(\nabla v) n
                \end{split}
            \]
            Thus, taking account for all elements in the matrix we get $\partial^{2} _{n} (D^2v) = D^2( \partial^{2} _{n}v)$.
    \end{enumerate}
    Proof is complete.

\end{proof}


\begin{figure}
    \centering
    \begin{tikzpicture}
        \coordinate (center) at (0, 0);

        % Reference hexagon vertices
        \coordinate (A1) at (0:2.5);
        \coordinate (A2) at (60:2.5);
        \coordinate (A3) at (120:2.5);
        \coordinate (A4) at (180:2.5);
        \coordinate (A5) at (240:2.5);
        \coordinate (A6) at (300:2.5);
        \coordinate (A7) at (300:2.5);


        \fill[green!40] (center) --(A1) -- (A2) -- (A3) -- (A4)  -- cycle;
        \fill[blue!30] (center) -- (A4) --(A5)--(A6)-- (A7) -- (A1) -- cycle;

        % Draw the individual edges
        \draw[dotted, line width=1.5pt] (center) -- (A1);
        \draw[dotted, line width=1.5pt] (center) -- (A2);
        \draw[dotted, line width=1.5pt] (center) -- (A3) -- (A4) -- (center);
        \draw (A2) -- (A3);
        \draw (A4) -- (A5);
        \draw (center) -- (A5) -- (A6) ;
        \draw (center) -- (A6) -- (A1);


        \begin{scope}[shift={(2.5,0)}]
            % Shifted
            \coordinate (center) at (0, 0);

            % Reference hexagon vertices
            \coordinate (B1) at (0:2.5);
            \coordinate (B2) at (60:2.5);
            \coordinate (B3) at (120:2.5);
            \coordinate (B5) at (240:2.5);
            \coordinate (B6) at (300:2.5);

            \fill[green!40] (center) --(B6) -- (B1) -- (B2) --(B3) -- cycle;
            \fill[blue!30] (center) -- (B5) --(B6) -- cycle;


            % Draw the individual edges
            \draw[dotted, line width=1.5pt] (center) -- (B2);
            \draw[dotted, line width=1.5pt] (center) -- (B3); % double
            \draw[dotted, line width=1.5pt] (center) -- (B1);
            \draw (B1) -- (B2);
            \draw (B2) -- (B3);
            \draw (B5) -- (B6);
            \draw[dotted, line width=1.5pt] (center) -- (B6) -- (B1);
        \end{scope}


        \coordinate (Ti) at (-0,-1.0);
        \coordinate (Tg) at (3.0,2.0);
        \node[below] at (Tg) {$\mathcal{T}_{\Gamma }$};
        \node[below] at (Ti) {$\mathcal{T}_{int }$};

        \coordinate (C1) at (-3,0);
        \coordinate (C2) at (5.2,-0.5);
        \draw[-, line width=2pt, >=stealth] ($(C2)$) to[bend right] node[midway, yshift=-0.3cm] {$\Gamma $} ($(C1)$);

        \begin{scope}[shift={(6.0,1.7)}]
            \draw[dotted, line width=1.5pt] (-1.5,0) -- (-1,0);
            \node[anchor=west] at (-1.0,0) {$\mathcal{F}_h^g$};
        \end{scope}

        % Symbol visualization
        % \draw[dotted, line width=1.5pt] (3.4,3.1) -- (3.9,3.1);
        % \node[anchor=west] at (4.0,3.1) {$\mathcal{F}_h^g$};
        %% Legend
        \draw (4.4,1.3) rectangle (5.8,2.1); % Legend box

\end{tikzpicture}

\caption{Illustration of $\mathcal{F} _{h}^{g}$ denoted as the dotted lines. The set is defined as all facets which belongs to cut cells $\mathcal{T} _{\Gamma }$ sharing a node with interior elements $\mathcal{T} _{int }$ .  }
\label{fig:illustration_F_g}
\end{figure}





\begin{lemma}
    \label{lemma:bi_inv_gh_lemma}
    For $v \in  V_{h}$ it holds that
        \begin{align}
            \label{eq:bi_inv_gh_1}
        \| v \|_{ \mathcal{T} _{h} }^{ 2 }  & \lesssim  \| v \|_{ \Omega  }^{ 2 }  + \sum_{j=0}^{k} h^{2j+1} ( \jump{ \partial ^{j}_{n} v }, \jump{ \partial ^{j}_{n} v}    )_{\mathcal{F}_{h}^{g}}\\
            \label{eq:bi_inv_gh_2}
        \| D ^2 v \|_{ \mathcal{T} _{h} }^{ 2 }  & \lesssim  \| D^2 v \|_{ \Omega  }^{ 2 }  + \sum_{j=0}^{k} h^{2j-3} ( \jump{ \partial ^{j}_{n} v }, \jump{ \partial ^{j}_{n}v }    )_{\mathcal{F}_{h}^{g}}
        \end{align}
        Here is $k$ the polynomial degree.
\end{lemma}

\begin{proof}
    We will dive the proof in two parts where we first prove \eqref{eq:bi_inv_gh_1} and then in the second part prove \eqref{eq:bi_inv_gh_2}.
    \begin{enumerate}[label=\arabic*)]
        \item
            First of all, notice that there is a patch $P(T) $ consisting of $\left\{ T_{i} \right\}_{i=1}^{l} $ mesh elements s.t. each pair $ \left\{ T_{i}, T_{i+1} \right\} $ share a facet $F_{i}$ and the last element $T_{l}$ has a so-called "fat"
            intersection.

            Let us define the following norm \[
            g_{F_{i}}^{L^{2}}( v,v)  = \sum_{j=0}^{k} h^{2j+1}( \jump{ \partial ^{j}_{n}v }, \jump{ \partial ^{j}_{n}v }    )_{F_{i} }
            \]
            where $F_{i} \in  \mathcal{F} ^{g}_{h}$ and polynomial degree $ k$. Using Lemma \ref{lemma:bi_local_facet_estimate} can we see that \[
            \| v \|_{ T_{i} }^{  } \lesssim \| v \|_{ T_{i+1} }^{ 2 } + g_{F_{i}}^{L^{2}}( v,v).
            \]
    Consequently, using induction over each pair $\left\{ T_{i}, T_{i+1} \right\} $ with a corresponding $F_{i}$, we obtain
            \[
                \begin{split}
            \| v \|_{ T_{1} }^{2  }  & \le  C \| v \|_{ T_{2} }^{ 2 } + g_{F_{1}}^{L^{2}}( v,v)\\
              & \le  C( C( \| v \|_{ T_{3} }^{ 2 } + g_{F_{2}}^{L^{2}}( v,v) ) + g_{F_{1}}^{L^{2}}( v,v) )\\
              & \lesssim    \| v \|_{ T_{l} }^{ 2 }  + \sum_{i=1}^{l-1} g_{F_{i}}^{L^{2}}( v,v)  \\
              & \lesssim    \| v \|_{ T_{l} \cap \Omega  }^{ 2 }  + \sum_{i=1}^{l-1} g_{F_{i}}^{L^{2}}( v,v)
                \end{split}
            \]
            Here the last steps arise from the fact that $\|  v \|_{ T_{l} }^{  } \lesssim  \|  v \|_{ T_{l} \cap \Omega  }^{  }  $, which is a consequence of the fat intersection property.
Summation over the intersected triangles $\mathcal{T} _{\Gamma }$ implies,
            \[
                    \| v \|_{ \mathcal{T} _{\Gamma } }^{2  } \lesssim \| v \|_{ \mathcal{T} _{\Gamma}\cap \Omega  }^{2  }+ \sum_{i=1}^{l-1} g_{F_{l}}^{L^{2}}( v,v) \\
                     = \| v \|_{ \mathcal{T}_{\Gamma } \cap \Omega   }^{ 2 }  + \sum_{j=0}^{k} h^{2j+1} ( \jump{ \partial ^{j}_{n} v }, \jump{ \partial ^{j}_{n}v }    )_{\mathcal{F}_{h}^{g}}
        \]
        And as a trivial extension this now also holds for the active mesh $\mathcal{T} _{h}$ , that is, \[
                    \| v \|_{ \mathcal{T} _{h } }^{2  } \lesssim  \| v \|_{ \mathcal{T}_{h } \cap \Omega   }^{ 2 }  + \sum_{j=0}^{k} h^{2j+1} ( \jump{ \partial ^{j}_{n} v }, \jump{ \partial ^{j}_{n}v }    )_{\mathcal{F}_{h}^{g}}.
        \]
        Hence, \eqref{eq:bi_inv_gh_1} holds and the first part of the proof is complete.

    \item We will simply start by replacing $v$  by $D^2 v$ and use the Proposition \ref{prop:hessian_change}.
        \[
            \begin{split}
                    \| D^2 v \|_{ \mathcal{T} _{h} }^{2  }&  \lesssim \| D^2 v \|_{ \Omega  }^{ 2 }  + \sum_{j=0}^{k} h^{2j+1} ( \jump{   \partial ^{j}_{n} D^2 v }, \jump{  \partial ^{j}_{n} D^2 v}    )_{\mathcal{F}_{h}^{g}} \\
                    &=  \| D^2 v \|_{ \Omega  }^{ 2 }  + \sum_{j=0}^{k} h^{2j+1} ( \jump{   D^2 \partial ^{j}_{n}  v }, \jump{  D^2 \partial ^{j}_{n}  v}    )_{\mathcal{F}_{h}^{g}}
            \end{split}
        \]
        Remark that $\|  D^2 v \|_{ T_{l} }^{  } \lesssim  \|  D^2 v \|_{ T_{l} \cap \Omega  }^{  }  $ using the fat-intersection property.
        Thus, it remains to show that \[
        \sum_{j=0}^{k} h^{2j+1} ( \jump{   D^2 \partial ^{j}_{n}  v }, \jump{  D^2 \partial ^{j}_{n}  v}    )_{\mathcal{F}_{h}^{g}} \lesssim  \sum_{j=0}^{k} h^{2j-3} ( \jump{    \partial ^{j}_{n}  v }, \jump{  \partial ^{j}_{n}  v}
        )_{\mathcal{F}_{h}^{g}}.
        \]
        Recall the decomposition procedure done in \eqref{eq:projection} for a facet $F$ , where $P_{F} := I - n_{} \oplus n_{} $ and $Q_{F} = n_{} \oplus n_{}$. We apply this so we can to decompose the Hessian evaluated on the facet  s.t. \[
        D^2 v  \mid _{F} = Q_{F}D^2v + P_{F} D^2 v .
        \]
        % Recall from basic finite element theory .
        \red{
        Recall from \eqref{eq:fund_inv_est} and \eqref{eq:degrade} the inverse estimates $\| D^2v \|_{T  }^{ 2 }  \le h^{-4} \| v \|_{ T }^{2  }    $ and $ \| D^2v \|_{F  }^{2  } \le h^{-1} \| D^2 v  \|_{T  }^{ 2 }   $.}
        That is, applying the decomposition we get the following estimates \[
            \begin{split}
        \| \jump{ P_{F}   D^2 \partial _{n}^{j} v }\|_{ F }^{ 2 } & = \| P_{F} D^2 \jump{ \partial _{n}^{j} v }   \|_{ F  }^{ 2} = \| (I- n\otimes n) D^2 \jump{ \partial _{n}^{j} v }   \|_{ F  }^{ 2}    \\
        & \lesssim  \|  (I- n\otimes n) \|_{l^{\infty} }^{  } \| D^2 \jump{ \partial _{n}^{j} v }   \|_{ F  }^{ 2} \lesssim   \|  D^2 \jump{ \partial _{n}^{j} v }   \|_{ F  }^{ 2} \lesssim h^{-4} \|  \jump{ \partial ^{j}_{n} v }   \|_{ F }^{  2}. \\
        \| \jump{ Q_{F}   D^2 \partial _{n}^{j} v }\|_{ F }^{ 2 } & = \| Q_{F} D^2 \jump{ \partial _{n}^{j} v }   \|_{ F  }^{ 2} = \| n\otimes n D^2 \jump{ \partial _{n}^{j} v }   \|_{ F  }^{ 2}    \\
        & \lesssim  \| n \otimes n \|_{l^{\infty} }^{  }  \|  D^2 \jump{ \partial _{n}^{j} v }   \|_{ F  }^{ 2} \lesssim h^{-4} \|  \jump{ \partial ^{j}_{n} v }   \|_{ F }^{  2}.
            \end{split}
        \]
        Here is $\| \cdot  \|_{l^{\infty}  }^{  } $ denoted as the discrete matrix inf-norm.
        Thus, using this on the decomposition can we admit that
         \[
             \begin{split}
            h^{2j +1} \| \jump{D^2 \partial ^{j}_{n}  v}  \|_{ F }^{2  } & \lesssim h^{2j +1} \| Q_{F} \jump{ \partial ^{j}_{n}  v }  \|_{ F }^{2  } + h^{2j +1} \| P_{F} D^2 \jump{ \partial ^{j}_{n}  v }       \|_{ F }^{2  } \\
            &  \lesssim   h^{2j -3}  \| \jump{ \partial ^{j}_{n}v }   \|_{ F }^{ 2 }
             \end{split}
        \]
        % \red{Seems like the projection argument is not necessarry. You can simply prove it using direcetly the estimate $\| D^2v \|_{F  }^{ 2 }\le h^{-2} \| \nabla v \|_{ F }^{ 2 } \le h^{-4} \| v \|_{ F }^{2  }    $ s.t.
        % $$
        % \| D^2 \jump{ \partial _{n}^{j} v } \|_{F  }^{ 2 } \le h^{-4} \| \jump{ \partial _{n}^{j} v } \|_{ F }^{2  }
    % $$ }

        Thus, fulfilling \eqref{eq:bi_inv_gh_2}.
        \[
            \begin{split}
         \sum_{j=0}^{k} h^{2j+1} ( \jump{   D^2 \partial ^{j}_{n}  v }, \jump{  D^2 \partial ^{j}_{n}  v}    )_{\mathcal{F}_{h}^{g}} & \lesssim  \sum_{j=0}^{k} h^{2j-3}  (\jump{    \partial ^{j}_{n}  v }, \jump{ \partial ^{j}_{n}  v }  )_{\mathcal{F} _{h}^{g}  }^{2  }
            \end{split}
        \]

    \end{enumerate}
    Hence, the proof is complete.

\end{proof}



Finally, we now have the tools we need to construct an candidate for the ghost penalty for which satisfies all assumptions.

\begin{proposition}[Face-based ghost penalty]
    Let $k\ge  2$ be the order of the polynomial basis in $V_{h}$ .
    For any set of positive parameters $\left\{ \gamma _{j} \right\} _{j=0}^{k}$, the ghost penalty defined as \[
    g^{}_{h}( v,w)  := \sum_{j=1}^{k} \sum_{F \in \mathcal{F} _{h}^{g}}^{} \gamma _{j} h^{2j-3}_{F} ( \jump{ \partial ^{j}_{n} v }, \jump{ \partial ^{j}_{n} w }  ) _{F} \text{ for any } v,w \in V_{h},
    \]
    satisfies the Assumption \ref{as:EP1} and \ref{as:EP2}.
\end{proposition}

\begin{proof}
    First of all, from Lemma \ref{lemma:bi_inv_gh_lemma} is it clear that \(
    \| D^2v \|_{ \mathcal{T} _{h} }^{  }  \lesssim \| D^2 v \|_{ \Omega   }^{  } + \abs{ v } _{g_{h}}.
    \)
    Thus, the Assumption \ref{as:EP1} holds and it only remains to check Assumption \ref{as:EP2}, that is, $ \abs{ C^{e}_{h} v }_{g_{h}} \lesssim h^{r-1} \| v \|_{r, \Omega   }^{  }$.
    Let $v  \in H^{s}( \Omega ) $, $s\ge 3$,  and $r = \min\{s,k-1\} $
    From definition is \[
        \begin{split}
        \abs{ C^{e}_{h} v }_{g_{h}}^{2} & = \sum_{j=1}^{k}  \gamma _{j} h^{2j-3} ( \jump{ \partial ^{j}_{n} v }, \jump{ \partial ^{j}_{n} v }  ) _{\mathcal{F}_{h}^{g} } \\
& = \sum_{j=1}^{r-1}  \gamma _{j} h^{2j-3} ( \jump{ \partial ^{j}_{n} v }, \jump{ \partial ^{j}_{n} v }  ) _{\mathcal{F}_{h}^{g} } + \sum_{j=r}^{k}  \gamma _{j} h^{2j-3} ( \jump{ \partial ^{j}_{n} v }, \jump{ \partial ^{j}_{n} v }  ) _{\mathcal{F}_{h}^{g} } \\
&= \sum_{j=r}^{k}  \gamma _{j} h^{2j-3} ( \jump{ \partial ^{j}_{n} v }, \jump{ \partial ^{j}_{n} v }  ) _{\mathcal{F}_{h}^{g} } \\
        \end{split}
    \]

    where the jump vanishes for the for the first $r-1$ terms, that is,  $\jump{ \partial ^{j}_{n} v } = 0  \  \forall s \le r-1$.





\end{proof}

    % % 
\subsection{Results}%
\label{sub:results}

\todo[inline]{Move numerical results/experiments into separate chapter}


Since we have in the analysis assumed that $\Gamma $ is $C^2 $ will we run our test problems on a circle with radius $r=1$ with order $k=2$.

\subsubsection{Test Problem 1}%
\label{ssub:test_problem_1}


We introduce the following manufactured solution $u_{ex}(x,y) = (x^2 + y^2 - 1)^2 \sin(2\pi x) \cos(2\pi y)$. To compute the corresponding functions $f, g_{1}$ and $g_{2}$ did we use algorithmic differentiation. The method was implemented in Julia
using the FEM package Gridap \cite{verdugo22,julia17}. You can see the corresponding results in Figure \ref{fig:CutFEM_error1} and Table \ref{table:CutFEM_error1}.


\begin{table}[h!]
    \caption{Convergence analysis of the numerical method}
    \label{table:CutFEM_error1}
    \begin{tabular}{rrrrrrrrr}
    \hline\hline
    \textbf{$n$} & \textbf{$\Vert e \Vert_{L^2}$} & \textbf{EOC} & \textbf{$ \Vert e \Vert_{H^1}$} & \textbf{EOC} & \textbf{$\Vert e \Vert_{ a_h,* }$} & \textbf{EOC} & \textbf{Cond number} & \textbf{ndofs} \\\hline
    4 & 4.0E-01 &  & 3.6E+00 &  & 3.6E+01 &  & 1.1E+06 & 8.1E+01 \\
    8 & 1.7E-01 & 1.27 & 1.3E+00 & 1.43 & 2.0E+01 & 0.89 & 3.0E+06 & 2.4E+02 \\
    16 & 2.4E-02 & 2.77 & 2.9E-01 & 2.19 & 9.2E+00 & 1.09 & 1.2E+08 & 8.3E+02 \\
    32 & 6.0E-03 & 2.01 & 7.2E-02 & 2.04 & 4.5E+00 & 1.02 & 3.2E+08 & 3.0E+03 \\
    64 & 2.1E-03 & 1.54 & 1.8E-02 & 1.98 & 2.3E+00 & 1.00 & 4.2E+10 & 1.1E+04 \\
    128 & 5.2E-04 & 2.00 & 4.5E-03 & 2.01 & 1.1E+00 & 1.00 & 4.4E+11 & 4.3E+04 \\
    256 & 8.8E-05 & 2.57 & 1.1E-03 & 2.02 & 5.7E-01 & 1.00 & 2.5E+18 & 1.7E+05 \\\hline\hline
  \end{tabular}
\end{table}


\begin{figure}[h!]
    \centering
    % Recommended preamble:
% \usetikzlibrary{arrows.meta}
% \usetikzlibrary{backgrounds}
% \usepgfplotslibrary{patchplots}
% \usepgfplotslibrary{fillbetween}
% \pgfplotsset{%
%     layers/standard/.define layer set={%
%         background,axis background,axis grid,axis ticks,axis lines,axis tick labels,pre main,main,axis descriptions,axis foreground%
%     }{
%         grid style={/pgfplots/on layer=axis grid},%
%         tick style={/pgfplots/on layer=axis ticks},%
%         axis line style={/pgfplots/on layer=axis lines},%
%         label style={/pgfplots/on layer=axis descriptions},%
%         legend style={/pgfplots/on layer=axis descriptions},%
%         title style={/pgfplots/on layer=axis descriptions},%
%         colorbar style={/pgfplots/on layer=axis descriptions},%
%         ticklabel style={/pgfplots/on layer=axis tick labels},%
%         axis background@ style={/pgfplots/on layer=axis background},%
%         3d box foreground style={/pgfplots/on layer=axis foreground},%
%     },
% }

\begin{tikzpicture}[/tikz/background rectangle/.style={fill={rgb,1:red,1.0;green,1.0;blue,1.0}, fill opacity={1.0}, draw opacity={1.0}}, show background rectangle]
\begin{axis}[point meta max={nan}, point meta min={nan}, legend cell align={left}, legend columns={1}, title={}, title style={at={{(0.5,1)}}, anchor={south}, font={{\fontsize{14 pt}{18.2 pt}\selectfont}}, color={rgb,1:red,0.0;green,0.0;blue,0.0}, draw opacity={1.0}, rotate={0.0}, align={center}}, legend style={color={rgb,1:red,0.0;green,0.0;blue,0.0}, draw opacity={1.0}, line width={1}, solid, fill={rgb,1:red,1.0;green,1.0;blue,1.0}, fill opacity={1.0}, text opacity={1.0}, font={{\fontsize{14 pt}{18.2 pt}\selectfont}}, text={rgb,1:red,0.0;green,0.0;blue,0.0}, cells={anchor={center}}, at={(0.98, 0.02)}, anchor={south east}}, axis background/.style={fill={rgb,1:red,1.0;green,1.0;blue,1.0}, opacity={1.0}}, anchor={north west}, xshift={1.0mm}, yshift={-1.0mm}, width={150.4mm}, height={99.6mm}, scaled x ticks={false}, xlabel={h}, x tick style={color={rgb,1:red,0.0;green,0.0;blue,0.0}, opacity={1.0}}, x tick label style={color={rgb,1:red,0.0;green,0.0;blue,0.0}, opacity={1.0}, rotate={0}}, xlabel style={at={(ticklabel cs:0.5)}, anchor=near ticklabel, at={{(ticklabel cs:0.5)}}, anchor={near ticklabel}, font={{\fontsize{11 pt}{14.3 pt}\selectfont}}, color={rgb,1:red,0.0;green,0.0;blue,0.0}, draw opacity={1.0}, rotate={0.0}}, xmode={log}, log basis x={2}, xmajorgrids={true}, xmin={0.0034480585792603714}, xmax={0.2832209713239497}, xticklabels={{$2^{-6}$,$2^{-3}$}}, xtick={{0.015625,0.125}}, xtick align={inside}, xticklabel style={font={{\fontsize{8 pt}{10.4 pt}\selectfont}}, color={rgb,1:red,0.0;green,0.0;blue,0.0}, draw opacity={1.0}, rotate={0.0}}, x grid style={color={rgb,1:red,0.0;green,0.0;blue,0.0}, draw opacity={0.1}, line width={0.5}, solid}, extra x ticks={{0.009375,0.0109375,0.0125,0.0140625,0.015625,0.015625,0.01875,0.021875,0.025,0.028125,0.03125,0.03125,0.0375,0.04375,0.05,0.05625,0.01875,0.021875,0.025,0.028125,0.03125,0.03125,0.0375,0.04375,0.05,0.05625,0.0625,0.0625,0.075,0.0875,0.1,0.1125,0.15,0.175,0.2,0.225,0.25,0.25}}, extra x tick labels={}, extra x tick style={grid={major}, x grid style={color={rgb,1:red,0.0;green,0.0;blue,0.0}, draw opacity={0.05}, line width={0.5}, solid}, major tick length={0.1cm}}, axis x line*={left}, x axis line style={color={rgb,1:red,0.0;green,0.0;blue,0.0}, draw opacity={1.0}, line width={1}, solid}, scaled y ticks={false}, ylabel={}, y tick style={color={rgb,1:red,0.0;green,0.0;blue,0.0}, opacity={1.0}}, y tick label style={color={rgb,1:red,0.0;green,0.0;blue,0.0}, opacity={1.0}, rotate={0}}, ylabel style={at={(ticklabel cs:0.5)}, anchor=near ticklabel, at={{(ticklabel cs:0.5)}}, anchor={near ticklabel}, font={{\fontsize{11 pt}{14.3 pt}\selectfont}}, color={rgb,1:red,0.0;green,0.0;blue,0.0}, draw opacity={1.0}, rotate={0.0}}, ymode={log}, log basis y={2}, ymajorgrids={true}, ymin={5.946451976698402e-5}, ymax={53.612500463633474}, yticklabels={{$2^{-8}$,$2^{0}$}}, ytick={{0.00390625,1.0}}, ytick align={inside}, yticklabel style={font={{\fontsize{8 pt}{10.4 pt}\selectfont}}, color={rgb,1:red,0.0;green,0.0;blue,0.0}, draw opacity={1.0}, rotate={0.0}}, y grid style={color={rgb,1:red,0.0;green,0.0;blue,0.0}, draw opacity={0.1}, line width={0.5}, solid}, extra y ticks={{0.00234375,0.002734375,0.003125,0.003515625,0.00390625,0.00390625,0.0046875,0.00546875,0.00625,0.00703125,0.0078125,0.0078125,0.009375,0.0109375,0.0125,0.0140625,0.015625,0.015625,0.01875,0.021875,0.025,0.028125,0.03125,0.03125,0.0375,0.04375,0.05,0.05625,0.0625,0.0625,0.075,0.0875,0.1,0.1125,0.125,0.125,0.15,0.175,0.2,0.225,0.25,0.25,0.3,0.35,0.4,0.45,0.0046875,0.00546875,0.00625,0.00703125,0.0078125,0.0078125,0.009375,0.0109375,0.0125,0.0140625,0.015625,0.015625,0.01875,0.021875,0.025,0.028125,0.03125,0.03125,0.0375,0.04375,0.05,0.05625,0.0625,0.0625,0.075,0.0875,0.1,0.1125,0.125,0.125,0.15,0.175,0.2,0.225,0.25,0.25,0.3,0.35,0.4,0.45,0.5,0.5,0.6,0.7,0.8,0.9,1.2,1.4,1.6,1.8,2.0,2.0,2.4,2.8,3.2,3.6,4.0,4.0,4.8,5.6,6.4,7.2,8.0,8.0,9.6,11.2,12.8,14.4,16.0,16.0,19.2,22.4,25.6,28.8,32.0,32.0,38.4,44.8,51.2}}, extra y tick labels={}, extra y tick style={grid={major}, y grid style={color={rgb,1:red,0.0;green,0.0;blue,0.0}, draw opacity={0.05}, line width={0.5}, solid}, major tick length={0.1cm}}, axis y line*={left}, y axis line style={color={rgb,1:red,0.0;green,0.0;blue,0.0}, draw opacity={1.0}, line width={1}, solid}, colorbar={false}]
    \addplot[color={rgb,1:red,0.0;green,0.6056;blue,0.9787}, name path={15c93f99-2ad7-4f7c-be7b-bf680797e714}, draw opacity={1.0}, line width={1}, solid]
        table[row sep={\\}]
        {
            \\
            0.25  0.3996521510998364  \\
            0.125  0.16517715848302073  \\
            0.0625  0.0242737770545615  \\
            0.03125  0.006008832719834156  \\
            0.015625  0.00207187118447813  \\
            0.0078125  0.0005196665181438592  \\
            0.00390625  8.76588729067292e-5  \\
        }
        ;
    \addlegendentry {$\Vert e \Vert_{L^2}$}
    \addplot[color={rgb,1:red,0.0;green,0.6056;blue,0.9787}, name path={b2f0eb05-4d85-4092-a283-10fd5ff7f9f5}, only marks, draw opacity={1.0}, line width={0}, solid, mark={*}, mark size={3.0 pt}, mark repeat={1}, mark options={color={rgb,1:red,0.0;green,0.0;blue,0.0}, draw opacity={1.0}, fill={rgb,1:red,0.0;green,0.6056;blue,0.9787}, fill opacity={1.0}, line width={0.75}, rotate={0}, solid}, forget plot]
        table[row sep={\\}]
        {
            \\
            0.25  0.3996521510998364  \\
            0.125  0.16517715848302073  \\
            0.0625  0.0242737770545615  \\
            0.03125  0.006008832719834156  \\
            0.015625  0.00207187118447813  \\
            0.0078125  0.0005196665181438592  \\
            0.00390625  8.76588729067292e-5  \\
        }
        ;
    \addplot[color={rgb,1:red,0.8889;green,0.4356;blue,0.2781}, name path={493c36fb-1c2a-457c-aa00-29d587b33ee5}, draw opacity={1.0}, line width={1}, solid]
        table[row sep={\\}]
        {
            \\
            0.25  3.6204094462218896  \\
            0.125  1.345186069843398  \\
            0.0625  0.29478878904950073  \\
            0.03125  0.07170478762130418  \\
            0.015625  0.018236405835013764  \\
            0.0078125  0.004529417406616602  \\
            0.00390625  0.0011160186204957883  \\
        }
        ;
    \addlegendentry {$\Vert e \Vert_{H^1}$}
    \addplot[color={rgb,1:red,0.8889;green,0.4356;blue,0.2781}, name path={59039888-40e4-4457-aa21-dcdd061a1121}, only marks, draw opacity={1.0}, line width={0}, solid, mark={*}, mark size={3.0 pt}, mark repeat={1}, mark options={color={rgb,1:red,0.0;green,0.0;blue,0.0}, draw opacity={1.0}, fill={rgb,1:red,0.8889;green,0.4356;blue,0.2781}, fill opacity={1.0}, line width={0.75}, rotate={0}, solid}, forget plot]
        table[row sep={\\}]
        {
            \\
            0.25  3.6204094462218896  \\
            0.125  1.345186069843398  \\
            0.0625  0.29478878904950073  \\
            0.03125  0.07170478762130418  \\
            0.015625  0.018236405835013764  \\
            0.0078125  0.004529417406616602  \\
            0.00390625  0.0011160186204957883  \\
        }
        ;
    \addplot[color={rgb,1:red,0.2422;green,0.6433;blue,0.3044}, name path={bdb9dc23-0d16-4d25-bc47-79e7af7c204b}, draw opacity={1.0}, line width={1}, solid]
        table[row sep={\\}]
        {
            \\
            0.25  36.368726722842005  \\
            0.125  19.57282507000903  \\
            0.0625  9.20543801021562  \\
            0.03125  4.549118581752081  \\
            0.015625  2.2712752068995794  \\
            0.0078125  1.1353039284956243  \\
            0.00390625  0.5675569226015551  \\
        }
        ;
    \addlegendentry {$\Vert e \Vert_{a_{h,*}}$}
    \addplot[color={rgb,1:red,0.2422;green,0.6433;blue,0.3044}, name path={c73cf9ff-8332-4228-a897-081aabbb6b7b}, only marks, draw opacity={1.0}, line width={0}, solid, mark={*}, mark size={3.0 pt}, mark repeat={1}, mark options={color={rgb,1:red,0.0;green,0.0;blue,0.0}, draw opacity={1.0}, fill={rgb,1:red,0.2422;green,0.6433;blue,0.3044}, fill opacity={1.0}, line width={0.75}, rotate={0}, solid}, forget plot]
        table[row sep={\\}]
        {
            \\
            0.25  36.368726722842005  \\
            0.125  19.57282507000903  \\
            0.0625  9.20543801021562  \\
            0.03125  4.549118581752081  \\
            0.015625  2.2712752068995794  \\
            0.0078125  1.1353039284956243  \\
            0.00390625  0.5675569226015551  \\
        }
        ;
\end{axis}
\end{tikzpicture}

    \caption{The plot presents the $L^2$ and $H^1$ error norms and the error in the energy norm ($\Vert e \Vert_{a_h,*}$).}
    \label{fig:CutFEM_error1}
\end{figure}


    % 
\newpage
\section{Numerical results}%
\label{sec:numerical_results}


\subsection{Numerical Results for CutCIP Biharmonic Equation  }%
\label{sub:numerical_results_for_cutcip_biharmonic_equation_}

\begin{itemize}
    \item EOC Test on circle and its configuration
        \begin{enumerate}[label=\arabic*)]
            \item Penalty parameters $\gamma , \gamma _{g1}, \gamma _{g2} = 20,10, 0.1$
            \item Background mesh $L = 3.11$ and circle $R=1$, flower $r0, r1 = L0.3, L0.1$
        \end{enumerate}
    \item Translation
        \begin{enumerate}[label=\arabic*)]
            \item Background mesh $L = 3.11$ and $R=1$
        \end{enumerate}
\end{itemize}


In this section will we provide a numerical study of the methods studied in Section \ref{sec:cutcip_biharmonic_problem}.

\subsubsection{EOC test}%
\label{ssub:eoc_test}

Here we consider the manufactured solution $l,r,m = (2, 1, 1) $ s.t.
\[
u_{\text{ex}}(x,y) = (x^2- y^2 -1) \sin\left(\frac{2\pi m}{l}x_1\right)\cos\left(\frac{2\pi r}{l}y\right)
\]
On a square background mesh $L=3.11$ with a circle domain of radius $R=1$.

\begin{table}
  \begin{tabular}{rrrrrrrrrrr}
    \noalign{\hrule height 2pt}
    \textbf{$h/L$} & \textbf{$n$} & \textbf{$\Vert e \Vert_{L^2}$} & \textbf{EOC} & \textbf{$ \Vert e \Vert_{H^1}$} & \textbf{EOC} & \textbf{$\Vert e \Vert_{ a_h,* }$} & \textbf{EOC} & \textbf{$\kappa_{\infty}(A)$} & \textbf{EOC} & \textbf{ndofs} \\\noalign{\hrule height 2pt}
    1/8 & 8 & 1.8E-01 & NaN & 1.9E+00 & NaN & 2.6E+01 & NaN & 1.9E+06 & NaN & 1.7E+02 \\
    1/16 & 16 & 4.2E-02 & 2.11 & 4.8E-01 & 1.97 & 1.2E+01 & 1.13 & 2.5E+07 & -3.70 & 5.8E+02 \\
    1/32 & 32 & 9.9E-03 & 2.08 & 1.2E-01 & 2.02 & 5.7E+00 & 1.06 & 4.0E+08 & -4.01 & 2.0E+03 \\
    1/64 & 64 & 2.3E-03 & 2.13 & 2.9E-02 & 2.03 & 2.8E+00 & 1.02 & 6.1E+09 & -3.96 & 7.6E+03 \\
    1/128 & 128 & 5.4E-04 & 2.06 & 6.9E-03 & 2.06 & 1.4E+00 & 1.01 & 9.5E+10 & -3.95 & 2.9E+04 \\
    1/256 & 256 & 1.3E-04 & 2.03 & 1.7E-03 & 2.03 & 6.9E-01 & 1.01 & 1.5E+12 & -3.97 & 1.2E+05 \\
    1/512 & 512 & 3.3E-05 & 2.02 & 4.2E-04 & 2.02 & 3.5E-01 & 1.00 & 2.4E+13 & -3.99 & 4.6E+05 \\
    1/1024 & 1024 & 1.4E-05 & 1.22 & 1.1E-04 & 1.95 & 1.7E-01 & 1.00 & 3.7E+14 & -3.98 & 1.8E+06 \\\noalign{\hrule height 2pt}
  \end{tabular}


\caption{Convergence rates for the Hessian-based method applied to the circular domain with side length $L=2.7$, using parameters $\gamma=20$, $\gamma_1=10$, and $\gamma_2=0.1$}
\end{table}
\begin{figure}
% Recommended preamble:
% \usetikzlibrary{arrows.meta}
% \usetikzlibrary{backgrounds}
% \usepgfplotslibrary{patchplots}
% \usepgfplotslibrary{fillbetween}
% \pgfplotsset{%
%     layers/standard/.define layer set={%
%         background,axis background,axis grid,axis ticks,axis lines,axis tick labels,pre main,main,axis descriptions,axis foreground%
%     }{
%         grid style={/pgfplots/on layer=axis grid},%
%         tick style={/pgfplots/on layer=axis ticks},%
%         axis line style={/pgfplots/on layer=axis lines},%
%         label style={/pgfplots/on layer=axis descriptions},%
%         legend style={/pgfplots/on layer=axis descriptions},%
%         title style={/pgfplots/on layer=axis descriptions},%
%         colorbar style={/pgfplots/on layer=axis descriptions},%
%         ticklabel style={/pgfplots/on layer=axis tick labels},%
%         axis background@ style={/pgfplots/on layer=axis background},%
%         3d box foreground style={/pgfplots/on layer=axis foreground},%
%     },
% }

\begin{tikzpicture}[/tikz/background rectangle/.style={fill={rgb,1:red,1.0;green,1.0;blue,1.0}, fill opacity={1.0}, draw opacity={1.0}}, show background rectangle]
\begin{axis}[point meta max={nan}, point meta min={nan}, legend cell align={left}, legend columns={1}, title={}, title style={at={{(0.5,1)}}, anchor={south}, font={{\fontsize{14 pt}{18.2 pt}\selectfont}}, color={rgb,1:red,0.0;green,0.0;blue,0.0}, draw opacity={1.0}, rotate={0.0}, align={center}}, legend style={color={rgb,1:red,0.0;green,0.0;blue,0.0}, draw opacity={1.0}, line width={1}, solid, fill={rgb,1:red,1.0;green,1.0;blue,1.0}, fill opacity={1.0}, text opacity={1.0}, font={{\fontsize{14 pt}{18.2 pt}\selectfont}}, text={rgb,1:red,0.0;green,0.0;blue,0.0}, cells={anchor={center}}, at={(1.02, 1)}, anchor={north west}}, axis background/.style={fill={rgb,1:red,1.0;green,1.0;blue,1.0}, opacity={1.0}}, anchor={north west}, xshift={1.0mm}, yshift={-1.0mm}, width={120.0mm}, height={74.2mm}, scaled x ticks={false}, xlabel={h}, x tick style={color={rgb,1:red,0.0;green,0.0;blue,0.0}, opacity={1.0}}, x tick label style={color={rgb,1:red,0.0;green,0.0;blue,0.0}, opacity={1.0}, rotate={0}}, xlabel style={at={(ticklabel cs:0.5)}, anchor=near ticklabel, at={{(ticklabel cs:0.5)}}, anchor={near ticklabel}, font={{\fontsize{11 pt}{14.3 pt}\selectfont}}, color={rgb,1:red,0.0;green,0.0;blue,0.0}, draw opacity={1.0}, rotate={0.0}}, xmode={log}, log basis x={2}, xmajorgrids={true}, xmin={0.0017240292896301857}, xmax={0.14161048566197484}, xticklabels={{$2^{-7.5}$,$2^{-5.0}$}}, xtick={{0.005524271728019903,0.03125}}, xtick align={inside}, xticklabel style={font={{\fontsize{8 pt}{10.4 pt}\selectfont}}, color={rgb,1:red,0.0;green,0.0;blue,0.0}, draw opacity={1.0}, rotate={0.0}}, x grid style={color={rgb,1:red,0.0;green,0.0;blue,0.0}, draw opacity={0.1}, line width={0.5}, solid}, extra x ticks={{0.003314563036811942,0.003866990209613932,0.004419417382415922,0.004971844555217913,0.005524271728019903,0.005524271728019903,0.006629126073623884,0.007733980419227864,0.008838834764831844,0.009943689110435826,0.006629126073623884,0.007733980419227864,0.008838834764831844,0.009943689110435826,0.011048543456039806,0.011048543456039806,0.013258252147247768,0.015467960838455728,0.017677669529663688,0.01988737822087165,0.0375,0.04375,0.05,0.05625,0.0625,0.0625,0.075,0.0875,0.1,0.1125}}, extra x tick labels={}, extra x tick style={grid={major}, x grid style={color={rgb,1:red,0.0;green,0.0;blue,0.0}, draw opacity={0.05}, line width={0.5}, solid}, major tick length={0.1cm}}, axis x line*={left}, x axis line style={color={rgb,1:red,0.0;green,0.0;blue,0.0}, draw opacity={1.0}, line width={1}, solid}, scaled y ticks={false}, ylabel={$\Vert e \Vert_{}$}, y tick style={color={rgb,1:red,0.0;green,0.0;blue,0.0}, opacity={1.0}}, y tick label style={color={rgb,1:red,0.0;green,0.0;blue,0.0}, opacity={1.0}, rotate={0}}, ylabel style={at={(ticklabel cs:0.5)}, anchor=near ticklabel, at={{(ticklabel cs:0.5)}}, anchor={near ticklabel}, font={{\fontsize{11 pt}{14.3 pt}\selectfont}}, color={rgb,1:red,0.0;green,0.0;blue,0.0}, draw opacity={1.0}, rotate={0.0}}, ymode={log}, log basis y={2}, ymajorgrids={true}, ymin={1.3862504738420095e-5}, ymax={10.93956609845569}, yticklabels={{$2^{-10}$,$2^{0}$}}, ytick={{0.0009765625,1.0}}, ytick align={inside}, yticklabel style={font={{\fontsize{8 pt}{10.4 pt}\selectfont}}, color={rgb,1:red,0.0;green,0.0;blue,0.0}, draw opacity={1.0}, rotate={0.0}}, y grid style={color={rgb,1:red,0.0;green,0.0;blue,0.0}, draw opacity={0.1}, line width={0.5}, solid}, extra y ticks={{0.0005859375,0.00068359375,0.00078125,0.00087890625,0.0009765625,0.0009765625,0.001171875,0.0013671875,0.0015625,0.0017578125,0.001953125,0.001953125,0.00234375,0.002734375,0.003125,0.003515625,0.00390625,0.00390625,0.0046875,0.00546875,0.00625,0.00703125,0.0078125,0.0078125,0.009375,0.0109375,0.0125,0.0140625,0.015625,0.015625,0.01875,0.021875,0.025,0.028125,0.03125,0.03125,0.0375,0.04375,0.05,0.05625,0.0625,0.0625,0.075,0.0875,0.1,0.1125,0.125,0.125,0.15,0.175,0.2,0.225,0.25,0.25,0.3,0.35,0.4,0.45,0.001171875,0.0013671875,0.0015625,0.0017578125,0.001953125,0.001953125,0.00234375,0.002734375,0.003125,0.003515625,0.00390625,0.00390625,0.0046875,0.00546875,0.00625,0.00703125,0.0078125,0.0078125,0.009375,0.0109375,0.0125,0.0140625,0.015625,0.015625,0.01875,0.021875,0.025,0.028125,0.03125,0.03125,0.0375,0.04375,0.05,0.05625,0.0625,0.0625,0.075,0.0875,0.1,0.1125,0.125,0.125,0.15,0.175,0.2,0.225,0.25,0.25,0.3,0.35,0.4,0.45,0.5,0.5,0.6,0.7,0.8,0.9,1.2,1.4,1.6,1.8,2.0,2.0,2.4,2.8,3.2,3.6,4.0,4.0,4.8,5.6,6.4,7.2,8.0,8.0,9.6}}, extra y tick labels={}, extra y tick style={grid={major}, y grid style={color={rgb,1:red,0.0;green,0.0;blue,0.0}, draw opacity={0.05}, line width={0.5}, solid}, major tick length={0.1cm}}, axis y line*={left}, y axis line style={color={rgb,1:red,0.0;green,0.0;blue,0.0}, draw opacity={1.0}, line width={1}, solid}, colorbar={false}]
    \addplot[color={rgb,1:red,0.0;green,0.6056;blue,0.9787}, name path={9218d1bc-144b-41cc-9d0d-cc01b677eba6}, draw opacity={1.0}, line width={1}, solid]
        table[row sep={\\}]
        {
            \\
            0.125  0.12485282013359456  \\
            0.0625  0.021519566026825893  \\
            0.03125  0.006108410278138697  \\
            0.015625  0.0013316391108902747  \\
            0.0078125  0.0002951129467558774  \\
            0.00390625  6.955804210218113e-5  \\
            0.001953125  2.0358342512059304e-5  \\
        }
        ;
    \addlegendentry {$\Vert e \Vert_{L^2}$}
    \addplot[color={rgb,1:red,0.0;green,0.6056;blue,0.9787}, name path={50ec2bef-2fa7-4f27-9c87-20e33eda3668}, only marks, draw opacity={1.0}, line width={0}, solid, mark={*}, mark size={3.0 pt}, mark repeat={1}, mark options={color={rgb,1:red,0.0;green,0.0;blue,0.0}, draw opacity={1.0}, fill={rgb,1:red,0.0;green,0.6056;blue,0.9787}, fill opacity={1.0}, line width={0.75}, rotate={0}, solid}, forget plot]
        table[row sep={\\}]
        {
            \\
            0.125  0.12485282013359456  \\
            0.0625  0.021519566026825893  \\
            0.03125  0.006108410278138697  \\
            0.015625  0.0013316391108902747  \\
            0.0078125  0.0002951129467558774  \\
            0.00390625  6.955804210218113e-5  \\
            0.001953125  2.0358342512059304e-5  \\
        }
        ;
    \addplot[color={rgb,1:red,0.8889;green,0.4356;blue,0.2781}, name path={e771d2fe-4eaf-429c-9680-a071abe9483b}, draw opacity={1.0}, line width={1}, solid]
        table[row sep={\\}]
        {
            \\
            0.125  0.7621211610548966  \\
            0.0625  0.18586984920759084  \\
            0.03125  0.05567729731816812  \\
            0.015625  0.01259148044540304  \\
            0.0078125  0.002695581170454147  \\
            0.00390625  0.0006099332861921217  \\
            0.001953125  0.0001433590482844144  \\
        }
        ;
    \addlegendentry {$\Vert e \Vert_{H^1}$}
    \addplot[color={rgb,1:red,0.8889;green,0.4356;blue,0.2781}, name path={415f23ed-cbb6-40a9-9ba1-dccf4160f6b5}, only marks, draw opacity={1.0}, line width={0}, solid, mark={*}, mark size={3.0 pt}, mark repeat={1}, mark options={color={rgb,1:red,0.0;green,0.0;blue,0.0}, draw opacity={1.0}, fill={rgb,1:red,0.8889;green,0.4356;blue,0.2781}, fill opacity={1.0}, line width={0.75}, rotate={0}, solid}, forget plot]
        table[row sep={\\}]
        {
            \\
            0.125  0.7621211610548966  \\
            0.0625  0.18586984920759084  \\
            0.03125  0.05567729731816812  \\
            0.015625  0.01259148044540304  \\
            0.0078125  0.002695581170454147  \\
            0.00390625  0.0006099332861921217  \\
            0.001953125  0.0001433590482844144  \\
        }
        ;
    \addplot[color={rgb,1:red,0.2422;green,0.6433;blue,0.3044}, name path={b462b987-2b48-4523-a6aa-ce2520e22a2f}, draw opacity={1.0}, line width={1}, solid]
        table[row sep={\\}]
        {
            \\
            0.125  7.449024240862031  \\
            0.0625  3.775104225092707  \\
            0.03125  1.8456105206495865  \\
            0.015625  0.8627174111673411  \\
            0.0078125  0.4082356919935078  \\
            0.00390625  0.1985035554213105  \\
            0.001953125  0.09770054621981272  \\
        }
        ;
    \addlegendentry {$\Vert e \Vert_{a_{h,*}}$}
    \addplot[color={rgb,1:red,0.2422;green,0.6433;blue,0.3044}, name path={fcc71a29-4ba9-4cf1-b6a6-18abab25ad7d}, only marks, draw opacity={1.0}, line width={0}, solid, mark={*}, mark size={3.0 pt}, mark repeat={1}, mark options={color={rgb,1:red,0.0;green,0.0;blue,0.0}, draw opacity={1.0}, fill={rgb,1:red,0.2422;green,0.6433;blue,0.3044}, fill opacity={1.0}, line width={0.75}, rotate={0}, solid}, forget plot]
        table[row sep={\\}]
        {
            \\
            0.125  7.449024240862031  \\
            0.0625  3.775104225092707  \\
            0.03125  1.8456105206495865  \\
            0.015625  0.8627174111673411  \\
            0.0078125  0.4082356919935078  \\
            0.00390625  0.1985035554213105  \\
            0.001953125  0.09770054621981272  \\
        }
        ;
\end{axis}
\end{tikzpicture}

\caption{Convergence rates for the Hessian-based method applied to the circular domain with side length $L=2.7$, using parameters $\gamma=20$, $\gamma_1=10$, and $\gamma_2=0.1$}
\end{figure}

\begin{table}
  \begin{tabular}{rrrrrrrrrrr}
    \noalign{\hrule height 2pt}
    \textbf{$h/L$} & \textbf{$n$} & \textbf{$\Vert e \Vert_{L^2}$} & \textbf{EOC} & \textbf{$ \Vert e \Vert_{H^1}$} & \textbf{EOC} & \textbf{$\Vert e \Vert_{ a_h,* }$} & \textbf{EOC} & \textbf{$\kappa_{\infty}(A)$} & \textbf{EOC} & \textbf{ndofs} \\\noalign{\hrule height 2pt}
    1/8 & 8 & 1.8E-01 & NaN & 1.9E+00 & NaN & 2.6E+01 & NaN & 1.9E+06 & NaN & 1.7E+02 \\
    1/16 & 16 & 4.2E-02 & 2.11 & 4.8E-01 & 1.97 & 1.2E+01 & 1.13 & 2.5E+07 & -3.70 & 5.8E+02 \\
    1/32 & 32 & 9.9E-03 & 2.08 & 1.2E-01 & 2.02 & 5.7E+00 & 1.06 & 4.0E+08 & -4.01 & 2.0E+03 \\
    1/64 & 64 & 2.3E-03 & 2.13 & 2.9E-02 & 2.03 & 2.8E+00 & 1.02 & 6.1E+09 & -3.96 & 7.6E+03 \\
    1/128 & 128 & 5.4E-04 & 2.06 & 6.9E-03 & 2.06 & 1.4E+00 & 1.01 & 9.5E+10 & -3.95 & 2.9E+04 \\
    1/256 & 256 & 1.3E-04 & 2.03 & 1.7E-03 & 2.03 & 6.9E-01 & 1.01 & 1.5E+12 & -3.97 & 1.2E+05 \\
    1/512 & 512 & 3.3E-05 & 2.02 & 4.2E-04 & 2.02 & 3.5E-01 & 1.00 & 2.4E+13 & -3.99 & 4.6E+05 \\
    1/1024 & 1024 & 1.4E-05 & 1.22 & 1.1E-04 & 1.95 & 1.7E-01 & 1.00 & 3.7E+14 & -3.98 & 1.8E+06 \\\noalign{\hrule height 2pt}
  \end{tabular}

\caption{Convergence rates for the Laplacian-based method applied to the circular domain with side length $L=2.7$, using parameters $\gamma=20$, $\gamma_1=10$, and $\gamma_2=0.1$}
\end{table}
\begin{figure}
% Recommended preamble:
% \usetikzlibrary{arrows.meta}
% \usetikzlibrary{backgrounds}
% \usepgfplotslibrary{patchplots}
% \usepgfplotslibrary{fillbetween}
% \pgfplotsset{%
%     layers/standard/.define layer set={%
%         background,axis background,axis grid,axis ticks,axis lines,axis tick labels,pre main,main,axis descriptions,axis foreground%
%     }{
%         grid style={/pgfplots/on layer=axis grid},%
%         tick style={/pgfplots/on layer=axis ticks},%
%         axis line style={/pgfplots/on layer=axis lines},%
%         label style={/pgfplots/on layer=axis descriptions},%
%         legend style={/pgfplots/on layer=axis descriptions},%
%         title style={/pgfplots/on layer=axis descriptions},%
%         colorbar style={/pgfplots/on layer=axis descriptions},%
%         ticklabel style={/pgfplots/on layer=axis tick labels},%
%         axis background@ style={/pgfplots/on layer=axis background},%
%         3d box foreground style={/pgfplots/on layer=axis foreground},%
%     },
% }

\begin{tikzpicture}[/tikz/background rectangle/.style={fill={rgb,1:red,1.0;green,1.0;blue,1.0}, fill opacity={1.0}, draw opacity={1.0}}, show background rectangle]
\begin{axis}[point meta max={nan}, point meta min={nan}, legend cell align={left}, legend columns={1}, title={}, title style={at={{(0.5,1)}}, anchor={south}, font={{\fontsize{14 pt}{18.2 pt}\selectfont}}, color={rgb,1:red,0.0;green,0.0;blue,0.0}, draw opacity={1.0}, rotate={0.0}, align={center}}, legend style={color={rgb,1:red,0.0;green,0.0;blue,0.0}, draw opacity={1.0}, line width={1}, solid, fill={rgb,1:red,1.0;green,1.0;blue,1.0}, fill opacity={1.0}, text opacity={1.0}, font={{\fontsize{14 pt}{18.2 pt}\selectfont}}, text={rgb,1:red,0.0;green,0.0;blue,0.0}, cells={anchor={center}}, at={(1.02, 1)}, anchor={north west}}, axis background/.style={fill={rgb,1:red,1.0;green,1.0;blue,1.0}, opacity={1.0}}, anchor={north west}, xshift={1.0mm}, yshift={-1.0mm}, width={120.0mm}, height={74.2mm}, scaled x ticks={false}, xlabel={h}, x tick style={color={rgb,1:red,0.0;green,0.0;blue,0.0}, opacity={1.0}}, x tick label style={color={rgb,1:red,0.0;green,0.0;blue,0.0}, opacity={1.0}, rotate={0}}, xlabel style={at={(ticklabel cs:0.5)}, anchor=near ticklabel, at={{(ticklabel cs:0.5)}}, anchor={near ticklabel}, font={{\fontsize{11 pt}{14.3 pt}\selectfont}}, color={rgb,1:red,0.0;green,0.0;blue,0.0}, draw opacity={1.0}, rotate={0.0}}, xmode={log}, log basis x={2}, xmajorgrids={true}, xmin={0.0017240292896301857}, xmax={0.14161048566197484}, xticklabels={{$2^{-7.5}$,$2^{-5.0}$}}, xtick={{0.005524271728019903,0.03125}}, xtick align={inside}, xticklabel style={font={{\fontsize{8 pt}{10.4 pt}\selectfont}}, color={rgb,1:red,0.0;green,0.0;blue,0.0}, draw opacity={1.0}, rotate={0.0}}, x grid style={color={rgb,1:red,0.0;green,0.0;blue,0.0}, draw opacity={0.1}, line width={0.5}, solid}, extra x ticks={{0.003314563036811942,0.003866990209613932,0.004419417382415922,0.004971844555217913,0.005524271728019903,0.005524271728019903,0.006629126073623884,0.007733980419227864,0.008838834764831844,0.009943689110435826,0.006629126073623884,0.007733980419227864,0.008838834764831844,0.009943689110435826,0.011048543456039806,0.011048543456039806,0.013258252147247768,0.015467960838455728,0.017677669529663688,0.01988737822087165,0.0375,0.04375,0.05,0.05625,0.0625,0.0625,0.075,0.0875,0.1,0.1125}}, extra x tick labels={}, extra x tick style={grid={major}, x grid style={color={rgb,1:red,0.0;green,0.0;blue,0.0}, draw opacity={0.05}, line width={0.5}, solid}, major tick length={0.1cm}}, axis x line*={left}, x axis line style={color={rgb,1:red,0.0;green,0.0;blue,0.0}, draw opacity={1.0}, line width={1}, solid}, scaled y ticks={false}, ylabel={$\Vert e \Vert_{}$}, y tick style={color={rgb,1:red,0.0;green,0.0;blue,0.0}, opacity={1.0}}, y tick label style={color={rgb,1:red,0.0;green,0.0;blue,0.0}, opacity={1.0}, rotate={0}}, ylabel style={at={(ticklabel cs:0.5)}, anchor=near ticklabel, at={{(ticklabel cs:0.5)}}, anchor={near ticklabel}, font={{\fontsize{11 pt}{14.3 pt}\selectfont}}, color={rgb,1:red,0.0;green,0.0;blue,0.0}, draw opacity={1.0}, rotate={0.0}}, ymode={log}, log basis y={2}, ymajorgrids={true}, ymin={1.3862504738420095e-5}, ymax={10.93956609845569}, yticklabels={{$2^{-10}$,$2^{0}$}}, ytick={{0.0009765625,1.0}}, ytick align={inside}, yticklabel style={font={{\fontsize{8 pt}{10.4 pt}\selectfont}}, color={rgb,1:red,0.0;green,0.0;blue,0.0}, draw opacity={1.0}, rotate={0.0}}, y grid style={color={rgb,1:red,0.0;green,0.0;blue,0.0}, draw opacity={0.1}, line width={0.5}, solid}, extra y ticks={{0.0005859375,0.00068359375,0.00078125,0.00087890625,0.0009765625,0.0009765625,0.001171875,0.0013671875,0.0015625,0.0017578125,0.001953125,0.001953125,0.00234375,0.002734375,0.003125,0.003515625,0.00390625,0.00390625,0.0046875,0.00546875,0.00625,0.00703125,0.0078125,0.0078125,0.009375,0.0109375,0.0125,0.0140625,0.015625,0.015625,0.01875,0.021875,0.025,0.028125,0.03125,0.03125,0.0375,0.04375,0.05,0.05625,0.0625,0.0625,0.075,0.0875,0.1,0.1125,0.125,0.125,0.15,0.175,0.2,0.225,0.25,0.25,0.3,0.35,0.4,0.45,0.001171875,0.0013671875,0.0015625,0.0017578125,0.001953125,0.001953125,0.00234375,0.002734375,0.003125,0.003515625,0.00390625,0.00390625,0.0046875,0.00546875,0.00625,0.00703125,0.0078125,0.0078125,0.009375,0.0109375,0.0125,0.0140625,0.015625,0.015625,0.01875,0.021875,0.025,0.028125,0.03125,0.03125,0.0375,0.04375,0.05,0.05625,0.0625,0.0625,0.075,0.0875,0.1,0.1125,0.125,0.125,0.15,0.175,0.2,0.225,0.25,0.25,0.3,0.35,0.4,0.45,0.5,0.5,0.6,0.7,0.8,0.9,1.2,1.4,1.6,1.8,2.0,2.0,2.4,2.8,3.2,3.6,4.0,4.0,4.8,5.6,6.4,7.2,8.0,8.0,9.6}}, extra y tick labels={}, extra y tick style={grid={major}, y grid style={color={rgb,1:red,0.0;green,0.0;blue,0.0}, draw opacity={0.05}, line width={0.5}, solid}, major tick length={0.1cm}}, axis y line*={left}, y axis line style={color={rgb,1:red,0.0;green,0.0;blue,0.0}, draw opacity={1.0}, line width={1}, solid}, colorbar={false}]
    \addplot[color={rgb,1:red,0.0;green,0.6056;blue,0.9787}, name path={9218d1bc-144b-41cc-9d0d-cc01b677eba6}, draw opacity={1.0}, line width={1}, solid]
        table[row sep={\\}]
        {
            \\
            0.125  0.12485282013359456  \\
            0.0625  0.021519566026825893  \\
            0.03125  0.006108410278138697  \\
            0.015625  0.0013316391108902747  \\
            0.0078125  0.0002951129467558774  \\
            0.00390625  6.955804210218113e-5  \\
            0.001953125  2.0358342512059304e-5  \\
        }
        ;
    \addlegendentry {$\Vert e \Vert_{L^2}$}
    \addplot[color={rgb,1:red,0.0;green,0.6056;blue,0.9787}, name path={50ec2bef-2fa7-4f27-9c87-20e33eda3668}, only marks, draw opacity={1.0}, line width={0}, solid, mark={*}, mark size={3.0 pt}, mark repeat={1}, mark options={color={rgb,1:red,0.0;green,0.0;blue,0.0}, draw opacity={1.0}, fill={rgb,1:red,0.0;green,0.6056;blue,0.9787}, fill opacity={1.0}, line width={0.75}, rotate={0}, solid}, forget plot]
        table[row sep={\\}]
        {
            \\
            0.125  0.12485282013359456  \\
            0.0625  0.021519566026825893  \\
            0.03125  0.006108410278138697  \\
            0.015625  0.0013316391108902747  \\
            0.0078125  0.0002951129467558774  \\
            0.00390625  6.955804210218113e-5  \\
            0.001953125  2.0358342512059304e-5  \\
        }
        ;
    \addplot[color={rgb,1:red,0.8889;green,0.4356;blue,0.2781}, name path={e771d2fe-4eaf-429c-9680-a071abe9483b}, draw opacity={1.0}, line width={1}, solid]
        table[row sep={\\}]
        {
            \\
            0.125  0.7621211610548966  \\
            0.0625  0.18586984920759084  \\
            0.03125  0.05567729731816812  \\
            0.015625  0.01259148044540304  \\
            0.0078125  0.002695581170454147  \\
            0.00390625  0.0006099332861921217  \\
            0.001953125  0.0001433590482844144  \\
        }
        ;
    \addlegendentry {$\Vert e \Vert_{H^1}$}
    \addplot[color={rgb,1:red,0.8889;green,0.4356;blue,0.2781}, name path={415f23ed-cbb6-40a9-9ba1-dccf4160f6b5}, only marks, draw opacity={1.0}, line width={0}, solid, mark={*}, mark size={3.0 pt}, mark repeat={1}, mark options={color={rgb,1:red,0.0;green,0.0;blue,0.0}, draw opacity={1.0}, fill={rgb,1:red,0.8889;green,0.4356;blue,0.2781}, fill opacity={1.0}, line width={0.75}, rotate={0}, solid}, forget plot]
        table[row sep={\\}]
        {
            \\
            0.125  0.7621211610548966  \\
            0.0625  0.18586984920759084  \\
            0.03125  0.05567729731816812  \\
            0.015625  0.01259148044540304  \\
            0.0078125  0.002695581170454147  \\
            0.00390625  0.0006099332861921217  \\
            0.001953125  0.0001433590482844144  \\
        }
        ;
    \addplot[color={rgb,1:red,0.2422;green,0.6433;blue,0.3044}, name path={b462b987-2b48-4523-a6aa-ce2520e22a2f}, draw opacity={1.0}, line width={1}, solid]
        table[row sep={\\}]
        {
            \\
            0.125  7.449024240862031  \\
            0.0625  3.775104225092707  \\
            0.03125  1.8456105206495865  \\
            0.015625  0.8627174111673411  \\
            0.0078125  0.4082356919935078  \\
            0.00390625  0.1985035554213105  \\
            0.001953125  0.09770054621981272  \\
        }
        ;
    \addlegendentry {$\Vert e \Vert_{a_{h,*}}$}
    \addplot[color={rgb,1:red,0.2422;green,0.6433;blue,0.3044}, name path={fcc71a29-4ba9-4cf1-b6a6-18abab25ad7d}, only marks, draw opacity={1.0}, line width={0}, solid, mark={*}, mark size={3.0 pt}, mark repeat={1}, mark options={color={rgb,1:red,0.0;green,0.0;blue,0.0}, draw opacity={1.0}, fill={rgb,1:red,0.2422;green,0.6433;blue,0.3044}, fill opacity={1.0}, line width={0.75}, rotate={0}, solid}, forget plot]
        table[row sep={\\}]
        {
            \\
            0.125  7.449024240862031  \\
            0.0625  3.775104225092707  \\
            0.03125  1.8456105206495865  \\
            0.015625  0.8627174111673411  \\
            0.0078125  0.4082356919935078  \\
            0.00390625  0.1985035554213105  \\
            0.001953125  0.09770054621981272  \\
        }
        ;
\end{axis}
\end{tikzpicture}

\caption{Convergence rates for the Laplacian-based method applied to the circular domain with side length $L=2.7$, using parameters $\gamma=20$, $\gamma_1=10$, and $\gamma_2=0.1$}
\end{figure}



\begin{table}
  \begin{tabular}{rrrrrrrrrrr}
    \noalign{\hrule height 2pt}
    \textbf{$h/L$} & \textbf{$n$} & \textbf{$\Vert e \Vert_{L^2}$} & \textbf{EOC} & \textbf{$ \Vert e \Vert_{H^1}$} & \textbf{EOC} & \textbf{$\Vert e \Vert_{ a_h,* }$} & \textbf{EOC} & \textbf{$\kappa_{\infty}(A)$} & \textbf{EOC} & \textbf{ndofs} \\\noalign{\hrule height 2pt}
    1/8 & 8 & 1.8E-01 & NaN & 1.9E+00 & NaN & 2.6E+01 & NaN & 1.9E+06 & NaN & 1.7E+02 \\
    1/16 & 16 & 4.2E-02 & 2.11 & 4.8E-01 & 1.97 & 1.2E+01 & 1.13 & 2.5E+07 & -3.70 & 5.8E+02 \\
    1/32 & 32 & 9.9E-03 & 2.08 & 1.2E-01 & 2.02 & 5.7E+00 & 1.06 & 4.0E+08 & -4.01 & 2.0E+03 \\
    1/64 & 64 & 2.3E-03 & 2.13 & 2.9E-02 & 2.03 & 2.8E+00 & 1.02 & 6.1E+09 & -3.96 & 7.6E+03 \\
    1/128 & 128 & 5.4E-04 & 2.06 & 6.9E-03 & 2.06 & 1.4E+00 & 1.01 & 9.5E+10 & -3.95 & 2.9E+04 \\
    1/256 & 256 & 1.3E-04 & 2.03 & 1.7E-03 & 2.03 & 6.9E-01 & 1.01 & 1.5E+12 & -3.97 & 1.2E+05 \\
    1/512 & 512 & 3.3E-05 & 2.02 & 4.2E-04 & 2.02 & 3.5E-01 & 1.00 & 2.4E+13 & -3.99 & 4.6E+05 \\
    1/1024 & 1024 & 1.4E-05 & 1.22 & 1.1E-04 & 1.95 & 1.7E-01 & 1.00 & 3.7E+14 & -3.98 & 1.8E+06 \\\noalign{\hrule height 2pt}
  \end{tabular}

\caption{Convergence rates for the Laplacian-based method applied to the Flower domain with side length $L=2.7$, using parameters $\gamma=20$, $\gamma_1=10$, and $\gamma_2=1$}
\end{table}
\begin{figure}
% Recommended preamble:
% \usetikzlibrary{arrows.meta}
% \usetikzlibrary{backgrounds}
% \usepgfplotslibrary{patchplots}
% \usepgfplotslibrary{fillbetween}
% \pgfplotsset{%
%     layers/standard/.define layer set={%
%         background,axis background,axis grid,axis ticks,axis lines,axis tick labels,pre main,main,axis descriptions,axis foreground%
%     }{
%         grid style={/pgfplots/on layer=axis grid},%
%         tick style={/pgfplots/on layer=axis ticks},%
%         axis line style={/pgfplots/on layer=axis lines},%
%         label style={/pgfplots/on layer=axis descriptions},%
%         legend style={/pgfplots/on layer=axis descriptions},%
%         title style={/pgfplots/on layer=axis descriptions},%
%         colorbar style={/pgfplots/on layer=axis descriptions},%
%         ticklabel style={/pgfplots/on layer=axis tick labels},%
%         axis background@ style={/pgfplots/on layer=axis background},%
%         3d box foreground style={/pgfplots/on layer=axis foreground},%
%     },
% }

\begin{tikzpicture}[/tikz/background rectangle/.style={fill={rgb,1:red,1.0;green,1.0;blue,1.0}, fill opacity={1.0}, draw opacity={1.0}}, show background rectangle]
\begin{axis}[point meta max={nan}, point meta min={nan}, legend cell align={left}, legend columns={1}, title={}, title style={at={{(0.5,1)}}, anchor={south}, font={{\fontsize{14 pt}{18.2 pt}\selectfont}}, color={rgb,1:red,0.0;green,0.0;blue,0.0}, draw opacity={1.0}, rotate={0.0}, align={center}}, legend style={color={rgb,1:red,0.0;green,0.0;blue,0.0}, draw opacity={1.0}, line width={1}, solid, fill={rgb,1:red,1.0;green,1.0;blue,1.0}, fill opacity={1.0}, text opacity={1.0}, font={{\fontsize{14 pt}{18.2 pt}\selectfont}}, text={rgb,1:red,0.0;green,0.0;blue,0.0}, cells={anchor={center}}, at={(1.02, 1)}, anchor={north west}}, axis background/.style={fill={rgb,1:red,1.0;green,1.0;blue,1.0}, opacity={1.0}}, anchor={north west}, xshift={1.0mm}, yshift={-1.0mm}, width={120.0mm}, height={74.2mm}, scaled x ticks={false}, xlabel={h}, x tick style={color={rgb,1:red,0.0;green,0.0;blue,0.0}, opacity={1.0}}, x tick label style={color={rgb,1:red,0.0;green,0.0;blue,0.0}, opacity={1.0}, rotate={0}}, xlabel style={at={(ticklabel cs:0.5)}, anchor=near ticklabel, at={{(ticklabel cs:0.5)}}, anchor={near ticklabel}, font={{\fontsize{11 pt}{14.3 pt}\selectfont}}, color={rgb,1:red,0.0;green,0.0;blue,0.0}, draw opacity={1.0}, rotate={0.0}}, xmode={log}, log basis x={2}, xmajorgrids={true}, xmin={0.0017240292896301857}, xmax={0.14161048566197484}, xticklabels={{$2^{-7.5}$,$2^{-5.0}$}}, xtick={{0.005524271728019903,0.03125}}, xtick align={inside}, xticklabel style={font={{\fontsize{8 pt}{10.4 pt}\selectfont}}, color={rgb,1:red,0.0;green,0.0;blue,0.0}, draw opacity={1.0}, rotate={0.0}}, x grid style={color={rgb,1:red,0.0;green,0.0;blue,0.0}, draw opacity={0.1}, line width={0.5}, solid}, extra x ticks={{0.003314563036811942,0.003866990209613932,0.004419417382415922,0.004971844555217913,0.005524271728019903,0.005524271728019903,0.006629126073623884,0.007733980419227864,0.008838834764831844,0.009943689110435826,0.006629126073623884,0.007733980419227864,0.008838834764831844,0.009943689110435826,0.011048543456039806,0.011048543456039806,0.013258252147247768,0.015467960838455728,0.017677669529663688,0.01988737822087165,0.0375,0.04375,0.05,0.05625,0.0625,0.0625,0.075,0.0875,0.1,0.1125}}, extra x tick labels={}, extra x tick style={grid={major}, x grid style={color={rgb,1:red,0.0;green,0.0;blue,0.0}, draw opacity={0.05}, line width={0.5}, solid}, major tick length={0.1cm}}, axis x line*={left}, x axis line style={color={rgb,1:red,0.0;green,0.0;blue,0.0}, draw opacity={1.0}, line width={1}, solid}, scaled y ticks={false}, ylabel={$\Vert e \Vert_{}$}, y tick style={color={rgb,1:red,0.0;green,0.0;blue,0.0}, opacity={1.0}}, y tick label style={color={rgb,1:red,0.0;green,0.0;blue,0.0}, opacity={1.0}, rotate={0}}, ylabel style={at={(ticklabel cs:0.5)}, anchor=near ticklabel, at={{(ticklabel cs:0.5)}}, anchor={near ticklabel}, font={{\fontsize{11 pt}{14.3 pt}\selectfont}}, color={rgb,1:red,0.0;green,0.0;blue,0.0}, draw opacity={1.0}, rotate={0.0}}, ymode={log}, log basis y={2}, ymajorgrids={true}, ymin={1.3862504738420095e-5}, ymax={10.93956609845569}, yticklabels={{$2^{-10}$,$2^{0}$}}, ytick={{0.0009765625,1.0}}, ytick align={inside}, yticklabel style={font={{\fontsize{8 pt}{10.4 pt}\selectfont}}, color={rgb,1:red,0.0;green,0.0;blue,0.0}, draw opacity={1.0}, rotate={0.0}}, y grid style={color={rgb,1:red,0.0;green,0.0;blue,0.0}, draw opacity={0.1}, line width={0.5}, solid}, extra y ticks={{0.0005859375,0.00068359375,0.00078125,0.00087890625,0.0009765625,0.0009765625,0.001171875,0.0013671875,0.0015625,0.0017578125,0.001953125,0.001953125,0.00234375,0.002734375,0.003125,0.003515625,0.00390625,0.00390625,0.0046875,0.00546875,0.00625,0.00703125,0.0078125,0.0078125,0.009375,0.0109375,0.0125,0.0140625,0.015625,0.015625,0.01875,0.021875,0.025,0.028125,0.03125,0.03125,0.0375,0.04375,0.05,0.05625,0.0625,0.0625,0.075,0.0875,0.1,0.1125,0.125,0.125,0.15,0.175,0.2,0.225,0.25,0.25,0.3,0.35,0.4,0.45,0.001171875,0.0013671875,0.0015625,0.0017578125,0.001953125,0.001953125,0.00234375,0.002734375,0.003125,0.003515625,0.00390625,0.00390625,0.0046875,0.00546875,0.00625,0.00703125,0.0078125,0.0078125,0.009375,0.0109375,0.0125,0.0140625,0.015625,0.015625,0.01875,0.021875,0.025,0.028125,0.03125,0.03125,0.0375,0.04375,0.05,0.05625,0.0625,0.0625,0.075,0.0875,0.1,0.1125,0.125,0.125,0.15,0.175,0.2,0.225,0.25,0.25,0.3,0.35,0.4,0.45,0.5,0.5,0.6,0.7,0.8,0.9,1.2,1.4,1.6,1.8,2.0,2.0,2.4,2.8,3.2,3.6,4.0,4.0,4.8,5.6,6.4,7.2,8.0,8.0,9.6}}, extra y tick labels={}, extra y tick style={grid={major}, y grid style={color={rgb,1:red,0.0;green,0.0;blue,0.0}, draw opacity={0.05}, line width={0.5}, solid}, major tick length={0.1cm}}, axis y line*={left}, y axis line style={color={rgb,1:red,0.0;green,0.0;blue,0.0}, draw opacity={1.0}, line width={1}, solid}, colorbar={false}]
    \addplot[color={rgb,1:red,0.0;green,0.6056;blue,0.9787}, name path={9218d1bc-144b-41cc-9d0d-cc01b677eba6}, draw opacity={1.0}, line width={1}, solid]
        table[row sep={\\}]
        {
            \\
            0.125  0.12485282013359456  \\
            0.0625  0.021519566026825893  \\
            0.03125  0.006108410278138697  \\
            0.015625  0.0013316391108902747  \\
            0.0078125  0.0002951129467558774  \\
            0.00390625  6.955804210218113e-5  \\
            0.001953125  2.0358342512059304e-5  \\
        }
        ;
    \addlegendentry {$\Vert e \Vert_{L^2}$}
    \addplot[color={rgb,1:red,0.0;green,0.6056;blue,0.9787}, name path={50ec2bef-2fa7-4f27-9c87-20e33eda3668}, only marks, draw opacity={1.0}, line width={0}, solid, mark={*}, mark size={3.0 pt}, mark repeat={1}, mark options={color={rgb,1:red,0.0;green,0.0;blue,0.0}, draw opacity={1.0}, fill={rgb,1:red,0.0;green,0.6056;blue,0.9787}, fill opacity={1.0}, line width={0.75}, rotate={0}, solid}, forget plot]
        table[row sep={\\}]
        {
            \\
            0.125  0.12485282013359456  \\
            0.0625  0.021519566026825893  \\
            0.03125  0.006108410278138697  \\
            0.015625  0.0013316391108902747  \\
            0.0078125  0.0002951129467558774  \\
            0.00390625  6.955804210218113e-5  \\
            0.001953125  2.0358342512059304e-5  \\
        }
        ;
    \addplot[color={rgb,1:red,0.8889;green,0.4356;blue,0.2781}, name path={e771d2fe-4eaf-429c-9680-a071abe9483b}, draw opacity={1.0}, line width={1}, solid]
        table[row sep={\\}]
        {
            \\
            0.125  0.7621211610548966  \\
            0.0625  0.18586984920759084  \\
            0.03125  0.05567729731816812  \\
            0.015625  0.01259148044540304  \\
            0.0078125  0.002695581170454147  \\
            0.00390625  0.0006099332861921217  \\
            0.001953125  0.0001433590482844144  \\
        }
        ;
    \addlegendentry {$\Vert e \Vert_{H^1}$}
    \addplot[color={rgb,1:red,0.8889;green,0.4356;blue,0.2781}, name path={415f23ed-cbb6-40a9-9ba1-dccf4160f6b5}, only marks, draw opacity={1.0}, line width={0}, solid, mark={*}, mark size={3.0 pt}, mark repeat={1}, mark options={color={rgb,1:red,0.0;green,0.0;blue,0.0}, draw opacity={1.0}, fill={rgb,1:red,0.8889;green,0.4356;blue,0.2781}, fill opacity={1.0}, line width={0.75}, rotate={0}, solid}, forget plot]
        table[row sep={\\}]
        {
            \\
            0.125  0.7621211610548966  \\
            0.0625  0.18586984920759084  \\
            0.03125  0.05567729731816812  \\
            0.015625  0.01259148044540304  \\
            0.0078125  0.002695581170454147  \\
            0.00390625  0.0006099332861921217  \\
            0.001953125  0.0001433590482844144  \\
        }
        ;
    \addplot[color={rgb,1:red,0.2422;green,0.6433;blue,0.3044}, name path={b462b987-2b48-4523-a6aa-ce2520e22a2f}, draw opacity={1.0}, line width={1}, solid]
        table[row sep={\\}]
        {
            \\
            0.125  7.449024240862031  \\
            0.0625  3.775104225092707  \\
            0.03125  1.8456105206495865  \\
            0.015625  0.8627174111673411  \\
            0.0078125  0.4082356919935078  \\
            0.00390625  0.1985035554213105  \\
            0.001953125  0.09770054621981272  \\
        }
        ;
    \addlegendentry {$\Vert e \Vert_{a_{h,*}}$}
    \addplot[color={rgb,1:red,0.2422;green,0.6433;blue,0.3044}, name path={fcc71a29-4ba9-4cf1-b6a6-18abab25ad7d}, only marks, draw opacity={1.0}, line width={0}, solid, mark={*}, mark size={3.0 pt}, mark repeat={1}, mark options={color={rgb,1:red,0.0;green,0.0;blue,0.0}, draw opacity={1.0}, fill={rgb,1:red,0.2422;green,0.6433;blue,0.3044}, fill opacity={1.0}, line width={0.75}, rotate={0}, solid}, forget plot]
        table[row sep={\\}]
        {
            \\
            0.125  7.449024240862031  \\
            0.0625  3.775104225092707  \\
            0.03125  1.8456105206495865  \\
            0.015625  0.8627174111673411  \\
            0.0078125  0.4082356919935078  \\
            0.00390625  0.1985035554213105  \\
            0.001953125  0.09770054621981272  \\
        }
        ;
\end{axis}
\end{tikzpicture}

\caption{Convergence rates for the Laplacian-based method applied to the Flower domain with side length $L=2.7$, using parameters $\gamma=20$, $\gamma_1=10$, and $\gamma_2=1$}
\end{figure}

\begin{figure}\centering
\subfloat[]{\label{sub:fig:refine_a}
        \begin{tikzpicture}[scale=0.9]

            % Domain is blue
            \fill[blue!30] (0.1, 0.1) circle (1.5cm);
            \draw[black] (0.1, 0.1) circle (1.5cm);
            \node at (0.2, 0.2) {$\Omega$};

            % Background mesh
            \foreach \i in {-2.5, -2, ..., 2.5} {
                \draw[line width=0.1pt, shift={(-2.5,\i)}, opacity=0.2] (0,0) -- (5,0);
                \draw[line width=0.1pt, shift={(\i,-2.5)},opacity=0.2] (0,0) -- (0,5);
            }

        \end{tikzpicture}

}\hfill
\subfloat[]{\label{sub:fig:refine_b}
        \begin{tikzpicture}[scale=0.9]

            % Domain is blue
            \fill[blue!30] (0.1, 0.1) circle (1.5cm);
            \draw[black] (0.1, 0.1) circle (1.5cm);
            \node at (0.2, 0.2) {$\Omega$};

            % Background mesh
            \foreach \i in {-2.5, -2.25, ..., 2.5} {
                \draw[line width=0.1pt, shift={(-2.5,\i)}, opacity=0.2] (0,0) -- (5,0);
                \draw[line width=0.1pt, shift={(\i,-2.5)},opacity=0.2] (0,0) -- (0,5);
            }

        \end{tikzpicture}


}
\hfill
\subfloat[]{\label{sub:fig:refine_c}
        \begin{tikzpicture}[scale=0.9]

            % Domain is blue
            \fill[blue!30] (0.1, 0.1) circle (1.5cm);
            \draw[black] (0.1, 0.1) circle (1.5cm);
            \node at (0.2, 0.2) {$\Omega$};

            % Background mesh
            \foreach \i in {-2.5, -2.375, ..., 2.5} {
                \draw[line width=0.1pt, shift={(-2.5,\i)}, opacity=0.2] (0,0) -- (5,0);
                \draw[line width=0.1pt, shift={(\i,-2.5)},opacity=0.2] (0,0) -- (0,5);
            }

        \end{tikzpicture}
}
    \caption{Illustration of the domain $\Omega $ defined as a circle with radius $R$. The background mesh is a square domain with side lengths $L$ with three refinements of the mesh size $h$.}
\end{figure}


Here we consider the manufactured solution $l,r,m = (2, 1, 1) $ s.t.
\[
u_{\text{ex}}(x,y) = \sin\left(\frac{2\pi m}{l}x_1\right)\cos\left(\frac{2\pi r}{l}y\right)
\]
on the flower domain for the Laplace equation. Here is the flower domain described as
 Recall the polar coordinates $r = \sqrt{x^2 + y^2}$ and $\theta = atan( x,y) $. Then we define the flower level set function $\phi$ to be on the form
 \begin{equation}
 \label{eq:flower}
\phi(x, y) = r_0 + r_1  cos(5 \theta) - r = 0
 \end{equation}
Here we define the parameters $r_0= 0.3L$ and $ r_1=0.1L$.
\begin{figure}
    \centering
\begin{tikzpicture}
  \def\rZero{0.6} % Define r_0
  \def\rOne{0.2} % Define r_1
  \begin{axis}[
    xmin=-1, xmax=1,
    ymin=-1, ymax=1,
    axis lines=center,
    axis equal,
    hide axis,  % This will hide the axis
    % xlabel={$x$},
    % ylabel={$y$},
    % grid=both,
    % minor tick num=1,
    domain=0:2*pi,
    samples=200
  ]

  \addplot[thick, parametric, smooth, fill=blue!30] ({(\rZero + \rOne*cos(5*deg(x)))*sin(deg(x))} , {(\rZero + \rOne*cos(5*deg(x)))*cos(deg(x))});
  \node at (0.1, 0.1) {$\Omega$};
  \node at (0.6, 0.6) {$\phi( x,y ) = 0 $};

  % Background mesh
  \foreach \i in {-1, -0.75, ..., 1} {
      \addplot[ thin, opacity=0.3, forget plot] coordinates {(\i,-1) (\i,1)};
      \addplot[ thin, opacity=0.3, forget plot] coordinates {(-1,\i) (1,\i)};
  }

  \end{axis}
\end{tikzpicture}
\caption{Illustration of the flower domain associated with the level set function \eqref{eq:flower} s.t. $\phi ( x,y) =0$ for $r_{0} = 0.3L$ and $r_{1} = 0.1L$ }
\end{figure}

\subsubsection{Translation test}%
\label{ssub:translation_test}


Here we have the translation test $\delta $ from $0$ to $2 \sqrt{2}h $.


\begin{figure}
    \centering
    \begin{tikzpicture}[scale=0.9]

        % Domain is blue
        % \fill[blue!30] (0.1, 0.1) circle (1.5cm);

        \coordinate (p0) at (0.1, 0.1);  % the origin of the circle

        \draw[black,line width=0.8pt, opacity=1.0 ] (0.1, 0.1) circle (1.1cm);
        \draw[black,line width=0.8pt, opacity=0.7 ] (0.4535, 0.4535) circle (1.1cm);
        \draw[black,line width=0.8pt, opacity=0.3 ] (0.807, 0.807) circle (1.1cm);
        \draw[black,line width=0.8pt, opacity=0.2 ] (1.16, 1.16) circle (1.1cm);

        % Drawing points
        \fill[black] (p0) circle (2pt);
        \pgfmathsetmacro{\pOneX}{0.1 + 1.5*sqrt(2)*0.5}
        \pgfmathsetmacro{\pOneY}{0.1 + 1.5*sqrt(2)*0.5}
        \coordinate (p1) at (\pOneX, \pOneY);

        \draw[->, thick] (p0) -- (p1) node[midway,below] {$\delta$};
        % Background mesh
        \foreach \i in {-2.5, -2, ..., 2.5} {
            \draw[line width=0.1pt, shift={(-2.5,\i)}, opacity=0.2] (0,0) -- (5,0);
            \draw[line width=0.1pt, shift={(\i,-2.5)},opacity=0.2] (0,0) -- (0,5);
        }

    \end{tikzpicture}
    \caption{Illustration of the translation test with translation $\delta$ from  $(0,2 \sqrt{2}h)  $ }
\end{figure}


\begin{figure}[htp]
\centering

\subfloat[]{%
% Recommended preamble:
% \usetikzlibrary{arrows.meta}
% \usetikzlibrary{backgrounds}
% \usepgfplotslibrary{patchplots}
% \usepgfplotslibrary{fillbetween}
% \pgfplotsset{%
%     layers/standard/.define layer set={%
%         background,axis background,axis grid,axis ticks,axis lines,axis tick labels,pre main,main,axis descriptions,axis foreground%
%     }{
%         grid style={/pgfplots/on layer=axis grid},%
%         tick style={/pgfplots/on layer=axis ticks},%
%         axis line style={/pgfplots/on layer=axis lines},%
%         label style={/pgfplots/on layer=axis descriptions},%
%         legend style={/pgfplots/on layer=axis descriptions},%
%         title style={/pgfplots/on layer=axis descriptions},%
%         colorbar style={/pgfplots/on layer=axis descriptions},%
%         ticklabel style={/pgfplots/on layer=axis tick labels},%
%         axis background@ style={/pgfplots/on layer=axis background},%
%         3d box foreground style={/pgfplots/on layer=axis foreground},%
%     },
% }

\begin{tikzpicture}[/tikz/background rectangle/.style={fill={rgb,1:red,1.0;green,1.0;blue,1.0}, fill opacity={1.0}, draw opacity={1.0}}, show background rectangle]
\begin{axis}[point meta max={nan}, point meta min={nan}, legend cell align={left}, legend columns={1}, title={}, title style={at={{(0.5,1)}}, anchor={south}, font={{\fontsize{14 pt}{18.2 pt}\selectfont}}, color={rgb,1:red,0.0;green,0.0;blue,0.0}, draw opacity={1.0}, rotate={0.0}, align={center}}, legend style={color={rgb,1:red,0.0;green,0.0;blue,0.0}, draw opacity={1.0}, line width={1}, solid, fill={rgb,1:red,1.0;green,1.0;blue,1.0}, fill opacity={1.0}, text opacity={1.0}, font={{\fontsize{12 pt}{15.600000000000001 pt}\selectfont}}, text={rgb,1:red,0.0;green,0.0;blue,0.0}, cells={anchor={center}}, at={(1.02, 1)}, anchor={north west}}, axis background/.style={fill={rgb,1:red,1.0;green,1.0;blue,1.0}, opacity={1.0}}, anchor={north west}, xshift={1.0mm}, yshift={-1.0mm}, width={145.4mm}, height={48.8mm}, scaled x ticks={false}, xlabel={}, x tick style={color={rgb,1:red,0.0;green,0.0;blue,0.0}, opacity={1.0}}, x tick label style={color={rgb,1:red,0.0;green,0.0;blue,0.0}, opacity={1.0}, rotate={0}}, xlabel style={at={(ticklabel cs:0.5)}, anchor=near ticklabel, at={{(ticklabel cs:0.5)}}, anchor={near ticklabel}, font={{\fontsize{11 pt}{14.3 pt}\selectfont}}, color={rgb,1:red,0.0;green,0.0;blue,0.0}, draw opacity={1.0}, rotate={0.0}}, xmajorgrids={true}, xmin={-0.014318912319027599}, xmax={0.49161598961994724}, xticklabels={{$0.0$,$0.1$,$0.2$,$0.3$,$0.4$}}, xtick={{0.0,0.1,0.2,0.30000000000000004,0.4}}, xtick align={inside}, xticklabel style={font={{\fontsize{8 pt}{10.4 pt}\selectfont}}, color={rgb,1:red,0.0;green,0.0;blue,0.0}, draw opacity={1.0}, rotate={0.0}}, x grid style={color={rgb,1:red,0.0;green,0.0;blue,0.0}, draw opacity={0.1}, line width={0.5}, solid}, axis x line*={left}, x axis line style={color={rgb,1:red,0.0;green,0.0;blue,0.0}, draw opacity={1.0}, line width={1}, solid}, scaled y ticks={false}, ylabel={}, y tick style={color={rgb,1:red,0.0;green,0.0;blue,0.0}, opacity={1.0}}, y tick label style={color={rgb,1:red,0.0;green,0.0;blue,0.0}, opacity={1.0}, rotate={0}}, ylabel style={at={(ticklabel cs:0.5)}, anchor=near ticklabel, at={{(ticklabel cs:0.5)}}, anchor={near ticklabel}, font={{\fontsize{11 pt}{14.3 pt}\selectfont}}, color={rgb,1:red,0.0;green,0.0;blue,0.0}, draw opacity={1.0}, rotate={0.0}}, ymode={log}, log basis y={10}, ymajorgrids={true}, ymin={0.20417379446695233}, ymax={4.8977881936844765e23}, yticklabels={{$10^{0}$,$10^{5}$,$10^{10}$,$10^{15}$,$10^{20}$}}, ytick={{1.0,100000.0,1.0e10,1.0e15,1.0e20}}, ytick align={inside}, yticklabel style={font={{\fontsize{8 pt}{10.4 pt}\selectfont}}, color={rgb,1:red,0.0;green,0.0;blue,0.0}, draw opacity={1.0}, rotate={0.0}}, y grid style={color={rgb,1:red,0.0;green,0.0;blue,0.0}, draw opacity={0.1}, line width={0.5}, solid}, axis y line*={left}, y axis line style={color={rgb,1:red,0.0;green,0.0;blue,0.0}, draw opacity={1.0}, line width={1}, solid}, colorbar={false}]
    [\addlegendimage{empty legend}] \addlegendentry[font={{\fontsize{11 pt}{14.3 pt}\selectfont}}, text={rgb,1:red,0.0;green,0.0;blue,0.0}] {\hspace{-.6cm}{\textbf{$(\gamma, \gamma_1, \gamma_2)$}}}
    \addplot[color={rgb,1:red,0.0;green,0.0;blue,1.0}, name path={6b6e567c-0536-4ccb-85ac-ff2f726f8688}, draw opacity={1.0}, line width={1}, solid]
        table[row sep={\\}]
        {
            \\
            0.0  2.4570950441198234e7  \\
            0.0009565071689397187  2.4575177817058776e7  \\
            0.0019130143378794373  2.457939432153537e7  \\
            0.002869521506819156  2.458359993262882e7  \\
            0.0038260286757588746  2.458779465674972e7  \\
            0.004782535844698593  2.4591978499855544e7  \\
            0.005739043013638312  2.459615149276407e7  \\
            0.0066955501825780315  2.4600313711532637e7  \\
            0.007652057351517749  2.4604465240478504e7  \\
            0.008608564520457468  2.4608606200320814e7  \\
            0.009565071689397187  2.461273674546548e7  \\
            0.010521578858336907  2.4616857075354844e7  \\
            0.011478086027276624  2.462096741405805e7  \\
            0.012434593196216343  2.4625068048178114e7  \\
            0.013391100365156063  2.4629159291351434e7  \\
            0.01434760753409578  2.4633241499038637e7  \\
            0.015304114703035498  2.4637315092569552e7  \\
            0.016260621871975217  2.4641380518918514e7  \\
            0.017217129040914936  2.907160783788938e7  \\
            0.018173636209854658  3.01736550733328e7  \\
            0.019130143378794373  3.0161159674663432e7  \\
            0.020086650547734092  3.014553019906001e7  \\
            0.021043157716673814  3.0126793965248656e7  \\
            0.02199966488561353  3.0104971701026954e7  \\
            0.022956172054553248  3.008007733370636e7  \\
            0.02391267922349297  3.0052119811621603e7  \\
            0.024869186392432685  3.0021104900275115e7  \\
            0.025825693561372404  2.9987036114925243e7  \\
            0.026782200730312126  2.9949914985242825e7  \\
            0.027738707899251844  2.9909740491733182e7  \\
            0.02869521506819156  2.9866507994435318e7  \\
            0.02965172223713128  2.9820208143328153e7  \\
            0.030608229406070997  2.9921530302125357e7  \\
            0.031564736575010716  3.004689185085508e7  \\
            0.032521243743950434  3.0102519944424547e7  \\
            0.03347775091289015  3.0075287682572857e7  \\
            0.03443425808182987  3.00129865087198e7  \\
            0.0353907652507696  2.9947594719055988e7  \\
            0.036347272419709316  2.988531269685183e7  \\
            0.03730377958864903  2.999752899209736e7  \\
            0.038260286757588746  3.0099219153132316e7  \\
            0.039216793926528465  3.0190026228974443e7  \\
            0.040173301095468184  3.0269684739036746e7  \\
            0.04112980826440791  3.0338010495639905e7  \\
            0.04208631543334763  3.039489075425501e7  \\
            0.043042822602287346  3.0440274686266273e7  \\
            0.04399932977122706  3.0474454318662822e7  \\
            0.04495583694016678  3.0497795129235696e7  \\
            0.045912344109106495  3.0341716710751604e7  \\
            0.046868851278046214  3.0771626401125457e7  \\
            0.04782535844698594  3.076151458712502e7  \\
            0.04878186561592566  3.0749446394353345e7  \\
            0.04973837278486537  3.073509797094306e7  \\
            0.05069487995380509  3.0718203611125216e7  \\
            0.05165138712274481  3.069876201103822e7  \\
            0.052607894291684526  3.06767678203732e7  \\
            0.05356440146062425  3.0652212190914962e7  \\
            0.05452090862956397  3.0625083300759733e7  \\
            0.05547741579850369  3.0595366822776075e7  \\
            0.0564339229674434  3.056304643253768e7  \\
            0.05739043013638312  3.0528104234280046e7  \\
            0.05834693730532284  3.049052128220792e7  \\
            0.05930344447426256  3.0335482560209706e7  \\
            0.060259951643202275  3.0331179618278597e7  \\
            0.061216458812141994  3.0326477862591956e7  \\
            0.06217296598108171  3.032137135337371e7  \\
            0.06312947315002143  3.0315854469616923e7  \\
            0.06408598031896115  3.0309922067050803e7  \\
            0.06504248748790087  3.0303569589425206e7  \\
            0.06599899465684059  3.0296793219212066e7  \\
            0.0669555018257803  3.0289589822027303e7  \\
            0.06791200899472002  3.0284066007855423e7  \\
            0.06886851616365974  3.0278268767731234e7  \\
            0.06982502333259948  3.027204638185062e7  \\
            0.0707815305015392  3.026539292424087e7  \\
            0.07173803767047891  3.0258609103185315e7  \\
            0.07269454483941863  3.025252874422468e7  \\
            0.07365105200835834  3.026546132730754e7  \\
            0.07460755917729806  3.0282587366997167e7  \\
            0.07556406634623777  3.0290099721479796e7  \\
            0.07652057351517749  3.0707451620736662e7  \\
            0.07747708068411721  3.0714726576558206e7  \\
            0.07843358785305693  3.071931162905674e7  \\
            0.07939009502199665  3.0721196993066136e7  \\
            0.08034660219093637  3.0720367076699384e7  \\
            0.08130310935987609  3.112420066965612e7  \\
            0.08225961652881582  3.1108195263269275e7  \\
            0.08321612369775554  3.1091651801831283e7  \\
            0.08417263086669526  3.1074561353713505e7  \\
            0.08512913803563497  3.1056914988972094e7  \\
            0.08608564520457469  3.103870374273628e7  \\
            0.0870421523735144  3.1019918641520966e7  \\
            0.08799865954245412  3.1000550734756198e7  \\
            0.08895516671139383  3.098059106057635e7  \\
            0.08991167388033355  3.0960030676214512e7  \\
            0.09086818104927327  3.093886065484303e7  \\
            0.09182468821821299  3.091707213473941e7  \\
            0.09278119538715271  3.0894656260436695e7  \\
            0.09373770255609243  3.0871604273380786e7  \\
            0.09469420972503216  3.0847907456246473e7  \\
            0.09565071689397188  3.082355716856913e7  \\
            0.0966072240629116  3.0798544869435646e7  \\
            0.09756373123185132  3.077286208489275e7  \\
            0.09852023840079104  3.0802839391262364e7  \\
            0.09947674556973074  3.0794974625745077e7  \\
            0.10043325273867046  3.0782357008080017e7  \\
            0.10138975990761018  3.076508009316329e7  \\
            0.1023462670765499  3.0743226851306308e7  \\
            0.10330277424548961  3.0716871362234257e7  \\
            0.10425928141442933  3.068608048380502e7  \\
            0.10521578858336905  3.0650915350635044e7  \\
            0.10617229575230877  3.0611432740062624e7  \\
            0.1071288029212485  3.0567686386869814e7  \\
            0.10808531009018822  3.0519728125949744e7  \\
            0.10904181725912794  3.0467608958313305e7  \\
            0.10999832442806766  3.0411380024363186e7  \\
            0.11095483159700738  3.0351093489779077e7  \\
            0.1119113387659471  3.028680333180436e7  \\
            0.1128678459348868  3.021856601931354e7  \\
            0.11382435310382652  3.0146441204509348e7  \\
            0.11478086027276624  3.0070492202054854e7  \\
            0.11573736744170596  2.999078653415546e7  \\
            0.11669387461064568  2.990739633085573e7  \\
            0.1176503817795854  2.9820398654502526e7  \\
            0.11860688894852511  2.9729875815354533e7  \\
            0.11956339611746483  2.9683319727943894e7  \\
            0.12051990328640455  2.9775572556039862e7  \\
            0.12147641045534427  2.986434344364219e7  \\
            0.12243291762428399  2.994954734457105e7  \\
            0.1233894247932237  3.0031104590266336e7  \\
            0.12434593196216343  3.0108940642423145e7  \\
            0.12530243913110314  3.0182985717834253e7  \\
            0.12625894630004286  3.025317446277133e7  \\
            0.12721545346898258  3.0319445441135745e7  \\
            0.1281719606379223  3.0381740672812056e7  \\
            0.12912846780686202  3.044000502522762e7  \\
            0.13008497497580174  3.0494185508416772e7  \\
            0.13104148214474146  3.054423058240205e7  \\
            0.13199798931368117  3.0590089297724113e7  \\
            0.1329544964826209  3.0631710350395396e7  \\
            0.1339110036515606  3.0669041105470937e7  \\
            0.13486751082050033  3.070202644595319e7  \\
            0.13582401798944005  3.073060759898191e7  \\
            0.13678052515837977  3.075472073663347e7  \\
            0.1377370323273195  3.0774295582931016e7  \\
            0.13869353949625923  3.0789253838792805e7  \\
            0.13965004666519895  3.0799507466265783e7  \\
            0.14060655383413867  3.0759766658113934e7  \\
            0.1415630610030784  3.0785787822977725e7  \\
            0.1425195681720181  3.0811134324066084e7  \\
            0.14347607534095783  3.083581455384143e7  \\
            0.14443258250989754  3.085983702791809e7  \\
            0.14538908967883726  3.088321034622242e7  \\
            0.14634559684777695  3.0905943185495753e7  \\
            0.14730210401671667  3.0928044278174613e7  \\
            0.1482586111856564  3.094952244300753e7  \\
            0.1492151183545961  3.0970386527896952e7  \\
            0.15017162552353583  3.099064544400026e7  \\
            0.15112813269247555  3.1010308113748e7  \\
            0.15208463986141527  3.1029383498960465e7  \\
            0.15304114703035498  3.1047880543252785e7  \\
            0.1539976541992947  3.1065808235873237e7  \\
            0.15495416136823442  3.1083175516989663e7  \\
            0.15591066853717414  3.1099991353891876e7  \\
            0.15686717570611386  3.1116264672947556e7  \\
            0.15782368287505358  3.071892721989415e7  \\
            0.1587801900439933  3.072112263113543e7  \\
            0.15973669721293302  3.0720592560371246e7  \\
            0.16069320438187273  3.071735578501008e7  \\
            0.16164971155081245  3.0711424902309578e7  \\
            0.16260621871975217  3.070280701640095e7  \\
            0.16356272588869192  3.028756890506517e7  \\
            0.16451923305763164  3.027520267830903e7  \\
            0.16547574022657136  3.0253408298281793e7  \\
            0.16643224739551107  3.0255624442407522e7  \\
            0.1673887545644508  3.026190263755165e7  \\
            0.1683452617333905  3.026877392038857e7  \\
            0.16930176890233023  3.027521109110161e7  \\
            0.17025827607126995  3.028122016581096e7  \\
            0.17121478324020967  3.0286807018515814e7  \\
            0.17217129040914939  3.0293245076011453e7  \\
            0.17312779757808908  3.0300234606798284e7  \\
            0.1740843047470288  3.0306798598103933e7  \\
            0.1750408119159685  3.0312940514241602e7  \\
            0.17599731908490823  3.0318664549035527e7  \\
            0.17695382625384795  3.0323975562352255e7  \\
            0.17791033342278767  3.0328878969629582e7  \\
            0.1788668405917274  3.0333380566261657e7  \\
            0.1798233477606671  3.047073344989923e7  \\
            0.18077985492960683  3.0509644111982655e7  \\
            0.18173636209854654  3.0545904257205084e7  \\
            0.18269286926748626  3.0579533241080735e7  \\
            0.18364937643642598  3.0610549524144746e7  \\
            0.1846058836053657  3.063897028156015e7  \\
            0.18556239077430542  3.066481089942761e7  \\
            0.18651889794324514  3.068808450350587e7  \\
            0.18747540511218486  3.070880149652774e7  \\
            0.18843191228112458  3.0726969055326845e7  \\
            0.18938841945006432  3.0742590559392318e7  \\
            0.19034492661900404  3.0755665084410887e7  \\
            0.19130143378794376  3.0766890594557967e7  \\
            0.19225794095688348  3.033877680218117e7  \\
            0.1932144481258232  3.0344449802868787e7  \\
            0.19417095529476291  3.0487550792879924e7  \\
            0.19512746246370263  3.0458654324025735e7  \\
            0.19608396963264235  3.0419020755771715e7  \\
            0.19704047680158207  3.0367885808494892e7  \\
            0.1979969839705218  3.0305272795080584e7  \\
            0.19895349113946148  3.02312623205531e7  \\
            0.1999099983084012  3.014600165498742e7  \\
            0.20086650547734092  3.004971437743299e7  \\
            0.20182301264628064  2.9942710415348355e7  \\
            0.20277951981522035  2.9913742699475113e7  \\
            0.20373602698416007  2.998067646755411e7  \\
            0.2046925341530998  3.0044523909437925e7  \\
            0.2056490413220395  3.0105277288137622e7  \\
            0.20660554849097923  3.008417027192958e7  \\
            0.20756205565991895  2.9992178280128945e7  \\
            0.20851856282885867  2.983644933996319e7  \\
            0.20947506999779839  2.984374228712415e7  \\
            0.2104315771667381  2.9888506993970137e7  \\
            0.21138808433567782  2.9930209627512947e7  \\
            0.21234459150461754  2.9968857126821715e7  \\
            0.21330109867355726  3.0004452133418042e7  \\
            0.214257605842497  3.0036994294982463e7  \\
            0.21521411301143673  3.00664810884507e7  \\
            0.21617062018037644  3.0092907918077253e7  \\
            0.21712712734931616  3.0116267612076912e7  \\
            0.21808363451825588  3.0136548989436653e7  \\
            0.2190401416871956  3.015373489637281e7  \\
            0.21999664885613532  3.0167801020986613e7  \\
            0.22095315602507504  2.907045198528429e7  \\
            0.22190966319401476  2.90719838190647e7  \\
            0.22286617036295447  2.4639348789469156e7  \\
            0.2238226775318942  2.4635279345810197e7  \\
            0.2247791847008339  2.4631201497335e7  \\
            0.2257356918697736  2.4627114818020053e7  \\
            0.22669219903871332  2.4623018926657304e7  \\
            0.22764870620765304  2.4618913472579613e7  \\
            0.22860521337659276  2.4614798167866807e7  \\
            0.22956172054553248  2.4610672760244556e7  \\
            0.2305182277144722  2.4606537028148998e7  \\
            0.23147473488341191  2.460239080608322e7  \\
            0.23243124205235163  2.4598233943630286e7  \\
            0.23338774922129135  2.459406634688763e7  \\
            0.23434425639023107  2.4589887935538877e7  \\
            0.2353007635591708  2.4585698650929164e7  \\
            0.2362572707281105  2.4581498485237643e7  \\
            0.23721377789705023  2.457728742830275e7  \\
            0.23817028506598995  2.4573065485630758e7  \\
            0.23912679223492966  2.4573065482373662e7  \\
            0.24008329940386938  2.457728742529129e7  \\
            0.2410398065728091  2.4581498486895002e7  \\
            0.24199631374174882  2.458569865382936e7  \\
            0.24295282091068854  2.4589887935241114e7  \\
            0.24390932807962826  2.459406634834946e7  \\
            0.24486583524856798  2.4598233948105216e7  \\
            0.2458223424175077  2.460239080223208e7  \\
            0.2467788495864474  2.46065370279639e7  \\
            0.24773535675538713  2.4610672756986853e7  \\
            0.24869186392432685  2.461479817131057e7  \\
            0.24964837109326657  2.461891347184445e7  \\
            0.2506048782622063  2.4623018923826165e7  \\
            0.25156138543114603  2.4627114818794254e7  \\
            0.2525178926000857  2.4631201500742715e7  \\
            0.25347439976902547  2.4635279346763823e7  \\
            0.25443090693796516  2.4639348795084365e7  \\
            0.2553874141069049  2.9071983783845942e7  \\
            0.2563439212758446  2.9070451950598136e7  \\
            0.25730042844478435  3.01678009788245e7  \\
            0.25825693561372404  3.0153734865744185e7  \\
            0.2592134427826638  3.0136548951409586e7  \\
            0.2601699499516035  3.0116267571264926e7  \\
            0.2611264571205432  3.0092907877280835e7  \\
            0.2620829642894829  3.0066481046867378e7  \\
            0.26303947145842266  3.003699426113173e7  \\
            0.26399597862736235  3.0004452094181716e7  \\
            0.2649524857963021  2.9968857093126345e7  \\
            0.2659089929652418  2.99302095937406e7  \\
            0.26686550013418153  2.9888506958824255e7  \\
            0.2678220073031212  2.9843742251627482e7  \\
            0.26877851447206097  2.9836449375373743e7  \\
            0.26973502164100066  2.9992178310932744e7  \\
            0.2706915288099404  3.008417030059203e7  \\
            0.2716480359788801  3.010527725209657e7  \\
            0.27260454314781984  3.0044523874954846e7  \\
            0.27356105031675954  2.9980676430762e7  \\
            0.2745175574856993  2.9913742664578613e7  \\
            0.275474064654639  2.9942710449694216e7  \\
            0.2764305718235787  3.0049714409293562e7  \\
            0.27738707899251847  3.0146001691738196e7  \\
            0.27834358616145816  3.0231262356109045e7  \\
            0.2793000933303979  3.030527281502146e7  \\
            0.2802566004993376  3.0367885833233006e7  \\
            0.28121310766827734  3.04190207692425e7  \\
            0.28216961483721703  3.0458654332807556e7  \\
            0.2831261220061568  3.0487550802924424e7  \\
            0.28408262917509647  3.0344449771424025e7  \\
            0.2850391363440362  3.033877680241027e7  \\
            0.2859956435129759  3.0766890560436744e7  \\
            0.28695215068191565  3.0755665040552486e7  \\
            0.28790865785085534  3.0742590515173417e7  \\
            0.2888651650197951  3.072696901698025e7  \\
            0.2898216721887348  3.0708801464417815e7  \\
            0.2907781793576745  3.068808445893715e7  \\
            0.29173468652661416  3.0664810856705133e7  \\
            0.2926911936955539  3.063897024290705e7  \\
            0.2936477008644936  3.061054949263729e7  \\
            0.29460420803343335  3.057953319898957e7  \\
            0.29556071520237304  3.0545904219456498e7  \\
            0.2965172223713128  3.050964406797651e7  \\
            0.2974737295402525  3.0470733406598862e7  \\
            0.2984302367091922  3.033338056168998e7  \\
            0.2993867438781319  3.0328878968684807e7  \\
            0.30034325104707166  3.0323975567967772e7  \\
            0.3012997582160114  3.031866454672022e7  \\
            0.3022562653849511  3.03129405027257e7  \\
            0.30321277255389084  3.0306798595403954e7  \\
            0.30416927972283053  3.030023460266286e7  \\
            0.3051257868917703  3.0293245066209886e7  \\
            0.30608229406070997  3.0286807013139594e7  \\
            0.3070388012296497  3.0281220161414873e7  \\
            0.3079953083985894  3.02752110857404e7  \\
            0.30895181556752915  3.026877391659413e7  \\
            0.30990832273646884  3.0261902631018434e7  \\
            0.3108648299054086  3.0255624439983536e7  \\
            0.3118213370743483  3.025340833444095e7  \\
            0.31277784424328803  3.027520270679611e7  \\
            0.3137343514122277  3.0287568941408742e7  \\
            0.31469085858116747  3.0702807055033367e7  \\
            0.31564736575010716  3.0711424937590323e7  \\
            0.3166038729190469  3.071735582099643e7  \\
            0.3175603800879866  3.072059259378963e7  \\
            0.31851688725692634  3.0721122668494757e7  \\
            0.31947339442586603  3.071892725537317e7  \\
            0.3204299015948058  3.111626466844076e7  \\
            0.32138640876374547  3.1099991340138778e7  \\
            0.3223429159326852  3.108317550425999e7  \\
            0.3232994231016249  3.1065808216852106e7  \\
            0.32425593027056465  3.104788053543975e7  \\
            0.32521243743950434  3.1029383479557272e7  \\
            0.3261689446084441  3.1010308099328134e7  \\
            0.32712545177738384  3.0990645428899355e7  \\
            0.3280819589463235  3.097038651052753e7  \\
            0.3290384661152633  3.0949522421516128e7  \\
            0.32999497328420296  3.0928044264919538e7  \\
            0.3309514804531427  3.090594316230229e7  \\
            0.3319079876220824  3.0883210333416384e7  \\
            0.33286449479102215  3.0859837008729413e7  \\
            0.33382100195996184  3.083581453549271e7  \\
            0.3347775091289016  3.08111342973057e7  \\
            0.3357340162978413  3.078578780087762e7  \\
            0.336690523466781  3.0759766639093738e7  \\
            0.3376470306357207  3.0799507427978713e7  \\
            0.33860353780466046  3.0789253806501605e7  \\
            0.33956004497360015  3.0774295549455315e7  \\
            0.3405165521425399  3.0754720695578974e7  \\
            0.3414730593114796  3.0730607564238086e7  \\
            0.34242956648041933  3.0702026408382367e7  \\
            0.343386073649359  3.0669041059616394e7  \\
            0.34434258081829877  3.063171031172745e7  \\
            0.3452990879872384  3.059008925423312e7  \\
            0.34625559515617815  3.054423055075131e7  \\
            0.34721210232511784  3.0494185467334226e7  \\
            0.3481686094940576  3.0440004982362393e7  \\
            0.3491251166629973  3.038174063866572e7  \\
            0.350081623831937  3.0319445397941872e7  \\
            0.3510381310008768  3.0253174418414272e7  \\
            0.35199463816981647  3.018298568030272e7  \\
            0.3529511453387562  3.010894060455665e7  \\
            0.3539076525076959  3.0031104558951944e7  \\
            0.35486415967663565  2.9949547302579995e7  \\
            0.35582066684557534  2.9864343409090124e7  \\
            0.3567771740145151  2.9775572524318885e7  \\
            0.3577336811834548  2.9683319683590423e7  \\
            0.3586901883523945  2.9729875851715174e7  \\
            0.3596466955213342  2.9820398684070054e7  \\
            0.36060320269027396  2.9907396364054307e7  \\
            0.36155970985921365  2.999078657556043e7  \\
            0.3625162170281534  3.007049223366633e7  \\
            0.3634727241970931  3.014644123456949e7  \\
            0.36442923136603284  3.021856606096142e7  \\
            0.3653857385349725  3.0286803367960364e7  \\
            0.3663422457039123  3.0351093533642422e7  \\
            0.36729875287285196  3.0411380058893193e7  \\
            0.3682552600417917  3.0467608989377838e7  \\
            0.3692117672107314  3.051972815728923e7  \\
            0.37016827437967115  3.0567686423309222e7  \\
            0.37112478154861084  3.0611432771321993e7  \\
            0.3720812887175506  3.0650915386305217e7  \\
            0.3730377958864903  3.0686080518180557e7  \\
            0.37399430305543  3.0716871399124347e7  \\
            0.3749508102243697  3.0743226886467367e7  \\
            0.37590731739330946  3.076508012939139e7  \\
            0.37686382456224915  3.0782357036973022e7  \\
            0.3778203317311889  3.079497466202439e7  \\
            0.37877683890012864  3.0802839425481606e7  \\
            0.37973334606906833  3.0772862095347285e7  \\
            0.3806898532380081  3.079854488607746e7  \\
            0.38164636040694777  3.082355719096101e7  \\
            0.3826028675758875  3.0847907471594803e7  \\
            0.3835593747448272  3.0871604297254715e7  \\
            0.38451588191376695  3.0894656282668184e7  \\
            0.38547238908270665  3.0917072144794162e7  \\
            0.3864288962516464  3.0938860675577164e7  \\
            0.3873854034205861  3.0960030692258693e7  \\
            0.38834191058952583  3.09805910728807e7  \\
            0.3892984177584655  3.1000550753608078e7  \\
            0.39025492492740527  3.1019918659720954e7  \\
            0.39121143209634496  3.1038703754629113e7  \\
            0.3921679392652847  3.1056914996125918e7  \\
            0.3931244464342244  3.1074561370037157e7  \\
            0.39408095360316414  3.1091651805671602e7  \\
            0.39503746077210383  3.1108195271524064e7  \\
            0.3959939679410436  3.1124200679101568e7  \\
            0.39695047510998327  3.0720367044622265e7  \\
            0.39790698227892296  3.07211969570261e7  \\
            0.39886348944786265  3.0719311586340696e7  \\
            0.3998199966168024  3.0714726537941653e7  \\
            0.4007765037857421  3.070745157826378e7  \\
            0.40173301095468184  3.0290099680435043e7  \\
            0.4026895181236216  3.0282587324158646e7  \\
            0.4036460252925613  3.026546128350644e7  \\
            0.404602532461501  3.025252874853202e7  \\
            0.4055590396304407  3.0258609105273053e7  \\
            0.40651554679938046  3.0265392930485714e7  \\
            0.40747205396832015  3.0272046386204123e7  \\
            0.4084285611372599  3.0278268772783525e7  \\
            0.4093850683061996  3.028406601181682e7  \\
            0.41034157547513933  3.028958982317097e7  \\
            0.411298082644079  3.029679322055194e7  \\
            0.41225458981301877  3.0303569598112736e7  \\
            0.41321109698195846  3.0309922068390656e7  \\
            0.4141676041508982  3.031585447098303e7  \\
            0.4151241113198379  3.0321371358573902e7  \\
            0.41608061848877764  3.0326477868632194e7  \\
            0.41703712565771733  3.0331179618346915e7  \\
            0.4179936328266571  3.0335482557138223e7  \\
            0.41895013999559677  3.0490521318382397e7  \\
            0.4199066471645365  3.0528104272652082e7  \\
            0.4208631543334762  3.0563046463415507e7  \\
            0.42181966150241595  3.0595366856836822e7  \\
            0.42277616867135565  3.062508333946821e7  \\
            0.4237326758402954  3.065221222926261e7  \\
            0.4246891830092351  3.0676767850883096e7  \\
            0.42564569017817483  3.0698762046829924e7  \\
            0.4266021973471145  3.071820364809627e7  \\
            0.42755870451605427  3.073509800733173e7  \\
            0.428515211684994  3.0749446428056683e7  \\
            0.4294717188539337  3.0761514622226343e7  \\
            0.43042822602287345  3.0771626430848684e7  \\
            0.43138473319181314  3.0341716742912374e7  \\
            0.4323412403607529  3.0497795118115388e7  \\
            0.4332977475296926  3.04744543101396e7  \\
            0.4342542546986323  3.0440274669962056e7  \\
            0.435210761867572  3.039489073781782e7  \\
            0.43616726903651176  3.033801047451784e7  \\
            0.43712377620545145  3.026968470659531e7  \\
            0.4380802833743912  3.0190026198443636e7  \\
            0.4390367905433309  3.0099219119303986e7  \\
            0.43999329771227064  2.999752896140491e7  \\
            0.4409498048812103  2.9885312654516544e7  \\
            0.4419063120501501  2.9947594754010115e7  \\
            0.44286281921908976  3.001298654301016e7  \\
            0.4438193263880295  3.0075287713397577e7  \\
            0.4447758335569692  3.010251990591001e7  \\
            0.44573234072590895  3.004689181892752e7  \\
            0.44668884789484864  2.9921530265269466e7  \\
            0.4476453550637884  2.982020817819402e7  \\
            0.4486018622327281  2.9866508027666938e7  \\
            0.4495583694016678  2.9909740526872296e7  \\
            0.4505148765706075  2.9949915016551923e7  \\
            0.4514713837395472  2.9987036150972072e7  \\
            0.45242789090848695  3.0021104930279184e7  \\
            0.45338439807742664  3.0052119847611327e7  \\
            0.4543409052463664  3.0080077370535545e7  \\
            0.4552974124153061  3.0104971734610923e7  \\
            0.4562539195842458  3.0126793994229548e7  \\
            0.4572104267531855  3.014553022954214e7  \\
            0.45816693392212526  3.0161159707895633e7  \\
            0.45912344109106495  3.0173655108579237e7  \\
            0.4600799482600047  2.907160786676577e7  \\
            0.4610364554289444  2.4641380512272116e7  \\
            0.46199296259788414  2.463731508761835e7  \\
            0.46294946976682383  2.463324150187979e7  \\
            0.4639059769357636  2.462915928578683e7  \\
            0.46486248410470327  2.4625068044178445e7  \\
            0.465818991273643  2.4620967416665543e7  \\
            0.4667754984425827  2.4616857067734804e7  \\
            0.46773200561152245  2.461273673887529e7  \\
            0.46868851278046214  2.4608606194144487e7  \\
            0.4696450199494019  2.4604465234844e7  \\
            0.4706015271183416  2.4600313714814134e7  \\
            0.4715580342872813  2.459615149524198e7  \\
            0.472514541456221  2.4591978496477418e7  \\
            0.47347104862516076  2.4587794650482167e7  \\
            0.47442755579410045  2.45835999353068e7  \\
            0.4753840629630402  2.4579394320014752e7  \\
            0.4763405701319799  2.4575177812048133e7  \\
            0.47729707730091964  2.4570950435266025e7  \\
        }
        ;
    \addlegendentry {$(20.0, 10.0, 0.5) $}
    \addplot[color={rgb,1:red,1.0;green,0.0;blue,0.0}, name path={c8b1aeb1-58b2-4cb9-ab42-5a47cc9639fc}, draw opacity={1.0}, line width={1}, solid]
        table[row sep={\\}]
        {
            \\
            0.0  8.939008519076015e9  \\
            0.0009565071689397187  1.3826755787084549e10  \\
            0.0019130143378794373  2.1845573828891613e10  \\
            0.002869521506819156  3.538174415400113e10  \\
            0.0038260286757588746  5.8993659861484276e10  \\
            0.004782535844698593  1.0177583977302077e11  \\
            0.005739043013638312  1.8282061669420212e11  \\
            0.0066955501825780315  3.446639302476399e11  \\
            0.007652057351517749  6.89042815846331e11  \\
            0.008608564520457468  1.4804593582444592e12  \\
            0.009565071689397187  3.4851347619935e12  \\
            0.010521578858336907  9.236258543164941e12  \\
            0.011478086027276624  2.8720547666490316e13  \\
            0.012434593196216343  1.1210196377109917e14  \\
            0.013391100365156063  6.203310260652621e14  \\
            0.01434760753409578  6.291201174936238e15  \\
            0.015304114703035498  2.4025306959790736e17  \\
            0.016260621871975217  3.55470011833026e21  \\
            0.017217129040914936  7.785483774539897e10  \\
            0.018173636209854658  2.996175105921447e16  \\
            0.019130143378794373  3.588855389296854e13  \\
            0.020086650547734092  2.035004609196768e12  \\
            0.021043157716673814  3.1169748548576184e11  \\
            0.02199966488561353  1.6805173352975668e11  \\
            0.022956172054553248  1.9834396781299347e11  \\
            0.02391267922349297  2.3514235775175546e11  \\
            0.024869186392432685  2.8008142360864594e11  \\
            0.025825693561372404  3.352710693984947e11  \\
            0.026782200730312126  4.034519664406695e11  \\
            0.027738707899251844  4.8821275052523566e11  \\
            0.02869521506819156  5.942896068316727e11  \\
            0.02965172223713128  7.279864851202438e11  \\
            0.030608229406070997  8.97770034122171e11  \\
            0.031564736575010716  1.1151215289547913e12  \\
            0.032521243743950434  1.3957706584429585e12  \\
            0.03347775091289015  7.331095137519722e16  \\
            0.03443425808182987  5.726433923219986e14  \\
            0.0353907652507696  4.413350665252235e13  \\
            0.036347272419709316  7.709867950139438e12  \\
            0.03730377958864903  4.918922352527987e12  \\
            0.038260286757588746  6.540412630540542e12  \\
            0.039216793926528465  8.81332243490471e12  \\
            0.040173301095468184  2.2038707928091516e13  \\
            0.04112980826440791  7.905020691219112e13  \\
            0.04208631543334763  3.863276708726628e14  \\
            0.043042822602287346  3.181028598630001e15  \\
            0.04399932977122706  7.67895260921663e16  \\
            0.04495583694016678  7.702821455234086e19  \\
            0.045912344109106495  1.1898364914478716e14  \\
            0.046868851278046214  1.0e23  \\
            0.04782535844698594  3.591128470196195e17  \\
            0.04878186561592566  8.347170780733437e15  \\
            0.04973837278486537  1.0332318833868214e15  \\
            0.05069487995380509  2.0324596669290772e15  \\
            0.05165138712274481  4.3286654245749505e15  \\
            0.052607894291684526  1.0193937556510028e16  \\
            0.05356440146062425  2.7375431702328508e16  \\
            0.05452090862956397  8.792407143649294e16  \\
            0.05547741579850369  3.657711037139831e17  \\
            0.0564339229674434  2.2886999040912635e18  \\
            0.05739043013638312  3.0174786798037524e19  \\
            0.05834693730532284  2.507079648437468e21  \\
            0.05930344447426256  6.413319971353387e11  \\
            0.060259951643202275  5.197635062694138e11  \\
            0.061216458812141994  4.635708839360576e11  \\
            0.06217296598108171  4.873515048735434e11  \\
            0.06312947315002143  9.58510870658198e11  \\
            0.06408598031896115  2.0669078828236188e11  \\
            0.06504248748790087  1.1024039376921675e12  \\
            0.06599899465684059  1.7832277070095862e11  \\
            0.0669555018257803  7.784904239791014e10  \\
            0.06791200899472002  4.190993459346536e10  \\
            0.06886851616365974  2.492979658960681e10  \\
            0.06982502333259948  1.5762747525208256e10  \\
            0.0707815305015392  1.039805904742587e10  \\
            0.07173803767047891  7.080448979150426e9  \\
            0.07269454483941863  4.94349255549799e9  \\
            0.07365105200835834  3.5225793623125157e9  \\
            0.07460755917729806  2.553252163515623e9  \\
            0.07556406634623777  1.8778012403913858e9  \\
            0.07652057351517749  1.8863112925164388e18  \\
            0.07747708068411721  1.1428839127126374e16  \\
            0.07843358785305693  8.93436533829132e14  \\
            0.07939009502199665  1.7319663757546144e14  \\
            0.08034660219093637  5.551828405550436e13  \\
            0.08130310935987609  3.5890161193655007e21  \\
            0.08225961652881582  2.1331931286717658e17  \\
            0.08321612369775554  5.77684600578272e15  \\
            0.08417263086669526  5.880881801127709e14  \\
            0.08512913803563497  1.0910073074500139e14  \\
            0.08608564520457469  2.8560561546684836e13  \\
            0.0870421523735144  9.351881753552654e12  \\
            0.08799865954245412  3.583466857048173e12  \\
            0.08895516671139383  1.5428090897684578e12  \\
            0.08991167388033355  8.870065362011023e11  \\
            0.09086818104927327  2.0438540622339612e12  \\
            0.09182468821821299  5.247966193918293e12  \\
            0.09278119538715271  1.55537970655395e13  \\
            0.09373770255609243  5.62349960543442e13  \\
            0.09469420972503216  2.7260416596308953e14  \\
            0.09565071689397188  2.1264888078442828e15  \\
            0.0966072240629116  4.113125927164131e16  \\
            0.09756373123185132  9.970487192059294e18  \\
            0.09852023840079104  6.685260179500391e9  \\
            0.09947674556973074  4.498524920148409e9  \\
            0.10043325273867046  3.0650881697404766e9  \\
            0.10138975990761018  2.1092175131508179e9  \\
            0.1023462670765499  1.461990625991943e9  \\
            0.10330277424548961  1.0177047419567645e9  \\
            0.10425928141442933  7.08929626481291e8  \\
            0.10521578858336905  4.9188722030327296e8  \\
            0.10617229575230877  3.376907549250713e8  \\
            0.1071288029212485  2.8472631735809135e8  \\
            0.10808531009018822  3.102367849793143e8  \\
            0.10904181725912794  3.416813476194751e8  \\
            0.10999832442806766  3.887957551412766e9  \\
            0.11095483159700738  4.253124966885904e8  \\
            0.1119113387659471  4.833222352996191e8  \\
            0.1128678459348868  5.61743138602973e8  \\
            0.11382435310382652  6.716009031609267e8  \\
            0.11478086027276624  8.357410948807799e8  \\
            0.11573736744170596  1.1066240806914692e9  \\
            0.11669387461064568  1.6363778983017652e9  \\
            0.1176503817795854  3.107029699920634e9  \\
            0.11860688894852511  2.1239921095444874e10  \\
            0.11956339611746483  1.3898858081225788e10  \\
            0.12051990328640455  5.51606715014098e9  \\
            0.12147641045534427  2.147630844542372e9  \\
            0.12243291762428399  1.3204618669572704e9  \\
            0.1233894247932237  9.522842008965001e8  \\
            0.12434593196216343  7.446768016756355e8  \\
            0.12530243913110314  6.117009804762638e8  \\
            0.12625894630004286  5.1947969307288504e8  \\
            0.12721545346898258  4.521549684671089e8  \\
            0.1281719606379223  4.03236432859101e8  \\
            0.12912846780686202  5.354016427797664e8  \\
            0.13008497497580174  3.2499944299956614e8  \\
            0.13104148214474146  2.968927300190607e8  \\
            0.13199798931368117  2.77840565367776e8  \\
            0.1329544964826209  4.0831967059139436e8  \\
            0.1339110036515606  5.910210648783476e8  \\
            0.13486751082050033  8.495450316450083e8  \\
            0.13582401798944005  1.2193834558686647e9  \\
            0.13678052515837977  1.7547541135499775e9  \\
            0.1377370323273195  2.539888402776487e9  \\
            0.13869353949625923  3.7080728491060514e9  \\
            0.13965004666519895  5.474530377005848e9  \\
            0.14060655383413867  3.970279216084333e21  \\
            0.1415630610030784  3.649957542077611e17  \\
            0.1425195681720181  7.957706539463643e15  \\
            0.14347607534095783  7.048960821224888e14  \\
            0.14443258250989754  1.1836062515927497e14  \\
            0.14538908967883726  2.8715387773999113e13  \\
            0.14634559684777695  8.84899105208793e12  \\
            0.14730210401671667  3.2250422047459424e12  \\
            0.1482586111856564  1.330591672755854e12  \\
            0.1492151183545961  1.048350831400494e12  \\
            0.15017162552353583  2.3216969669685884e12  \\
            0.15112813269247555  5.693203341189602e12  \\
            0.15208463986141527  1.5971483457117389e13  \\
            0.15304114703035498  5.3970220555716445e13  \\
            0.1539976541992947  2.399406089147199e14  \\
            0.15495416136823442  1.665670452068793e15  \\
            0.15591066853717414  2.7177448594204284e16  \\
            0.15686717570611386  4.767118441470547e18  \\
            0.15782368287505358  3.6270098287766336e13  \\
            0.1587801900439933  9.318011192054942e13  \\
            0.15973669721293302  3.637650052230261e14  \\
            0.16069320438187273  2.7206020476165125e15  \\
            0.16164971155081245  8.278898632109947e16  \\
            0.16260621871975217  3.170982008768662e21  \\
            0.16356272588869192  2.185937743566649e9  \\
            0.16451923305763164  2.9931319063012733e9  \\
            0.16547574022657136  4.1633357442313323e9  \\
            0.16643224739551107  5.899719870860133e9  \\
            0.1673887545644508  8.550554334646245e9  \\
            0.1683452617333905  1.2744783633035063e10  \\
            0.16930176890233023  1.970078845536606e10  \\
            0.17025827607126995  3.2023277640785427e10  \\
            0.17121478324020967  5.618840146125923e10  \\
            0.17217129040914939  1.1320020820254913e11  \\
            0.17312779757808908  3.311439058297294e11  \\
            0.1740843047470288  9.960007082675403e11  \\
            0.1750408119159685  8.205002480235715e11  \\
            0.17599731908490823  5.747792900343732e11  \\
            0.17695382625384795  4.616065782864312e11  \\
            0.17791033342278767  4.838357632433135e11  \\
            0.1788668405917274  5.715665854299962e11  \\
            0.1798233477606671  1.0e23  \\
            0.18077985492960683  1.880915504455856e20  \\
            0.18173636209854654  7.302170090143355e18  \\
            0.18269286926748626  8.572243612569379e17  \\
            0.18364937643642598  1.724174173049993e17  \\
            0.1846058836053657  4.778576757819e16  \\
            0.18556239077430542  1.6392789204592964e16  \\
            0.18651889794324514  6.549086844182744e15  \\
            0.18747540511218486  2.933447300140283e15  \\
            0.18843191228112458  1.4363250356468668e15  \\
            0.18938841945006432  2.469165093236452e15  \\
            0.19034492661900404  3.934251636460751e16  \\
            0.19130143378794376  1.3107063314234294e19  \\
            0.19225794095688348  1.502950578877144e14  \\
            0.1932144481258232  9.497218714118734e13  \\
            0.19417095529476291  9.636223092038993e17  \\
            0.19512746246370263  1.2797030897787598e16  \\
            0.19608396963264235  1.0167616433720786e15  \\
            0.19704047680158207  1.6654178649726662e14  \\
            0.1979969839705218  4.047887401291157e13  \\
            0.19895349113946148  1.2621765474385648e13  \\
            0.1999099983084012  7.579033497360532e12  \\
            0.20086650547734092  5.662963778379549e12  \\
            0.20182301264628064  4.2857075732441987e12  \\
            0.20277951981522035  1.7217932112436363e13  \\
            0.20373602698416007  1.3732230847964486e14  \\
            0.2046925341530998  3.9114237831312885e15  \\
            0.2056490413220395  3.877325320447809e19  \\
            0.20660554849097923  1.2463404195795527e12  \\
            0.20756205565991895  9.996340515554292e11  \\
            0.20851856282885867  8.07732717236578e11  \\
            0.20947506999779839  6.572155446506666e11  \\
            0.2104315771667381  5.382353744394856e11  \\
            0.21138808433567782  4.434949815971707e11  \\
            0.21234459150461754  3.6753622210994934e11  \\
            0.21330109867355726  3.062406434332074e11  \\
            0.214257605842497  2.5647503694394843e11  \\
            0.21521411301143673  2.158422266682666e11  \\
            0.21617062018037644  1.824714791376804e11  \\
            0.21712712734931616  1.549356813520545e11  \\
            0.21808363451825588  7.360678436284402e11  \\
            0.2190401416871956  7.0902734951026e12  \\
            0.21999664885613532  3.693506569177321e14  \\
            0.22095315602507504  8.374085347065341e10  \\
            0.22190966319401476  7.244225590590582e10  \\
            0.22286617036295447  5.382080475934783e18  \\
            0.2238226775318942  3.012250128709419e16  \\
            0.2247791847008339  1.7838811261471512e15  \\
            0.2257356918697736  2.4962965496134753e14  \\
            0.22669219903871332  5.483273524643095e13  \\
            0.22764870620765304  1.5910806985432812e13  \\
            0.22860521337659276  5.578137813179613e12  \\
            0.22956172054553248  2.242445083580996e12  \\
            0.2305182277144722  9.998296929809229e11  \\
            0.23147473488341191  4.8340932694167505e11  \\
            0.23243124205235163  2.4936483909064435e11  \\
            0.23338774922129135  1.3565838803183849e11  \\
            0.23434425639023107  7.712888740616837e10  \\
            0.2353007635591708  4.550811971692307e10  \\
            0.2362572707281105  2.7708736620728508e10  \\
            0.23721377789705023  1.7329988042644665e10  \\
            0.23817028506598995  1.1090283414452368e10  \\
            0.23912679223492966  1.1090283414139172e10  \\
            0.24008329940386938  1.7329988043978703e10  \\
            0.2410398065728091  2.7708736620775986e10  \\
            0.24199631374174882  4.5508119714886406e10  \\
            0.24295282091068854  7.71288873981817e10  \\
            0.24390932807962826  1.3565838803464e11  \\
            0.24486583524856798  2.4936483914891165e11  \\
            0.2458223424175077  4.8340932651965204e11  \\
            0.2467788495864474  9.998296922018705e11  \\
            0.24773535675538713  2.242445084487568e12  \\
            0.24869186392432685  5.578137795680781e12  \\
            0.24964837109326657  1.5910806973751203e13  \\
            0.2506048782622063  5.483273572025131e13  \\
            0.25156138543114603  2.4962966509762903e14  \\
            0.2525178926000857  1.7838811608314235e15  \\
            0.25347439976902547  3.012250690553355e16  \\
            0.25443090693796516  5.382103226247797e18  \\
            0.2553874141069049  7.244225588394377e10  \\
            0.2563439212758446  8.37408534530838e10  \\
            0.25730042844478435  3.693506341769237e14  \\
            0.25825693561372404  7.090273500198148e12  \\
            0.2592134427826638  7.360678442293765e11  \\
            0.2601699499516035  1.549356813179392e11  \\
            0.2611264571205432  1.824714791204574e11  \\
            0.2620829642894829  2.1584222671034622e11  \\
            0.26303947145842266  2.5647503681358554e11  \\
            0.26399597862736235  3.0624064364909045e11  \\
            0.2649524857963021  3.675362217833041e11  \\
            0.2659089929652418  4.434949810588284e11  \\
            0.26686550013418153  5.3823537376505853e11  \\
            0.2678220073031212  6.572155445410918e11  \\
            0.26877851447206097  8.077327154815587e11  \\
            0.26973502164100066  9.996340504743337e11  \\
            0.2706915288099404  1.2463404167783813e12  \\
            0.2716480359788801  3.8773349413164474e19  \\
            0.27260454314781984  3.911423705650689e15  \\
            0.27356105031675954  1.3732230841341134e14  \\
            0.2745175574856993  1.7217932201631633e13  \\
            0.275474064654639  4.2857075890063604e12  \\
            0.2764305718235787  5.662963799629337e12  \\
            0.27738707899251847  7.579033508704118e12  \\
            0.27834358616145816  1.2621765486422367e13  \\
            0.2793000933303979  4.047887447052634e13  \\
            0.2802566004993376  1.6654178963773938e14  \\
            0.28121310766827734  1.0167616394084674e15  \\
            0.28216961483721703  1.2797029614236274e16  \\
            0.2831261220061568  9.63620661912073e17  \\
            0.28408262917509647  9.497218900385217e13  \\
            0.2850391363440362  1.502950593301985e14  \\
            0.2859956435129759  1.3107142470685385e19  \\
            0.28695215068191565  3.934247711412966e16  \\
            0.28790865785085534  2.46916462134663e15  \\
            0.2888651650197951  1.4363251751567542e15  \\
            0.2898216721887348  2.9334474137693165e15  \\
            0.2907781793576745  6.549086097695477e15  \\
            0.29173468652661416  1.6392788635041996e16  \\
            0.2926911936955539  4.778580170326555e16  \\
            0.2936477008644936  1.724173093563409e17  \\
            0.29460420803343335  8.572224728146414e17  \\
            0.29556071520237304  7.302184110153609e18  \\
            0.2965172223713128  1.8809632043395973e20  \\
            0.2974737295402525  1.0e23  \\
            0.2984302367091922  5.715665868782766e11  \\
            0.2993867438781319  4.838357631939054e11  \\
            0.30034325104707166  4.6160657645445184e11  \\
            0.3012997582160114  5.747792891894912e11  \\
            0.3022562653849511  8.205002500504387e11  \\
            0.30321277255389084  9.960006894733274e11  \\
            0.30416927972283053  3.3114390668372754e11  \\
            0.3051257868917703  1.1320020836855856e11  \\
            0.30608229406070997  5.618840146481923e10  \\
            0.3070388012296497  3.202327764537808e10  \\
            0.3079953083985894  1.9700788460485195e10  \\
            0.30895181556752915  1.2744783631959135e10  \\
            0.30990832273646884  8.550554335050983e9  \\
            0.3108648299054086  5.899719870844479e9  \\
            0.3118213370743483  4.1633357443354e9  \\
            0.31277784424328803  2.9931319063571534e9  \\
            0.3137343514122277  2.185937743780786e9  \\
            0.31469085858116747  3.171199303433156e21  \\
            0.31564736575010716  8.278916507534013e16  \\
            0.3166038729190469  2.7206017795135745e15  \\
            0.3175603800879866  3.637650134421436e14  \\
            0.31851688725692634  9.318011292889261e13  \\
            0.31947339442586603  3.627009841759382e13  \\
            0.3204299015948058  4.767123935370036e18  \\
            0.32138640876374547  2.7177460178967964e16  \\
            0.3223429159326852  1.665670658654935e15  \\
            0.3232994231016249  2.399405991552644e14  \\
            0.32425593027056465  5.3970220281856445e13  \\
            0.32521243743950434  1.597148351345897e13  \\
            0.3261689446084441  5.693203333795265e12  \\
            0.32712545177738384  2.321696959633732e12  \\
            0.3280819589463235  1.0483508325636621e12  \\
            0.3290384661152633  1.3305916731647246e12  \\
            0.32999497328420296  3.225042200001061e12  \\
            0.3309514804531427  8.848991054252979e12  \\
            0.3319079876220824  2.8715387859630863e13  \\
            0.33286449479102215  1.1836062973569095e14  \\
            0.33382100195996184  7.048961892758899e14  \\
            0.3347775091289016  7.957705711363891e15  \\
            0.3357340162978413  3.6499492015346675e17  \\
            0.336690523466781  3.9641470631418837e21  \\
            0.3376470306357207  5.47453037693907e9  \\
            0.33860353780466046  3.708072848927601e9  \\
            0.33956004497360015  2.5398884023105364e9  \\
            0.3405165521425399  1.7547541132923229e9  \\
            0.3414730593114796  1.219383455641906e9  \\
            0.34242956648041933  8.4954503142732e8  \\
            0.343386073649359  5.910210646634351e8  \\
            0.34434258081829877  4.083196704064087e8  \\
            0.3452990879872384  2.778405651850825e8  \\
            0.34625559515617815  2.968927299442431e8  \\
            0.34721210232511784  3.2499944291563976e8  \\
            0.3481686094940576  5.354018906750611e8  \\
            0.3491251166629973  4.032364327089749e8  \\
            0.350081623831937  4.521549682928944e8  \\
            0.3510381310008768  5.194796928264311e8  \\
            0.35199463816981647  6.117009801018426e8  \\
            0.3529511453387562  7.446768011191039e8  \\
            0.3539076525076959  9.522841998868822e8  \\
            0.35486415967663565  1.320461865104572e9  \\
            0.35582066684557534  2.1476308391377506e9  \\
            0.3567771740145151  5.5160671130117035e9  \\
            0.3577336811834548  1.3898858246210106e10  \\
            0.3586901883523945  2.123992157921117e10  \\
            0.3596466955213342  3.10702971164347e9  \\
            0.36060320269027396  1.6363779013641076e9  \\
            0.36155970985921365  1.1066240819771802e9  \\
            0.3625162170281534  8.357410955841044e8  \\
            0.3634727241970931  6.716009036298727e8  \\
            0.36442923136603284  5.617431389073638e8  \\
            0.3653857385349725  4.8332223551624143e8  \\
            0.3663422457039123  4.25312496854842e8  \\
            0.36729875287285196  3.887958542583211e9  \\
            0.3682552600417917  3.4168134770167744e8  \\
            0.3692117672107314  3.102367850581196e8  \\
            0.37016827437967115  2.8355193168983305e8  \\
            0.37112478154861084  3.376907551058403e8  \\
            0.3720812887175506  4.9188722049709105e8  \\
            0.3730377958864903  7.08929626693751e8  \\
            0.37399430305543  1.0177047421845659e9  \\
            0.3749508102243697  1.4619906262273726e9  \\
            0.37590731739330946  2.109217513427814e9  \\
            0.37686382456224915  3.065088169997933e9  \\
            0.3778203317311889  4.498524920056059e9  \\
            0.37877683890012864  6.685260179669541e9  \\
            0.37973334606906833  9.970995267355378e18  \\
            0.3806898532380081  4.113122970435722e16  \\
            0.38164636040694777  2.126489034617244e15  \\
            0.3826028675758875  2.72604157510071e14  \\
            0.3835593747448272  5.623499540375681e13  \\
            0.38451588191376695  1.5553796955433014e13  \\
            0.38547238908270665  5.247966196937274e12  \\
            0.3864288962516464  2.0438540566218887e12  \\
            0.3873854034205861  8.870065359801608e11  \\
            0.38834191058952583  1.5428090868480955e12  \\
            0.3892984177584655  3.5834668549357544e12  \\
            0.39025492492740527  9.351881736049934e12  \\
            0.39121143209634496  2.8560561596154375e13  \\
            0.3921679392652847  1.0910073094838531e14  \\
            0.3931244464342244  5.880881667366294e14  \\
            0.39408095360316414  5.776844889465489e15  \\
            0.39503746077210383  2.1331947510641267e17  \\
            0.3959939679410436  3.588971256460002e21  \\
            0.39695047510998327  5.551828644731052e13  \\
            0.39790698227892296  1.7319665009492894e14  \\
            0.39886348944786265  8.934365144975569e14  \\
            0.3998199966168024  1.1428840697020432e16  \\
            0.4007765037857421  1.8863076153658985e18  \\
            0.40173301095468184  1.877801240202731e9  \\
            0.4026895181236216  2.5532521633889074e9  \\
            0.4036460252925613  3.5225793622105703e9  \\
            0.404602532461501  4.9434925546541395e9  \\
            0.4055590396304407  7.080448978635124e9  \\
            0.40651554679938046  1.039805904719232e10  \\
            0.40747205396832015  1.5762747519277908e10  \\
            0.4084285611372599  2.49297965725033e10  \\
            0.4093850683061996  4.190993462026292e10  \\
            0.41034157547513933  7.784904240260962e10  \\
            0.411298082644079  1.7832277071639294e11  \\
            0.41225458981301877  1.1024039391407253e12  \\
            0.41321109698195846  2.0669078925443057e11  \\
            0.4141676041508982  9.58510874092565e11  \\
            0.4151241113198379  4.873515057237074e11  \\
            0.41608061848877764  4.635708832868283e11  \\
            0.41703712565771733  5.197635054672267e11  \\
            0.4179936328266571  6.413319935305663e11  \\
            0.41895013999559677  2.507149638829668e21  \\
            0.4199066471645365  3.0175550257703494e19  \\
            0.4208631543334762  2.2887182411473684e18  \\
            0.42181966150241595  3.657711210405116e17  \\
            0.42277616867135565  8.792413321482389e16  \\
            0.4237326758402954  2.7375420190616524e16  \\
            0.4246891830092351  1.0193939692319934e16  \\
            0.42564569017817483  4.3286654056610485e15  \\
            0.4266021973471145  2.0324596329421492e15  \\
            0.42755870451605427  1.0332318644922778e15  \\
            0.428515211684994  8.347171016302455e15  \\
            0.4294717188539337  3.591132837784516e17  \\
            0.43042822602287345  1.0e23  \\
            0.43138473319181314  1.189836511884752e14  \\
            0.4323412403607529  7.70275485538257e19  \\
            0.4332977475296926  7.678951579238325e16  \\
            0.4342542546986323  3.1810287573180515e15  \\
            0.435210761867572  3.863276742989779e14  \\
            0.43616726903651176  7.905020588950702e13  \\
            0.43712377620545145  2.203870776809219e13  \\
            0.4380802833743912  8.813322377371637e12  \\
            0.4390367905433309  6.540412638359799e12  \\
            0.43999329771227064  4.918922339234216e12  \\
            0.4409498048812103  7.709867929344142e12  \\
            0.4419063120501501  4.413350658082104e13  \\
            0.44286281921908976  5.726433814628032e14  \\
            0.4438193263880295  7.331102483261962e16  \\
            0.4447758335569692  1.395770660427463e12  \\
            0.44573234072590895  1.11512152650111e12  \\
            0.44668884789484864  8.977700332702312e11  \\
            0.4476453550637884  7.279864848845486e11  \\
            0.4486018622327281  5.942896070728412e11  \\
            0.4495583694016678  4.8821275052602625e11  \\
            0.4505148765706075  4.0345196636198315e11  \\
            0.4514713837395472  3.352710694579099e11  \\
            0.45242789090848695  2.800814233360137e11  \\
            0.45338439807742664  2.3514235793164914e11  \\
            0.4543409052463664  1.9834396776625415e11  \\
            0.4552974124153061  1.6805173355428244e11  \\
            0.4562539195842458  3.116974854605454e11  \\
            0.4572104267531855  2.0350046020344106e12  \\
            0.45816693392212526  3.588855382002776e13  \\
            0.45912344109106495  2.996174018629903e16  \\
            0.4600799482600047  7.785483773810352e10  \\
            0.4610364554289444  3.5553759915410546e21  \\
            0.46199296259788414  2.4025309010050826e17  \\
            0.46294946976682383  6.291202039713533e15  \\
            0.4639059769357636  6.203310418008186e14  \\
            0.46486248410470327  1.1210196471018256e14  \\
            0.465818991273643  2.8720547905707555e13  \\
            0.4667754984425827  9.23625851210237e12  \\
            0.46773200561152245  3.4851347714121226e12  \\
            0.46868851278046214  1.4804593571298154e12  \\
            0.4696450199494019  6.89042816285205e11  \\
            0.4706015271183416  3.4466393013397577e11  \\
            0.4715580342872813  1.8282061664928934e11  \\
            0.472514541456221  1.0177583977105667e11  \\
            0.47347104862516076  5.899365986999215e10  \\
            0.47442755579410045  3.5381744151352715e10  \\
            0.4753840629630402  2.1845573828555214e10  \\
            0.4763405701319799  1.3826755786590078e10  \\
            0.47729707730091964  8.939008519524015e9  \\
        }
        ;
    \addlegendentry {$(20.0, 0.0, 0.0) $}
    \addplot[color={rgb,1:red,0.0;green,0.0;blue,0.0}, name path={bb9883bc-9c1c-4831-924c-16db6468bf41}, only marks, draw opacity={0.0}, line width={0}, solid, mark={*}, mark size={0.00075 pt}, mark repeat={1}, mark options={color={rgb,1:red,0.0;green,0.0;blue,0.0}, draw opacity={1.0}, fill={rgb,1:red,0.0;green,0.0;blue,0.0}, fill opacity={0.0}, line width={0.75}, rotate={0}, solid}, forget plot]
        table[row sep={\\}]
        {
            \\
            0.0  1.0  \\
        }
        ;
\end{axis}
\end{tikzpicture}

}

\subfloat[]{%
% Recommended preamble:
% \usetikzlibrary{arrows.meta}
% \usetikzlibrary{backgrounds}
% \usepgfplotslibrary{patchplots}
% \usepgfplotslibrary{fillbetween}
% \pgfplotsset{%
%     layers/standard/.define layer set={%
%         background,axis background,axis grid,axis ticks,axis lines,axis tick labels,pre main,main,axis descriptions,axis foreground%
%     }{
%         grid style={/pgfplots/on layer=axis grid},%
%         tick style={/pgfplots/on layer=axis ticks},%
%         axis line style={/pgfplots/on layer=axis lines},%
%         label style={/pgfplots/on layer=axis descriptions},%
%         legend style={/pgfplots/on layer=axis descriptions},%
%         title style={/pgfplots/on layer=axis descriptions},%
%         colorbar style={/pgfplots/on layer=axis descriptions},%
%         ticklabel style={/pgfplots/on layer=axis tick labels},%
%         axis background@ style={/pgfplots/on layer=axis background},%
%         3d box foreground style={/pgfplots/on layer=axis foreground},%
%     },
% }

\begin{tikzpicture}[/tikz/background rectangle/.style={fill={rgb,1:red,1.0;green,1.0;blue,1.0}, fill opacity={1.0}, draw opacity={1.0}}, show background rectangle]
\begin{axis}[point meta max={nan}, point meta min={nan}, legend cell align={left}, legend columns={1}, title={}, title style={at={{(0.5,1)}}, anchor={south}, font={{\fontsize{14 pt}{18.2 pt}\selectfont}}, color={rgb,1:red,0.0;green,0.0;blue,0.0}, draw opacity={1.0}, rotate={0.0}, align={center}}, legend style={color={rgb,1:red,0.0;green,0.0;blue,0.0}, draw opacity={1.0}, line width={1}, solid, fill={rgb,1:red,1.0;green,1.0;blue,1.0}, fill opacity={1.0}, text opacity={1.0}, font={{\fontsize{8 pt}{10.4 pt}\selectfont}}, text={rgb,1:red,0.0;green,0.0;blue,0.0}, cells={anchor={center}}, at={(1.02, 1)}, anchor={north west}}, axis background/.style={fill={rgb,1:red,1.0;green,1.0;blue,1.0}, opacity={1.0}}, anchor={north west}, xshift={1.0mm}, yshift={-1.0mm}, width={120.0mm}, height={74.2mm}, scaled x ticks={false}, xlabel={}, x tick style={color={rgb,1:red,0.0;green,0.0;blue,0.0}, opacity={1.0}}, x tick label style={color={rgb,1:red,0.0;green,0.0;blue,0.0}, opacity={1.0}, rotate={0}}, xlabel style={at={(ticklabel cs:0.5)}, anchor=near ticklabel, at={{(ticklabel cs:0.5)}}, anchor={near ticklabel}, font={{\fontsize{11 pt}{14.3 pt}\selectfont}}, color={rgb,1:red,0.0;green,0.0;blue,0.0}, draw opacity={1.0}, rotate={0.0}}, xmajorgrids={true}, xmin={-0.01649326567117626}, xmax={0.5662687880437169}, xticklabels={{$0.0$,$0.1$,$0.2$,$0.3$,$0.4$,$0.5$}}, xtick={{0.0,0.1,0.2,0.30000000000000004,0.4,0.5}}, xtick align={inside}, xticklabel style={font={{\fontsize{8 pt}{10.4 pt}\selectfont}}, color={rgb,1:red,0.0;green,0.0;blue,0.0}, draw opacity={1.0}, rotate={0.0}}, x grid style={color={rgb,1:red,0.0;green,0.0;blue,0.0}, draw opacity={0.1}, line width={0.5}, solid}, axis x line*={left}, x axis line style={color={rgb,1:red,0.0;green,0.0;blue,0.0}, draw opacity={1.0}, line width={1}, solid}, scaled y ticks={false}, ylabel={}, y tick style={color={rgb,1:red,0.0;green,0.0;blue,0.0}, opacity={1.0}}, y tick label style={color={rgb,1:red,0.0;green,0.0;blue,0.0}, opacity={1.0}, rotate={0}}, ylabel style={at={(ticklabel cs:0.5)}, anchor=near ticklabel, at={{(ticklabel cs:0.5)}}, anchor={near ticklabel}, font={{\fontsize{11 pt}{14.3 pt}\selectfont}}, color={rgb,1:red,0.0;green,0.0;blue,0.0}, draw opacity={1.0}, rotate={0.0}}, ymode={log}, log basis y={10}, ymajorgrids={true}, ymin={0.01568959833291757}, ymax={310.0432599127799}, yticklabels={{$10^{0}$,$10^{2}$}}, ytick={{1.0,100.0}}, ytick align={inside}, yticklabel style={font={{\fontsize{8 pt}{10.4 pt}\selectfont}}, color={rgb,1:red,0.0;green,0.0;blue,0.0}, draw opacity={1.0}, rotate={0.0}}, y grid style={color={rgb,1:red,0.0;green,0.0;blue,0.0}, draw opacity={0.1}, line width={0.5}, solid}, axis y line*={left}, y axis line style={color={rgb,1:red,0.0;green,0.0;blue,0.0}, draw opacity={1.0}, line width={1}, solid}, colorbar={false}]
    [\addlegendimage{empty legend}] \addlegendentry[font={{\fontsize{11 pt}{14.3 pt}\selectfont}}, text={rgb,1:red,0.0;green,0.0;blue,0.0}] {\hspace{-.6cm}{\textbf{$(\gamma, \gamma_1, \gamma_2)$}}}
    \addplot[color={rgb,1:red,0.0;green,0.0;blue,1.0}, name path={2c1fd526-2e49-4f87-902f-6efdd7440f87}, draw opacity={1.0}, line width={1}, solid]
        table[row sep={\\}]
        {
            \\
            0.0  0.020917285515940576  \\
            0.001101754553852787  0.020923277405693396  \\
            0.002203509107705574  0.02095259884817224  \\
            0.0033052636615583607  0.021210608859436497  \\
            0.004407018215411148  0.02092872126936171  \\
            0.005508772769263934  0.021143004310123593  \\
            0.0066105273231167215  0.021657483547878085  \\
            0.007712281876969509  0.021047325950646737  \\
            0.008814036430822295  0.020939386664037805  \\
            0.009915790984675084  0.02094833507852258  \\
            0.011017545538527868  0.02098046716597757  \\
            0.012119300092380656  0.020966902138977804  \\
            0.013221054646233443  0.020969023525817948  \\
            0.01432280920008623  0.020990140280396672  \\
            0.015424563753939018  0.02106022175816107  \\
            0.016526318307791804  0.02335626532772011  \\
            0.01762807286164459  0.02085337015470323  \\
            0.018729827415497377  0.020895203482808405  \\
            0.019831581969350167  0.020904072231252874  \\
            0.020933336523202953  0.020906705973777424  \\
            0.022035091077055736  0.020908294291033927  \\
            0.023136845630908526  0.0209103108091734  \\
            0.024238600184761313  0.020913116463502447  \\
            0.0253403547386141  0.020916549185172044  \\
            0.026442109292466886  0.020920018430185403  \\
            0.027543863846319676  0.020922342418639345  \\
            0.02864561840017246  0.020950233234048782  \\
            0.029747372954025245  0.02095768456300192  \\
            0.030849127507878035  0.020966157600381854  \\
            0.03195088206173082  0.020975955762615434  \\
            0.03305263661558361  0.020987617606710505  \\
            0.034154391169436395  0.021001888193277793  \\
            0.03525614572328918  0.02101943406506624  \\
            0.03635790027714197  0.02104005558664877  \\
            0.037459654830994754  0.02106120080396599  \\
            0.03856140938484754  0.02107619545345478  \\
            0.039663163938700334  0.021070832791045577  \\
            0.04076491849255312  0.021085302505939124  \\
            0.04186667304640591  0.02110054728437575  \\
            0.042968427600258687  0.021116657400902558  \\
            0.04407018215411147  0.021133728373512916  \\
            0.045171936707964266  0.021151858421404005  \\
            0.04627369126181705  0.021171143791028014  \\
            0.04737544581566984  0.021191671043068953  \\
            0.048477200369522626  0.021213505192713512  \\
            0.04957895492337542  0.02123667254849131  \\
            0.0506807094772282  0.021261137323978344  \\
            0.051782464031080985  0.021247445603706366  \\
            0.05288421858493377  0.02118907609024813  \\
            0.05398597313878656  0.021283487207684988  \\
            0.05508772769263935  0.07868225329624191  \\
            0.05618948224649214  0.021483023951101833  \\
            0.05729123680034492  0.021240827344540438  \\
            0.058392991354197704  0.02116838859109928  \\
            0.05949474590805049  0.02113131557077329  \\
            0.060596500461903284  0.021108329273390517  \\
            0.06169825501575607  0.02109302777401753  \\
            0.06280000956960886  0.021082803813534758  \\
            0.06390176412346164  0.021076421914151513  \\
            0.06500351867731442  0.021073287069480157  \\
            0.06610527323116722  0.02107318800894597  \\
            0.06720702778502  0.02107622006265468  \\
            0.06830878233887279  0.021082811379266257  \\
            0.06941053689272557  0.021093858232740844  \\
            0.07051229144657836  0.021111050853871754  \\
            0.07161404600043114  0.021137635801904656  \\
            0.07271580055428394  0.021180378659072484  \\
            0.07381755510813673  0.021255559825779383  \\
            0.07491930966198951  0.021413079233555725  \\
            0.0760210642158423  0.021896587657047988  \\
            0.07712281876969508  0.02791116389996988  \\
            0.07822457332354787  0.02142118321888716  \\
            0.07932632787740067  0.020758303098044748  \\
            0.08042808243125345  0.0207752413421832  \\
            0.08152983698510624  0.020805972295152263  \\
            0.08263159153895902  0.020828699044193513  \\
            0.08373334609281181  0.02084342713247511  \\
            0.0848351006466646  0.020850396280801094  \\
            0.08593685520051737  0.020847466101603263  \\
            0.08703860975437017  0.020826458705156283  \\
            0.08814036430822295  0.020761536864284946  \\
            0.08924211886207574  0.033525536447638296  \\
            0.09034387341592853  0.021306972685231986  \\
            0.09144562796978131  0.02114835729291415  \\
            0.0925473825236341  0.021108513893177923  \\
            0.09364913707748689  0.021096238862374352  \\
            0.09475089163133968  0.021094849106071203  \\
            0.09585264618519247  0.021099142559731218  \\
            0.09695440073904525  0.021107087513401953  \\
            0.09805615529289805  0.021117871471949972  \\
            0.09915790984675084  0.021131339393211417  \\
            0.1002596644006036  0.02113892981165414  \\
            0.1013614189544564  0.021158806005604793  \\
            0.10246317350830918  0.02118680697282225  \\
            0.10356492806216197  0.0212372517050231  \\
            0.10466668261601475  0.021404850153481574  \\
            0.10576843716986754  0.07521564401671728  \\
            0.10687019172372034  0.02114373593988083  \\
            0.10797194627757312  0.021153055106389405  \\
            0.10907370083142591  0.0211782475947956  \\
            0.1101754553852787  0.021202103511898714  \\
            0.11127720993913148  0.021224968600177946  \\
            0.11237896449298428  0.021247498292942646  \\
            0.11348071904683706  0.02127009608264589  \\
            0.11458247360068984  0.021292990918206706  \\
            0.11568422815454263  0.021316305983418923  \\
            0.11678598270839541  0.02134009691674261  \\
            0.1178877372622482  0.021364371609204798  \\
            0.11898949181610098  0.021389099735920722  \\
            0.12009124636995377  0.021414216227719735  \\
            0.12119300092380657  0.02143962074619351  \\
            0.12229475547765935  0.021465174129362944  \\
            0.12339651003151214  0.02149069242242062  \\
            0.12449826458536492  0.02151593950820229  \\
            0.12560001913921773  0.021540621851650843  \\
            0.1267017736930705  0.021564398750274404  \\
            0.1278035282469233  0.021586966013327286  \\
            0.12890528280077607  0.021608540919649043  \\
            0.13000703735462885  0.02163423631239983  \\
            0.13110879190848163  0.021679665132729644  \\
            0.13221054646233443  0.021605921213478884  \\
            0.1333123010161872  0.021600328485052042  \\
            0.13441405557004  0.021571325749666716  \\
            0.1355158101238928  0.021646545342081793  \\
            0.13661756467774558  0.02156918814351828  \\
            0.13771931923159836  0.021696875173416556  \\
            0.13882107378545114  0.021616674204910907  \\
            0.13992282833930392  0.021545143270674635  \\
            0.14102458289315672  0.021588736191582775  \\
            0.1421263374470095  0.02160579775905835  \\
            0.14322809200086228  0.02159626610396467  \\
            0.1443298465547151  0.02166925590120548  \\
            0.14543160110856787  0.021619662641954543  \\
            0.14653335566242065  0.021597658172249748  \\
            0.14763511021627346  0.021575722671790067  \\
            0.14873686477012624  0.021552540947288052  \\
            0.14983861932397902  0.02152828159601507  \\
            0.1509403738778318  0.02150328859082469  \\
            0.1520421284316846  0.0214778830519626  \\
            0.15314388298553738  0.02145232997805193  \\
            0.15424563753939016  0.02142683877560054  \\
            0.15534739209324297  0.021401570361699148  \\
            0.15644914664709575  0.02137664405758331  \\
            0.15755090120094853  0.02135214244820222  \\
            0.15865265575480134  0.02132811352665496  \\
            0.15975441030865412  0.02130456941831492  \\
            0.1608561648625069  0.021281480271912425  \\
            0.1619579194163597  0.021258760383297916  \\
            0.16305967397021248  0.021236240669539195  \\
            0.16416142852406526  0.021213616936095418  \\
            0.16526318307791804  0.021190366924983255  \\
            0.16636493763177085  0.02116578338470482  \\
            0.16746669218562363  0.021142054824173806  \\
            0.16856844673947638  0.0212818724526968  \\
            0.1696702012933292  0.02191009488169024  \\
            0.17077195584718197  0.021288387375493465  \\
            0.17187371040103475  0.021207184341043316  \\
            0.17297546495488753  0.021171195395535112  \\
            0.17407721950874033  0.021148089287303287  \\
            0.1751789740625931  0.021130384892525347  \\
            0.1762807286164459  0.021124093554885516  \\
            0.1773824831702987  0.021111992582042167  \\
            0.17848423772415148  0.02110258277813746  \\
            0.17958599227800426  0.021096316441157916  \\
            0.18068774683185707  0.021094496998841108  \\
            0.18178950138570985  0.021100322563128708  \\
            0.18289125593956262  0.021122736864359483  \\
            0.18399301049341543  0.021196962581315978  \\
            0.1850947650472682  0.021683982876173912  \\
            0.186196519601121  0.02079253967273185  \\
            0.18729827415497377  0.020801943262244705  \\
            0.18840002870882658  0.02084009449312762  \\
            0.18950178326267936  0.02085035964909163  \\
            0.19060353781653214  0.020847847828991538  \\
            0.19170529237038494  0.020836935550112925  \\
            0.19280704692423772  0.020818332045231997  \\
            0.1939088014780905  0.02079135851066556  \\
            0.1950105560319433  0.020760218344025257  \\
            0.1961123105857961  0.02082802233093389  \\
            0.19721406513964887  0.03791535240185756  \\
            0.19831581969350168  0.022766260071455366  \\
            0.19941757424735443  0.021574237721866563  \\
            0.2005193288012072  0.021317515847841445  \\
            0.20162108335506  0.021211951568540333  \\
            0.2027228379089128  0.021156149416933105  \\
            0.20382459246276557  0.02112271655069878  \\
            0.20492634701661835  0.021101397944548975  \\
            0.20602810157047116  0.021087572795423202  \\
            0.20712985612432394  0.021078915188233203  \\
            0.20823161067817672  0.02107419156947027  \\
            0.2093333652320295  0.02107276590202478  \\
            0.2104351197858823  0.021074386018616786  \\
            0.2115368743397351  0.02107910762160728  \\
            0.21263862889358787  0.02108731647573912  \\
            0.21374038344744067  0.021099878049625057  \\
            0.21484213800129345  0.021118567652978686  \\
            0.21594389255514623  0.02114736393224503  \\
            0.21704564710899904  0.021197361870918943  \\
            0.21814740166285182  0.021315942585360386  \\
            0.2192491562167046  0.022137734464122805  \\
            0.2203509107705574  0.02165397457118227  \\
            0.2214526653244102  0.02118014417133343  \\
            0.22255441987826297  0.021214762384232984  \\
            0.22365617443211575  0.02128563866282979  \\
            0.22475792898596855  0.02124850543433208  \\
            0.22585968353982133  0.021224664142040734  \\
            0.2269614380936741  0.021202152149917365  \\
            0.22806319264752692  0.021180970192922356  \\
            0.22916494720137967  0.02116106948885145  \\
            0.23026670175523245  0.021142371935204074  \\
            0.23136845630908526  0.021124784477336508  \\
            0.23247021086293804  0.02110820874341889  \\
            0.23357196541679082  0.021092547103575023  \\
            0.2346737199706436  0.02107770616906461  \\
            0.2357754745244964  0.0210805793357639  \\
            0.23687722907834918  0.02107002142469121  \\
            0.23797898363220196  0.021050831215955187  \\
            0.23908073818605477  0.021029389380270564  \\
            0.24018249273990755  0.021010165275495988  \\
            0.24128424729376033  0.02099430516999374  \\
            0.24238600184761314  0.020981434999864112  \\
            0.24348775640146592  0.02097078732747522  \\
            0.2445895109553187  0.020961705246682914  \\
            0.24569126550917147  0.02095377226804009  \\
            0.24679302006302428  0.020922216184489976  \\
            0.24789477461687706  0.02092122416863899  \\
            0.24899652917072984  0.020918134613437923  \\
            0.25009828372458265  0.020914588363133816  \\
            0.25120003827843546  0.02091143372867306  \\
            0.2523017928322882  0.020909042150213903  \\
            0.253403547386141  0.020907345966840424  \\
            0.25450530193999377  0.020905561320934567  \\
            0.2556070564938466  0.020900980735142138  \\
            0.2567088110476994  0.02088274040454783  \\
            0.25781056560155213  0.020768001561791555  \\
            0.25891232015540494  0.027924588342842623  \\
            0.2600140747092577  0.02101411235992611  \\
            0.2611158292631105  0.020976165686923245  \\
            0.26221758381696325  0.020966098042338368  \\
            0.26331933837081606  0.023685083087616236  \\
            0.26442109292466887  0.020964750074802068  \\
            0.2655228474785216  0.02093912742760571  \\
            0.2666246020323744  0.020957958013585305  \\
            0.26772635658622723  0.034818497849062276  \\
            0.26882811114008  0.03167673858175101  \\
            0.2699298656939328  0.020937017209808017  \\
            0.2710316202477856  0.320224786103316  \\
            0.27213337480163835  0.021001243980118902  \\
            0.27323512935549116  0.02093285888698676  \\
            0.27433688390934396  0.020918666817066136  \\
            0.2754386384631967  0.020918671287433337  \\
            0.27654039301704947  0.020932870991145137  \\
            0.2776421475709023  0.02100124791560271  \\
            0.2787439021247551  0.320102530309526  \\
            0.27984565667860783  0.02093701337779468  \\
            0.28094741123246064  0.0316739382415783  \\
            0.28204916578631345  0.03481864044181153  \\
            0.2831509203401662  0.02095799474339919  \\
            0.284252674894019  0.02093919450474277  \\
            0.2853544294478718  0.02096484446124673  \\
            0.28645618400172457  0.02368681307505564  \\
            0.2875579385555774  0.02096627041532929  \\
            0.2886596931094302  0.020976347808029136  \\
            0.28976144766328293  0.021014322034997267  \\
            0.29086320221713574  0.027957055996264153  \\
            0.29196495677098855  0.020767890804895998  \\
            0.2930667113248413  0.020882801955969664  \\
            0.2941684658786941  0.02090108796306556  \\
            0.2952702204325469  0.020905695419550077  \\
            0.29637197498639967  0.020907499453453913  \\
            0.2974737295402525  0.020909211000626964  \\
            0.2985754840941052  0.02091161533046061  \\
            0.29967723864795803  0.020914780633862626  \\
            0.30077899320181084  0.0209183354832666  \\
            0.3018807477556636  0.02092143116281089  \\
            0.3029825023095164  0.02092242563304082  \\
            0.3040842568633692  0.0209538424836551  \\
            0.30518601141722196  0.020961779361700638  \\
            0.30628776597107477  0.020970863296355603  \\
            0.3073895205249276  0.020981510453851607  \\
            0.3084912750787803  0.02099437718564571  \\
            0.30959302963263313  0.021010230091792745  \\
            0.31069478418648594  0.0210294424112827  \\
            0.3117965387403387  0.021050868431929864  \\
            0.3128982932941915  0.021070044731345206  \\
            0.3140000478480443  0.021080621625835596  \\
            0.31510180240189706  0.021077976165091387  \\
            0.31620355695574986  0.02109282251182264  \\
            0.31730531150960267  0.021108488307066343  \\
            0.3184070660634554  0.02112506666253365  \\
            0.31950882061730823  0.02114265487845929  \\
            0.32061057517116104  0.021161350937848086  \\
            0.3217123297250138  0.02118124744077201  \\
            0.3228140842788666  0.021202421961843698  \\
            0.3239158388327194  0.021224922682000845  \\
            0.32501759338657216  0.021248748194430796  \\
            0.32611934794042496  0.02128555271599915  \\
            0.3272211024942777  0.02121473496462329  \\
            0.3283228570481305  0.02118014416693206  \\
            0.32942461160198333  0.02165399727551033  \\
            0.3305263661558361  0.02213746736581705  \\
            0.3316281207096889  0.021315892991643413  \\
            0.3327298752635417  0.021197356071911207  \\
            0.33383162981739445  0.021147382000155677  \\
            0.33493338437124726  0.021118603039125657  \\
            0.3360351389251  0.02109992758677041  \\
            0.33713689347895276  0.021087378309774127  \\
            0.33823864803280557  0.021079180570163038  \\
            0.3393404025866584  0.02107446932708979  \\
            0.3404421571405111  0.02107285914877982  \\
            0.34154391169436393  0.021074294641512285  \\
            0.34264566624821674  0.021079028316881663  \\
            0.3437474208020695  0.02108769665437511  \\
            0.3448491753559223  0.021101533852660746  \\
            0.34595092990977505  0.021122866886827872  \\
            0.34705268446362786  0.021156318524408625  \\
            0.34815443901748067  0.0212121479892831  \\
            0.3492561935713334  0.02131775908611858  \\
            0.3503579481251862  0.02157458701412206  \\
            0.35145970267903903  0.02276706819028004  \\
            0.3525614572328918  0.037920604400035286  \\
            0.3536632117867446  0.020828066585146097  \\
            0.3547649663405974  0.020760246515702782  \\
            0.35586672089445015  0.020791405597070864  \\
            0.35696847544830296  0.020818395598388217  \\
            0.35807023000215576  0.020837012615742282  \\
            0.3591719845560085  0.020847936711837677  \\
            0.3602737391098613  0.02085045994992876  \\
            0.36137549366371413  0.020840207673260457  \\
            0.3624772482175669  0.020802076052272717  \\
            0.3635790027714197  0.020792744352576283  \\
            0.3646807573252725  0.021684000645892775  \\
            0.36578251187912525  0.02119704963137665  \\
            0.36688426643297806  0.02112284285561472  \\
            0.36798602098683086  0.021100440366224906  \\
            0.3690877755406836  0.021094624462926155  \\
            0.3701895300945364  0.021096452865173927  \\
            0.3712912846483892  0.021102728111727823  \\
            0.372393039202242  0.021112147192200965  \\
            0.3734947937560948  0.021124258183308167  \\
            0.37459654830994754  0.021130529358214755  \\
            0.37569830286380035  0.021148218042445086  \\
            0.37680005741765316  0.02117131586029625  \\
            0.3779018119715059  0.021207304141896952  \\
            0.3790035665253587  0.021288512980143406  \\
            0.3801053210792115  0.02191002985063643  \\
            0.3812070756330643  0.021281214791584846  \\
            0.3823088301869171  0.02114193998071699  \\
            0.3834105847407699  0.0211657397390872  \\
            0.38451233929462264  0.021190349169393653  \\
            0.38561409384847545  0.02121361261450698  \\
            0.38671584840232825  0.021236245000415446  \\
            0.387817602956181  0.021258771261063205  \\
            0.3889193575100338  0.021281496756650156  \\
            0.3900211120638866  0.021304591157961544  \\
            0.3911228666177394  0.02132814050265835  \\
            0.3922246211715922  0.02135217485649121  \\
            0.393326375725445  0.02137668225629826  \\
            0.39442813027929774  0.021401614852196313  \\
            0.39552988483315055  0.021426890211671462  \\
            0.39663163938700335  0.021452389193806102  \\
            0.39773339394085605  0.021477951109653632  \\
            0.39883514849470886  0.021503366851298503  \\
            0.3999369030485616  0.02152837181602946  \\
            0.4010386576024144  0.02155264542814812  \\
            0.4021404121562672  0.021575844511289743  \\
            0.40324216671012  0.02159780184265595  \\
            0.4043439212639728  0.021619836251709965  \\
            0.4054456758178256  0.02166949780198956  \\
            0.40654743037167834  0.02159643582936438  \\
            0.40764918492553115  0.021606036285290484  \\
            0.40875093947938396  0.021589023871834975  \\
            0.4098526940332367  0.021545506929645245  \\
            0.4109544485870895  0.02161706033290978  \\
            0.4120562031409423  0.021696830228124724  \\
            0.4131579576947951  0.021568814690025716  \\
            0.4142597122486479  0.02164619442240607  \\
            0.4153614668025007  0.021571010207829658  \\
            0.41646322135635344  0.021600066403315778  \\
            0.41756497591020625  0.02160571106810497  \\
            0.418666730464059  0.02167985698112634  \\
            0.4197684850179118  0.0216340401630562  \\
            0.4208702395717646  0.02160838375801579  \\
            0.42197199412561737  0.02158683395092585  \\
            0.4230737486794702  0.021564286042581763  \\
            0.424175503233323  0.021540524830928977  \\
            0.42527725778717573  0.021515855517662847  \\
            0.42637901234102854  0.021490619455061314  \\
            0.42748076689488135  0.021465110642324526  \\
            0.4285825214487341  0.02143956553721427  \\
            0.4296842760025869  0.02141416835688892  \\
            0.4307860305564397  0.02138905846297824  \\
            0.43188778511029247  0.02136433635965986  \\
            0.4329895396641453  0.021340067260121876  \\
            0.4340912942179981  0.021316281638405613  \\
            0.43519304877185083  0.021292971786302895  \\
            0.43629480332570364  0.021270082328705347  \\
            0.43739655787955645  0.02124749051807431  \\
            0.4384983124334092  0.021224968219908658  \\
            0.439600066987262  0.021202113669441358  \\
            0.4407018215411148  0.021178275811120403  \\
            0.44180357609496756  0.021153123702695845  \\
            0.4429053306488204  0.021143959318076066  \\
            0.4440070852026732  0.07525378783478039  \\
            0.44510883975652593  0.02140473290051358  \\
            0.44621059431037874  0.021237129275322373  \\
            0.4473123488642315  0.021186687840785538  \\
            0.4484141034180843  0.021158682314260203  \\
            0.4495158579719371  0.0211387941343406  \\
            0.45061761252578986  0.021131169342242278  \\
            0.45171936707964266  0.021117711970455547  \\
            0.45282112163349547  0.021106937610330667  \\
            0.4539228761873482  0.021099001703411894  \\
            0.45502463074120103  0.02109471712440521  \\
            0.45612638529505384  0.02109611608145847  \\
            0.4572281398489066  0.021108401567670083  \\
            0.45832989440275934  0.021148259190456098  \\
            0.45943164895661215  0.021306904622250676  \\
            0.4605334035104649  0.03352670188798115  \\
            0.4616351580643177  0.02076138472227477  \\
            0.4627369126181705  0.020826337270860484  \\
            0.46383866717202327  0.02084735972230207  \\
            0.4649404217258761  0.020850301728847602  \\
            0.4660421762797288  0.02084334402478836  \\
            0.46714393083358163  0.02082862844293871  \\
            0.46824568538743444  0.020805916537897037  \\
            0.4693474399412872  0.020775203737184597  \\
            0.47044919449514  0.020758280063276742  \\
            0.4715509490489928  0.02142092009478608  \\
            0.47265270360284556  0.027908617888134844  \\
            0.47375445815669837  0.02189611077584631  \\
            0.4748562127105512  0.02141279587808422  \\
            0.4759579672644039  0.021255343692463118  \\
            0.47705972181825673  0.021180197390184476  \\
            0.47816147637210954  0.021137476806474893  \\
            0.4792632309259623  0.021110908121617904  \\
            0.4803649854798151  0.021093728567120742  \\
            0.4814667400336679  0.0210826930033661  \\
            0.48256849458752066  0.02107611201538616  \\
            0.48367024914137347  0.021073089855663852  \\
            0.4847720036952263  0.021073198758177057  \\
            0.485873758249079  0.021076343713018808  \\
            0.48697551280293183  0.021082736304738316  \\
            0.48807726735678464  0.02109297190685268  \\
            0.4891790219106374  0.021108286520808935  \\
            0.4902807764644902  0.02113128831708373  \\
            0.49138253101834295  0.021168381244516928  \\
            0.49248428557219576  0.021240850573392575  \\
            0.49358604012604856  0.021483124338736555  \\
            0.4946877946799013  0.07868957587532484  \\
            0.4957895492337541  0.02128348310742274  \\
            0.49689130378760693  0.02118908935739027  \\
            0.4979930583414597  0.021247487217152854  \\
            0.4990948128953125  0.02126090436922863  \\
            0.5001965674491653  0.02123642129068222  \\
            0.501298322003018  0.02121324049631405  \\
            0.5024000765568709  0.02119139707271736  \\
            0.5035018311107237  0.021170864072704482  \\
            0.5046035856645764  0.021151575917021662  \\
            0.5057053402184292  0.02113344555398223  \\
            0.506807094772282  0.021116376315570656  \\
            0.5079088493261348  0.021100269625243796  \\
            0.5090106038799875  0.0210850296624403  \\
            0.5101123584338404  0.021070565898111175  \\
            0.5112141129876931  0.021076171844257023  \\
            0.5123158675415459  0.021061171592926562  \\
            0.5134176220953988  0.02104001012644688  \\
            0.5145193766492515  0.021019374542255907  \\
            0.5156211312031043  0.021001819247004734  \\
            0.5167228857569571  0.020987543460420987  \\
            0.5178246403108099  0.020975879726997788  \\
            0.5189263948646626  0.02096608228669032  \\
            0.5200281494185154  0.02095761215571924  \\
            0.5211299039723682  0.020950165675180765  \\
            0.522231658526221  0.02092213354114296  \\
            0.5233334130800738  0.02091981414574647  \\
            0.5244351676339265  0.02091635234077252  \\
            0.5255369221877794  0.020912929277734738  \\
            0.5266386767416321  0.020910135304333256  \\
            0.5277404312954849  0.02090813273554966  \\
            0.5288421858493377  0.02090656153524039  \\
            0.5299439404031905  0.020903950258044736  \\
            0.5310456949570432  0.02089511520064469  \\
            0.5321474495108961  0.020853354070469145  \\
            0.5332492040647488  0.02335344372677489  \\
            0.5343509586186016  0.0210599449925504  \\
            0.5354527131724545  0.020989949206498902  \\
            0.5365544677263072  0.02096884690419118  \\
            0.53765622228016  0.020966735200146802  \\
            0.5387579768340128  0.020980288372851336  \\
            0.5398597313878656  0.020948254152947945  \\
            0.5409614859417183  0.020939334049179366  \\
            0.5420632404955712  0.021047304196043254  \\
            0.543164995049424  0.02165779260473442  \\
            0.5442667496032767  0.021143022147135114  \\
            0.5453685041571296  0.020928846649758415  \\
            0.5464702587109823  0.02121068977691851  \\
            0.5475720132648351  0.020952585825662107  \\
            0.5486737678186879  0.020923268719708697  \\
            0.5497755223725407  0.020917285516110638  \\
        }
        ;
    \addlegendentry {$(20, 0, 0) $}
    \addplot[color={rgb,1:red,0.0;green,0.0;blue,1.0}, name path={f6c21950-05e0-483b-9d10-ec892bc78bf5}, draw opacity={1.0}, line width={1}, dashed, forget plot]
        table[row sep={\\}]
        {
            \\
            0.0  0.16665010541520522  \\
            0.001101754553852787  0.16666945710467837  \\
            0.002203509107705574  0.16675651276749287  \\
            0.0033052636615583607  0.16734153070040855  \\
            0.004407018215411148  0.16711314880858952  \\
            0.005508772769263934  0.16686869956729297  \\
            0.0066105273231167215  0.16699850796321375  \\
            0.007712281876969509  0.1689598905214798  \\
            0.008814036430822295  0.16668466694018305  \\
            0.009915790984675084  0.16660700239545323  \\
            0.011017545538527868  0.16660752214826471  \\
            0.012119300092380656  0.16668358238506392  \\
            0.013221054646233443  0.1666451988244561  \\
            0.01432280920008623  0.1667106181809403  \\
            0.015424563753939018  0.16738184685745647  \\
            0.016526318307791804  0.258058129209573  \\
            0.01762807286164459  0.16673255249463295  \\
            0.018729827415497377  0.1664166013208665  \\
            0.019831581969350167  0.16633038540266704  \\
            0.020933336523202953  0.16626910685019855  \\
            0.022035091077055736  0.16621412362227603  \\
            0.023136845630908526  0.16616171015278913  \\
            0.024238600184761313  0.1661105001233273  \\
            0.0253403547386141  0.1660593069986587  \\
            0.026442109292466886  0.166006393039424  \\
            0.027543863846319676  0.16594880355960206  \\
            0.02864561840017246  0.1659365875729268  \\
            0.029747372954025245  0.1658999784285305  \\
            0.030849127507878035  0.16586220228399096  \\
            0.03195088206173082  0.16582264101532082  \\
            0.03305263661558361  0.16578054711361015  \\
            0.034154391169436395  0.1657348991762377  \\
            0.03525614572328918  0.16568392556717546  \\
            0.03635790027714197  0.16562365888574132  \\
            0.037459654830994754  0.1655430309785207  \\
            0.03856140938484754  0.16539946149912163  \\
            0.039663163938700334  0.16519238075325868  \\
            0.04076491849255312  0.16516652127969048  \\
            0.04186667304640591  0.16514333937700815  \\
            0.042968427600258687  0.16512305103377214  \\
            0.04407018215411147  0.16510586992208068  \\
            0.045171936707964266  0.16509199829995005  \\
            0.04627369126181705  0.16508161225844611  \\
            0.04737544581566984  0.16507483947923463  \\
            0.048477200369522626  0.16507172756712796  \\
            0.04957895492337542  0.1650722013974929  \\
            0.0506807094772282  0.1650760092079363  \\
            0.051782464031080985  0.16508964286357472  \\
            0.05288421858493377  0.16519913670433717  \\
            0.05398597313878656  0.16606644469558235  \\
            0.05508772769263935  0.3985083227906712  \\
            0.05618948224649214  0.16427505082687374  \\
            0.05729123680034492  0.1641302625710679  \\
            0.058392991354197704  0.16411093395838316  \\
            0.05949474590805049  0.1640935242671908  \\
            0.060596500461903284  0.16407537707046  \\
            0.06169825501575607  0.1640591787009844  \\
            0.06280000956960886  0.1640474594256593  \\
            0.06390176412346164  0.16404258695401197  \\
            0.06500351867731442  0.1640471693385721  \\
            0.06610527323116722  0.16406452644623407  \\
            0.06720702778502  0.16409933501001528  \\
            0.06830878233887279  0.16415869169339226  \\
            0.06941053689272557  0.16425410020707867  \\
            0.07051229144657836  0.16440558078332634  \\
            0.07161404600043114  0.16465113328777028  \\
            0.07271580055428394  0.165071681939807  \\
            0.07381755510813673  0.16587034907808587  \\
            0.07491930966198951  0.16770743979781977  \\
            0.0760210642158423  0.1740610559576244  \\
            0.07712281876969508  0.26260709338779664  \\
            0.07822457332354787  0.18520009917028168  \\
            0.07932632787740067  0.1647536637853987  \\
            0.08042808243125345  0.1628634014317877  \\
            0.08152983698510624  0.1624393954331923  \\
            0.08263159153895902  0.16230332299659242  \\
            0.08373334609281181  0.1622413136902919  \\
            0.0848351006466646  0.1621939688175433  \\
            0.08593685520051737  0.16213957027619233  \\
            0.08703860975437017  0.1620884955402695  \\
            0.08814036430822295  0.162523339430562  \\
            0.08924211886207574  0.3029207992718088  \\
            0.09034387341592853  0.16527235190348047  \\
            0.09144562796978131  0.1637190117039149  \\
            0.0925473825236341  0.16331463538842408  \\
            0.09364913707748689  0.16314895167887872  \\
            0.09475089163133968  0.16307264329513646  \\
            0.09585264618519247  0.16304288404022624  \\
            0.09695440073904525  0.1630451096164483  \\
            0.09805615529289805  0.16307625919790866  \\
            0.09915790984675084  0.16314130022937476  \\
            0.1002596644006036  0.1632416318431117  \\
            0.1013614189544564  0.16344609068186575  \\
            0.10246317350830918  0.1638435939331616  \\
            0.10356492806216197  0.1648245918262831  \\
            0.10466668261601475  0.16919341530002982  \\
            0.10576843716986754  1.2607139639230687  \\
            0.10687019172372034  0.16384393212341042  \\
            0.10797194627757312  0.1622676361031101  \\
            0.10907370083142591  0.16214616687149414  \\
            0.1101754553852787  0.1621671932483585  \\
            0.11127720993913148  0.16220839967742715  \\
            0.11237896449298428  0.16224960508712266  \\
            0.11348071904683706  0.16228726591561443  \\
            0.11458247360068984  0.16232126529575622  \\
            0.11568422815454263  0.1623521613036777  \\
            0.11678598270839541  0.16238053688521617  \\
            0.1178877372622482  0.16240685303278796  \\
            0.11898949181610098  0.1624314308059138  \\
            0.12009124636995377  0.16245446361131927  \\
            0.12119300092380657  0.162476032818397  \\
            0.12229475547765935  0.1624961204924681  \\
            0.12339651003151214  0.1625146207409526  \\
            0.12449826458536492  0.162531358823025  \\
            0.12560001913921773  0.16254614898195754  \\
            0.1267017736930705  0.16255900993931502  \\
            0.1278035282469233  0.16257111329464405  \\
            0.12890528280077607  0.16259052461062262  \\
            0.13000703735462885  0.16270598742303458  \\
            0.13110879190848163  0.18878399201064042  \\
            0.13221054646233443  0.16271145961057878  \\
            0.1333123010161872  0.16254917998536939  \\
            0.13441405557004  0.1624555694551151  \\
            0.1355158101238928  0.16249084313954454  \\
            0.13661756467774558  0.16236158668608688  \\
            0.13771931923159836  0.16249807255227314  \\
            0.13882107378545114  0.16243722807277872  \\
            0.13992282833930392  0.16239864400600168  \\
            0.14102458289315672  0.16249986118638438  \\
            0.1421263374470095  0.16260360875695903  \\
            0.14322809200086228  0.16310104278310936  \\
            0.1443298465547151  0.16334567732754104  \\
            0.14543160110856787  0.16261667498645738  \\
            0.14653335566242065  0.16257792309742794  \\
            0.14763511021627346  0.1625642697710701  \\
            0.14873686477012624  0.16255209890621505  \\
            0.14983861932397902  0.16253830887999132  \\
            0.1509403738778318  0.16252252260788924  \\
            0.1520421284316846  0.16250486947862075  \\
            0.15314388298553738  0.16248554181582076  \\
            0.15424563753939016  0.16246468391060517  \\
            0.15534739209324297  0.16244235965771914  \\
            0.15644914664709575  0.1624185390486014  \\
            0.15755090120094853  0.1623930866328765  \\
            0.15865265575480134  0.1623657471905295  \\
            0.15975441030865412  0.16233613035726313  \\
            0.1608561648625069  0.16230370790265686  \\
            0.1619579194163597  0.16226787547809604  \\
            0.16305967397021248  0.16222826962282946  \\
            0.16416142852406526  0.1621861007853642  \\
            0.16526318307791804  0.16215004940426092  \\
            0.16636493763177085  0.16216823517886775  \\
            0.16746669218562363  0.1625965565340019  \\
            0.16856844673947638  0.17172957193563812  \\
            0.1696702012933292  0.1843040096354133  \\
            0.17077195584718197  0.1660445910359992  \\
            0.17187371040103475  0.1642056955480693  \\
            0.17297546495488753  0.16360610538958112  \\
            0.17407721950874033  0.16332597697244358  \\
            0.1751789740625931  0.1631713397160805  \\
            0.1762807286164459  0.16309966786152041  \\
            0.1773824831702987  0.1630526149962499  \\
            0.17848423772415148  0.1630359442909857  \\
            0.17958599227800426  0.16304865832702098  \\
            0.18068774683185707  0.16309862296549985  \\
            0.18178950138570985  0.16321074566091964  \\
            0.18289125593956262  0.16346171035200613  \\
            0.18399301049341543  0.16418159007102714  \\
            0.1850947650472682  0.16930697287960803  \\
            0.186196519601121  0.1671949073303529  \\
            0.18729827415497377  0.16212371476239434  \\
            0.18840002870882658  0.16210633423215376  \\
            0.18950178326267936  0.16216485117191  \\
            0.19060353781653214  0.16221404856527455  \\
            0.19170529237038494  0.16226448506581384  \\
            0.19280704692423772  0.16235136960166638  \\
            0.1939088014780905  0.1625827543100118  \\
            0.1950105560319433  0.1634273324528767  \\
            0.1961123105857961  0.16855756658816737  \\
            0.19721406513964887  0.44534708527295874  \\
            0.19831581969350168  0.1865736932106372  \\
            0.19941757424735443  0.16974045993233952  \\
            0.2005193288012072  0.16657228099997387  \\
            0.20162108335506  0.16539904133312694  \\
            0.2027228379089128  0.16482891928576807  \\
            0.20382459246276557  0.16451035619483187  \\
            0.20492634701661835  0.16431823313227506  \\
            0.20602810157047116  0.16419799305121152  \\
            0.20712985612432394  0.1641223768270539  \\
            0.20823161067817672  0.16407634760866843  \\
            0.2093333652320295  0.1640509431281003  \\
            0.2104351197858823  0.16404045303728884  \\
            0.2115368743397351  0.164040985902693  \\
            0.21263862889358787  0.16404966065464524  \\
            0.21374038344744067  0.16406408024210273  \\
            0.21484213800129345  0.1640819261454212  \\
            0.21594389255514623  0.1641007380076321  \\
            0.21704564710899904  0.1641195081375656  \\
            0.21814740166285182  0.16416475168332167  \\
            0.2192491562167046  0.16532060698647116  \\
            0.2203509107705574  0.16913006092379107  \\
            0.2214526653244102  0.16542601650007685  \\
            0.22255441987826297  0.16511810701267987  \\
            0.22365617443211575  0.16509684384371978  \\
            0.22475792898596855  0.165069982862126  \\
            0.22585968353982133  0.1650676716926123  \\
            0.2269614380936741  0.16506886181524535  \\
            0.22806319264752692  0.16507370772460972  \\
            0.22916494720137967  0.16508221568583  \\
            0.23026670175523245  0.16509429214106786  \\
            0.23136845630908526  0.16510978073517577  \\
            0.23247021086293804  0.16512848902930774  \\
            0.23357196541679082  0.16515020674262826  \\
            0.2346737199706436  0.1651747174485505  \\
            0.2357754745244964  0.16523753457696308  \\
            0.23687722907834918  0.16548115006156033  \\
            0.23797898363220196  0.16558309228456083  \\
            0.23908073818605477  0.16565121133553046  \\
            0.24018249273990755  0.1657060082287284  \\
            0.24128424729376033  0.16575392942215306  \\
            0.24238600184761314  0.16579755674786925  \\
            0.24348775640146592  0.1658381960902186  \\
            0.2445895109553187  0.16587670516313097  \\
            0.24569126550917147  0.1659137619170011  \\
            0.24679302006302428  0.16590999268155548  \\
            0.24789477461687706  0.16597155638127453  \\
            0.24899652917072984  0.1660262066550879  \\
            0.25009828372458265  0.16607791025163218  \\
            0.25120003827843546  0.16612890003908026  \\
            0.2523017928322882  0.1661805582540367  \\
            0.253403547386141  0.1662340237567564  \\
            0.25450530193999377  0.166291328984472  \\
            0.2556070564938466  0.16636024096214633  \\
            0.2567088110476994  0.16649372874094565  \\
            0.25781056560155213  0.1683824847864863  \\
            0.25891232015540494  0.5046329735263034  \\
            0.2600140747092577  0.16683306503002607  \\
            0.2611158292631105  0.1666617064692562  \\
            0.26221758381696325  0.16664515909479677  \\
            0.26331933837081606  0.29219160371512715  \\
            0.26442109292466887  0.1666108064708464  \\
            0.2655228474785216  0.16662262144071177  \\
            0.2666246020323744  0.1669139181472158  \\
            0.26772635658622723  0.23725013566583322  \\
            0.26882811114008  0.17955044893650915  \\
            0.2699298656939328  0.1664757982520889  \\
            0.2710316202477856  1.0761065600728088  \\
            0.27213337480163835  0.1668844898397161  \\
            0.27323512935549116  0.16669914244719178  \\
            0.27433688390934396  0.1666546235456757  \\
            0.2754386384631967  0.16665464463976207  \\
            0.27654039301704947  0.16669916508331173  \\
            0.2776421475709023  0.16688407215048656  \\
            0.2787439021247551  1.070512559654798  \\
            0.27984565667860783  0.16647520712116906  \\
            0.28094741123246064  0.17952810393307508  \\
            0.28204916578631345  0.2374465450549463  \\
            0.2831509203401662  0.1669134732514904  \\
            0.284252674894019  0.16662238793895287  \\
            0.2853544294478718  0.16661058054449301  \\
            0.28645618400172457  0.290470701636139  \\
            0.2875579385555774  0.166646509248907  \\
            0.2886596931094302  0.16666310943579238  \\
            0.28976144766328293  0.16683432956024374  \\
            0.29086320221713574  0.5047190255645536  \\
            0.29196495677098855  0.16838918319774884  \\
            0.2930667113248413  0.16650108555204057  \\
            0.2941684658786941  0.16636743233254392  \\
            0.2952702204325469  0.16629847645340318  \\
            0.29637197498639967  0.1662411656859764  \\
            0.2974737295402525  0.16618769766090716  \\
            0.2985754840941052  0.16613602251142726  \\
            0.29967723864795803  0.16608499060205917  \\
            0.30077899320181084  0.16603321111833955  \\
            0.3018807477556636  0.16597844226581618  \\
            0.3029825023095164  0.16591670496383368  \\
            0.3040842568633692  0.1659183941922934  \\
            0.30518601141722196  0.16588127245045978  \\
            0.30628776597107477  0.165842687663157  \\
            0.3073895205249276  0.16580196493717894  \\
            0.3084912750787803  0.16575825096091307  \\
            0.30959302963263313  0.16571024541208956  \\
            0.31069478418648594  0.1656553723679496  \\
            0.3117965387403387  0.16558719042560915  \\
            0.3128982932941915  0.16548520388570334  \\
            0.3140000478480443  0.1652416023786069  \\
            0.31510180240189706  0.16517912975144045  \\
            0.31620355695574986  0.1651545821535893  \\
            0.31730531150960267  0.1651328200186051  \\
            0.3184070660634554  0.16511405902429552  \\
            0.31950882061730823  0.16509850858651348  \\
            0.32061057517116104  0.16508636014636624  \\
            0.3217123297250138  0.16507776891536885  \\
            0.3228140842788666  0.16507282715458438  \\
            0.3239158388327194  0.16507152715657722  \\
            0.32501759338657216  0.16507371286745773  \\
            0.32611934794042496  0.1650968439644216  \\
            0.3272211024942777  0.1651152355627392  \\
            0.3283228570481305  0.1654130220947931  \\
            0.32942461160198333  0.1689877585717776  \\
            0.3305263661558361  0.1652251565761992  \\
            0.3316281207096889  0.1641571965057082  \\
            0.3327298752635417  0.16411929591621177  \\
            0.33383162981739445  0.16410245542705748  \\
            0.33493338437124726  0.1640843797958684  \\
            0.3360351389251  0.16406686816076774  \\
            0.33713689347895276  0.16405261181993844  \\
            0.33823864803280557  0.1640440150480328  \\
            0.3393404025866584  0.1640435117599763  \\
            0.3404421571405111  0.16405400014575292  \\
            0.34154391169436393  0.16407937942878242  \\
            0.34264566624821674  0.1641253624142658  \\
            0.3437474208020695  0.16420090975719062  \\
            0.3448491753559223  0.16432105214635076  \\
            0.34595092990977505  0.1645130353571231  \\
            0.34705268446362786  0.1648313886016112  \\
            0.34815443901748067  0.16540116743245237  \\
            0.3492561935713334  0.16657375773390484  \\
            0.3503579481251862  0.1697403315273482  \\
            0.35145970267903903  0.18656614784224557  \\
            0.3525614572328918  0.4452724121061749  \\
            0.3536632117867446  0.1685597612685935  \\
            0.3547649663405974  0.16343112491770861  \\
            0.35586672089445015  0.16258665372484984  \\
            0.35696847544830296  0.16235519760368067  \\
            0.35807023000215576  0.16226820229285752  \\
            0.3591719845560085  0.16221763099169378  \\
            0.3602737391098613  0.16216826066998133  \\
            0.36137549366371413  0.16210949586973064  \\
            0.3624772482175669  0.16212650850042176  \\
            0.3635790027714197  0.16720112983854035  \\
            0.3646807573252725  0.16931987530313306  \\
            0.36578251187912525  0.16418803580334695  \\
            0.36688426643297806  0.16346708638819982  \\
            0.36798602098683086  0.16321569428626626  \\
            0.3690877755406836  0.16310333758765969  \\
            0.3701895300945364  0.16305321841766116  \\
            0.3712912846483892  0.1630403853874667  \\
            0.372393039202242  0.16305694926059452  \\
            0.3734947937560948  0.16310388949381077  \\
            0.37459654830994754  0.16317500070231797  \\
            0.37569830286380035  0.1633289530469772  \\
            0.37680005741765316  0.16360838185482815  \\
            0.3779018119715059  0.1642070606292649  \\
            0.3790035665253587  0.16604388089939717  \\
            0.3801053210792115  0.18428370443685263  \\
            0.3812070756330643  0.17171554990698065  \\
            0.3823088301869171  0.16259654710727683  \\
            0.3834105847407699  0.16216922675928072  \\
            0.38451233929462264  0.1621511988982774  \\
            0.38561409384847545  0.16218724533171247  \\
            0.38671584840232825  0.1622293672234627  \\
            0.387817602956181  0.1622689163396298  \\
            0.3889193575100338  0.16230469250126053  \\
            0.3900211120638866  0.16233706259575273  \\
            0.3911228666177394  0.16236663200209206  \\
            0.3922246211715922  0.1623939291389357  \\
            0.393326375725445  0.1624193443293288  \\
            0.39442813027929774  0.1624431327837618  \\
            0.39552988483315055  0.1624654300864707  \\
            0.39663163938700335  0.16248626655183282  \\
            0.39773339394085605  0.1625055787098392  \\
            0.39883514849470886  0.16252322262754929  \\
            0.3999369030485616  0.16253900578872046  \\
            0.4010386576024144  0.16255279683512658  \\
            0.4021404121562672  0.16256496566770226  \\
            0.40324216671012  0.16257858998665545  \\
            0.4043439212639728  0.16261719188863424  \\
            0.4054456758178256  0.16334537646847988  \\
            0.40654743037167834  0.1631053506678583  \\
            0.40764918492553115  0.16260640379561886  \\
            0.40875093947938396  0.1625026840734496  \\
            0.4098526940332367  0.16240181760865524  \\
            0.4109544485870895  0.1624374432724633  \\
            0.4120562031409423  0.16249773863327846  \\
            0.4131579576947951  0.16236107329850796  \\
            0.4142597122486479  0.1624909198209004  \\
            0.4153614668025007  0.16245260793016866  \\
            0.41646322135635344  0.16254642677974557  \\
            0.41756497591020625  0.16270838137826268  \\
            0.418666730464059  0.1887402604774263  \\
            0.4197684850179118  0.16270571066005393  \\
            0.4208702395717646  0.1625899045223959  \\
            0.42197199412561737  0.16257042561318324  \\
            0.4230737486794702  0.16255831179259853  \\
            0.424175503233323  0.16254545182158342  \\
            0.42527725778717573  0.16253066106358943  \\
            0.42637901234102854  0.1625139169115801  \\
            0.42748076689488135  0.1624954042783615  \\
            0.4285825214487341  0.1624752980761927  \\
            0.4296842760025869  0.16245370462642855  \\
            0.4307860305564397  0.16243064224070833  \\
            0.43188778511029247  0.1624060297712601  \\
            0.4329895396641453  0.16237967386525778  \\
            0.4340912942179981  0.16235125341707976  \\
            0.43519304877185083  0.1623203074603019  \\
            0.43629480332570364  0.16228625354513115  \\
            0.43739655787955645  0.16224853552378984  \\
            0.4384983124334092  0.16220727620882827  \\
            0.439600066987262  0.16216603710611657  \\
            0.4407018215411148  0.16214505967074683  \\
            0.44180357609496756  0.16226692584443117  \\
            0.4429053306488204  0.16384619117318058  \\
            0.4440070852026732  1.2615305708581765  \\
            0.44510883975652593  0.16919749874518317  \\
            0.44621059431037874  0.16482396513176098  \\
            0.4473123488642315  0.1638417221121308  \\
            0.4484141034180843  0.16344345552204934  \\
            0.4495158579719371  0.1632383168245964  \\
            0.45061761252578986  0.16313714327182077  \\
            0.45171936707964266  0.163071979381864  \\
            0.45282112163349547  0.16304072231146013  \\
            0.4539228761873482  0.1630383861863423  \\
            0.45502463074120103  0.16306801235876786  \\
            0.45612638529505384  0.16314413437119188  \\
            0.4572281398489066  0.1633095105478969  \\
            0.45832989440275934  0.16371324685026672  \\
            0.45943164895661215  0.16526443452879608  \\
            0.4605334035104649  0.30280362661339555  \\
            0.4616351580643177  0.16252059589012105  \\
            0.4627369126181705  0.16208550435033994  \\
            0.46383866717202327  0.16213627190284846  \\
            0.4649404217258761  0.16219046641639207  \\
            0.4660421762797288  0.16223766044602542  \\
            0.46714393083358163  0.16229954743125266  \\
            0.46824568538743444  0.1624355235757106  \\
            0.4693474399412872  0.16285951062861012  \\
            0.47044919449514  0.16475022399969844  \\
            0.4715509490489928  0.18520379149061403  \\
            0.47265270360284556  0.2626407064171822  \\
            0.47375445815669837  0.1740632178252702  \\
            0.4748562127105512  0.16770655498029566  \\
            0.4759579672644039  0.1658684892263987  \\
            0.47705972181825673  0.16506936096149785  \\
            0.47816147637210954  0.16464854759754516  \\
            0.4792632309259623  0.1644028251651643  \\
            0.4803649854798151  0.16425122813189388  \\
            0.4814667400336679  0.1641557374546809  \\
            0.48256849458752066  0.16409632366899862  \\
            0.48367024914137347  0.16406147934530846  \\
            0.4847720036952263  0.16404410819877113  \\
            0.485873758249079  0.16403953833379908  \\
            0.48697551280293183  0.16404446157652822  \\
            0.48807726735678464  0.16405629486872145  \\
            0.4891790219106374  0.16407272594875869  \\
            0.4902807764644902  0.1640913620720323  \\
            0.49138253101834295  0.16410992825911058  \\
            0.49248428557219576  0.16413277284489525  \\
            0.49358604012604856  0.16429701433818453  \\
            0.4946877946799013  0.4067248799672251  \\
            0.4957895492337541  0.16609926041188003  \\
            0.49689130378760693  0.1652052362773204  \\
            0.4979930583414597  0.16509072684280648  \\
            0.4990948128953125  0.16507234827994524  \\
            0.5001965674491653  0.16506840661433847  \\
            0.501298322003018  0.1650678153171278  \\
            0.5024000765568709  0.16507082455454866  \\
            0.5035018311107237  0.16507750794468187  \\
            0.5046035856645764  0.1650878165113094  \\
            0.5057053402184292  0.16510162135282488  \\
            0.506807094772282  0.16511874530862147  \\
            0.5079088493261348  0.16513898519140072  \\
            0.5090106038799875  0.16516212652424594  \\
            0.5101123584338404  0.1651879526215469  \\
            0.5112141129876931  0.16539541661193308  \\
            0.5123158675415459  0.1655389578974689  \\
            0.5134176220953988  0.16561953122279116  \\
            0.5145193766492515  0.16567972783261734  \\
            0.5156211312031043  0.16573062047754314  \\
            0.5167228857569571  0.16577618216752124  \\
            0.5178246403108099  0.16581819041867807  \\
            0.5189263948646626  0.16585767167361132  \\
            0.5200281494185154  0.1658953771643309  \\
            0.5211299039723682  0.16593192745202054  \\
            0.522231658526221  0.1659419964627153  \\
            0.5233334130800738  0.1659994418930165  \\
            0.5244351676339265  0.16605225984802255  \\
            0.5255369221877794  0.16610339500915308  \\
            0.5266386767416321  0.16615457665033895  \\
            0.5277404312954849  0.16620698200723882  \\
            0.5288421858493377  0.1662619640966209  \\
            0.5299439404031905  0.16632322432015073  \\
            0.5310456949570432  0.16640935060415787  \\
            0.5321474495108961  0.1667250532001099  \\
            0.5332492040647488  0.25823213988242966  \\
            0.5343509586186016  0.16738020288685884  \\
            0.5354527131724545  0.16670927430895688  \\
            0.5365544677263072  0.16664378403714544  \\
            0.53765622228016  0.1666826245066006  \\
            0.5387579768340128  0.16660600943828957  \\
            0.5398597313878656  0.16660720016131444  \\
            0.5409614859417183  0.16668496612934697  \\
            0.5420632404955712  0.16896439757659332  \\
            0.543164995049424  0.16700203698486102  \\
            0.5442667496032767  0.16686818628197603  \\
            0.5453685041571296  0.16711958873366106  \\
            0.5464702587109823  0.16734424502360948  \\
            0.5475720132648351  0.1667565736532289  \\
            0.5486737678186879  0.16666942307071317  \\
            0.5497755223725407  0.16665010541542544  \\
        }
        ;
    \addplot[color={rgb,1:red,0.0;green,0.0;blue,1.0}, name path={5b20e200-af15-47a9-aa80-27d983b30ac4}, draw opacity={1.0}, line width={1}, dotted, forget plot]
        table[row sep={\\}]
        {
            \\
            0.0  3.6833424269105066  \\
            0.001101754553852787  3.6924773201815837  \\
            0.002203509107705574  3.7403297704366154  \\
            0.0033052636615583607  4.214148218418403  \\
            0.004407018215411148  11.356299928208822  \\
            0.005508772769263934  4.421068928745918  \\
            0.0066105273231167215  5.478748391230931  \\
            0.007712281876969509  6.892411515121708  \\
            0.008814036430822295  3.9474398711822007  \\
            0.009915790984675084  3.8087315127236154  \\
            0.011017545538527868  3.8254094026868803  \\
            0.012119300092380656  3.8389470254709344  \\
            0.013221054646233443  3.8532720925946116  \\
            0.01432280920008623  4.085551679860199  \\
            0.015424563753939018  5.702472552440729  \\
            0.016526318307791804  43.871826826530175  \\
            0.01762807286164459  4.8966585841604315  \\
            0.018729827415497377  3.9746428578107778  \\
            0.019831581969350167  3.787381770567385  \\
            0.020933336523202953  3.720399765340314  \\
            0.022035091077055736  3.689455901092693  \\
            0.023136845630908526  3.673049240598128  \\
            0.024238600184761313  3.663737767169948  \\
            0.0253403547386141  3.658489933048697  \\
            0.026442109292466886  3.6561936985652976  \\
            0.027543863846319676  3.65852653860357  \\
            0.02864561840017246  3.646746882520617  \\
            0.029747372954025245  3.6461271379055153  \\
            0.030849127507878035  3.6461009249180316  \\
            0.03195088206173082  3.646579254739299  \\
            0.03305263661558361  3.6475308099774026  \\
            0.034154391169436395  3.6489867214984164  \\
            0.03525614572328918  3.651067455263194  \\
            0.03635790027714197  3.654054300079787  \\
            0.037459654830994754  3.6586096061145867  \\
            0.03856140938484754  3.6680133878882546  \\
            0.039663163938700334  3.6543726428003267  \\
            0.04076491849255312  3.6567428041283327  \\
            0.04186667304640591  3.6596832624078828  \\
            0.042968427600258687  3.6632427146175566  \\
            0.04407018215411147  3.6674703030029776  \\
            0.045171936707964266  3.6724115764304917  \\
            0.04627369126181705  3.678102751577577  \\
            0.04737544581566984  3.684562874890961  \\
            0.048477200369522626  3.6917836557969848  \\
            0.04957895492337542  3.69971715425971  \\
            0.0506807094772282  3.708262286632193  \\
            0.051782464031080985  3.8240826059921527  \\
            0.05288421858493377  3.896281788240878  \\
            0.05398597313878656  5.010618766955159  \\
            0.05508772769263935  56.60342159519081  \\
            0.05618948224649214  4.167786709644592  \\
            0.05729123680034492  3.7856978819987934  \\
            0.058392991354197704  3.714844306499369  \\
            0.05949474590805049  3.6947432690318127  \\
            0.060596500461903284  3.69061714571081  \\
            0.06169825501575607  3.694128050108717  \\
            0.06280000956960886  3.7028843024708604  \\
            0.06390176412346164  3.7164989684856886  \\
            0.06500351867731442  3.735621059779808  \\
            0.06610527323116722  3.7617850888879985  \\
            0.06720702778502  3.7976532610122873  \\
            0.06830878233887279  3.847663048743862  \\
            0.06941053689272557  3.9193749041676833  \\
            0.07051229144657836  4.026311879778441  \\
            0.07161404600043114  4.194381210594864  \\
            0.07271580055428394  4.477991862983304  \\
            0.07381755510813673  5.006862771940095  \\
            0.07491930966198951  6.155822265719687  \\
            0.0760210642158423  9.471474837837375  \\
            0.07712281876969508  34.625809779308874  \\
            0.07822457332354787  18.546069601219866  \\
            0.07932632787740067  8.11183753173703  \\
            0.08042808243125345  5.8299414022809275  \\
            0.08152983698510624  4.9447519861812514  \\
            0.08263159153895902  4.522404582843299  \\
            0.08373334609281181  4.306804456877982  \\
            0.0848351006466646  4.214613495199626  \\
            0.08593685520051737  4.2442361003241285  \\
            0.08703860975437017  4.536095039601956  \\
            0.08814036430822295  6.21636079139032  \\
            0.08924211886207574  49.375417988195196  \\
            0.09034387341592853  5.4899563875828425  \\
            0.09144562796978131  4.279895976809477  \\
            0.0925473825236341  3.9841016095234973  \\
            0.09364913707748689  3.869412107516007  \\
            0.09475089163133968  3.815413192778609  \\
            0.09585264618519247  3.7897303678690943  \\
            0.09695440073904525  3.7819041960102093  \\
            0.09805615529289805  3.789712072888899  \\
            0.09915790984675084  3.816448093032271  \\
            0.1002596644006036  3.877187156505453  \\
            0.1013614189544564  3.989381589802086  \\
            0.10246317350830918  4.240060065982633  \\
            0.10356492806216197  4.912984688822828  \\
            0.10466668261601475  7.691284238108815  \\
            0.10576843716986754  234.3380183248194  \\
            0.10687019172372034  7.49665551877458  \\
            0.10797194627757312  4.9503594935926785  \\
            0.10907370083142591  4.2851067676978225  \\
            0.1101754553852787  4.018893662431172  \\
            0.11127720993913148  3.8863789407850775  \\
            0.11237896449298428  3.810817709552267  \\
            0.11348071904683706  3.7635698326680775  \\
            0.11458247360068984  3.7320217368881186  \\
            0.11568422815454263  3.709927389720816  \\
            0.11678598270839541  3.6939139786988138  \\
            0.1178877372622482  3.68203877911524  \\
            0.11898949181610098  3.673131270091643  \\
            0.12009124636995377  3.6664713922169248  \\
            0.12119300092380657  3.661627496939536  \\
            0.12229475547765935  3.6583817340232625  \\
            0.12339651003151214  3.656719513339381  \\
            0.12449826458536492  3.6568987975440614  \\
            0.12560001913921773  3.65968982225098  \\
            0.1267017736930705  3.667130596933306  \\
            0.1278035282469233  3.685355777137582  \\
            0.12890528280077607  3.739631626027992  \\
            0.13000703735462885  4.0402824413599365  \\
            0.13110879190848163  21.30401638856547  \\
            0.13221054646233443  3.935732696426377  \\
            0.1333123010161872  3.743170315990809  \\
            0.13441405557004  3.756638289101804  \\
            0.1355158101238928  3.6658308854716872  \\
            0.13661756467774558  3.793530537085839  \\
            0.13771931923159836  3.6480418223292133  \\
            0.13882107378545114  3.6766648439722873  \\
            0.13992282833930392  3.9487512683897905  \\
            0.14102458289315672  3.73241638454273  \\
            0.1421263374470095  3.791038482490964  \\
            0.14322809200086228  4.571337309914866  \\
            0.1443298465547151  5.253298824518656  \\
            0.14543160110856787  3.8167615359389115  \\
            0.14653335566242065  3.7043182718593646  \\
            0.14763511021627346  3.6741555454341808  \\
            0.14873686477012624  3.662610748741965  \\
            0.14983861932397902  3.657890973377008  \\
            0.1509403738778318  3.6565526059024767  \\
            0.1520421284316846  3.6573492899119655  \\
            0.15314388298553738  3.6598153410087337  \\
            0.15424563753939016  3.663844780816785  \\
            0.15534739209324297  3.6695565532932592  \\
            0.15644914664709575  3.6772701617780688  \\
            0.15755090120094853  3.6875475870954473  \\
            0.15865265575480134  3.7013062630287568  \\
            0.15975441030865412  3.7200474634857303  \\
            0.1608561648625069  3.746313184678472  \\
            0.1619579194163597  3.7846501564749486  \\
            0.16305967397021248  3.8438235423127773  \\
            0.16416142852406526  3.9425156708553373  \\
            0.16526318307791804  4.126452739067625  \\
            0.16636493763177085  4.532672431048963  \\
            0.16746669218562363  5.735698180203577  \\
            0.16856844673947638  13.187860263178555  \\
            0.1696702012933292  14.521033528772941  \\
            0.17077195584718197  5.745129091688099  \\
            0.17187371040103475  4.485287559638875  \\
            0.17297546495488753  4.08825807421144  \\
            0.17407721950874033  3.922567207049678  \\
            0.1751789740625931  3.856042158624977  \\
            0.1762807286164459  3.800255802216279  \\
            0.1773824831702987  3.7837735269632353  \\
            0.17848423772415148  3.783858680740172  \\
            0.17958599227800426  3.7999594766122238  \\
            0.18068774683185707  3.837687483733549  \\
            0.18178950138570985  3.9153626428463157  \\
            0.18289125593956262  4.093010391732394  \\
            0.18399301049341543  4.642309709379657  \\
            0.1850947650472682  8.312925263587115  \\
            0.186196519601121  11.510711857930287  \\
            0.18729827415497377  4.990340756810062  \\
            0.18840002870882658  4.335504227699105  \\
            0.18950178326267936  4.210878896442365  \\
            0.19060353781653214  4.247116105277936  \\
            0.19170529237038494  4.396112742037144  \\
            0.19280704692423772  4.697916976300424  \\
            0.1939088014780905  5.299329034964145  \\
            0.1950105560319433  6.668071540888263  \\
            0.1961123105857961  10.972104401783838  \\
            0.19721406513964887  76.32978340521382  \\
            0.19831581969350168  14.447446855168963  \\
            0.19941757424735443  7.312922716333026  \\
            0.2005193288012072  5.458474110416769  \\
            0.20162108335506  4.697690316876714  \\
            0.2027228379089128  4.316216385440358  \\
            0.20382459246276557  4.10018164580834  \\
            0.20492634701661835  3.96715668544549  \\
            0.20602810157047116  3.8800978832696402  \\
            0.20712985612432394  3.82047494693068  \\
            0.20823161067817672  3.778252331332333  \\
            0.2093333652320295  3.7476663532083423  \\
            0.2104351197858823  3.725281793462984  \\
            0.2115368743397351  3.709053032869829  \\
            0.21263862889358787  3.697897362772071  \\
            0.21374038344744067  3.691632674591583  \\
            0.21484213800129345  3.6914337112656357  \\
            0.21594389255514623  3.7017669365797783  \\
            0.21704564710899904  3.7389421477474754  \\
            0.21814740166285182  3.888038828342158  \\
            0.2192491562167046  5.379878973411684  \\
            0.2203509107705574  6.422169434785005  \\
            0.2214526653244102  4.067638013542551  \\
            0.22255441987826297  3.832297969070113  \\
            0.22365617443211575  4.054588126221157  \\
            0.22475792898596855  3.703937899173449  \\
            0.22585968353982133  3.69568190960638  \\
            0.2269614380936741  3.688095684593358  \\
            0.22806319264752692  3.6812520204730097  \\
            0.22916494720137967  3.6751773610803795  \\
            0.23026670175523245  3.669864890408561  \\
            0.23136845630908526  3.665285834989438  \\
            0.23247021086293804  3.6613984565870474  \\
            0.23357196541679082  3.6581547968496833  \\
            0.2346737199706436  3.6555055115163304  \\
            0.2357754745244964  3.739404158340728  \\
            0.23687722907834918  3.6620560025712554  \\
            0.23797898363220196  3.656079018464526  \\
            0.23908073818605477  3.6524359483687903  \\
            0.24018249273990755  3.649953986729501  \\
            0.24128424729376033  3.6482095725884265  \\
            0.24238600184761314  3.6470153929105047  \\
            0.24348775640146592  3.6463008337817873  \\
            0.2445895109553187  3.6460677190569584  \\
            0.24569126550917147  3.646375709452422  \\
            0.24679302006302428  3.671804443367219  \\
            0.24789477461687706  3.6563778059210783  \\
            0.24899652917072984  3.65699985392524  \\
            0.25009828372458265  3.660727307792638  \\
            0.25120003827843546  3.6677695410403954  \\
            0.2523017928322882  3.680069439771572  \\
            0.253403547386141  3.7023186180744863  \\
            0.25450530193999377  3.746815187412785  \\
            0.2556070564938466  3.8540273034644508  \\
            0.2567088110476994  4.2265227765544084  \\
            0.25781056560155213  7.897359329728197  \\
            0.25891232015540494  91.29004828535926  \\
            0.2600140747092577  4.442985981101767  \\
            0.2611158292631105  3.9311978587480683  \\
            0.26221758381696325  3.81596285672374  \\
            0.26331933837081606  66.70797500250262  \\
            0.26442109292466887  3.875234456089491  \\
            0.2655228474785216  3.841193402691257  \\
            0.2666246020323744  4.294762349265451  \\
            0.26772635658622723  38.38734081863993  \\
            0.26882811114008  11.188311895667416  \\
            0.2699298656939328  4.49000617775321  \\
            0.2710316202477856  211.74691893360475  \\
            0.27213337480163835  3.826644423539467  \\
            0.27323512935549116  3.707639239347696  \\
            0.27433688390934396  3.6854345174227627  \\
            0.2754386384631967  3.6854290176073783  \\
            0.27654039301704947  3.707616029372306  \\
            0.2776421475709023  3.8265463817069625  \\
            0.2787439021247551  211.67133860446737  \\
            0.27984565667860783  4.489670349773424  \\
            0.28094741123246064  11.18663498179925  \\
            0.28204916578631345  38.38224316872619  \\
            0.2831509203401662  4.294677880922602  \\
            0.284252674894019  3.84117249481472  \\
            0.2853544294478718  3.8752305551641753  \\
            0.28645618400172457  66.75073046371318  \\
            0.2875579385555774  3.8160029504873383  \\
            0.2886596931094302  3.931260768939875  \\
            0.28976144766328293  4.443192428376597  \\
            0.29086320221713574  91.29654501530754  \\
            0.29196495677098855  7.897864336864879  \\
            0.2930667113248413  4.226599092935942  \\
            0.2941684658786941  3.8540481067855623  \\
            0.2952702204325469  3.7468176759612284  \\
            0.29637197498639967  3.7023124155305873  \\
            0.2974737295402525  3.6800581852561813  \\
            0.2985754840941052  3.6677549608041455  \\
            0.29967723864795803  3.6607103472499434  \\
            0.30077899320181084  3.6569810486997696  \\
            0.3018807477556636  3.656357277924224  \\
            0.3029825023095164  3.671778642903511  \\
            0.3040842568633692  3.646354779279134  \\
            0.30518601141722196  3.646046816226761  \\
            0.30628776597107477  3.6462803005366458  \\
            0.3073895205249276  3.646995578539017  \\
            0.3084912750787803  3.64819080775521  \\
            0.30959302963263313  3.6499365285838405  \\
            0.31069478418648594  3.652419909611568  \\
            0.3117965387403387  3.6560643314158114  \\
            0.3128982932941915  3.662042516474057  \\
            0.3140000478480443  3.739405618452028  \\
            0.31510180240189706  3.6554893799304837  \\
            0.31620355695574986  3.6581387506498366  \\
            0.31730531150960267  3.6613825132244737  \\
            0.3184070660634554  3.665270002874635  \\
            0.31950882061730823  3.669849166061249  \\
            0.32061057517116104  3.6751617258474836  \\
            0.3217123297250138  3.6812364370675694  \\
            0.3228140842788666  3.6880800939471112  \\
            0.3239158388327194  3.6956662288867164  \\
            0.32501759338657216  3.703922022270452  \\
            0.32611934794042496  4.054353951291024  \\
            0.3272211024942777  3.8323024173548625  \\
            0.3283228570481305  4.067636953943484  \\
            0.32942461160198333  6.422033196560763  \\
            0.3305263661558361  5.379720465059663  \\
            0.3316281207096889  3.8880092316649737  \\
            0.3327298752635417  3.738930663221301  \\
            0.33383162981739445  3.7017622868487416  \\
            0.33493338437124726  3.6914333337396883  \\
            0.3360351389251  3.6916359196622155  \\
            0.33713689347895276  3.6979042687105386  \\
            0.33823864803280557  3.7090640202397753  \\
            0.3393404025866584  3.7252976049990445  \\
            0.3404421571405111  3.74768809882262  \\
            0.34154391169436393  3.7782816157970593  \\
            0.34264566624821674  3.8205141050091878  \\
            0.3437474208020695  3.880150393108243  \\
            0.3448491753559223  3.9672279124813885  \\
            0.34595092990977505  4.100280248339057  \\
            0.34705268446362786  4.316357154755891  \\
            0.34815443901748067  4.697900470400176  \\
            0.3492561935713334  5.458809408036767  \\
            0.3503579481251862  7.3135221221633655  \\
            0.35145970267903903  14.448893793349539  \\
            0.3525614572328918  76.33774504416485  \\
            0.3536632117867446  10.973094079147852  \\
            0.3547649663405974  6.668526464031374  \\
            0.35586672089445015  5.299584647535529  \\
            0.35696847544830296  4.698072047878249  \\
            0.35807023000215576  4.396208101788943  \\
            0.3591719845560085  4.247169581759017  \\
            0.3602737391098613  4.210894317963077  \\
            0.36137549366371413  4.335466703214606  \\
            0.3624772482175669  4.990171836306285  \\
            0.3635790027714197  11.509735801548013  \\
            0.3646807573252725  8.312227475975883  \\
            0.36578251187912525  4.642148785256001  \\
            0.36688426643297806  4.0929476892711385  \\
            0.36798602098683086  3.9153327997130116  \\
            0.3690877755406836  3.8376725290922766  \\
            0.3701895300945364  3.7999531195587792  \\
            0.3712912846483892  3.7838586915066186  \\
            0.372393039202242  3.783779623622697  \\
            0.3734947937560948  3.8002691510820403  \\
            0.37459654830994754  3.8560658783423105  \\
            0.37569830286380035  3.9226016486366992  \\
            0.37680005741765316  4.088316909295054  \\
            0.3779018119715059  4.485397276315445  \\
            0.3790035665253587  5.745373751681264  \\
            0.3801053210792115  14.521972696974904  \\
            0.3812070756330643  13.18864695746216  \\
            0.3823088301869171  5.735904622495614  \\
            0.3834105847407699  4.532761066508769  \\
            0.38451233929462264  4.12649779193383  \\
            0.38561409384847545  3.94254049579044  \\
            0.38671584840232825  3.843837569615683  \\
            0.387817602956181  3.7846578397352806  \\
            0.3889193575100338  3.746316875280014  \\
            0.3900211120638866  3.7200485113104946  \\
            0.3911228666177394  3.701305495172611  \\
            0.3922246211715922  3.687545537431039  \\
            0.393326375725445  3.677267189777991  \\
            0.39442813027929774  3.6695529095804846  \\
            0.39552988483315055  3.6638406449421033  \\
            0.39663163938700335  3.659810844202172  \\
            0.39773339394085605  3.6573445287720334  \\
            0.39883514849470886  3.6565476504369423  \\
            0.3999369030485616  3.657885869629038  \\
            0.4010386576024144  3.6626055151762604  \\
            0.4021404121562672  3.6741501527522664  \\
            0.40324216671012  3.7043125583380996  \\
            0.4043439212639728  3.816754691336004  \\
            0.4054456758178256  5.25327953529663  \\
            0.40654743037167834  4.571324744965942  \\
            0.40764918492553115  3.791032713427005  \\
            0.40875093947938396  3.7324142617168468  \\
            0.4098526940332367  3.948647769077468  \\
            0.4109544485870895  3.6766604841021113  \\
            0.4120562031409423  3.6480420905497364  \\
            0.4131579576947951  3.7934245270029847  \\
            0.4142597122486479  3.665832266629562  \\
            0.4153614668025007  3.756637852740339  \\
            0.41646322135635344  3.7431742378677604  \\
            0.41756497591020625  3.935739796176736  \\
            0.418666730464059  21.304132767440432  \\
            0.4197684850179118  4.040291444405872  \\
            0.4208702395717646  3.739637698475118  \\
            0.42197199412561737  3.6853612934515705  \\
            0.4230737486794702  3.667135901622607  \\
            0.424175503233323  3.6596949910380885  \\
            0.42527725778717573  3.6569038314189113  \\
            0.42637901234102854  3.65672437888065  \\
            0.42748076689488135  3.658386373275983  \\
            0.4285825214487341  3.6616318272755652  \\
            0.4296842760025869  3.6664753010520292  \\
            0.4307860305564397  3.6731346041625135  \\
            0.43188778511029247  3.6820413269343955  \\
            0.4329895396641453  3.693915441297536  \\
            0.4340912942179981  3.7099273310973526  \\
            0.43519304877185083  3.732019496425666  \\
            0.43629480332570364  3.7635643613430423  \\
            0.43739655787955645  3.8108072419504087  \\
            0.4384983124334092  3.8863602792205074  \\
            0.439600066987262  4.018860430397085  \\
            0.4407018215411148  4.285044469789346  \\
            0.44180357609496756  4.950228182143481  \\
            0.4429053306488204  7.496297063926716  \\
            0.4440070852026732  234.32370970688174  \\
            0.44510883975652593  7.690865316964096  \\
            0.44621059431037874  4.912825872265587  \\
            0.4473123488642315  4.239980949767776  \\
            0.4484141034180843  3.989336877099688  \\
            0.4495158579719371  3.8771605550291093  \\
            0.45061761252578986  3.8164301591199457  \\
            0.45171936707964266  3.789702590281933  \\
            0.45282112163349547  3.7819011986315663  \\
            0.4539228761873482  3.7897334075159907  \\
            0.45502463074120103  3.8154233968962905  \\
            0.45612638529505384  3.8694332917409517  \\
            0.4572281398489066  3.984144221907359  \\
            0.45832989440275934  4.279992816574297  \\
            0.45943164895661215  5.490256286555892  \\
            0.4605334035104649  49.38144936248425  \\
            0.4616351580643177  6.216717264795943  \\
            0.4627369126181705  4.5361792890532815  \\
            0.46383866717202327  4.244243357742762  \\
            0.4649404217258761  4.214578661832671  \\
            0.4660421762797288  4.30673133291769  \\
            0.46714393083358163  4.522282654436671  \\
            0.46824568538743444  4.944553999281937  \\
            0.4693474399412872  5.82960520780376  \\
            0.47044919449514  8.111193452377888  \\
            0.4715509490489928  18.544232629390415  \\
            0.47265270360284556  34.62218883281108  \\
            0.47375445815669837  9.470603100287798  \\
            0.4748562127105512  6.155382627149818  \\
            0.4759579672644039  5.006600027995139  \\
            0.47705972181825673  4.477821009215133  \\
            0.47816147637210954  4.194263937834741  \\
            0.4792632309259623  4.026228338454209  \\
            0.4803649854798151  3.9193138679231687  \\
            0.4814667400336679  3.8476177445938564  \\
            0.48256849458752066  3.7976193905140225  \\
            0.48367024914137347  3.761759812954316  \\
            0.4847720036952263  3.735602446836064  \\
            0.485873758249079  3.7164856828250645  \\
            0.48697551280293183  3.7028754284429737  \\
            0.48807726735678464  3.694123006385524  \\
            0.4891790219106374  3.690615685805533  \\
            0.4902807764644902  3.694745631836601  \\
            0.49138253101834295  3.71485181739352  \\
            0.49248428557219576  3.7857156595118697  \\
            0.49358604012604856  4.167844548584532  \\
            0.4946877946799013  56.60627148104107  \\
            0.4957895492337541  5.010517231686638  \\
            0.49689130378760693  3.8962821647099886  \\
            0.4979930583414597  3.8240675501454886  \\
            0.4990948128953125  3.7082783075513377  \\
            0.5001965674491653  3.6997329184737646  \\
            0.501298322003018  3.691799279619843  \\
            0.5024000765568709  3.684578453073202  \\
            0.5035018311107237  3.6781183549399614  \\
            0.5046035856645764  3.672427252804864  \\
            0.5057053402184292  3.667486079931499  \\
            0.506807094772282  3.6632586027369776  \\
            0.5079088493261348  3.6596992588663  \\
            0.5090106038799875  3.6567588956702664  \\
            0.5101123584338404  3.654388808360292  \\
            0.5112141129876931  3.66802604242706  \\
            0.5123158675415459  3.658623680863725  \\
            0.5134176220953988  3.6540696444873726  \\
            0.5145193766492515  3.6510842064200584  \\
            0.5156211312031043  3.649004857715932  \\
            0.5167228857569571  3.6475501379311654  \\
            0.5178246403108099  3.6465994719869843  \\
            0.5189263948646626  3.6461216864513957  \\
            0.5200281494185154  3.6461480964702724  \\
            0.5211299039723682  3.6467677010413344  \\
            0.522231658526221  3.6585483060789046  \\
            0.5233334130800738  3.656213337849972  \\
            0.5244351676339265  3.658507862236037  \\
            0.5255369221877794  3.6637536257993477  \\
            0.5266386767416321  3.6730623112214347  \\
            0.5277404312954849  3.6894649228150382  \\
            0.5288421858493377  3.7204022877964835  \\
            0.5299439404031905  3.787372079231493  \\
            0.5310456949570432  3.9746031965233803  \\
            0.5321474495108961  4.896493976654565  \\
            0.5332492040647488  43.86976101195644  \\
            0.5343509586186016  5.701887856087375  \\
            0.5354527131724545  4.08544673267238  \\
            0.5365544677263072  3.85322879007001  \\
            0.53765622228016  3.838853405417694  \\
            0.5387579768340128  3.8253253628002652  \\
            0.5398597313878656  3.808742487192769  \\
            0.5409614859417183  3.9474792843047295  \\
            0.5420632404955712  6.8926174214611295  \\
            0.543164995049424  5.4791612817243145  \\
            0.5442667496032767  4.4213298533110645  \\
            0.5453685041571296  11.358633015541743  \\
            0.5464702587109823  4.21448427575817  \\
            0.5475720132648351  3.7403739029333583  \\
            0.5486737678186879  3.692489738939825  \\
            0.5497755223725407  3.6833424269120014  \\
        }
        ;
    \addplot[color={rgb,1:red,1.0;green,0.0;blue,0.0}, name path={d1d4f0a7-6234-408c-ba9c-70e1f4463e54}, draw opacity={1.0}, line width={1}, solid]
        table[row sep={\\}]
        {
            \\
            0.0  0.02252223919973492  \\
            0.001101754553852787  0.022522460517840178  \\
            0.002203509107705574  0.02252312308002555  \\
            0.0033052636615583607  0.02252423259282162  \\
            0.004407018215411148  0.02252579856018541  \\
            0.005508772769263934  0.02252783414989028  \\
            0.0066105273231167215  0.02253035573352383  \\
            0.007712281876969509  0.022533381785989733  \\
            0.008814036430822295  0.022536930655673738  \\
            0.009915790984675084  0.0225410164095066  \\
            0.011017545538527868  0.02261141061710291  \\
            0.012119300092380656  0.022625739610801155  \\
            0.013221054646233443  0.02264323196977175  \\
            0.01432280920008623  0.02266041516418808  \\
            0.015424563753939018  0.02267616927455328  \\
            0.016526318307791804  0.02276530929874224  \\
            0.01762807286164459  0.02278566723648542  \\
            0.018729827415497377  0.02280539349561686  \\
            0.019831581969350167  0.022822042138170497  \\
            0.020933336523202953  0.022835129099133218  \\
            0.022035091077055736  0.022845024670999033  \\
            0.023136845630908526  0.02285246049738421  \\
            0.024238600184761313  0.022858285241870148  \\
            0.0253403547386141  0.022863322586573646  \\
            0.026442109292466886  0.02286827867916406  \\
            0.027543863846319676  0.02287368628249625  \\
            0.02864561840017246  0.022952679865681527  \\
            0.029747372954025245  0.022963364324187708  \\
            0.030849127507878035  0.02297426621456374  \\
            0.03195088206173082  0.022985442221665455  \\
            0.03305263661558361  0.02299707609617836  \\
            0.034154391169436395  0.023009487902785963  \\
            0.03525614572328918  0.023023002248379124  \\
            0.03635790027714197  0.023037674414515347  \\
            0.037459654830994754  0.02305302569415784  \\
            0.03856140938484754  0.023067756442694858  \\
            0.039663163938700334  0.023078673874791355  \\
            0.04076491849255312  0.023084833695697936  \\
            0.04186667304640591  0.023089289053653834  \\
            0.042968427600258687  0.023091871049717824  \\
            0.04407018215411147  0.023092425465789376  \\
            0.045171936707964266  0.023090814595223025  \\
            0.04627369126181705  0.02308691744534848  \\
            0.04737544581566984  0.02308062862446256  \\
            0.048477200369522626  0.023071856487283313  \\
            0.04957895492337542  0.023060521317577708  \\
            0.0506807094772282  0.023046554519368052  \\
            0.051782464031080985  0.02272978090383761  \\
            0.05288421858493377  0.022736068620428113  \\
            0.05398597313878656  0.022741774732869213  \\
            0.05508772769263935  0.022746884439074813  \\
            0.05618948224649214  0.022751392880231452  \\
            0.05729123680034492  0.022755304471800877  \\
            0.058392991354197704  0.02275863209910957  \\
            0.05949474590805049  0.022761396196767775  \\
            0.060596500461903284  0.022763623747514544  \\
            0.06169825501575607  0.02276534723853056  \\
            0.06280000956960886  0.022766603621087025  \\
            0.06390176412346164  0.022767433300524872  \\
            0.06500351867731442  0.02276787919857959  \\
            0.06610527323116722  0.022767985900735623  \\
            0.06720702778502  0.022767798910198967  \\
            0.06830878233887279  0.02276736401044186  \\
            0.06941053689272557  0.022766726734225508  \\
            0.07051229144657836  0.022765931935565132  \\
            0.07161404600043114  0.022765023448343594  \\
            0.07271580055428394  0.022764043821997464  \\
            0.07381755510813673  0.02276303411506106  \\
            0.07491930966198951  0.022762033738877765  \\
            0.0760210642158423  0.022761080330298108  \\
            0.07712281876969508  0.022760209651989983  \\
            0.07822457332354787  0.022759455504425413  \\
            0.07932632787740067  0.022758849647835273  \\
            0.08042808243125345  0.022758421728362378  \\
            0.08152983698510624  0.022758199206006736  \\
            0.08263159153895902  0.022758207282831823  \\
            0.08373334609281181  0.02275846882950579  \\
            0.0848351006466646  0.022759004309322428  \\
            0.08593685520051737  0.022759831702890154  \\
            0.08703860975437017  0.022760966423461308  \\
            0.08814036430822295  0.022762421231537853  \\
            0.08924211886207574  0.022764206138815794  \\
            0.09034387341592853  0.022766328300807187  \\
            0.09144562796978131  0.022768791897896438  \\
            0.0925473825236341  0.022771597994348497  \\
            0.09364913707748689  0.022774744377074167  \\
            0.09475089163133968  0.022778225365757375  \\
            0.09585264618519247  0.022782031587106143  \\
            0.09695440073904525  0.022786149714562727  \\
            0.09805615529289805  0.022790562159054922  \\
            0.09915790984675084  0.022795246707899237  \\
            0.1002596644006036  0.022845777586142032  \\
            0.1013614189544564  0.022848406897893643  \\
            0.10246317350830918  0.022851328729350053  \\
            0.10356492806216197  0.02285452224280881  \\
            0.10466668261601475  0.022857962267111252  \\
            0.10576843716986754  0.02286161878291014  \\
            0.10687019172372034  0.022865456441043847  \\
            0.10797194627757312  0.02286943413220351  \\
            0.10907370083142591  0.022873504642535004  \\
            0.1101754553852787  0.022877614419398777  \\
            0.11127720993913148  0.022881703486470235  \\
            0.11237896449298428  0.022885705540550562  \\
            0.11348071904683706  0.02288954826259199  \\
            0.11458247360068984  0.022893153866411716  \\
            0.11568422815454263  0.02289643990063589  \\
            0.11678598270839541  0.022899320298919716  \\
            0.1178877372622482  0.0229017066621637  \\
            0.11898949181610098  0.02290350972309055  \\
            0.12009124636995377  0.022904640935328652  \\
            0.12119300092380657  0.022905014105263268  \\
            0.12229475547765935  0.022904546975058504  \\
            0.12339651003151214  0.022903162665422767  \\
            0.12449826458536492  0.022900790893770333  \\
            0.12560001913921773  0.022897368905069312  \\
            0.1267017736930705  0.022892842070212426  \\
            0.1278035282469233  0.022887164151745614  \\
            0.12890528280077607  0.02288029724714828  \\
            0.13000703735462885  0.02287221147076698  \\
            0.13110879190848163  0.022862884436637053  \\
            0.13221054646233443  0.022852300634769552  \\
            0.1333123010161872  0.02284045080073764  \\
            0.13441405557004  0.022827331378744042  \\
            0.1355158101238928  0.02319391648626782  \\
            0.13661756467774558  0.0231844209593413  \\
            0.13771931923159836  0.023543812281399158  \\
            0.13882107378545114  0.02318911062244909  \\
            0.13992282833930392  0.02282011577352675  \\
            0.14102458289315672  0.022833890723333864  \\
            0.1421263374470095  0.022846392568451022  \\
            0.14322809200086228  0.022857622861530758  \\
            0.1443298465547151  0.022867588521769663  \\
            0.14543160110856787  0.02287630231261727  \\
            0.14653335566242065  0.022883783464439873  \\
            0.14763511021627346  0.02289005833690437  \\
            0.14873686477012624  0.022895161013381477  \\
            0.14983861932397902  0.022899133738942052  \\
            0.1509403738778318  0.022902027123862127  \\
            0.1520421284316846  0.02290390004751472  \\
            0.15314388298553738  0.02290481923361425  \\
            0.15424563753939016  0.02290485848385319  \\
            0.15534739209324297  0.022904097598839598  \\
            0.15644914664709575  0.022902621041464562  \\
            0.15755090120094853  0.022900516412774918  \\
            0.15865265575480134  0.022897872832788033  \\
            0.15975441030865412  0.02289477932192385  \\
            0.1608561648625069  0.022891323261374108  \\
            0.1619579194163597  0.02288758901289339  \\
            0.16305967397021248  0.02288365674718404  \\
            0.16416142852406526  0.02287960151098599  \\
            0.16526318307791804  0.02287549254833876  \\
            0.16636493763177085  0.022871392867524733  \\
            0.16746669218562363  0.02286735903314253  \\
            0.16856844673947638  0.02286344115377056  \\
            0.1696702012933292  0.022859683036741723  \\
            0.17077195584718197  0.02285612246674328  \\
            0.17187371040103475  0.022852791583883317  \\
            0.17297546495488753  0.022849717328201066  \\
            0.17407721950874033  0.022846921921611665  \\
            0.1751789740625931  0.02284442337308698  \\
            0.1762807286164459  0.022792717614508837  \\
            0.1773824831702987  0.022788168045849367  \\
            0.17848423772415148  0.022783902221249073  \\
            0.17958599227800426  0.022779939989500175  \\
            0.18068774683185707  0.02277629676616118  \\
            0.18178950138570985  0.02277298386349965  \\
            0.18289125593956262  0.022770008778214473  \\
            0.18399301049341543  0.02276737543039406  \\
            0.1850947650472682  0.022765084373425417  \\
            0.186196519601121  0.022763132966148274  \\
            0.18729827415497377  0.022761515525754557  \\
            0.18840002870882658  0.022760223455942277  \\
            0.18950178326267936  0.02275924535938181  \\
            0.19060353781653214  0.022758567139611132  \\
            0.19170529237038494  0.022758172091764922  \\
            0.19280704692423772  0.02275804098641429  \\
            0.1939088014780905  0.022758152148669043  \\
            0.1950105560319433  0.022758481536719073  \\
            0.1961123105857961  0.022759002812701726  \\
            0.19721406513964887  0.022759687417346543  \\
            0.19831581969350168  0.022760504642826158  \\
            0.19941757424735443  0.022761421700791814  \\
            0.2005193288012072  0.022762403804061512  \\
            0.20162108335506  0.022763414243882858  \\
            0.2027228379089128  0.02276441448613942  \\
            0.20382459246276557  0.022765364282073337  \\
            0.20492634701661835  0.022766221807754224  \\
            0.20602810157047116  0.022766943846488613  \\
            0.20712985612432394  0.022767486020496035  \\
            0.20823161067817672  0.02276780309204926  \\
            0.2093333652320295  0.022767849346848154  \\
            0.2104351197858823  0.022767579072719334  \\
            0.2115368743397351  0.022766947142518788  \\
            0.21263862889358787  0.022765909706381215  \\
            0.21374038344744067  0.022764424992454513  \\
            0.21484213800129345  0.02276245420505258  \\
            0.21594389255514623  0.022759962500682948  \\
            0.21704564710899904  0.02275692001843812  \\
            0.21814740166285182  0.022753302926562975  \\
            0.2192491562167046  0.022749094451181555  \\
            0.2203509107705574  0.022744285840070187  \\
            0.2214526653244102  0.022738877230799866  \\
            0.22255441987826297  0.022732878388106476  \\
            0.22365617443211575  0.02272630930435743  \\
            0.22475792898596855  0.02305352023600203  \\
            0.22585968353982133  0.023066150843495975  \\
            0.2269614380936741  0.023076184449240526  \\
            0.22806319264752692  0.023083695430504994  \\
            0.22916494720137967  0.023088769387156403  \\
            0.23026670175523245  0.023091504615371173  \\
            0.23136845630908526  0.023092014072563463  \\
            0.23247021086293804  0.02309042696255973  \\
            0.23357196541679082  0.023086889259248965  \\
            0.2346737199706436  0.0230815627206596  \\
            0.2357754745244964  0.023073546659831502  \\
            0.23687722907834918  0.023060435071742373  \\
            0.23797898363220196  0.02304512400192021  \\
            0.23908073818605477  0.023029997893866574  \\
            0.24018249273990755  0.023015871430716296  \\
            0.24128424729376033  0.023002931813267142  \\
            0.24238600184761314  0.022990950244706602  \\
            0.24348775640146592  0.022979577988876395  \\
            0.2445895109553187  0.02296855701381392  \\
            0.24569126550917147  0.022957774864398532  \\
            0.24679302006302428  0.022876339907885554  \\
            0.24789477461687706  0.022870583477606482  \\
            0.24899652917072984  0.02286547703668604  \\
            0.25009828372458265  0.02286058312972435  \\
            0.25120003827843546  0.022855277731876653  \\
            0.2523017928322882  0.022848786933424563  \\
            0.253403547386141  0.022840260816330595  \\
            0.25450530193999377  0.022828887294966776  \\
            0.2556070564938466  0.022814074394818577  \\
            0.2567088110476994  0.022795791055741126  \\
            0.25781056560155213  0.02277529825944428  \\
            0.25891232015540494  0.022756855815259325  \\
            0.2600140747092577  0.02266834246336077  \\
            0.2611158292631105  0.022651820243446204  \\
            0.26221758381696325  0.022634254747324944  \\
            0.26331933837081606  0.022617674765002795  \\
            0.26442109292466887  0.022543228017219735  \\
            0.2655228474785216  0.0225388759658583  \\
            0.2666246020323744  0.022535063849918435  \\
            0.26772635658622723  0.0225317823045093  \\
            0.26882811114008  0.02252901447948204  \\
            0.2699298656939328  0.022526741619659476  \\
            0.2710316202477856  0.022524946133770648  \\
            0.27213337480163835  0.022523613185420857  \\
            0.27323512935549116  0.02252273142649426  \\
            0.27433688390934396  0.022522293264074605  \\
            0.2754386384631967  0.022522294911205106  \\
            0.27654039301704947  0.022522736376040842  \\
            0.2776421475709023  0.02252362146206092  \\
            0.2787439021247551  0.022524957779386394  \\
            0.27984565667860783  0.02252675669529479  \\
            0.28094741123246064  0.022529033068046187  \\
            0.28204916578631345  0.022531804514626677  \\
            0.2831509203401662  0.0225350898224362  \\
            0.284252674894019  0.022538905882494444  \\
            0.2853544294478718  0.022543262113874366  \\
            0.28645618400172457  0.02261779490152133  \\
            0.2875579385555774  0.022634389199431008  \\
            0.2886596931094302  0.02265196480587566  \\
            0.28976144766328293  0.022668494493007955  \\
            0.29086320221713574  0.022756788740353924  \\
            0.29196495677098855  0.022775282942089834  \\
            0.2930667113248413  0.022795828165636044  \\
            0.2941684658786941  0.022814161399130516  \\
            0.2952702204325469  0.02282902016228435  \\
            0.29637197498639967  0.022840434970815327  \\
            0.2974737295402525  0.022848997780915228  \\
            0.2985754840941052  0.02285552091996113  \\
            0.29967723864795803  0.02286085461807836  \\
            0.30077899320181084  0.022865773028575193  \\
            0.3018807477556636  0.02287090025692195  \\
            0.3029825023095164  0.02287667365995881  \\
            0.3040842568633692  0.022957995659939383  \\
            0.30518601141722196  0.022968786712849687  \\
            0.30628776597107477  0.022979812107524377  \\
            0.3073895205249276  0.022991184384374672  \\
            0.3084912750787803  0.023003161883947396  \\
            0.30959302963263313  0.02301609375264776  \\
            0.31069478418648594  0.023030208991153343  \\
            0.3117965387403387  0.02304532004671061  \\
            0.3128982932941915  0.023060611422292238  \\
            0.3140000478480443  0.023073698673796033  \\
            0.31510180240189706  0.023081955798170767  \\
            0.31620355695574986  0.023087285299798987  \\
            0.31730531150960267  0.023090824234455975  \\
            0.3184070660634554  0.023092410708971136  \\
            0.31950882061730823  0.023091898637532266  \\
            0.32061057517116104  0.023089158738054987  \\
            0.3217123297250138  0.02308407802037713  \\
            0.3228140842788666  0.023076558215077675  \\
            0.3239158388327194  0.023066513831756997  \\
            0.32501759338657216  0.023053870707081515  \\
            0.32611934794042496  0.02272642642237038  \\
            0.3272211024942777  0.022732996133357598  \\
            0.3283228570481305  0.022738995541956113  \\
            0.32942461160198333  0.02274440466654795  \\
            0.3305263661558361  0.022749213750424283  \\
            0.3316281207096889  0.022753422664901622  \\
            0.3327298752635417  0.02275704016914189  \\
            0.33383162981739445  0.022760083044013685  \\
            0.33493338437124726  0.02276257512674079  \\
            0.3360351389251  0.02276454628357798  \\
            0.33713689347895276  0.022766031362731535  \\
            0.33823864803280557  0.022767069164709882  \\
            0.3393404025866584  0.02276770146550202  \\
            0.3404421571405111  0.022767972118606462  \\
            0.34154391169436393  0.022767926255310565  \\
            0.34264566624821674  0.02276760959197968  \\
            0.3437474208020695  0.02276706784573526  \\
            0.3448491753559223  0.022766346258478374  \\
            0.34595092990977505  0.02276548921114483  \\
            0.34705268446362786  0.022764539924639744  \\
            0.34815443901748067  0.0227635402254274  \\
            0.3492561935713334  0.022762530365410674  \\
            0.3503579481251862  0.02276154888360081  \\
            0.35145970267903903  0.0227606324903891  \\
            0.3525614572328918  0.02275981597823222  \\
            0.3536632117867446  0.022759132136485966  \\
            0.3547649663405974  0.02275861167776881  \\
            0.35586672089445015  0.02275828316347135  \\
            0.35696847544830296  0.022758172934867624  \\
            0.35807023000215576  0.022758305036134664  \\
            0.3591719845560085  0.022758701145057333  \\
            0.3602737391098613  0.022759380492383196  \\
            0.36137549366371413  0.022760359786494325  \\
            0.3624772482175669  0.022761653125010477  \\
            0.3635790027714197  0.02276327190748423  \\
            0.3646807573252725  0.022765224730216128  \\
            0.36578251187912525  0.022767517278509165  \\
            0.36688426643297806  0.022770152193129257  \\
            0.36798602098683086  0.02277312892111404  \\
            0.3690877755406836  0.022776443542250942  \\
            0.3701895300945364  0.022780088558977285  \\
            0.3712912846483892  0.022784052657956542  \\
            0.372393039202242  0.022788320422366625  \\
            0.3734947937560948  0.02279287200030219  \\
            0.37459654830994754  0.022844578080821992  \\
            0.37569830286380035  0.022847054529777054  \\
            0.37680005741765316  0.022849832455392575  \\
            0.3779018119715059  0.022852892991616157  \\
            0.3790035665253587  0.022856213197915677  \\
            0.3801053210792115  0.022859765542491833  \\
            0.3812070756330643  0.022863517398967678  \\
            0.3823088301869171  0.022867430586852165  \\
            0.3834105847407699  0.02287146098158213  \\
            0.38451233929462264  0.022875558220839455  \\
            0.38561409384847545  0.022879665540029764  \\
            0.38671584840232825  0.022883719774557366  \\
            0.387817602956181  0.022887651560019307  \\
            0.3889193575100338  0.022891385759327428  \\
            0.3900211120638866  0.022894842138153227  \\
            0.3911228666177394  0.02289793629087399  \\
            0.3922246211715922  0.02290058081223652  \\
            0.393326375725445  0.0229026866755249  \\
            0.39442813027929774  0.022904164771426337  \\
            0.39552988483315055  0.022904927527144552  \\
            0.39663163938700335  0.022904890524487936  \\
            0.39773339394085605  0.02290397402736851  \\
            0.39883514849470886  0.022902104319867226  \\
            0.3999369030485616  0.022899214785901126  \\
            0.4010386576024144  0.022895246680908213  \\
            0.4021404121562672  0.022890149558304956  \\
            0.40324216671012  0.022883881373352937  \\
            0.4043439212639728  0.022876408285769596  \\
            0.4054456758178256  0.02286770423395861  \\
            0.40654743037167834  0.022857750349765914  \\
            0.40764918492553115  0.022846534322458933  \\
            0.40875093947938396  0.02283404979811589  \\
            0.4098526940332367  0.02282029593811797  \\
            0.4109544485870895  0.023189347821229632  \\
            0.4120562031409423  0.023543791803665018  \\
            0.4131579576947951  0.02318416652591496  \\
            0.4142597122486479  0.02319369505986814  \\
            0.4153614668025007  0.022827162282316106  \\
            0.41646322135635344  0.02284030080891486  \\
            0.41756497591020625  0.022852166356644533  \\
            0.418666730464059  0.0228627631167361  \\
            0.4197684850179118  0.02287210085831907  \\
            0.4208702395717646  0.022880195495413997  \\
            0.42197199412561737  0.0228870697429361  \\
            0.4230737486794702  0.022892753754410966  \\
            0.424175503233323  0.02289728565368415  \\
            0.42527725778717573  0.022900711859750462  \\
            0.42637901234102854  0.022903087149650284  \\
            0.42748076689488135  0.02290447439987218  \\
            0.4285825214487341  0.02290494398968486  \\
            0.4296842760025869  0.022904572871601768  \\
            0.4307860305564397  0.022903443359451926  \\
            0.43188778511029247  0.022901641682318636  \\
            0.4329895396641453  0.022899256407035835  \\
            0.4340912942179981  0.022896376801761247  \\
            0.43519304877185083  0.02289309125215095  \\
            0.43629480332570364  0.022889485789772197  \\
            0.43739655787955645  0.022885642812812135  \\
            0.4384983124334092  0.02288164003056699  \\
            0.439600066987262  0.022877549657830083  \\
            0.4407018215411148  0.022873437860238606  \\
            0.44180357609496756  0.02286936443747049  \\
            0.4429053306488204  0.022865382716283456  \\
            0.4440070852026732  0.02286153962659867  \\
            0.44510883975652593  0.022857875922020526  \\
            0.44621059431037874  0.02285442651438595  \\
            0.4473123488642315  0.022851220883918607  \\
            0.4484141034180843  0.02284828355028953  \\
            0.4495158579719371  0.02284563456541934  \\
            0.45061761252578986  0.02279509129228612  \\
            0.45171936707964266  0.022790408786633548  \\
            0.45282112163349547  0.022785998317058336  \\
            0.4539228761873482  0.022781882093163685  \\
            0.45502463074120103  0.02277807770226886  \\
            0.45612638529505384  0.02277459846972818  \\
            0.4572281398489066  0.022771453767071157  \\
            0.45832989440275934  0.022768649275449353  \\
            0.45943164895661215  0.022766187207418687  \\
            0.4605334035104649  0.022764066499093613  \\
            0.4616351580643177  0.022762282970645895  \\
            0.4627369126181705  0.022760829467607626  \\
            0.46383866717202327  0.02275969597981101  \\
            0.4649404217258761  0.022758869748826432  \\
            0.4660421762797288  0.022758335362875578  \\
            0.46714393083358163  0.022758074844502665  \\
            0.46824568538743444  0.02275806773221115  \\
            0.4693474399412872  0.022758291157565464  \\
            0.47044919449514  0.022758719922218305  \\
            0.4715509490489928  0.022759326568936925  \\
            0.47265270360284556  0.022760081454031498  \\
            0.47375445815669837  0.02276095282074683  \\
            0.4748562127105512  0.022761906872242364  \\
            0.4759579672644039  0.02276290784898857  \\
            0.47705972181825673  0.022763918116856786  \\
            0.47816147637210954  0.02276489826896281  \\
            0.4792632309259623  0.022765807249403806  \\
            0.4803649854798151  0.022766602512399436  \\
            0.4814667400336679  0.022767240227834187  \\
            0.48256849458752066  0.022767675545378882  \\
            0.48367024914137347  0.022767862934727767  \\
            0.4847720036952263  0.022767756617857165  \\
            0.485873758249079  0.02276731109422887  \\
            0.48697551280293183  0.0227664817822745  \\
            0.48807726735678464  0.02276522576461096  \\
            0.4891790219106374  0.022763502640191283  \\
            0.4902807764644902  0.022761275462843816  \\
            0.49138253101834295  0.02275851174990042  \\
            0.49248428557219576  0.022755184524361805  \\
            0.49358604012604856  0.02275127335769574  \\
            0.4946877946799013  0.02274676537174066  \\
            0.4957895492337541  0.02274165615808227  \\
            0.49689130378760693  0.022735950584615132  \\
            0.4979930583414597  0.022729663463703687  \\
            0.4990948128953125  0.023046210855619496  \\
            0.5001965674491653  0.023060164388987846  \\
            0.501298322003018  0.023071487876362297  \\
            0.5024000765568709  0.023080250193640107  \\
            0.5035018311107237  0.023086531213692918  \\
            0.5046035856645764  0.023090422648635717  \\
            0.5057053402184292  0.02309202988376888  \\
            0.506807094772282  0.023091473854472883  \\
            0.5079088493261348  0.02308889217213352  \\
            0.5090106038799875  0.023084438929311456  \\
            0.5101123584338404  0.023078282882308152  \\
            0.5112141129876931  0.023067591799235555  \\
            0.5123158675415459  0.023052838869313124  \\
            0.5134176220953988  0.023037470317879515  \\
            0.5145193766492515  0.023022785097607832  \\
            0.5156211312031043  0.02300926126913367  \\
            0.5167228857569571  0.02299684350573064  \\
            0.5178246403108099  0.022985207557149275  \\
            0.5189263948646626  0.022974033744605247  \\
            0.5200281494185154  0.022963138519642174  \\
            0.5211299039723682  0.022952465172409495  \\
            0.522231658526221  0.022873360527882466  \\
            0.5233334130800738  0.022867971826826285  \\
            0.5244351676339265  0.022863038381575464  \\
            0.5255369221877794  0.022858027417276973  \\
            0.5266386767416321  0.022852232955359007  \\
            0.5277404312954849  0.02284483160704954  \\
            0.5288421858493377  0.02283497500631668  \\
            0.5299439404031905  0.02282193164821084  \\
            0.5310456949570432  0.02280533099914659  \\
            0.5321474495108961  0.02278565617186002  \\
            0.5332492040647488  0.02276535087654462  \\
            0.5343509586186016  0.02267601417935072  \\
            0.5354527131724545  0.02266026660937356  \\
            0.5365544677263072  0.022643092052052224  \\
            0.53765622228016  0.02262561165993905  \\
            0.5387579768340128  0.022611299976115006  \\
            0.5398597313878656  0.02254098443635172  \\
            0.5409614859417183  0.022536902736827666  \\
            0.5420632404955712  0.022533357714602194  \\
            0.543164995049424  0.02253033534959996  \\
            0.5442667496032767  0.022527817329595433  \\
            0.5453685041571296  0.02252578520841771  \\
            0.5464702587109823  0.02252422263810186  \\
            0.5475720132648351  0.022523116470944494  \\
            0.5486737678186879  0.022522457221445545  \\
            0.5497755223725407  0.02252223919972479  \\
        }
        ;
    \addlegendentry {$(20, 10, 0.1) $}
    \addplot[color={rgb,1:red,1.0;green,0.0;blue,0.0}, name path={fc4c513b-684e-4a29-8eaa-0acd368e3f5d}, draw opacity={1.0}, line width={1}, dashed, forget plot]
        table[row sep={\\}]
        {
            \\
            0.0  0.18069911663312657  \\
            0.001101754553852787  0.1806977799152342  \\
            0.002203509107705574  0.18069375841459812  \\
            0.0033052636615583607  0.180687053160145  \\
            0.004407018215411148  0.180677665744694  \\
            0.005508772769263934  0.18066559810574387  \\
            0.0066105273231167215  0.18065085220545207  \\
            0.007712281876969509  0.18063342950609484  \\
            0.008814036430822295  0.18061332994727958  \\
            0.009915790984675084  0.1805905498040117  \\
            0.011017545538527868  0.18093236245361455  \\
            0.012119300092380656  0.18094037254314288  \\
            0.013221054646233443  0.18094161103192286  \\
            0.01432280920008623  0.18093541252963147  \\
            0.015424563753939018  0.18092249509114536  \\
            0.016526318307791804  0.18125364689791015  \\
            0.01762807286164459  0.18125806503542877  \\
            0.018729827415497377  0.18125408523844885  \\
            0.019831581969350167  0.18124266445031376  \\
            0.020933336523202953  0.1812256269701158  \\
            0.022035091077055736  0.18120512481282589  \\
            0.023136845630908526  0.18118337653389516  \\
            0.024238600184761313  0.18116247809500627  \\
            0.0253403547386141  0.1811442438883092  \\
            0.026442109292466886  0.1811300860885158  \\
            0.027543863846319676  0.18112093997565076  \\
            0.02864561840017246  0.18220334681612507  \\
            0.029747372954025245  0.18221687146687104  \\
            0.030849127507878035  0.1822338718791959  \\
            0.03195088206173082  0.18225360886462103  \\
            0.03305263661558361  0.18227512092301446  \\
            0.034154391169436395  0.18229730412254372  \\
            0.03525614572328918  0.1823189329300067  \\
            0.03635790027714197  0.18233856085846048  \\
            0.037459654830994754  0.18235406816358044  \\
            0.03856140938484754  0.18236058942447672  \\
            0.039663163938700334  0.18234089034317957  \\
            0.04076491849255312  0.18230328568060344  \\
            0.04186667304640591  0.18226442032519607  \\
            0.042968427600258687  0.18222410476749282  \\
            0.04407018215411147  0.18218215628161288  \\
            0.045171936707964266  0.1821384037171641  \\
            0.04627369126181705  0.18209269032009692  \\
            0.04737544581566984  0.18204487452517146  \\
            0.048477200369522626  0.18199482904225497  \\
            0.04957895492337542  0.18194243890157438  \\
            0.0506807094772282  0.18188759944098107  \\
            0.051782464031080985  0.17998069474335177  \\
            0.05288421858493377  0.17994280621276382  \\
            0.05398597313878656  0.17990488869385735  \\
            0.05508772769263935  0.17986692814303293  \\
            0.05618948224649214  0.179828917598276  \\
            0.05729123680034492  0.17979085698616354  \\
            0.058392991354197704  0.17975275274159305  \\
            0.05949474590805049  0.17971461729448482  \\
            0.060596500461903284  0.1796764684740737  \\
            0.06169825501575607  0.17963832887478504  \\
            0.06280000956960886  0.17960022522198255  \\
            0.06390176412346164  0.17956218776501856  \\
            0.06500351867731442  0.17952424971746497  \\
            0.06610527323116722  0.17948644675245465  \\
            0.06720702778502  0.17944881655791473  \\
            0.06830878233887279  0.17941139844284293  \\
            0.06941053689272557  0.17937423298672153  \\
            0.07051229144657836  0.17933736172023915  \\
            0.07161404600043114  0.17930082682200488  \\
            0.07271580055428394  0.17926467082301076  \\
            0.07381755510813673  0.17922893630592832  \\
            0.07491930966198951  0.17919366559545227  \\
            0.0760210642158423  0.17915890043536534  \\
            0.07712281876969508  0.179124681653842  \\
            0.07822457332354787  0.1790910488183051  \\
            0.07932632787740067  0.17905803988585625  \\
            0.08042808243125345  0.1790256908551305  \\
            0.08152983698510624  0.17899403542593706  \\
            0.08263159153895902  0.1789631046725595  \\
            0.08373334609281181  0.17893292673671676  \\
            0.0848351006466646  0.17890352654210692  \\
            0.08593685520051737  0.1788749255354098  \\
            0.08703860975437017  0.17884714145028108  \\
            0.08814036430822295  0.1788201880948251  \\
            0.08924211886207574  0.1787940751570766  \\
            0.09034387341592853  0.17876880802292375  \\
            0.09144562796978131  0.17874438759881772  \\
            0.0925473825236341  0.17872081012807434  \\
            0.09364913707748689  0.17869806699227578  \\
            0.09475089163133968  0.17867614448223862  \\
            0.09585264618519247  0.17865502352694235  \\
            0.09695440073904525  0.17863467936423072  \\
            0.09805615529289805  0.17861508113792243  \\
            0.09915790984675084  0.17859619140402097  \\
            0.1002596644006036  0.1785472594580839  \\
            0.1013614189544564  0.1785289699911921  \\
            0.10246317350830918  0.17851144721958037  \\
            0.10356492806216197  0.17849468258803694  \\
            0.10466668261601475  0.17847866337577056  \\
            0.10576843716986754  0.178463371568461  \\
            0.10687019172372034  0.1784487827883928  \\
            0.10797194627757312  0.17843486530290362  \\
            0.10907370083142591  0.17842157914437576  \\
            0.1101754553852787  0.17840887537551764  \\
            0.11127720993913148  0.17839669554426685  \\
            0.11237896449298428  0.178384971372402  \\
            0.11348071904683706  0.17837362472523322  \\
            0.11458247360068984  0.17836256790075575  \\
            0.11568422815454263  0.17835170426740946  \\
            0.11678598270839541  0.17834092926063447  \\
            0.1178877372622482  0.17833013172211307  \\
            0.11898949181610098  0.1783191955366565  \\
            0.12009124636995377  0.17830800149339654  \\
            0.12119300092380657  0.17829642926399222  \\
            0.12229475547765935  0.1782843593757873  \\
            0.12339651003151214  0.1782716750425319  \\
            0.12449826458536492  0.17825826372097772  \\
            0.12560001913921773  0.17824401827732594  \\
            0.1267017736930705  0.1782288376801967  \\
            0.1278035282469233  0.1782126271711573  \\
            0.12890528280077607  0.17819529791014624  \\
            0.13000703735462885  0.17817676613229144  \\
            0.13110879190848163  0.17815695188853908  \\
            0.13221054646233443  0.17813577746855638  \\
            0.1333123010161872  0.1781131656277593  \\
            0.13441405557004  0.1780890377494124  \\
            0.1355158101238928  0.18058527079212935  \\
            0.13661756467774558  0.18056956148726547  \\
            0.13771931923159836  0.183020632611722  \\
            0.13882107378545114  0.18057842476779612  \\
            0.13992282833930392  0.17807370017165539  \\
            0.14102458289315672  0.17809875081306797  \\
            0.1421263374470095  0.1781222122718822  \\
            0.14322809200086228  0.17814416903458294  \\
            0.1443298465547151  0.17816470294125233  \\
            0.14543160110856787  0.17818389496817563  \\
            0.14653335566242065  0.17820182704153545  \\
            0.14763511021627346  0.17821858373472232  \\
            0.14873686477012624  0.17823425371646465  \\
            0.14983861932397902  0.1782489308370304  \\
            0.1509403738778318  0.178262714763146  \\
            0.1520421284316846  0.1782757111024639  \\
            0.15314388298553738  0.17828803099959534  \\
            0.15424563753939016  0.1782997902272865  \\
            0.15534739209324297  0.1783111078371609  \\
            0.15644914664709575  0.17832210447190872  \\
            0.15755090120094853  0.17833290046235684  \\
            0.15865265575480134  0.17834361384446057  \\
            0.15975441030865412  0.1783543584299819  \\
            0.1608561648625069  0.1783652420435493  \\
            0.1619579194163597  0.1783763650189967  \\
            0.16305967397021248  0.17838781901458955  \\
            0.16416142852406526  0.17839968617708996  \\
            0.16526318307791804  0.17841203865756228  \\
            0.16636493763177085  0.1784249384602363  \\
            0.16746669218562363  0.1784384375893514  \\
            0.16856844673947638  0.1784525784514875  \\
            0.1696702012933292  0.17846739446881602  \\
            0.17077195584718197  0.1784829108580147  \\
            0.17187371040103475  0.17849914553875004  \\
            0.17297546495488753  0.1785161101383926  \\
            0.17407721950874033  0.17853381106710636  \\
            0.1751789740625931  0.17855225064655092  \\
            0.1762807286164459  0.1786010082116877  \\
            0.1773824831702987  0.17862025808864077  \\
            0.17848423772415148  0.17864023594619943  \\
            0.17958599227800426  0.1786609758128653  \\
            0.18068774683185707  0.178682505391905  \\
            0.18178950138570985  0.1787048465069601  \\
            0.18289125593956262  0.17872801546656586  \\
            0.18399301049341543  0.17875202336133694  \\
            0.1850947650472682  0.17877687631286632  \\
            0.186196519601121  0.17880257568527363  \\
            0.18729827415497377  0.17882911827533965  \\
            0.18840002870882658  0.17885649649062701  \\
            0.18950178326267936  0.17888469852873118  \\
            0.19060353781653214  0.17891370856448174  \\
            0.19170529237038494  0.1789435069524926  \\
            0.19280704692423772  0.1789740704508472  \\
            0.1939088014780905  0.17900537246830547  \\
            0.1950105560319433  0.17903738333719216  \\
            0.1961123105857961  0.1790700706089264  \\
            0.19721406513964887  0.1791033993718171  \\
            0.19831581969350168  0.17913733258553347  \\
            0.19941757424735443  0.17917183142534077  \\
            0.2005193288012072  0.17920685563220057  \\
            0.20162108335506  0.1792423638607821  \\
            0.2027228379089128  0.17927831401973526  \\
            0.20382459246276557  0.17931466360092047  \\
            0.20492634701661835  0.17935136999538495  \\
            0.20602810157047116  0.1793883907987949  \\
            0.20712985612432394  0.1794256841075257  \\
            0.20823161067817672  0.17946320881797656  \\
            0.2093333652320295  0.17950092493625666  \\
            0.2104351197858823  0.17953879391138836  \\
            0.2115368743397351  0.17957677900633534  \\
            0.21263862889358787  0.17961484571645098  \\
            0.21374038344744067  0.1796529622440686  \\
            0.21484213800129345  0.1796911000310689  \\
            0.21594389255514623  0.17972923434357338  \\
            0.21704564710899904  0.17976734489529297  \\
            0.21814740166285182  0.17980541648484108  \\
            0.2192491562167046  0.17984343961381705  \\
            0.2203509107705574  0.17988141104562386  \\
            0.2214526653244102  0.17991933425474313  \\
            0.22255441987826297  0.17995721971588202  \\
            0.22365617443211575  0.17999508497501304  \\
            0.22475792898596855  0.18191101255821973  \\
            0.22585968353982133  0.1819645149990041  \\
            0.2269614380936741  0.18201563052805184  \\
            0.22806319264752692  0.18206446806406926  \\
            0.22916494720137967  0.1821111478083933  \\
            0.23026670175523245  0.18215580336936374  \\
            0.23136845630908526  0.18219858329831454  \\
            0.23247021086293804  0.18223965121352387  \\
            0.23357196541679082  0.1822791840107486  \\
            0.2346737199706436  0.18231736802704462  \\
            0.2357754745244964  0.18235117785024174  \\
            0.23687722907834918  0.18235450373221165  \\
            0.23797898363220196  0.1823424013098921  \\
            0.23908073818605477  0.1823244058799537  \\
            0.24018249273990755  0.18230350494516875  \\
            0.24128424729376033  0.1822813812741595  \\
            0.24238600184761314  0.1822593465040792  \\
            0.24348775640146592  0.18223856304584696  \\
            0.2445895109553187  0.1822200706072351  \\
            0.24569126550917147  0.18220473119537467  \\
            0.24679302006302428  0.1811130573216426  \\
            0.24789477461687706  0.18111949049823184  \\
            0.24899652917072984  0.18113122274198976  \\
            0.25009828372458265  0.1811475686141682  \\
            0.25120003827843546  0.1811673459906127  \\
            0.2523017928322882  0.18118892759588226  \\
            0.253403547386141  0.1812103389923882  \\
            0.25450530193999377  0.1812293998238172  \\
            0.2556070564938466  0.18124389727405182  \\
            0.2567088110476994  0.18125179889394735  \\
            0.25781056560155213  0.18125160882476585  \\
            0.25891232015540494  0.18124324112015844  \\
            0.2600140747092577  0.18092805538281861  \\
            0.2611158292631105  0.18093774089072842  \\
            0.26221758381696325  0.18094025580617978  \\
            0.26331933837081606  0.1809353900011063  \\
            0.26442109292466887  0.18057805477154104  \\
            0.2655228474785216  0.1806021863485334  \\
            0.2666246020323744  0.18062363379479385  \\
            0.26772635658622723  0.18064240398024844  \\
            0.26882811114008  0.18065849829035352  \\
            0.2699298656939328  0.18067191573297747  \\
            0.2710316202477856  0.18068265440393783  \\
            0.27213337480163835  0.18069071220378988  \\
            0.27323512935549116  0.18069608723572902  \\
            0.27433688390934396  0.18069877806734788  \\
            0.2754386384631967  0.1806987839098953  \\
            0.27654039301704947  0.1806961047137331  \\
            0.2776421475709023  0.1806907411680786  \\
            0.2787439021247551  0.180682694606497  \\
            0.27984565667860783  0.18067196682642223  \\
            0.28094741123246064  0.18065855982661302  \\
            0.28204916578631345  0.18064247540636608  \\
            0.2831509203401662  0.1806237144439454  \\
            0.284252674894019  0.1806022754237335  \\
            0.2853544294478718  0.18057815131674276  \\
            0.28645618400172457  0.1809370658697796  \\
            0.2875579385555774  0.18094193046515522  \\
            0.2886596931094302  0.18093941364599564  \\
            0.28976144766328293  0.18092972559838286  \\
            0.29086320221713574  0.18124854642469582  \\
            0.29196495677098855  0.1812569032747157  \\
            0.2930667113248413  0.1812570954532479  \\
            0.2941684658786941  0.18124920271837958  \\
            0.2952702204325469  0.18123471732150345  \\
            0.29637197498639967  0.1812156695063063  \\
            0.2974737295402525  0.1811942705510521  \\
            0.2985754840941052  0.1811726994539406  \\
            0.29967723864795803  0.18115292917150888  \\
            0.30077899320181084  0.18113658524807416  \\
            0.3018807477556636  0.1811248478363797  \\
            0.3029825023095164  0.18111840026143539  \\
            0.3040842568633692  0.18220964003292067  \\
            0.30518601141722196  0.18222497699461734  \\
            0.30628776597107477  0.1822434543266352  \\
            0.3073895205249276  0.18226420928602585  \\
            0.3084912750787803  0.1822862020455642  \\
            0.30959302963263313  0.1823082705338068  \\
            0.31069478418648594  0.1823291036456231  \\
            0.3117965387403387  0.1823470192016389  \\
            0.3128982932941915  0.1823590303835546  \\
            0.3140000478480443  0.18235560159071  \\
            0.31510180240189706  0.18232223365473305  \\
            0.31620355695574986  0.1822840225246488  \\
            0.31730531150960267  0.1822444554661795  \\
            0.3184070660634554  0.18220334565206428  \\
            0.31950882061730823  0.18216051566255745  \\
            0.32061057517116104  0.18211580132287572  \\
            0.3217123297250138  0.18206905349264726  \\
            0.3228140842788666  0.18202013794298105  \\
            0.3239158388327194  0.18196893381855447  \\
            0.32501759338657216  0.1819153315105965  \\
            0.32611934794042496  0.1799996344573617  \\
            0.3272211024942777  0.17996175294663677  \\
            0.3283228570481305  0.17992385212130324  \\
            0.32942461160198333  0.17988591439732418  \\
            0.3305263661558361  0.17984792928229398  \\
            0.3316281207096889  0.17980989329553754  \\
            0.3327298752635417  0.17977180967489706  \\
            0.33383162981739445  0.17973368792327346  \\
            0.33493338437124726  0.17969554324802833  \\
            0.3360351389251  0.17965739594098973  \\
            0.33713689347895276  0.17961927074028688  \\
            0.33823864803280557  0.17958119620655152  \\
            0.3393404025866584  0.17954320413774308  \\
            0.3404421571405111  0.1795053290365007  \\
            0.34154391169436393  0.17946760763579034  \\
            0.34264566624821674  0.17943007848022602  \\
            0.3437474208020695  0.1793927815555033  \\
            0.3448491753559223  0.179355757955381  \\
            0.34595092990977505  0.17931904957174494  \\
            0.34705268446362786  0.17928269879642017  \\
            0.34815443901748067  0.17924674822425513  \\
            0.3492561935713334  0.17921124034840608  \\
            0.3503579481251862  0.17917621724439842  \\
            0.35145970267903903  0.17914172024069785  \\
            0.3525614572328918  0.17910778957870327  \\
            0.3536632117867446  0.1790744640643816  \\
            0.3547649663405974  0.17904178071880644  \\
            0.35586672089445015  0.17900977443294527  \\
            0.35696847544830296  0.17897847763383548  \\
            0.35807023000215576  0.17894791996618867  \\
            0.3591719845560085  0.17891812799700968  \\
            0.3602737391098613  0.1788891249423304  \\
            0.36137549366371413  0.17886093042029108  \\
            0.3624772482175669  0.17883356022706381  \\
            0.3635790027714197  0.17880702613421431  \\
            0.3646807573252725  0.1787813357008085  \\
            0.36578251187912525  0.17875649209540956  \\
            0.36688426643297806  0.17873249391651355  \\
            0.36798602098683086  0.1787093350029453  \\
            0.3690877755406836  0.17868700422178282  \\
            0.3701895300945364  0.17866548522019654  \\
            0.3712912846483892  0.17864475612691716  \\
            0.372393039202242  0.1786247891884638  \\
            0.3734947937560948  0.17860555032266257  \\
            0.37459654830994754  0.17855669352265607  \\
            0.37569830286380035  0.17853801848420436  \\
            0.37680005741765316  0.17852011321906872  \\
            0.3779018119715059  0.17850297082035704  \\
            0.3790035665253587  0.17848658079348162  \\
            0.3801053210792115  0.178470927908878  \\
            0.3812070756330643  0.17845599110001578  \\
            0.3823088301869171  0.17844174242941263  \\
            0.3834105847407699  0.17842814614870092  \\
            0.38451233929462264  0.1784151578865396  \\
            0.38561409384847545  0.17840272400296142  \\
            0.38671584840232825  0.17839078115510462  \\
            0.387817602956181  0.17837925612119063  \\
            0.3889193575100338  0.17836806592527907  \\
            0.3900211120638866  0.17835711829863823  \\
            0.3911228666177394  0.17834631249648994  \\
            0.3922246211715922  0.17833554046990524  \\
            0.393326375725445  0.1783246883613001  \\
            0.39442813027929774  0.17831363826510935  \\
            0.39552988483315055  0.17830227016170355  \\
            0.39663163938700335  0.17829046390932662  \\
            0.39773339394085605  0.17827810116402074  \\
            0.39883514849470886  0.17826506708855927  \\
            0.3999369030485616  0.1782512517264142  \\
            0.4010386576024144  0.17823655094086252  \\
            0.4021404121562672  0.17822086684830996  \\
            0.40324216671012  0.17820410772603074  \\
            0.4043439212639728  0.1781861874048306  \\
            0.4054456758178256  0.17816702420788286  \\
            0.40654743037167834  0.17814653951772488  \\
            0.40764918492553115  0.17812465608587302  \\
            0.40875093947938396  0.17810129620941284  \\
            0.4098526940332367  0.17807637991955483  \\
            0.4109544485870895  0.18057767329289612  \\
            0.4120562031409423  0.1830204928160024  \\
            0.4131579576947951  0.180570195725607  \\
            0.4142597122486479  0.18058612892278086  \\
            0.4153614668025007  0.1780864295693821  \\
            0.41646322135635344  0.17811067482798115  \\
            0.41756497591020625  0.17813337358351589  \\
            0.418666730464059  0.1781546087851599  \\
            0.4197684850179118  0.17817446161237924  \\
            0.4208702395717646  0.17819301329369017  \\
            0.42197199412561737  0.1782103468796932  \\
            0.4230737486794702  0.1782265488301906  \\
            0.424175503233323  0.17824171029561048  \\
            0.42527725778717573  0.17825592798605028  \\
            0.42637901234102854  0.17826930455610196  \\
            0.42748076689488135  0.17828194846520204  \\
            0.4285825214487341  0.17829397331494964  \\
            0.4296842760025869  0.17830549671004933  \\
            0.4307860305564397  0.17831663872537318  \\
            0.43188778511029247  0.17832752009344283  \\
            0.4329895396641453  0.17833826024553537  \\
            0.4340912942179981  0.17834897533831384  \\
            0.43519304877185083  0.17835977639630135  \\
            0.43629480332570364  0.1783707676675586  \\
            0.43739655787955645  0.17838204527486645  \\
            0.4384983124334092  0.1783936962029964  \\
            0.439600066987262  0.17840579763986478  \\
            0.4407018215411148  0.1784184166625315  \\
            0.44180357609496756  0.1784316102399309  \\
            0.4429053306488204  0.1784454255127662  \\
            0.4440070852026732  0.17845990030754913  \\
            0.44510883975652593  0.17847506383725625  \\
            0.44621059431037874  0.17849093755006581  \\
            0.4473123488642315  0.17850753608881628  \\
            0.4484141034180843  0.17852486833417974  \\
            0.4495158579719371  0.17854293850842543  \\
            0.45061761252578986  0.17859164377074954  \\
            0.45171936707964266  0.17861054454066408  \\
            0.45282112163349547  0.17863015373896493  \\
            0.4539228761873482  0.1786505087543683  \\
            0.45502463074120103  0.17867164039116973  \\
            0.45612638529505384  0.17869357336265762  \\
            0.4572281398489066  0.1787163266938123  \\
            0.45832989440275934  0.17873991405055467  \\
            0.45943164895661215  0.17876434401050625  \\
            0.4605334035104649  0.17878962029178658  \\
            0.4616351580643177  0.1788157419518766  \\
            0.4627369126181705  0.17884270357096138  \\
            0.46383866717202327  0.17887049542893008  \\
            0.4649404217258761  0.17889910368779044  \\
            0.4660421762797288  0.1789285105855666  \\
            0.46714393083358163  0.17895869464938208  \\
            0.46824568538743444  0.17898963093012266  \\
            0.4693474399412872  0.17902129126280794  \\
            0.47044919449514  0.17905364455078993  \\
            0.4715509490489928  0.17908665707315033  \\
            0.47265270360284556  0.17912029281122535  \\
            0.47375445815669837  0.1791545137888402  \\
            0.4748562127105512  0.17918928042062213  \\
            0.4759579672644039  0.17922455186101224  \\
            0.47705972181825673  0.17926028634981112  \\
            0.47816147637210954  0.179296441546855  \\
            0.4792632309259623  0.17933297485506128  \\
            0.4803649854798151  0.1793698437301966  \\
            0.4814667400336679  0.17940700598115417  \\
            0.48256849458752066  0.1794444200669949  \\
            0.48367024914137347  0.17948204539855123  \\
            0.4847720036952263  0.17951984265993087  \\
            0.485873758249079  0.17955777415803845  \\
            0.48697551280293183  0.17959580421621563  \\
            0.48807726735678464  0.17963389962050352  \\
            0.4891790219106374  0.17967203012285493  \\
            0.4902807764644902  0.17971016900124798  \\
            0.49138253101834295  0.17974829366624548  \\
            0.49248428557219576  0.1797863862945955  \\
            0.49358604012604856  0.17982443446189708  \\
            0.4946877946799013  0.17986243173628383  \\
            0.4957895492337541  0.17990037818946591  \\
            0.49689130378760693  0.1799382807720694  \\
            0.4979930583414597  0.17997615350128493  \\
            0.4990948128953125  0.18188333487647204  \\
            0.5001965674491653  0.18193806856187045  \\
            0.501298322003018  0.18199036456016174  \\
            0.5024000765568709  0.18204032682241653  \\
            0.5035018311107237  0.18208806964704494  \\
            0.5046035856645764  0.1821337196874035  \\
            0.5057053402184292  0.18217741790394548  \\
            0.506807094772282  0.18221932047761802  \\
            0.5079088493261348  0.18225959801874045  \\
            0.5090106038799875  0.1822984327455543  \\
            0.5101123584338404  0.1823360136955358  \\
            0.5112141129876931  0.1823561125885399  \\
            0.5123158675415459  0.18234949451025784  \\
            0.5134176220953988  0.18233390156461818  \\
            0.5145193766492515  0.1823141997078133  \\
            0.5156211312031043  0.18229250932769997  \\
            0.5167228857569571  0.18227027747090843  \\
            0.5178246403108099  0.18224873013836787  \\
            0.5189263948646626  0.18222897140216646  \\
            0.5200281494185154  0.18221196234277512  \\
            0.5211299039723682  0.18219844110449085  \\
            0.522231658526221  0.1811155885490647  \\
            0.5233334130800738  0.18112472514661065  \\
            0.5244351676339265  0.1811388815951415  \\
            0.5255369221877794  0.18115712055605918  \\
            0.5266386767416321  0.1811780279955711  \\
            0.5277404312954849  0.1811997879217646  \\
            0.5288421858493377  0.18122030297958855  \\
            0.5299439404031905  0.18123735321467777  \\
            0.5310456949570432  0.18124878481438902  \\
            0.5321474495108961  0.18125277068557338  \\
            0.5332492040647488  0.18124834917685922  \\
            0.5343509586186016  0.1809208263037117  \\
            0.5354527131724545  0.1809337409787818  \\
            0.5365544677263072  0.18093993723406088  \\
            0.53765622228016  0.18093869720470085  \\
            0.5387579768340128  0.18093068611210375  \\
            0.5398597313878656  0.18059045686280426  \\
            0.5409614859417183  0.18061324497636735  \\
            0.5420632404955712  0.1806333533774384  \\
            0.543164995049424  0.180650785648365  \\
            0.5442667496032767  0.18066554172850385  \\
            0.5453685041571296  0.18067762004716392  \\
            0.5464702587109823  0.18068701853950023  \\
            0.5475720132648351  0.18069373516868387  \\
            0.5486737678186879  0.18069776824255876  \\
            0.5497755223725407  0.18069911663323196  \\
        }
        ;
    \addplot[color={rgb,1:red,1.0;green,0.0;blue,0.0}, name path={8ea002dd-aef3-4d07-8273-8a7aa4844086}, draw opacity={1.0}, line width={1}, dotted, forget plot]
        table[row sep={\\}]
        {
            \\
            0.0  3.797683922388714  \\
            0.001101754553852787  3.7977184146194385  \\
            0.002203509107705574  3.797822160211135  \\
            0.0033052636615583607  3.7979957961342494  \\
            0.004407018215411148  3.7982403432937186  \\
            0.005508772769263934  3.7985571447510598  \\
            0.0066105273231167215  3.7989477781868253  \\
            0.007712281876969509  3.7994139421166038  \\
            0.008814036430822295  3.799957315536582  \\
            0.009915790984675084  3.8005793902181737  \\
            0.011017545538527868  3.808965848620352  \\
            0.012119300092380656  3.8056822983812664  \\
            0.013221054646233443  3.8037448549337856  \\
            0.01432280920008623  3.802623042567514  \\
            0.015424563753939018  3.802014273442349  \\
            0.016526318307791804  3.8065825233458748  \\
            0.01762807286164459  3.80270599222836  \\
            0.018729827415497377  3.80016614588595  \\
            0.019831581969350167  3.7983460143813264  \\
            0.020933336523202953  3.796830416973339  \\
            0.022035091077055736  3.795306881638736  \\
            0.023136845630908526  3.7935267238872745  \\
            0.024238600184761313  3.7912963976608234  \\
            0.0253403547386141  3.7884843084455326  \\
            0.026442109292466886  3.785032794512539  \\
            0.027543863846319676  3.7809662588604773  \\
            0.02864561840017246  3.7792415333090177  \\
            0.029747372954025245  3.7743047904161675  \\
            0.030849127507878035  3.7692131485958877  \\
            0.03195088206173082  3.7641681773040268  \\
            0.03305263661558361  3.759345276890193  \\
            0.034154391169436395  3.7548769133356346  \\
            0.03525614572328918  3.75084763949142  \\
            0.03635790027714197  3.747299701413508  \\
            0.037459654830994754  3.744246957586141  \\
            0.03856140938484754  3.741692305082493  \\
            0.039663163938700334  3.7400131854046026  \\
            0.04076491849255312  3.738936809455839  \\
            0.04186667304640591  3.7379130912356624  \\
            0.042968427600258687  3.736946328382536  \\
            0.04407018215411147  3.7360414965005115  \\
            0.045171936707964266  3.7352042068373605  \\
            0.04627369126181705  3.7344406851130207  \\
            0.04737544581566984  3.7337578097201156  \\
            0.048477200369522626  3.733163250695898  \\
            0.04957895492337542  3.7326657546586963  \\
            0.0506807094772282  3.732275630404172  \\
            0.051782464031080985  3.7222356526254687  \\
            0.05288421858493377  3.7211865580964996  \\
            0.05398597313878656  3.7201581536167376  \\
            0.05508772769263935  3.7191499198865046  \\
            0.05618948224649214  3.7181613609070063  \\
            0.05729123680034492  3.7171920306358768  \\
            0.058392991354197704  3.7162415514330616  \\
            0.05949474590805049  3.7153096264027554  \\
            0.060596500461903284  3.7143960467838917  \\
            0.06169825501575607  3.7135006950875327  \\
            0.06280000956960886  3.712623544472733  \\
            0.06390176412346164  3.7117646548045387  \\
            0.06500351867731442  3.710924165830212  \\
            0.06610527323116722  3.7101022879638568  \\
            0.06720702778502  3.709299291199042  \\
            0.06830878233887279  3.7085154927005592  \\
            0.06941053689272557  3.7077512436290596  \\
            0.07051229144657836  3.7070069157231518  \\
            0.07161404600043114  3.7062828881181447  \\
            0.07271580055428394  3.70557953480444  \\
            0.07381755510813673  3.7048972130468534  \\
            0.07491930966198951  3.7042362529875743  \\
            0.0760210642158423  3.703596948574821  \\
            0.07712281876969508  3.702979549862795  \\
            0.07822457332354787  3.7023842566677665  \\
            0.07932632787740067  3.701811213499052  \\
            0.08042808243125345  3.7012605056441967  \\
            0.08152983698510624  3.7007321562570157  \\
            0.08263159153895902  3.7002261242810652  \\
            0.08373334609281181  3.6997423030347  \\
            0.0848351006466646  3.699280519284425  \\
            0.08593685520051737  3.6988405326365807  \\
            0.08703860975437017  3.698422035092797  \\
            0.08814036430822295  3.6980246506099923  \\
            0.08924211886207574  3.6976479345224726  \\
            0.09034387341592853  3.6972913726766743  \\
            0.09144562796978131  3.6969543801220106  \\
            0.0925473825236341  3.69663629919019  \\
            0.09364913707748689  3.696336396758601  \\
            0.09475089163133968  3.696053860451331  \\
            0.09585264618519247  3.6957877934603536  \\
            0.09695440073904525  3.6955372075575785  \\
            0.09805615529289805  3.6953010137235123  \\
            0.09915790984675084  3.695078009581346  \\
            0.1002596644006036  3.6935752799013226  \\
            0.1013614189544564  3.693358135222936  \\
            0.10246317350830918  3.69315038389056  \\
            0.10356492806216197  3.6929515845266274  \\
            0.10466668261601475  3.692761252589456  \\
            0.10576843716986754  3.69257886353796  \\
            0.10687019172372034  3.692403859758729  \\
            0.10797194627757312  3.692235661114485  \\
            0.10907370083142591  3.6920736789664863  \\
            0.1101754553852787  3.6919173334604976  \\
            0.11127720993913148  3.691766073718448  \\
            0.11237896449298428  3.6916194003720983  \\
            0.11348071904683706  3.69147688960459  \\
            0.11458247360068984  3.6913382175571936  \\
            0.11568422815454263  3.691203183642098  \\
            0.11678598270839541  3.691071731033477  \\
            0.1178877372622482  3.690943962431663  \\
            0.11898949181610098  3.690820149191304  \\
            0.12009124636995377  3.6907007320914267  \\
            0.12119300092380657  3.690586312453618  \\
            0.12229475547765935  3.6904776329565903  \\
            0.12339651003151214  3.6903755482943223  \\
            0.12449826458536492  3.6902809866935087  \\
            0.12560001913921773  3.6901949041184277  \\
            0.1267017736930705  3.6901182336384837  \\
            0.1278035282469233  3.690051832794273  \\
            0.12890528280077607  3.689996431855964  \\
            0.13000703735462885  3.689952585564061  \\
            0.13110879190848163  3.6899206303818533  \\
            0.13221054646233443  3.6899006484906978  \\
            0.1333123010161872  3.6898924388479903  \\
            0.13441405557004  3.6898954946584697  \\
            0.1355158101238928  3.6945062793606422  \\
            0.13661756467774558  3.6945449221747326  \\
            0.13771931923159836  3.6992172580151195  \\
            0.13882107378545114  3.6945247964569816  \\
            0.13992282833930392  3.6899051536426644  \\
            0.14102458289315672  3.6898968143788506  \\
            0.1421263374470095  3.689899381069381  \\
            0.14322809200086228  3.6899135199505926  \\
            0.1443298465547151  3.6899395815146656  \\
            0.14543160110856787  3.6899776206287593  \\
            0.14653335566242065  3.6900274223170606  \\
            0.14763511021627346  3.6900885343595835  \\
            0.14873686477012624  3.690160306881026  \\
            0.14983861932397902  3.6902419381449167  \\
            0.1509403738778318  3.6903325249073995  \\
            0.1520421284316846  3.6904311149959823  \\
            0.15314388298553738  3.690536759331778  \\
            0.15424563753939016  3.690648560497008  \\
            0.15534739209324297  3.690765715152911  \\
            0.15644914664709575  3.69088754813123  \\
            0.15755090120094853  3.69101353676803  \\
            0.15865265575480134  3.6911433248994476  \\
            0.15975441030865412  3.691276726787025  \\
            0.1608561648625069  3.6914137219763523  \\
            0.1619579194163597  3.6915544426211175  \\
            0.16305967397021248  3.691699155119606  \\
            0.16416142852406526  3.691848237996564  \\
            0.16526318307791804  3.6920021578600455  \\
            0.16636493763177085  3.6921614450376423  \\
            0.16746669218562363  3.6923266701953623  \\
            0.16856844673947638  3.692498422924997  \\
            0.1696702012933292  3.692677292990318  \\
            0.17077195584718197  3.6928638546891857  \\
            0.17187371040103475  3.6930586546028246  \\
            0.17297546495488753  3.6932622029063995  \\
            0.17407721950874033  3.6934749683784958  \\
            0.1751789740625931  3.6936973772784234  \\
            0.1762807286164459  3.695198686502809  \\
            0.1773824831702987  3.6954282254756494  \\
            0.17848423772415148  3.6956715696235847  \\
            0.17958599227800426  3.6959298629535593  \\
            0.18068774683185707  3.6962041428509145  \\
            0.18178950138570985  3.6964953509308898  \\
            0.18289125593956262  3.6968043411307394  \\
            0.18399301049341543  3.69713188573317  \\
            0.1850947650472682  3.697478679807663  \\
            0.186196519601121  3.6978453444449477  \\
            0.18729827415497377  3.6982324290563673  \\
            0.18840002870882658  3.698640412966302  \\
            0.18950178326267936  3.699069706479426  \\
            0.19060353781653214  3.699520651583126  \\
            0.19170529237038494  3.6999935224375227  \\
            0.19280704692423772  3.7004885257997566  \\
            0.1939088014780905  3.7010058015311427  \\
            0.1950105560319433  3.701545423342097  \\
            0.1961123105857961  3.702107399943639  \\
            0.19721406513964887  3.702691676769502  \\
            0.19831581969350168  3.703298138447875  \\
            0.19941757424735443  3.703926612197145  \\
            0.2005193288012072  3.7045768722954526  \\
            0.20162108335506  3.7052486457774125  \\
            0.2027228379089128  3.7059416194459742  \\
            0.20382459246276557  3.7066554482600202  \\
            0.20492634701661835  3.7073897650793275  \\
            0.20602810157047116  3.708144191672622  \\
            0.20712985612432394  3.708918350812304  \\
            0.20823161067817672  3.7097118791797734  \\
            0.2093333652320295  3.71052444071835  \\
            0.2104351197858823  3.7113557399908963  \\
            0.2115368743397351  3.712205535038414  \\
            0.21263862889358787  3.713073649195531  \\
            0.21374038344744067  3.7139599813079234  \\
            0.21484213800129345  3.7148645138133793  \\
            0.21594389255514623  3.715787318182714  \\
            0.21704564710899904  3.716728557256274  \\
            0.21814740166285182  3.7176884840446713  \\
            0.2192491562167046  3.7186674365357253  \\
            0.2203509107705574  3.7196658279411072  \\
            0.2214526653244102  3.7206841314939276  \\
            0.22255441987826297  3.7217228582579533  \\
            0.22365617443211575  3.722782525053324  \\
            0.22475792898596855  3.732473827975327  \\
            0.22585968353982133  3.7329193485577057  \\
            0.2269614380936741  3.733466805560594  \\
            0.22806319264752692  3.7341067480821954  \\
            0.22916494720137967  3.734831015613473  \\
            0.23026670175523245  3.7356323605361927  \\
            0.23136845630908526  3.736504242164519  \\
            0.23247021086293804  3.737440743471507  \\
            0.23357196541679082  3.7384365675604476  \\
            0.2346737199706436  3.7394870738886934  \\
            0.2357754745244964  3.740606182028814  \\
            0.23687722907834918  3.742925299091857  \\
            0.23797898363220196  3.7457293520847483  \\
            0.23908073818605477  3.749030610312223  \\
            0.24018249273990755  3.7528224334277365  \\
            0.24128424729376033  3.757078945133161  \\
            0.24238600184761314  3.76173804227831  \\
            0.24348775640146592  3.766691323633747  \\
            0.2445895109553187  3.771783525187013  \\
            0.24569126550917147  3.7768233600654137  \\
            0.24679302006302428  3.7787494387312948  \\
            0.24789477461687706  3.783088880033095  \\
            0.24899652917072984  3.786855869370918  \\
            0.25009828372458265  3.7899852105703102  \\
            0.25120003827843546  3.7924939796698345  \\
            0.2523017928322882  3.794478252194822  \\
            0.253403547386141  3.796101980543501  \\
            0.25450530193999377  3.7975864129590176  \\
            0.2556070564938466  3.7992095694544883  \\
            0.2567088110476994  3.8013276154675104  \\
            0.25781056560155213  3.8044383622479505  \\
            0.25891232015540494  3.8093339426744595  \\
            0.2600140747092577  3.802273431444899  \\
            0.2611158292631105  3.803109475747574  \\
            0.26221758381696325  3.804590921747688  \\
            0.26331933837081606  3.807112674937476  \\
            0.26442109292466887  3.800920814126657  \\
            0.2655228474785216  3.8002588865233253  \\
            0.2666246020323744  3.799676275668409  \\
            0.26772635658622723  3.7991716555881263  \\
            0.26882811114008  3.7987434180989292  \\
            0.2699298656939328  3.7983898551753033  \\
            0.2710316202477856  3.798109312793656  \\
            0.27213337480163835  3.7979003179589736  \\
            0.27323512935549116  3.797761679318983  \\
            0.27433688390934396  3.797692561744326  \\
            0.2754386384631967  3.7976925354754147  \\
            0.27654039301704947  3.7977616004839247  \\
            0.2776421475709023  3.7979001864733886  \\
            0.2787439021247551  3.798109128520179  \\
            0.27984565667860783  3.7983896179246264  \\
            0.28094741123246064  3.798743127632001  \\
            0.28204916578631345  3.7991713116173265  \\
            0.2831509203401662  3.7996758778559103  \\
            0.284252674894019  3.8002584344826675  \\
            0.2853544294478718  3.800920307430794  \\
            0.28645618400172457  3.8071078261112454  \\
            0.2875579385555774  3.804585944915427  \\
            0.2886596931094302  3.80310440828051  \\
            0.28976144766328293  3.802268303322145  \\
            0.29086320221713574  3.8093227458593546  \\
            0.29196495677098855  3.8044258325420603  \\
            0.2930667113248413  3.801314070165447  \\
            0.2941684658786941  3.7991952294650653  \\
            0.2952702204325469  3.797571433959923  \\
            0.29637197498639967  3.796086474243435  \\
            0.2974737295402525  3.794462301016162  \\
            0.2985754840941052  3.7924776465346843  \\
            0.29967723864795803  3.7899685449303404  \\
            0.30077899320181084  3.786838910418834  \\
            0.3018807477556636  3.7830716582274873  \\
            0.3029825023095164  3.7787319772511543  \\
            0.3040842568633692  3.7768054558643813  \\
            0.30518601141722196  3.7717653963208644  \\
            0.30628776597107477  3.766672998949528  \\
            0.3073895205249276  3.7617195680434494  \\
            0.3084912750787803  3.757060389559977  \\
            0.30959302963263313  3.752803886797283  \\
            0.31069478418648594  3.7490121812626  \\
            0.3117965387403387  3.7457111617324848  \\
            0.3128982932941915  3.7429074742715907  \\
            0.3140000478480443  3.7405888584239566  \\
            0.31510180240189706  3.7394686562898176  \\
            0.31620355695574986  3.738418115904142  \\
            0.31730531150960267  3.737422294790829  \\
            0.3184070660634554  3.7364858348245162  \\
            0.31950882061730823  3.7356140342450033  \\
            0.32061057517116104  3.734812811412178  \\
            0.3217123297250138  3.7340887083096215  \\
            0.3228140842788666  3.7334489737868215  \\
            0.3239158388327194  3.7329017694718  \\
            0.32501759338657216  3.7324565472158104  \\
            0.32611934794042496  3.7227681181027172  \\
            0.3272211024942777  3.7217084866913264  \\
            0.3283228570481305  3.720669802220761  \\
            0.32942461160198333  3.7196515473132874  \\
            0.3305263661558361  3.7186532103511407  \\
            0.3316281207096889  3.7176743175596965  \\
            0.3327298752635417  3.7167144552082854  \\
            0.33383162981739445  3.7157732848158247  \\
            0.33493338437124726  3.7148505529091387  \\
            0.3360351389251  3.7139460962165103  \\
            0.33713689347895276  3.713059842868406  \\
            0.33823864803280557  3.7121918100607645  \\
            0.3393404025866584  3.711342098613808  \\
            0.3404421571405111  3.710510884890073  \\
            0.34154391169436393  3.7096984105746444  \\
            0.34264566624821674  3.7089049708583572  \\
            0.3437474208020695  3.7081309015783686  \\
            0.3448491753559223  3.7073765658566797  \\
            0.34595092990977505  3.7066423407468605  \\
            0.34705268446362786  3.705928604325644  \\
            0.34815443901748067  3.7052357235984155  \\
            0.3492561935713334  3.7045640434879403  \\
            0.3503579481251862  3.7039138770873077  \\
            0.35145970267903903  3.7032854972748828  \\
            0.3525614572328918  3.702679129696347  \\
            0.3536632117867446  3.7020949470730256  \\
            0.3547649663405974  3.7015330647259472  \\
            0.35586672089445015  3.700993537183973  \\
            0.35696847544830296  3.7004763557091565  \\
            0.35807023000215576  3.6999814465757512  \\
            0.3591719845560085  3.6995086699173245  \\
            0.3602737391098613  3.699057818983312  \\
            0.36137549366371413  3.698628619628874  \\
            0.3624772482175669  3.698220729893792  \\
            0.3635790027714197  3.697833739509881  \\
            0.3646807573252725  3.697467169200534  \\
            0.36578251187912525  3.6971204696104465  \\
            0.36688426643297806  3.6967930197163135  \\
            0.36798602098683086  3.696484124524915  \\
            0.3690877755406836  3.6961930118391004  \\
            0.3701895300945364  3.6959188278165764  \\
            0.3712912846483892  3.6956606309451248  \\
            0.372393039202242  3.695417383950225  \\
            0.3734947937560948  3.695187942943858  \\
            0.37459654830994754  3.6936875009869086  \\
            0.37569830286380035  3.693465507500073  \\
            0.37680005741765316  3.6932531119524916  \\
            0.3779018119715059  3.6930498945028614  \\
            0.3790035665253587  3.692855392039173  \\
            0.3801053210792115  3.692669099385044  \\
            0.3812070756330643  3.692490474367781  \\
            0.3823088301869171  3.6923189465727644  \\
            0.3834105847407699  3.692153929648817  \\
            0.38451233929462264  3.691994836996541  \\
            0.38561409384847545  3.691841100563473  \\
            0.38671584840232825  3.6916921922924093  \\
            0.387817602956181  3.6915476475322753  \\
            0.3889193575100338  3.691407089425194  \\
            0.3900211120638866  3.6912702529667096  \\
            0.3911228666177394  3.691137007138567  \\
            0.3922246211715922  3.6910073732783117  \\
            0.393326375725445  3.690881537758976  \\
            0.39442813027929774  3.690759857130065  \\
            0.39552988483315055  3.6906428541893894  \\
            0.39663163938700335  3.690531203984372  \\
            0.39773339394085605  3.690425709473944  \\
            0.39883514849470886  3.6903272674369902  \\
            0.3999369030485616  3.6902368260601865  \\
            0.4010386576024144  3.6901553363824777  \\
            0.4021404121562672  3.690083700298761  \\
            0.40324216671012  3.690022718019722  \\
            0.4043439212639728  3.6899730377777535  \\
            0.4054456758178256  3.689935110114992  \\
            0.40654743037167834  3.689909148409841  \\
            0.40764918492553115  3.6898950964210506  \\
            0.40875093947938396  3.689892602697892  \\
            0.4098526940332367  3.6899010006996393  \\
            0.4109544485870895  3.6945246391330495  \\
            0.4120562031409423  3.699217292135886  \\
            0.4131579576947951  3.694545059654916  \\
            0.4142597122486479  3.6945064596307957  \\
            0.4153614668025007  3.6898996752056754  \\
            0.41646322135635344  3.689896685234444  \\
            0.41756497591020625  3.6899049748926362  \\
            0.418666730464059  3.6899250503100687  \\
            0.4197684850179118  3.689957111336815  \\
            0.4208702395717646  3.6900010742852807  \\
            0.42197199412561737  3.6900566010379583  \\
            0.4230737486794702  3.690123135182572  \\
            0.424175503233323  3.690199944855422  \\
            0.42527725778717573  3.6902861710721866  \\
            0.42637901234102854  3.690380879518839  \\
            0.42748076689488135  3.6904831132155107  \\
            0.4285825214487341  3.6905919431686347  \\
            0.4296842760025869  3.6907065141757065  \\
            0.4307860305564397  3.690826083306987  \\
            0.43188778511029247  3.6909500492484026  \\
            0.4329895396641453  3.6910779714806106  \\
            0.4340912942179981  3.6912095791599704  \\
            0.43519304877185083  3.691344770343953  \\
            0.43629480332570364  3.6914836028676317  \\
            0.43739655787955645  3.6916262785820444  \\
            0.4384983124334092  3.691773122874661  \\
            0.439600066987262  3.691924561371403  \\
            0.4407018215411148  3.692081095550935  \\
            0.44180357609496756  3.69224327872903  \\
            0.4429053306488204  3.6924116935581246  \\
            0.4440070852026732  3.692586931873818  \\
            0.44510883975652593  3.6927695774552416  \\
            0.44621059431037874  3.6929601920529906  \\
            0.4473123488642315  3.693159304904504  \\
            0.4484141034180843  3.6933674058730412  \\
            0.4495158579719371  3.6935849423667806  \\
            0.45061761252578986  3.6950887038133646  \\
            0.45171936707964266  3.695311806374897  \\
            0.45282112163349547  3.6955480977534787  \\
            0.4539228761873482  3.695798780447814  \\
            0.45502463074120103  3.6960649435920727  \\
            0.45612638529505384  3.6963475755213966  \\
            0.4572281398489066  3.6966475731433976  \\
            0.45832989440275934  3.6969657489233705  \\
            0.45943164895661215  3.697302836065409  \\
            0.4605334035104649  3.6976594923093984  \\
            0.4616351580643177  3.6980363026679965  \\
            0.4627369126181705  3.698433781346844  \\
            0.46383866717202327  3.6988523730535494  \\
            0.4649404217258761  3.6992924538626943  \\
            0.4660421762797288  3.699754331794516  \\
            0.46714393083358163  3.7002382472531545  \\
            0.46824568538743444  3.700744373472998  \\
            0.4693474399412872  3.7012728171258376  \\
            0.47044919449514  3.7018236192464378  \\
            0.4715509490489928  3.702396756648802  \\
            0.47265270360284556  3.7029921440021303  \\
            0.47375445815669837  3.7036096367411266  \\
            0.4748562127105512  3.7042490349811845  \\
            0.4759579672644039  3.704910088586743  \\
            0.47705972181825673  3.7055925035145885  \\
            0.47816147637210954  3.706295949511239  \\
            0.4792632309259623  3.707020069185637  \\
            0.4803649854798151  3.70776448840268  \\
            0.4814667400336679  3.708528827862829  \\
            0.48256849458752066  3.7093127156426235  \\
            0.48367024914137347  3.710115800374074  \\
            0.4847720036952263  3.7109377646584494  \\
            0.485873758249079  3.7117783382434406  \\
            0.48697551280293183  3.7126373104265684  \\
            0.48807726735678464  3.7135145411420822  \\
            0.4891790219106374  3.7144099701749385  \\
            0.4902807764644902  3.7153236239832768  \\
            0.49138253101834295  3.7162556196414673  \\
            0.49248428557219576  3.7172061654630553  \\
            0.49358604012604856  3.7181755578658455  \\
            0.4946877946799013  3.7191641739829397  \\
            0.4957895492337541  3.72017245932655  \\
            0.49689130378760693  3.7212009093458738  \\
            0.4979930583414597  3.722250042782385  \\
            0.4990948128953125  3.732292744656371  \\
            0.5001965674491653  3.7326831903366275  \\
            0.501298322003018  3.7331809617762555  \\
            0.5024000765568709  3.73377575101254  \\
            0.5035018311107237  3.7344588124713325  \\
            0.5046035856645764  3.7352224772957148  \\
            0.5057053402184292  3.736059868363127  \\
            0.506807094772282  3.7369647612724815  \\
            0.5079088493261348  3.737931546116054  \\
            0.5090106038799875  3.7389552486302065  \\
            0.5101123584338404  3.7400315724959703  \\
            0.5112141129876931  3.7417098984030446  \\
            0.5123158675415459  3.7442649811581847  \\
            0.5134176220953988  3.7473180267685358  \\
            0.5145193766492515  3.750866141845376  \\
            0.5156211312031043  3.754895476989776  \\
            0.5167228857569571  3.7593638017497355  \\
            0.5178246403108099  3.7641865838618593  \\
            0.5189263948646626  3.769231379914222  \\
            0.5200281494185154  3.7743228097411627  \\
            0.5211299039723682  3.77925931779231  \\
            0.522231658526221  3.7809836030391955  \\
            0.5233334130800738  3.7850498881974666  \\
            0.5244351676339265  3.7885011250687355  \\
            0.5255369221877794  3.791312902539353  \\
            0.5266386767416321  3.7935428729636222  \\
            0.5277404312954849  3.7953226192817895  \\
            0.5288421858493377  3.7968456714758236  \\
            0.5299439404031905  3.7983606901762745  \\
            0.5310456949570432  3.800180111476934  \\
            0.5321474495108961  3.8027190624879106  \\
            0.5332492040647488  3.8065944339024216  \\
            0.5343509586186016  3.8020194214449794  \\
            0.5354527131724545  3.802628143899211  \\
            0.5365544677263072  3.8037498811160004  \\
            0.53765622228016  3.8056872166936055  \\
            0.5387579768340128  3.8089706141438198  \\
            0.5398597313878656  3.800579869531509  \\
            0.5409614859417183  3.79995774041193  \\
            0.5420632404955712  3.79941431296285  \\
            0.543164995049424  3.7989480953667054  \\
            0.5442667496032767  3.798557408577135  \\
            0.5453685041571296  3.79824055402931  \\
            0.5464702587109823  3.7979959539932437  \\
            0.5475720132648351  3.797822265357961  \\
            0.5486737678186879  3.797718467164688  \\
            0.5497755223725407  3.7976839223888548  \\
        }
        ;
\end{axis}
\end{tikzpicture}

}


\caption{ Translation results for Hessian $n=16$ for $500$ iterations, $\delta =(0, 2 \sqrt{2}h) $ for $L=2.7$ with a circle $R=1$       }

\end{figure}

\begin{figure}[htp]
\centering

\subfloat[]{%
% Recommended preamble:
% \usetikzlibrary{arrows.meta}
% \usetikzlibrary{backgrounds}
% \usepgfplotslibrary{patchplots}
% \usepgfplotslibrary{fillbetween}
% \pgfplotsset{%
%     layers/standard/.define layer set={%
%         background,axis background,axis grid,axis ticks,axis lines,axis tick labels,pre main,main,axis descriptions,axis foreground%
%     }{
%         grid style={/pgfplots/on layer=axis grid},%
%         tick style={/pgfplots/on layer=axis ticks},%
%         axis line style={/pgfplots/on layer=axis lines},%
%         label style={/pgfplots/on layer=axis descriptions},%
%         legend style={/pgfplots/on layer=axis descriptions},%
%         title style={/pgfplots/on layer=axis descriptions},%
%         colorbar style={/pgfplots/on layer=axis descriptions},%
%         ticklabel style={/pgfplots/on layer=axis tick labels},%
%         axis background@ style={/pgfplots/on layer=axis background},%
%         3d box foreground style={/pgfplots/on layer=axis foreground},%
%     },
% }

\begin{tikzpicture}[/tikz/background rectangle/.style={fill={rgb,1:red,1.0;green,1.0;blue,1.0}, fill opacity={1.0}, draw opacity={1.0}}, show background rectangle]
\begin{axis}[point meta max={nan}, point meta min={nan}, legend cell align={left}, legend columns={1}, title={}, title style={at={{(0.5,1)}}, anchor={south}, font={{\fontsize{14 pt}{18.2 pt}\selectfont}}, color={rgb,1:red,0.0;green,0.0;blue,0.0}, draw opacity={1.0}, rotate={0.0}, align={center}}, legend style={color={rgb,1:red,0.0;green,0.0;blue,0.0}, draw opacity={1.0}, line width={1}, solid, fill={rgb,1:red,1.0;green,1.0;blue,1.0}, fill opacity={1.0}, text opacity={1.0}, font={{\fontsize{12 pt}{15.600000000000001 pt}\selectfont}}, text={rgb,1:red,0.0;green,0.0;blue,0.0}, cells={anchor={center}}, at={(1.02, 1)}, anchor={north west}}, axis background/.style={fill={rgb,1:red,1.0;green,1.0;blue,1.0}, opacity={1.0}}, anchor={north west}, xshift={1.0mm}, yshift={-1.0mm}, width={145.4mm}, height={48.8mm}, scaled x ticks={false}, xlabel={}, x tick style={color={rgb,1:red,0.0;green,0.0;blue,0.0}, opacity={1.0}}, x tick label style={color={rgb,1:red,0.0;green,0.0;blue,0.0}, opacity={1.0}, rotate={0}}, xlabel style={at={(ticklabel cs:0.5)}, anchor=near ticklabel, at={{(ticklabel cs:0.5)}}, anchor={near ticklabel}, font={{\fontsize{11 pt}{14.3 pt}\selectfont}}, color={rgb,1:red,0.0;green,0.0;blue,0.0}, draw opacity={1.0}, rotate={0.0}}, xmajorgrids={true}, xmin={-0.014318912319027599}, xmax={0.49161598961994724}, xticklabels={{$0.0$,$0.1$,$0.2$,$0.3$,$0.4$}}, xtick={{0.0,0.1,0.2,0.30000000000000004,0.4}}, xtick align={inside}, xticklabel style={font={{\fontsize{8 pt}{10.4 pt}\selectfont}}, color={rgb,1:red,0.0;green,0.0;blue,0.0}, draw opacity={1.0}, rotate={0.0}}, x grid style={color={rgb,1:red,0.0;green,0.0;blue,0.0}, draw opacity={0.1}, line width={0.5}, solid}, axis x line*={left}, x axis line style={color={rgb,1:red,0.0;green,0.0;blue,0.0}, draw opacity={1.0}, line width={1}, solid}, scaled y ticks={false}, ylabel={}, y tick style={color={rgb,1:red,0.0;green,0.0;blue,0.0}, opacity={1.0}}, y tick label style={color={rgb,1:red,0.0;green,0.0;blue,0.0}, opacity={1.0}, rotate={0}}, ylabel style={at={(ticklabel cs:0.5)}, anchor=near ticklabel, at={{(ticklabel cs:0.5)}}, anchor={near ticklabel}, font={{\fontsize{11 pt}{14.3 pt}\selectfont}}, color={rgb,1:red,0.0;green,0.0;blue,0.0}, draw opacity={1.0}, rotate={0.0}}, ymode={log}, log basis y={10}, ymajorgrids={true}, ymin={0.20417379446695233}, ymax={4.8977881936844765e23}, yticklabels={{$10^{0}$,$10^{5}$,$10^{10}$,$10^{15}$,$10^{20}$}}, ytick={{1.0,100000.0,1.0e10,1.0e15,1.0e20}}, ytick align={inside}, yticklabel style={font={{\fontsize{8 pt}{10.4 pt}\selectfont}}, color={rgb,1:red,0.0;green,0.0;blue,0.0}, draw opacity={1.0}, rotate={0.0}}, y grid style={color={rgb,1:red,0.0;green,0.0;blue,0.0}, draw opacity={0.1}, line width={0.5}, solid}, axis y line*={left}, y axis line style={color={rgb,1:red,0.0;green,0.0;blue,0.0}, draw opacity={1.0}, line width={1}, solid}, colorbar={false}]
    [\addlegendimage{empty legend}] \addlegendentry[font={{\fontsize{11 pt}{14.3 pt}\selectfont}}, text={rgb,1:red,0.0;green,0.0;blue,0.0}] {\hspace{-.6cm}{\textbf{$(\gamma, \gamma_1, \gamma_2)$}}}
    \addplot[color={rgb,1:red,0.0;green,0.0;blue,1.0}, name path={6b6e567c-0536-4ccb-85ac-ff2f726f8688}, draw opacity={1.0}, line width={1}, solid]
        table[row sep={\\}]
        {
            \\
            0.0  2.4570950441198234e7  \\
            0.0009565071689397187  2.4575177817058776e7  \\
            0.0019130143378794373  2.457939432153537e7  \\
            0.002869521506819156  2.458359993262882e7  \\
            0.0038260286757588746  2.458779465674972e7  \\
            0.004782535844698593  2.4591978499855544e7  \\
            0.005739043013638312  2.459615149276407e7  \\
            0.0066955501825780315  2.4600313711532637e7  \\
            0.007652057351517749  2.4604465240478504e7  \\
            0.008608564520457468  2.4608606200320814e7  \\
            0.009565071689397187  2.461273674546548e7  \\
            0.010521578858336907  2.4616857075354844e7  \\
            0.011478086027276624  2.462096741405805e7  \\
            0.012434593196216343  2.4625068048178114e7  \\
            0.013391100365156063  2.4629159291351434e7  \\
            0.01434760753409578  2.4633241499038637e7  \\
            0.015304114703035498  2.4637315092569552e7  \\
            0.016260621871975217  2.4641380518918514e7  \\
            0.017217129040914936  2.907160783788938e7  \\
            0.018173636209854658  3.01736550733328e7  \\
            0.019130143378794373  3.0161159674663432e7  \\
            0.020086650547734092  3.014553019906001e7  \\
            0.021043157716673814  3.0126793965248656e7  \\
            0.02199966488561353  3.0104971701026954e7  \\
            0.022956172054553248  3.008007733370636e7  \\
            0.02391267922349297  3.0052119811621603e7  \\
            0.024869186392432685  3.0021104900275115e7  \\
            0.025825693561372404  2.9987036114925243e7  \\
            0.026782200730312126  2.9949914985242825e7  \\
            0.027738707899251844  2.9909740491733182e7  \\
            0.02869521506819156  2.9866507994435318e7  \\
            0.02965172223713128  2.9820208143328153e7  \\
            0.030608229406070997  2.9921530302125357e7  \\
            0.031564736575010716  3.004689185085508e7  \\
            0.032521243743950434  3.0102519944424547e7  \\
            0.03347775091289015  3.0075287682572857e7  \\
            0.03443425808182987  3.00129865087198e7  \\
            0.0353907652507696  2.9947594719055988e7  \\
            0.036347272419709316  2.988531269685183e7  \\
            0.03730377958864903  2.999752899209736e7  \\
            0.038260286757588746  3.0099219153132316e7  \\
            0.039216793926528465  3.0190026228974443e7  \\
            0.040173301095468184  3.0269684739036746e7  \\
            0.04112980826440791  3.0338010495639905e7  \\
            0.04208631543334763  3.039489075425501e7  \\
            0.043042822602287346  3.0440274686266273e7  \\
            0.04399932977122706  3.0474454318662822e7  \\
            0.04495583694016678  3.0497795129235696e7  \\
            0.045912344109106495  3.0341716710751604e7  \\
            0.046868851278046214  3.0771626401125457e7  \\
            0.04782535844698594  3.076151458712502e7  \\
            0.04878186561592566  3.0749446394353345e7  \\
            0.04973837278486537  3.073509797094306e7  \\
            0.05069487995380509  3.0718203611125216e7  \\
            0.05165138712274481  3.069876201103822e7  \\
            0.052607894291684526  3.06767678203732e7  \\
            0.05356440146062425  3.0652212190914962e7  \\
            0.05452090862956397  3.0625083300759733e7  \\
            0.05547741579850369  3.0595366822776075e7  \\
            0.0564339229674434  3.056304643253768e7  \\
            0.05739043013638312  3.0528104234280046e7  \\
            0.05834693730532284  3.049052128220792e7  \\
            0.05930344447426256  3.0335482560209706e7  \\
            0.060259951643202275  3.0331179618278597e7  \\
            0.061216458812141994  3.0326477862591956e7  \\
            0.06217296598108171  3.032137135337371e7  \\
            0.06312947315002143  3.0315854469616923e7  \\
            0.06408598031896115  3.0309922067050803e7  \\
            0.06504248748790087  3.0303569589425206e7  \\
            0.06599899465684059  3.0296793219212066e7  \\
            0.0669555018257803  3.0289589822027303e7  \\
            0.06791200899472002  3.0284066007855423e7  \\
            0.06886851616365974  3.0278268767731234e7  \\
            0.06982502333259948  3.027204638185062e7  \\
            0.0707815305015392  3.026539292424087e7  \\
            0.07173803767047891  3.0258609103185315e7  \\
            0.07269454483941863  3.025252874422468e7  \\
            0.07365105200835834  3.026546132730754e7  \\
            0.07460755917729806  3.0282587366997167e7  \\
            0.07556406634623777  3.0290099721479796e7  \\
            0.07652057351517749  3.0707451620736662e7  \\
            0.07747708068411721  3.0714726576558206e7  \\
            0.07843358785305693  3.071931162905674e7  \\
            0.07939009502199665  3.0721196993066136e7  \\
            0.08034660219093637  3.0720367076699384e7  \\
            0.08130310935987609  3.112420066965612e7  \\
            0.08225961652881582  3.1108195263269275e7  \\
            0.08321612369775554  3.1091651801831283e7  \\
            0.08417263086669526  3.1074561353713505e7  \\
            0.08512913803563497  3.1056914988972094e7  \\
            0.08608564520457469  3.103870374273628e7  \\
            0.0870421523735144  3.1019918641520966e7  \\
            0.08799865954245412  3.1000550734756198e7  \\
            0.08895516671139383  3.098059106057635e7  \\
            0.08991167388033355  3.0960030676214512e7  \\
            0.09086818104927327  3.093886065484303e7  \\
            0.09182468821821299  3.091707213473941e7  \\
            0.09278119538715271  3.0894656260436695e7  \\
            0.09373770255609243  3.0871604273380786e7  \\
            0.09469420972503216  3.0847907456246473e7  \\
            0.09565071689397188  3.082355716856913e7  \\
            0.0966072240629116  3.0798544869435646e7  \\
            0.09756373123185132  3.077286208489275e7  \\
            0.09852023840079104  3.0802839391262364e7  \\
            0.09947674556973074  3.0794974625745077e7  \\
            0.10043325273867046  3.0782357008080017e7  \\
            0.10138975990761018  3.076508009316329e7  \\
            0.1023462670765499  3.0743226851306308e7  \\
            0.10330277424548961  3.0716871362234257e7  \\
            0.10425928141442933  3.068608048380502e7  \\
            0.10521578858336905  3.0650915350635044e7  \\
            0.10617229575230877  3.0611432740062624e7  \\
            0.1071288029212485  3.0567686386869814e7  \\
            0.10808531009018822  3.0519728125949744e7  \\
            0.10904181725912794  3.0467608958313305e7  \\
            0.10999832442806766  3.0411380024363186e7  \\
            0.11095483159700738  3.0351093489779077e7  \\
            0.1119113387659471  3.028680333180436e7  \\
            0.1128678459348868  3.021856601931354e7  \\
            0.11382435310382652  3.0146441204509348e7  \\
            0.11478086027276624  3.0070492202054854e7  \\
            0.11573736744170596  2.999078653415546e7  \\
            0.11669387461064568  2.990739633085573e7  \\
            0.1176503817795854  2.9820398654502526e7  \\
            0.11860688894852511  2.9729875815354533e7  \\
            0.11956339611746483  2.9683319727943894e7  \\
            0.12051990328640455  2.9775572556039862e7  \\
            0.12147641045534427  2.986434344364219e7  \\
            0.12243291762428399  2.994954734457105e7  \\
            0.1233894247932237  3.0031104590266336e7  \\
            0.12434593196216343  3.0108940642423145e7  \\
            0.12530243913110314  3.0182985717834253e7  \\
            0.12625894630004286  3.025317446277133e7  \\
            0.12721545346898258  3.0319445441135745e7  \\
            0.1281719606379223  3.0381740672812056e7  \\
            0.12912846780686202  3.044000502522762e7  \\
            0.13008497497580174  3.0494185508416772e7  \\
            0.13104148214474146  3.054423058240205e7  \\
            0.13199798931368117  3.0590089297724113e7  \\
            0.1329544964826209  3.0631710350395396e7  \\
            0.1339110036515606  3.0669041105470937e7  \\
            0.13486751082050033  3.070202644595319e7  \\
            0.13582401798944005  3.073060759898191e7  \\
            0.13678052515837977  3.075472073663347e7  \\
            0.1377370323273195  3.0774295582931016e7  \\
            0.13869353949625923  3.0789253838792805e7  \\
            0.13965004666519895  3.0799507466265783e7  \\
            0.14060655383413867  3.0759766658113934e7  \\
            0.1415630610030784  3.0785787822977725e7  \\
            0.1425195681720181  3.0811134324066084e7  \\
            0.14347607534095783  3.083581455384143e7  \\
            0.14443258250989754  3.085983702791809e7  \\
            0.14538908967883726  3.088321034622242e7  \\
            0.14634559684777695  3.0905943185495753e7  \\
            0.14730210401671667  3.0928044278174613e7  \\
            0.1482586111856564  3.094952244300753e7  \\
            0.1492151183545961  3.0970386527896952e7  \\
            0.15017162552353583  3.099064544400026e7  \\
            0.15112813269247555  3.1010308113748e7  \\
            0.15208463986141527  3.1029383498960465e7  \\
            0.15304114703035498  3.1047880543252785e7  \\
            0.1539976541992947  3.1065808235873237e7  \\
            0.15495416136823442  3.1083175516989663e7  \\
            0.15591066853717414  3.1099991353891876e7  \\
            0.15686717570611386  3.1116264672947556e7  \\
            0.15782368287505358  3.071892721989415e7  \\
            0.1587801900439933  3.072112263113543e7  \\
            0.15973669721293302  3.0720592560371246e7  \\
            0.16069320438187273  3.071735578501008e7  \\
            0.16164971155081245  3.0711424902309578e7  \\
            0.16260621871975217  3.070280701640095e7  \\
            0.16356272588869192  3.028756890506517e7  \\
            0.16451923305763164  3.027520267830903e7  \\
            0.16547574022657136  3.0253408298281793e7  \\
            0.16643224739551107  3.0255624442407522e7  \\
            0.1673887545644508  3.026190263755165e7  \\
            0.1683452617333905  3.026877392038857e7  \\
            0.16930176890233023  3.027521109110161e7  \\
            0.17025827607126995  3.028122016581096e7  \\
            0.17121478324020967  3.0286807018515814e7  \\
            0.17217129040914939  3.0293245076011453e7  \\
            0.17312779757808908  3.0300234606798284e7  \\
            0.1740843047470288  3.0306798598103933e7  \\
            0.1750408119159685  3.0312940514241602e7  \\
            0.17599731908490823  3.0318664549035527e7  \\
            0.17695382625384795  3.0323975562352255e7  \\
            0.17791033342278767  3.0328878969629582e7  \\
            0.1788668405917274  3.0333380566261657e7  \\
            0.1798233477606671  3.047073344989923e7  \\
            0.18077985492960683  3.0509644111982655e7  \\
            0.18173636209854654  3.0545904257205084e7  \\
            0.18269286926748626  3.0579533241080735e7  \\
            0.18364937643642598  3.0610549524144746e7  \\
            0.1846058836053657  3.063897028156015e7  \\
            0.18556239077430542  3.066481089942761e7  \\
            0.18651889794324514  3.068808450350587e7  \\
            0.18747540511218486  3.070880149652774e7  \\
            0.18843191228112458  3.0726969055326845e7  \\
            0.18938841945006432  3.0742590559392318e7  \\
            0.19034492661900404  3.0755665084410887e7  \\
            0.19130143378794376  3.0766890594557967e7  \\
            0.19225794095688348  3.033877680218117e7  \\
            0.1932144481258232  3.0344449802868787e7  \\
            0.19417095529476291  3.0487550792879924e7  \\
            0.19512746246370263  3.0458654324025735e7  \\
            0.19608396963264235  3.0419020755771715e7  \\
            0.19704047680158207  3.0367885808494892e7  \\
            0.1979969839705218  3.0305272795080584e7  \\
            0.19895349113946148  3.02312623205531e7  \\
            0.1999099983084012  3.014600165498742e7  \\
            0.20086650547734092  3.004971437743299e7  \\
            0.20182301264628064  2.9942710415348355e7  \\
            0.20277951981522035  2.9913742699475113e7  \\
            0.20373602698416007  2.998067646755411e7  \\
            0.2046925341530998  3.0044523909437925e7  \\
            0.2056490413220395  3.0105277288137622e7  \\
            0.20660554849097923  3.008417027192958e7  \\
            0.20756205565991895  2.9992178280128945e7  \\
            0.20851856282885867  2.983644933996319e7  \\
            0.20947506999779839  2.984374228712415e7  \\
            0.2104315771667381  2.9888506993970137e7  \\
            0.21138808433567782  2.9930209627512947e7  \\
            0.21234459150461754  2.9968857126821715e7  \\
            0.21330109867355726  3.0004452133418042e7  \\
            0.214257605842497  3.0036994294982463e7  \\
            0.21521411301143673  3.00664810884507e7  \\
            0.21617062018037644  3.0092907918077253e7  \\
            0.21712712734931616  3.0116267612076912e7  \\
            0.21808363451825588  3.0136548989436653e7  \\
            0.2190401416871956  3.015373489637281e7  \\
            0.21999664885613532  3.0167801020986613e7  \\
            0.22095315602507504  2.907045198528429e7  \\
            0.22190966319401476  2.90719838190647e7  \\
            0.22286617036295447  2.4639348789469156e7  \\
            0.2238226775318942  2.4635279345810197e7  \\
            0.2247791847008339  2.4631201497335e7  \\
            0.2257356918697736  2.4627114818020053e7  \\
            0.22669219903871332  2.4623018926657304e7  \\
            0.22764870620765304  2.4618913472579613e7  \\
            0.22860521337659276  2.4614798167866807e7  \\
            0.22956172054553248  2.4610672760244556e7  \\
            0.2305182277144722  2.4606537028148998e7  \\
            0.23147473488341191  2.460239080608322e7  \\
            0.23243124205235163  2.4598233943630286e7  \\
            0.23338774922129135  2.459406634688763e7  \\
            0.23434425639023107  2.4589887935538877e7  \\
            0.2353007635591708  2.4585698650929164e7  \\
            0.2362572707281105  2.4581498485237643e7  \\
            0.23721377789705023  2.457728742830275e7  \\
            0.23817028506598995  2.4573065485630758e7  \\
            0.23912679223492966  2.4573065482373662e7  \\
            0.24008329940386938  2.457728742529129e7  \\
            0.2410398065728091  2.4581498486895002e7  \\
            0.24199631374174882  2.458569865382936e7  \\
            0.24295282091068854  2.4589887935241114e7  \\
            0.24390932807962826  2.459406634834946e7  \\
            0.24486583524856798  2.4598233948105216e7  \\
            0.2458223424175077  2.460239080223208e7  \\
            0.2467788495864474  2.46065370279639e7  \\
            0.24773535675538713  2.4610672756986853e7  \\
            0.24869186392432685  2.461479817131057e7  \\
            0.24964837109326657  2.461891347184445e7  \\
            0.2506048782622063  2.4623018923826165e7  \\
            0.25156138543114603  2.4627114818794254e7  \\
            0.2525178926000857  2.4631201500742715e7  \\
            0.25347439976902547  2.4635279346763823e7  \\
            0.25443090693796516  2.4639348795084365e7  \\
            0.2553874141069049  2.9071983783845942e7  \\
            0.2563439212758446  2.9070451950598136e7  \\
            0.25730042844478435  3.01678009788245e7  \\
            0.25825693561372404  3.0153734865744185e7  \\
            0.2592134427826638  3.0136548951409586e7  \\
            0.2601699499516035  3.0116267571264926e7  \\
            0.2611264571205432  3.0092907877280835e7  \\
            0.2620829642894829  3.0066481046867378e7  \\
            0.26303947145842266  3.003699426113173e7  \\
            0.26399597862736235  3.0004452094181716e7  \\
            0.2649524857963021  2.9968857093126345e7  \\
            0.2659089929652418  2.99302095937406e7  \\
            0.26686550013418153  2.9888506958824255e7  \\
            0.2678220073031212  2.9843742251627482e7  \\
            0.26877851447206097  2.9836449375373743e7  \\
            0.26973502164100066  2.9992178310932744e7  \\
            0.2706915288099404  3.008417030059203e7  \\
            0.2716480359788801  3.010527725209657e7  \\
            0.27260454314781984  3.0044523874954846e7  \\
            0.27356105031675954  2.9980676430762e7  \\
            0.2745175574856993  2.9913742664578613e7  \\
            0.275474064654639  2.9942710449694216e7  \\
            0.2764305718235787  3.0049714409293562e7  \\
            0.27738707899251847  3.0146001691738196e7  \\
            0.27834358616145816  3.0231262356109045e7  \\
            0.2793000933303979  3.030527281502146e7  \\
            0.2802566004993376  3.0367885833233006e7  \\
            0.28121310766827734  3.04190207692425e7  \\
            0.28216961483721703  3.0458654332807556e7  \\
            0.2831261220061568  3.0487550802924424e7  \\
            0.28408262917509647  3.0344449771424025e7  \\
            0.2850391363440362  3.033877680241027e7  \\
            0.2859956435129759  3.0766890560436744e7  \\
            0.28695215068191565  3.0755665040552486e7  \\
            0.28790865785085534  3.0742590515173417e7  \\
            0.2888651650197951  3.072696901698025e7  \\
            0.2898216721887348  3.0708801464417815e7  \\
            0.2907781793576745  3.068808445893715e7  \\
            0.29173468652661416  3.0664810856705133e7  \\
            0.2926911936955539  3.063897024290705e7  \\
            0.2936477008644936  3.061054949263729e7  \\
            0.29460420803343335  3.057953319898957e7  \\
            0.29556071520237304  3.0545904219456498e7  \\
            0.2965172223713128  3.050964406797651e7  \\
            0.2974737295402525  3.0470733406598862e7  \\
            0.2984302367091922  3.033338056168998e7  \\
            0.2993867438781319  3.0328878968684807e7  \\
            0.30034325104707166  3.0323975567967772e7  \\
            0.3012997582160114  3.031866454672022e7  \\
            0.3022562653849511  3.03129405027257e7  \\
            0.30321277255389084  3.0306798595403954e7  \\
            0.30416927972283053  3.030023460266286e7  \\
            0.3051257868917703  3.0293245066209886e7  \\
            0.30608229406070997  3.0286807013139594e7  \\
            0.3070388012296497  3.0281220161414873e7  \\
            0.3079953083985894  3.02752110857404e7  \\
            0.30895181556752915  3.026877391659413e7  \\
            0.30990832273646884  3.0261902631018434e7  \\
            0.3108648299054086  3.0255624439983536e7  \\
            0.3118213370743483  3.025340833444095e7  \\
            0.31277784424328803  3.027520270679611e7  \\
            0.3137343514122277  3.0287568941408742e7  \\
            0.31469085858116747  3.0702807055033367e7  \\
            0.31564736575010716  3.0711424937590323e7  \\
            0.3166038729190469  3.071735582099643e7  \\
            0.3175603800879866  3.072059259378963e7  \\
            0.31851688725692634  3.0721122668494757e7  \\
            0.31947339442586603  3.071892725537317e7  \\
            0.3204299015948058  3.111626466844076e7  \\
            0.32138640876374547  3.1099991340138778e7  \\
            0.3223429159326852  3.108317550425999e7  \\
            0.3232994231016249  3.1065808216852106e7  \\
            0.32425593027056465  3.104788053543975e7  \\
            0.32521243743950434  3.1029383479557272e7  \\
            0.3261689446084441  3.1010308099328134e7  \\
            0.32712545177738384  3.0990645428899355e7  \\
            0.3280819589463235  3.097038651052753e7  \\
            0.3290384661152633  3.0949522421516128e7  \\
            0.32999497328420296  3.0928044264919538e7  \\
            0.3309514804531427  3.090594316230229e7  \\
            0.3319079876220824  3.0883210333416384e7  \\
            0.33286449479102215  3.0859837008729413e7  \\
            0.33382100195996184  3.083581453549271e7  \\
            0.3347775091289016  3.08111342973057e7  \\
            0.3357340162978413  3.078578780087762e7  \\
            0.336690523466781  3.0759766639093738e7  \\
            0.3376470306357207  3.0799507427978713e7  \\
            0.33860353780466046  3.0789253806501605e7  \\
            0.33956004497360015  3.0774295549455315e7  \\
            0.3405165521425399  3.0754720695578974e7  \\
            0.3414730593114796  3.0730607564238086e7  \\
            0.34242956648041933  3.0702026408382367e7  \\
            0.343386073649359  3.0669041059616394e7  \\
            0.34434258081829877  3.063171031172745e7  \\
            0.3452990879872384  3.059008925423312e7  \\
            0.34625559515617815  3.054423055075131e7  \\
            0.34721210232511784  3.0494185467334226e7  \\
            0.3481686094940576  3.0440004982362393e7  \\
            0.3491251166629973  3.038174063866572e7  \\
            0.350081623831937  3.0319445397941872e7  \\
            0.3510381310008768  3.0253174418414272e7  \\
            0.35199463816981647  3.018298568030272e7  \\
            0.3529511453387562  3.010894060455665e7  \\
            0.3539076525076959  3.0031104558951944e7  \\
            0.35486415967663565  2.9949547302579995e7  \\
            0.35582066684557534  2.9864343409090124e7  \\
            0.3567771740145151  2.9775572524318885e7  \\
            0.3577336811834548  2.9683319683590423e7  \\
            0.3586901883523945  2.9729875851715174e7  \\
            0.3596466955213342  2.9820398684070054e7  \\
            0.36060320269027396  2.9907396364054307e7  \\
            0.36155970985921365  2.999078657556043e7  \\
            0.3625162170281534  3.007049223366633e7  \\
            0.3634727241970931  3.014644123456949e7  \\
            0.36442923136603284  3.021856606096142e7  \\
            0.3653857385349725  3.0286803367960364e7  \\
            0.3663422457039123  3.0351093533642422e7  \\
            0.36729875287285196  3.0411380058893193e7  \\
            0.3682552600417917  3.0467608989377838e7  \\
            0.3692117672107314  3.051972815728923e7  \\
            0.37016827437967115  3.0567686423309222e7  \\
            0.37112478154861084  3.0611432771321993e7  \\
            0.3720812887175506  3.0650915386305217e7  \\
            0.3730377958864903  3.0686080518180557e7  \\
            0.37399430305543  3.0716871399124347e7  \\
            0.3749508102243697  3.0743226886467367e7  \\
            0.37590731739330946  3.076508012939139e7  \\
            0.37686382456224915  3.0782357036973022e7  \\
            0.3778203317311889  3.079497466202439e7  \\
            0.37877683890012864  3.0802839425481606e7  \\
            0.37973334606906833  3.0772862095347285e7  \\
            0.3806898532380081  3.079854488607746e7  \\
            0.38164636040694777  3.082355719096101e7  \\
            0.3826028675758875  3.0847907471594803e7  \\
            0.3835593747448272  3.0871604297254715e7  \\
            0.38451588191376695  3.0894656282668184e7  \\
            0.38547238908270665  3.0917072144794162e7  \\
            0.3864288962516464  3.0938860675577164e7  \\
            0.3873854034205861  3.0960030692258693e7  \\
            0.38834191058952583  3.09805910728807e7  \\
            0.3892984177584655  3.1000550753608078e7  \\
            0.39025492492740527  3.1019918659720954e7  \\
            0.39121143209634496  3.1038703754629113e7  \\
            0.3921679392652847  3.1056914996125918e7  \\
            0.3931244464342244  3.1074561370037157e7  \\
            0.39408095360316414  3.1091651805671602e7  \\
            0.39503746077210383  3.1108195271524064e7  \\
            0.3959939679410436  3.1124200679101568e7  \\
            0.39695047510998327  3.0720367044622265e7  \\
            0.39790698227892296  3.07211969570261e7  \\
            0.39886348944786265  3.0719311586340696e7  \\
            0.3998199966168024  3.0714726537941653e7  \\
            0.4007765037857421  3.070745157826378e7  \\
            0.40173301095468184  3.0290099680435043e7  \\
            0.4026895181236216  3.0282587324158646e7  \\
            0.4036460252925613  3.026546128350644e7  \\
            0.404602532461501  3.025252874853202e7  \\
            0.4055590396304407  3.0258609105273053e7  \\
            0.40651554679938046  3.0265392930485714e7  \\
            0.40747205396832015  3.0272046386204123e7  \\
            0.4084285611372599  3.0278268772783525e7  \\
            0.4093850683061996  3.028406601181682e7  \\
            0.41034157547513933  3.028958982317097e7  \\
            0.411298082644079  3.029679322055194e7  \\
            0.41225458981301877  3.0303569598112736e7  \\
            0.41321109698195846  3.0309922068390656e7  \\
            0.4141676041508982  3.031585447098303e7  \\
            0.4151241113198379  3.0321371358573902e7  \\
            0.41608061848877764  3.0326477868632194e7  \\
            0.41703712565771733  3.0331179618346915e7  \\
            0.4179936328266571  3.0335482557138223e7  \\
            0.41895013999559677  3.0490521318382397e7  \\
            0.4199066471645365  3.0528104272652082e7  \\
            0.4208631543334762  3.0563046463415507e7  \\
            0.42181966150241595  3.0595366856836822e7  \\
            0.42277616867135565  3.062508333946821e7  \\
            0.4237326758402954  3.065221222926261e7  \\
            0.4246891830092351  3.0676767850883096e7  \\
            0.42564569017817483  3.0698762046829924e7  \\
            0.4266021973471145  3.071820364809627e7  \\
            0.42755870451605427  3.073509800733173e7  \\
            0.428515211684994  3.0749446428056683e7  \\
            0.4294717188539337  3.0761514622226343e7  \\
            0.43042822602287345  3.0771626430848684e7  \\
            0.43138473319181314  3.0341716742912374e7  \\
            0.4323412403607529  3.0497795118115388e7  \\
            0.4332977475296926  3.04744543101396e7  \\
            0.4342542546986323  3.0440274669962056e7  \\
            0.435210761867572  3.039489073781782e7  \\
            0.43616726903651176  3.033801047451784e7  \\
            0.43712377620545145  3.026968470659531e7  \\
            0.4380802833743912  3.0190026198443636e7  \\
            0.4390367905433309  3.0099219119303986e7  \\
            0.43999329771227064  2.999752896140491e7  \\
            0.4409498048812103  2.9885312654516544e7  \\
            0.4419063120501501  2.9947594754010115e7  \\
            0.44286281921908976  3.001298654301016e7  \\
            0.4438193263880295  3.0075287713397577e7  \\
            0.4447758335569692  3.010251990591001e7  \\
            0.44573234072590895  3.004689181892752e7  \\
            0.44668884789484864  2.9921530265269466e7  \\
            0.4476453550637884  2.982020817819402e7  \\
            0.4486018622327281  2.9866508027666938e7  \\
            0.4495583694016678  2.9909740526872296e7  \\
            0.4505148765706075  2.9949915016551923e7  \\
            0.4514713837395472  2.9987036150972072e7  \\
            0.45242789090848695  3.0021104930279184e7  \\
            0.45338439807742664  3.0052119847611327e7  \\
            0.4543409052463664  3.0080077370535545e7  \\
            0.4552974124153061  3.0104971734610923e7  \\
            0.4562539195842458  3.0126793994229548e7  \\
            0.4572104267531855  3.014553022954214e7  \\
            0.45816693392212526  3.0161159707895633e7  \\
            0.45912344109106495  3.0173655108579237e7  \\
            0.4600799482600047  2.907160786676577e7  \\
            0.4610364554289444  2.4641380512272116e7  \\
            0.46199296259788414  2.463731508761835e7  \\
            0.46294946976682383  2.463324150187979e7  \\
            0.4639059769357636  2.462915928578683e7  \\
            0.46486248410470327  2.4625068044178445e7  \\
            0.465818991273643  2.4620967416665543e7  \\
            0.4667754984425827  2.4616857067734804e7  \\
            0.46773200561152245  2.461273673887529e7  \\
            0.46868851278046214  2.4608606194144487e7  \\
            0.4696450199494019  2.4604465234844e7  \\
            0.4706015271183416  2.4600313714814134e7  \\
            0.4715580342872813  2.459615149524198e7  \\
            0.472514541456221  2.4591978496477418e7  \\
            0.47347104862516076  2.4587794650482167e7  \\
            0.47442755579410045  2.45835999353068e7  \\
            0.4753840629630402  2.4579394320014752e7  \\
            0.4763405701319799  2.4575177812048133e7  \\
            0.47729707730091964  2.4570950435266025e7  \\
        }
        ;
    \addlegendentry {$(20.0, 10.0, 0.5) $}
    \addplot[color={rgb,1:red,1.0;green,0.0;blue,0.0}, name path={c8b1aeb1-58b2-4cb9-ab42-5a47cc9639fc}, draw opacity={1.0}, line width={1}, solid]
        table[row sep={\\}]
        {
            \\
            0.0  8.939008519076015e9  \\
            0.0009565071689397187  1.3826755787084549e10  \\
            0.0019130143378794373  2.1845573828891613e10  \\
            0.002869521506819156  3.538174415400113e10  \\
            0.0038260286757588746  5.8993659861484276e10  \\
            0.004782535844698593  1.0177583977302077e11  \\
            0.005739043013638312  1.8282061669420212e11  \\
            0.0066955501825780315  3.446639302476399e11  \\
            0.007652057351517749  6.89042815846331e11  \\
            0.008608564520457468  1.4804593582444592e12  \\
            0.009565071689397187  3.4851347619935e12  \\
            0.010521578858336907  9.236258543164941e12  \\
            0.011478086027276624  2.8720547666490316e13  \\
            0.012434593196216343  1.1210196377109917e14  \\
            0.013391100365156063  6.203310260652621e14  \\
            0.01434760753409578  6.291201174936238e15  \\
            0.015304114703035498  2.4025306959790736e17  \\
            0.016260621871975217  3.55470011833026e21  \\
            0.017217129040914936  7.785483774539897e10  \\
            0.018173636209854658  2.996175105921447e16  \\
            0.019130143378794373  3.588855389296854e13  \\
            0.020086650547734092  2.035004609196768e12  \\
            0.021043157716673814  3.1169748548576184e11  \\
            0.02199966488561353  1.6805173352975668e11  \\
            0.022956172054553248  1.9834396781299347e11  \\
            0.02391267922349297  2.3514235775175546e11  \\
            0.024869186392432685  2.8008142360864594e11  \\
            0.025825693561372404  3.352710693984947e11  \\
            0.026782200730312126  4.034519664406695e11  \\
            0.027738707899251844  4.8821275052523566e11  \\
            0.02869521506819156  5.942896068316727e11  \\
            0.02965172223713128  7.279864851202438e11  \\
            0.030608229406070997  8.97770034122171e11  \\
            0.031564736575010716  1.1151215289547913e12  \\
            0.032521243743950434  1.3957706584429585e12  \\
            0.03347775091289015  7.331095137519722e16  \\
            0.03443425808182987  5.726433923219986e14  \\
            0.0353907652507696  4.413350665252235e13  \\
            0.036347272419709316  7.709867950139438e12  \\
            0.03730377958864903  4.918922352527987e12  \\
            0.038260286757588746  6.540412630540542e12  \\
            0.039216793926528465  8.81332243490471e12  \\
            0.040173301095468184  2.2038707928091516e13  \\
            0.04112980826440791  7.905020691219112e13  \\
            0.04208631543334763  3.863276708726628e14  \\
            0.043042822602287346  3.181028598630001e15  \\
            0.04399932977122706  7.67895260921663e16  \\
            0.04495583694016678  7.702821455234086e19  \\
            0.045912344109106495  1.1898364914478716e14  \\
            0.046868851278046214  1.0e23  \\
            0.04782535844698594  3.591128470196195e17  \\
            0.04878186561592566  8.347170780733437e15  \\
            0.04973837278486537  1.0332318833868214e15  \\
            0.05069487995380509  2.0324596669290772e15  \\
            0.05165138712274481  4.3286654245749505e15  \\
            0.052607894291684526  1.0193937556510028e16  \\
            0.05356440146062425  2.7375431702328508e16  \\
            0.05452090862956397  8.792407143649294e16  \\
            0.05547741579850369  3.657711037139831e17  \\
            0.0564339229674434  2.2886999040912635e18  \\
            0.05739043013638312  3.0174786798037524e19  \\
            0.05834693730532284  2.507079648437468e21  \\
            0.05930344447426256  6.413319971353387e11  \\
            0.060259951643202275  5.197635062694138e11  \\
            0.061216458812141994  4.635708839360576e11  \\
            0.06217296598108171  4.873515048735434e11  \\
            0.06312947315002143  9.58510870658198e11  \\
            0.06408598031896115  2.0669078828236188e11  \\
            0.06504248748790087  1.1024039376921675e12  \\
            0.06599899465684059  1.7832277070095862e11  \\
            0.0669555018257803  7.784904239791014e10  \\
            0.06791200899472002  4.190993459346536e10  \\
            0.06886851616365974  2.492979658960681e10  \\
            0.06982502333259948  1.5762747525208256e10  \\
            0.0707815305015392  1.039805904742587e10  \\
            0.07173803767047891  7.080448979150426e9  \\
            0.07269454483941863  4.94349255549799e9  \\
            0.07365105200835834  3.5225793623125157e9  \\
            0.07460755917729806  2.553252163515623e9  \\
            0.07556406634623777  1.8778012403913858e9  \\
            0.07652057351517749  1.8863112925164388e18  \\
            0.07747708068411721  1.1428839127126374e16  \\
            0.07843358785305693  8.93436533829132e14  \\
            0.07939009502199665  1.7319663757546144e14  \\
            0.08034660219093637  5.551828405550436e13  \\
            0.08130310935987609  3.5890161193655007e21  \\
            0.08225961652881582  2.1331931286717658e17  \\
            0.08321612369775554  5.77684600578272e15  \\
            0.08417263086669526  5.880881801127709e14  \\
            0.08512913803563497  1.0910073074500139e14  \\
            0.08608564520457469  2.8560561546684836e13  \\
            0.0870421523735144  9.351881753552654e12  \\
            0.08799865954245412  3.583466857048173e12  \\
            0.08895516671139383  1.5428090897684578e12  \\
            0.08991167388033355  8.870065362011023e11  \\
            0.09086818104927327  2.0438540622339612e12  \\
            0.09182468821821299  5.247966193918293e12  \\
            0.09278119538715271  1.55537970655395e13  \\
            0.09373770255609243  5.62349960543442e13  \\
            0.09469420972503216  2.7260416596308953e14  \\
            0.09565071689397188  2.1264888078442828e15  \\
            0.0966072240629116  4.113125927164131e16  \\
            0.09756373123185132  9.970487192059294e18  \\
            0.09852023840079104  6.685260179500391e9  \\
            0.09947674556973074  4.498524920148409e9  \\
            0.10043325273867046  3.0650881697404766e9  \\
            0.10138975990761018  2.1092175131508179e9  \\
            0.1023462670765499  1.461990625991943e9  \\
            0.10330277424548961  1.0177047419567645e9  \\
            0.10425928141442933  7.08929626481291e8  \\
            0.10521578858336905  4.9188722030327296e8  \\
            0.10617229575230877  3.376907549250713e8  \\
            0.1071288029212485  2.8472631735809135e8  \\
            0.10808531009018822  3.102367849793143e8  \\
            0.10904181725912794  3.416813476194751e8  \\
            0.10999832442806766  3.887957551412766e9  \\
            0.11095483159700738  4.253124966885904e8  \\
            0.1119113387659471  4.833222352996191e8  \\
            0.1128678459348868  5.61743138602973e8  \\
            0.11382435310382652  6.716009031609267e8  \\
            0.11478086027276624  8.357410948807799e8  \\
            0.11573736744170596  1.1066240806914692e9  \\
            0.11669387461064568  1.6363778983017652e9  \\
            0.1176503817795854  3.107029699920634e9  \\
            0.11860688894852511  2.1239921095444874e10  \\
            0.11956339611746483  1.3898858081225788e10  \\
            0.12051990328640455  5.51606715014098e9  \\
            0.12147641045534427  2.147630844542372e9  \\
            0.12243291762428399  1.3204618669572704e9  \\
            0.1233894247932237  9.522842008965001e8  \\
            0.12434593196216343  7.446768016756355e8  \\
            0.12530243913110314  6.117009804762638e8  \\
            0.12625894630004286  5.1947969307288504e8  \\
            0.12721545346898258  4.521549684671089e8  \\
            0.1281719606379223  4.03236432859101e8  \\
            0.12912846780686202  5.354016427797664e8  \\
            0.13008497497580174  3.2499944299956614e8  \\
            0.13104148214474146  2.968927300190607e8  \\
            0.13199798931368117  2.77840565367776e8  \\
            0.1329544964826209  4.0831967059139436e8  \\
            0.1339110036515606  5.910210648783476e8  \\
            0.13486751082050033  8.495450316450083e8  \\
            0.13582401798944005  1.2193834558686647e9  \\
            0.13678052515837977  1.7547541135499775e9  \\
            0.1377370323273195  2.539888402776487e9  \\
            0.13869353949625923  3.7080728491060514e9  \\
            0.13965004666519895  5.474530377005848e9  \\
            0.14060655383413867  3.970279216084333e21  \\
            0.1415630610030784  3.649957542077611e17  \\
            0.1425195681720181  7.957706539463643e15  \\
            0.14347607534095783  7.048960821224888e14  \\
            0.14443258250989754  1.1836062515927497e14  \\
            0.14538908967883726  2.8715387773999113e13  \\
            0.14634559684777695  8.84899105208793e12  \\
            0.14730210401671667  3.2250422047459424e12  \\
            0.1482586111856564  1.330591672755854e12  \\
            0.1492151183545961  1.048350831400494e12  \\
            0.15017162552353583  2.3216969669685884e12  \\
            0.15112813269247555  5.693203341189602e12  \\
            0.15208463986141527  1.5971483457117389e13  \\
            0.15304114703035498  5.3970220555716445e13  \\
            0.1539976541992947  2.399406089147199e14  \\
            0.15495416136823442  1.665670452068793e15  \\
            0.15591066853717414  2.7177448594204284e16  \\
            0.15686717570611386  4.767118441470547e18  \\
            0.15782368287505358  3.6270098287766336e13  \\
            0.1587801900439933  9.318011192054942e13  \\
            0.15973669721293302  3.637650052230261e14  \\
            0.16069320438187273  2.7206020476165125e15  \\
            0.16164971155081245  8.278898632109947e16  \\
            0.16260621871975217  3.170982008768662e21  \\
            0.16356272588869192  2.185937743566649e9  \\
            0.16451923305763164  2.9931319063012733e9  \\
            0.16547574022657136  4.1633357442313323e9  \\
            0.16643224739551107  5.899719870860133e9  \\
            0.1673887545644508  8.550554334646245e9  \\
            0.1683452617333905  1.2744783633035063e10  \\
            0.16930176890233023  1.970078845536606e10  \\
            0.17025827607126995  3.2023277640785427e10  \\
            0.17121478324020967  5.618840146125923e10  \\
            0.17217129040914939  1.1320020820254913e11  \\
            0.17312779757808908  3.311439058297294e11  \\
            0.1740843047470288  9.960007082675403e11  \\
            0.1750408119159685  8.205002480235715e11  \\
            0.17599731908490823  5.747792900343732e11  \\
            0.17695382625384795  4.616065782864312e11  \\
            0.17791033342278767  4.838357632433135e11  \\
            0.1788668405917274  5.715665854299962e11  \\
            0.1798233477606671  1.0e23  \\
            0.18077985492960683  1.880915504455856e20  \\
            0.18173636209854654  7.302170090143355e18  \\
            0.18269286926748626  8.572243612569379e17  \\
            0.18364937643642598  1.724174173049993e17  \\
            0.1846058836053657  4.778576757819e16  \\
            0.18556239077430542  1.6392789204592964e16  \\
            0.18651889794324514  6.549086844182744e15  \\
            0.18747540511218486  2.933447300140283e15  \\
            0.18843191228112458  1.4363250356468668e15  \\
            0.18938841945006432  2.469165093236452e15  \\
            0.19034492661900404  3.934251636460751e16  \\
            0.19130143378794376  1.3107063314234294e19  \\
            0.19225794095688348  1.502950578877144e14  \\
            0.1932144481258232  9.497218714118734e13  \\
            0.19417095529476291  9.636223092038993e17  \\
            0.19512746246370263  1.2797030897787598e16  \\
            0.19608396963264235  1.0167616433720786e15  \\
            0.19704047680158207  1.6654178649726662e14  \\
            0.1979969839705218  4.047887401291157e13  \\
            0.19895349113946148  1.2621765474385648e13  \\
            0.1999099983084012  7.579033497360532e12  \\
            0.20086650547734092  5.662963778379549e12  \\
            0.20182301264628064  4.2857075732441987e12  \\
            0.20277951981522035  1.7217932112436363e13  \\
            0.20373602698416007  1.3732230847964486e14  \\
            0.2046925341530998  3.9114237831312885e15  \\
            0.2056490413220395  3.877325320447809e19  \\
            0.20660554849097923  1.2463404195795527e12  \\
            0.20756205565991895  9.996340515554292e11  \\
            0.20851856282885867  8.07732717236578e11  \\
            0.20947506999779839  6.572155446506666e11  \\
            0.2104315771667381  5.382353744394856e11  \\
            0.21138808433567782  4.434949815971707e11  \\
            0.21234459150461754  3.6753622210994934e11  \\
            0.21330109867355726  3.062406434332074e11  \\
            0.214257605842497  2.5647503694394843e11  \\
            0.21521411301143673  2.158422266682666e11  \\
            0.21617062018037644  1.824714791376804e11  \\
            0.21712712734931616  1.549356813520545e11  \\
            0.21808363451825588  7.360678436284402e11  \\
            0.2190401416871956  7.0902734951026e12  \\
            0.21999664885613532  3.693506569177321e14  \\
            0.22095315602507504  8.374085347065341e10  \\
            0.22190966319401476  7.244225590590582e10  \\
            0.22286617036295447  5.382080475934783e18  \\
            0.2238226775318942  3.012250128709419e16  \\
            0.2247791847008339  1.7838811261471512e15  \\
            0.2257356918697736  2.4962965496134753e14  \\
            0.22669219903871332  5.483273524643095e13  \\
            0.22764870620765304  1.5910806985432812e13  \\
            0.22860521337659276  5.578137813179613e12  \\
            0.22956172054553248  2.242445083580996e12  \\
            0.2305182277144722  9.998296929809229e11  \\
            0.23147473488341191  4.8340932694167505e11  \\
            0.23243124205235163  2.4936483909064435e11  \\
            0.23338774922129135  1.3565838803183849e11  \\
            0.23434425639023107  7.712888740616837e10  \\
            0.2353007635591708  4.550811971692307e10  \\
            0.2362572707281105  2.7708736620728508e10  \\
            0.23721377789705023  1.7329988042644665e10  \\
            0.23817028506598995  1.1090283414452368e10  \\
            0.23912679223492966  1.1090283414139172e10  \\
            0.24008329940386938  1.7329988043978703e10  \\
            0.2410398065728091  2.7708736620775986e10  \\
            0.24199631374174882  4.5508119714886406e10  \\
            0.24295282091068854  7.71288873981817e10  \\
            0.24390932807962826  1.3565838803464e11  \\
            0.24486583524856798  2.4936483914891165e11  \\
            0.2458223424175077  4.8340932651965204e11  \\
            0.2467788495864474  9.998296922018705e11  \\
            0.24773535675538713  2.242445084487568e12  \\
            0.24869186392432685  5.578137795680781e12  \\
            0.24964837109326657  1.5910806973751203e13  \\
            0.2506048782622063  5.483273572025131e13  \\
            0.25156138543114603  2.4962966509762903e14  \\
            0.2525178926000857  1.7838811608314235e15  \\
            0.25347439976902547  3.012250690553355e16  \\
            0.25443090693796516  5.382103226247797e18  \\
            0.2553874141069049  7.244225588394377e10  \\
            0.2563439212758446  8.37408534530838e10  \\
            0.25730042844478435  3.693506341769237e14  \\
            0.25825693561372404  7.090273500198148e12  \\
            0.2592134427826638  7.360678442293765e11  \\
            0.2601699499516035  1.549356813179392e11  \\
            0.2611264571205432  1.824714791204574e11  \\
            0.2620829642894829  2.1584222671034622e11  \\
            0.26303947145842266  2.5647503681358554e11  \\
            0.26399597862736235  3.0624064364909045e11  \\
            0.2649524857963021  3.675362217833041e11  \\
            0.2659089929652418  4.434949810588284e11  \\
            0.26686550013418153  5.3823537376505853e11  \\
            0.2678220073031212  6.572155445410918e11  \\
            0.26877851447206097  8.077327154815587e11  \\
            0.26973502164100066  9.996340504743337e11  \\
            0.2706915288099404  1.2463404167783813e12  \\
            0.2716480359788801  3.8773349413164474e19  \\
            0.27260454314781984  3.911423705650689e15  \\
            0.27356105031675954  1.3732230841341134e14  \\
            0.2745175574856993  1.7217932201631633e13  \\
            0.275474064654639  4.2857075890063604e12  \\
            0.2764305718235787  5.662963799629337e12  \\
            0.27738707899251847  7.579033508704118e12  \\
            0.27834358616145816  1.2621765486422367e13  \\
            0.2793000933303979  4.047887447052634e13  \\
            0.2802566004993376  1.6654178963773938e14  \\
            0.28121310766827734  1.0167616394084674e15  \\
            0.28216961483721703  1.2797029614236274e16  \\
            0.2831261220061568  9.63620661912073e17  \\
            0.28408262917509647  9.497218900385217e13  \\
            0.2850391363440362  1.502950593301985e14  \\
            0.2859956435129759  1.3107142470685385e19  \\
            0.28695215068191565  3.934247711412966e16  \\
            0.28790865785085534  2.46916462134663e15  \\
            0.2888651650197951  1.4363251751567542e15  \\
            0.2898216721887348  2.9334474137693165e15  \\
            0.2907781793576745  6.549086097695477e15  \\
            0.29173468652661416  1.6392788635041996e16  \\
            0.2926911936955539  4.778580170326555e16  \\
            0.2936477008644936  1.724173093563409e17  \\
            0.29460420803343335  8.572224728146414e17  \\
            0.29556071520237304  7.302184110153609e18  \\
            0.2965172223713128  1.8809632043395973e20  \\
            0.2974737295402525  1.0e23  \\
            0.2984302367091922  5.715665868782766e11  \\
            0.2993867438781319  4.838357631939054e11  \\
            0.30034325104707166  4.6160657645445184e11  \\
            0.3012997582160114  5.747792891894912e11  \\
            0.3022562653849511  8.205002500504387e11  \\
            0.30321277255389084  9.960006894733274e11  \\
            0.30416927972283053  3.3114390668372754e11  \\
            0.3051257868917703  1.1320020836855856e11  \\
            0.30608229406070997  5.618840146481923e10  \\
            0.3070388012296497  3.202327764537808e10  \\
            0.3079953083985894  1.9700788460485195e10  \\
            0.30895181556752915  1.2744783631959135e10  \\
            0.30990832273646884  8.550554335050983e9  \\
            0.3108648299054086  5.899719870844479e9  \\
            0.3118213370743483  4.1633357443354e9  \\
            0.31277784424328803  2.9931319063571534e9  \\
            0.3137343514122277  2.185937743780786e9  \\
            0.31469085858116747  3.171199303433156e21  \\
            0.31564736575010716  8.278916507534013e16  \\
            0.3166038729190469  2.7206017795135745e15  \\
            0.3175603800879866  3.637650134421436e14  \\
            0.31851688725692634  9.318011292889261e13  \\
            0.31947339442586603  3.627009841759382e13  \\
            0.3204299015948058  4.767123935370036e18  \\
            0.32138640876374547  2.7177460178967964e16  \\
            0.3223429159326852  1.665670658654935e15  \\
            0.3232994231016249  2.399405991552644e14  \\
            0.32425593027056465  5.3970220281856445e13  \\
            0.32521243743950434  1.597148351345897e13  \\
            0.3261689446084441  5.693203333795265e12  \\
            0.32712545177738384  2.321696959633732e12  \\
            0.3280819589463235  1.0483508325636621e12  \\
            0.3290384661152633  1.3305916731647246e12  \\
            0.32999497328420296  3.225042200001061e12  \\
            0.3309514804531427  8.848991054252979e12  \\
            0.3319079876220824  2.8715387859630863e13  \\
            0.33286449479102215  1.1836062973569095e14  \\
            0.33382100195996184  7.048961892758899e14  \\
            0.3347775091289016  7.957705711363891e15  \\
            0.3357340162978413  3.6499492015346675e17  \\
            0.336690523466781  3.9641470631418837e21  \\
            0.3376470306357207  5.47453037693907e9  \\
            0.33860353780466046  3.708072848927601e9  \\
            0.33956004497360015  2.5398884023105364e9  \\
            0.3405165521425399  1.7547541132923229e9  \\
            0.3414730593114796  1.219383455641906e9  \\
            0.34242956648041933  8.4954503142732e8  \\
            0.343386073649359  5.910210646634351e8  \\
            0.34434258081829877  4.083196704064087e8  \\
            0.3452990879872384  2.778405651850825e8  \\
            0.34625559515617815  2.968927299442431e8  \\
            0.34721210232511784  3.2499944291563976e8  \\
            0.3481686094940576  5.354018906750611e8  \\
            0.3491251166629973  4.032364327089749e8  \\
            0.350081623831937  4.521549682928944e8  \\
            0.3510381310008768  5.194796928264311e8  \\
            0.35199463816981647  6.117009801018426e8  \\
            0.3529511453387562  7.446768011191039e8  \\
            0.3539076525076959  9.522841998868822e8  \\
            0.35486415967663565  1.320461865104572e9  \\
            0.35582066684557534  2.1476308391377506e9  \\
            0.3567771740145151  5.5160671130117035e9  \\
            0.3577336811834548  1.3898858246210106e10  \\
            0.3586901883523945  2.123992157921117e10  \\
            0.3596466955213342  3.10702971164347e9  \\
            0.36060320269027396  1.6363779013641076e9  \\
            0.36155970985921365  1.1066240819771802e9  \\
            0.3625162170281534  8.357410955841044e8  \\
            0.3634727241970931  6.716009036298727e8  \\
            0.36442923136603284  5.617431389073638e8  \\
            0.3653857385349725  4.8332223551624143e8  \\
            0.3663422457039123  4.25312496854842e8  \\
            0.36729875287285196  3.887958542583211e9  \\
            0.3682552600417917  3.4168134770167744e8  \\
            0.3692117672107314  3.102367850581196e8  \\
            0.37016827437967115  2.8355193168983305e8  \\
            0.37112478154861084  3.376907551058403e8  \\
            0.3720812887175506  4.9188722049709105e8  \\
            0.3730377958864903  7.08929626693751e8  \\
            0.37399430305543  1.0177047421845659e9  \\
            0.3749508102243697  1.4619906262273726e9  \\
            0.37590731739330946  2.109217513427814e9  \\
            0.37686382456224915  3.065088169997933e9  \\
            0.3778203317311889  4.498524920056059e9  \\
            0.37877683890012864  6.685260179669541e9  \\
            0.37973334606906833  9.970995267355378e18  \\
            0.3806898532380081  4.113122970435722e16  \\
            0.38164636040694777  2.126489034617244e15  \\
            0.3826028675758875  2.72604157510071e14  \\
            0.3835593747448272  5.623499540375681e13  \\
            0.38451588191376695  1.5553796955433014e13  \\
            0.38547238908270665  5.247966196937274e12  \\
            0.3864288962516464  2.0438540566218887e12  \\
            0.3873854034205861  8.870065359801608e11  \\
            0.38834191058952583  1.5428090868480955e12  \\
            0.3892984177584655  3.5834668549357544e12  \\
            0.39025492492740527  9.351881736049934e12  \\
            0.39121143209634496  2.8560561596154375e13  \\
            0.3921679392652847  1.0910073094838531e14  \\
            0.3931244464342244  5.880881667366294e14  \\
            0.39408095360316414  5.776844889465489e15  \\
            0.39503746077210383  2.1331947510641267e17  \\
            0.3959939679410436  3.588971256460002e21  \\
            0.39695047510998327  5.551828644731052e13  \\
            0.39790698227892296  1.7319665009492894e14  \\
            0.39886348944786265  8.934365144975569e14  \\
            0.3998199966168024  1.1428840697020432e16  \\
            0.4007765037857421  1.8863076153658985e18  \\
            0.40173301095468184  1.877801240202731e9  \\
            0.4026895181236216  2.5532521633889074e9  \\
            0.4036460252925613  3.5225793622105703e9  \\
            0.404602532461501  4.9434925546541395e9  \\
            0.4055590396304407  7.080448978635124e9  \\
            0.40651554679938046  1.039805904719232e10  \\
            0.40747205396832015  1.5762747519277908e10  \\
            0.4084285611372599  2.49297965725033e10  \\
            0.4093850683061996  4.190993462026292e10  \\
            0.41034157547513933  7.784904240260962e10  \\
            0.411298082644079  1.7832277071639294e11  \\
            0.41225458981301877  1.1024039391407253e12  \\
            0.41321109698195846  2.0669078925443057e11  \\
            0.4141676041508982  9.58510874092565e11  \\
            0.4151241113198379  4.873515057237074e11  \\
            0.41608061848877764  4.635708832868283e11  \\
            0.41703712565771733  5.197635054672267e11  \\
            0.4179936328266571  6.413319935305663e11  \\
            0.41895013999559677  2.507149638829668e21  \\
            0.4199066471645365  3.0175550257703494e19  \\
            0.4208631543334762  2.2887182411473684e18  \\
            0.42181966150241595  3.657711210405116e17  \\
            0.42277616867135565  8.792413321482389e16  \\
            0.4237326758402954  2.7375420190616524e16  \\
            0.4246891830092351  1.0193939692319934e16  \\
            0.42564569017817483  4.3286654056610485e15  \\
            0.4266021973471145  2.0324596329421492e15  \\
            0.42755870451605427  1.0332318644922778e15  \\
            0.428515211684994  8.347171016302455e15  \\
            0.4294717188539337  3.591132837784516e17  \\
            0.43042822602287345  1.0e23  \\
            0.43138473319181314  1.189836511884752e14  \\
            0.4323412403607529  7.70275485538257e19  \\
            0.4332977475296926  7.678951579238325e16  \\
            0.4342542546986323  3.1810287573180515e15  \\
            0.435210761867572  3.863276742989779e14  \\
            0.43616726903651176  7.905020588950702e13  \\
            0.43712377620545145  2.203870776809219e13  \\
            0.4380802833743912  8.813322377371637e12  \\
            0.4390367905433309  6.540412638359799e12  \\
            0.43999329771227064  4.918922339234216e12  \\
            0.4409498048812103  7.709867929344142e12  \\
            0.4419063120501501  4.413350658082104e13  \\
            0.44286281921908976  5.726433814628032e14  \\
            0.4438193263880295  7.331102483261962e16  \\
            0.4447758335569692  1.395770660427463e12  \\
            0.44573234072590895  1.11512152650111e12  \\
            0.44668884789484864  8.977700332702312e11  \\
            0.4476453550637884  7.279864848845486e11  \\
            0.4486018622327281  5.942896070728412e11  \\
            0.4495583694016678  4.8821275052602625e11  \\
            0.4505148765706075  4.0345196636198315e11  \\
            0.4514713837395472  3.352710694579099e11  \\
            0.45242789090848695  2.800814233360137e11  \\
            0.45338439807742664  2.3514235793164914e11  \\
            0.4543409052463664  1.9834396776625415e11  \\
            0.4552974124153061  1.6805173355428244e11  \\
            0.4562539195842458  3.116974854605454e11  \\
            0.4572104267531855  2.0350046020344106e12  \\
            0.45816693392212526  3.588855382002776e13  \\
            0.45912344109106495  2.996174018629903e16  \\
            0.4600799482600047  7.785483773810352e10  \\
            0.4610364554289444  3.5553759915410546e21  \\
            0.46199296259788414  2.4025309010050826e17  \\
            0.46294946976682383  6.291202039713533e15  \\
            0.4639059769357636  6.203310418008186e14  \\
            0.46486248410470327  1.1210196471018256e14  \\
            0.465818991273643  2.8720547905707555e13  \\
            0.4667754984425827  9.23625851210237e12  \\
            0.46773200561152245  3.4851347714121226e12  \\
            0.46868851278046214  1.4804593571298154e12  \\
            0.4696450199494019  6.89042816285205e11  \\
            0.4706015271183416  3.4466393013397577e11  \\
            0.4715580342872813  1.8282061664928934e11  \\
            0.472514541456221  1.0177583977105667e11  \\
            0.47347104862516076  5.899365986999215e10  \\
            0.47442755579410045  3.5381744151352715e10  \\
            0.4753840629630402  2.1845573828555214e10  \\
            0.4763405701319799  1.3826755786590078e10  \\
            0.47729707730091964  8.939008519524015e9  \\
        }
        ;
    \addlegendentry {$(20.0, 0.0, 0.0) $}
    \addplot[color={rgb,1:red,0.0;green,0.0;blue,0.0}, name path={bb9883bc-9c1c-4831-924c-16db6468bf41}, only marks, draw opacity={0.0}, line width={0}, solid, mark={*}, mark size={0.00075 pt}, mark repeat={1}, mark options={color={rgb,1:red,0.0;green,0.0;blue,0.0}, draw opacity={1.0}, fill={rgb,1:red,0.0;green,0.0;blue,0.0}, fill opacity={0.0}, line width={0.75}, rotate={0}, solid}, forget plot]
        table[row sep={\\}]
        {
            \\
            0.0  1.0  \\
        }
        ;
\end{axis}
\end{tikzpicture}

}

\subfloat[]{%
% Recommended preamble:
% \usetikzlibrary{arrows.meta}
% \usetikzlibrary{backgrounds}
% \usepgfplotslibrary{patchplots}
% \usepgfplotslibrary{fillbetween}
% \pgfplotsset{%
%     layers/standard/.define layer set={%
%         background,axis background,axis grid,axis ticks,axis lines,axis tick labels,pre main,main,axis descriptions,axis foreground%
%     }{
%         grid style={/pgfplots/on layer=axis grid},%
%         tick style={/pgfplots/on layer=axis ticks},%
%         axis line style={/pgfplots/on layer=axis lines},%
%         label style={/pgfplots/on layer=axis descriptions},%
%         legend style={/pgfplots/on layer=axis descriptions},%
%         title style={/pgfplots/on layer=axis descriptions},%
%         colorbar style={/pgfplots/on layer=axis descriptions},%
%         ticklabel style={/pgfplots/on layer=axis tick labels},%
%         axis background@ style={/pgfplots/on layer=axis background},%
%         3d box foreground style={/pgfplots/on layer=axis foreground},%
%     },
% }

\begin{tikzpicture}[/tikz/background rectangle/.style={fill={rgb,1:red,1.0;green,1.0;blue,1.0}, fill opacity={1.0}, draw opacity={1.0}}, show background rectangle]
\begin{axis}[point meta max={nan}, point meta min={nan}, legend cell align={left}, legend columns={1}, title={}, title style={at={{(0.5,1)}}, anchor={south}, font={{\fontsize{14 pt}{18.2 pt}\selectfont}}, color={rgb,1:red,0.0;green,0.0;blue,0.0}, draw opacity={1.0}, rotate={0.0}, align={center}}, legend style={color={rgb,1:red,0.0;green,0.0;blue,0.0}, draw opacity={1.0}, line width={1}, solid, fill={rgb,1:red,1.0;green,1.0;blue,1.0}, fill opacity={1.0}, text opacity={1.0}, font={{\fontsize{8 pt}{10.4 pt}\selectfont}}, text={rgb,1:red,0.0;green,0.0;blue,0.0}, cells={anchor={center}}, at={(1.02, 1)}, anchor={north west}}, axis background/.style={fill={rgb,1:red,1.0;green,1.0;blue,1.0}, opacity={1.0}}, anchor={north west}, xshift={1.0mm}, yshift={-1.0mm}, width={120.0mm}, height={74.2mm}, scaled x ticks={false}, xlabel={}, x tick style={color={rgb,1:red,0.0;green,0.0;blue,0.0}, opacity={1.0}}, x tick label style={color={rgb,1:red,0.0;green,0.0;blue,0.0}, opacity={1.0}, rotate={0}}, xlabel style={at={(ticklabel cs:0.5)}, anchor=near ticklabel, at={{(ticklabel cs:0.5)}}, anchor={near ticklabel}, font={{\fontsize{11 pt}{14.3 pt}\selectfont}}, color={rgb,1:red,0.0;green,0.0;blue,0.0}, draw opacity={1.0}, rotate={0.0}}, xmajorgrids={true}, xmin={-0.01649326567117626}, xmax={0.5662687880437169}, xticklabels={{$0.0$,$0.1$,$0.2$,$0.3$,$0.4$,$0.5$}}, xtick={{0.0,0.1,0.2,0.30000000000000004,0.4,0.5}}, xtick align={inside}, xticklabel style={font={{\fontsize{8 pt}{10.4 pt}\selectfont}}, color={rgb,1:red,0.0;green,0.0;blue,0.0}, draw opacity={1.0}, rotate={0.0}}, x grid style={color={rgb,1:red,0.0;green,0.0;blue,0.0}, draw opacity={0.1}, line width={0.5}, solid}, axis x line*={left}, x axis line style={color={rgb,1:red,0.0;green,0.0;blue,0.0}, draw opacity={1.0}, line width={1}, solid}, scaled y ticks={false}, ylabel={}, y tick style={color={rgb,1:red,0.0;green,0.0;blue,0.0}, opacity={1.0}}, y tick label style={color={rgb,1:red,0.0;green,0.0;blue,0.0}, opacity={1.0}, rotate={0}}, ylabel style={at={(ticklabel cs:0.5)}, anchor=near ticklabel, at={{(ticklabel cs:0.5)}}, anchor={near ticklabel}, font={{\fontsize{11 pt}{14.3 pt}\selectfont}}, color={rgb,1:red,0.0;green,0.0;blue,0.0}, draw opacity={1.0}, rotate={0.0}}, ymode={log}, log basis y={10}, ymajorgrids={true}, ymin={0.01568959833291757}, ymax={310.0432599127799}, yticklabels={{$10^{0}$,$10^{2}$}}, ytick={{1.0,100.0}}, ytick align={inside}, yticklabel style={font={{\fontsize{8 pt}{10.4 pt}\selectfont}}, color={rgb,1:red,0.0;green,0.0;blue,0.0}, draw opacity={1.0}, rotate={0.0}}, y grid style={color={rgb,1:red,0.0;green,0.0;blue,0.0}, draw opacity={0.1}, line width={0.5}, solid}, axis y line*={left}, y axis line style={color={rgb,1:red,0.0;green,0.0;blue,0.0}, draw opacity={1.0}, line width={1}, solid}, colorbar={false}]
    [\addlegendimage{empty legend}] \addlegendentry[font={{\fontsize{11 pt}{14.3 pt}\selectfont}}, text={rgb,1:red,0.0;green,0.0;blue,0.0}] {\hspace{-.6cm}{\textbf{$(\gamma, \gamma_1, \gamma_2)$}}}
    \addplot[color={rgb,1:red,0.0;green,0.0;blue,1.0}, name path={2c1fd526-2e49-4f87-902f-6efdd7440f87}, draw opacity={1.0}, line width={1}, solid]
        table[row sep={\\}]
        {
            \\
            0.0  0.020917285515940576  \\
            0.001101754553852787  0.020923277405693396  \\
            0.002203509107705574  0.02095259884817224  \\
            0.0033052636615583607  0.021210608859436497  \\
            0.004407018215411148  0.02092872126936171  \\
            0.005508772769263934  0.021143004310123593  \\
            0.0066105273231167215  0.021657483547878085  \\
            0.007712281876969509  0.021047325950646737  \\
            0.008814036430822295  0.020939386664037805  \\
            0.009915790984675084  0.02094833507852258  \\
            0.011017545538527868  0.02098046716597757  \\
            0.012119300092380656  0.020966902138977804  \\
            0.013221054646233443  0.020969023525817948  \\
            0.01432280920008623  0.020990140280396672  \\
            0.015424563753939018  0.02106022175816107  \\
            0.016526318307791804  0.02335626532772011  \\
            0.01762807286164459  0.02085337015470323  \\
            0.018729827415497377  0.020895203482808405  \\
            0.019831581969350167  0.020904072231252874  \\
            0.020933336523202953  0.020906705973777424  \\
            0.022035091077055736  0.020908294291033927  \\
            0.023136845630908526  0.0209103108091734  \\
            0.024238600184761313  0.020913116463502447  \\
            0.0253403547386141  0.020916549185172044  \\
            0.026442109292466886  0.020920018430185403  \\
            0.027543863846319676  0.020922342418639345  \\
            0.02864561840017246  0.020950233234048782  \\
            0.029747372954025245  0.02095768456300192  \\
            0.030849127507878035  0.020966157600381854  \\
            0.03195088206173082  0.020975955762615434  \\
            0.03305263661558361  0.020987617606710505  \\
            0.034154391169436395  0.021001888193277793  \\
            0.03525614572328918  0.02101943406506624  \\
            0.03635790027714197  0.02104005558664877  \\
            0.037459654830994754  0.02106120080396599  \\
            0.03856140938484754  0.02107619545345478  \\
            0.039663163938700334  0.021070832791045577  \\
            0.04076491849255312  0.021085302505939124  \\
            0.04186667304640591  0.02110054728437575  \\
            0.042968427600258687  0.021116657400902558  \\
            0.04407018215411147  0.021133728373512916  \\
            0.045171936707964266  0.021151858421404005  \\
            0.04627369126181705  0.021171143791028014  \\
            0.04737544581566984  0.021191671043068953  \\
            0.048477200369522626  0.021213505192713512  \\
            0.04957895492337542  0.02123667254849131  \\
            0.0506807094772282  0.021261137323978344  \\
            0.051782464031080985  0.021247445603706366  \\
            0.05288421858493377  0.02118907609024813  \\
            0.05398597313878656  0.021283487207684988  \\
            0.05508772769263935  0.07868225329624191  \\
            0.05618948224649214  0.021483023951101833  \\
            0.05729123680034492  0.021240827344540438  \\
            0.058392991354197704  0.02116838859109928  \\
            0.05949474590805049  0.02113131557077329  \\
            0.060596500461903284  0.021108329273390517  \\
            0.06169825501575607  0.02109302777401753  \\
            0.06280000956960886  0.021082803813534758  \\
            0.06390176412346164  0.021076421914151513  \\
            0.06500351867731442  0.021073287069480157  \\
            0.06610527323116722  0.02107318800894597  \\
            0.06720702778502  0.02107622006265468  \\
            0.06830878233887279  0.021082811379266257  \\
            0.06941053689272557  0.021093858232740844  \\
            0.07051229144657836  0.021111050853871754  \\
            0.07161404600043114  0.021137635801904656  \\
            0.07271580055428394  0.021180378659072484  \\
            0.07381755510813673  0.021255559825779383  \\
            0.07491930966198951  0.021413079233555725  \\
            0.0760210642158423  0.021896587657047988  \\
            0.07712281876969508  0.02791116389996988  \\
            0.07822457332354787  0.02142118321888716  \\
            0.07932632787740067  0.020758303098044748  \\
            0.08042808243125345  0.0207752413421832  \\
            0.08152983698510624  0.020805972295152263  \\
            0.08263159153895902  0.020828699044193513  \\
            0.08373334609281181  0.02084342713247511  \\
            0.0848351006466646  0.020850396280801094  \\
            0.08593685520051737  0.020847466101603263  \\
            0.08703860975437017  0.020826458705156283  \\
            0.08814036430822295  0.020761536864284946  \\
            0.08924211886207574  0.033525536447638296  \\
            0.09034387341592853  0.021306972685231986  \\
            0.09144562796978131  0.02114835729291415  \\
            0.0925473825236341  0.021108513893177923  \\
            0.09364913707748689  0.021096238862374352  \\
            0.09475089163133968  0.021094849106071203  \\
            0.09585264618519247  0.021099142559731218  \\
            0.09695440073904525  0.021107087513401953  \\
            0.09805615529289805  0.021117871471949972  \\
            0.09915790984675084  0.021131339393211417  \\
            0.1002596644006036  0.02113892981165414  \\
            0.1013614189544564  0.021158806005604793  \\
            0.10246317350830918  0.02118680697282225  \\
            0.10356492806216197  0.0212372517050231  \\
            0.10466668261601475  0.021404850153481574  \\
            0.10576843716986754  0.07521564401671728  \\
            0.10687019172372034  0.02114373593988083  \\
            0.10797194627757312  0.021153055106389405  \\
            0.10907370083142591  0.0211782475947956  \\
            0.1101754553852787  0.021202103511898714  \\
            0.11127720993913148  0.021224968600177946  \\
            0.11237896449298428  0.021247498292942646  \\
            0.11348071904683706  0.02127009608264589  \\
            0.11458247360068984  0.021292990918206706  \\
            0.11568422815454263  0.021316305983418923  \\
            0.11678598270839541  0.02134009691674261  \\
            0.1178877372622482  0.021364371609204798  \\
            0.11898949181610098  0.021389099735920722  \\
            0.12009124636995377  0.021414216227719735  \\
            0.12119300092380657  0.02143962074619351  \\
            0.12229475547765935  0.021465174129362944  \\
            0.12339651003151214  0.02149069242242062  \\
            0.12449826458536492  0.02151593950820229  \\
            0.12560001913921773  0.021540621851650843  \\
            0.1267017736930705  0.021564398750274404  \\
            0.1278035282469233  0.021586966013327286  \\
            0.12890528280077607  0.021608540919649043  \\
            0.13000703735462885  0.02163423631239983  \\
            0.13110879190848163  0.021679665132729644  \\
            0.13221054646233443  0.021605921213478884  \\
            0.1333123010161872  0.021600328485052042  \\
            0.13441405557004  0.021571325749666716  \\
            0.1355158101238928  0.021646545342081793  \\
            0.13661756467774558  0.02156918814351828  \\
            0.13771931923159836  0.021696875173416556  \\
            0.13882107378545114  0.021616674204910907  \\
            0.13992282833930392  0.021545143270674635  \\
            0.14102458289315672  0.021588736191582775  \\
            0.1421263374470095  0.02160579775905835  \\
            0.14322809200086228  0.02159626610396467  \\
            0.1443298465547151  0.02166925590120548  \\
            0.14543160110856787  0.021619662641954543  \\
            0.14653335566242065  0.021597658172249748  \\
            0.14763511021627346  0.021575722671790067  \\
            0.14873686477012624  0.021552540947288052  \\
            0.14983861932397902  0.02152828159601507  \\
            0.1509403738778318  0.02150328859082469  \\
            0.1520421284316846  0.0214778830519626  \\
            0.15314388298553738  0.02145232997805193  \\
            0.15424563753939016  0.02142683877560054  \\
            0.15534739209324297  0.021401570361699148  \\
            0.15644914664709575  0.02137664405758331  \\
            0.15755090120094853  0.02135214244820222  \\
            0.15865265575480134  0.02132811352665496  \\
            0.15975441030865412  0.02130456941831492  \\
            0.1608561648625069  0.021281480271912425  \\
            0.1619579194163597  0.021258760383297916  \\
            0.16305967397021248  0.021236240669539195  \\
            0.16416142852406526  0.021213616936095418  \\
            0.16526318307791804  0.021190366924983255  \\
            0.16636493763177085  0.02116578338470482  \\
            0.16746669218562363  0.021142054824173806  \\
            0.16856844673947638  0.0212818724526968  \\
            0.1696702012933292  0.02191009488169024  \\
            0.17077195584718197  0.021288387375493465  \\
            0.17187371040103475  0.021207184341043316  \\
            0.17297546495488753  0.021171195395535112  \\
            0.17407721950874033  0.021148089287303287  \\
            0.1751789740625931  0.021130384892525347  \\
            0.1762807286164459  0.021124093554885516  \\
            0.1773824831702987  0.021111992582042167  \\
            0.17848423772415148  0.02110258277813746  \\
            0.17958599227800426  0.021096316441157916  \\
            0.18068774683185707  0.021094496998841108  \\
            0.18178950138570985  0.021100322563128708  \\
            0.18289125593956262  0.021122736864359483  \\
            0.18399301049341543  0.021196962581315978  \\
            0.1850947650472682  0.021683982876173912  \\
            0.186196519601121  0.02079253967273185  \\
            0.18729827415497377  0.020801943262244705  \\
            0.18840002870882658  0.02084009449312762  \\
            0.18950178326267936  0.02085035964909163  \\
            0.19060353781653214  0.020847847828991538  \\
            0.19170529237038494  0.020836935550112925  \\
            0.19280704692423772  0.020818332045231997  \\
            0.1939088014780905  0.02079135851066556  \\
            0.1950105560319433  0.020760218344025257  \\
            0.1961123105857961  0.02082802233093389  \\
            0.19721406513964887  0.03791535240185756  \\
            0.19831581969350168  0.022766260071455366  \\
            0.19941757424735443  0.021574237721866563  \\
            0.2005193288012072  0.021317515847841445  \\
            0.20162108335506  0.021211951568540333  \\
            0.2027228379089128  0.021156149416933105  \\
            0.20382459246276557  0.02112271655069878  \\
            0.20492634701661835  0.021101397944548975  \\
            0.20602810157047116  0.021087572795423202  \\
            0.20712985612432394  0.021078915188233203  \\
            0.20823161067817672  0.02107419156947027  \\
            0.2093333652320295  0.02107276590202478  \\
            0.2104351197858823  0.021074386018616786  \\
            0.2115368743397351  0.02107910762160728  \\
            0.21263862889358787  0.02108731647573912  \\
            0.21374038344744067  0.021099878049625057  \\
            0.21484213800129345  0.021118567652978686  \\
            0.21594389255514623  0.02114736393224503  \\
            0.21704564710899904  0.021197361870918943  \\
            0.21814740166285182  0.021315942585360386  \\
            0.2192491562167046  0.022137734464122805  \\
            0.2203509107705574  0.02165397457118227  \\
            0.2214526653244102  0.02118014417133343  \\
            0.22255441987826297  0.021214762384232984  \\
            0.22365617443211575  0.02128563866282979  \\
            0.22475792898596855  0.02124850543433208  \\
            0.22585968353982133  0.021224664142040734  \\
            0.2269614380936741  0.021202152149917365  \\
            0.22806319264752692  0.021180970192922356  \\
            0.22916494720137967  0.02116106948885145  \\
            0.23026670175523245  0.021142371935204074  \\
            0.23136845630908526  0.021124784477336508  \\
            0.23247021086293804  0.02110820874341889  \\
            0.23357196541679082  0.021092547103575023  \\
            0.2346737199706436  0.02107770616906461  \\
            0.2357754745244964  0.0210805793357639  \\
            0.23687722907834918  0.02107002142469121  \\
            0.23797898363220196  0.021050831215955187  \\
            0.23908073818605477  0.021029389380270564  \\
            0.24018249273990755  0.021010165275495988  \\
            0.24128424729376033  0.02099430516999374  \\
            0.24238600184761314  0.020981434999864112  \\
            0.24348775640146592  0.02097078732747522  \\
            0.2445895109553187  0.020961705246682914  \\
            0.24569126550917147  0.02095377226804009  \\
            0.24679302006302428  0.020922216184489976  \\
            0.24789477461687706  0.02092122416863899  \\
            0.24899652917072984  0.020918134613437923  \\
            0.25009828372458265  0.020914588363133816  \\
            0.25120003827843546  0.02091143372867306  \\
            0.2523017928322882  0.020909042150213903  \\
            0.253403547386141  0.020907345966840424  \\
            0.25450530193999377  0.020905561320934567  \\
            0.2556070564938466  0.020900980735142138  \\
            0.2567088110476994  0.02088274040454783  \\
            0.25781056560155213  0.020768001561791555  \\
            0.25891232015540494  0.027924588342842623  \\
            0.2600140747092577  0.02101411235992611  \\
            0.2611158292631105  0.020976165686923245  \\
            0.26221758381696325  0.020966098042338368  \\
            0.26331933837081606  0.023685083087616236  \\
            0.26442109292466887  0.020964750074802068  \\
            0.2655228474785216  0.02093912742760571  \\
            0.2666246020323744  0.020957958013585305  \\
            0.26772635658622723  0.034818497849062276  \\
            0.26882811114008  0.03167673858175101  \\
            0.2699298656939328  0.020937017209808017  \\
            0.2710316202477856  0.320224786103316  \\
            0.27213337480163835  0.021001243980118902  \\
            0.27323512935549116  0.02093285888698676  \\
            0.27433688390934396  0.020918666817066136  \\
            0.2754386384631967  0.020918671287433337  \\
            0.27654039301704947  0.020932870991145137  \\
            0.2776421475709023  0.02100124791560271  \\
            0.2787439021247551  0.320102530309526  \\
            0.27984565667860783  0.02093701337779468  \\
            0.28094741123246064  0.0316739382415783  \\
            0.28204916578631345  0.03481864044181153  \\
            0.2831509203401662  0.02095799474339919  \\
            0.284252674894019  0.02093919450474277  \\
            0.2853544294478718  0.02096484446124673  \\
            0.28645618400172457  0.02368681307505564  \\
            0.2875579385555774  0.02096627041532929  \\
            0.2886596931094302  0.020976347808029136  \\
            0.28976144766328293  0.021014322034997267  \\
            0.29086320221713574  0.027957055996264153  \\
            0.29196495677098855  0.020767890804895998  \\
            0.2930667113248413  0.020882801955969664  \\
            0.2941684658786941  0.02090108796306556  \\
            0.2952702204325469  0.020905695419550077  \\
            0.29637197498639967  0.020907499453453913  \\
            0.2974737295402525  0.020909211000626964  \\
            0.2985754840941052  0.02091161533046061  \\
            0.29967723864795803  0.020914780633862626  \\
            0.30077899320181084  0.0209183354832666  \\
            0.3018807477556636  0.02092143116281089  \\
            0.3029825023095164  0.02092242563304082  \\
            0.3040842568633692  0.0209538424836551  \\
            0.30518601141722196  0.020961779361700638  \\
            0.30628776597107477  0.020970863296355603  \\
            0.3073895205249276  0.020981510453851607  \\
            0.3084912750787803  0.02099437718564571  \\
            0.30959302963263313  0.021010230091792745  \\
            0.31069478418648594  0.0210294424112827  \\
            0.3117965387403387  0.021050868431929864  \\
            0.3128982932941915  0.021070044731345206  \\
            0.3140000478480443  0.021080621625835596  \\
            0.31510180240189706  0.021077976165091387  \\
            0.31620355695574986  0.02109282251182264  \\
            0.31730531150960267  0.021108488307066343  \\
            0.3184070660634554  0.02112506666253365  \\
            0.31950882061730823  0.02114265487845929  \\
            0.32061057517116104  0.021161350937848086  \\
            0.3217123297250138  0.02118124744077201  \\
            0.3228140842788666  0.021202421961843698  \\
            0.3239158388327194  0.021224922682000845  \\
            0.32501759338657216  0.021248748194430796  \\
            0.32611934794042496  0.02128555271599915  \\
            0.3272211024942777  0.02121473496462329  \\
            0.3283228570481305  0.02118014416693206  \\
            0.32942461160198333  0.02165399727551033  \\
            0.3305263661558361  0.02213746736581705  \\
            0.3316281207096889  0.021315892991643413  \\
            0.3327298752635417  0.021197356071911207  \\
            0.33383162981739445  0.021147382000155677  \\
            0.33493338437124726  0.021118603039125657  \\
            0.3360351389251  0.02109992758677041  \\
            0.33713689347895276  0.021087378309774127  \\
            0.33823864803280557  0.021079180570163038  \\
            0.3393404025866584  0.02107446932708979  \\
            0.3404421571405111  0.02107285914877982  \\
            0.34154391169436393  0.021074294641512285  \\
            0.34264566624821674  0.021079028316881663  \\
            0.3437474208020695  0.02108769665437511  \\
            0.3448491753559223  0.021101533852660746  \\
            0.34595092990977505  0.021122866886827872  \\
            0.34705268446362786  0.021156318524408625  \\
            0.34815443901748067  0.0212121479892831  \\
            0.3492561935713334  0.02131775908611858  \\
            0.3503579481251862  0.02157458701412206  \\
            0.35145970267903903  0.02276706819028004  \\
            0.3525614572328918  0.037920604400035286  \\
            0.3536632117867446  0.020828066585146097  \\
            0.3547649663405974  0.020760246515702782  \\
            0.35586672089445015  0.020791405597070864  \\
            0.35696847544830296  0.020818395598388217  \\
            0.35807023000215576  0.020837012615742282  \\
            0.3591719845560085  0.020847936711837677  \\
            0.3602737391098613  0.02085045994992876  \\
            0.36137549366371413  0.020840207673260457  \\
            0.3624772482175669  0.020802076052272717  \\
            0.3635790027714197  0.020792744352576283  \\
            0.3646807573252725  0.021684000645892775  \\
            0.36578251187912525  0.02119704963137665  \\
            0.36688426643297806  0.02112284285561472  \\
            0.36798602098683086  0.021100440366224906  \\
            0.3690877755406836  0.021094624462926155  \\
            0.3701895300945364  0.021096452865173927  \\
            0.3712912846483892  0.021102728111727823  \\
            0.372393039202242  0.021112147192200965  \\
            0.3734947937560948  0.021124258183308167  \\
            0.37459654830994754  0.021130529358214755  \\
            0.37569830286380035  0.021148218042445086  \\
            0.37680005741765316  0.02117131586029625  \\
            0.3779018119715059  0.021207304141896952  \\
            0.3790035665253587  0.021288512980143406  \\
            0.3801053210792115  0.02191002985063643  \\
            0.3812070756330643  0.021281214791584846  \\
            0.3823088301869171  0.02114193998071699  \\
            0.3834105847407699  0.0211657397390872  \\
            0.38451233929462264  0.021190349169393653  \\
            0.38561409384847545  0.02121361261450698  \\
            0.38671584840232825  0.021236245000415446  \\
            0.387817602956181  0.021258771261063205  \\
            0.3889193575100338  0.021281496756650156  \\
            0.3900211120638866  0.021304591157961544  \\
            0.3911228666177394  0.02132814050265835  \\
            0.3922246211715922  0.02135217485649121  \\
            0.393326375725445  0.02137668225629826  \\
            0.39442813027929774  0.021401614852196313  \\
            0.39552988483315055  0.021426890211671462  \\
            0.39663163938700335  0.021452389193806102  \\
            0.39773339394085605  0.021477951109653632  \\
            0.39883514849470886  0.021503366851298503  \\
            0.3999369030485616  0.02152837181602946  \\
            0.4010386576024144  0.02155264542814812  \\
            0.4021404121562672  0.021575844511289743  \\
            0.40324216671012  0.02159780184265595  \\
            0.4043439212639728  0.021619836251709965  \\
            0.4054456758178256  0.02166949780198956  \\
            0.40654743037167834  0.02159643582936438  \\
            0.40764918492553115  0.021606036285290484  \\
            0.40875093947938396  0.021589023871834975  \\
            0.4098526940332367  0.021545506929645245  \\
            0.4109544485870895  0.02161706033290978  \\
            0.4120562031409423  0.021696830228124724  \\
            0.4131579576947951  0.021568814690025716  \\
            0.4142597122486479  0.02164619442240607  \\
            0.4153614668025007  0.021571010207829658  \\
            0.41646322135635344  0.021600066403315778  \\
            0.41756497591020625  0.02160571106810497  \\
            0.418666730464059  0.02167985698112634  \\
            0.4197684850179118  0.0216340401630562  \\
            0.4208702395717646  0.02160838375801579  \\
            0.42197199412561737  0.02158683395092585  \\
            0.4230737486794702  0.021564286042581763  \\
            0.424175503233323  0.021540524830928977  \\
            0.42527725778717573  0.021515855517662847  \\
            0.42637901234102854  0.021490619455061314  \\
            0.42748076689488135  0.021465110642324526  \\
            0.4285825214487341  0.02143956553721427  \\
            0.4296842760025869  0.02141416835688892  \\
            0.4307860305564397  0.02138905846297824  \\
            0.43188778511029247  0.02136433635965986  \\
            0.4329895396641453  0.021340067260121876  \\
            0.4340912942179981  0.021316281638405613  \\
            0.43519304877185083  0.021292971786302895  \\
            0.43629480332570364  0.021270082328705347  \\
            0.43739655787955645  0.02124749051807431  \\
            0.4384983124334092  0.021224968219908658  \\
            0.439600066987262  0.021202113669441358  \\
            0.4407018215411148  0.021178275811120403  \\
            0.44180357609496756  0.021153123702695845  \\
            0.4429053306488204  0.021143959318076066  \\
            0.4440070852026732  0.07525378783478039  \\
            0.44510883975652593  0.02140473290051358  \\
            0.44621059431037874  0.021237129275322373  \\
            0.4473123488642315  0.021186687840785538  \\
            0.4484141034180843  0.021158682314260203  \\
            0.4495158579719371  0.0211387941343406  \\
            0.45061761252578986  0.021131169342242278  \\
            0.45171936707964266  0.021117711970455547  \\
            0.45282112163349547  0.021106937610330667  \\
            0.4539228761873482  0.021099001703411894  \\
            0.45502463074120103  0.02109471712440521  \\
            0.45612638529505384  0.02109611608145847  \\
            0.4572281398489066  0.021108401567670083  \\
            0.45832989440275934  0.021148259190456098  \\
            0.45943164895661215  0.021306904622250676  \\
            0.4605334035104649  0.03352670188798115  \\
            0.4616351580643177  0.02076138472227477  \\
            0.4627369126181705  0.020826337270860484  \\
            0.46383866717202327  0.02084735972230207  \\
            0.4649404217258761  0.020850301728847602  \\
            0.4660421762797288  0.02084334402478836  \\
            0.46714393083358163  0.02082862844293871  \\
            0.46824568538743444  0.020805916537897037  \\
            0.4693474399412872  0.020775203737184597  \\
            0.47044919449514  0.020758280063276742  \\
            0.4715509490489928  0.02142092009478608  \\
            0.47265270360284556  0.027908617888134844  \\
            0.47375445815669837  0.02189611077584631  \\
            0.4748562127105512  0.02141279587808422  \\
            0.4759579672644039  0.021255343692463118  \\
            0.47705972181825673  0.021180197390184476  \\
            0.47816147637210954  0.021137476806474893  \\
            0.4792632309259623  0.021110908121617904  \\
            0.4803649854798151  0.021093728567120742  \\
            0.4814667400336679  0.0210826930033661  \\
            0.48256849458752066  0.02107611201538616  \\
            0.48367024914137347  0.021073089855663852  \\
            0.4847720036952263  0.021073198758177057  \\
            0.485873758249079  0.021076343713018808  \\
            0.48697551280293183  0.021082736304738316  \\
            0.48807726735678464  0.02109297190685268  \\
            0.4891790219106374  0.021108286520808935  \\
            0.4902807764644902  0.02113128831708373  \\
            0.49138253101834295  0.021168381244516928  \\
            0.49248428557219576  0.021240850573392575  \\
            0.49358604012604856  0.021483124338736555  \\
            0.4946877946799013  0.07868957587532484  \\
            0.4957895492337541  0.02128348310742274  \\
            0.49689130378760693  0.02118908935739027  \\
            0.4979930583414597  0.021247487217152854  \\
            0.4990948128953125  0.02126090436922863  \\
            0.5001965674491653  0.02123642129068222  \\
            0.501298322003018  0.02121324049631405  \\
            0.5024000765568709  0.02119139707271736  \\
            0.5035018311107237  0.021170864072704482  \\
            0.5046035856645764  0.021151575917021662  \\
            0.5057053402184292  0.02113344555398223  \\
            0.506807094772282  0.021116376315570656  \\
            0.5079088493261348  0.021100269625243796  \\
            0.5090106038799875  0.0210850296624403  \\
            0.5101123584338404  0.021070565898111175  \\
            0.5112141129876931  0.021076171844257023  \\
            0.5123158675415459  0.021061171592926562  \\
            0.5134176220953988  0.02104001012644688  \\
            0.5145193766492515  0.021019374542255907  \\
            0.5156211312031043  0.021001819247004734  \\
            0.5167228857569571  0.020987543460420987  \\
            0.5178246403108099  0.020975879726997788  \\
            0.5189263948646626  0.02096608228669032  \\
            0.5200281494185154  0.02095761215571924  \\
            0.5211299039723682  0.020950165675180765  \\
            0.522231658526221  0.02092213354114296  \\
            0.5233334130800738  0.02091981414574647  \\
            0.5244351676339265  0.02091635234077252  \\
            0.5255369221877794  0.020912929277734738  \\
            0.5266386767416321  0.020910135304333256  \\
            0.5277404312954849  0.02090813273554966  \\
            0.5288421858493377  0.02090656153524039  \\
            0.5299439404031905  0.020903950258044736  \\
            0.5310456949570432  0.02089511520064469  \\
            0.5321474495108961  0.020853354070469145  \\
            0.5332492040647488  0.02335344372677489  \\
            0.5343509586186016  0.0210599449925504  \\
            0.5354527131724545  0.020989949206498902  \\
            0.5365544677263072  0.02096884690419118  \\
            0.53765622228016  0.020966735200146802  \\
            0.5387579768340128  0.020980288372851336  \\
            0.5398597313878656  0.020948254152947945  \\
            0.5409614859417183  0.020939334049179366  \\
            0.5420632404955712  0.021047304196043254  \\
            0.543164995049424  0.02165779260473442  \\
            0.5442667496032767  0.021143022147135114  \\
            0.5453685041571296  0.020928846649758415  \\
            0.5464702587109823  0.02121068977691851  \\
            0.5475720132648351  0.020952585825662107  \\
            0.5486737678186879  0.020923268719708697  \\
            0.5497755223725407  0.020917285516110638  \\
        }
        ;
    \addlegendentry {$(20, 0, 0) $}
    \addplot[color={rgb,1:red,0.0;green,0.0;blue,1.0}, name path={f6c21950-05e0-483b-9d10-ec892bc78bf5}, draw opacity={1.0}, line width={1}, dashed, forget plot]
        table[row sep={\\}]
        {
            \\
            0.0  0.16665010541520522  \\
            0.001101754553852787  0.16666945710467837  \\
            0.002203509107705574  0.16675651276749287  \\
            0.0033052636615583607  0.16734153070040855  \\
            0.004407018215411148  0.16711314880858952  \\
            0.005508772769263934  0.16686869956729297  \\
            0.0066105273231167215  0.16699850796321375  \\
            0.007712281876969509  0.1689598905214798  \\
            0.008814036430822295  0.16668466694018305  \\
            0.009915790984675084  0.16660700239545323  \\
            0.011017545538527868  0.16660752214826471  \\
            0.012119300092380656  0.16668358238506392  \\
            0.013221054646233443  0.1666451988244561  \\
            0.01432280920008623  0.1667106181809403  \\
            0.015424563753939018  0.16738184685745647  \\
            0.016526318307791804  0.258058129209573  \\
            0.01762807286164459  0.16673255249463295  \\
            0.018729827415497377  0.1664166013208665  \\
            0.019831581969350167  0.16633038540266704  \\
            0.020933336523202953  0.16626910685019855  \\
            0.022035091077055736  0.16621412362227603  \\
            0.023136845630908526  0.16616171015278913  \\
            0.024238600184761313  0.1661105001233273  \\
            0.0253403547386141  0.1660593069986587  \\
            0.026442109292466886  0.166006393039424  \\
            0.027543863846319676  0.16594880355960206  \\
            0.02864561840017246  0.1659365875729268  \\
            0.029747372954025245  0.1658999784285305  \\
            0.030849127507878035  0.16586220228399096  \\
            0.03195088206173082  0.16582264101532082  \\
            0.03305263661558361  0.16578054711361015  \\
            0.034154391169436395  0.1657348991762377  \\
            0.03525614572328918  0.16568392556717546  \\
            0.03635790027714197  0.16562365888574132  \\
            0.037459654830994754  0.1655430309785207  \\
            0.03856140938484754  0.16539946149912163  \\
            0.039663163938700334  0.16519238075325868  \\
            0.04076491849255312  0.16516652127969048  \\
            0.04186667304640591  0.16514333937700815  \\
            0.042968427600258687  0.16512305103377214  \\
            0.04407018215411147  0.16510586992208068  \\
            0.045171936707964266  0.16509199829995005  \\
            0.04627369126181705  0.16508161225844611  \\
            0.04737544581566984  0.16507483947923463  \\
            0.048477200369522626  0.16507172756712796  \\
            0.04957895492337542  0.1650722013974929  \\
            0.0506807094772282  0.1650760092079363  \\
            0.051782464031080985  0.16508964286357472  \\
            0.05288421858493377  0.16519913670433717  \\
            0.05398597313878656  0.16606644469558235  \\
            0.05508772769263935  0.3985083227906712  \\
            0.05618948224649214  0.16427505082687374  \\
            0.05729123680034492  0.1641302625710679  \\
            0.058392991354197704  0.16411093395838316  \\
            0.05949474590805049  0.1640935242671908  \\
            0.060596500461903284  0.16407537707046  \\
            0.06169825501575607  0.1640591787009844  \\
            0.06280000956960886  0.1640474594256593  \\
            0.06390176412346164  0.16404258695401197  \\
            0.06500351867731442  0.1640471693385721  \\
            0.06610527323116722  0.16406452644623407  \\
            0.06720702778502  0.16409933501001528  \\
            0.06830878233887279  0.16415869169339226  \\
            0.06941053689272557  0.16425410020707867  \\
            0.07051229144657836  0.16440558078332634  \\
            0.07161404600043114  0.16465113328777028  \\
            0.07271580055428394  0.165071681939807  \\
            0.07381755510813673  0.16587034907808587  \\
            0.07491930966198951  0.16770743979781977  \\
            0.0760210642158423  0.1740610559576244  \\
            0.07712281876969508  0.26260709338779664  \\
            0.07822457332354787  0.18520009917028168  \\
            0.07932632787740067  0.1647536637853987  \\
            0.08042808243125345  0.1628634014317877  \\
            0.08152983698510624  0.1624393954331923  \\
            0.08263159153895902  0.16230332299659242  \\
            0.08373334609281181  0.1622413136902919  \\
            0.0848351006466646  0.1621939688175433  \\
            0.08593685520051737  0.16213957027619233  \\
            0.08703860975437017  0.1620884955402695  \\
            0.08814036430822295  0.162523339430562  \\
            0.08924211886207574  0.3029207992718088  \\
            0.09034387341592853  0.16527235190348047  \\
            0.09144562796978131  0.1637190117039149  \\
            0.0925473825236341  0.16331463538842408  \\
            0.09364913707748689  0.16314895167887872  \\
            0.09475089163133968  0.16307264329513646  \\
            0.09585264618519247  0.16304288404022624  \\
            0.09695440073904525  0.1630451096164483  \\
            0.09805615529289805  0.16307625919790866  \\
            0.09915790984675084  0.16314130022937476  \\
            0.1002596644006036  0.1632416318431117  \\
            0.1013614189544564  0.16344609068186575  \\
            0.10246317350830918  0.1638435939331616  \\
            0.10356492806216197  0.1648245918262831  \\
            0.10466668261601475  0.16919341530002982  \\
            0.10576843716986754  1.2607139639230687  \\
            0.10687019172372034  0.16384393212341042  \\
            0.10797194627757312  0.1622676361031101  \\
            0.10907370083142591  0.16214616687149414  \\
            0.1101754553852787  0.1621671932483585  \\
            0.11127720993913148  0.16220839967742715  \\
            0.11237896449298428  0.16224960508712266  \\
            0.11348071904683706  0.16228726591561443  \\
            0.11458247360068984  0.16232126529575622  \\
            0.11568422815454263  0.1623521613036777  \\
            0.11678598270839541  0.16238053688521617  \\
            0.1178877372622482  0.16240685303278796  \\
            0.11898949181610098  0.1624314308059138  \\
            0.12009124636995377  0.16245446361131927  \\
            0.12119300092380657  0.162476032818397  \\
            0.12229475547765935  0.1624961204924681  \\
            0.12339651003151214  0.1625146207409526  \\
            0.12449826458536492  0.162531358823025  \\
            0.12560001913921773  0.16254614898195754  \\
            0.1267017736930705  0.16255900993931502  \\
            0.1278035282469233  0.16257111329464405  \\
            0.12890528280077607  0.16259052461062262  \\
            0.13000703735462885  0.16270598742303458  \\
            0.13110879190848163  0.18878399201064042  \\
            0.13221054646233443  0.16271145961057878  \\
            0.1333123010161872  0.16254917998536939  \\
            0.13441405557004  0.1624555694551151  \\
            0.1355158101238928  0.16249084313954454  \\
            0.13661756467774558  0.16236158668608688  \\
            0.13771931923159836  0.16249807255227314  \\
            0.13882107378545114  0.16243722807277872  \\
            0.13992282833930392  0.16239864400600168  \\
            0.14102458289315672  0.16249986118638438  \\
            0.1421263374470095  0.16260360875695903  \\
            0.14322809200086228  0.16310104278310936  \\
            0.1443298465547151  0.16334567732754104  \\
            0.14543160110856787  0.16261667498645738  \\
            0.14653335566242065  0.16257792309742794  \\
            0.14763511021627346  0.1625642697710701  \\
            0.14873686477012624  0.16255209890621505  \\
            0.14983861932397902  0.16253830887999132  \\
            0.1509403738778318  0.16252252260788924  \\
            0.1520421284316846  0.16250486947862075  \\
            0.15314388298553738  0.16248554181582076  \\
            0.15424563753939016  0.16246468391060517  \\
            0.15534739209324297  0.16244235965771914  \\
            0.15644914664709575  0.1624185390486014  \\
            0.15755090120094853  0.1623930866328765  \\
            0.15865265575480134  0.1623657471905295  \\
            0.15975441030865412  0.16233613035726313  \\
            0.1608561648625069  0.16230370790265686  \\
            0.1619579194163597  0.16226787547809604  \\
            0.16305967397021248  0.16222826962282946  \\
            0.16416142852406526  0.1621861007853642  \\
            0.16526318307791804  0.16215004940426092  \\
            0.16636493763177085  0.16216823517886775  \\
            0.16746669218562363  0.1625965565340019  \\
            0.16856844673947638  0.17172957193563812  \\
            0.1696702012933292  0.1843040096354133  \\
            0.17077195584718197  0.1660445910359992  \\
            0.17187371040103475  0.1642056955480693  \\
            0.17297546495488753  0.16360610538958112  \\
            0.17407721950874033  0.16332597697244358  \\
            0.1751789740625931  0.1631713397160805  \\
            0.1762807286164459  0.16309966786152041  \\
            0.1773824831702987  0.1630526149962499  \\
            0.17848423772415148  0.1630359442909857  \\
            0.17958599227800426  0.16304865832702098  \\
            0.18068774683185707  0.16309862296549985  \\
            0.18178950138570985  0.16321074566091964  \\
            0.18289125593956262  0.16346171035200613  \\
            0.18399301049341543  0.16418159007102714  \\
            0.1850947650472682  0.16930697287960803  \\
            0.186196519601121  0.1671949073303529  \\
            0.18729827415497377  0.16212371476239434  \\
            0.18840002870882658  0.16210633423215376  \\
            0.18950178326267936  0.16216485117191  \\
            0.19060353781653214  0.16221404856527455  \\
            0.19170529237038494  0.16226448506581384  \\
            0.19280704692423772  0.16235136960166638  \\
            0.1939088014780905  0.1625827543100118  \\
            0.1950105560319433  0.1634273324528767  \\
            0.1961123105857961  0.16855756658816737  \\
            0.19721406513964887  0.44534708527295874  \\
            0.19831581969350168  0.1865736932106372  \\
            0.19941757424735443  0.16974045993233952  \\
            0.2005193288012072  0.16657228099997387  \\
            0.20162108335506  0.16539904133312694  \\
            0.2027228379089128  0.16482891928576807  \\
            0.20382459246276557  0.16451035619483187  \\
            0.20492634701661835  0.16431823313227506  \\
            0.20602810157047116  0.16419799305121152  \\
            0.20712985612432394  0.1641223768270539  \\
            0.20823161067817672  0.16407634760866843  \\
            0.2093333652320295  0.1640509431281003  \\
            0.2104351197858823  0.16404045303728884  \\
            0.2115368743397351  0.164040985902693  \\
            0.21263862889358787  0.16404966065464524  \\
            0.21374038344744067  0.16406408024210273  \\
            0.21484213800129345  0.1640819261454212  \\
            0.21594389255514623  0.1641007380076321  \\
            0.21704564710899904  0.1641195081375656  \\
            0.21814740166285182  0.16416475168332167  \\
            0.2192491562167046  0.16532060698647116  \\
            0.2203509107705574  0.16913006092379107  \\
            0.2214526653244102  0.16542601650007685  \\
            0.22255441987826297  0.16511810701267987  \\
            0.22365617443211575  0.16509684384371978  \\
            0.22475792898596855  0.165069982862126  \\
            0.22585968353982133  0.1650676716926123  \\
            0.2269614380936741  0.16506886181524535  \\
            0.22806319264752692  0.16507370772460972  \\
            0.22916494720137967  0.16508221568583  \\
            0.23026670175523245  0.16509429214106786  \\
            0.23136845630908526  0.16510978073517577  \\
            0.23247021086293804  0.16512848902930774  \\
            0.23357196541679082  0.16515020674262826  \\
            0.2346737199706436  0.1651747174485505  \\
            0.2357754745244964  0.16523753457696308  \\
            0.23687722907834918  0.16548115006156033  \\
            0.23797898363220196  0.16558309228456083  \\
            0.23908073818605477  0.16565121133553046  \\
            0.24018249273990755  0.1657060082287284  \\
            0.24128424729376033  0.16575392942215306  \\
            0.24238600184761314  0.16579755674786925  \\
            0.24348775640146592  0.1658381960902186  \\
            0.2445895109553187  0.16587670516313097  \\
            0.24569126550917147  0.1659137619170011  \\
            0.24679302006302428  0.16590999268155548  \\
            0.24789477461687706  0.16597155638127453  \\
            0.24899652917072984  0.1660262066550879  \\
            0.25009828372458265  0.16607791025163218  \\
            0.25120003827843546  0.16612890003908026  \\
            0.2523017928322882  0.1661805582540367  \\
            0.253403547386141  0.1662340237567564  \\
            0.25450530193999377  0.166291328984472  \\
            0.2556070564938466  0.16636024096214633  \\
            0.2567088110476994  0.16649372874094565  \\
            0.25781056560155213  0.1683824847864863  \\
            0.25891232015540494  0.5046329735263034  \\
            0.2600140747092577  0.16683306503002607  \\
            0.2611158292631105  0.1666617064692562  \\
            0.26221758381696325  0.16664515909479677  \\
            0.26331933837081606  0.29219160371512715  \\
            0.26442109292466887  0.1666108064708464  \\
            0.2655228474785216  0.16662262144071177  \\
            0.2666246020323744  0.1669139181472158  \\
            0.26772635658622723  0.23725013566583322  \\
            0.26882811114008  0.17955044893650915  \\
            0.2699298656939328  0.1664757982520889  \\
            0.2710316202477856  1.0761065600728088  \\
            0.27213337480163835  0.1668844898397161  \\
            0.27323512935549116  0.16669914244719178  \\
            0.27433688390934396  0.1666546235456757  \\
            0.2754386384631967  0.16665464463976207  \\
            0.27654039301704947  0.16669916508331173  \\
            0.2776421475709023  0.16688407215048656  \\
            0.2787439021247551  1.070512559654798  \\
            0.27984565667860783  0.16647520712116906  \\
            0.28094741123246064  0.17952810393307508  \\
            0.28204916578631345  0.2374465450549463  \\
            0.2831509203401662  0.1669134732514904  \\
            0.284252674894019  0.16662238793895287  \\
            0.2853544294478718  0.16661058054449301  \\
            0.28645618400172457  0.290470701636139  \\
            0.2875579385555774  0.166646509248907  \\
            0.2886596931094302  0.16666310943579238  \\
            0.28976144766328293  0.16683432956024374  \\
            0.29086320221713574  0.5047190255645536  \\
            0.29196495677098855  0.16838918319774884  \\
            0.2930667113248413  0.16650108555204057  \\
            0.2941684658786941  0.16636743233254392  \\
            0.2952702204325469  0.16629847645340318  \\
            0.29637197498639967  0.1662411656859764  \\
            0.2974737295402525  0.16618769766090716  \\
            0.2985754840941052  0.16613602251142726  \\
            0.29967723864795803  0.16608499060205917  \\
            0.30077899320181084  0.16603321111833955  \\
            0.3018807477556636  0.16597844226581618  \\
            0.3029825023095164  0.16591670496383368  \\
            0.3040842568633692  0.1659183941922934  \\
            0.30518601141722196  0.16588127245045978  \\
            0.30628776597107477  0.165842687663157  \\
            0.3073895205249276  0.16580196493717894  \\
            0.3084912750787803  0.16575825096091307  \\
            0.30959302963263313  0.16571024541208956  \\
            0.31069478418648594  0.1656553723679496  \\
            0.3117965387403387  0.16558719042560915  \\
            0.3128982932941915  0.16548520388570334  \\
            0.3140000478480443  0.1652416023786069  \\
            0.31510180240189706  0.16517912975144045  \\
            0.31620355695574986  0.1651545821535893  \\
            0.31730531150960267  0.1651328200186051  \\
            0.3184070660634554  0.16511405902429552  \\
            0.31950882061730823  0.16509850858651348  \\
            0.32061057517116104  0.16508636014636624  \\
            0.3217123297250138  0.16507776891536885  \\
            0.3228140842788666  0.16507282715458438  \\
            0.3239158388327194  0.16507152715657722  \\
            0.32501759338657216  0.16507371286745773  \\
            0.32611934794042496  0.1650968439644216  \\
            0.3272211024942777  0.1651152355627392  \\
            0.3283228570481305  0.1654130220947931  \\
            0.32942461160198333  0.1689877585717776  \\
            0.3305263661558361  0.1652251565761992  \\
            0.3316281207096889  0.1641571965057082  \\
            0.3327298752635417  0.16411929591621177  \\
            0.33383162981739445  0.16410245542705748  \\
            0.33493338437124726  0.1640843797958684  \\
            0.3360351389251  0.16406686816076774  \\
            0.33713689347895276  0.16405261181993844  \\
            0.33823864803280557  0.1640440150480328  \\
            0.3393404025866584  0.1640435117599763  \\
            0.3404421571405111  0.16405400014575292  \\
            0.34154391169436393  0.16407937942878242  \\
            0.34264566624821674  0.1641253624142658  \\
            0.3437474208020695  0.16420090975719062  \\
            0.3448491753559223  0.16432105214635076  \\
            0.34595092990977505  0.1645130353571231  \\
            0.34705268446362786  0.1648313886016112  \\
            0.34815443901748067  0.16540116743245237  \\
            0.3492561935713334  0.16657375773390484  \\
            0.3503579481251862  0.1697403315273482  \\
            0.35145970267903903  0.18656614784224557  \\
            0.3525614572328918  0.4452724121061749  \\
            0.3536632117867446  0.1685597612685935  \\
            0.3547649663405974  0.16343112491770861  \\
            0.35586672089445015  0.16258665372484984  \\
            0.35696847544830296  0.16235519760368067  \\
            0.35807023000215576  0.16226820229285752  \\
            0.3591719845560085  0.16221763099169378  \\
            0.3602737391098613  0.16216826066998133  \\
            0.36137549366371413  0.16210949586973064  \\
            0.3624772482175669  0.16212650850042176  \\
            0.3635790027714197  0.16720112983854035  \\
            0.3646807573252725  0.16931987530313306  \\
            0.36578251187912525  0.16418803580334695  \\
            0.36688426643297806  0.16346708638819982  \\
            0.36798602098683086  0.16321569428626626  \\
            0.3690877755406836  0.16310333758765969  \\
            0.3701895300945364  0.16305321841766116  \\
            0.3712912846483892  0.1630403853874667  \\
            0.372393039202242  0.16305694926059452  \\
            0.3734947937560948  0.16310388949381077  \\
            0.37459654830994754  0.16317500070231797  \\
            0.37569830286380035  0.1633289530469772  \\
            0.37680005741765316  0.16360838185482815  \\
            0.3779018119715059  0.1642070606292649  \\
            0.3790035665253587  0.16604388089939717  \\
            0.3801053210792115  0.18428370443685263  \\
            0.3812070756330643  0.17171554990698065  \\
            0.3823088301869171  0.16259654710727683  \\
            0.3834105847407699  0.16216922675928072  \\
            0.38451233929462264  0.1621511988982774  \\
            0.38561409384847545  0.16218724533171247  \\
            0.38671584840232825  0.1622293672234627  \\
            0.387817602956181  0.1622689163396298  \\
            0.3889193575100338  0.16230469250126053  \\
            0.3900211120638866  0.16233706259575273  \\
            0.3911228666177394  0.16236663200209206  \\
            0.3922246211715922  0.1623939291389357  \\
            0.393326375725445  0.1624193443293288  \\
            0.39442813027929774  0.1624431327837618  \\
            0.39552988483315055  0.1624654300864707  \\
            0.39663163938700335  0.16248626655183282  \\
            0.39773339394085605  0.1625055787098392  \\
            0.39883514849470886  0.16252322262754929  \\
            0.3999369030485616  0.16253900578872046  \\
            0.4010386576024144  0.16255279683512658  \\
            0.4021404121562672  0.16256496566770226  \\
            0.40324216671012  0.16257858998665545  \\
            0.4043439212639728  0.16261719188863424  \\
            0.4054456758178256  0.16334537646847988  \\
            0.40654743037167834  0.1631053506678583  \\
            0.40764918492553115  0.16260640379561886  \\
            0.40875093947938396  0.1625026840734496  \\
            0.4098526940332367  0.16240181760865524  \\
            0.4109544485870895  0.1624374432724633  \\
            0.4120562031409423  0.16249773863327846  \\
            0.4131579576947951  0.16236107329850796  \\
            0.4142597122486479  0.1624909198209004  \\
            0.4153614668025007  0.16245260793016866  \\
            0.41646322135635344  0.16254642677974557  \\
            0.41756497591020625  0.16270838137826268  \\
            0.418666730464059  0.1887402604774263  \\
            0.4197684850179118  0.16270571066005393  \\
            0.4208702395717646  0.1625899045223959  \\
            0.42197199412561737  0.16257042561318324  \\
            0.4230737486794702  0.16255831179259853  \\
            0.424175503233323  0.16254545182158342  \\
            0.42527725778717573  0.16253066106358943  \\
            0.42637901234102854  0.1625139169115801  \\
            0.42748076689488135  0.1624954042783615  \\
            0.4285825214487341  0.1624752980761927  \\
            0.4296842760025869  0.16245370462642855  \\
            0.4307860305564397  0.16243064224070833  \\
            0.43188778511029247  0.1624060297712601  \\
            0.4329895396641453  0.16237967386525778  \\
            0.4340912942179981  0.16235125341707976  \\
            0.43519304877185083  0.1623203074603019  \\
            0.43629480332570364  0.16228625354513115  \\
            0.43739655787955645  0.16224853552378984  \\
            0.4384983124334092  0.16220727620882827  \\
            0.439600066987262  0.16216603710611657  \\
            0.4407018215411148  0.16214505967074683  \\
            0.44180357609496756  0.16226692584443117  \\
            0.4429053306488204  0.16384619117318058  \\
            0.4440070852026732  1.2615305708581765  \\
            0.44510883975652593  0.16919749874518317  \\
            0.44621059431037874  0.16482396513176098  \\
            0.4473123488642315  0.1638417221121308  \\
            0.4484141034180843  0.16344345552204934  \\
            0.4495158579719371  0.1632383168245964  \\
            0.45061761252578986  0.16313714327182077  \\
            0.45171936707964266  0.163071979381864  \\
            0.45282112163349547  0.16304072231146013  \\
            0.4539228761873482  0.1630383861863423  \\
            0.45502463074120103  0.16306801235876786  \\
            0.45612638529505384  0.16314413437119188  \\
            0.4572281398489066  0.1633095105478969  \\
            0.45832989440275934  0.16371324685026672  \\
            0.45943164895661215  0.16526443452879608  \\
            0.4605334035104649  0.30280362661339555  \\
            0.4616351580643177  0.16252059589012105  \\
            0.4627369126181705  0.16208550435033994  \\
            0.46383866717202327  0.16213627190284846  \\
            0.4649404217258761  0.16219046641639207  \\
            0.4660421762797288  0.16223766044602542  \\
            0.46714393083358163  0.16229954743125266  \\
            0.46824568538743444  0.1624355235757106  \\
            0.4693474399412872  0.16285951062861012  \\
            0.47044919449514  0.16475022399969844  \\
            0.4715509490489928  0.18520379149061403  \\
            0.47265270360284556  0.2626407064171822  \\
            0.47375445815669837  0.1740632178252702  \\
            0.4748562127105512  0.16770655498029566  \\
            0.4759579672644039  0.1658684892263987  \\
            0.47705972181825673  0.16506936096149785  \\
            0.47816147637210954  0.16464854759754516  \\
            0.4792632309259623  0.1644028251651643  \\
            0.4803649854798151  0.16425122813189388  \\
            0.4814667400336679  0.1641557374546809  \\
            0.48256849458752066  0.16409632366899862  \\
            0.48367024914137347  0.16406147934530846  \\
            0.4847720036952263  0.16404410819877113  \\
            0.485873758249079  0.16403953833379908  \\
            0.48697551280293183  0.16404446157652822  \\
            0.48807726735678464  0.16405629486872145  \\
            0.4891790219106374  0.16407272594875869  \\
            0.4902807764644902  0.1640913620720323  \\
            0.49138253101834295  0.16410992825911058  \\
            0.49248428557219576  0.16413277284489525  \\
            0.49358604012604856  0.16429701433818453  \\
            0.4946877946799013  0.4067248799672251  \\
            0.4957895492337541  0.16609926041188003  \\
            0.49689130378760693  0.1652052362773204  \\
            0.4979930583414597  0.16509072684280648  \\
            0.4990948128953125  0.16507234827994524  \\
            0.5001965674491653  0.16506840661433847  \\
            0.501298322003018  0.1650678153171278  \\
            0.5024000765568709  0.16507082455454866  \\
            0.5035018311107237  0.16507750794468187  \\
            0.5046035856645764  0.1650878165113094  \\
            0.5057053402184292  0.16510162135282488  \\
            0.506807094772282  0.16511874530862147  \\
            0.5079088493261348  0.16513898519140072  \\
            0.5090106038799875  0.16516212652424594  \\
            0.5101123584338404  0.1651879526215469  \\
            0.5112141129876931  0.16539541661193308  \\
            0.5123158675415459  0.1655389578974689  \\
            0.5134176220953988  0.16561953122279116  \\
            0.5145193766492515  0.16567972783261734  \\
            0.5156211312031043  0.16573062047754314  \\
            0.5167228857569571  0.16577618216752124  \\
            0.5178246403108099  0.16581819041867807  \\
            0.5189263948646626  0.16585767167361132  \\
            0.5200281494185154  0.1658953771643309  \\
            0.5211299039723682  0.16593192745202054  \\
            0.522231658526221  0.1659419964627153  \\
            0.5233334130800738  0.1659994418930165  \\
            0.5244351676339265  0.16605225984802255  \\
            0.5255369221877794  0.16610339500915308  \\
            0.5266386767416321  0.16615457665033895  \\
            0.5277404312954849  0.16620698200723882  \\
            0.5288421858493377  0.1662619640966209  \\
            0.5299439404031905  0.16632322432015073  \\
            0.5310456949570432  0.16640935060415787  \\
            0.5321474495108961  0.1667250532001099  \\
            0.5332492040647488  0.25823213988242966  \\
            0.5343509586186016  0.16738020288685884  \\
            0.5354527131724545  0.16670927430895688  \\
            0.5365544677263072  0.16664378403714544  \\
            0.53765622228016  0.1666826245066006  \\
            0.5387579768340128  0.16660600943828957  \\
            0.5398597313878656  0.16660720016131444  \\
            0.5409614859417183  0.16668496612934697  \\
            0.5420632404955712  0.16896439757659332  \\
            0.543164995049424  0.16700203698486102  \\
            0.5442667496032767  0.16686818628197603  \\
            0.5453685041571296  0.16711958873366106  \\
            0.5464702587109823  0.16734424502360948  \\
            0.5475720132648351  0.1667565736532289  \\
            0.5486737678186879  0.16666942307071317  \\
            0.5497755223725407  0.16665010541542544  \\
        }
        ;
    \addplot[color={rgb,1:red,0.0;green,0.0;blue,1.0}, name path={5b20e200-af15-47a9-aa80-27d983b30ac4}, draw opacity={1.0}, line width={1}, dotted, forget plot]
        table[row sep={\\}]
        {
            \\
            0.0  3.6833424269105066  \\
            0.001101754553852787  3.6924773201815837  \\
            0.002203509107705574  3.7403297704366154  \\
            0.0033052636615583607  4.214148218418403  \\
            0.004407018215411148  11.356299928208822  \\
            0.005508772769263934  4.421068928745918  \\
            0.0066105273231167215  5.478748391230931  \\
            0.007712281876969509  6.892411515121708  \\
            0.008814036430822295  3.9474398711822007  \\
            0.009915790984675084  3.8087315127236154  \\
            0.011017545538527868  3.8254094026868803  \\
            0.012119300092380656  3.8389470254709344  \\
            0.013221054646233443  3.8532720925946116  \\
            0.01432280920008623  4.085551679860199  \\
            0.015424563753939018  5.702472552440729  \\
            0.016526318307791804  43.871826826530175  \\
            0.01762807286164459  4.8966585841604315  \\
            0.018729827415497377  3.9746428578107778  \\
            0.019831581969350167  3.787381770567385  \\
            0.020933336523202953  3.720399765340314  \\
            0.022035091077055736  3.689455901092693  \\
            0.023136845630908526  3.673049240598128  \\
            0.024238600184761313  3.663737767169948  \\
            0.0253403547386141  3.658489933048697  \\
            0.026442109292466886  3.6561936985652976  \\
            0.027543863846319676  3.65852653860357  \\
            0.02864561840017246  3.646746882520617  \\
            0.029747372954025245  3.6461271379055153  \\
            0.030849127507878035  3.6461009249180316  \\
            0.03195088206173082  3.646579254739299  \\
            0.03305263661558361  3.6475308099774026  \\
            0.034154391169436395  3.6489867214984164  \\
            0.03525614572328918  3.651067455263194  \\
            0.03635790027714197  3.654054300079787  \\
            0.037459654830994754  3.6586096061145867  \\
            0.03856140938484754  3.6680133878882546  \\
            0.039663163938700334  3.6543726428003267  \\
            0.04076491849255312  3.6567428041283327  \\
            0.04186667304640591  3.6596832624078828  \\
            0.042968427600258687  3.6632427146175566  \\
            0.04407018215411147  3.6674703030029776  \\
            0.045171936707964266  3.6724115764304917  \\
            0.04627369126181705  3.678102751577577  \\
            0.04737544581566984  3.684562874890961  \\
            0.048477200369522626  3.6917836557969848  \\
            0.04957895492337542  3.69971715425971  \\
            0.0506807094772282  3.708262286632193  \\
            0.051782464031080985  3.8240826059921527  \\
            0.05288421858493377  3.896281788240878  \\
            0.05398597313878656  5.010618766955159  \\
            0.05508772769263935  56.60342159519081  \\
            0.05618948224649214  4.167786709644592  \\
            0.05729123680034492  3.7856978819987934  \\
            0.058392991354197704  3.714844306499369  \\
            0.05949474590805049  3.6947432690318127  \\
            0.060596500461903284  3.69061714571081  \\
            0.06169825501575607  3.694128050108717  \\
            0.06280000956960886  3.7028843024708604  \\
            0.06390176412346164  3.7164989684856886  \\
            0.06500351867731442  3.735621059779808  \\
            0.06610527323116722  3.7617850888879985  \\
            0.06720702778502  3.7976532610122873  \\
            0.06830878233887279  3.847663048743862  \\
            0.06941053689272557  3.9193749041676833  \\
            0.07051229144657836  4.026311879778441  \\
            0.07161404600043114  4.194381210594864  \\
            0.07271580055428394  4.477991862983304  \\
            0.07381755510813673  5.006862771940095  \\
            0.07491930966198951  6.155822265719687  \\
            0.0760210642158423  9.471474837837375  \\
            0.07712281876969508  34.625809779308874  \\
            0.07822457332354787  18.546069601219866  \\
            0.07932632787740067  8.11183753173703  \\
            0.08042808243125345  5.8299414022809275  \\
            0.08152983698510624  4.9447519861812514  \\
            0.08263159153895902  4.522404582843299  \\
            0.08373334609281181  4.306804456877982  \\
            0.0848351006466646  4.214613495199626  \\
            0.08593685520051737  4.2442361003241285  \\
            0.08703860975437017  4.536095039601956  \\
            0.08814036430822295  6.21636079139032  \\
            0.08924211886207574  49.375417988195196  \\
            0.09034387341592853  5.4899563875828425  \\
            0.09144562796978131  4.279895976809477  \\
            0.0925473825236341  3.9841016095234973  \\
            0.09364913707748689  3.869412107516007  \\
            0.09475089163133968  3.815413192778609  \\
            0.09585264618519247  3.7897303678690943  \\
            0.09695440073904525  3.7819041960102093  \\
            0.09805615529289805  3.789712072888899  \\
            0.09915790984675084  3.816448093032271  \\
            0.1002596644006036  3.877187156505453  \\
            0.1013614189544564  3.989381589802086  \\
            0.10246317350830918  4.240060065982633  \\
            0.10356492806216197  4.912984688822828  \\
            0.10466668261601475  7.691284238108815  \\
            0.10576843716986754  234.3380183248194  \\
            0.10687019172372034  7.49665551877458  \\
            0.10797194627757312  4.9503594935926785  \\
            0.10907370083142591  4.2851067676978225  \\
            0.1101754553852787  4.018893662431172  \\
            0.11127720993913148  3.8863789407850775  \\
            0.11237896449298428  3.810817709552267  \\
            0.11348071904683706  3.7635698326680775  \\
            0.11458247360068984  3.7320217368881186  \\
            0.11568422815454263  3.709927389720816  \\
            0.11678598270839541  3.6939139786988138  \\
            0.1178877372622482  3.68203877911524  \\
            0.11898949181610098  3.673131270091643  \\
            0.12009124636995377  3.6664713922169248  \\
            0.12119300092380657  3.661627496939536  \\
            0.12229475547765935  3.6583817340232625  \\
            0.12339651003151214  3.656719513339381  \\
            0.12449826458536492  3.6568987975440614  \\
            0.12560001913921773  3.65968982225098  \\
            0.1267017736930705  3.667130596933306  \\
            0.1278035282469233  3.685355777137582  \\
            0.12890528280077607  3.739631626027992  \\
            0.13000703735462885  4.0402824413599365  \\
            0.13110879190848163  21.30401638856547  \\
            0.13221054646233443  3.935732696426377  \\
            0.1333123010161872  3.743170315990809  \\
            0.13441405557004  3.756638289101804  \\
            0.1355158101238928  3.6658308854716872  \\
            0.13661756467774558  3.793530537085839  \\
            0.13771931923159836  3.6480418223292133  \\
            0.13882107378545114  3.6766648439722873  \\
            0.13992282833930392  3.9487512683897905  \\
            0.14102458289315672  3.73241638454273  \\
            0.1421263374470095  3.791038482490964  \\
            0.14322809200086228  4.571337309914866  \\
            0.1443298465547151  5.253298824518656  \\
            0.14543160110856787  3.8167615359389115  \\
            0.14653335566242065  3.7043182718593646  \\
            0.14763511021627346  3.6741555454341808  \\
            0.14873686477012624  3.662610748741965  \\
            0.14983861932397902  3.657890973377008  \\
            0.1509403738778318  3.6565526059024767  \\
            0.1520421284316846  3.6573492899119655  \\
            0.15314388298553738  3.6598153410087337  \\
            0.15424563753939016  3.663844780816785  \\
            0.15534739209324297  3.6695565532932592  \\
            0.15644914664709575  3.6772701617780688  \\
            0.15755090120094853  3.6875475870954473  \\
            0.15865265575480134  3.7013062630287568  \\
            0.15975441030865412  3.7200474634857303  \\
            0.1608561648625069  3.746313184678472  \\
            0.1619579194163597  3.7846501564749486  \\
            0.16305967397021248  3.8438235423127773  \\
            0.16416142852406526  3.9425156708553373  \\
            0.16526318307791804  4.126452739067625  \\
            0.16636493763177085  4.532672431048963  \\
            0.16746669218562363  5.735698180203577  \\
            0.16856844673947638  13.187860263178555  \\
            0.1696702012933292  14.521033528772941  \\
            0.17077195584718197  5.745129091688099  \\
            0.17187371040103475  4.485287559638875  \\
            0.17297546495488753  4.08825807421144  \\
            0.17407721950874033  3.922567207049678  \\
            0.1751789740625931  3.856042158624977  \\
            0.1762807286164459  3.800255802216279  \\
            0.1773824831702987  3.7837735269632353  \\
            0.17848423772415148  3.783858680740172  \\
            0.17958599227800426  3.7999594766122238  \\
            0.18068774683185707  3.837687483733549  \\
            0.18178950138570985  3.9153626428463157  \\
            0.18289125593956262  4.093010391732394  \\
            0.18399301049341543  4.642309709379657  \\
            0.1850947650472682  8.312925263587115  \\
            0.186196519601121  11.510711857930287  \\
            0.18729827415497377  4.990340756810062  \\
            0.18840002870882658  4.335504227699105  \\
            0.18950178326267936  4.210878896442365  \\
            0.19060353781653214  4.247116105277936  \\
            0.19170529237038494  4.396112742037144  \\
            0.19280704692423772  4.697916976300424  \\
            0.1939088014780905  5.299329034964145  \\
            0.1950105560319433  6.668071540888263  \\
            0.1961123105857961  10.972104401783838  \\
            0.19721406513964887  76.32978340521382  \\
            0.19831581969350168  14.447446855168963  \\
            0.19941757424735443  7.312922716333026  \\
            0.2005193288012072  5.458474110416769  \\
            0.20162108335506  4.697690316876714  \\
            0.2027228379089128  4.316216385440358  \\
            0.20382459246276557  4.10018164580834  \\
            0.20492634701661835  3.96715668544549  \\
            0.20602810157047116  3.8800978832696402  \\
            0.20712985612432394  3.82047494693068  \\
            0.20823161067817672  3.778252331332333  \\
            0.2093333652320295  3.7476663532083423  \\
            0.2104351197858823  3.725281793462984  \\
            0.2115368743397351  3.709053032869829  \\
            0.21263862889358787  3.697897362772071  \\
            0.21374038344744067  3.691632674591583  \\
            0.21484213800129345  3.6914337112656357  \\
            0.21594389255514623  3.7017669365797783  \\
            0.21704564710899904  3.7389421477474754  \\
            0.21814740166285182  3.888038828342158  \\
            0.2192491562167046  5.379878973411684  \\
            0.2203509107705574  6.422169434785005  \\
            0.2214526653244102  4.067638013542551  \\
            0.22255441987826297  3.832297969070113  \\
            0.22365617443211575  4.054588126221157  \\
            0.22475792898596855  3.703937899173449  \\
            0.22585968353982133  3.69568190960638  \\
            0.2269614380936741  3.688095684593358  \\
            0.22806319264752692  3.6812520204730097  \\
            0.22916494720137967  3.6751773610803795  \\
            0.23026670175523245  3.669864890408561  \\
            0.23136845630908526  3.665285834989438  \\
            0.23247021086293804  3.6613984565870474  \\
            0.23357196541679082  3.6581547968496833  \\
            0.2346737199706436  3.6555055115163304  \\
            0.2357754745244964  3.739404158340728  \\
            0.23687722907834918  3.6620560025712554  \\
            0.23797898363220196  3.656079018464526  \\
            0.23908073818605477  3.6524359483687903  \\
            0.24018249273990755  3.649953986729501  \\
            0.24128424729376033  3.6482095725884265  \\
            0.24238600184761314  3.6470153929105047  \\
            0.24348775640146592  3.6463008337817873  \\
            0.2445895109553187  3.6460677190569584  \\
            0.24569126550917147  3.646375709452422  \\
            0.24679302006302428  3.671804443367219  \\
            0.24789477461687706  3.6563778059210783  \\
            0.24899652917072984  3.65699985392524  \\
            0.25009828372458265  3.660727307792638  \\
            0.25120003827843546  3.6677695410403954  \\
            0.2523017928322882  3.680069439771572  \\
            0.253403547386141  3.7023186180744863  \\
            0.25450530193999377  3.746815187412785  \\
            0.2556070564938466  3.8540273034644508  \\
            0.2567088110476994  4.2265227765544084  \\
            0.25781056560155213  7.897359329728197  \\
            0.25891232015540494  91.29004828535926  \\
            0.2600140747092577  4.442985981101767  \\
            0.2611158292631105  3.9311978587480683  \\
            0.26221758381696325  3.81596285672374  \\
            0.26331933837081606  66.70797500250262  \\
            0.26442109292466887  3.875234456089491  \\
            0.2655228474785216  3.841193402691257  \\
            0.2666246020323744  4.294762349265451  \\
            0.26772635658622723  38.38734081863993  \\
            0.26882811114008  11.188311895667416  \\
            0.2699298656939328  4.49000617775321  \\
            0.2710316202477856  211.74691893360475  \\
            0.27213337480163835  3.826644423539467  \\
            0.27323512935549116  3.707639239347696  \\
            0.27433688390934396  3.6854345174227627  \\
            0.2754386384631967  3.6854290176073783  \\
            0.27654039301704947  3.707616029372306  \\
            0.2776421475709023  3.8265463817069625  \\
            0.2787439021247551  211.67133860446737  \\
            0.27984565667860783  4.489670349773424  \\
            0.28094741123246064  11.18663498179925  \\
            0.28204916578631345  38.38224316872619  \\
            0.2831509203401662  4.294677880922602  \\
            0.284252674894019  3.84117249481472  \\
            0.2853544294478718  3.8752305551641753  \\
            0.28645618400172457  66.75073046371318  \\
            0.2875579385555774  3.8160029504873383  \\
            0.2886596931094302  3.931260768939875  \\
            0.28976144766328293  4.443192428376597  \\
            0.29086320221713574  91.29654501530754  \\
            0.29196495677098855  7.897864336864879  \\
            0.2930667113248413  4.226599092935942  \\
            0.2941684658786941  3.8540481067855623  \\
            0.2952702204325469  3.7468176759612284  \\
            0.29637197498639967  3.7023124155305873  \\
            0.2974737295402525  3.6800581852561813  \\
            0.2985754840941052  3.6677549608041455  \\
            0.29967723864795803  3.6607103472499434  \\
            0.30077899320181084  3.6569810486997696  \\
            0.3018807477556636  3.656357277924224  \\
            0.3029825023095164  3.671778642903511  \\
            0.3040842568633692  3.646354779279134  \\
            0.30518601141722196  3.646046816226761  \\
            0.30628776597107477  3.6462803005366458  \\
            0.3073895205249276  3.646995578539017  \\
            0.3084912750787803  3.64819080775521  \\
            0.30959302963263313  3.6499365285838405  \\
            0.31069478418648594  3.652419909611568  \\
            0.3117965387403387  3.6560643314158114  \\
            0.3128982932941915  3.662042516474057  \\
            0.3140000478480443  3.739405618452028  \\
            0.31510180240189706  3.6554893799304837  \\
            0.31620355695574986  3.6581387506498366  \\
            0.31730531150960267  3.6613825132244737  \\
            0.3184070660634554  3.665270002874635  \\
            0.31950882061730823  3.669849166061249  \\
            0.32061057517116104  3.6751617258474836  \\
            0.3217123297250138  3.6812364370675694  \\
            0.3228140842788666  3.6880800939471112  \\
            0.3239158388327194  3.6956662288867164  \\
            0.32501759338657216  3.703922022270452  \\
            0.32611934794042496  4.054353951291024  \\
            0.3272211024942777  3.8323024173548625  \\
            0.3283228570481305  4.067636953943484  \\
            0.32942461160198333  6.422033196560763  \\
            0.3305263661558361  5.379720465059663  \\
            0.3316281207096889  3.8880092316649737  \\
            0.3327298752635417  3.738930663221301  \\
            0.33383162981739445  3.7017622868487416  \\
            0.33493338437124726  3.6914333337396883  \\
            0.3360351389251  3.6916359196622155  \\
            0.33713689347895276  3.6979042687105386  \\
            0.33823864803280557  3.7090640202397753  \\
            0.3393404025866584  3.7252976049990445  \\
            0.3404421571405111  3.74768809882262  \\
            0.34154391169436393  3.7782816157970593  \\
            0.34264566624821674  3.8205141050091878  \\
            0.3437474208020695  3.880150393108243  \\
            0.3448491753559223  3.9672279124813885  \\
            0.34595092990977505  4.100280248339057  \\
            0.34705268446362786  4.316357154755891  \\
            0.34815443901748067  4.697900470400176  \\
            0.3492561935713334  5.458809408036767  \\
            0.3503579481251862  7.3135221221633655  \\
            0.35145970267903903  14.448893793349539  \\
            0.3525614572328918  76.33774504416485  \\
            0.3536632117867446  10.973094079147852  \\
            0.3547649663405974  6.668526464031374  \\
            0.35586672089445015  5.299584647535529  \\
            0.35696847544830296  4.698072047878249  \\
            0.35807023000215576  4.396208101788943  \\
            0.3591719845560085  4.247169581759017  \\
            0.3602737391098613  4.210894317963077  \\
            0.36137549366371413  4.335466703214606  \\
            0.3624772482175669  4.990171836306285  \\
            0.3635790027714197  11.509735801548013  \\
            0.3646807573252725  8.312227475975883  \\
            0.36578251187912525  4.642148785256001  \\
            0.36688426643297806  4.0929476892711385  \\
            0.36798602098683086  3.9153327997130116  \\
            0.3690877755406836  3.8376725290922766  \\
            0.3701895300945364  3.7999531195587792  \\
            0.3712912846483892  3.7838586915066186  \\
            0.372393039202242  3.783779623622697  \\
            0.3734947937560948  3.8002691510820403  \\
            0.37459654830994754  3.8560658783423105  \\
            0.37569830286380035  3.9226016486366992  \\
            0.37680005741765316  4.088316909295054  \\
            0.3779018119715059  4.485397276315445  \\
            0.3790035665253587  5.745373751681264  \\
            0.3801053210792115  14.521972696974904  \\
            0.3812070756330643  13.18864695746216  \\
            0.3823088301869171  5.735904622495614  \\
            0.3834105847407699  4.532761066508769  \\
            0.38451233929462264  4.12649779193383  \\
            0.38561409384847545  3.94254049579044  \\
            0.38671584840232825  3.843837569615683  \\
            0.387817602956181  3.7846578397352806  \\
            0.3889193575100338  3.746316875280014  \\
            0.3900211120638866  3.7200485113104946  \\
            0.3911228666177394  3.701305495172611  \\
            0.3922246211715922  3.687545537431039  \\
            0.393326375725445  3.677267189777991  \\
            0.39442813027929774  3.6695529095804846  \\
            0.39552988483315055  3.6638406449421033  \\
            0.39663163938700335  3.659810844202172  \\
            0.39773339394085605  3.6573445287720334  \\
            0.39883514849470886  3.6565476504369423  \\
            0.3999369030485616  3.657885869629038  \\
            0.4010386576024144  3.6626055151762604  \\
            0.4021404121562672  3.6741501527522664  \\
            0.40324216671012  3.7043125583380996  \\
            0.4043439212639728  3.816754691336004  \\
            0.4054456758178256  5.25327953529663  \\
            0.40654743037167834  4.571324744965942  \\
            0.40764918492553115  3.791032713427005  \\
            0.40875093947938396  3.7324142617168468  \\
            0.4098526940332367  3.948647769077468  \\
            0.4109544485870895  3.6766604841021113  \\
            0.4120562031409423  3.6480420905497364  \\
            0.4131579576947951  3.7934245270029847  \\
            0.4142597122486479  3.665832266629562  \\
            0.4153614668025007  3.756637852740339  \\
            0.41646322135635344  3.7431742378677604  \\
            0.41756497591020625  3.935739796176736  \\
            0.418666730464059  21.304132767440432  \\
            0.4197684850179118  4.040291444405872  \\
            0.4208702395717646  3.739637698475118  \\
            0.42197199412561737  3.6853612934515705  \\
            0.4230737486794702  3.667135901622607  \\
            0.424175503233323  3.6596949910380885  \\
            0.42527725778717573  3.6569038314189113  \\
            0.42637901234102854  3.65672437888065  \\
            0.42748076689488135  3.658386373275983  \\
            0.4285825214487341  3.6616318272755652  \\
            0.4296842760025869  3.6664753010520292  \\
            0.4307860305564397  3.6731346041625135  \\
            0.43188778511029247  3.6820413269343955  \\
            0.4329895396641453  3.693915441297536  \\
            0.4340912942179981  3.7099273310973526  \\
            0.43519304877185083  3.732019496425666  \\
            0.43629480332570364  3.7635643613430423  \\
            0.43739655787955645  3.8108072419504087  \\
            0.4384983124334092  3.8863602792205074  \\
            0.439600066987262  4.018860430397085  \\
            0.4407018215411148  4.285044469789346  \\
            0.44180357609496756  4.950228182143481  \\
            0.4429053306488204  7.496297063926716  \\
            0.4440070852026732  234.32370970688174  \\
            0.44510883975652593  7.690865316964096  \\
            0.44621059431037874  4.912825872265587  \\
            0.4473123488642315  4.239980949767776  \\
            0.4484141034180843  3.989336877099688  \\
            0.4495158579719371  3.8771605550291093  \\
            0.45061761252578986  3.8164301591199457  \\
            0.45171936707964266  3.789702590281933  \\
            0.45282112163349547  3.7819011986315663  \\
            0.4539228761873482  3.7897334075159907  \\
            0.45502463074120103  3.8154233968962905  \\
            0.45612638529505384  3.8694332917409517  \\
            0.4572281398489066  3.984144221907359  \\
            0.45832989440275934  4.279992816574297  \\
            0.45943164895661215  5.490256286555892  \\
            0.4605334035104649  49.38144936248425  \\
            0.4616351580643177  6.216717264795943  \\
            0.4627369126181705  4.5361792890532815  \\
            0.46383866717202327  4.244243357742762  \\
            0.4649404217258761  4.214578661832671  \\
            0.4660421762797288  4.30673133291769  \\
            0.46714393083358163  4.522282654436671  \\
            0.46824568538743444  4.944553999281937  \\
            0.4693474399412872  5.82960520780376  \\
            0.47044919449514  8.111193452377888  \\
            0.4715509490489928  18.544232629390415  \\
            0.47265270360284556  34.62218883281108  \\
            0.47375445815669837  9.470603100287798  \\
            0.4748562127105512  6.155382627149818  \\
            0.4759579672644039  5.006600027995139  \\
            0.47705972181825673  4.477821009215133  \\
            0.47816147637210954  4.194263937834741  \\
            0.4792632309259623  4.026228338454209  \\
            0.4803649854798151  3.9193138679231687  \\
            0.4814667400336679  3.8476177445938564  \\
            0.48256849458752066  3.7976193905140225  \\
            0.48367024914137347  3.761759812954316  \\
            0.4847720036952263  3.735602446836064  \\
            0.485873758249079  3.7164856828250645  \\
            0.48697551280293183  3.7028754284429737  \\
            0.48807726735678464  3.694123006385524  \\
            0.4891790219106374  3.690615685805533  \\
            0.4902807764644902  3.694745631836601  \\
            0.49138253101834295  3.71485181739352  \\
            0.49248428557219576  3.7857156595118697  \\
            0.49358604012604856  4.167844548584532  \\
            0.4946877946799013  56.60627148104107  \\
            0.4957895492337541  5.010517231686638  \\
            0.49689130378760693  3.8962821647099886  \\
            0.4979930583414597  3.8240675501454886  \\
            0.4990948128953125  3.7082783075513377  \\
            0.5001965674491653  3.6997329184737646  \\
            0.501298322003018  3.691799279619843  \\
            0.5024000765568709  3.684578453073202  \\
            0.5035018311107237  3.6781183549399614  \\
            0.5046035856645764  3.672427252804864  \\
            0.5057053402184292  3.667486079931499  \\
            0.506807094772282  3.6632586027369776  \\
            0.5079088493261348  3.6596992588663  \\
            0.5090106038799875  3.6567588956702664  \\
            0.5101123584338404  3.654388808360292  \\
            0.5112141129876931  3.66802604242706  \\
            0.5123158675415459  3.658623680863725  \\
            0.5134176220953988  3.6540696444873726  \\
            0.5145193766492515  3.6510842064200584  \\
            0.5156211312031043  3.649004857715932  \\
            0.5167228857569571  3.6475501379311654  \\
            0.5178246403108099  3.6465994719869843  \\
            0.5189263948646626  3.6461216864513957  \\
            0.5200281494185154  3.6461480964702724  \\
            0.5211299039723682  3.6467677010413344  \\
            0.522231658526221  3.6585483060789046  \\
            0.5233334130800738  3.656213337849972  \\
            0.5244351676339265  3.658507862236037  \\
            0.5255369221877794  3.6637536257993477  \\
            0.5266386767416321  3.6730623112214347  \\
            0.5277404312954849  3.6894649228150382  \\
            0.5288421858493377  3.7204022877964835  \\
            0.5299439404031905  3.787372079231493  \\
            0.5310456949570432  3.9746031965233803  \\
            0.5321474495108961  4.896493976654565  \\
            0.5332492040647488  43.86976101195644  \\
            0.5343509586186016  5.701887856087375  \\
            0.5354527131724545  4.08544673267238  \\
            0.5365544677263072  3.85322879007001  \\
            0.53765622228016  3.838853405417694  \\
            0.5387579768340128  3.8253253628002652  \\
            0.5398597313878656  3.808742487192769  \\
            0.5409614859417183  3.9474792843047295  \\
            0.5420632404955712  6.8926174214611295  \\
            0.543164995049424  5.4791612817243145  \\
            0.5442667496032767  4.4213298533110645  \\
            0.5453685041571296  11.358633015541743  \\
            0.5464702587109823  4.21448427575817  \\
            0.5475720132648351  3.7403739029333583  \\
            0.5486737678186879  3.692489738939825  \\
            0.5497755223725407  3.6833424269120014  \\
        }
        ;
    \addplot[color={rgb,1:red,1.0;green,0.0;blue,0.0}, name path={d1d4f0a7-6234-408c-ba9c-70e1f4463e54}, draw opacity={1.0}, line width={1}, solid]
        table[row sep={\\}]
        {
            \\
            0.0  0.02252223919973492  \\
            0.001101754553852787  0.022522460517840178  \\
            0.002203509107705574  0.02252312308002555  \\
            0.0033052636615583607  0.02252423259282162  \\
            0.004407018215411148  0.02252579856018541  \\
            0.005508772769263934  0.02252783414989028  \\
            0.0066105273231167215  0.02253035573352383  \\
            0.007712281876969509  0.022533381785989733  \\
            0.008814036430822295  0.022536930655673738  \\
            0.009915790984675084  0.0225410164095066  \\
            0.011017545538527868  0.02261141061710291  \\
            0.012119300092380656  0.022625739610801155  \\
            0.013221054646233443  0.02264323196977175  \\
            0.01432280920008623  0.02266041516418808  \\
            0.015424563753939018  0.02267616927455328  \\
            0.016526318307791804  0.02276530929874224  \\
            0.01762807286164459  0.02278566723648542  \\
            0.018729827415497377  0.02280539349561686  \\
            0.019831581969350167  0.022822042138170497  \\
            0.020933336523202953  0.022835129099133218  \\
            0.022035091077055736  0.022845024670999033  \\
            0.023136845630908526  0.02285246049738421  \\
            0.024238600184761313  0.022858285241870148  \\
            0.0253403547386141  0.022863322586573646  \\
            0.026442109292466886  0.02286827867916406  \\
            0.027543863846319676  0.02287368628249625  \\
            0.02864561840017246  0.022952679865681527  \\
            0.029747372954025245  0.022963364324187708  \\
            0.030849127507878035  0.02297426621456374  \\
            0.03195088206173082  0.022985442221665455  \\
            0.03305263661558361  0.02299707609617836  \\
            0.034154391169436395  0.023009487902785963  \\
            0.03525614572328918  0.023023002248379124  \\
            0.03635790027714197  0.023037674414515347  \\
            0.037459654830994754  0.02305302569415784  \\
            0.03856140938484754  0.023067756442694858  \\
            0.039663163938700334  0.023078673874791355  \\
            0.04076491849255312  0.023084833695697936  \\
            0.04186667304640591  0.023089289053653834  \\
            0.042968427600258687  0.023091871049717824  \\
            0.04407018215411147  0.023092425465789376  \\
            0.045171936707964266  0.023090814595223025  \\
            0.04627369126181705  0.02308691744534848  \\
            0.04737544581566984  0.02308062862446256  \\
            0.048477200369522626  0.023071856487283313  \\
            0.04957895492337542  0.023060521317577708  \\
            0.0506807094772282  0.023046554519368052  \\
            0.051782464031080985  0.02272978090383761  \\
            0.05288421858493377  0.022736068620428113  \\
            0.05398597313878656  0.022741774732869213  \\
            0.05508772769263935  0.022746884439074813  \\
            0.05618948224649214  0.022751392880231452  \\
            0.05729123680034492  0.022755304471800877  \\
            0.058392991354197704  0.02275863209910957  \\
            0.05949474590805049  0.022761396196767775  \\
            0.060596500461903284  0.022763623747514544  \\
            0.06169825501575607  0.02276534723853056  \\
            0.06280000956960886  0.022766603621087025  \\
            0.06390176412346164  0.022767433300524872  \\
            0.06500351867731442  0.02276787919857959  \\
            0.06610527323116722  0.022767985900735623  \\
            0.06720702778502  0.022767798910198967  \\
            0.06830878233887279  0.02276736401044186  \\
            0.06941053689272557  0.022766726734225508  \\
            0.07051229144657836  0.022765931935565132  \\
            0.07161404600043114  0.022765023448343594  \\
            0.07271580055428394  0.022764043821997464  \\
            0.07381755510813673  0.02276303411506106  \\
            0.07491930966198951  0.022762033738877765  \\
            0.0760210642158423  0.022761080330298108  \\
            0.07712281876969508  0.022760209651989983  \\
            0.07822457332354787  0.022759455504425413  \\
            0.07932632787740067  0.022758849647835273  \\
            0.08042808243125345  0.022758421728362378  \\
            0.08152983698510624  0.022758199206006736  \\
            0.08263159153895902  0.022758207282831823  \\
            0.08373334609281181  0.02275846882950579  \\
            0.0848351006466646  0.022759004309322428  \\
            0.08593685520051737  0.022759831702890154  \\
            0.08703860975437017  0.022760966423461308  \\
            0.08814036430822295  0.022762421231537853  \\
            0.08924211886207574  0.022764206138815794  \\
            0.09034387341592853  0.022766328300807187  \\
            0.09144562796978131  0.022768791897896438  \\
            0.0925473825236341  0.022771597994348497  \\
            0.09364913707748689  0.022774744377074167  \\
            0.09475089163133968  0.022778225365757375  \\
            0.09585264618519247  0.022782031587106143  \\
            0.09695440073904525  0.022786149714562727  \\
            0.09805615529289805  0.022790562159054922  \\
            0.09915790984675084  0.022795246707899237  \\
            0.1002596644006036  0.022845777586142032  \\
            0.1013614189544564  0.022848406897893643  \\
            0.10246317350830918  0.022851328729350053  \\
            0.10356492806216197  0.02285452224280881  \\
            0.10466668261601475  0.022857962267111252  \\
            0.10576843716986754  0.02286161878291014  \\
            0.10687019172372034  0.022865456441043847  \\
            0.10797194627757312  0.02286943413220351  \\
            0.10907370083142591  0.022873504642535004  \\
            0.1101754553852787  0.022877614419398777  \\
            0.11127720993913148  0.022881703486470235  \\
            0.11237896449298428  0.022885705540550562  \\
            0.11348071904683706  0.02288954826259199  \\
            0.11458247360068984  0.022893153866411716  \\
            0.11568422815454263  0.02289643990063589  \\
            0.11678598270839541  0.022899320298919716  \\
            0.1178877372622482  0.0229017066621637  \\
            0.11898949181610098  0.02290350972309055  \\
            0.12009124636995377  0.022904640935328652  \\
            0.12119300092380657  0.022905014105263268  \\
            0.12229475547765935  0.022904546975058504  \\
            0.12339651003151214  0.022903162665422767  \\
            0.12449826458536492  0.022900790893770333  \\
            0.12560001913921773  0.022897368905069312  \\
            0.1267017736930705  0.022892842070212426  \\
            0.1278035282469233  0.022887164151745614  \\
            0.12890528280077607  0.02288029724714828  \\
            0.13000703735462885  0.02287221147076698  \\
            0.13110879190848163  0.022862884436637053  \\
            0.13221054646233443  0.022852300634769552  \\
            0.1333123010161872  0.02284045080073764  \\
            0.13441405557004  0.022827331378744042  \\
            0.1355158101238928  0.02319391648626782  \\
            0.13661756467774558  0.0231844209593413  \\
            0.13771931923159836  0.023543812281399158  \\
            0.13882107378545114  0.02318911062244909  \\
            0.13992282833930392  0.02282011577352675  \\
            0.14102458289315672  0.022833890723333864  \\
            0.1421263374470095  0.022846392568451022  \\
            0.14322809200086228  0.022857622861530758  \\
            0.1443298465547151  0.022867588521769663  \\
            0.14543160110856787  0.02287630231261727  \\
            0.14653335566242065  0.022883783464439873  \\
            0.14763511021627346  0.02289005833690437  \\
            0.14873686477012624  0.022895161013381477  \\
            0.14983861932397902  0.022899133738942052  \\
            0.1509403738778318  0.022902027123862127  \\
            0.1520421284316846  0.02290390004751472  \\
            0.15314388298553738  0.02290481923361425  \\
            0.15424563753939016  0.02290485848385319  \\
            0.15534739209324297  0.022904097598839598  \\
            0.15644914664709575  0.022902621041464562  \\
            0.15755090120094853  0.022900516412774918  \\
            0.15865265575480134  0.022897872832788033  \\
            0.15975441030865412  0.02289477932192385  \\
            0.1608561648625069  0.022891323261374108  \\
            0.1619579194163597  0.02288758901289339  \\
            0.16305967397021248  0.02288365674718404  \\
            0.16416142852406526  0.02287960151098599  \\
            0.16526318307791804  0.02287549254833876  \\
            0.16636493763177085  0.022871392867524733  \\
            0.16746669218562363  0.02286735903314253  \\
            0.16856844673947638  0.02286344115377056  \\
            0.1696702012933292  0.022859683036741723  \\
            0.17077195584718197  0.02285612246674328  \\
            0.17187371040103475  0.022852791583883317  \\
            0.17297546495488753  0.022849717328201066  \\
            0.17407721950874033  0.022846921921611665  \\
            0.1751789740625931  0.02284442337308698  \\
            0.1762807286164459  0.022792717614508837  \\
            0.1773824831702987  0.022788168045849367  \\
            0.17848423772415148  0.022783902221249073  \\
            0.17958599227800426  0.022779939989500175  \\
            0.18068774683185707  0.02277629676616118  \\
            0.18178950138570985  0.02277298386349965  \\
            0.18289125593956262  0.022770008778214473  \\
            0.18399301049341543  0.02276737543039406  \\
            0.1850947650472682  0.022765084373425417  \\
            0.186196519601121  0.022763132966148274  \\
            0.18729827415497377  0.022761515525754557  \\
            0.18840002870882658  0.022760223455942277  \\
            0.18950178326267936  0.02275924535938181  \\
            0.19060353781653214  0.022758567139611132  \\
            0.19170529237038494  0.022758172091764922  \\
            0.19280704692423772  0.02275804098641429  \\
            0.1939088014780905  0.022758152148669043  \\
            0.1950105560319433  0.022758481536719073  \\
            0.1961123105857961  0.022759002812701726  \\
            0.19721406513964887  0.022759687417346543  \\
            0.19831581969350168  0.022760504642826158  \\
            0.19941757424735443  0.022761421700791814  \\
            0.2005193288012072  0.022762403804061512  \\
            0.20162108335506  0.022763414243882858  \\
            0.2027228379089128  0.02276441448613942  \\
            0.20382459246276557  0.022765364282073337  \\
            0.20492634701661835  0.022766221807754224  \\
            0.20602810157047116  0.022766943846488613  \\
            0.20712985612432394  0.022767486020496035  \\
            0.20823161067817672  0.02276780309204926  \\
            0.2093333652320295  0.022767849346848154  \\
            0.2104351197858823  0.022767579072719334  \\
            0.2115368743397351  0.022766947142518788  \\
            0.21263862889358787  0.022765909706381215  \\
            0.21374038344744067  0.022764424992454513  \\
            0.21484213800129345  0.02276245420505258  \\
            0.21594389255514623  0.022759962500682948  \\
            0.21704564710899904  0.02275692001843812  \\
            0.21814740166285182  0.022753302926562975  \\
            0.2192491562167046  0.022749094451181555  \\
            0.2203509107705574  0.022744285840070187  \\
            0.2214526653244102  0.022738877230799866  \\
            0.22255441987826297  0.022732878388106476  \\
            0.22365617443211575  0.02272630930435743  \\
            0.22475792898596855  0.02305352023600203  \\
            0.22585968353982133  0.023066150843495975  \\
            0.2269614380936741  0.023076184449240526  \\
            0.22806319264752692  0.023083695430504994  \\
            0.22916494720137967  0.023088769387156403  \\
            0.23026670175523245  0.023091504615371173  \\
            0.23136845630908526  0.023092014072563463  \\
            0.23247021086293804  0.02309042696255973  \\
            0.23357196541679082  0.023086889259248965  \\
            0.2346737199706436  0.0230815627206596  \\
            0.2357754745244964  0.023073546659831502  \\
            0.23687722907834918  0.023060435071742373  \\
            0.23797898363220196  0.02304512400192021  \\
            0.23908073818605477  0.023029997893866574  \\
            0.24018249273990755  0.023015871430716296  \\
            0.24128424729376033  0.023002931813267142  \\
            0.24238600184761314  0.022990950244706602  \\
            0.24348775640146592  0.022979577988876395  \\
            0.2445895109553187  0.02296855701381392  \\
            0.24569126550917147  0.022957774864398532  \\
            0.24679302006302428  0.022876339907885554  \\
            0.24789477461687706  0.022870583477606482  \\
            0.24899652917072984  0.02286547703668604  \\
            0.25009828372458265  0.02286058312972435  \\
            0.25120003827843546  0.022855277731876653  \\
            0.2523017928322882  0.022848786933424563  \\
            0.253403547386141  0.022840260816330595  \\
            0.25450530193999377  0.022828887294966776  \\
            0.2556070564938466  0.022814074394818577  \\
            0.2567088110476994  0.022795791055741126  \\
            0.25781056560155213  0.02277529825944428  \\
            0.25891232015540494  0.022756855815259325  \\
            0.2600140747092577  0.02266834246336077  \\
            0.2611158292631105  0.022651820243446204  \\
            0.26221758381696325  0.022634254747324944  \\
            0.26331933837081606  0.022617674765002795  \\
            0.26442109292466887  0.022543228017219735  \\
            0.2655228474785216  0.0225388759658583  \\
            0.2666246020323744  0.022535063849918435  \\
            0.26772635658622723  0.0225317823045093  \\
            0.26882811114008  0.02252901447948204  \\
            0.2699298656939328  0.022526741619659476  \\
            0.2710316202477856  0.022524946133770648  \\
            0.27213337480163835  0.022523613185420857  \\
            0.27323512935549116  0.02252273142649426  \\
            0.27433688390934396  0.022522293264074605  \\
            0.2754386384631967  0.022522294911205106  \\
            0.27654039301704947  0.022522736376040842  \\
            0.2776421475709023  0.02252362146206092  \\
            0.2787439021247551  0.022524957779386394  \\
            0.27984565667860783  0.02252675669529479  \\
            0.28094741123246064  0.022529033068046187  \\
            0.28204916578631345  0.022531804514626677  \\
            0.2831509203401662  0.0225350898224362  \\
            0.284252674894019  0.022538905882494444  \\
            0.2853544294478718  0.022543262113874366  \\
            0.28645618400172457  0.02261779490152133  \\
            0.2875579385555774  0.022634389199431008  \\
            0.2886596931094302  0.02265196480587566  \\
            0.28976144766328293  0.022668494493007955  \\
            0.29086320221713574  0.022756788740353924  \\
            0.29196495677098855  0.022775282942089834  \\
            0.2930667113248413  0.022795828165636044  \\
            0.2941684658786941  0.022814161399130516  \\
            0.2952702204325469  0.02282902016228435  \\
            0.29637197498639967  0.022840434970815327  \\
            0.2974737295402525  0.022848997780915228  \\
            0.2985754840941052  0.02285552091996113  \\
            0.29967723864795803  0.02286085461807836  \\
            0.30077899320181084  0.022865773028575193  \\
            0.3018807477556636  0.02287090025692195  \\
            0.3029825023095164  0.02287667365995881  \\
            0.3040842568633692  0.022957995659939383  \\
            0.30518601141722196  0.022968786712849687  \\
            0.30628776597107477  0.022979812107524377  \\
            0.3073895205249276  0.022991184384374672  \\
            0.3084912750787803  0.023003161883947396  \\
            0.30959302963263313  0.02301609375264776  \\
            0.31069478418648594  0.023030208991153343  \\
            0.3117965387403387  0.02304532004671061  \\
            0.3128982932941915  0.023060611422292238  \\
            0.3140000478480443  0.023073698673796033  \\
            0.31510180240189706  0.023081955798170767  \\
            0.31620355695574986  0.023087285299798987  \\
            0.31730531150960267  0.023090824234455975  \\
            0.3184070660634554  0.023092410708971136  \\
            0.31950882061730823  0.023091898637532266  \\
            0.32061057517116104  0.023089158738054987  \\
            0.3217123297250138  0.02308407802037713  \\
            0.3228140842788666  0.023076558215077675  \\
            0.3239158388327194  0.023066513831756997  \\
            0.32501759338657216  0.023053870707081515  \\
            0.32611934794042496  0.02272642642237038  \\
            0.3272211024942777  0.022732996133357598  \\
            0.3283228570481305  0.022738995541956113  \\
            0.32942461160198333  0.02274440466654795  \\
            0.3305263661558361  0.022749213750424283  \\
            0.3316281207096889  0.022753422664901622  \\
            0.3327298752635417  0.02275704016914189  \\
            0.33383162981739445  0.022760083044013685  \\
            0.33493338437124726  0.02276257512674079  \\
            0.3360351389251  0.02276454628357798  \\
            0.33713689347895276  0.022766031362731535  \\
            0.33823864803280557  0.022767069164709882  \\
            0.3393404025866584  0.02276770146550202  \\
            0.3404421571405111  0.022767972118606462  \\
            0.34154391169436393  0.022767926255310565  \\
            0.34264566624821674  0.02276760959197968  \\
            0.3437474208020695  0.02276706784573526  \\
            0.3448491753559223  0.022766346258478374  \\
            0.34595092990977505  0.02276548921114483  \\
            0.34705268446362786  0.022764539924639744  \\
            0.34815443901748067  0.0227635402254274  \\
            0.3492561935713334  0.022762530365410674  \\
            0.3503579481251862  0.02276154888360081  \\
            0.35145970267903903  0.0227606324903891  \\
            0.3525614572328918  0.02275981597823222  \\
            0.3536632117867446  0.022759132136485966  \\
            0.3547649663405974  0.02275861167776881  \\
            0.35586672089445015  0.02275828316347135  \\
            0.35696847544830296  0.022758172934867624  \\
            0.35807023000215576  0.022758305036134664  \\
            0.3591719845560085  0.022758701145057333  \\
            0.3602737391098613  0.022759380492383196  \\
            0.36137549366371413  0.022760359786494325  \\
            0.3624772482175669  0.022761653125010477  \\
            0.3635790027714197  0.02276327190748423  \\
            0.3646807573252725  0.022765224730216128  \\
            0.36578251187912525  0.022767517278509165  \\
            0.36688426643297806  0.022770152193129257  \\
            0.36798602098683086  0.02277312892111404  \\
            0.3690877755406836  0.022776443542250942  \\
            0.3701895300945364  0.022780088558977285  \\
            0.3712912846483892  0.022784052657956542  \\
            0.372393039202242  0.022788320422366625  \\
            0.3734947937560948  0.02279287200030219  \\
            0.37459654830994754  0.022844578080821992  \\
            0.37569830286380035  0.022847054529777054  \\
            0.37680005741765316  0.022849832455392575  \\
            0.3779018119715059  0.022852892991616157  \\
            0.3790035665253587  0.022856213197915677  \\
            0.3801053210792115  0.022859765542491833  \\
            0.3812070756330643  0.022863517398967678  \\
            0.3823088301869171  0.022867430586852165  \\
            0.3834105847407699  0.02287146098158213  \\
            0.38451233929462264  0.022875558220839455  \\
            0.38561409384847545  0.022879665540029764  \\
            0.38671584840232825  0.022883719774557366  \\
            0.387817602956181  0.022887651560019307  \\
            0.3889193575100338  0.022891385759327428  \\
            0.3900211120638866  0.022894842138153227  \\
            0.3911228666177394  0.02289793629087399  \\
            0.3922246211715922  0.02290058081223652  \\
            0.393326375725445  0.0229026866755249  \\
            0.39442813027929774  0.022904164771426337  \\
            0.39552988483315055  0.022904927527144552  \\
            0.39663163938700335  0.022904890524487936  \\
            0.39773339394085605  0.02290397402736851  \\
            0.39883514849470886  0.022902104319867226  \\
            0.3999369030485616  0.022899214785901126  \\
            0.4010386576024144  0.022895246680908213  \\
            0.4021404121562672  0.022890149558304956  \\
            0.40324216671012  0.022883881373352937  \\
            0.4043439212639728  0.022876408285769596  \\
            0.4054456758178256  0.02286770423395861  \\
            0.40654743037167834  0.022857750349765914  \\
            0.40764918492553115  0.022846534322458933  \\
            0.40875093947938396  0.02283404979811589  \\
            0.4098526940332367  0.02282029593811797  \\
            0.4109544485870895  0.023189347821229632  \\
            0.4120562031409423  0.023543791803665018  \\
            0.4131579576947951  0.02318416652591496  \\
            0.4142597122486479  0.02319369505986814  \\
            0.4153614668025007  0.022827162282316106  \\
            0.41646322135635344  0.02284030080891486  \\
            0.41756497591020625  0.022852166356644533  \\
            0.418666730464059  0.0228627631167361  \\
            0.4197684850179118  0.02287210085831907  \\
            0.4208702395717646  0.022880195495413997  \\
            0.42197199412561737  0.0228870697429361  \\
            0.4230737486794702  0.022892753754410966  \\
            0.424175503233323  0.02289728565368415  \\
            0.42527725778717573  0.022900711859750462  \\
            0.42637901234102854  0.022903087149650284  \\
            0.42748076689488135  0.02290447439987218  \\
            0.4285825214487341  0.02290494398968486  \\
            0.4296842760025869  0.022904572871601768  \\
            0.4307860305564397  0.022903443359451926  \\
            0.43188778511029247  0.022901641682318636  \\
            0.4329895396641453  0.022899256407035835  \\
            0.4340912942179981  0.022896376801761247  \\
            0.43519304877185083  0.02289309125215095  \\
            0.43629480332570364  0.022889485789772197  \\
            0.43739655787955645  0.022885642812812135  \\
            0.4384983124334092  0.02288164003056699  \\
            0.439600066987262  0.022877549657830083  \\
            0.4407018215411148  0.022873437860238606  \\
            0.44180357609496756  0.02286936443747049  \\
            0.4429053306488204  0.022865382716283456  \\
            0.4440070852026732  0.02286153962659867  \\
            0.44510883975652593  0.022857875922020526  \\
            0.44621059431037874  0.02285442651438595  \\
            0.4473123488642315  0.022851220883918607  \\
            0.4484141034180843  0.02284828355028953  \\
            0.4495158579719371  0.02284563456541934  \\
            0.45061761252578986  0.02279509129228612  \\
            0.45171936707964266  0.022790408786633548  \\
            0.45282112163349547  0.022785998317058336  \\
            0.4539228761873482  0.022781882093163685  \\
            0.45502463074120103  0.02277807770226886  \\
            0.45612638529505384  0.02277459846972818  \\
            0.4572281398489066  0.022771453767071157  \\
            0.45832989440275934  0.022768649275449353  \\
            0.45943164895661215  0.022766187207418687  \\
            0.4605334035104649  0.022764066499093613  \\
            0.4616351580643177  0.022762282970645895  \\
            0.4627369126181705  0.022760829467607626  \\
            0.46383866717202327  0.02275969597981101  \\
            0.4649404217258761  0.022758869748826432  \\
            0.4660421762797288  0.022758335362875578  \\
            0.46714393083358163  0.022758074844502665  \\
            0.46824568538743444  0.02275806773221115  \\
            0.4693474399412872  0.022758291157565464  \\
            0.47044919449514  0.022758719922218305  \\
            0.4715509490489928  0.022759326568936925  \\
            0.47265270360284556  0.022760081454031498  \\
            0.47375445815669837  0.02276095282074683  \\
            0.4748562127105512  0.022761906872242364  \\
            0.4759579672644039  0.02276290784898857  \\
            0.47705972181825673  0.022763918116856786  \\
            0.47816147637210954  0.02276489826896281  \\
            0.4792632309259623  0.022765807249403806  \\
            0.4803649854798151  0.022766602512399436  \\
            0.4814667400336679  0.022767240227834187  \\
            0.48256849458752066  0.022767675545378882  \\
            0.48367024914137347  0.022767862934727767  \\
            0.4847720036952263  0.022767756617857165  \\
            0.485873758249079  0.02276731109422887  \\
            0.48697551280293183  0.0227664817822745  \\
            0.48807726735678464  0.02276522576461096  \\
            0.4891790219106374  0.022763502640191283  \\
            0.4902807764644902  0.022761275462843816  \\
            0.49138253101834295  0.02275851174990042  \\
            0.49248428557219576  0.022755184524361805  \\
            0.49358604012604856  0.02275127335769574  \\
            0.4946877946799013  0.02274676537174066  \\
            0.4957895492337541  0.02274165615808227  \\
            0.49689130378760693  0.022735950584615132  \\
            0.4979930583414597  0.022729663463703687  \\
            0.4990948128953125  0.023046210855619496  \\
            0.5001965674491653  0.023060164388987846  \\
            0.501298322003018  0.023071487876362297  \\
            0.5024000765568709  0.023080250193640107  \\
            0.5035018311107237  0.023086531213692918  \\
            0.5046035856645764  0.023090422648635717  \\
            0.5057053402184292  0.02309202988376888  \\
            0.506807094772282  0.023091473854472883  \\
            0.5079088493261348  0.02308889217213352  \\
            0.5090106038799875  0.023084438929311456  \\
            0.5101123584338404  0.023078282882308152  \\
            0.5112141129876931  0.023067591799235555  \\
            0.5123158675415459  0.023052838869313124  \\
            0.5134176220953988  0.023037470317879515  \\
            0.5145193766492515  0.023022785097607832  \\
            0.5156211312031043  0.02300926126913367  \\
            0.5167228857569571  0.02299684350573064  \\
            0.5178246403108099  0.022985207557149275  \\
            0.5189263948646626  0.022974033744605247  \\
            0.5200281494185154  0.022963138519642174  \\
            0.5211299039723682  0.022952465172409495  \\
            0.522231658526221  0.022873360527882466  \\
            0.5233334130800738  0.022867971826826285  \\
            0.5244351676339265  0.022863038381575464  \\
            0.5255369221877794  0.022858027417276973  \\
            0.5266386767416321  0.022852232955359007  \\
            0.5277404312954849  0.02284483160704954  \\
            0.5288421858493377  0.02283497500631668  \\
            0.5299439404031905  0.02282193164821084  \\
            0.5310456949570432  0.02280533099914659  \\
            0.5321474495108961  0.02278565617186002  \\
            0.5332492040647488  0.02276535087654462  \\
            0.5343509586186016  0.02267601417935072  \\
            0.5354527131724545  0.02266026660937356  \\
            0.5365544677263072  0.022643092052052224  \\
            0.53765622228016  0.02262561165993905  \\
            0.5387579768340128  0.022611299976115006  \\
            0.5398597313878656  0.02254098443635172  \\
            0.5409614859417183  0.022536902736827666  \\
            0.5420632404955712  0.022533357714602194  \\
            0.543164995049424  0.02253033534959996  \\
            0.5442667496032767  0.022527817329595433  \\
            0.5453685041571296  0.02252578520841771  \\
            0.5464702587109823  0.02252422263810186  \\
            0.5475720132648351  0.022523116470944494  \\
            0.5486737678186879  0.022522457221445545  \\
            0.5497755223725407  0.02252223919972479  \\
        }
        ;
    \addlegendentry {$(20, 10, 0.1) $}
    \addplot[color={rgb,1:red,1.0;green,0.0;blue,0.0}, name path={fc4c513b-684e-4a29-8eaa-0acd368e3f5d}, draw opacity={1.0}, line width={1}, dashed, forget plot]
        table[row sep={\\}]
        {
            \\
            0.0  0.18069911663312657  \\
            0.001101754553852787  0.1806977799152342  \\
            0.002203509107705574  0.18069375841459812  \\
            0.0033052636615583607  0.180687053160145  \\
            0.004407018215411148  0.180677665744694  \\
            0.005508772769263934  0.18066559810574387  \\
            0.0066105273231167215  0.18065085220545207  \\
            0.007712281876969509  0.18063342950609484  \\
            0.008814036430822295  0.18061332994727958  \\
            0.009915790984675084  0.1805905498040117  \\
            0.011017545538527868  0.18093236245361455  \\
            0.012119300092380656  0.18094037254314288  \\
            0.013221054646233443  0.18094161103192286  \\
            0.01432280920008623  0.18093541252963147  \\
            0.015424563753939018  0.18092249509114536  \\
            0.016526318307791804  0.18125364689791015  \\
            0.01762807286164459  0.18125806503542877  \\
            0.018729827415497377  0.18125408523844885  \\
            0.019831581969350167  0.18124266445031376  \\
            0.020933336523202953  0.1812256269701158  \\
            0.022035091077055736  0.18120512481282589  \\
            0.023136845630908526  0.18118337653389516  \\
            0.024238600184761313  0.18116247809500627  \\
            0.0253403547386141  0.1811442438883092  \\
            0.026442109292466886  0.1811300860885158  \\
            0.027543863846319676  0.18112093997565076  \\
            0.02864561840017246  0.18220334681612507  \\
            0.029747372954025245  0.18221687146687104  \\
            0.030849127507878035  0.1822338718791959  \\
            0.03195088206173082  0.18225360886462103  \\
            0.03305263661558361  0.18227512092301446  \\
            0.034154391169436395  0.18229730412254372  \\
            0.03525614572328918  0.1823189329300067  \\
            0.03635790027714197  0.18233856085846048  \\
            0.037459654830994754  0.18235406816358044  \\
            0.03856140938484754  0.18236058942447672  \\
            0.039663163938700334  0.18234089034317957  \\
            0.04076491849255312  0.18230328568060344  \\
            0.04186667304640591  0.18226442032519607  \\
            0.042968427600258687  0.18222410476749282  \\
            0.04407018215411147  0.18218215628161288  \\
            0.045171936707964266  0.1821384037171641  \\
            0.04627369126181705  0.18209269032009692  \\
            0.04737544581566984  0.18204487452517146  \\
            0.048477200369522626  0.18199482904225497  \\
            0.04957895492337542  0.18194243890157438  \\
            0.0506807094772282  0.18188759944098107  \\
            0.051782464031080985  0.17998069474335177  \\
            0.05288421858493377  0.17994280621276382  \\
            0.05398597313878656  0.17990488869385735  \\
            0.05508772769263935  0.17986692814303293  \\
            0.05618948224649214  0.179828917598276  \\
            0.05729123680034492  0.17979085698616354  \\
            0.058392991354197704  0.17975275274159305  \\
            0.05949474590805049  0.17971461729448482  \\
            0.060596500461903284  0.1796764684740737  \\
            0.06169825501575607  0.17963832887478504  \\
            0.06280000956960886  0.17960022522198255  \\
            0.06390176412346164  0.17956218776501856  \\
            0.06500351867731442  0.17952424971746497  \\
            0.06610527323116722  0.17948644675245465  \\
            0.06720702778502  0.17944881655791473  \\
            0.06830878233887279  0.17941139844284293  \\
            0.06941053689272557  0.17937423298672153  \\
            0.07051229144657836  0.17933736172023915  \\
            0.07161404600043114  0.17930082682200488  \\
            0.07271580055428394  0.17926467082301076  \\
            0.07381755510813673  0.17922893630592832  \\
            0.07491930966198951  0.17919366559545227  \\
            0.0760210642158423  0.17915890043536534  \\
            0.07712281876969508  0.179124681653842  \\
            0.07822457332354787  0.1790910488183051  \\
            0.07932632787740067  0.17905803988585625  \\
            0.08042808243125345  0.1790256908551305  \\
            0.08152983698510624  0.17899403542593706  \\
            0.08263159153895902  0.1789631046725595  \\
            0.08373334609281181  0.17893292673671676  \\
            0.0848351006466646  0.17890352654210692  \\
            0.08593685520051737  0.1788749255354098  \\
            0.08703860975437017  0.17884714145028108  \\
            0.08814036430822295  0.1788201880948251  \\
            0.08924211886207574  0.1787940751570766  \\
            0.09034387341592853  0.17876880802292375  \\
            0.09144562796978131  0.17874438759881772  \\
            0.0925473825236341  0.17872081012807434  \\
            0.09364913707748689  0.17869806699227578  \\
            0.09475089163133968  0.17867614448223862  \\
            0.09585264618519247  0.17865502352694235  \\
            0.09695440073904525  0.17863467936423072  \\
            0.09805615529289805  0.17861508113792243  \\
            0.09915790984675084  0.17859619140402097  \\
            0.1002596644006036  0.1785472594580839  \\
            0.1013614189544564  0.1785289699911921  \\
            0.10246317350830918  0.17851144721958037  \\
            0.10356492806216197  0.17849468258803694  \\
            0.10466668261601475  0.17847866337577056  \\
            0.10576843716986754  0.178463371568461  \\
            0.10687019172372034  0.1784487827883928  \\
            0.10797194627757312  0.17843486530290362  \\
            0.10907370083142591  0.17842157914437576  \\
            0.1101754553852787  0.17840887537551764  \\
            0.11127720993913148  0.17839669554426685  \\
            0.11237896449298428  0.178384971372402  \\
            0.11348071904683706  0.17837362472523322  \\
            0.11458247360068984  0.17836256790075575  \\
            0.11568422815454263  0.17835170426740946  \\
            0.11678598270839541  0.17834092926063447  \\
            0.1178877372622482  0.17833013172211307  \\
            0.11898949181610098  0.1783191955366565  \\
            0.12009124636995377  0.17830800149339654  \\
            0.12119300092380657  0.17829642926399222  \\
            0.12229475547765935  0.1782843593757873  \\
            0.12339651003151214  0.1782716750425319  \\
            0.12449826458536492  0.17825826372097772  \\
            0.12560001913921773  0.17824401827732594  \\
            0.1267017736930705  0.1782288376801967  \\
            0.1278035282469233  0.1782126271711573  \\
            0.12890528280077607  0.17819529791014624  \\
            0.13000703735462885  0.17817676613229144  \\
            0.13110879190848163  0.17815695188853908  \\
            0.13221054646233443  0.17813577746855638  \\
            0.1333123010161872  0.1781131656277593  \\
            0.13441405557004  0.1780890377494124  \\
            0.1355158101238928  0.18058527079212935  \\
            0.13661756467774558  0.18056956148726547  \\
            0.13771931923159836  0.183020632611722  \\
            0.13882107378545114  0.18057842476779612  \\
            0.13992282833930392  0.17807370017165539  \\
            0.14102458289315672  0.17809875081306797  \\
            0.1421263374470095  0.1781222122718822  \\
            0.14322809200086228  0.17814416903458294  \\
            0.1443298465547151  0.17816470294125233  \\
            0.14543160110856787  0.17818389496817563  \\
            0.14653335566242065  0.17820182704153545  \\
            0.14763511021627346  0.17821858373472232  \\
            0.14873686477012624  0.17823425371646465  \\
            0.14983861932397902  0.1782489308370304  \\
            0.1509403738778318  0.178262714763146  \\
            0.1520421284316846  0.1782757111024639  \\
            0.15314388298553738  0.17828803099959534  \\
            0.15424563753939016  0.1782997902272865  \\
            0.15534739209324297  0.1783111078371609  \\
            0.15644914664709575  0.17832210447190872  \\
            0.15755090120094853  0.17833290046235684  \\
            0.15865265575480134  0.17834361384446057  \\
            0.15975441030865412  0.1783543584299819  \\
            0.1608561648625069  0.1783652420435493  \\
            0.1619579194163597  0.1783763650189967  \\
            0.16305967397021248  0.17838781901458955  \\
            0.16416142852406526  0.17839968617708996  \\
            0.16526318307791804  0.17841203865756228  \\
            0.16636493763177085  0.1784249384602363  \\
            0.16746669218562363  0.1784384375893514  \\
            0.16856844673947638  0.1784525784514875  \\
            0.1696702012933292  0.17846739446881602  \\
            0.17077195584718197  0.1784829108580147  \\
            0.17187371040103475  0.17849914553875004  \\
            0.17297546495488753  0.1785161101383926  \\
            0.17407721950874033  0.17853381106710636  \\
            0.1751789740625931  0.17855225064655092  \\
            0.1762807286164459  0.1786010082116877  \\
            0.1773824831702987  0.17862025808864077  \\
            0.17848423772415148  0.17864023594619943  \\
            0.17958599227800426  0.1786609758128653  \\
            0.18068774683185707  0.178682505391905  \\
            0.18178950138570985  0.1787048465069601  \\
            0.18289125593956262  0.17872801546656586  \\
            0.18399301049341543  0.17875202336133694  \\
            0.1850947650472682  0.17877687631286632  \\
            0.186196519601121  0.17880257568527363  \\
            0.18729827415497377  0.17882911827533965  \\
            0.18840002870882658  0.17885649649062701  \\
            0.18950178326267936  0.17888469852873118  \\
            0.19060353781653214  0.17891370856448174  \\
            0.19170529237038494  0.1789435069524926  \\
            0.19280704692423772  0.1789740704508472  \\
            0.1939088014780905  0.17900537246830547  \\
            0.1950105560319433  0.17903738333719216  \\
            0.1961123105857961  0.1790700706089264  \\
            0.19721406513964887  0.1791033993718171  \\
            0.19831581969350168  0.17913733258553347  \\
            0.19941757424735443  0.17917183142534077  \\
            0.2005193288012072  0.17920685563220057  \\
            0.20162108335506  0.1792423638607821  \\
            0.2027228379089128  0.17927831401973526  \\
            0.20382459246276557  0.17931466360092047  \\
            0.20492634701661835  0.17935136999538495  \\
            0.20602810157047116  0.1793883907987949  \\
            0.20712985612432394  0.1794256841075257  \\
            0.20823161067817672  0.17946320881797656  \\
            0.2093333652320295  0.17950092493625666  \\
            0.2104351197858823  0.17953879391138836  \\
            0.2115368743397351  0.17957677900633534  \\
            0.21263862889358787  0.17961484571645098  \\
            0.21374038344744067  0.1796529622440686  \\
            0.21484213800129345  0.1796911000310689  \\
            0.21594389255514623  0.17972923434357338  \\
            0.21704564710899904  0.17976734489529297  \\
            0.21814740166285182  0.17980541648484108  \\
            0.2192491562167046  0.17984343961381705  \\
            0.2203509107705574  0.17988141104562386  \\
            0.2214526653244102  0.17991933425474313  \\
            0.22255441987826297  0.17995721971588202  \\
            0.22365617443211575  0.17999508497501304  \\
            0.22475792898596855  0.18191101255821973  \\
            0.22585968353982133  0.1819645149990041  \\
            0.2269614380936741  0.18201563052805184  \\
            0.22806319264752692  0.18206446806406926  \\
            0.22916494720137967  0.1821111478083933  \\
            0.23026670175523245  0.18215580336936374  \\
            0.23136845630908526  0.18219858329831454  \\
            0.23247021086293804  0.18223965121352387  \\
            0.23357196541679082  0.1822791840107486  \\
            0.2346737199706436  0.18231736802704462  \\
            0.2357754745244964  0.18235117785024174  \\
            0.23687722907834918  0.18235450373221165  \\
            0.23797898363220196  0.1823424013098921  \\
            0.23908073818605477  0.1823244058799537  \\
            0.24018249273990755  0.18230350494516875  \\
            0.24128424729376033  0.1822813812741595  \\
            0.24238600184761314  0.1822593465040792  \\
            0.24348775640146592  0.18223856304584696  \\
            0.2445895109553187  0.1822200706072351  \\
            0.24569126550917147  0.18220473119537467  \\
            0.24679302006302428  0.1811130573216426  \\
            0.24789477461687706  0.18111949049823184  \\
            0.24899652917072984  0.18113122274198976  \\
            0.25009828372458265  0.1811475686141682  \\
            0.25120003827843546  0.1811673459906127  \\
            0.2523017928322882  0.18118892759588226  \\
            0.253403547386141  0.1812103389923882  \\
            0.25450530193999377  0.1812293998238172  \\
            0.2556070564938466  0.18124389727405182  \\
            0.2567088110476994  0.18125179889394735  \\
            0.25781056560155213  0.18125160882476585  \\
            0.25891232015540494  0.18124324112015844  \\
            0.2600140747092577  0.18092805538281861  \\
            0.2611158292631105  0.18093774089072842  \\
            0.26221758381696325  0.18094025580617978  \\
            0.26331933837081606  0.1809353900011063  \\
            0.26442109292466887  0.18057805477154104  \\
            0.2655228474785216  0.1806021863485334  \\
            0.2666246020323744  0.18062363379479385  \\
            0.26772635658622723  0.18064240398024844  \\
            0.26882811114008  0.18065849829035352  \\
            0.2699298656939328  0.18067191573297747  \\
            0.2710316202477856  0.18068265440393783  \\
            0.27213337480163835  0.18069071220378988  \\
            0.27323512935549116  0.18069608723572902  \\
            0.27433688390934396  0.18069877806734788  \\
            0.2754386384631967  0.1806987839098953  \\
            0.27654039301704947  0.1806961047137331  \\
            0.2776421475709023  0.1806907411680786  \\
            0.2787439021247551  0.180682694606497  \\
            0.27984565667860783  0.18067196682642223  \\
            0.28094741123246064  0.18065855982661302  \\
            0.28204916578631345  0.18064247540636608  \\
            0.2831509203401662  0.1806237144439454  \\
            0.284252674894019  0.1806022754237335  \\
            0.2853544294478718  0.18057815131674276  \\
            0.28645618400172457  0.1809370658697796  \\
            0.2875579385555774  0.18094193046515522  \\
            0.2886596931094302  0.18093941364599564  \\
            0.28976144766328293  0.18092972559838286  \\
            0.29086320221713574  0.18124854642469582  \\
            0.29196495677098855  0.1812569032747157  \\
            0.2930667113248413  0.1812570954532479  \\
            0.2941684658786941  0.18124920271837958  \\
            0.2952702204325469  0.18123471732150345  \\
            0.29637197498639967  0.1812156695063063  \\
            0.2974737295402525  0.1811942705510521  \\
            0.2985754840941052  0.1811726994539406  \\
            0.29967723864795803  0.18115292917150888  \\
            0.30077899320181084  0.18113658524807416  \\
            0.3018807477556636  0.1811248478363797  \\
            0.3029825023095164  0.18111840026143539  \\
            0.3040842568633692  0.18220964003292067  \\
            0.30518601141722196  0.18222497699461734  \\
            0.30628776597107477  0.1822434543266352  \\
            0.3073895205249276  0.18226420928602585  \\
            0.3084912750787803  0.1822862020455642  \\
            0.30959302963263313  0.1823082705338068  \\
            0.31069478418648594  0.1823291036456231  \\
            0.3117965387403387  0.1823470192016389  \\
            0.3128982932941915  0.1823590303835546  \\
            0.3140000478480443  0.18235560159071  \\
            0.31510180240189706  0.18232223365473305  \\
            0.31620355695574986  0.1822840225246488  \\
            0.31730531150960267  0.1822444554661795  \\
            0.3184070660634554  0.18220334565206428  \\
            0.31950882061730823  0.18216051566255745  \\
            0.32061057517116104  0.18211580132287572  \\
            0.3217123297250138  0.18206905349264726  \\
            0.3228140842788666  0.18202013794298105  \\
            0.3239158388327194  0.18196893381855447  \\
            0.32501759338657216  0.1819153315105965  \\
            0.32611934794042496  0.1799996344573617  \\
            0.3272211024942777  0.17996175294663677  \\
            0.3283228570481305  0.17992385212130324  \\
            0.32942461160198333  0.17988591439732418  \\
            0.3305263661558361  0.17984792928229398  \\
            0.3316281207096889  0.17980989329553754  \\
            0.3327298752635417  0.17977180967489706  \\
            0.33383162981739445  0.17973368792327346  \\
            0.33493338437124726  0.17969554324802833  \\
            0.3360351389251  0.17965739594098973  \\
            0.33713689347895276  0.17961927074028688  \\
            0.33823864803280557  0.17958119620655152  \\
            0.3393404025866584  0.17954320413774308  \\
            0.3404421571405111  0.1795053290365007  \\
            0.34154391169436393  0.17946760763579034  \\
            0.34264566624821674  0.17943007848022602  \\
            0.3437474208020695  0.1793927815555033  \\
            0.3448491753559223  0.179355757955381  \\
            0.34595092990977505  0.17931904957174494  \\
            0.34705268446362786  0.17928269879642017  \\
            0.34815443901748067  0.17924674822425513  \\
            0.3492561935713334  0.17921124034840608  \\
            0.3503579481251862  0.17917621724439842  \\
            0.35145970267903903  0.17914172024069785  \\
            0.3525614572328918  0.17910778957870327  \\
            0.3536632117867446  0.1790744640643816  \\
            0.3547649663405974  0.17904178071880644  \\
            0.35586672089445015  0.17900977443294527  \\
            0.35696847544830296  0.17897847763383548  \\
            0.35807023000215576  0.17894791996618867  \\
            0.3591719845560085  0.17891812799700968  \\
            0.3602737391098613  0.1788891249423304  \\
            0.36137549366371413  0.17886093042029108  \\
            0.3624772482175669  0.17883356022706381  \\
            0.3635790027714197  0.17880702613421431  \\
            0.3646807573252725  0.1787813357008085  \\
            0.36578251187912525  0.17875649209540956  \\
            0.36688426643297806  0.17873249391651355  \\
            0.36798602098683086  0.1787093350029453  \\
            0.3690877755406836  0.17868700422178282  \\
            0.3701895300945364  0.17866548522019654  \\
            0.3712912846483892  0.17864475612691716  \\
            0.372393039202242  0.1786247891884638  \\
            0.3734947937560948  0.17860555032266257  \\
            0.37459654830994754  0.17855669352265607  \\
            0.37569830286380035  0.17853801848420436  \\
            0.37680005741765316  0.17852011321906872  \\
            0.3779018119715059  0.17850297082035704  \\
            0.3790035665253587  0.17848658079348162  \\
            0.3801053210792115  0.178470927908878  \\
            0.3812070756330643  0.17845599110001578  \\
            0.3823088301869171  0.17844174242941263  \\
            0.3834105847407699  0.17842814614870092  \\
            0.38451233929462264  0.1784151578865396  \\
            0.38561409384847545  0.17840272400296142  \\
            0.38671584840232825  0.17839078115510462  \\
            0.387817602956181  0.17837925612119063  \\
            0.3889193575100338  0.17836806592527907  \\
            0.3900211120638866  0.17835711829863823  \\
            0.3911228666177394  0.17834631249648994  \\
            0.3922246211715922  0.17833554046990524  \\
            0.393326375725445  0.1783246883613001  \\
            0.39442813027929774  0.17831363826510935  \\
            0.39552988483315055  0.17830227016170355  \\
            0.39663163938700335  0.17829046390932662  \\
            0.39773339394085605  0.17827810116402074  \\
            0.39883514849470886  0.17826506708855927  \\
            0.3999369030485616  0.1782512517264142  \\
            0.4010386576024144  0.17823655094086252  \\
            0.4021404121562672  0.17822086684830996  \\
            0.40324216671012  0.17820410772603074  \\
            0.4043439212639728  0.1781861874048306  \\
            0.4054456758178256  0.17816702420788286  \\
            0.40654743037167834  0.17814653951772488  \\
            0.40764918492553115  0.17812465608587302  \\
            0.40875093947938396  0.17810129620941284  \\
            0.4098526940332367  0.17807637991955483  \\
            0.4109544485870895  0.18057767329289612  \\
            0.4120562031409423  0.1830204928160024  \\
            0.4131579576947951  0.180570195725607  \\
            0.4142597122486479  0.18058612892278086  \\
            0.4153614668025007  0.1780864295693821  \\
            0.41646322135635344  0.17811067482798115  \\
            0.41756497591020625  0.17813337358351589  \\
            0.418666730464059  0.1781546087851599  \\
            0.4197684850179118  0.17817446161237924  \\
            0.4208702395717646  0.17819301329369017  \\
            0.42197199412561737  0.1782103468796932  \\
            0.4230737486794702  0.1782265488301906  \\
            0.424175503233323  0.17824171029561048  \\
            0.42527725778717573  0.17825592798605028  \\
            0.42637901234102854  0.17826930455610196  \\
            0.42748076689488135  0.17828194846520204  \\
            0.4285825214487341  0.17829397331494964  \\
            0.4296842760025869  0.17830549671004933  \\
            0.4307860305564397  0.17831663872537318  \\
            0.43188778511029247  0.17832752009344283  \\
            0.4329895396641453  0.17833826024553537  \\
            0.4340912942179981  0.17834897533831384  \\
            0.43519304877185083  0.17835977639630135  \\
            0.43629480332570364  0.1783707676675586  \\
            0.43739655787955645  0.17838204527486645  \\
            0.4384983124334092  0.1783936962029964  \\
            0.439600066987262  0.17840579763986478  \\
            0.4407018215411148  0.1784184166625315  \\
            0.44180357609496756  0.1784316102399309  \\
            0.4429053306488204  0.1784454255127662  \\
            0.4440070852026732  0.17845990030754913  \\
            0.44510883975652593  0.17847506383725625  \\
            0.44621059431037874  0.17849093755006581  \\
            0.4473123488642315  0.17850753608881628  \\
            0.4484141034180843  0.17852486833417974  \\
            0.4495158579719371  0.17854293850842543  \\
            0.45061761252578986  0.17859164377074954  \\
            0.45171936707964266  0.17861054454066408  \\
            0.45282112163349547  0.17863015373896493  \\
            0.4539228761873482  0.1786505087543683  \\
            0.45502463074120103  0.17867164039116973  \\
            0.45612638529505384  0.17869357336265762  \\
            0.4572281398489066  0.1787163266938123  \\
            0.45832989440275934  0.17873991405055467  \\
            0.45943164895661215  0.17876434401050625  \\
            0.4605334035104649  0.17878962029178658  \\
            0.4616351580643177  0.1788157419518766  \\
            0.4627369126181705  0.17884270357096138  \\
            0.46383866717202327  0.17887049542893008  \\
            0.4649404217258761  0.17889910368779044  \\
            0.4660421762797288  0.1789285105855666  \\
            0.46714393083358163  0.17895869464938208  \\
            0.46824568538743444  0.17898963093012266  \\
            0.4693474399412872  0.17902129126280794  \\
            0.47044919449514  0.17905364455078993  \\
            0.4715509490489928  0.17908665707315033  \\
            0.47265270360284556  0.17912029281122535  \\
            0.47375445815669837  0.1791545137888402  \\
            0.4748562127105512  0.17918928042062213  \\
            0.4759579672644039  0.17922455186101224  \\
            0.47705972181825673  0.17926028634981112  \\
            0.47816147637210954  0.179296441546855  \\
            0.4792632309259623  0.17933297485506128  \\
            0.4803649854798151  0.1793698437301966  \\
            0.4814667400336679  0.17940700598115417  \\
            0.48256849458752066  0.1794444200669949  \\
            0.48367024914137347  0.17948204539855123  \\
            0.4847720036952263  0.17951984265993087  \\
            0.485873758249079  0.17955777415803845  \\
            0.48697551280293183  0.17959580421621563  \\
            0.48807726735678464  0.17963389962050352  \\
            0.4891790219106374  0.17967203012285493  \\
            0.4902807764644902  0.17971016900124798  \\
            0.49138253101834295  0.17974829366624548  \\
            0.49248428557219576  0.1797863862945955  \\
            0.49358604012604856  0.17982443446189708  \\
            0.4946877946799013  0.17986243173628383  \\
            0.4957895492337541  0.17990037818946591  \\
            0.49689130378760693  0.1799382807720694  \\
            0.4979930583414597  0.17997615350128493  \\
            0.4990948128953125  0.18188333487647204  \\
            0.5001965674491653  0.18193806856187045  \\
            0.501298322003018  0.18199036456016174  \\
            0.5024000765568709  0.18204032682241653  \\
            0.5035018311107237  0.18208806964704494  \\
            0.5046035856645764  0.1821337196874035  \\
            0.5057053402184292  0.18217741790394548  \\
            0.506807094772282  0.18221932047761802  \\
            0.5079088493261348  0.18225959801874045  \\
            0.5090106038799875  0.1822984327455543  \\
            0.5101123584338404  0.1823360136955358  \\
            0.5112141129876931  0.1823561125885399  \\
            0.5123158675415459  0.18234949451025784  \\
            0.5134176220953988  0.18233390156461818  \\
            0.5145193766492515  0.1823141997078133  \\
            0.5156211312031043  0.18229250932769997  \\
            0.5167228857569571  0.18227027747090843  \\
            0.5178246403108099  0.18224873013836787  \\
            0.5189263948646626  0.18222897140216646  \\
            0.5200281494185154  0.18221196234277512  \\
            0.5211299039723682  0.18219844110449085  \\
            0.522231658526221  0.1811155885490647  \\
            0.5233334130800738  0.18112472514661065  \\
            0.5244351676339265  0.1811388815951415  \\
            0.5255369221877794  0.18115712055605918  \\
            0.5266386767416321  0.1811780279955711  \\
            0.5277404312954849  0.1811997879217646  \\
            0.5288421858493377  0.18122030297958855  \\
            0.5299439404031905  0.18123735321467777  \\
            0.5310456949570432  0.18124878481438902  \\
            0.5321474495108961  0.18125277068557338  \\
            0.5332492040647488  0.18124834917685922  \\
            0.5343509586186016  0.1809208263037117  \\
            0.5354527131724545  0.1809337409787818  \\
            0.5365544677263072  0.18093993723406088  \\
            0.53765622228016  0.18093869720470085  \\
            0.5387579768340128  0.18093068611210375  \\
            0.5398597313878656  0.18059045686280426  \\
            0.5409614859417183  0.18061324497636735  \\
            0.5420632404955712  0.1806333533774384  \\
            0.543164995049424  0.180650785648365  \\
            0.5442667496032767  0.18066554172850385  \\
            0.5453685041571296  0.18067762004716392  \\
            0.5464702587109823  0.18068701853950023  \\
            0.5475720132648351  0.18069373516868387  \\
            0.5486737678186879  0.18069776824255876  \\
            0.5497755223725407  0.18069911663323196  \\
        }
        ;
    \addplot[color={rgb,1:red,1.0;green,0.0;blue,0.0}, name path={8ea002dd-aef3-4d07-8273-8a7aa4844086}, draw opacity={1.0}, line width={1}, dotted, forget plot]
        table[row sep={\\}]
        {
            \\
            0.0  3.797683922388714  \\
            0.001101754553852787  3.7977184146194385  \\
            0.002203509107705574  3.797822160211135  \\
            0.0033052636615583607  3.7979957961342494  \\
            0.004407018215411148  3.7982403432937186  \\
            0.005508772769263934  3.7985571447510598  \\
            0.0066105273231167215  3.7989477781868253  \\
            0.007712281876969509  3.7994139421166038  \\
            0.008814036430822295  3.799957315536582  \\
            0.009915790984675084  3.8005793902181737  \\
            0.011017545538527868  3.808965848620352  \\
            0.012119300092380656  3.8056822983812664  \\
            0.013221054646233443  3.8037448549337856  \\
            0.01432280920008623  3.802623042567514  \\
            0.015424563753939018  3.802014273442349  \\
            0.016526318307791804  3.8065825233458748  \\
            0.01762807286164459  3.80270599222836  \\
            0.018729827415497377  3.80016614588595  \\
            0.019831581969350167  3.7983460143813264  \\
            0.020933336523202953  3.796830416973339  \\
            0.022035091077055736  3.795306881638736  \\
            0.023136845630908526  3.7935267238872745  \\
            0.024238600184761313  3.7912963976608234  \\
            0.0253403547386141  3.7884843084455326  \\
            0.026442109292466886  3.785032794512539  \\
            0.027543863846319676  3.7809662588604773  \\
            0.02864561840017246  3.7792415333090177  \\
            0.029747372954025245  3.7743047904161675  \\
            0.030849127507878035  3.7692131485958877  \\
            0.03195088206173082  3.7641681773040268  \\
            0.03305263661558361  3.759345276890193  \\
            0.034154391169436395  3.7548769133356346  \\
            0.03525614572328918  3.75084763949142  \\
            0.03635790027714197  3.747299701413508  \\
            0.037459654830994754  3.744246957586141  \\
            0.03856140938484754  3.741692305082493  \\
            0.039663163938700334  3.7400131854046026  \\
            0.04076491849255312  3.738936809455839  \\
            0.04186667304640591  3.7379130912356624  \\
            0.042968427600258687  3.736946328382536  \\
            0.04407018215411147  3.7360414965005115  \\
            0.045171936707964266  3.7352042068373605  \\
            0.04627369126181705  3.7344406851130207  \\
            0.04737544581566984  3.7337578097201156  \\
            0.048477200369522626  3.733163250695898  \\
            0.04957895492337542  3.7326657546586963  \\
            0.0506807094772282  3.732275630404172  \\
            0.051782464031080985  3.7222356526254687  \\
            0.05288421858493377  3.7211865580964996  \\
            0.05398597313878656  3.7201581536167376  \\
            0.05508772769263935  3.7191499198865046  \\
            0.05618948224649214  3.7181613609070063  \\
            0.05729123680034492  3.7171920306358768  \\
            0.058392991354197704  3.7162415514330616  \\
            0.05949474590805049  3.7153096264027554  \\
            0.060596500461903284  3.7143960467838917  \\
            0.06169825501575607  3.7135006950875327  \\
            0.06280000956960886  3.712623544472733  \\
            0.06390176412346164  3.7117646548045387  \\
            0.06500351867731442  3.710924165830212  \\
            0.06610527323116722  3.7101022879638568  \\
            0.06720702778502  3.709299291199042  \\
            0.06830878233887279  3.7085154927005592  \\
            0.06941053689272557  3.7077512436290596  \\
            0.07051229144657836  3.7070069157231518  \\
            0.07161404600043114  3.7062828881181447  \\
            0.07271580055428394  3.70557953480444  \\
            0.07381755510813673  3.7048972130468534  \\
            0.07491930966198951  3.7042362529875743  \\
            0.0760210642158423  3.703596948574821  \\
            0.07712281876969508  3.702979549862795  \\
            0.07822457332354787  3.7023842566677665  \\
            0.07932632787740067  3.701811213499052  \\
            0.08042808243125345  3.7012605056441967  \\
            0.08152983698510624  3.7007321562570157  \\
            0.08263159153895902  3.7002261242810652  \\
            0.08373334609281181  3.6997423030347  \\
            0.0848351006466646  3.699280519284425  \\
            0.08593685520051737  3.6988405326365807  \\
            0.08703860975437017  3.698422035092797  \\
            0.08814036430822295  3.6980246506099923  \\
            0.08924211886207574  3.6976479345224726  \\
            0.09034387341592853  3.6972913726766743  \\
            0.09144562796978131  3.6969543801220106  \\
            0.0925473825236341  3.69663629919019  \\
            0.09364913707748689  3.696336396758601  \\
            0.09475089163133968  3.696053860451331  \\
            0.09585264618519247  3.6957877934603536  \\
            0.09695440073904525  3.6955372075575785  \\
            0.09805615529289805  3.6953010137235123  \\
            0.09915790984675084  3.695078009581346  \\
            0.1002596644006036  3.6935752799013226  \\
            0.1013614189544564  3.693358135222936  \\
            0.10246317350830918  3.69315038389056  \\
            0.10356492806216197  3.6929515845266274  \\
            0.10466668261601475  3.692761252589456  \\
            0.10576843716986754  3.69257886353796  \\
            0.10687019172372034  3.692403859758729  \\
            0.10797194627757312  3.692235661114485  \\
            0.10907370083142591  3.6920736789664863  \\
            0.1101754553852787  3.6919173334604976  \\
            0.11127720993913148  3.691766073718448  \\
            0.11237896449298428  3.6916194003720983  \\
            0.11348071904683706  3.69147688960459  \\
            0.11458247360068984  3.6913382175571936  \\
            0.11568422815454263  3.691203183642098  \\
            0.11678598270839541  3.691071731033477  \\
            0.1178877372622482  3.690943962431663  \\
            0.11898949181610098  3.690820149191304  \\
            0.12009124636995377  3.6907007320914267  \\
            0.12119300092380657  3.690586312453618  \\
            0.12229475547765935  3.6904776329565903  \\
            0.12339651003151214  3.6903755482943223  \\
            0.12449826458536492  3.6902809866935087  \\
            0.12560001913921773  3.6901949041184277  \\
            0.1267017736930705  3.6901182336384837  \\
            0.1278035282469233  3.690051832794273  \\
            0.12890528280077607  3.689996431855964  \\
            0.13000703735462885  3.689952585564061  \\
            0.13110879190848163  3.6899206303818533  \\
            0.13221054646233443  3.6899006484906978  \\
            0.1333123010161872  3.6898924388479903  \\
            0.13441405557004  3.6898954946584697  \\
            0.1355158101238928  3.6945062793606422  \\
            0.13661756467774558  3.6945449221747326  \\
            0.13771931923159836  3.6992172580151195  \\
            0.13882107378545114  3.6945247964569816  \\
            0.13992282833930392  3.6899051536426644  \\
            0.14102458289315672  3.6898968143788506  \\
            0.1421263374470095  3.689899381069381  \\
            0.14322809200086228  3.6899135199505926  \\
            0.1443298465547151  3.6899395815146656  \\
            0.14543160110856787  3.6899776206287593  \\
            0.14653335566242065  3.6900274223170606  \\
            0.14763511021627346  3.6900885343595835  \\
            0.14873686477012624  3.690160306881026  \\
            0.14983861932397902  3.6902419381449167  \\
            0.1509403738778318  3.6903325249073995  \\
            0.1520421284316846  3.6904311149959823  \\
            0.15314388298553738  3.690536759331778  \\
            0.15424563753939016  3.690648560497008  \\
            0.15534739209324297  3.690765715152911  \\
            0.15644914664709575  3.69088754813123  \\
            0.15755090120094853  3.69101353676803  \\
            0.15865265575480134  3.6911433248994476  \\
            0.15975441030865412  3.691276726787025  \\
            0.1608561648625069  3.6914137219763523  \\
            0.1619579194163597  3.6915544426211175  \\
            0.16305967397021248  3.691699155119606  \\
            0.16416142852406526  3.691848237996564  \\
            0.16526318307791804  3.6920021578600455  \\
            0.16636493763177085  3.6921614450376423  \\
            0.16746669218562363  3.6923266701953623  \\
            0.16856844673947638  3.692498422924997  \\
            0.1696702012933292  3.692677292990318  \\
            0.17077195584718197  3.6928638546891857  \\
            0.17187371040103475  3.6930586546028246  \\
            0.17297546495488753  3.6932622029063995  \\
            0.17407721950874033  3.6934749683784958  \\
            0.1751789740625931  3.6936973772784234  \\
            0.1762807286164459  3.695198686502809  \\
            0.1773824831702987  3.6954282254756494  \\
            0.17848423772415148  3.6956715696235847  \\
            0.17958599227800426  3.6959298629535593  \\
            0.18068774683185707  3.6962041428509145  \\
            0.18178950138570985  3.6964953509308898  \\
            0.18289125593956262  3.6968043411307394  \\
            0.18399301049341543  3.69713188573317  \\
            0.1850947650472682  3.697478679807663  \\
            0.186196519601121  3.6978453444449477  \\
            0.18729827415497377  3.6982324290563673  \\
            0.18840002870882658  3.698640412966302  \\
            0.18950178326267936  3.699069706479426  \\
            0.19060353781653214  3.699520651583126  \\
            0.19170529237038494  3.6999935224375227  \\
            0.19280704692423772  3.7004885257997566  \\
            0.1939088014780905  3.7010058015311427  \\
            0.1950105560319433  3.701545423342097  \\
            0.1961123105857961  3.702107399943639  \\
            0.19721406513964887  3.702691676769502  \\
            0.19831581969350168  3.703298138447875  \\
            0.19941757424735443  3.703926612197145  \\
            0.2005193288012072  3.7045768722954526  \\
            0.20162108335506  3.7052486457774125  \\
            0.2027228379089128  3.7059416194459742  \\
            0.20382459246276557  3.7066554482600202  \\
            0.20492634701661835  3.7073897650793275  \\
            0.20602810157047116  3.708144191672622  \\
            0.20712985612432394  3.708918350812304  \\
            0.20823161067817672  3.7097118791797734  \\
            0.2093333652320295  3.71052444071835  \\
            0.2104351197858823  3.7113557399908963  \\
            0.2115368743397351  3.712205535038414  \\
            0.21263862889358787  3.713073649195531  \\
            0.21374038344744067  3.7139599813079234  \\
            0.21484213800129345  3.7148645138133793  \\
            0.21594389255514623  3.715787318182714  \\
            0.21704564710899904  3.716728557256274  \\
            0.21814740166285182  3.7176884840446713  \\
            0.2192491562167046  3.7186674365357253  \\
            0.2203509107705574  3.7196658279411072  \\
            0.2214526653244102  3.7206841314939276  \\
            0.22255441987826297  3.7217228582579533  \\
            0.22365617443211575  3.722782525053324  \\
            0.22475792898596855  3.732473827975327  \\
            0.22585968353982133  3.7329193485577057  \\
            0.2269614380936741  3.733466805560594  \\
            0.22806319264752692  3.7341067480821954  \\
            0.22916494720137967  3.734831015613473  \\
            0.23026670175523245  3.7356323605361927  \\
            0.23136845630908526  3.736504242164519  \\
            0.23247021086293804  3.737440743471507  \\
            0.23357196541679082  3.7384365675604476  \\
            0.2346737199706436  3.7394870738886934  \\
            0.2357754745244964  3.740606182028814  \\
            0.23687722907834918  3.742925299091857  \\
            0.23797898363220196  3.7457293520847483  \\
            0.23908073818605477  3.749030610312223  \\
            0.24018249273990755  3.7528224334277365  \\
            0.24128424729376033  3.757078945133161  \\
            0.24238600184761314  3.76173804227831  \\
            0.24348775640146592  3.766691323633747  \\
            0.2445895109553187  3.771783525187013  \\
            0.24569126550917147  3.7768233600654137  \\
            0.24679302006302428  3.7787494387312948  \\
            0.24789477461687706  3.783088880033095  \\
            0.24899652917072984  3.786855869370918  \\
            0.25009828372458265  3.7899852105703102  \\
            0.25120003827843546  3.7924939796698345  \\
            0.2523017928322882  3.794478252194822  \\
            0.253403547386141  3.796101980543501  \\
            0.25450530193999377  3.7975864129590176  \\
            0.2556070564938466  3.7992095694544883  \\
            0.2567088110476994  3.8013276154675104  \\
            0.25781056560155213  3.8044383622479505  \\
            0.25891232015540494  3.8093339426744595  \\
            0.2600140747092577  3.802273431444899  \\
            0.2611158292631105  3.803109475747574  \\
            0.26221758381696325  3.804590921747688  \\
            0.26331933837081606  3.807112674937476  \\
            0.26442109292466887  3.800920814126657  \\
            0.2655228474785216  3.8002588865233253  \\
            0.2666246020323744  3.799676275668409  \\
            0.26772635658622723  3.7991716555881263  \\
            0.26882811114008  3.7987434180989292  \\
            0.2699298656939328  3.7983898551753033  \\
            0.2710316202477856  3.798109312793656  \\
            0.27213337480163835  3.7979003179589736  \\
            0.27323512935549116  3.797761679318983  \\
            0.27433688390934396  3.797692561744326  \\
            0.2754386384631967  3.7976925354754147  \\
            0.27654039301704947  3.7977616004839247  \\
            0.2776421475709023  3.7979001864733886  \\
            0.2787439021247551  3.798109128520179  \\
            0.27984565667860783  3.7983896179246264  \\
            0.28094741123246064  3.798743127632001  \\
            0.28204916578631345  3.7991713116173265  \\
            0.2831509203401662  3.7996758778559103  \\
            0.284252674894019  3.8002584344826675  \\
            0.2853544294478718  3.800920307430794  \\
            0.28645618400172457  3.8071078261112454  \\
            0.2875579385555774  3.804585944915427  \\
            0.2886596931094302  3.80310440828051  \\
            0.28976144766328293  3.802268303322145  \\
            0.29086320221713574  3.8093227458593546  \\
            0.29196495677098855  3.8044258325420603  \\
            0.2930667113248413  3.801314070165447  \\
            0.2941684658786941  3.7991952294650653  \\
            0.2952702204325469  3.797571433959923  \\
            0.29637197498639967  3.796086474243435  \\
            0.2974737295402525  3.794462301016162  \\
            0.2985754840941052  3.7924776465346843  \\
            0.29967723864795803  3.7899685449303404  \\
            0.30077899320181084  3.786838910418834  \\
            0.3018807477556636  3.7830716582274873  \\
            0.3029825023095164  3.7787319772511543  \\
            0.3040842568633692  3.7768054558643813  \\
            0.30518601141722196  3.7717653963208644  \\
            0.30628776597107477  3.766672998949528  \\
            0.3073895205249276  3.7617195680434494  \\
            0.3084912750787803  3.757060389559977  \\
            0.30959302963263313  3.752803886797283  \\
            0.31069478418648594  3.7490121812626  \\
            0.3117965387403387  3.7457111617324848  \\
            0.3128982932941915  3.7429074742715907  \\
            0.3140000478480443  3.7405888584239566  \\
            0.31510180240189706  3.7394686562898176  \\
            0.31620355695574986  3.738418115904142  \\
            0.31730531150960267  3.737422294790829  \\
            0.3184070660634554  3.7364858348245162  \\
            0.31950882061730823  3.7356140342450033  \\
            0.32061057517116104  3.734812811412178  \\
            0.3217123297250138  3.7340887083096215  \\
            0.3228140842788666  3.7334489737868215  \\
            0.3239158388327194  3.7329017694718  \\
            0.32501759338657216  3.7324565472158104  \\
            0.32611934794042496  3.7227681181027172  \\
            0.3272211024942777  3.7217084866913264  \\
            0.3283228570481305  3.720669802220761  \\
            0.32942461160198333  3.7196515473132874  \\
            0.3305263661558361  3.7186532103511407  \\
            0.3316281207096889  3.7176743175596965  \\
            0.3327298752635417  3.7167144552082854  \\
            0.33383162981739445  3.7157732848158247  \\
            0.33493338437124726  3.7148505529091387  \\
            0.3360351389251  3.7139460962165103  \\
            0.33713689347895276  3.713059842868406  \\
            0.33823864803280557  3.7121918100607645  \\
            0.3393404025866584  3.711342098613808  \\
            0.3404421571405111  3.710510884890073  \\
            0.34154391169436393  3.7096984105746444  \\
            0.34264566624821674  3.7089049708583572  \\
            0.3437474208020695  3.7081309015783686  \\
            0.3448491753559223  3.7073765658566797  \\
            0.34595092990977505  3.7066423407468605  \\
            0.34705268446362786  3.705928604325644  \\
            0.34815443901748067  3.7052357235984155  \\
            0.3492561935713334  3.7045640434879403  \\
            0.3503579481251862  3.7039138770873077  \\
            0.35145970267903903  3.7032854972748828  \\
            0.3525614572328918  3.702679129696347  \\
            0.3536632117867446  3.7020949470730256  \\
            0.3547649663405974  3.7015330647259472  \\
            0.35586672089445015  3.700993537183973  \\
            0.35696847544830296  3.7004763557091565  \\
            0.35807023000215576  3.6999814465757512  \\
            0.3591719845560085  3.6995086699173245  \\
            0.3602737391098613  3.699057818983312  \\
            0.36137549366371413  3.698628619628874  \\
            0.3624772482175669  3.698220729893792  \\
            0.3635790027714197  3.697833739509881  \\
            0.3646807573252725  3.697467169200534  \\
            0.36578251187912525  3.6971204696104465  \\
            0.36688426643297806  3.6967930197163135  \\
            0.36798602098683086  3.696484124524915  \\
            0.3690877755406836  3.6961930118391004  \\
            0.3701895300945364  3.6959188278165764  \\
            0.3712912846483892  3.6956606309451248  \\
            0.372393039202242  3.695417383950225  \\
            0.3734947937560948  3.695187942943858  \\
            0.37459654830994754  3.6936875009869086  \\
            0.37569830286380035  3.693465507500073  \\
            0.37680005741765316  3.6932531119524916  \\
            0.3779018119715059  3.6930498945028614  \\
            0.3790035665253587  3.692855392039173  \\
            0.3801053210792115  3.692669099385044  \\
            0.3812070756330643  3.692490474367781  \\
            0.3823088301869171  3.6923189465727644  \\
            0.3834105847407699  3.692153929648817  \\
            0.38451233929462264  3.691994836996541  \\
            0.38561409384847545  3.691841100563473  \\
            0.38671584840232825  3.6916921922924093  \\
            0.387817602956181  3.6915476475322753  \\
            0.3889193575100338  3.691407089425194  \\
            0.3900211120638866  3.6912702529667096  \\
            0.3911228666177394  3.691137007138567  \\
            0.3922246211715922  3.6910073732783117  \\
            0.393326375725445  3.690881537758976  \\
            0.39442813027929774  3.690759857130065  \\
            0.39552988483315055  3.6906428541893894  \\
            0.39663163938700335  3.690531203984372  \\
            0.39773339394085605  3.690425709473944  \\
            0.39883514849470886  3.6903272674369902  \\
            0.3999369030485616  3.6902368260601865  \\
            0.4010386576024144  3.6901553363824777  \\
            0.4021404121562672  3.690083700298761  \\
            0.40324216671012  3.690022718019722  \\
            0.4043439212639728  3.6899730377777535  \\
            0.4054456758178256  3.689935110114992  \\
            0.40654743037167834  3.689909148409841  \\
            0.40764918492553115  3.6898950964210506  \\
            0.40875093947938396  3.689892602697892  \\
            0.4098526940332367  3.6899010006996393  \\
            0.4109544485870895  3.6945246391330495  \\
            0.4120562031409423  3.699217292135886  \\
            0.4131579576947951  3.694545059654916  \\
            0.4142597122486479  3.6945064596307957  \\
            0.4153614668025007  3.6898996752056754  \\
            0.41646322135635344  3.689896685234444  \\
            0.41756497591020625  3.6899049748926362  \\
            0.418666730464059  3.6899250503100687  \\
            0.4197684850179118  3.689957111336815  \\
            0.4208702395717646  3.6900010742852807  \\
            0.42197199412561737  3.6900566010379583  \\
            0.4230737486794702  3.690123135182572  \\
            0.424175503233323  3.690199944855422  \\
            0.42527725778717573  3.6902861710721866  \\
            0.42637901234102854  3.690380879518839  \\
            0.42748076689488135  3.6904831132155107  \\
            0.4285825214487341  3.6905919431686347  \\
            0.4296842760025869  3.6907065141757065  \\
            0.4307860305564397  3.690826083306987  \\
            0.43188778511029247  3.6909500492484026  \\
            0.4329895396641453  3.6910779714806106  \\
            0.4340912942179981  3.6912095791599704  \\
            0.43519304877185083  3.691344770343953  \\
            0.43629480332570364  3.6914836028676317  \\
            0.43739655787955645  3.6916262785820444  \\
            0.4384983124334092  3.691773122874661  \\
            0.439600066987262  3.691924561371403  \\
            0.4407018215411148  3.692081095550935  \\
            0.44180357609496756  3.69224327872903  \\
            0.4429053306488204  3.6924116935581246  \\
            0.4440070852026732  3.692586931873818  \\
            0.44510883975652593  3.6927695774552416  \\
            0.44621059431037874  3.6929601920529906  \\
            0.4473123488642315  3.693159304904504  \\
            0.4484141034180843  3.6933674058730412  \\
            0.4495158579719371  3.6935849423667806  \\
            0.45061761252578986  3.6950887038133646  \\
            0.45171936707964266  3.695311806374897  \\
            0.45282112163349547  3.6955480977534787  \\
            0.4539228761873482  3.695798780447814  \\
            0.45502463074120103  3.6960649435920727  \\
            0.45612638529505384  3.6963475755213966  \\
            0.4572281398489066  3.6966475731433976  \\
            0.45832989440275934  3.6969657489233705  \\
            0.45943164895661215  3.697302836065409  \\
            0.4605334035104649  3.6976594923093984  \\
            0.4616351580643177  3.6980363026679965  \\
            0.4627369126181705  3.698433781346844  \\
            0.46383866717202327  3.6988523730535494  \\
            0.4649404217258761  3.6992924538626943  \\
            0.4660421762797288  3.699754331794516  \\
            0.46714393083358163  3.7002382472531545  \\
            0.46824568538743444  3.700744373472998  \\
            0.4693474399412872  3.7012728171258376  \\
            0.47044919449514  3.7018236192464378  \\
            0.4715509490489928  3.702396756648802  \\
            0.47265270360284556  3.7029921440021303  \\
            0.47375445815669837  3.7036096367411266  \\
            0.4748562127105512  3.7042490349811845  \\
            0.4759579672644039  3.704910088586743  \\
            0.47705972181825673  3.7055925035145885  \\
            0.47816147637210954  3.706295949511239  \\
            0.4792632309259623  3.707020069185637  \\
            0.4803649854798151  3.70776448840268  \\
            0.4814667400336679  3.708528827862829  \\
            0.48256849458752066  3.7093127156426235  \\
            0.48367024914137347  3.710115800374074  \\
            0.4847720036952263  3.7109377646584494  \\
            0.485873758249079  3.7117783382434406  \\
            0.48697551280293183  3.7126373104265684  \\
            0.48807726735678464  3.7135145411420822  \\
            0.4891790219106374  3.7144099701749385  \\
            0.4902807764644902  3.7153236239832768  \\
            0.49138253101834295  3.7162556196414673  \\
            0.49248428557219576  3.7172061654630553  \\
            0.49358604012604856  3.7181755578658455  \\
            0.4946877946799013  3.7191641739829397  \\
            0.4957895492337541  3.72017245932655  \\
            0.49689130378760693  3.7212009093458738  \\
            0.4979930583414597  3.722250042782385  \\
            0.4990948128953125  3.732292744656371  \\
            0.5001965674491653  3.7326831903366275  \\
            0.501298322003018  3.7331809617762555  \\
            0.5024000765568709  3.73377575101254  \\
            0.5035018311107237  3.7344588124713325  \\
            0.5046035856645764  3.7352224772957148  \\
            0.5057053402184292  3.736059868363127  \\
            0.506807094772282  3.7369647612724815  \\
            0.5079088493261348  3.737931546116054  \\
            0.5090106038799875  3.7389552486302065  \\
            0.5101123584338404  3.7400315724959703  \\
            0.5112141129876931  3.7417098984030446  \\
            0.5123158675415459  3.7442649811581847  \\
            0.5134176220953988  3.7473180267685358  \\
            0.5145193766492515  3.750866141845376  \\
            0.5156211312031043  3.754895476989776  \\
            0.5167228857569571  3.7593638017497355  \\
            0.5178246403108099  3.7641865838618593  \\
            0.5189263948646626  3.769231379914222  \\
            0.5200281494185154  3.7743228097411627  \\
            0.5211299039723682  3.77925931779231  \\
            0.522231658526221  3.7809836030391955  \\
            0.5233334130800738  3.7850498881974666  \\
            0.5244351676339265  3.7885011250687355  \\
            0.5255369221877794  3.791312902539353  \\
            0.5266386767416321  3.7935428729636222  \\
            0.5277404312954849  3.7953226192817895  \\
            0.5288421858493377  3.7968456714758236  \\
            0.5299439404031905  3.7983606901762745  \\
            0.5310456949570432  3.800180111476934  \\
            0.5321474495108961  3.8027190624879106  \\
            0.5332492040647488  3.8065944339024216  \\
            0.5343509586186016  3.8020194214449794  \\
            0.5354527131724545  3.802628143899211  \\
            0.5365544677263072  3.8037498811160004  \\
            0.53765622228016  3.8056872166936055  \\
            0.5387579768340128  3.8089706141438198  \\
            0.5398597313878656  3.800579869531509  \\
            0.5409614859417183  3.79995774041193  \\
            0.5420632404955712  3.79941431296285  \\
            0.543164995049424  3.7989480953667054  \\
            0.5442667496032767  3.798557408577135  \\
            0.5453685041571296  3.79824055402931  \\
            0.5464702587109823  3.7979959539932437  \\
            0.5475720132648351  3.797822265357961  \\
            0.5486737678186879  3.797718467164688  \\
            0.5497755223725407  3.7976839223888548  \\
        }
        ;
\end{axis}
\end{tikzpicture}

}


\caption{ Translation results for Laplace $n=16$ for $500$ iterations, $\delta =(0, 2 \sqrt{2}h) $ for $L=2.7$ with a circle $R=1$       }

\end{figure}

    % 
\newpage
\section{Cahn Hilliard equation }%
\label{sec:cahn_hilliard_equation}


\subsection{Introduction to numerical methods for Cahn Hilliard}%
\label{sub:introduction_to_numerical_methods_for_cahn_hilliard}

hello




\newpage


    % \section{Conclusion}

We have presented several interesting multiphysics processes to consider when modelling cell membrane dynamics. First, we introduced an elastic bending energy on evolving surfaces using the Canham-Helrich energy functional \eqref{eq:CH} and the equivalent Willmore energy functional. Secondly, we established that the diffusion of a two-phase system can be modelled using the Ginzburg-Landau energy functional \eqref{eq:GL} and discussed how we could combine the energy functional models to do multiphysics problems. 

To demonstrate the techniques used to solve physical problems on evolving domains, we chose the Willmore energy functional \eqref{eq:WE}. The background theory of calculus on surfaces and evolutionary dynamics was briefly discussed and then applied to define the strong and weak (or integral) formulation of the system dynamics. We then established a basic numerical framework and introduced the evolution surface finite element method.

    % \newpage
\section{Appendix}%
\label{sec:appendix}

\subsection{$L_{2} \left( \Omega  \right)$ space }%
\label{sub:_l__2_space_}

Using the definition from \cite{manzoni2021optimal} and we let $\Omega $ be a an open set in $\mathbb{R} ^{d}$ and $p \in \mathbb{R} $  such that $p \ge 1$. Then we denote
$L^{p}\left( \Omega  \right) $ to be the set of measurable function $u: \Omega \to \mathbb{R} $ such that  it is equipped
in a finite Banach space \[
\|u\|_{L^{p}\left( \Omega  \right)}^{} = \left( \int_{\Omega }^{} \left\lvert u \right\rvert ^{p}
\right)^{\frac{1}{p}}.
\]

Now let $u,v: \Omega  \to \mathbb{R} $. Then is $L_{2}\left( \Omega  \right)$ a Hilbert space when the inner product is
finite such that this exists \[
\left( u,v \right)_{L^{p}\left( \Omega   \right)} = \int_{\Omega }^{} uv  .
\]










    % \newpage
    \printbibliography

    \includepdf[pages=-]{results/illustration/back_page.pdf}




\end{document}
