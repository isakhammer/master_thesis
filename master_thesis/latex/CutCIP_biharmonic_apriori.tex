
\subsection{A priori error estimate}%
\label{sec:a_priori_estimates}


For the proposed method, we want to derive a priori error estimate with respect to both the  $\| \cdot  \|_{a_{h},*   }^{  } $-norm and the  $\| \cdot  \|_{ \Omega  }^{
} $-norm.
We will construct a suitable (quasi-)interpolation operator, here we use the Clement quasi interpolation operator which in contrast to the standard Lagrange nodal interpolation iterator is also defined for low regularity function in $u \in L^{2}(
\Omega ) $.
In combination with discrete coercivity this allows you to derive an a priori estimate in the energy norm. Finally, we use a standard duality argument, i.e. Aubin-Nitsche trick, to derive the $L^{2}$ error estimate.

Recall that for $v \in H^{1}( \mathcal{T } _{h}) $ these inequalities holds $\forall T \in \mathcal{T} _{h}$ such that \[
\begin{split}
    \| v \|_{ \partial T }^{  } &\lesssim h^{-\frac{1}{2}}_{T}\|  v \|_{ T }^{  }+ h^{\frac{1}{2}} \| \nabla v \|_{T  }^{   }  , \\
    \| v \|_{ \Gamma \cap T }^{  } &\lesssim  h^{-\frac{1}{2}} \| v \|_{T  }^{  }   + h^{\frac{1}{2}}_{T} \| \nabla v \|_{ T }^{  },
\end{split}
\]
for proof see \cite[Lemma 4.2]{hansbo2003finite}.

Assume that $\Omega $ has a boundary $\Gamma $ in $C^{1}$, then it does exist an bounded extension operator, \[
    ( \cdot ) ^{e}: H^{m}( \Omega )  \to H^{m} ( \mathbb{R} ^{d}),
\]
for all  $v \in H^{m}( \Omega )$ which satisfies
\begin{equation}
    \begin{split}
 v^{e}| _{\Omega } =   v,  \\
\| v^{e} \|_{ m,\mathbb{R} ^{d}  }^{  } & \lesssim \| v \|_{ m, \Omega  }^{  }.
    \end{split}
\end{equation}
For more information, see \cite[Theorem 9.7]{brezis2011functional} and \cite[p.181, p.185]{stein1970singular}. To simplify, we use the notation $ v := v^{e}   $ for $v \in \mathbb{R} ^{d} \setminus \Omega $.

Recall the Lemma \ref{lemma:clements}, we construct the Cléments interpolator combined with the extension operator such that $C_{h}^{e}: H^{m}( \mathbb{R} ^{d}) \to V_{h}$ such that  $C ^{e} _{h} v := C _{h} v^{e} $.
We can immediately observe that the interpolation satisfies the global error estimates. Let $v \in H^{s}( \Omega ) $ and $r = \mathrm{min}(s , k+1) $  that is,
\begin{align}
    \| v - C _{h}^{e} v \|_{  l, \mathcal{T} _{h} }^{  } & \lesssim h^{r-l}\sum_{T \in \mathcal{T}_h} \| v \|_{ r, \omega(T) }^{  }, \quad 0\le l\le r \\
    \| v - C ^{e}_{h}v \|_{ l,\mathcal{F} _{h} }^{  } & \lesssim h^{r-l-\frac{1}{2}}\sum_{T \in \mathcal{T}_h} \| v \|_{ r, \omega(F)  }^{  }, \quad 0  \le  l \le   r-\frac{1}{2} \\
\| v - C ^{e}_{h}v \|_{ l, \Gamma }^{  } & \lesssim h^{r-l-\frac{1}{2}} \sum_{T \in \mathcal{T}_h}  \| v \|_{ r,  \omega(T)  }^{  }, \quad 0  \le  l \le  r-\frac{1}{2}
\end{align}
 and arguing that $ \sum_{T}^{} \| v \|_{s,\omega ( T)   }^{  } \le C  \| v \|_{s, \mathcal{T}_{h}   }^{  } $ where $C$ is some constant decided by the maximum number of elements in a patch
    $\omega( T)  $ for all $T \in \mathcal{T} _{h}$. This also holds for the inequality $ \sum_{T}^{} \| v \|_{s,\omega ( F)   }^{  } \le C  \| v \|_{s, \mathcal{T}_{h}   }^{  } $. Hence, we have perhaps an even more useful set of inequalities.
\begin{align}
    \label{eq:bi_projection_estimates_1}
    \| v - C _{h}^{e} v \|_{  l, \mathcal{T} _{h} }^{  } & \lesssim h^{r-l} \| v \|_{ r, \mathcal{T} _{h} }^{  }, \quad 0\le l\le r \\
    \label{eq:bi_projection_estimates_2}
    \| v - C ^{e}_{h}v \|_{ l,\mathcal{F} _{h} }^{  } & \lesssim h^{r-l-\frac{1}{2}} \| v \|_{ r, \mathcal{T} _{h}  }^{  }, \quad 0  \le  l \le   r-\frac{1}{2} \\
    \label{eq:bi_projection_estimates_3}
\| v - C ^{e}_{h}v \|_{ l, \Gamma }^{  } & \lesssim h^{r-l-\frac{1}{2}}   \| v \|_{ r,  \mathcal{T} _{h}  }^{  }, \quad 0  \le  l \le  r-\frac{1}{2}
\end{align}
% \todo[inline]{ Maybe hard to argue \eqref{eq:bi_projection_estimates_3} to hold on $\Gamma $, but may be related to some generalization of \eqref{eq:bi_n_cut_inverse_1} and \eqref{eq:bi_cut_inverse_1}. Anyhow, \eqref{eq:bi_projection_estimates_2} and
% \eqref{eq:bi_projection_estimates_3} was never used in the proof of Lemma \ref{lemma:astar_estimate} since we used inverse estimates and ended up with \eqref{eq:bi_projection_estimates_1} instead on all of them.}
Naturally can we see this is the tools we need to construct an estimate for the energy norm.

\begin{lemma}
    \label{lemma:astar_estimate}
    Let $u \in H^{s}( \Omega ) $ for $s\ge 3$ be a exact solution to $\eqref{eq:cont_weak_problem} $ and let $k$ be the polynomial order. Then we have
    \begin{equation}
    \|  u - C_{h}u \|_{ a_{h},*  }^{  } \lesssim h^{r-2} \| u \|_{ r, \Omega  }^{  }, \quad r = \mathrm{min} ( s, k+1)
    \end{equation}
\end{lemma}
\begin{proof}
    By definition is
    \begin{equation}
        \begin{split}
            \| u - C_{h}^{e}u \|_{ a_{h}, * }^{  2}  =& \ \alpha  \|  ( u - C_{h}^{e}u) \|_{ \mathcal{T} _{h} \cap \Omega  }^{ 2}  + \| D^2 ( u - C_{h}^{e}u ) \|_{\mathcal{T} _{h} \cap \Omega   }^{ 2 } \\  &  + \gamma \| h^{-\frac{1}{2}} \jump{ \partial _{n} (u -
        C_{h}^{e} u) }   \|_{ \mathcal{F}_{h}^{}\cap \Omega    }^{ 2
        } + \gamma \| h^{-\frac{1}{2}}  \partial _{n} (u - C_{h}^{e}u)    \|_{ \Gamma   }^{ 2 } \\
          & + \| h^{\frac{1}{2}} \mean{ \partial _{nn} (u - C_{h}^{e}u) }   \|_{\mathcal{F} _{h}^{} \cap \Omega   }^{  2} +  \| h^{\frac{1}{2}} \partial _{nn}(u - C_{h}^{e}u)     \|_{ \Gamma }^{  2}.
        \end{split}
    \end{equation}

    The strategy is to bound each term individually.
             Starting with the first two terms we get
             \begin{equation}
        \begin{split}
            \alpha  \|  ( u - C_{h}^{e}u) \|_{ \mathcal{T} _{h} \cap \Omega  }^{ 2}    & \lesssim \|  ( u - C_{h}^{e}u) \|_{0,\mathcal{T} _{h}  }^{ 2} \\
             &    \lesssim  h^{2(r-0)}    \| u \|_{r,\mathcal{T}_{h} }^{  2}\lesssim  h^{2(r-2)}    \| u \|_{r,\mathcal{T}_{h} }^{  2} \\
            \| D^2 ( u - C_{h}^{e}u ) \|_{\mathcal{T} _{h} \cap \Omega   }^{ 2 } &  \lesssim  \|  u - C_{h}^{e}u  \|_{2,\mathcal{T} _{h}   }^{ 2 }   =  \|  u - C_{h}^{e}u  \|_{2,\mathcal{T} _{h}   }^{ 2 } \\
                                                                                 & \lesssim  h^{2(r-2)} \| u \|_{ r, \mathcal{T}_{h}}^{ 2 }.
        \end{split}
             \end{equation}
    Here we simply used \eqref{eq:bi_projection_estimates_1}.
    Recall the first order inverse estimate \eqref{eq:bi_n_cut_inverse_2} and that $\| \jump{ \partial _{n} u }   \|_{ \mathcal{F} _{h} }^{  } \le \| \partial _{n^{+}} u^{+}   \|_{ \mathcal{F} _{h} }^{  } +
    \|\partial _{n^{-}} u^{-}   \|_{ \mathcal{F} _{h} }^{  } \lesssim  \|\partial _{n} u \|_{ \partial\mathcal{T }_{h}  }^{2  }  $, hence, this implies $\| \jump{ \partial _{n} u }   \|_{   \mathcal{F}_{h}\cap \Omega    }^{2} \lesssim h^{-1} \| \nabla u \|_{ \mathcal{T}
        _{h} }^{2  }  $. Using this and the inequality \eqref{eq:bi_projection_estimates_1} we therefore can observe,
            \begin{equation}
        \begin{split}
            \gamma \| h^{-\frac{1}{2}} \jump{ \partial _{n} ( u - C_{h}^{e}u ) }   \|_{ \mathcal{F}_{h} \cap \Omega   }^{  2}   & \lesssim
            h^{-1} \|   \partial _{n} ( u - C_{h}^{e}u )    \|_{  \partial \mathcal{T} _{h} }^{2  } \lesssim  h^{-2} \|   \nabla  ( u - C_{h}^{e}u )    \|_{ \mathcal{T} _{h}\cap \Omega  }^{2  }\\
&=  h^{-2} \|    u - C_{h}^{e}u     \|_{ 1,\mathcal{T} _{h} }^{2  } \lesssim  h^{ -2 } h^{ 2(r - 1)  } \| u \|_{ r, \mathcal{T}_{h}   }^{  2} \\
& \lesssim   h^{ 2(r - 2)  } \| u \|_{ r, \mathcal{T}_{h}   }^{  2} \\
        \end{split}
            \end{equation}
        .
    And for the boundary term we do a similar procedure, but instead use the first order inverse estimate \eqref{eq:bi_n_cut_inverse_1}.
        \begin{equation}
        \begin{split}
            \gamma \| h^{-\frac{1}{2}}  \partial _{n} ( u - C_{h}^{e}u ) \|_{ \Gamma    }^{  2} & \lesssim h^{-2} \|   \nabla  ( u - C_{h}^{e}u )    \|_{ \mathcal{T}_{h} }^{2  }  \lesssim  h^{-2} \|    u - C_{h}^{e}u     \|_{1,2, \mathcal{T}_{h}   }^{2  } \\
            & \lesssim h^{-2} h^{2(r-1)}   \| u \|_{r, \mathcal{T}_{h}   }^{  2}  \lesssim  h^{2(r- 2)}   \| u \|_{r, \mathcal{T}_{h}   }^{  2}
        \end{split}
        \end{equation}
            Recall that $\| \mean{ u }   \|_{ \mathcal{F} _{h} }^{  } \le \| u^{+} \|_{ \mathcal{F} _{h}  }^{  } + \| u^{-} \|_{ \mathcal{F} _{h}  }^{  }   \lesssim  \| u \|_{ \partial\mathcal{T }_{h}  }^{2  }  $ and the second order inverse
            inequality \eqref{eq:bi_cut_inverse_2}. It is clear
            by using \eqref{eq:bi_projection_estimates_1} that this holds.
            \begin{equation}
                \begin{split}
 \| h^{\frac{1}{2}} \mean{ \partial _{nn} (u - C_{h}^{e}u) }   \|_{\mathcal{F} _{h}^{} \cap \Omega   }^{  2} &  \lesssim h^{} \|   \partial _{nn} (u - C_{h}^{e}u)    \|_{\partial \mathcal{T} _{h} \cap \Omega    }^{  2}  \lesssim h^{} h^{-1}  \|   D^2 (u - C_{h}^{e}u)    \|_{ \mathcal{T} _{h}   }^{  2} \\
                                                                                                                &  = \|   u - C_{h}^{e}u    \|_{ 2, \mathcal{T} _{h}   }^{  2}  \lesssim h^{2(r - 2)}  \| u \|_{r, \mathcal{T} _{h} }^{  }
                \end{split}
            \end{equation}
                Similarly we can easily see by using \eqref{eq:bi_projection_estimates_1} and the second order boundary inverse inequality  \eqref{eq:bi_cut_inverse_1} that this must hold,

                \begin{equation}
              \| h^{\frac{1}{2}} \partial _{nn}(u - C_{h}^{e}u)     \|_{ \Gamma }^{  2} \lesssim   \|  D^2(u - C_{h}^{e}u)     \|_{ \mathcal{T}_{h}   }^{  2} \lesssim \|  u - C_{h}^{e}u \|_{ 2, \mathcal{T}_{h} }^{2  } \lesssim h^{2(r-2)}  \| u \|_{ r, \mathcal{T}_{h}   }^{2  }
                \end{equation}

    Thus, all elements is bounded by $ h^{2(r-2)}  \| u \|_{ r, \mathcal{T}_{h}   }^{2  } $  and the proof is complete.
\end{proof}

\begin{lemma}[Weak galerkin orthogonality]
Let $u \in H^{s}( \Omega )  $, $ s\ge 3 $  be the exact solution to   \eqref{eq:cont_weak_problem} and $u_{h} \in V_{h}$ is a discrete solution to \eqref{eq:discrete_CutCIP_prob}. Then is \[
    a_{h}( u - u_{h}, v) = g_{h} ( u_{h}, v) \quad \forall v \in V_{h}.
    \]
\end{lemma}

\begin{proof}
   From the definition of the problem \eqref{eq:discrete_CutCIP_prob} and utilizing that $a_{h}( u,v) = l(v ) \forall v \in V_{h} $ can we easily observe that \[
       \begin{split}
   l(v ) & =  A_{h}( u_{h},v) =  a_{h}( u,v)  = a_{h}( u_{h},v)+g_{h}( u_{h},v)
       \end{split}
   \]
    Hence, it is clear that $a_{h}( u -  u_{h}, v) = g_{h}( u_{h},v)  $.
\end{proof}

\begin{assumption}[EP2]
    \label{as:bi_EP2}
    For $v \in H^{s}( \Omega ) $ and $r = \min \{s,k+1 \} $, the semi-norm $\abs{ \ \cdot \  }_{g_{h}} $ satisfies the following estimate, \[
    \abs{ C _{h}^{e} v } _{g_{h}} \lesssim  h^{r-1} \| v \|_{ r,\Omega  }^{  }.
    \]
\end{assumption}


\begin{theorem}
    \label{thm:apriori_result}
    Let $u \in H^{s}( \Omega ) $ , $s\ge 3$ a solution to \eqref{eq:cont_weak_problem} and let $u \in V_{h}$ of order $k\ge 2$ be the discrete solution to \eqref{eq:discrete_CutCIP_prob}. Then for $r = \min_{}\{s, k+2\} $ the error $e = u - u_{h}$ satisfies
    \begin{align}
        \label{eq:bi_apriori_1}
            \| e \|_{ a_{h},* }^{  } &\lesssim   h^{r-2} \| u \|_{ r,\Omega  }^{  }\\
        \label{eq:bi_apriori_2}
        \| e \|_{ \Omega  }^{  } &\lesssim   h^{r-\mathrm{max}\left\{ 0, 3-k \right\} } \| u \|_{ r,\Omega  }^{  }
    \end{align}

\end{theorem}

\begin{proof}
    We will divide the proof into two steps.
    \begin{enumerate}[label=\arabic*)]
        \item We want to prove that $\| e \|_{ a_{h},* }^{  } \lesssim   h^{r-1} \| u \|_{ r,\Omega  }^{  }$.
    Let $e = u - u_{h}$ consist of $e = e_{h} + e_{\pi }$, where the discrete error has the form $e_{h} = C _{h}^{e} u - u_{h}$ and the interpolation error $e_{\pi } = u - C _{h} ^{e}u$. We can then observe that
    \[
        \begin{split}
    \| u - u_{h} \|_{ a_{h} }^{  } & \lesssim  \| u - C_{h}^{e} u + C_{h}^{e}u - u_{h} \|_{ a_{h},* }^{  } \\
    & \le \|  u - C_{h}^{e} u \|_{a_{h},*  }^{  } +  \| C_{h}^{e}u - u_{h} \|_{a_{h},*  }^{  }\\
                                     & \le \| e_{\pi } \|_{a_{h},*}^{  } + \| e_{h} \|_{A_{h}  }^{  }
        \end{split}
    \]
    \red{Using Lemma \ref{lemma:astar_estimate}, can we see that $\| e_{\pi } \|_{a_{h},*}^{  } \lesssim h^{r-1} \| u \|_{ r,\Omega  }^{  }  $ is already fulfilled.} So it remains to check the discrete part. From Lemma \ref{lemma:bi_Ah_coercive}, \ref{lemma:bi_Ah_bounded}, the weak Galerkin orthogonality and Assumption \ref{as:bi_EP2} is it natural to see that, \[
    \begin{split}
\| e_{h} \|_{ A_{h} }^{ 2 } & \lesssim a_{h}( C _{h}^{e} u - u_{h}, e_{h}) + g_{h}( C _{h}^{e}u - u_{h}, e_{h}) \\
 & = a_{h}( C _{h}^{e} u - u, e_{h}) + a_{h}( u - u_{h}, e_{h}) + g_{h}( C _{h}^{e}u - u_{h}, e_{h}) \\
 & = a_{h}( C _{h}^{e} u - u, e_{h}) + g_{h}( C _{h}^{e}u, e_{h}) \\
 & \lesssim h^{r-1} \| u \|_{ r, \Omega  }^{  } \| e_{h} \|_{ A_{h} }^{  }.
    \end{split}
\]



The last line of the calculations above was the Assumption \ref{as:bi_EP2}, i.e.,
\[
    \begin{split}
        a_{h}( C _{h}^{e} u - u, e_{h}) + g_{h}( C _{h}^{e}u, e_{h}) &\lesssim \| C _{h}^{e} u - u \|_{a_{h},*  }^{  } \| e_{h} \|_{a_{h}  }^{  }
        + \abs{ C _{h}^{e}u }_{g_{h}} \abs{e_{h}  }_{g_{h}} \\
         &\lesssim \| C _{h}^{e} u - u \|_{a_{h},*  }^{  } \| e_{h} \|_{a_{h}  }^{  } + h^{r-1} \| e_{h} \|_{r, \Omega   }^{  }\abs{e_{h}  }_{g_{h}} \\
         &\lesssim (\| C _{h}^{e} u - u \|_{a_{h},*  }^{  } + h^{r-1} \| e_{h} \|_{r, \Omega   }^{  }) \|e_{h}\|_{A_{h}} \\
         &\lesssim  h^{r-1} \| u \|_{r, \Omega   }^{  } \|e_{h}\|_{A_{h}}.
    \end{split}
\]
Here we noticed that $\| e_{h} \|_{a_{h}  }^{  } + \abs{e_{h}  }_{g_{h}} \lesssim \| e_{h} \|_{ A_{h} }^{  }  $. We also argued that $\| C _{h}^{e} u - u \|_{a_{h},*  }^{  } \lesssim h^{r-1}\| u \|_{ r,\Omega  }^{  }  $ from Lemma
\ref{lemma:astar_estimate}.
Hence, the first part of the proof is complete.

    \item We want to show that $ \| e \|_{ \Omega  }^{  } \lesssim   h^{r} \| u \|_{ r,\Omega  }^{  }$. The idea is to apply the so-called Aubin-Nitsche duality trick while being aware of the ghost penalty $g_{h}$. Let us denote the following
        observation.
        Assume that $e:= u -u_{h} \in L^{2}( \Omega ) $ and $\psi  \in H^{4}( \Omega ) $.
        Let the corresponding dual problem to \eqref{eq:bi_problem} be
        \begin{equation}
            \begin{split}
            \Delta ^2 \psi &= e  \quad  \text{ in } \Omega  \\
            \partial _{n} \psi &= 0 \quad \text{ on } \Gamma \\
            \partial _{n} \Delta \psi & = 0 \quad  \text{ on } \Gamma   \\
            \end{split}
        \end{equation}

        This implies that it exists a $\psi \in H^{4}( \Omega ) $ such that $a_{h}( \psi, v ) = ( e,v)_{\Omega } \ \forall v \in V_{h}  $. Hence, we can easily observe that \begin{equation}
            \label{eq:ni_1}
            \begin{split}
        \| e \|_{ \Omega  }^{ 2 }  & = ( e,e)_\Omega   = ( e, \Delta ^2 \psi )_{\Omega } \\
        &= a_{h}( e, \psi ) = a_{h}( u-uh, \psi ) \\
        &= a_{h}( u-u_h, \psi + C^{e}_{h}\psi  - C^{e}_{h}\psi )  \\
        &= a_{h}( u-u_h, \psi   - C^{e}_{h}\psi ) +  a_{h}( u-u_h, C^{e}_{h}\psi )  \\
        &= a_{h}( u-u_h, \psi  - C^{e}_{h}\psi )  \\
        &\le   \|u-u_{h}  \|_{a_{h},*  }^{  }  \| \psi  - C^{e}_{h}\psi \|_{a_{h},*  }^{  }    \\
            \end{split}
        \end{equation}

        Here we applied the Galerkin orthogonality $ a_{h}( u-uh, C^{e}_{h}\psi ) = 0$.
        Using the a priori estimate \eqref{lemma:astar_estimate}  is it clear that
        \begin{equation}
            \label{eq:ni_2}
        \|u-u_{h}  \|_{a_{h},*  }^{  }  \le h^{r} \| u \|_{r,\Omega  }^{  }\quad  \text{and}\quad  \| \psi  - C^{e}_{h}\psi \|_{a_{h},*  }^{  }  \le h^{r} \| \psi  \|_{4, \Omega   }^{  }.
        \end{equation}

        And then standard inverse estimate \eqref{eq:inv1} can we see
        \begin{equation}
            \label{eq:ni_3}
             h^{r} \| u \|_{r,\Omega  }^{  }  \le h^{r-2} \| u \|_{\Omega  }^{  } \quad  \text{and}\quad  h^{\widetilde{r} -2} \| \psi  \|_{4,\Omega  }^{  }  \le h^{\widetilde{r}-2} \| \psi  \|_{\Omega }^{  }.
        \end{equation}
        Here is $r = \mathrm{max}(3,k+1)$ and $ \widetilde{r} = \mathrm{max}(4,k+1)$.
        Combining \eqref{eq:ni_1}, \eqref{eq:ni_2} and \eqref{eq:ni_3} we have, \begin{equation}
            \| e \|_{\Omega   }^{ 2 } \lesssim  h^{r-2} \| u \|_{\Omega   }^{  } \|  \psi \|_{ \Omega  }^{  }.
        \end{equation}

        Using that $\| \psi  \|_{ \Omega  }^{  } \le \| e \|_{ \Omega   }^{  }  $ is it easy to see that \begin{equation}
            \| e \|_{\Omega   }^{  } \lesssim  h^{r-2} \| u \|_{\Omega   }^{  }
        \end{equation}

    \end{enumerate}
\end{proof}


