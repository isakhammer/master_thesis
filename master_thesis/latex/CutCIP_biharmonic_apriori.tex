
\subsection{A priori estimates}%
\label{sec:a_priori_estimates}


In this section, we will only briefly overview the important assumptions and definitions and then prove that the ghost penalty does affect the convergence rate specifically for the energy norm.
However, the $C^{0}$ discrete solution is to rough for the standard Lagrange interpolation operator. Hence, this motivates us to introduce a method to interpolate a non-smooth function, the so-called Cléments interpolation operator.

\subsubsection{Cléments interpolation}%
\label{ssub:clement_operator}

\begin{figure}[h!]
\begin{minipage}{.5\linewidth}
\centering
\subfloat[]{
    \label{fig:macroelements:a}
    \begin{tikzpicture}[scale=0.5]
        % Arbitrary triangle
        \coordinate (A1) at (0,0);
        \coordinate (B1) at (4,1);
        \coordinate (C1) at (1,3);
        \fill [red!30] (A1) -- (B1) -- (C1) -- cycle;
        \draw (A1) -- (B1) -- (C1) -- cycle;

        % Centroid
        \coordinate (ai) at (barycentric cs:A1=1,B1=1,C1=1);
        \fill (ai) circle (2pt);
        \node[anchor=north west] at (ai) {$a_i$};

        % Reference triangle
        \coordinate (A2) at ($(A1) + (6, 0)$);
        \coordinate (B2) at ($(A2) + (0, 3)$);
        \coordinate (C2) at ($(A2) + (3, 0)$);
        \fill [blue!30] (A2) -- (B2) -- (C2) -- cycle;
        \draw (A2) -- (B2) -- (C2) -- cycle;

        % Centroid of the reference triangle
        \coordinate (ahi) at (barycentric cs:A2=1,B2=1,C2=1);
        \fill (ahi) circle (2pt);
        \node[anchor=north west] at (ahi) {$\hat{a}_{j( i)} $};

        \draw[->, thick, >=stealth] ($(ahi)+(0.2,+0.2)$) to[bend right] node[midway, above] {$G_{A_i}$} ($(ai)+(0.2,+0.3)$);

    \end{tikzpicture}
}
\end{minipage}%
\begin{minipage}{.5\linewidth}
\centering
\subfloat[]{
    \label{fig:macroelements:b}
    \begin{tikzpicture}[scale=0.5]
    % Arbitrary square
    \coordinate (A1) at (0,0);
    \coordinate (B1) at (3,1);
    \coordinate (C1) at (4,4);
    \coordinate (D1) at (1,3);
    \draw (A1) -- (B1) -- (C1) -- (D1) -- cycle;
    \fill[red!30] (A1) -- (B1) -- (C1) -- (D1) -- cycle;

    % Draw edge from D1 to B1
    \draw (D1) -- (B1);

    % Pick a point on the edge and label it as a_i
    \coordinate (a_i) at ($(D1)!.6!(B1)$);
    \fill (a_i) circle (2pt);
    \node[anchor=north] at (a_i) {$a_i$};

    % Reference equilateral triangle
    \coordinate (A2) at ($(A1) + (6, 0)$);
    \coordinate (B2) at ($(A2) + (2, 3.464)$); % 3.464 = 2 * sqrt(3)
    \coordinate (C2) at ($(A2) + (4, 0)$);
    \fill[blue!30] (A2) -- (B2) -- (C2) -- cycle;
    \draw (A2) -- (B2) -- (C2) -- cycle;

    % Divide the equilateral triangle into two right triangles
    \coordinate (M) at ($(A2)!.5!(C2)$);
    \draw (B2) -- (M);

    % Pick a point on the shared edge and label it as ahat_i
    \coordinate (ahat_i) at ($(B2)!.7!(M)$);
    \fill (ahat_i) circle (2pt);
    \node[anchor=north west] at (ahat_i) {$\hat{a}_{j( i)} $};

    % Draw the mapping G_{A_i} from ahat_i to a_i
    \draw[->, thick, >=stealth] ($(ahat_i)+(0.2,0.2)$) to[bend right] node[midway, above] {$G_{A_i}$} ($(a_i)+(0.2,0.2)$);
\end{tikzpicture}
}
\end{minipage}\par\medskip
\centering
\subfloat[]{
    \label{fig:macroelements:c}
 \begin{tikzpicture}[scale=0.5]

    % Central vertex a_i
    \coordinate (ah_i) at (0, 0);

    % Reference hexagon
    \foreach \angle in {0, 60, ..., 300} {
        \coordinate (A) at (\angle:2.5);
        \coordinate (B) at (\angle + 60:2.5);
        \fill[blue!30] (ah_i) -- (A) -- (B) -- cycle;
        \draw (ah_i) -- (A) -- (B) -- cycle;
    }

    \fill (ah_i) circle (2pt);
    \node[anchor=east, yshift=0.25cm] at (ah_i) {$\hat{a}_{j( i) }$};

    \coordinate (A1) at (5, 0);
    \coordinate (B1) at (8, 0);
    \coordinate (C1) at (8, 3);
    \coordinate (D1) at (7, 4);
    \coordinate (E1) at (3.5, 4);
    \coordinate (F1) at (3, 2);
    \fill[red!30] (A1) -- (B1) -- (C1) -- (D1) -- (E1) -- (F1) -- cycle;
    \draw (A1) -- (B1) -- (C1) -- (D1) -- (E1) -- (F1) -- cycle;

    \coordinate (ai) at (barycentric cs:A1=1,B1=1,C1=1,D1=1,E1=1,F1=1);
    % Draw lines from vertices to a_i
    \draw (A1) -- (ai);
    \draw (B1) -- (ai);
    \draw (C1) -- (ai);
    \draw (D1) -- (ai);
    \draw (E1) -- (ai);
    \draw (F1) -- (ai);
    % Centroid
    \fill (ai) circle (2pt);
    \node[anchor=south, yshift=0.2cm] at (ai) {$a_i$};

    \draw[->, thick, >=stealth] ($(ah_i)+(0.2,- 0.2)$) to[bend right] node[midway, above, yshift=0.1cm] {$G_{A_i}$} ($(ai)+(-0.2,-0.2)$);

\end{tikzpicture}
}

\caption{Illustration of the different cases when mapping from the reference macroelement $\widehat{A}_{j( i) }$  to the domain $A_{i}$,  $G_{A_{i}}: \widehat{A}_{j( i) } \to A_{i}$. Here we have defined $\hat{a}_{j(i)} \in \widehat{A}_{j(i)}$ s.t. $G( \hat{a}_{j( i) })
= a_i$. }

\label{fig:macroelements}
\end{figure}


We want to compute the expected convergence rate of the energy norm \eqref{eq:bi_Ah_norm}. An important tool in the process is the Cléments interpolation operator, $C_{h}$.
It is used for interpolation on non smooth functions by applying an regularization on so-called macroelements. Let us denote the $\mathcal{P}_{c}^{k}( \Omega )  $ to be an $H^{1}$ conformal polynomial space. We denote $\left\{ a_{1}, \ldots, a_{N}
\right\} $ to be the Lagrange nodes. Associated with each node $a_{i}$ we denote the macroelement $A_{i}$ to consist of all simplices containing $a_{i}$. Let $n_{cf}$ be the number of configurations for the macroelement, then we define the index $j:
\left\{ 1,\ldots,N \right\} \to \left\{ 1, \ldots, n_{cf} \right\}  $ s.t. $j( i) $ is the index associated with the reference configuration $\widehat{A}_{j(i) }$ for corresponding macroelement $A_{i}$ for an illustration, see
Figure \ref{fig:macroelements}.

Let us define a $C^{0}$-diffeomorphism $G_{A_{i}}:
\widehat{A}_{j( i) } \to A_{i}$ s.t. for all $\widehat{T} \in \widehat{A}_{j( i) } $ is the restriction $G_{A_{i}  \mid \widehat{T}}$ affine. The Cléments interpolation operator $C_{h}$ is defined as a $L^2$-projection onto the macroelements. That is, given
a reference macroelement $\widehat{A}_{j( i) }$ and a function $\hat{v} \in L^{1}( \widehat{A}_{j( i) })  $, then $\widehat{C}_{j( i) } \hat{v}$  is the unique polynomial in $\mathcal{P}^{k} ( \widehat{A}_{j( i) })  $ s.t. \[
\int_{  \widehat{A}_{j( i) }}^{} ( \widehat{C}_{j( i) } \hat{v} - \hat{v}) p \ dx  = 0 \quad  \forall p \in \mathcal{P}^{k} ( \widehat{A}_{j( i) })
\]
Finally, we define the Cléments interpolator $C_{h} : L^{1}( \Omega )  \to \mathcal{P} ^{k}_{c}(\Omega  ) $ s.t.
\[
C_{h} v = \sum_{i=1}^{N} \widehat{C}_{j( i) } ( v (G_{A_{i}}) (G^{-1}_{A_{i}}(a_{i})) )\phi _{i},
\]
where $\phi _{i}$ is the corresponding polynomial basis at node $a_{i}$.

Recall the general Sobolev norm notation,
\[
\| u \|_{ m,p,T }^{  } = \left( \sum_{ \left\lvert \alpha  \right\rvert \le m}^{} \int_{T}^{}  \left\lvert  \partial ^{\alpha } u \right\rvert^{p} dx   \right)^{\frac{1}{2}}
\]
where we use the convenient notation $\| u \|_{L^2(T) }^{  } = \| u \|_{ T  }^{  } = \| u \|_{ 0,2,T  }^{  } $ and similarly $\| u \|_{ H^r( T )  }^{  } = \| u \|_{ r,T  }^{  } = \| u \|_{ r,2,T  }^{  }  $.


Finally, we have the following lemma

\begin{lemma}
    \label{lemma:clements}

We define the Clement interpolation as the projection
$C_{h}: H^{m} \left( \Omega  \right) \mapsto V_{h}$, where $V_{h}$ has the order $k$. Then does the following stability estimate hold,
\[
 \| C_{h} v \|_{H^{m}\left( \Omega  \right)   }^{  } \lesssim \| v \|_{ H^{m}\left( \Omega  \right)  }^{  } \quad \forall v \in H^{m}\left( \Omega  \right),
\]
and if the following conditions for an parameter $l$ is satisfied, it exists error estimates s.t.,
\[
    \begin{split}
      m\le l \le k+1  \implies \| v - C_{h} v \|_{ m,p,T   }^{  }  &  \lesssim h^{l-m}_{T} \| v \|_{l,p,\omega \left( T \right)  }^{  } \quad  \forall T \in \mathcal{T} _{h}, \forall v \in H^{l}( \omega \left( T \right)
      ), \\
      m +\frac{1}{2}\le l \le k+1  \implies \| v - C_{h} v \|_{ m,p,F }^{  } & \lesssim h^{l-m- \frac{1}{2}}_{T} \| v \|_{l,p,\omega \left( F \right)  }^{  } \quad  \forall \partial T \in \mathcal{T} _{h}, \forall v \in H^{l}( \omega \left( F
      \right)).
    \end{split}
\]

\end{lemma}


\begin{corollary}
    \label{cor:celement_apriori}
    Let $0 \le l \le k+1$ and let $0\le m \le \min_{} ( 1,l )$.
    Given Lemma \ref{lemma:clements}  then there exists an $C > 0$ s.t.
    \[
    \inf_{v_{h} \in \mathcal{P} ^{k}_{c}( \Omega ) } \| v - v_{h} \|_{  m,p,\Omega }^{  } \le C h^{l-m}  \| v \|_{ l,p,\Omega  }^{  }    \forall v \in W_{l,p}( \Omega ).
    \]
\end{corollary}
This result is very useful since it is now sufficient to show that a priori estimates holds given to prove convergence rate. For further detailed information about the Cléments interpolation, please investigate \cite[Chapter 1.6]{ern04}.
We will use these estimates to compute convergence rate given that Ceas' Lemma holds.


\subsubsection{Energy a priori estimates.  }%
\label{ssub:extension}
Recall that for $v \in H^{1}( \mathcal{T } _{h}) $ these inequalities holds $\forall T \in \mathcal{T} _{h}$ s.t. \[
\begin{split}
    \| v \|_{ \partial T }^{  } &\lesssim h^{-\frac{1}{2}}_{T}\|  v \|_{ T }^{  }+ h^{\frac{1}{2}} \| \nabla v \|_{T  }^{   }  , \\
    \| v \|_{ \Gamma \cap T }^{  } &\lesssim  h^{-\frac{1}{2}} \| v \|_{T  }^{  }   + h^{\frac{1}{2}}_{T} \| \nabla v \|_{ T }^{  }.
\end{split}
\]
For proof, see \cite[Lemma 4.2]{hansbo2003finite}.

A key idea is to utilize between the relationship between the physical space $\Omega $ and the active mesh $\mathcal{T}_{h}$. Assume that $\Omega $ has a boundary $\Gamma $ in $C^{1}$, then does it exist an bounded extension operator, \[
( \cdot ) ^{e}: W^{m,q}( \Omega )  \to W^{m,q} ( \Omega ^{e})
\]
for all  $v \in W^{m,q}( \Omega )$ where $0< m \le \infty$ and $1 \le q \le \infty$ which satisfies \[
    \begin{split}
 v^{e}| _{\Omega } =   v  \\
\| v^{e} \|_{ m,q,\Omega ^{e}  }^{  } & \lesssim \| v \|_{ m,q, \Omega  }^{  } \\
    \end{split}
\]
This extension theorem primarily utilizes \cite[Theorem 9.7]{brezis2011functional} by Brezis, with valuable context provided by \cite[p.181, p. 185]{stein1970singular}.
\begin{remark}
    Be aware that this theorem requires that we have a sufficiently smooth domain, hence, emphasizing the assumption of a sufficiently smooth boundary. Also keep in mind that the original theorem states the mapping $ ( \cdot ) ^{e}: W^{m,q}(
    \Omega )  \to W^{m,q} ( \mathbb{R} ^{d})$, but we restrict ourself to a slightly bigger extension $\Omega^{e} \supset   \Omega $.
    This is useful because we are now able to extend the function to the active set, while preserving the Sobolev regularity.
    \todo[inline]{ TODO: Not sure if this argument holds. At least from my point of view is this generalization is not trivial.}
\end{remark}
% \todo[inline]{ The physical space $\Omega $ is a subset of $\Omega ^{e}$, so I do not understand why $\| v^{e} \|_{ m,q,\Omega ^{e}  }^{  } \lesssim \| v \|_{ m,q, \Omega  }^{  } $ should hold.  I may also make a figure to illustrate $\Omega ^{e}$. }

Now construct the extended Sobolev space s.t. that $\Omega _{h}^{e} =\mathcal{T}_{h} \subset  \Omega^{e} $. We define an unfitted Cléments interpolator $C_{h}^{e}: H^{m}( \Omega ^{e}_{h}) \to V_{h}$
s.t.  $C ^{e} _{h} v := C _{h} v^{e} $.
We can immediately observe that the interpolation satisfies the global error estimates, that is,
\begin{align*}
    \| v - C _{h}^{e} v \|_{  m,2, \mathcal{T} _{h} }^{  } & \lesssim h^{l-m}\sum_{T \in \mathcal{T}_h} \| v \|_{ l,2, \omega(T) }^{  }, \quad m\le l\le k+1 \\
    \| v - C ^{e}_{h}v \|_{ m,2,\mathcal{F} _{h} }^{  } & \lesssim h^{l-m-\frac{1}{2}}\sum_{T \in \mathcal{T}_h} \| v \|_{ l,2, \omega(F)  }^{  }, \quad m+\frac{1}{2}  \le  l \le   k+1 \\
\| v - C ^{e}_{h}v \|_{ m,2, \Gamma }^{  } & \lesssim h^{l-m-\frac{1}{2}} \sum_{T \in \mathcal{T}_h}  \| v \|_{ l,2,  \omega(T)  }^{  }, \quad m+\frac{1}{2}  \le  l \le  k+1
\end{align*}
 and argue that $ \sum_{T}^{} \| v \|_{s,\omega ( T)   }^{  } \le C  \| v \|_{s, \mathcal{T}_{h}   }^{  } $ where $C$ is some constant decided by the maximum number of elements in a patch
    $\omega( T)  $ for all $T \in \mathcal{T} _{h}$. This also holds for the inequality $ \sum_{T}^{} \| v \|_{s,\omega ( F)   }^{  } \le C  \| v \|_{s, \mathcal{T}_{h}   }^{  } $. Hence, we have perhaps an even more useful set of inequalities.
\begin{align}
    \label{eq:bi_projection_estimates_1}
    \| v - C _{h}^{e} v \|_{  m,2, \mathcal{T} _{h} }^{  } & \lesssim h^{l-m}\sum_{T \in \mathcal{T}_h} \| v \|_{ l,2, \mathcal{T} _{h} }^{  }, \quad m\le l\le k+1 \\
    \label{eq:bi_projection_estimates_2}
    \| v - C ^{e}_{h}v \|_{ m,2,\mathcal{F} _{h} }^{  } & \lesssim h^{l-m-\frac{1}{2}}\sum_{T \in \mathcal{T}_h} \| v \|_{ l,2, \mathcal{T} _{h}  }^{  }, \quad m+\frac{1}{2}  \le  l \le   k+1 \\
    \label{eq:bi_projection_estimates_3}
\| v - C ^{e}_{h}v \|_{ m,2, \Gamma }^{  } & \lesssim h^{l-m-\frac{1}{2}} \sum_{T \in \mathcal{T}_h}  \| v \|_{ l,2,  \mathcal{T} _{h}  }^{  }, \quad m+\frac{1}{2}  \le  l \le  k+1
\end{align}
\todo[inline]{ Maybe hard to argue \eqref{eq:bi_projection_estimates_3} to hold on $\Gamma $, but may be related to some generalization of \eqref{eq:bi_n_cut_inverse_1} and \eqref{eq:bi_cut_inverse_1}. Anyhow, \eqref{eq:bi_projection_estimates_2} and
\eqref{eq:bi_projection_estimates_3} was never used in the proof of Lemma \ref{lemma:astar_estimate} since we used inverse estimates and ended up with \eqref{eq:bi_projection_estimates_1} instead on all of them.}
Naturally can we see this is the tools we need to construct an estimate for the energy norm.

\begin{lemma}
    \label{lemma:astar_estimate}
    Let $u \in H^{s}( \Omega ) $ for $s\ge 3$ be a exact solution. Then we have  \[
    \|  u - C_{h}u \|_{ a_{h},*  }^{  } \lesssim h^{s(s-2)} \| u \|_{ H^{s}( \Omega )  }^{  }
    \]

\end{lemma}
\begin{proof}
    By definition is
    \[
        \begin{split}
            \| u - C_{h}^{e}u \|_{ a_{h}, * }^{  2}  =& \ \| |\alpha |^{\frac{1}{2}} ( u - C_{h}^{e}u) \|_{ \mathcal{T} _{h} \cap \Omega  }^{ 2}  + \| D^2 ( u - C_{h}^{e}u ) \|_{\mathcal{T} _{h} \cap \Omega   }^{ 2 } \\  &  + \gamma \| h^{-\frac{1}{2}} \jump{ \partial _{n} (u -
        C_{h}^{e} u) }   \|_{ \mathcal{F}_{h}^{}\cap \Omega    }^{ 2
        } + \gamma \| h^{-\frac{1}{2}}  \partial _{n} (u - C_{h}^{e}u)    \|_{ \Gamma   }^{ 2 } \\
          & + \| h^{\frac{1}{2}} \mean{ \partial _{nn} (u - C_{h}^{e}u) }   \|_{\mathcal{F} _{h}^{} \cap \Omega   }^{  2} +  \| h^{\frac{1}{2}} \partial _{nn}(u - C_{h}^{e}u)     \|_{ \Gamma }^{  2}.
        \end{split}
    \]

    The strategy is to bound each term individually.
             Starting with the first two terms we get
    \[
        \begin{split}
            \| |\alpha |^{\frac{1}{2}} ( u - C_{h}^{e}u) \|_{ \mathcal{T} _{h} \cap \Omega  }^{ 2}    & \lesssim \|  ( u - C_{h}^{e}u) \|_{0,2,\mathcal{T} _{h}  }^{ 2} \\
             &    \lesssim  h^{2(s-0)}    \| u \|_{s,\mathcal{T}_{h} }^{  2}\lesssim  h^{2(s-2)}    \| u \|_{s,\mathcal{T}_{h} }^{  2} \\
            \| D^2 ( u - C_{h}^{e}u ) \|_{\mathcal{T} _{h} \cap \Omega   }^{ 2 } &  \lesssim  \|  u - C_{h}^{e}u  \|_{2,\mathcal{T} _{h}   }^{ 2 }   =  \|  u - C_{h}^{e}u  \|_{2,2,\mathcal{T} _{h}   }^{ 2 } \\
                                                                                 & \lesssim  h^{2(s-2)} \| u \|_{ s, \mathcal{T}_{h}}^{ 2 }.
        \end{split}
    \]
    Here we simply used \eqref{eq:bi_projection_estimates_1}.
    Recall the first order inverse estimate \eqref{eq:bi_n_cut_inverse_2} and that $\| \jump{ \partial _{n} u }   \|_{ \mathcal{F} _{h} }^{  } \le \| \partial _{n^{+}} u^{+}   \|_{ \mathcal{F} _{h} }^{  } +
    \|\partial _{n^{-}} u^{-}   \|_{ \mathcal{F} _{h} }^{  } \lesssim  \|\partial _{n} u \|_{ \partial\mathcal{T }_{h}  }^{2  }  $, hence, this implies $\| \jump{ \partial _{n} u }   \|_{   \mathcal{F}_{h}\cap \Omega    }^{2} \lesssim h^{-1} \| \nabla u \|_{ \mathcal{T}
        _{h} }^{2  }  $. Using this and the inequality \eqref{eq:bi_projection_estimates_1} we therefore can observe, \[
        \begin{split}
            \gamma \| h^{-\frac{1}{2}} \jump{ \partial _{n} ( u - C_{h}^{e}u ) }   \|_{ \mathcal{F}_{h} \cap \Omega   }^{  2}   & \lesssim
            h^{-1} \|   \partial _{n} ( u - C_{h}^{e}u )    \|_{  \partial \mathcal{T} _{h} }^{2  } \lesssim  h^{-2} \|   \nabla  ( u - C_{h}^{e}u )    \|_{ \mathcal{T} _{h}\cap \Omega  }^{2  }\\
&=  h^{-2} \|    u - C_{h}^{e}u     \|_{ 1,2,\mathcal{T} _{h} }^{2  } \lesssim  h^{ -2 } h^{ 2(s - 1)  } \| u \|_{ s, \mathcal{T}_{h}   }^{  2} \\
& \lesssim   h^{ 2(s - 2)  } \| u \|_{ s, \mathcal{T}_{h}   }^{  2} \\
        \end{split}
        .
    \]
    And for the boundary term we do a similar procedure, but instead use the first order inverse estimate \eqref{eq:bi_n_cut_inverse_1}. \[
        \begin{split}
            \gamma \| h^{-\frac{1}{2}}  \partial _{n} ( u - C_{h}^{e}u ) \|_{ \Gamma    }^{  2} & \lesssim h^{-2} \|   \nabla  ( u - C_{h}^{e}u )    \|_{ \mathcal{T}_{h} }^{2  }  \lesssim  h^{-2} \|    u - C_{h}^{e}u     \|_{1,2, \mathcal{T}_{h}   }^{2  } \\
            & \lesssim h^{-2} h^{2(s-1)}   \| u \|_{s, \mathcal{T}_{h}   }^{  2}  \lesssim  h^{2(s- 2)}   \| u \|_{s, \mathcal{T}_{h}   }^{  2}
        \end{split}
    \]
            Recall that $\| \mean{ u }   \|_{ \mathcal{F} _{h} }^{  } \le \| u^{+} \|_{ \mathcal{F} _{h}  }^{  } + \| u^{-} \|_{ \mathcal{F} _{h}  }^{  }   \lesssim  \| u \|_{ \partial\mathcal{T }_{h}  }^{2  }  $ and the second order inverse
            inequality \eqref{eq:bi_cut_inverse_2}. It is clear
            by using \eqref{eq:bi_projection_estimates_1} that this holds.
            \[
                \begin{split}
 \| h^{\frac{1}{2}} \mean{ \partial _{nn} (u - C_{h}^{e}u) }   \|_{\mathcal{F} _{h}^{} \cap \Omega   }^{  2} &  \lesssim h^{} \|   \partial _{nn} (u - C_{h}^{e}u)    \|_{\partial \mathcal{T} _{h} \cap \Omega    }^{  2}  \lesssim h^{} h^{-1}  \|   D^2 (u - C_{h}^{e}u)    \|_{ \mathcal{T} _{h}   }^{  2} \\
                                                                                                                &  = \|   u - C_{h}^{e}u    \|_{ 2, \mathcal{T} _{h}   }^{  2}  \lesssim h^{2(s - 2)}  \| u \|_{s, \mathcal{T} _{h} }^{  }
                \end{split}
            \]
                Similarly we can easily see by using \eqref{eq:bi_projection_estimates_1} and the second order boundary inverse inequality  \eqref{eq:bi_cut_inverse_1} that this must hold,

                \[
              \| h^{\frac{1}{2}} \partial _{nn}(u - C_{h}^{e}u)     \|_{ \Gamma }^{  2} \lesssim   \|  D^2(u - C_{h}^{e}u)     \|_{ \mathcal{T}_{h}   }^{  2} \lesssim \|  u - C_{h}^{e}u \|_{ 2, \mathcal{T}_{h} }^{2  } \lesssim h^{2(s-2)}  \| u \|_{ s, \mathcal{T}_{h}   }^{2  }
            \]

    Thus, all elements is bounded by $ h^{2(s-2)}  \| u \|_{ s, \mathcal{T}_{h}   }^{2  } $  and the proof is complete.
\end{proof}

\begin{lemma}[Weak galerkin orthogonality]
Let $u \in H^{s}( \Omega )  $, $ s\ge 3 $  be the exact solution and $u_{h} \in V_{h}$ is a discrete solution to \eqref{eq:discrete_CutCIP_prob}. Then is \[
    a_{h}( u - u_{h}, v) = g_{h} ( u_{h}, v) \quad \forall v \in V_{h}.
    \]
\end{lemma}

\begin{proof}
   From the definition of the problem \eqref{eq:discrete_CutCIP_prob}, utilizing that $a_{h}( u,v) = l(v ) \forall v \in V_{h} $ can we easily observe that \[
       \begin{split}
   l(v ) & =  A_{h}( u_{h},v) =  a_{h}( u,v)  = a_{h}( u_{h},v)+g_{h}( u_{h},v)
       \end{split}
   \]
    Hence, it is clear that $a_{h}( u -  u_{h}, v) = g_{h}( u_{h},v)  $.
\end{proof}

\begin{assumption}[EP2]
    \label{as:bi_EP2}
    For $v \in H^{s}( \Omega ) $ and $r = \min \{s,k+1 \} $, the semi-norm $\abs{ \cdot  }_{g_{h}} $ satisfies the following estimate, \[
    \abs{ C _{h}^{e} v } _{g_{h}} \lesssim  h^{r-1} \| v \|_{ r,\Omega  }^{  }.
    \]
\end{assumption}


\begin{theorem}
    \label{thm:apriori_result}
    Let $u \in H^{s}( \Omega ) $ , $s\ge 3$ a solution to \eqref{eq:bi_weak} and let $u \in V_{h}$ of order $k\ge 2$ be the discrete solution to \eqref{eq:cip_unfitted_hessian_form}. Then for $r = \min_{}\{s, k+1\} $ the error $e = u - u_{h}$ satisfies
    \begin{align}
        \label{eq:bi_apriori_1}
            \| e \|_{ a_{h},* }^{  } &\lesssim   h^{r-1} \| u \|_{ r,\Omega  }^{  }\\
        \label{eq:bi_apriori_2}
            \| e \|_{ \Omega  }^{  } &\lesssim   h^{r} \| u \|_{ r,\Omega  }^{  }
    \end{align}

\end{theorem}

\begin{proof}
    We will divide the proof into two steps.
    \begin{enumerate}[label=\arabic*)]
        \item We want to prove that $\| e \|_{ a_{h},* }^{  } \lesssim   h^{r-1} \| u \|_{ r,\Omega  }^{  }$.
    Let $e = u - u_{h}$ consist of $e = e_{h} + e_{\pi }$, where the discrete error has the form $e_{h} = C _{h}^{e} u - u_{h}$ and the interpolation error $e_{\pi } = u - C _{h} ^{e}u$. We can then observe that
    \[
        \begin{split}
    \| u - u_{h} \|_{ a_{h},* }^{  } &= \| u - C_{h}^{e} u + C_{h}^{e}u - u_{h} \|_{ a_{h},* }^{  } \\
    & \le \|  u - C_{h}^{e} u \|_{a_{h},*  }^{  } +  \| C_{h}^{e}u - u_{h} \|_{a_{h},*  }^{  }\\
                                     & \le \| e_{\pi } \|_{a_{h},*}^{  } + \| e_{h} \|_{A_{h},*  }^{  }
        \end{split}
    \]
    \red{Using Lemma \ref{lemma:astar_estimate}, can we see that $\| e_{\pi } \|_{a_{h},*}^{  } \lesssim h^{r-1} \| u \|_{ r,\Omega  }^{  }  $ is already fulfilled.} So it remains to check the discrete part. From Lemma \ref{lemma:bi_Ah_coercive}, \ref{lemma:bi_Ah_bounded}, the weak Galerkin orthogonality and Assumption \ref{as:bi_EP2} is it natural to see that, \[
    \begin{split}
\| e_{h} \|_{ A_{h},* }^{ 2 } & \lesssim a_{h}( C _{h}^{e} u - u_{h}, e_{h}) + g_{h}( C _{h}^{e}u - u_{h}, e_{h}) \\
 & = a_{h}( C _{h}^{e} u - u, e_{h}) + a_{h}( u - u_{h}, e_{h}) + g_{h}( C _{h}^{e}u - u_{h}, e_{h}) \\
 & = a_{h}( C _{h}^{e} u - u, e_{h}) + g_{h}( C _{h}^{e}u, e_{h}) \\
 & \lesssim h^{r-1} \| u \|_{ r, \Omega  }^{  } \| e_{h} \|_{ A_{h} }^{  }.
    \end{split}
\]



The last line of the calculations above comes from the fact that
\[
    \begin{split}
        a_{h}( C _{h}^{e} u - u, e_{h}) + g_{h}( C _{h}^{e}u, e_{h}) &\lesssim \| C _{h}^{e} u - u \|_{a_{h},*  }^{  } \| e_{h} \|_{a_{h},*  }^{  }
        + \abs{ C _{h}^{e}u }_{g_{h}} \abs{e_{h}  }_{g_{h}} \\
         &\lesssim \| C _{h}^{e} u - u \|_{a_{h},*  }^{  } \| e_{h} \|_{a_{h},*  }^{  } + h^{r-1} \| e_{h} \|_{r, \Omega   }^{  }\abs{e_{h}  }_{g_{h}} \\
         &\lesssim (\| C _{h}^{e} u - u \|_{a_{h},*  }^{  } + h^{r-1} \| e_{h} \|_{r, \Omega   }^{  }) \|e_{h}\|_{A_{h}} \\
         &\lesssim  h^{r-1} \| u \|_{r, \Omega   }^{  } \|e_{h}\|_{A_{h}}.
    \end{split}
\]
Here we noticed that $\| e_{h} \|_{a_{h},*  }^{  } + \abs{e_{h}  }_{g_{h}} \lesssim \| e_{h} \|_{ A_{h} }^{  }  $. We also argued that $\| C _{h}^{e} u - u \|_{a_{h},*  }^{  } \lesssim h^{r-1}\| u \|_{ r,\Omega  }^{  }  $ from Lemma
\ref{lemma:astar_estimate}.
\todo[inline]{ TODO: Need to show $ \| e_{h} \|_{a_{h},*  }^{  } + |e_{h}  |_{g_{h}} \le  \| e_{h} \|_{ A_{h} }^{  }  $ }
Hence, the first part of the proof is complete.

    \item We want to show that $ \| e \|_{ \Omega  }^{  } \lesssim   h^{r} \| u \|_{ r,\Omega  }^{  }$. The idea is to apply the so-called Aubin-Nitsche duality trick by being aware of the ghost penalty $g_{h}$. Let us denote the following observation.
    Because of the Assumption \ref{as:EP2} is there a function $\phi \in H^2( \Omega ) \cap H^{1}_{0}( \Omega ) $ and a $\psi \in L^2( \Omega )  $ s.t.
    $$-\Delta \phi = \psi  \text{ and }  \| \phi  \|_{ 2, \Omega  }^{  } \lesssim \| \psi  \|_{ \Omega  }^{  }.$$

    Because of the regularity $\phi $ can we write $ (e, \psi  )_{\Omega } = (e, -\Delta \phi )_{\Omega } = a_{h}(e, \phi )  $. Naturally, we can now seek to bound this operator. That is,
    \[
    \begin{split}
        (e, \psi  )_{\Omega } &= a_{h}( e, \phi ) \\
        &= a_{h}(e, \phi -  C_{h}^{e} \phi   ) + g_{h}(u_{h} - C_{h}^{e} u,   C^{e}_{h} \phi   )  +g_{h}(C^{e} _{h} u, C_{h}^{e} \phi   ) \\
        & \lesssim h^{r} \| u \|_{ r, \Omega  }^{  } \| \psi  \|_{ \Omega  }^{  }
    \end{split}
    \]

    In the first line we used the weak orthogonality, i.e.,
    \[
        \begin{split}
    a_{h}( e, \phi ) & = a_{h}( e, \phi  - C^{e} _{h} \phi ) + a_{h}( u - u_{h}, C^{e} _{h} \phi ) \\
     & = a_{h}( e, \phi  - C^{e} _{h} \phi ) + g_{h}(  u_{h}, C^{e} _{h} \phi ) \\
     & = a_{h}( e, \phi  - C^{e} _{h} \phi ) + g_{h}(  u_{h} - C ^{e}_{h}u, C^{e} _{h} \phi )+ g_{h}( C ^{e}_{h}u, C^{e} _{h} \phi ).
        \end{split}
    \]

    The last inequality appears after applying Cauchy-Schwartz and Assumption \ref{as:EP2}.
\[
    \begin{split}
        a_{h}( e, \phi  - C _{h} \phi ) &+ g_{h}(  u_{h} - C ^{e}_{h}u, C^{e} _{h} \phi )+ g_{h}( C ^{e}_{h}u, C _{h} \phi ) \\
                                          &\lesssim \| e \|_{ a_{h},*  }^{  } \| \phi - C _{h}^{e} \phi   \|_{a_{h},*  }^{  }   +
     \abs{\pi _{h}^{e} u - u_{h} }_{ g_{h}}^{  } \abs{ C^{e} _{h} \phi  }_{g_{h}}+ \abs{ C^{e} _{h} u } _{g_{h}} \abs{ C^{e} _{h} \phi  }_{g_{h}} \\
.
    \end{split}
\]

Using the result from the first part and the Assumption \ref{as:EP2}.
\[
    \begin{split}
        \| e \|_{ a_{h},*  }^{  } &\lesssim h^{r-1} \| u \|_{ r, \Omega  }^{  } \\
        \abs{ C^{e} _{h} u } _{g_{h}} &\lesssim h^{r-1}\| u \|_{2,\Omega   }^{  } \\
        \abs{ C^{e} _{h} u - u_{h} } _{g_{h}} &\lesssim \|  C^{e} _{h} u - u_{h} \|_{ A_{h}  }^{  }  \\
        \abs{ C^{e} _{h} \phi  } _{g_{h}} &\lesssim  \ldots \\
        \|\phi - C^{e} _{h} \phi  \|_{a_{h},*  }^{  }  &\lesssim  \ldots
    \end{split}
\]
    \todo[inline]{ TODO: Finish proof.  }

    \end{enumerate}
\end{proof}


% \todo[inline]{
% We define big-oh $O( h^{r})$ to be a upper bound s.t. it exists an $C>0$ s.t.  $\| e \|_{  }^{  } \le M h^{r} \text{ for all } h$, where $r$ is the order of convergence. Since we only will implement $k=1$.
% Hence, we have the construct the following Corollary.


% \begin{corollary}[Order of convergence]
%     Let $u \in H^{s}( \Omega ) $ , $s\ge 3$ a solution and let $u \in V_{h}$ of order $k = 2$ be the solution to \eqref{eq:Bi_a_h}. Then for $r = \min_{}\{s, k+1\} $ the error $e = u - u_{h}$ is bounded s.t . $ \| e \|_{ a_{h},* }^{  }$ is in $O( h)
%     $ and $\| e \|_{ L^{2} }^{  } $ is in $O( h^{2}) $. Or in other words, it exists and $M>0$
%     \begin{align}
%             \| e \|_{ a_{h},* }^{  } & \le    M h^{r-1}  \\
%             \| e \|_{ \Omega  }^{  } & \le    M h^{r}
%     \end{align}
% \end{corollary}
% \begin{proof}
%     It easily comes from the results $ \| u \|_{ r,\Omega  }^{  } = \| u \|_{r, 2,\Omega  }^{  } \le \| u \|_{r, \infty,\Omega  }^{  } = M $, where $M$ is a constant. The rest follows from the Theorem \ref{thm:apriori_result}.
% \end{proof}
% }
