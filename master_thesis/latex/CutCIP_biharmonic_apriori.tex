
\subsection{A priori error estimate}%
\label{sec:a_priori_estimates}


For the proposed method, we want to derive a priori error estimate with respect to both the  $\| \cdot  \|_{a_{h},*   }^{  } $-norm and the  $\| \cdot  \|_{ \Omega  }^{
} $-norm.  These estimates are geometrically robust in that they remain unaffected by specific cut configurations, thanks to the ghost penalty they incorporate.
First, we construct a suitable (quasi-)interpolation operator, here we use the Clement quasi interpolation operator which in contrast to the standard Lagrange nodal interpolation operator is also defined for low regularity function $u \in L^{2}(
\Omega ) $.
In combination with discrete coercivity this allows us to derive an a priori error estimate in the energy norm. Finally, we use a standard duality argument, also known as Aubin-Nitsche trick, to derive the $L^{2}( \Omega ) $-error estimate.

Recall that for $v \in H^{3}( \mathcal{T } _{h}) $ the following inequalities.
\begin{align}
    \label{eq:trace:1}
    \| \nabla v \|_{ \partial T }^{  } &\lesssim h^{-\frac{1}{2}}_{T}\|  \nabla v \|_{ T }^{  }+ h^{\frac{1}{2}} \| D^2 v \|_{T  }^{   }  , \\
    \label{eq:trace:2}
    \| \nabla v \|_{ \Gamma \cap T }^{  } &\lesssim  h^{-\frac{1}{2}} \| \nabla v \|_{T  }^{  }   + h^{\frac{1}{2}}_{T} \| D^2 v \|_{ T }^{  },\\
    \label{eq:trace:3}
    \| D^2 v \|_{ \Gamma \cap T }^{  } &\lesssim  h^{-\frac{1}{2}} \| D^2 v \|_{T  }^{  }   + h^{\frac{1}{2}}_{T} \| D^3 v \|_{ T }^{  },
\end{align}
holds $\forall T \in \mathcal{T} _{h}$, for proof see \cite[Lemma 4.2]{hansbo2003finite}.
In this context is $D^3v$ a tensor of third partial derivates, $\left[ D^3v \right] _{ijk} = \frac{\partial^3 v}{\partial x_i \partial x_j \partial x_k} \forall i,j,k \in \left\{ 1,\ldots,d \right\} $ where the norm $\| D^3 v\|_{\Omega   }^{ 2 } =
\int_{\Omega }^{} D^3 v : D^3 v \ dx   $ is defined via the standard Frobenius inner product.


Assume that $\Omega $ has a boundary $\Gamma $ in $C^{1}$, then there exists a bounded extension operator,
\begin{equation}
    ( \cdot ) ^{e}: H^{m}( \Omega )  \to H^{m} ( \mathbb{R} ^{d}),
\end{equation}
for all  $v \in H^{m}( \Omega )$ which satisfies
\begin{equation}
    \begin{split}
 v^{e}| _{\Omega } =   v,  \\
\| v^{e} \|_{ m,\mathbb{R} ^{d}  }^{  } & \lesssim \| v \|_{ m, \Omega  }^{  }.
    \end{split}
\end{equation}
For more information, see \cite[Theorem 9.7]{brezis2011functional} and \cite[p.181, p.185]{stein1970singular}. For the notation we simply write $ v := v^{e}   $ for $v \in \mathbb{R} ^{d} \setminus \Omega $.


Starting from Lemma \ref{lemma:clements}, assume $v \in H^{s}( \Omega ) $ and let $r = \mathrm{min}(s , k+1) $. Revisit the definition of $V_{h}$ from \eqref{eq:vh_energy}, which is a polynomial of degree $k$. We can then employ the combination of the Clément interpolator with the extension operator to create $C_{h}^{e}: H^{m}( \mathbb{R} ^{d}) \to V_{h}$, such that $C ^{e} _{h} v := C _{h} v^{e}$.
 Next, recall that $ \sum_{T}^{} \| v \|_{s,\omega ( T)   }^{  } \le C  \| v \|_{s, \mathcal{T}_{h}   }^{  } $ where $C$ is some constant decided by shape regularity of the mesh and the maximal number of different patches a single element can
 belong to. This also holds for the inequality $ \sum_{T}^{} \| v \|_{s,\omega ( F)   }^{  } \le C  \| v \|_{s, \mathcal{T}_{h}   }^{  } $.
The following estimates are thereby established.

\begin{align}
    \label{eq:bi_projection_estimates_1}
    \| v - C _{h}^{e} v \|_{  l, \mathcal{T} _{h} }^{  } & \lesssim h^{r-l}\sum_{T \in \mathcal{T}_h} \| v \|_{ r, \omega(T) }^{  } \lesssim  h^{r-l-\frac{1}{2}}  \| v \|_{ r, \Omega  }^{  }, \quad 0\le l\le r, \\
    \label{eq:bi_projection_estimates_2}
\| v - C ^{e}_{h}v \|_{ l, \partial \mathcal{T} _{h} }^{  } & \lesssim h^{r-l-\frac{1}{2}}\sum_{T \in \mathcal{T}_h} \sum_{F \in \partial T} \| v \|_{ r, \omega(F)  }^{  } \lesssim h^{r-l-\frac{1}{2}} \| v \|_{ r, \Omega   }^{  }, \quad 0  \le  l \le   r-\frac{1}{2}.
    % \label{eq:bi_projection_estimates_3}
    % \| v - C ^{e}_{h}v \|_{ l, \Gamma }^{  } & \lesssim h^{r-l-\frac{1}{2}} \sum_{T \in \mathcal{T}_h}  \| v \|_{ r,  \omega(T)  }^{  }\lesssim h^{r-l-\frac{1}{2}} \lesssim    \| v \|_{ r,  \Omega   }^{  }, \quad 0  \le  l \le  r-\frac{1}{2}.
\end{align}

% \todo[inline]{ Maybe hard to argue \eqref{eq:bi_projection_estimates_3} to hold on $\Gamma $, but may be related to some generalization of \eqref{eq:bi_n_cut_inverse_1} and \eqref{eq:bi_cut_inverse_1}. Anyhow, \eqref{eq:bi_projection_estimates_2} and
% \eqref{eq:bi_projection_estimates_3} was never used in the proof of Lemma \ref{lemma:astar_estimate} since we used inverse estimates and ended up with \eqref{eq:bi_projection_estimates_1} instead on all of them.}

\begin{lemma}
    \label{lemma:astar_estimate}
    Let $u \in H^{s}( \Omega ) $ for $s\ge 3$ be the exact solution to $\eqref{eq:cont_weak_problem} $ and let $k$ be the polynomial order of $V_{h}$. Set $r = \mathrm{min} ( s, k+1)$, then we have the interpolation estimates
    \begin{equation}
    \|  u - C_{h}u \|_{ a_{h},*  }^{  } \lesssim h^{r-2} \| u \|_{ r, \Omega  }^{  }.
    \end{equation}
\end{lemma}

\begin{proof}
    By definition,
    \begin{equation}
        \begin{split}
            \| u - C_{h}^{e}u \|_{ a_{h}, * }^{  2}  =& \ \alpha  \overbrace{\|  ( u - C_{h}^{e}u) \|_{ \mathcal{T} _{h} \cap \Omega  }^{ 2}}^{\mathrm{I} }   + \overbrace{\| D^2 ( u - C_{h}^{e}u ) \|_{\mathcal{T} _{h} \cap \Omega   }^{ 2
            }}^{\mathrm{II} }  \\  &  +
            \overbrace{\gamma \| h^{-\frac{1}{2}} \jump{ \partial _{n} (u -
        C_{h}^{e} u) }   \|_{ \mathcal{F}_{h}^{}\cap \Omega    }^{ 2
        }}^{\mathrm{III} }  + \overbrace{\gamma \| h^{-\frac{1}{2}}  \partial _{n} (u - C_{h}^{e}u)    \|_{ \Gamma   }^{ 2 }}^{\mathrm{IV} }  \\
          & + \overbrace{\| h^{\frac{1}{2}} \mean{ \partial _{nn} (u - C_{h}^{e}u) }   \|_{\mathcal{F} _{h}^{} \cap \Omega   }^{  2}}^{\mathrm{V} }  +  \overbrace{\| h^{\frac{1}{2}} \partial _{nn}(u - C_{h}^{e}u)     \|_{ \Gamma }^{  2}}^{\mathrm{VI}
          } \\
          =& \  \mathrm{I}  + \ldots + \mathrm{VI}.
        \end{split}
    \end{equation}
    The strategy is to bound each term individually.
    By initially focusing on the first two terms and employing equation \eqref{eq:bi_projection_estimates_1}, we can easily observe
             \begin{equation}
        \begin{split}
            \mathrm{I} +\mathrm{II}  & \lesssim \|   u - C_{h}^{e}u \|_{ \mathcal{T} _{h}  }^{ 2} + \|  D^2( u - C_{h}^{e}u )  \|_{\mathcal{T} _{h} }^{ 2 } \\
                                     & \lesssim  ( h^{2r}  + h^{2(r-2)} )\| u \|_{r,\mathcal{T}_{h} }^{  2} \lesssim h^{2(r -2)} \| u
                                     \|_{r, \mathcal{T} _{h} }^{2  }  .
        \end{split}
             \end{equation}
    From \eqref{eq:mean_jump_estimate} is it clear that $\| \jump{ \partial _{n} u }   \|_{ \mathcal{F}_{h}   }^{  } \lesssim \| \nabla  u \|_{ \partial  \mathcal{T} _{h} }^{  }   $. Hence, first applying the trace inequality \eqref{eq:trace:1}  and then
    \eqref{eq:bi_projection_estimates_1} is it clear that,
    \begin{equation}
        \begin{split}
            \mathrm{III}  & \lesssim h^{-1} \|  \nabla ( u - C_{h}^{e})  \|_{\partial \mathcal{T}_{h}   }^{2  }  \lesssim h^{-2} \| \nabla ( u - C^{e}_{h}u)  \|_{ \mathcal{T} _{h}
                          }^{ 2 } + \|  D^2  ( u - C_{h}^{e})  \|_{\mathcal{T}_{h}   }^{ 2 } \\
                          & \lesssim  ( h ^{2( r-1) -2 } + h^{2( r-2) } ) \| u  \|_{r, \mathcal{T}_{h}   }^{2  }  \lesssim  h^{2( r-2) }  \| u  \|_{r, \mathcal{T}_{h}   }^{ 2 }
    \end{split}
\end{equation}
% And for the boundary term we apply estimate \eqref{eq:bi_projection_estimates_3}
%         \begin{equation}
%             \mathrm{IV}   \lesssim h^{-1} \|  \nabla  ( u - C_{h}^{e}u ) \|_{ \Gamma  }^{2  }  \lesssim  h^{2( r - 2 )} \| u \|_{r, \mathcal{T}_{h}   }^{  }
%         \end{equation}
And for the boundary term we apply \eqref{eq:trace:2} and then \eqref{eq:bi_projection_estimates_1}
        \begin{equation}
            \begin{split}
            \mathrm{IV}   & \lesssim h^{-1} \|  \nabla  ( u - C_{h}^{e}u ) \|_{ \Gamma  }^{2  }    \lesssim h^{-2} \| \nabla ( u - C_{h}^{e}u )  \|_{ \mathcal{T}_{h}   }^{2  } + \| D^2( u - C_{h}^{e}u ) \|_{ \mathcal{T}_{h}   }^{ 2 } \\   & \lesssim  h^{2( r-2) }  \| u  \|_{r, \mathcal{T}_{h}   }^{ 2 }
            \end{split}
        \end{equation}
Again, from \eqref{eq:mean_jump_estimate} is it clear that $\| \mean{ \partial _{nn} u }   \|_{ \mathcal{F}_{h}   }^{  } \lesssim \| D^2  u \|_{ \partial  \mathcal{T} _{h} }^{  }   $, thus we see that,
        \begin{equation}
            \mathrm{V}   \lesssim h \|  D^2  ( u - C_{h}^{e}u ) \|_{\partial \mathcal{T}_{h}}^{2  }  \lesssim  h^{2( r - 2 )} \| u \|_{r, \mathcal{T}_{h}   }^{ 2 }.
        \end{equation}
        The final term we we apply \eqref{eq:trace:3} and then \eqref{eq:bi_projection_estimates_1}

        \begin{equation}
            \begin{split}
            \mathrm{VI} &    \lesssim h \|  D^2  ( u - C_{h}^{e}u ) \|_{\Gamma }^{2  }  \\
            & \lesssim  h^{-2} \| D^2 ( u - C_{h}^{e}u )  \|_{ \mathcal{T}_{h}   }^{2  } + \| D^3( u - C_{h}^{e}u ) \|_{ \mathcal{T}_{h}   }^{ 2 } \\
            % & \lesssim h^{-2}( \| u- C_{h}^{e} \|_{2,\mathcal{T}_{h}   }^{2  } + \| u- C_{h}^{e} \|_{3,\mathcal{T}_{h}   }^{2  }  )\\
            &  \lesssim h^{-2}( h^{2(r - 2 - \frac{1}{2})} + h^{2(r - 3 - \frac{1}{2})})\| u \|_{r,\Omega   }^{  } \\
            &  \lesssim ( h^{2r - 5)} + h^{2r - 6)})\| u \|_{r,\Omega   }^{  } \lesssim h^{2(r-2)} \| u \|_{r,\Omega   }^{  }
            \end{split}
        \end{equation}
    Hence, we have $ \| u - C_{h}^{e} u \|_{a_{h},*  }^{  } \lesssim   h^{r-2}  \| u \|_{ r, \mathcal{T}_{h}   }^{2  } $.
\end{proof}

\begin{lemma}[Weak Galerkin orthogonality]
Let $u \in H^{s}( \Omega )  $, $ s\ge 3 $  be the exact solution to   \eqref{eq:cont_weak_problem} and $u_{h} \in V_{h}$ is a discrete solution to \eqref{eq:discrete_CutCIP_prob}. Then is \[
    a_{h}( u - u_{h}, v_{h}) = g_{h} ( u_{h}, v_{h}) \quad \forall v_{h} \in V_{h}.
    \]
\end{lemma}

\begin{proof}
   From the definition of the problem \eqref{eq:discrete_CutCIP_prob} and utilizing that for $u \in H^{s}( \Omega ) $ we have the identity  $A_{h}( u,v_{h}) = a_{h}( u,v_{h}) = l(v_{h} )  \ \forall v_{h} \in V_{h} $. Consequently, it follows that \[
       \begin{split}
   l(v_{h} ) & =  A_{h}( u_{h},v_{h}) =  a_{h}( u,v_{h})  = a_{h}( u_{h},v_{h})+g_{h}( u_{h},v_{h})  \quad \forall  v_{h} \in  V_{h}.
       \end{split}
   \]
    Hence, we have $a_{h}( u -  u_{h}, v_{h}) = g_{h}( u_{h},v_{h})  $.
\end{proof}

\begin{assumption*}[EP2]
    For $v \in H^{s}( \Omega ) $ and $r = \min \{s,k+1 \} $, the semi-norm $\abs{ \ \cdot \  }_{g_{h}} $ is weakly consistent in the sense that
    \begin{equation}
        \label{as:bi_EP2}
        \abs{ C _{h}^{e} v } _{g_{h}} \lesssim  h^{r-2} \| v \|_{ r,\Omega  }^{  }.
    \end{equation}
\end{assumption*}

\begin{theorem}
    \label{thm:apriori_result}
    Let $u \in H^{s}( \Omega ) $ , $s\ge 3$ be a solution to \eqref{eq:cont_weak_problem} and let $u_{h} \in V_{h}$ of order $k\ge 2$ be the discrete solution to \eqref{eq:discrete_CutCIP_prob}. Then with $r = \min_{}\{s, k+1\} $ the error $e = u - u_{h}$ satisfies
    \begin{align}
        \label{eq:bi_apriori_1}
            \| e \|_{ a_{h},* }^{  } &\lesssim   h^{r-2} \| u \|_{ r,\Omega  }^{  }\\
        \label{eq:bi_apriori_2}
        \| e \|_{ \Omega  }^{  } &\lesssim   h^{r-\mathrm{max}\left\{ 0, 3-k \right\} } \| u \|_{ r,\Omega  }^{  }
    \end{align}

\end{theorem}
\begin{remark}
    Be aware that for $k=2$ the estimate \eqref{eq:bi_apriori_2} is suboptimal with $1$ order.
\end{remark}

\begin{proof}
    We will divide the proof into two steps.
    \\
        \textbf{Step 1.} We want to prove that $\| e \|_{ a_{h},* }^{  } \lesssim   h^{r-2} \| u \|_{ r,\Omega  }^{  }$.
    Decompose $e = u - u_{h}$ intro $e = e_{h} + e_{\pi }$, where we denote the discrete error $e_{h} = C _{h}^{e} u - u_{h}$ and the interpolation error $e_{\pi } = u - C _{h} ^{e}u$. We can then observe that
    \begin{equation}
        \begin{split}
    \| u - u_{h} \|_{ a_{h} }^{  } & \le   \| u - C_{h}^{e} u + C_{h}^{e}u - u_{h} \|_{ a_{h},* }^{  } \\
    & \le \|  u - C_{h}^{e} u \|_{a_{h},*  }^{  } +  \| C_{h}^{e}u - u_{h} \|_{a_{h},*  }^{  }\\
                                     & \lesssim  \| e_{\pi } \|_{a_{h},*}^{  } + \| e_{h} \|_{A_{h}  }^{  }
        \end{split}
    \end{equation}
    Using Lemma \ref{lemma:astar_estimate}, is it clear that $\| e_{\pi } \|_{a_{h},*}^{  } \lesssim h^{r-2} \| u \|_{ r,\Omega  }^{  }  $ is already fulfilled, hence, it remains to estimate $e_{h}$. From Lemma \ref{lemma:bi_Ah_coercive} and
    \ref{lemma:bi_Ah_bounded}, the weak Galerkin orthogonality and Assumption EP2 \eqref{as:bi_EP2} is it natural to arrive at,
    \begin{equation}
        \label{eq:apriori_energy1}
    \begin{split}
\| e_{h} \|_{ A_{h} }^{ 2 } & \lesssim a_{h}( C _{h}^{e} u - u_{h}, e_{h}) + g_{h}( C _{h}^{e}u - u_{h}, e_{h}) \\
 & = a_{h}( C _{h}^{e} u - u, e_{h}) + a_{h}( u - u_{h}, e_{h}) + g_{h}( C _{h}^{e}u - u_{h}, e_{h}) \\
 & = a_{h}( C _{h}^{e} u - u, e_{h}) + g_{h}( C _{h}^{e}u, e_{h}) \\
 % & \lesssim h^{r-2} \| u \|_{ r, \Omega  }^{  } \| e_{h} \|_{ A_{h} }^{  }.
    \end{split}
    \end{equation}
Hence, now utilizing the Assumption EP2 \eqref{as:bi_EP2} is it clear that
\begin{equation}
        \label{eq:apriori_energy2}
    \begin{split}
        a_{h}( C _{h}^{e} u - u, e_{h}) + g_{h}( C _{h}^{e}u, e_{h}) &\lesssim \| C _{h}^{e} u - u \|_{a_{h},*  }^{  } \| e_{h} \|_{a_{h}  }^{  }
        + \abs{ C _{h}^{e}u }_{g_{h}} \abs{e_{h}  }_{g_{h}} \\
         &\lesssim \| C _{h}^{e} u - u \|_{a_{h},*  }^{  } \| e_{h} \|_{a_{h}  }^{  } + h^{r-2} \| e_{h} \|_{r, \Omega   }^{  }\abs{e_{h}  }_{g_{h}} \\
         &\lesssim (\| C _{h}^{e} u - u \|_{a_{h},*  }^{  } + h^{r-2} \| e_{h} \|_{r, \Omega   }^{  }) \|e_{h}\|_{A_{h}} \\
         &\lesssim  h^{r-2} \| u \|_{r, \Omega   }^{  } \|e_{h}\|_{A_{h}}.
    \end{split}
\end{equation}
Here we noticed that $\| e_{h} \|_{a_{h}  }^{  } + \abs{e_{h}  }_{g_{h}} \lesssim \| e_{h} \|_{ A_{h} }^{  }  $, and used that $\| C _{h}^{e} u - u \|_{a_{h},*  }^{  } \lesssim h^{r-2}\| u \|_{ r,\Omega  }^{  }  $ from Lemma
\ref{lemma:astar_estimate}.

Finally, combining \eqref{eq:apriori_energy1} and \eqref{eq:apriori_energy2} is it clear that $\| e_{h} \|_{ A_{h}  }^{  } \lesssim h^{r-2} \| u \|_{r, \Omega   }^{  }  $.
Hence, the first part of the proof is complete.

        \textbf{Step 2.}
        We want to show that $ \| e \|_{ \Omega  }^{  } \lesssim   h^{r- \mathrm{max}(0,3-k)} \| u \|_{ r ,\Omega  }^{  }$. The idea is to apply the so-called Aubin-Nitsche duality trick while being aware of the ghost penalty $g_{h}$. Let us denote the following
        observation.
        Assume that $e:= u -u_{h} \in L^{2}( \Omega ) $ and $\psi  \in H^{4}( \Omega ) $.
        Let the corresponding dual problem to \eqref{eq:bi_problem} be
        \begin{equation}
            \begin{split}
            \Delta ^2 \psi &= e  \quad  \text{ in } \Omega  \\
            \partial _{n} \psi &= 0 \quad \text{ on } \Gamma \\
            \partial _{n} \Delta \psi & = 0 \quad  \text{ on } \Gamma   \\
            \end{split}
        \end{equation}

        This implies that it exists a $\psi \in H^{4}( \Omega ) $ such that $a_{h}(v, \psi ) = ( e,v)_{\Omega } \ \forall v \in V_{h}  $. Hence, we can easily observe that \begin{equation}
            \label{eq:ni_1}
            \begin{split}
        \| e \|_{ \Omega  }^{ 2 }  & = ( e,e)_\Omega   = ( e, \Delta ^2 \psi )_{\Omega } \\
        &= a_{h}(  \psi, e ) = a_{h}( u-u_h, \psi ) \\
        &= a_{h}( u-u_h, \psi + C^{e}_{h}\psi  - C^{e}_{h}\psi )  \\
        &= a_{h}( u-u_h, \psi   - C^{e}_{h}\psi ) +  a_{h}( u-u_h, C^{e}_{h}\psi )  \\
        &= a_{h}( u-u_h, \psi  - C^{e}_{h}\psi )  \\
        & \lesssim    \underbrace{\|u-u_{h}  \|_{a_{h},*  }^{  }}_{\mathrm{I} }   \underbrace{\| \psi  - C^{e}_{h}\psi \|_{a_{h},*  }^{  }}_{\mathrm{II} }
            \end{split}
        \end{equation}

        Now, for $\mathrm{I} $ we simply use the energy a priori estimate
        \begin{equation}
            \label{eq:ni_2}
            \mathrm{I} \lesssim h^{r-2} \| u \|_{ r,\Omega   }^{  }
        \end{equation}
        However, to estimate $\mathrm{II} $ we set $\widetilde{r} = \mathrm{min} ( 4, k+1)  $, where $4$ comes from the regularity $\psi \in H^{4}( \Omega ) $ and $\| \psi \|_{4,\Omega   }^{  } \lesssim \| e \|_{ \Omega   }^{  }  $, thus,
        \begin{equation}
            \label{eq:ni_3}
            \mathrm{II} \lesssim h^{\widetilde{r} -2} \| \psi \|_{ \widetilde{r},\Omega   }^{  } \lesssim h^{\widetilde{r} -2} \| e \|_{ \Omega   }^{  }.
        \end{equation}
        Hence, combining \eqref{eq:ni_1}, \eqref{eq:ni_2} and \eqref{eq:ni_3}  can we conclude \begin{equation}
            \| e \|_{\Omega   }^{  } \lesssim h^{r+ \widetilde{r} -4} \| u \|_{ r,\Omega   }^{  }
        \end{equation}


        Having a clear look at $\widetilde{r}$, wee see that \begin{equation}
            \widetilde{r} = \mathrm{min}(4, k+1) = \begin{cases}
                & 3, \quad  k=2 \\
                & 4, \quad  k\ge 3 \\
            \end{cases}
        \end{equation}

        So we have the following estimate,\begin{equation}
            \| e \|_{ \Omega  }^{  }  \lesssim  \| u \|_{ r,\Omega  }^{  }  \begin{cases}
                h^{r-1}, \quad k=2 \\
                h^{r-2}, \quad k\ge 3
            \end{cases}
        \end{equation}
        or equivalently $\| e \|_{\Omega   }^{  } \lesssim \| u \|_{\Omega   }^{r- \mathrm{max}(0,k-3)   } $.


\end{proof}


