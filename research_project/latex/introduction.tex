\section{Introduction}\label{sec:introduction}


Cell membrane dynamics has recently had many important applications. For instance, it has been linked to detecting deceases such as Alzheimer's disease, cancer cells and is as well important for development new methods and vaccines
\cite{small2006sorting}.
\todo[inline]{ Add more sources. }
One of the primary components of the cell membranes are lipids which serve many different functions. A key function is that it is consisting of a bilayer of lipids which controls the structural rigidity and the fluidity of the membrane \cite{ neidleman87}. It also turns out that the lipids often accumulate into so-called lipid rafts which serves as a rigid platform for proteins with special properties such as intracellular trafficking of lipids and lipid-anchored proteins \cite{Edidin03}.

Modelling of lipid rafts formation can be modelled as a two-phase separation problem based on minimization of the Ginzburg-Landau energy functional \cite{yushutin19}
\[
\mathcal{E}_{ch}  \left( \Gamma  \right) = \int_{\Gamma  }^{}\Psi \left( c \right) + \frac{\gamma}{2} \left\lvert \nabla c \right\rvert^{2} ,
\]
which is describing the chemical energy for a concentration $c: \Gamma \times \left[ 0,T \right] \mapsto  \left[ 0,1 \right]  $ over a surface membrane $\Gamma$. Several authors have solved this problem often results by deriving variants of Cahn
Hilliard Equation or Allen Cahn Equation if the concentration is not conserved both standstill and evolving domains \cite{yushutin19, udo97, ratz16,Gera2017, caetano21,yushutin19} .

Assuming that the system is a single-phase system can the elastic bending energy be modelled using the Canham Helrich energy functional \cite{wang08, udo97} \[
\mathcal{E} _{e}\left( \Gamma  \right) =   \int_{\Gamma }^{}  c_{b} H^{2} + c_{k} K
\].

Here is $H =  \kappa_1 + \kappa_2 $ denoted as the mean curvature and $K = \kappa_1 \kappa_2$ as the gaussian curvature with respectively $c_{b}$ and $c_{k}$ as tuning parameters and $\kappa_1$ and $\kappa_2$ as principal curvatures. Using the Gauss-Bonnet theorem can it be shown that the problem above is equivalent to the so-called Willmore energy
functional \cite{montiel2009curves, willmore1996riemannian}

\begin{equation}
\label{eq:WE}
\mathcal{E} \left( \Gamma  \right) = \int_{\Gamma }^{} \frac{1}{2} H ^2
.\end{equation}
This is a well known problem in the mathematical community \cite{ topping2000towards, marques2014willmore,link2013gradient}. In fact, it is a mathematical tool used to study the geometry of surfaces because it can be used to study the properties of minimal surfaces, which are surfaces with the least possible area for a given boundary. This is important in many areas of mathematics, including differential geometry, topology and mathematical physics \cite{koerber2021area,jakob2022singularities, rupp21}.



It has been established many numerical methods for for shape optimization problems \cite{sokolowski1992introduction,ito2008variational}, evolving surface partial differential equations (PDE) \cite{dziuk2013finite, dziuk2007finite,
binz2022convergent, barrett2007parametric, barrett2007variational, kovacs2019convergent} and specific
algorithms for the Willmore energy problem \eqref{eq:WE} \cite{palmurella2022parametric, dziuk2008computational, bonito2010parametric,  kovacs2021convergent, hu2022evolving}.

In this report will we establish a minimization of the energy functional \eqref{eq:WE}. However, we will first establish notation by including a section for definitions and important results from differential geometry. We will then derive the
underlying dynamics system of evolutionary surface PDE's \cite{dougan2012first} using the gradient flow technique inspired by shape optimization methods \cite{ito2008variational}. Lastly we will establish the model for the problem and discretize the
PDE by applying a proposed evolutionary surface FEM method \cite{kovacs2021convergent}.



