

\section{Evolutionary dynamics of the Willmore energy}%
\label{sec:evolutionary_pde_s_of_the_willmore_flow}

Recall that we define the velocity \eqref{eq:vel}to only have a normal component, i.e., $v\left( x \right)  = V \nu $.
The shape derivative for \eqref{eq:WE} in the direction of some velocity $v \in \left[ H^{1}\left( \Gamma  \right)  \right]^3   $ has the form \[
d\mathcal{E} \left( \Gamma; v  \right)  = \int_{\Gamma }^{} \left( - \Delta _{\Gamma } H + \frac{1}{2} H^{3} - H \left\lvert A \right\rvert^2  \right) V
\]
A complete derivation of the shape derivative can be found in \cite[Corally 4.7]{dougan2012first}. Consequently, by applying the gradient flow in \eqref{eq:gradient_flow} and using that $\| v \|_{ \Gamma  }^{  }  = \| V \|_{ \Gamma   }^{2  }   $ can
we easily see that,
\[
      \| V \|_{ \Gamma   }^{2  } = \int_{\Gamma }^{} V^2 = -d \mathcal{E} \left( \Gamma ;v \right).
\]
Hence, the gradient flow is equivalent to

\begin{equation}
\label{eq:gradient_velocity}
V  =   \Delta _{\Gamma } H + Q,
\end{equation}
\[
\]
where we denote the nonlinear term as $Q  = - \frac{1}{2} H^{3} + H \left\lvert A \right\rvert^2$.
\todo[inline]{ Need to define a way to compute $\Delta _{\Gamma } H $ in the background theory. }

From \cite[Lemma 2.1]{kovacs2021convergent} it derived that \eqref{eq:gradient_velocity} must satisfy the following material derivatives,
\[
    \begin{split}
\frac{D}{Dt}H & = - \left( \Delta _{\Gamma } + \left\lvert A \right\rvert ^2   \right) V, \\
\frac{D}{Dt} \nu & = \left( -\Delta _{\Gamma } + \left( HA - A^2 \right)  \right) z  \\
&  + \left\lvert \nabla _{\Gamma } H \right\rvert ^2   \nu - 2\left( \nabla _{\Gamma }\cdot \left( A \nabla _{\Gamma } H \right)  \right) \nu  \\
  & -A^2 \nabla _{\Gamma } H  - \nabla _{\Gamma } Q.
    \end{split}
\]
Here is the substitution variable introduced s.t. $z = \Delta  _{\Gamma } \nu  + \left\lvert A \right\rvert ^2 \nu $.

It exists methods which do not exploit the material derivatives \cite{bonito2010parametric, bartezzaghi2016isogeometric}.However, it turns out that including these material derivatives brings additional computational costs, but provides so-called full-order
approximation to the mean curvature, $H$, and the normal vector, $\nu $, and ,thus, allows us to construct rigorous convergence proofs for evolving surface FEM, see \cite{kovacs2021convergent, binz2022convergent}.

Recent work has proposed that current method may not conserve the mesh quality while the surface is restricted to evolve along the normal velocity. Thus, a new variation of the standard methods has been considered by introducing an tangential
velocity component via the equation  $H = - v\cdot \nu  $. Hence, also allowing the mesh to be less deformed, see more at \cite{hu2022evolving}. However, in this report will we not consider it.

Finally, we end up with the following second-order evolutionary system of PDE.
\begin{subequations}
    \begin{align}
\frac{d}{dt}H & = - \left( \Delta _{\Gamma } + \left\lvert A \right\rvert ^2   \right) V, \\
V  & =   \Delta _{\Gamma } H + Q, \\
\frac{d}{dt} \nu & = \left( -\Delta _{\Gamma } + \left( HA - A^2 \right)  \right) z \nonumber \\
& \quad   + \left\lvert \nabla _{\Gamma } H \right\rvert ^2   \nu - 2\left( \nabla _{\Gamma }\cdot \left( A \nabla _{\Gamma } H \right)  \right) \nu \nonumber  \\
  & \quad  -A^2 \nabla _{\Gamma } H  - \nabla _{\Gamma } Q, \\
z & = \Delta  _{\Gamma } \nu  + \left\lvert A \right\rvert ^2 \nu \\
\frac{d}{dt} x  &= v
    \end{align}
\end{subequations}
where the terms are
\[
    \begin{split}
Q & =  - \frac{1}{2} H^{3} + H \left\lvert A \right\rvert^2 \\
A & = \nabla _{\Gamma } \nu \\
v & = V\nu
    \end{split}
\]
Recall that the material derivative operator $\frac{D}{Dt} $ is simply equivalent to the time derivative $\frac{d}{dt}$, thus, it is now possible to apply normal techniques for time discretization.









