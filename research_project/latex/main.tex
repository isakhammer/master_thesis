

% \documentclass[12pt]{article}
% \documentclass[10.8pt, a4paper, USenglish, twocolumn]{article}

% https://cloudnet2016.ieee-cloudnet.org/authors/final-submission/
\documentclass[10pt,conference]{IEEEtran}



\usepackage{isaks_template} % Contains all included packages. See isaks_template.sty.

% latex margins
% \linespread{1.5}
\newgeometry{vmargin={15mm}, hmargin={25mm,37mm}}
%
\title{ {\Large \textbf{Mathematical modelling of cell membranes }} }
% IF ONE AUTHOR
%\address{Norwegian University of Science and Technology \\
%Department of Mathematical Sciences \\
%{\tt isakhammer@gmail.com}}
%

\author{Isak Hammer\footnote{isakhammer@gmail.com} \\{\small Supervisor: André Massing\footnote{Department of Mathematical Sciences, NTNU}} }
\date{\today}
\begin{document}

% Comment this out to remove todos
% https://tex.stackexchange.com/questions/4830/how-to-hide-todo-notes-without-deleting-them-manually
% \renewcommand{\todo}[1]{}
\maketitle
\begin{sloppy}
\textit{ \textbf{Note.} This article is submitted as an examination in the course TK8115 Numerical Optimal Control for the autumn semester 2022 at the Department of Engineering Cybernetics, NTNU. \\}


\begin{abstract}
This article aims to show the latest methods of mathematical modelling of biological cell membranes. We first presented a literature review on recent research about incorporating multi-physics processes into mathematical models. We then presented a mathematical and numerical shape optimization framework using a gradient flow method and a finite element method to solve cell membrane dynamics specifically for the elastic bending energy on evolving surfaces.
\end{abstract}

    \section{Introduction}\label{sec:introduction}

Cell membranes are the foundation of the origin of life, but also linked to the dynamics of virus al infections and genetic mutations since it controls what substances that can exit or enter the cell \cite{ hurley2010membrane}. In fact, a good
understanding of the cell membrane is important of engineering proteins to manipulate various intracellular processes in living systems \cite{rojas1998genetic}.

One of the primary components of the cell membranes are lipids which serve many different functions. A key function is that it is consisting of a bilayer of lipids which controls the structural rigidity and the fluidity of the membrane. Thus, elastic
bending forces, temperature and diffusion is essential on how a cell membrane will evolve \cite{udo97,neidleman87}.


\subsection{Elastic bending energy on evolving surfaces}%
\label{sub:willmore_flow}

Assuming that the system is a single-phase system, i.e., the lipids are uniformly distributed, can the elastic bending energy be modelled using the Canham-Helrich energy functional \cite{helfrich1973elastic, wang08, udo97}. Let us denote $b_{b},
b_{k}$ and $H_{0}$ as parameters based on physical models, then can the energy functional be denoted as,
\begin{equation}
\label{eq:CH}
\mathcal{E} _{CH}\left( \Gamma\left( t \right)   \right) =   \int_{\Gamma  }^{}  b_{b} \left( H- H_{0} \right) ^{2} + b_{k} K
.\end{equation}
Here is $H =  \kappa_1 + \kappa_2 $ denoted as the mean curvature and $K = \kappa_1 \kappa_2$ as the gaussian curvature with respectively and $\kappa_1$ and $\kappa_2$ as principal curvatures. $\Gamma \left( t
\right) = \Gamma  $ is here an evolving surface in $\mathbb{R} ^3$, for more info see section \ref{sec:background}.  Using the Gauss-Bonnet theorem can it be shown that the problem above is equivalent to the so-called Willmore energy
functional \cite{montiel2009curves, willmore1996riemannian},

\begin{equation}
\label{eq:WE}
\mathcal{E} _{W} \left( \Gamma\left( t \right)   \right) = \int_{\Gamma  }^{} \frac{1}{2} H ^2
.\end{equation}

This is a well known problem in the mathematical community \cite{ topping2000towards, marques2014willmore,link2013gradient,kuwert2012willmore}. In fact, it is a mathematical tool used to study the geometry of surfaces because it can be used to study
the diffeomorphism from a initial surface to a minimal energy configuration, which are surfaces with the least possible area for a given boundary. This is important in many areas of mathematics, including differential geometry, topology and mathematical physics \cite{koerber2021area,jakob2022singularities, rupp21}.

It has been established many numerical methods for for shape optimization problems \cite{sokolowski1992introduction,ito2008variational}, evolving surface partial differential equations (PDE) \cite{dziuk2013finite, dziuk2007finite,
binz2022convergent, barrett2007parametric, barrett2007variational, kovacs2019convergent, lehrenfeld2018stabilized} and specific
algorithms for the Willmore energy problem \eqref{eq:WE} \cite{palmurella2022parametric, dziuk2008computational, bonito2010parametric,  kovacs2021convergent, hu2022evolving}.

\subsection{Two-phase separation modelling on predefined evolving surfaces }%
\label{sub:two_phase_seperation_modelling_on_surfaces_}

It also turns out that the lipids often accumulate into so-called lipid rafts which serves as a rigid platform for proteins with special properties such as intracellular trafficking of lipids and lipid-anchored proteins \cite{ miller2020divide}. Modelling of
lipid rafts formation can be modelled as a two-phase separation problem based on minimization of the Ginzburg-Landau energy functional \cite{yushutin19},
\begin{equation}
\label{eq:GL}
\mathcal{E}_{GL}  \left( c   \right) = \int_{\Gamma\left(t  \right)   }^{}\Psi \left( c \right) + \frac{\gamma}{2} \left\lvert \nabla c \right\rvert^{2} ,
\end{equation}
which describes the chemical energy for a concentration $c: \Gamma\left( t \right)  \times \left[ 0,T \right] \mapsto  \left[ 0,1 \right]  $. Here is $ \Psi \left( c \right): \mathbb{R} \mapsto \mathbb{R} $ denoted as a nonlinear scalar chemical potential function. Keep in mind that unlike
the Willmore energy functional \eqref{eq:WE}, where the $\Gamma\left( t \right)  $ is determined by the elastic properties, should the energy functional \eqref{eq:GL} be interpreted as a chemical diffusion problem a predefined evolving domain $\Gamma \left( t \right) $.
Usually is this problem solved by deriving equivalent variants of partial differential equations (PDE) such as Allen-Cahn equation (or Cahn-Hilliard equation if the total concentration is globally conserved) on evolving domains. For further details,
see \cite{yushutin19,
udo97, ratz16,Gera2017, caetano21, elliott2015evolving}.

\subsection{Multiphysics problems on evolving surfaces}%

Ultimately will the cell membrane consists of interaction several kinds of physics (temperature, elasticity, chemical diffusion, internal fluid pressure etc.) \cite{udo97}. Hence, being able to model several processes may give unforeseen results.

An interesting example is to couple the energy functionals \eqref{eq:GL}  and \eqref{eq:CH}, since the lipid-rafts formation is said to change the elasticity properties of the membrane, may be a good model for how cell membranes evolve to specific shapes or execute cell division. One way
to couple the energy functionals is to let the parameters $b_{b}, b_{k}$ and $ H_{0} $ be some function of the time dependent concentration $c$, i.e.,
\[
    \begin{split}
        \mathcal{E}_{CHGL} \left( \Gamma\left( t \right) ,c\left( t \right)    \right) =  & \int_{\Gamma  }^{}  b_{b}\left( c \right)  \left( H- H_{0}\left( c \right)  \right) ^{2}  \\
        & + \int_{\Gamma   }^{} b_{k}\left( c \right)  K \\
        &+ \int_{\Gamma   }^{}\Psi \left( c \right) + \frac{\gamma}{2} \left\lvert \nabla c \right\rvert^{2} ,
    \end{split}
\]
For more information, see \cite{elliott2010surface}.

Recently have some authors also coupled diffusion processes and the so-called mean curvature energy, see \cite{burger2021interaction, elliott2022numerical}. It is well known that lipids travels
along the cell membrane in a fluidic manner, hence, it is also of interest to couple the Ginzburg-Landau energy functional \eqref{eq:GL} (or more specifically the Cahn-Hilliard equation) with the Navier-Stokes equation. Some methods has been proposed
methods for solving the problem on surfaces
and evolving surfaces, but it remains a field of active research \cite{olshanskii2022comparison}. As far as a author knows, coupling the Canham-Helrich energy functional \eqref{eq:CH}, Ginzburg-Landau energy functional \eqref{eq:GL}  and Navier-Stokes equation remains a open problem.

Some physical processes may require constant area and volume. This can simply be added by introducing respectively area and volume functionals, see \cite[Definition 2.5]{muller2013volume}.

Until now have all the models assumed that the membrane has no difference in internal and external pressure. As a matter of fact, osmotic pressure can be introduced by adding a energy functional using the van't Hoff formula. Let $V_{p}$ be the volume
of a closed evolving surface $\Gamma \left( t \right) $, we can then model the difference of internal and external pressure as,
\[
\Delta P \left( V_{p} \right) = P_{in} - P_{out} = iRT\left( \frac{n}{V_{p}} - \overline{c}  \right),
\]
where $i, R, T, \overline{c} $ and $n$ are the van't Hoff index, ideal gass constant, temperature , ambient molar concentration and molar amount of the enclosed solute. Then the energy
functional have the form,
\[
\mathcal{E} _{p}\left( \Gamma    \right)  = \int_{\Gamma   }^{   } \Delta P\left( V_{p} \right) ,
\]
For more information, see \cite{zhu2022mem3dg}.


\subsection{Outline of this report}%
\label{sub:outline_of_this_report}

The long-term goal would be to solve the multi-physics problems above. However, many of the problems above is fairly complicated to solve numerically and requires sophisticated techniques. Hence, in this report we focus on the latest research withing
the numerical methods of finding the minima of the energy functional \eqref{eq:WE}. However, we will first establish notation by including a section for definitions and important results from differential geometry and shape derivatives. We will then derive the
underlying dynamics system of evolutionary system dynamics using the gradient flow technique inspired by shape optimization methods based on the work done in \cite{ dougan2012first}. Lastly, we will establish the numerical methods of the system
dynamics by applying recent methods using an evolutionary surface finite element method (FEM) \cite{kovacs2021convergent, hu2022evolving}.



    

\section{Background Theory}%
\label{sec:background}


\subsection{Differential Calculus}%
\label{sub:differential_calculus}

This subsection is inspired by the notation used in \cite{kovacs2021convergent, dougan2012first}.
Let some initial surface $\Gamma^{0} \subset \mathbb{R} ^3  $ smooth compact and oriented surface with no boundary where we can assign a unique point $p \in \Gamma ^{0}$. We define define the time evolutionary surface to be on the form,
\[
    \begin{split}
\Gamma  = \Gamma \left( t \right) & = \Gamma \left( X \left(\cdot ,t \right)  \right) \\
                                  &= \left\{ X \left( p,t \right): \ p \in \Gamma^{0}  \right\}
    \end{split}
\]
transformed via the smooth mapping,
\[
X : \Gamma^{0} \times  \left[ 0,T \right]  \mapsto  \mathbb{R} ^3.
\]
An important regularity result is that if $\Gamma ^{0}$ is of class $C^{\infty}$, then $\Gamma $ is also of class $C^{\infty}$ for $\forall t \in \left[ 0,T \right] $ \cite{sokolowski1992introduction, dougan2012first}.
\todo[inline]{ Formally in \cite[p 48]{sokolowski1992introduction}, it might be an idea to formulate normal unit-vector regularity as $C^{\infty}$ }

We will define a unique evolutionary point $x \in \Gamma \left( t \right) $ based on the smooth mapping $X \left( p,t \right) = x$. A way to imagine this is to have a initial point in $\Gamma ^{0}$ and the mapping $X $ describes how this point will deform over time. The outer unit normal vector field of $\Gamma \left( t \right) $ is defined as the mapping $\nu : \Gamma \mapsto
\mathbb{R} ^{3}$.

Using the notation presented in \cite{dougan2012first} and \cite{kovacs2021convergent} can we define the basic surface differential operators. Consider a scalar function, $u: \Gamma \mapsto \mathbb{R} $, and a vector-valued function, $\hat{u}: \Gamma  \mapsto \mathbb{R} ^3$. We can then denote $ \nabla _{\Gamma } u: \Gamma \mapsto \mathbb{R} ^{3}$ as the tangential operator,
$$
\nabla_{\Gamma
} u  = \nabla u - \left<\nu, \nabla u \right> \nu.
$$
\todo[inline]{May be an idea to define a extension $\widetilde{u} \mid _{\Gamma }$ and look into regularity. See definitions in \cite{dziuk2013finite}.  }
Similarly, for the vector-valued function is the operator defined s.t.
$$\nabla_{\Gamma } \hat{u} = \left( \nabla_{\Gamma } u_{1},\nabla_{\Gamma } u_{2},\nabla_{\Gamma } u_{3}   \right)^{T}.$$ The surface divergence for a vector-valued function is defined as \[
\nabla_{\Gamma } \cdot \hat{u} = \nabla  \cdot \hat{u} - \nu^{T} D \hat{u} \cdot \nu
\]
Here $D\hat{u}$ denotes the Jacobian of $\hat{u}$. Similarly, the Laplace-Beltrami operator $\Delta _{\Gamma }u  : \Gamma \mapsto \mathbb{R}$  is proven to have the form \cite[Lemma 1]{xu2003eulerian},
\begin{equation*}
    \begin{split}
 \Delta _{\Gamma } u  & = \nabla _{\Gamma } \cdot  \nabla _{\Gamma }u \\
 &=  \Delta u  - \nu ^{T} D^2 u \cdot \nu - H \partial _{\nu } u
    \end{split}
.\end{equation*}
Here is $D^2u$ denotes as the Hessian of the scalar function $u$. In the case of a vector valued function is the operator defined as \[
\Delta _{\Gamma } \hat{u} = \left( \Delta _{\Gamma } u_{1}, \Delta _{\Gamma } u_{2}, \Delta _{\Gamma } u_{3} \right)^{T}
\]
A method to compute the mean curvature and the so-called Frobenius norm of matrix $A$ involves applying the
extended Weingarten map, $ A\left( x \right) = \nabla_{\Gamma } \nu \left( x \right) $, s.t. these identities holds ,
\begin{equation*}
    \begin{split}
    H & = tr(A) = k_{1} + k_{2}, \\
    \left\lvert A \right\rvert^{2}  & = k_{1}^2 + k_{2}^2,
    \end{split}
.\end{equation*}
see \cite{kovacs2021convergent}.
We may also want to use these definitions to introduce the following identities ,
\[
    \begin{split}
         \partial _{\nu } H & = - \left\lvert A \right\rvert ^{2}, \\
    \nabla _{\Gamma } H & = \Delta  _{\Gamma } \nu  + \left\lvert A \right\rvert ^2 \nu.  \\
    \end{split}
\]
Again, see Lemma 3.3 and Lemma 3.2 in \cite{dougan2012first}.

\subsection{Evolutionary Surface Dynamics}%
\label{sub:evolutionary_equations}

In this section will we develop a framework evolutionary surface dynamics.

First of all, we can denote the velocity $v: \Gamma \mapsto \mathbb{R} ^3$ to be
\begin{equation}
    \label{eq:vel}
\frac{dx }{ d t}  = v\left( x \right) \quad \forall x \in \Gamma \left( t \right) .
\end{equation}
Given a model of the velocity $v$ can we solve the ordinary differential equation (ODE) \eqref{eq:vel} and determine the evolution of a point on the surface $\Gamma\left( t \right)  $. In this article will we assume that the velocity only has a
normal component to the surface, i.e., it exists a scalar function $V: \Gamma \mapsto \mathbb{R} $ s.t. $v = V \nu  $.

% Define shape derivatives
Recall that the point $x = X \left( p,t \right)  $ is arising from the smooth mapping from the point $p $ in  $\Gamma ^{0} $ to $\Gamma \left( t \right) $. Now, let some arbitrary energy functional have the form,
\[
\mathcal{J}\left( \Gamma\left( t \right)   \right)  = \int_{\Gamma\left( t \right)  }^{} \varphi \left( x  \right) .
\]
For instance, in the case presented in \eqref{eq:WE} we define $\varphi = H ^2$.

Based on \cite{dougan2012first},the shape derivative of this energy functional at some time $t$ in the direction of the velocity $v\left( x \right) $ from \eqref{eq:vel} is defined as \[
d \mathcal{J} \left( \Gamma \left( t \right) ; v  \right) = \lim_{\varepsilon  \to 0} \frac{\mathcal{J}  \left( \Gamma \left( t + \varepsilon  \right) \right)  - \mathcal{J}\left( \Gamma \left( t  \right) \right)      }{\varepsilon }.
\]
For a more detailed description of the shape derivative, see \cite[Definition 2.19]{sokolowski1992introduction}.

Assume we have a scalar function $f: \Gamma\left( t \right)  \mapsto \mathbb{R}  $. Similarly, as for the shape derivative, we can now denote the material derivative at time $t$ in the direction of the velocity $v\left( x \right) $ as
\[
    \begin{split}
\frac{D}{Dt}  f\left( x,t; v \right)  & = \frac{d}{dt} f \left( X \left( p,t \right) , t \right) \\
&= \lim_{\varepsilon \to 0}  \frac{f \left( X \left( p, t + \varepsilon  \right)  \right) - f \left( X \left( p, t  \right)  \right) }{ \varepsilon },
    \end{split}
\]
see \cite[Definition 2.74]{sokolowski1992introduction}.


We denote the $L^2\left( \Gamma  \right)  $ as the space of all functions that are square-integrable with respect to the surface measure, i.e., \[
   L^{2}\left( \Gamma   \right)  = \left\{ u: \Gamma \mapsto \mathbb{R}  \mid  \int_{\Gamma }^{}    \left\lvert u \right\rvert ^{2} < \infty   \right\} \\
\]
Let $u,v \in L^{2}\left( \Gamma  \right) $, then can we define the norm and the inner-product \[
    \begin{split}
        \| u \|_{ L^{2}\left( \Gamma  \right)  }^{ 2 } & = \int_{\Gamma }^{} \left\lvert u \right\rvert ^2 \\
        \left( u, v \right)_{L^2\left( \Gamma  \right) } &= \int_{\Gamma }^{} uv  \\
    \end{split}
\]
In this paper will we also the shorthand notation $\| u \|_{ L^2\left( \Gamma  \right)  }^{  }  = \| u \|_{ \Gamma  }^{  } $ and $\left( u,v \right)_{L^2\left( \Gamma  \right) } = \left( u,v \right) _{\Gamma } $.
The Sobolev space $H^1\left( \Gamma  \right) $ is defined as the space of all functions and its first weak derivative with a finite $L^{2}$-norm, i.e,
\[
H^{1}\left( \Gamma  \right) = \left\{ f: \Gamma \mapsto \mathbb{R}   \mid  \int_{\Gamma }^{}
\left\lvert f \right\rvert ^2  + \left\lvert \nabla_\Gamma  f \right\rvert ^2 < \infty \right\},
\]
with the following norm and inner product $u,v \in H^{1}\left( \Gamma  \right) $,
\[
    \begin{split}
        \| u \|_{ H ^{1}\left( \Gamma  \right)  }^{  }  & = \| u \|_{ \Gamma  }^{  }  + \| \nabla_ \Gamma u \|_{ \Gamma  }^{  },  \\
        \left( u, v \right)_{H^1\left( \Gamma  \right) } &= \left( u,v \right) _{\Gamma } + \left( \nabla_\Gamma u, \nabla _{\Gamma } v  \right) _{\Gamma }.   \\
    \end{split}
\]

If we have a vector-valued function that $u: \Gamma  \mapsto  \mathbb{R} ^{3} $ where each element is in $H^1\left( \Gamma  \right) $ or $L^2\left( \Gamma  \right) $, then do we denote is as a member of respectively $\left[ H^{1}\left( \Gamma
\right)   \right]^3 $ or $\left[ L^2\left( \Gamma  \right)  \right] ^3$.




The method we will use in this paper to minimize the energy functional \eqref{eq:WE} is to compute the so-called gradient flow. The fundamental idea of the gradient flow is to give rise of evolutionary dynamics to decrease the overall energy
functional both in space and time, i.e., $\mathcal{J}\left( \Gamma \left( t_{2} \right)  \right) <   \mathcal{J}\left( \Gamma \left( t_{1} \right)\right)$ for all  $t_{2} > t_{1}$. For more information about gradient flows, see
    \cite{dogan2007discrete, dogan2005finite}. Now assume we have the velocity defined in \eqref{eq:vel} to be $v \in \left[ L^{2}\left( \Gamma  \right)  \right]^3 $, then we define the $L^2$  gradient flow s.t. \[
      \left( v,\varphi  \right) _{\Gamma  }  = - d \mathcal{J} \left( \Gamma ; \varphi  \right) \  \forall \varphi \in \left[ L^2\left( \Gamma  \right)   \right] ^3.
    \]
    It turns out that if $v \neq 0$, then is this equivalent to
    \begin{equation}
    \label{eq:gradient_flow}
d \mathcal{J} \left( \Gamma ; v \right) = -\| v \|_{ L^2\left( \Gamma  \right)  }^{^2  } < 0.
    .\end{equation}
see \cite{ito2008variational}.
Hence, we finally have a toolbox which can be used to model evolutionary dynamics for moving surfaces.












    

\section{Evolutionary dynamics of the Willmore flow}%
\label{sec:evolutionary_pde_s_of_the_willmore_flow}


The goal is to derive the evolutionary dynamics of the Willmore energy $\eqref{eq:WE} $.

Recall that we define the velocity \eqref{eq:vel}to only have a normal component, i.e., $v\left( x \right)  = V \nu $.
The shape derivative for \eqref{eq:WE} in the direction of some velocity $v \in \left[ H^{1}\left( \Gamma  \right)  \right]^3   $ has the form \[
d\mathcal{E} \left( \Gamma; v  \right)  = \int_{\Gamma }^{} \left( - \Delta _{\Gamma } H + \frac{1}{2} H^{3} - H \left\lvert A \right\rvert^2  \right) V
\]
A complete derivation of the shape derivative can be found in \cite[Corally 4.7]{dougan2012first}. Consequently, by applying the gradient flow in \eqref{eq:gradient_flow} and using that $\| v \|_{ \Gamma  }^{  }  = \| V \|_{ \Gamma   }^{2  }   $ can
we easily see that,
\[
      \| V \|_{ \Gamma   }^{2  } = \int_{\Gamma }^{} V^2 = -d \mathcal{E} \left( \Gamma ;v \right).
\]
Hence, the gradient flow is equivalent to

\begin{equation}
\label{eq:gradient_velocity}
V  =   \Delta _{\Gamma } H + Q,
\end{equation}
\[
\]
where we denote the nonlinear term as $Q  = - \frac{1}{2} H^{3} + H \left\lvert A \right\rvert^2$.
\todo[inline]{ Need to define a way to compute $\Delta _{\Gamma } H $ in the background theory. }

From \cite[Lemma 2.1]{kovacs2021convergent} it derived that \eqref{eq:gradient_velocity} must satisfy the following material derivatives,
\[
    \begin{split}
\frac{D}{Dt}H & = - \left( \Delta _{\Gamma } + \left\lvert A \right\rvert ^2   \right) V, \\
\frac{D}{Dt} \nu & = \left( -\Delta _{\Gamma } + \left( HA - A^2 \right)  \right) z  \\
&  + \left\lvert \nabla _{\Gamma } H \right\rvert ^2   \nu - 2\left( \nabla _{\Gamma }\cdot \left( A \nabla _{\Gamma } H \right)  \right) \nu  \\
  & -A^2 \nabla _{\Gamma } H  - \nabla _{\Gamma } Q.
    \end{split}
\]
Here is the substitution variable introduced s.t. $z = \Delta  _{\Gamma } \nu  + \left\lvert A \right\rvert ^2 \nu $.

It exists methods which do not exploit the material derivatives \cite{bonito2010parametric, bartezzaghi2016isogeometric}.However, it turns out that including these material derivatives brings additional computational costs, but provides so-called full-order
approximation to the mean curvature, $H$, and the normal vector, $\nu $, and ,thus, allows us to construct rigorous convergence proofs for evolving surface FEM, see \cite{kovacs2021convergent, binz2022convergent}.
\todo[inline]{ Should probably find sources with Reynolds theorem or something for why material derivatives is used. }

Recent work has proposed that current method may not conserve the mesh quality while the surface is restricted to evolve along the normal velocity. Thus, a new variation of the standard methods has been considered by introducing an tangential
velocity component via the equation  $H = - v\cdot \nu  $. Hence, also allowing the mesh to be less deformed, see more at \cite{hu2022evolving}. However, in this report will we not consider it.

Finally, we end up with the following strong form of the second-order evolutionary system of PDE.
\begin{subequations}
    \label{eq:WE_strong_form}
    \begin{align}
\frac{d}{dt}H & = - \left( \Delta _{\Gamma } + \left\lvert A \right\rvert ^2   \right) V, \\
V  & =   \Delta _{\Gamma } H + Q, \\
\frac{d}{dt} \nu & = \left( -\Delta _{\Gamma } + \left( HA - A^2 \right)  \right) z \nonumber \\
& \quad   + \left\lvert \nabla _{\Gamma } H \right\rvert ^2   \nu - 2\left( \nabla _{\Gamma }\cdot \left( A \nabla _{\Gamma } H \right)  \right) \nu \nonumber  \\
  & \quad  -A^2 \nabla _{\Gamma } H  - \nabla _{\Gamma } Q, \\
z & = \Delta  _{\Gamma } \nu  + \left\lvert A \right\rvert ^2 \nu \\
    \end{align}
\end{subequations}
where the terms are,
\[
    \begin{split}
\frac{d}{dt} x  &= v,\\
v & = V\nu, \\
Q & =  - \frac{1}{2} H^{3} + H \left\lvert A \right\rvert^2, \\
A & = \nabla _{\Gamma } \nu
    \end{split}
\]
Recall that the material derivative operator $\frac{D}{Dt} $ is simply equivalent to the time derivative $\frac{d}{dt}$, thus, it is now possible to apply normal techniques for time discretization.

\section{Weak formulation of Willmore flow}%
\label{sub:weak_formulation}

The FEM is based on approximating a PDE using the weak form.
From \cite{kovacs2021convergent} has it been shown that the weak form for the dynamics presented in \eqref{eq:WE_strong_form} can we written in the following weak form. We

Let us denote the following trial and test functions,
We want to find the following trial functions,
\[
( H \times V \times \nu  \times z ) \in    \left( H^{1}\left( \Gamma  \right), H^1 \left( \Gamma  \right), \left[ H^{1} \left( \Gamma  \right)  \right] ^3, \left[ H^{1}\left( \Gamma  \right)  \right]^3   \right),
\]
s.t. for all test functions,
\[
( \chi_{H}  \times \chi _{V} \times \chi _{\nu }  \times \chi _{z} ) \in    \left( H^{1}\left( \Gamma  \right), H^1 \left( \Gamma  \right), \left[ H^{1} \left( \Gamma  \right)  \right] ^3, \left[ H^{1}\left( \Gamma  \right)  \right]^3   \right),
\]
the following system of weak system of PDE's are satisfied.

\begin{equation}
\label{eq:WE_weak_form}
\begin{split}
\left( \frac{d}{dt}H, \chi _{H} \right)_{\Gamma } &=  \left( \nabla _{\Gamma } V, \chi _{H} \right) _{\Gamma } - \left( \left\lvert A \right\rvert ^2 V, \chi _{H} \right) _{\Gamma }   \\
\left( V, \chi _{V} \right) &= -\left( \nabla _{\Gamma } H, \nabla _{\Gamma } \chi _{V} \right) + \left( Q, \chi _{V} \right)   \\
\left( \frac{d}{dt} \nu , \chi _{\nu } \right)_{\Gamma } &= \left( \nabla _{\Gamma }z, \nabla _{\Gamma } \chi _{\nu } \right) _{\Gamma } + \left( ( HA - A^2  )z,\chi_\nu  \right)_{\Gamma }   \\
& \quad  + \left( \left\lvert \nabla _{\Gamma } H \right\rvert^2 \nu + A^2\nabla _{\Gamma } H, \chi _{\nu }  \right) + 2 \left( A \nabla _{\Gamma } H, \nabla _{\Gamma } \chi _{\nu } \nu  \right)   \\
 & \quad + \left( Q, \nabla _{\Gamma } \chi _{\nu }  \right)_{\Gamma } - \left( QH\nu , \chi _{\nu } \right) \\
\left( z, \chi _{z} \right) _{\Gamma }  &= - \left( \nabla _{\Gamma } \nu , \nabla _{\Gamma } \chi _{z} \right) + \left( \left\lvert A \right\rvert ^2 \nu , \chi _{z} \right)    \\
\end{split}
.\end{equation}














    
\section{Numerical Methods}%
\label{sec:numerical_methods}


 To be able to find a numerical approximation the weak formulation \eqref{eq:WE_weak_form} will we to formulate a evolving surface element method (ESFEM) for the evolutionary equation PDE's coupled with the evolving surface velocity. We will mostly
 use the notation and methods described in \cite{kovacs2021convergent, hu2022evolving}.

\subsection{ Triangular mesh for $\Gamma ^{0}$  }%
\label{sub:triangular_mesh}

Let initial surface $\Gamma ^{0} \in  \mathbb{R} ^3$ have the triangular mesh $\mathcal{T}_{h} $ constisting of triangles $T$. We will assume that the mesh is inform, i.e., for any $T_{1}, T_{2}$ $T_{1} \neq T_{2}$ and $T_{1} \cap T_{2} \neq \emptyset
$, then must $T_{1}$  and $T_{2}$ share either a vertex or a facet. Let diameter of each triangle $h_{T}$  be denoted as,
\[
    \begin{split}
    h_{T} & = \max_{x_{1}, x_{2} \in \Gamma ^{0}} dist(x_{1}, x_{2}), \\
h_{min} & = \min_{T \in \mathcal{T} _{h}} h_{T}, \\
h_{max} & = \max_{T \in \mathcal{T} _{h}} h_{T},
    \end{split}
\]
where $h_{min}$ and $h_{max}$ is the maximum and minimum diameter of an individual triangle $T$ in $\mathcal{T}_{h} $. We define the chunkiness parameter $c_{T} = h_{T}/r_{T}$, where $r_{T}$ is the largest ball inside inscribed in $T$. We will then
assume that the mesh is shape regular and quasi-uniform, i.e., $c_{T} \le c $ and $h_{max} \le c h_{min} $ for a  constant $c$ independent of $h_{T} $ and $T$.
For more information, see \cite{dziuk2007finite}.
\todo[inline]{  Apparently some quasi-uniform and shape regular definitions in \cite{dziuk2007finite}, need to check this.}

\subsection{Finite surface element}%
\label{sub:surface_element_method}
In this section is the goal to develop discrete analogs of the definitions we introduced in subsections \ref{sub:differential_calculus} and \ref{sub:evolutionary_equations}.


Let $\mathcal{T}_{h}  $ be the triangulation of $\Gamma ^{0}$. Now denote the vector $ \mathbf{x} = \mathbf{x} \left( t \right)  \in \mathbb{R} ^{3N}$ as a collection of all evolving nodes, $x_{j}\left( t \right)  \in  \mathbb{R} ^3$, $j \in  \mathcal{I} $, by a piecewise polynomial interpolation
with degree $k$. Here is the index of all the nodes defined as $\mathcal{I} =\left\{ 1, \ldots, N \right\}  $, where $N$ is the total number of nodes.

We denote the discretized surface, $\Gamma _{h} \left[ \mathbf{x}\left( t \right)  \right] $, to be the numerical approximation of the surface $\Gamma \left( t \right) $. As an initial condition do we construct the initial nodes, $
 x_{j}( 0 )  = p_{j} \forall j \in  \mathcal{I}  $, s.t.
 $\Gamma _{h}^{0  } = \Gamma _{h} \left[ \mathbf{x}\left( 0 \right)   \right] $ interpolates the initial surface, $\Gamma ^{0}$ in the nodal points $p_{j} \in \Gamma ^{0}, \  j \in  \mathcal{I} $.

An essential piece of the puzzle is the finite element basis on $\Gamma _{h}\left[ \mathbf{x} \right] $, which is denoted to have the form,
\[
 \phi_{j} \left[ \mathbf{x} \right]: \Gamma_{h}\left[ \mathbf{x} \right] \mapsto \mathbb{R} , \quad   j \in  \mathcal{I} .
\]
\todo[inline]{ How can the basis vectors move? And how is it related to the definition of $X_{h}$ later in the subsection?  }
In fact, the basis satisfies the identity, \[
\phi _{j} \left[ \mathbf{x} \right] \left( x_{i} \right)  = \delta _{ij} \quad \forall i,j \in  \mathcal{I} .
\]

Finally, we can now define the so-called evolving finite element space of $\Gamma _{h}\left[ \mathbf{x}\left( t \right)  \right] $ as, \[
    \begin{split}
S_{h}\left[ \mathbf{x} \right]   & = S_{h}\left( \Gamma _{h}\left[ \mathbf{x} \right]  \right) \\
 & = span\left\{ \sum_{j \in  \mathcal{I}  }^{}  c_{j} \phi _{j}\left[ \mathbf{x} \right]  \mid  c_{j} \in \mathbb{R}   \right\}.
    \end{split}
\]

The discretized surface mapping, $X_{h}: \Gamma_{h} ^{0} \mapsto  \Gamma _{h} \left[ \mathbf{x}\left( t \right)  \right] $, is denoted as \[
X_{h}\left( p_{h},t \right) = \sum_{j \in \mathcal{I} }^{}  x_{j}\left( t \right)  \phi \left[ x\left( 0 \right)  \right] \left( p_{h} \right), \  p_{h} \in \Gamma ^{0}_{h},
\]
and the problem obviously does satisfy the identities; $X_{h}\left( p_{h},0 \right)  = p_{h} \forall p_{h} \in \Gamma _{h}$, the evolutionary property $X_{h} \left( p_{j}, t \right)  = x_{j}\left( t \right)  \forall j \in \mathcal{I} $, and
\[
    \begin{split}
        \Gamma _{h} & = \Gamma_{h} \left( t \right) =   \Gamma_{h} \left[ \mathbf{x}\left( t \right)  \right]  \\
                                    &= \Gamma_{h}\left[ X_{h}\left( \cdot ,t \right)  \right] \\
&=\left\{ X_{h}\left( p_{h},t \right): p_{h} \in  \Gamma ^{0}_{h} \right\}.
    \end{split}
\]
\todo[inline]{TODO: Need to check if the definition of $\Gamma _{h}\left[ \mathbf{x}  \right] = \Gamma_{h}\left[ X_{h}\left( \cdot ,t \right)  \right] $ makes sense. Also made $\Gamma _{h} = \Gamma _{h}\left[ \mathbf{x}\left( t \right)  \right] $ som et kunstig innslag. }
For shorthand notation do we define the discretized point as $x_{h} = X_{h}\left( p_{h},t \right) \in \Gamma _{h} $.
Thus discretized velocity $v_{h}\left( x_{h},t \right): \Gamma _{h} \mapsto  \mathbb{R} ^{3} $ is defined as \[
    v_{h}\left( x_{h},t \right)  = \sum_{j \in \mathcal{I} }^{} v_{j}\left( t \right) \phi _{j}\left[ \mathbf{x}\left( t \right)  \right] \left( x_{h} \right)
\]
Hence, by introducing  $\mathbf{v} = \mathbf{v}\left( t \right)  $ as a collection of the velocity nodes,  $v_{j} \in \mathbb{R} ^3$, can we define the evolutionary dynamics of the nodal points $\mathbf{x}$ s.t. this holds,
\[
\frac{d}{dt}\mathbf{x}\left( t \right) = \mathbf{v}\left( t \right).
\]

Ultimately, the discrete material derivative of a function $f_{h}: \Gamma _{h} \mapsto  \mathbb{R} $ is simply defined in the direction in the velocity $c_{h}$  as \[
    \begin{split}
\frac{D}{Dt}f_{h}\left( x_{h}, t; v_{h}   \right) & = \frac{d}{dt} f_{h}\left( x_{h}, t; v_{h}   \right) \\
 & = \sum_{j \in \mathcal{I} }^{} \frac{d f_{j} \left( t \right) }{dt} \phi _{j}\left[ \mathbf{x} \right] \left( x_{h} \right)
    \end{split}
\]
where $f_{j}$ is the nodal variables.

\todo[inline]{ Is $ \frac{d f_{j} \left( t \right) }{dt}$  this consistent with  with the similar definition of the velocity? I guess so since $\dot{x_{j}} = v_{j}$. Anyhow, I am also wondering if the basis functions should have time derivative as
well.  }.



\subsection{Surface finite element method}%
\label{sub:surface_finite_element_method}

Let us denote the following trial functions,
\begin{equation}
\label{eq:FE_trial}
    \begin{split}
& ( H_{h} \times V_{h} \times \nu_{h}  \times z_{h} ) \in \\
&\left( S_{h}\left[ \mathbf{x} \right],S_{h}\left[ \mathbf{x} \right], S_{h}\left[ \mathbf{x} \right] ^3,S_{h}\left[ \mathbf{x} \right] ^3    \right),
    \end{split}
\end{equation}
and similarly the test functions,
\begin{equation}
\label{eq:FE_test}
    \begin{split}
& ( \chi_{h}^{H}  \times \chi_{h} ^{V} \times \chi_{h} ^{\nu }  \times \chi_{h} ^{z} ) \in \\
&\left( S_{h}\left[ \mathbf{x} \right],S_{h}\left[ \mathbf{x} \right], S_{h}\left[ \mathbf{x} \right] ^3,S_{h}\left[ \mathbf{x} \right] ^3    \right),
    \end{split}
.\end{equation}

Define the helping functions
\[
    \begin{split}
v_{h} & = V_{h}\nu_{h}, \\
Q_{h} & =  - \frac{1}{2} H_{h}^{3} + H_{h} \left\lvert A_{h} \right\rvert^2, \\
A_{h} & = \frac{1}{2}\left( \nabla _{\Gamma_{h} } \nu_{h} + \left( \nabla _{\Gamma _{h}}\nu _{h} \right) ^{T} \right)
    \end{split}
\]
\todo[inline]{ Why is the $A_{h}$ symmetric, and maybe $v_{h}$ has to be interpolated. \cite[p.13]{kovacs2021convergent}   }
\todo[inline]{ May introduce the interpolation $v_{h} = \widetilde{I}_{h}: C\left( \Gamma _{h}\left[ \mathbf{x} \right]  \mapsto  S_{h}\left[ \mathbf{x} \right]   \right)  $  check \cite[Remark 3.1]{kovacs2021convergent}, it important for the error analysis. }


 It has been proven that if you can find any trial functions \eqref{eq:FE_trial} that does satisfy the following weak formulation \eqref{eq:WE_weak_form} for all test functions \eqref{eq:WE_test}, then the solution is equivalent to solving the dynamics presented in \eqref{eq:WE_strong_form}. See \cite{kovacs2021convergent} for more information.

\begin{subequations}
\label{eq:FEM}
 \begin{align}
\left( \frac{d}{dt}H_{h}, \chi_{h} ^{H} \right)_{\Gamma_{h} } =&  \left( \nabla _{\Gamma_{h} } V_{h}, \nabla _{\Gamma_{h} }\chi_{h} ^{H} \right) _{\Gamma_{h} } \nonumber \\
                                                   &- \left( \left\lvert A_{h} \right\rvert ^2 V_{h}, \chi_{h} ^{H} \right) _{\Gamma_{h} },   \\
\left( V_{h}, \chi_{h} ^{V} \right)_{\Gamma } =& -\left( \nabla _{\Gamma_{h} } H_{h}, \nabla _{\Gamma_{h} } \chi_{h} ^{V} \right) \nonumber \\
 & + \left( Q_{h}, \chi_{h} ^{V} \right)_{\Gamma_{h} },   \\
\left( \frac{d}{dt} \nu_{h} , \chi_{h} ^{\nu } \right)_{\Gamma_{h} } =& \left( \nabla _{\Gamma_{h} }z_{h}, \nabla _{\Gamma_{h} } \chi_{h} ^{\nu } \right) _{\Gamma_{h} }\nonumber   \\
 & + \left( ( H_{h}A_{h} - A_{h}^2  )z_{h},\chi_{h}^\nu  \right)_{\Gamma_{h} }   \nonumber \\
 & + \left( \left\lvert \nabla _{\Gamma_{h} } H_{h} \right\rvert^2 \nu_{h} , \chi_{h} ^{\nu_{h} }  \right)_{\Gamma_{h}} \nonumber  \\
 & + \left(  A_{h}^2\nabla _{\Gamma_{h} } H_{h}, \chi_{h} ^{\nu_{h} }  \right)_{\Gamma_{h}} \nonumber  \\
 & + 2 \left( A_{h} \nabla _{\Gamma_{h} } H_{h}, \nabla _{\Gamma_{h} } \chi_{h} ^{\nu_{h} } \nu_{h}  \right)_{\Gamma_{h}} \nonumber    \\
 &  + \left( Q_{h}, \nabla _{\Gamma_{h} } \cdot \chi_{h} ^{\nu_{h} }  \right)_{\Gamma_{h} } \nonumber  \\
  &- \left( Q_{h}H_{h}\nu_{h} , \chi_{h} ^{\nu_{h} } \right) _{\Gamma_{h}},\\
\left( z_{h}, \chi_{h} ^{z} \right) _{\Gamma_{h} }  =& - \left( \nabla _{\Gamma_{h} } \nu_{h} , \nabla _{\Gamma_{h} } \chi_{h} ^{z_{h}} \right)_{\Gamma_{h}} \nonumber \\
 & + \left( \left\lvert A_{h} \right\rvert ^2 \nu_{h} , \chi_{h} ^{z} \right) \\
\frac{d}{dt}x_h &= v_h
 .\end{align}
\end{subequations}

Note that the time dependent variables $\nu, H $ and $x$ is initialized at $t=0$  based on $\Gamma ^{0}$.














    
\newpage
\section{Conclusion}%
\label{sec:conclusion}

We first presented the construction on the properties of the continuous interior penalty methods for the biharmonic equation with Cahn-Hilliard boundary conditions based on \cite{feng2007fully, brenner2012}.
Next, we constructed a stabilized cut finite element method for an unfitted mesh based the framework established in \cite{gurkan2019stabilized}.
This method involves the use of a so-called ghost penalty, which ensures stabilization by integrating
outside of the physical domain.
Theoretical evidence was provided to validate its stability, and an a priori analysis was conducted alongside corresponding numerical experiments.
Finally, we then demonstrated that the cut continuous interior penalty methods can successfully be applied to the nonlinear Cahn-Hilliard equation.


    \printbibliography
\end{sloppy}

\end{document}
