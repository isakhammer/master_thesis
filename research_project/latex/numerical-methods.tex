
\section{Numerical Methods}%
\label{sec:numerical_methods}


 To be able to find a numerical approximation the weak formulation \eqref{eq:WE_weak_form} will we to formulate a evolving surface element method (ESFEM) for the evolutionary equation PDE's coupled with the evolving surface velocity. We will mostly
 use the notation and methods described in \cite{kovacs2021convergent, hu2022evolving}.

\subsection{ Triangular mesh for $\Gamma ^{0}$  }%
\label{sub:triangular_mesh}

Let initial surface $\Gamma ^{0} \in  \mathbb{R} ^3$ have the triangular mesh $\mathcal{T}_{h} $ constisting of triangles $T$. We will assume that the mesh is inform, i.e., for any $T_{1}, T_{2}$ $T_{1} \neq T_{2}$ and $T_{1} \cap T_{2} \neq \emptyset
$, then must $T_{1}$  and $T_{2}$ share either a vertex or a facet. Let diameter of each triangle $h_{T}$  be denoted as,
\[
    \begin{split}
    h_{T} & = \max_{x_{1}, x_{2} \in \Gamma ^{0}} dist(x_{1}, x_{2}), \\
h_{min} & = \min_{T \in \mathcal{T} _{h}} h_{T}, \\
h_{max} & = \max_{T \in \mathcal{T} _{h}} h_{T},
    \end{split}
\]
where $h_{min}$ and $h_{max}$ is the maximum and minimum diameter of an individual triangle $T$ in $\mathcal{T}_{h} $. We define the chunkiness parameter $c_{T} = h_{T}/r_{T}$, where $r_{T}$ is the largest ball inside inscribed in $T$. We will then
assume that the mesh is shape regular and quasi-uniform, i.e., $c_{T} \le c $ and $h_{max} \le c h_{min} $ for a  constant $c$ independent of $h_{T} $ and $T$.
For more information, see \cite{dziuk2007finite}.
\todo[inline]{  Apparently some quasi-uniform and shape regular definitions in \cite{dziuk2007finite}, need to check this.}

\subsection{Finite surface element}%
\label{sub:surface_element_method}
In this section is the goal to develop discrete analogs of the definitions we introduced in subsections \ref{sub:differential_calculus} and \ref{sub:evolutionary_equations}.


Let $\mathcal{T}_{h}  $ be the triangulation of $\Gamma ^{0}$. Now denote the vector $ \mathbf{x} = \mathbf{x} \left( t \right)  \in \mathbb{R} ^{3N}$ as a collection of all evolving nodes, $x_{j}\left( t \right)  \in  \mathbb{R} ^3$, $j \in  \mathcal{I} $, by a piecewise polynomial interpolation
with degree $k$. Here is the index of all the nodes defined as $\mathcal{I} =\left\{ 1, \ldots, N \right\}  $, where $N$ is the total number of nodes.

We denote the discretized surface, $\Gamma _{h} \left[ \mathbf{x}\left( t \right)  \right] $, to be the numerical approximation of the surface $\Gamma \left( t \right) $. As an initial condition do we construct the initial nodes, $
 x_{j}( 0 )  = p_{j} \forall j \in  \mathcal{I}  $, s.t.
 $\Gamma _{h}^{0  } = \Gamma _{h} \left[ \mathbf{x}\left( 0 \right)   \right] $ interpolates the initial surface, $\Gamma ^{0}$ in the nodal points $p_{j} \in \Gamma ^{0}, \  j \in  \mathcal{I} $.

An essential piece of the puzzle is the finite element basis on $\Gamma _{h}\left[ \mathbf{x} \right] $, which is denoted to have the form,
\[
 \phi_{j} \left[ \mathbf{x} \right]: \Gamma_{h}\left[ \mathbf{x} \right] \mapsto \mathbb{R} , \quad   j \in  \mathcal{I} .
\]
\todo[inline]{ How can the basis vectors move? And how is it related to the definition of $X_{h}$ later in the subsection?  }
In fact, the basis satisfies the identity, \[
\phi _{j} \left[ \mathbf{x} \right] \left( x_{i} \right)  = \delta _{ij} \quad \forall i,j \in  \mathcal{I} .
\]

Finally, we can now define the so-called evolving finite element space of $\Gamma _{h}\left[ \mathbf{x}\left( t \right)  \right] $ as, \[
    \begin{split}
S_{h}\left[ \mathbf{x} \right]   & = S_{h}\left( \Gamma _{h}\left[ \mathbf{x} \right]  \right) \\
 & = span\left\{ \sum_{j \in  \mathcal{I}  }^{}  c_{j} \phi _{j}\left[ \mathbf{x} \right]  \mid  c_{j} \in \mathbb{R}   \right\}.
    \end{split}
\]

The discretized surface mapping, $X_{h}: \Gamma_{h} ^{0} \mapsto  \Gamma _{h} \left[ \mathbf{x}\left( t \right)  \right] $, is denoted as \[
X_{h}\left( p_{h},t \right) = \sum_{j \in \mathcal{I} }^{}  x_{j}\left( t \right)  \phi \left[ x\left( 0 \right)  \right] \left( p_{h} \right), \  p_{h} \in \Gamma ^{0}_{h},
\]
and the problem obviously does satisfy the identities; $X_{h}\left( p_{h},0 \right)  = p_{h} \forall p_{h} \in \Gamma _{h}$, the evolutionary property $X_{h} \left( p_{j}, t \right)  = x_{j}\left( t \right)  \forall j \in \mathcal{I} $, and
\[
    \begin{split}
        \Gamma _{h} & = \Gamma_{h} \left( t \right) =   \Gamma_{h} \left[ \mathbf{x}\left( t \right)  \right]  \\
                                    &= \Gamma_{h}\left[ X_{h}\left( \cdot ,t \right)  \right] \\
&=\left\{ X_{h}\left( p_{h},t \right): p_{h} \in  \Gamma ^{0}_{h} \right\}.
    \end{split}
\]
\todo[inline]{TODO: Need to check if the definition of $\Gamma _{h}\left[ \mathbf{x}  \right] = \Gamma_{h}\left[ X_{h}\left( \cdot ,t \right)  \right] $ makes sense. Also made $\Gamma _{h} = \Gamma _{h}\left[ \mathbf{x}\left( t \right)  \right] $ som et kunstig innslag. }
For shorthand notation do we define the discretized point as $x_{h} = X_{h}\left( p_{h},t \right) \in \Gamma _{h} $.
Thus discretized velocity $v_{h}\left( x_{h},t \right): \Gamma _{h} \mapsto  \mathbb{R} ^{3} $ is defined as \[
    v_{h}\left( x_{h},t \right)  = \sum_{j \in \mathcal{I} }^{} v_{j}\left( t \right) \phi _{j}\left[ \mathbf{x}\left( t \right)  \right] \left( x_{h} \right)
\]
Hence, by introducing  $\mathbf{v} = \mathbf{v}\left( t \right)  $ as a collection of the velocity nodes,  $v_{j} \in \mathbb{R} ^3$, can we define the evolutionary dynamics of the nodal points $\mathbf{x}$ s.t. this holds,
\[
\frac{d}{dt}\mathbf{x}\left( t \right) = \mathbf{v}\left( t \right).
\]

Ultimately, the discrete material derivative of a function $f_{h}: \Gamma _{h} \mapsto  \mathbb{R} $ is simply defined in the direction in the velocity $c_{h}$  as \[
    \begin{split}
\frac{D}{Dt}f_{h}\left( x_{h}, t; v_{h}   \right) & = \frac{d}{dt} f_{h}\left( x_{h}, t; v_{h}   \right) \\
 & = \sum_{j \in \mathcal{I} }^{} \frac{d f_{j} \left( t \right) }{dt} \phi _{j}\left[ \mathbf{x} \right] \left( x_{h} \right)
    \end{split}
\]
where $f_{j}$ is the nodal variables.

\todo[inline]{ Is $ \frac{d f_{j} \left( t \right) }{dt}$  this consistent with  with the similar definition of the velocity? I guess so since $\dot{x_{j}} = v_{j}$. Anyhow, I am also wondering if the basis functions should have time derivative as
well.  }.



\subsection{Surface finite element method}%
\label{sub:surface_finite_element_method}

Let us denote the following trial functions,
\begin{equation}
\label{eq:FE_trial}
    \begin{split}
& ( H_{h} \times V_{h} \times \nu_{h}  \times z_{h} ) \in \\
&\left( S_{h}\left[ \mathbf{x} \right],S_{h}\left[ \mathbf{x} \right], S_{h}\left[ \mathbf{x} \right] ^3,S_{h}\left[ \mathbf{x} \right] ^3    \right),
    \end{split}
\end{equation}
and similarly the test functions,
\begin{equation}
\label{eq:FE_test}
    \begin{split}
& ( \chi_{h}^{H}  \times \chi_{h} ^{V} \times \chi_{h} ^{\nu }  \times \chi_{h} ^{z} ) \in \\
&\left( S_{h}\left[ \mathbf{x} \right],S_{h}\left[ \mathbf{x} \right], S_{h}\left[ \mathbf{x} \right] ^3,S_{h}\left[ \mathbf{x} \right] ^3    \right),
    \end{split}
.\end{equation}

Define the helping functions
\[
    \begin{split}
v_{h} & = V_{h}\nu_{h}, \\
Q_{h} & =  - \frac{1}{2} H_{h}^{3} + H_{h} \left\lvert A_{h} \right\rvert^2, \\
A_{h} & = \frac{1}{2}\left( \nabla _{\Gamma_{h} } \nu_{h} + \left( \nabla _{\Gamma _{h}}\nu _{h} \right) ^{T} \right)
    \end{split}
\]
For the error analysis is it the discrete velocity denoted as, $v_{h} = \widetilde{I}_{h}\left( V_{h} \nu _{h} \right)  $,  where the interpolation operator is defined s.t. $ \widetilde{I}_{h}: C\left( \Gamma _{h}\left[ \mathbf{x} \right]  \mapsto  S_{h}\left[
\mathbf{x} \right]   \right)  $, however, in this article will this not be discussed, hence, we will note use the interpolation estimate. For more information see \cite[Remark 3.1]{kovacs2021convergent}.
\todo[inline]{ Why is the $A_{h}$ symmetric, and maybe $v_{h}$ has to be interpolated. \cite[p.13]{kovacs2021convergent}   }


 It has been proven that if you can find any trial functions \eqref{eq:FE_trial} that does satisfy the following weak formulation \eqref{eq:WE_weak_form} for all test functions \eqref{eq:WE_test}, then the solution is equivalent to solving the dynamics presented in \eqref{eq:WE_strong_form}. See \cite{kovacs2021convergent} for more information.

\begin{subequations}
\label{eq:FEM}
 \begin{align}
\left( \frac{d}{dt}H_{h}, \chi_{h} ^{H} \right)_{\Gamma_{h} } =&  \left( \nabla _{\Gamma_{h} } V_{h}, \nabla _{\Gamma_{h} }\chi_{h} ^{H} \right) _{\Gamma_{h} } \nonumber \\
                                                   &- \left( \left\lvert A_{h} \right\rvert ^2 V_{h}, \chi_{h} ^{H} \right) _{\Gamma_{h} },   \\
\left( V_{h}, \chi_{h} ^{V} \right)_{\Gamma } =& -\left( \nabla _{\Gamma_{h} } H_{h}, \nabla _{\Gamma_{h} } \chi_{h} ^{V} \right) \nonumber \\
 & + \left( Q_{h}, \chi_{h} ^{V} \right)_{\Gamma_{h} },   \\
\left( \frac{d}{dt} \nu_{h} , \chi_{h} ^{\nu } \right)_{\Gamma_{h} } =& \left( \nabla _{\Gamma_{h} }z_{h}, \nabla _{\Gamma_{h} } \chi_{h} ^{\nu } \right) _{\Gamma_{h} }\nonumber   \\
 & + \left( ( H_{h}A_{h} - A_{h}^2  )z_{h},\chi_{h}^\nu  \right)_{\Gamma_{h} }   \nonumber \\
 & + \left( \left\lvert \nabla _{\Gamma_{h} } H_{h} \right\rvert^2 \nu_{h} , \chi_{h} ^{\nu_{h} }  \right)_{\Gamma_{h}} \nonumber  \\
 & + \left(  A_{h}^2\nabla _{\Gamma_{h} } H_{h}, \chi_{h} ^{\nu_{h} }  \right)_{\Gamma_{h}} \nonumber  \\
 & + 2 \left( A_{h} \nabla _{\Gamma_{h} } H_{h}, \nabla _{\Gamma_{h} } \chi_{h} ^{\nu_{h} } \nu_{h}  \right)_{\Gamma_{h}} \nonumber    \\
 &  + \left( Q_{h}, \nabla _{\Gamma_{h} } \cdot \chi_{h} ^{\nu_{h} }  \right)_{\Gamma_{h} } \nonumber  \\
  &- \left( Q_{h}H_{h}\nu_{h} , \chi_{h} ^{\nu_{h} } \right) _{\Gamma_{h}},\\
\left( z_{h}, \chi_{h} ^{z} \right) _{\Gamma_{h} }  =& - \left( \nabla _{\Gamma_{h} } \nu_{h} , \nabla _{\Gamma_{h} } \chi_{h} ^{z_{h}} \right)_{\Gamma_{h}} \nonumber \\
 & + \left( \left\lvert A_{h} \right\rvert ^2 \nu_{h} , \chi_{h} ^{z} \right) \\
\frac{d}{dt}x_h &= v_h
 .\end{align}
\end{subequations}

Note that the time dependent variables $\nu, H $ and $x$ is initialized at $t=0$  based on $\Gamma ^{0}$.













