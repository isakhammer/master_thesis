
\section{Numerical Methods}%
\label{sec:numerical_methods}


 To be able to find a numerical approximation the weak formulation \eqref{eq:WE_weak_form} will we to formulate a evolving surface element method (ESFEM) for the evolutionary equation PDE's coupled with the evolving surface velocity. We will mostly
 use the notation and methods described in \cite{kovacs2021convergent, hu2022evolving}.

\subsection{ Triangular mesh for $\Gamma ^{0}$  }%
\label{sub:triangular_mesh}

Let initial surface $\Gamma ^{0} \in  \mathbb{R} ^3$ have the triangular mesh $\mathcal{T}_{h} $ constisting of triangles $T$. We will assume that the mesh is inform, i.e., for any $T_{1}, T_{2}$ $T_{1} \neq T_{2}$ and $T_{1} \cap T_{2} \neq \emptyset
$, then must $T_{1}$  and $T_{2}$ share either a vertex or a facet. Let diameter of each triangle $h_{T}$  be denoted as,
\[
    \begin{split}
    h_{T} & = \max_{x_{1}, x_{2} \in \Gamma ^{0}} dist(x_{1}, x_{2}), \\
h_{min} & = \min_{T \in \mathcal{T} _{h}} h_{T}, \\
h_{max} & = \max_{T \in \mathcal{T} _{h}} h_{T},
    \end{split}
\]
where $h_{min}$ and $h_{max}$ is the maximum and minimum diameter of an individual triangle $T$ in $\mathcal{T}_{h} $. We define the chunkiness parameter $c_{T} = h_{T}/r_{T}$, where $r_{T}$ is the largest ball inside inscribed in $T$. We will then
assume that the mesh is shape regular and quasi-uniform, i.e., $c_{T} \le c $ and $h_{max} \le c h_{min} $ for a  constant $c$ independent of $h_{T} $ and $T$.
For more information, see \cite{dziuk2007finite}.
\todo[inline]{  Apparently some quasi-uniform and shape regular definitions in \cite{dziuk2007finite}, need to check this.}

\subsection{Surface element method}%
\label{sub:surface_element_method}
In this section is the goal to develop discrete analogs of the definitions we introduced in subsection \ref{sub:differential_calculus}.



Let $\mathcal{T}_{h}  $ be the triangulation of $\Gamma ^{0}$. Now denote the vector $ \mathbf{x} = \mathbf{x} \left( t \right)  \in \mathbb{R} ^{3N}$ as a collection of all evolving nodes, $x_{j}\left( t \right)  \in  \mathbb{R} ^3$, $j = \mathcal{I} $, by a piecewise polynomial interpolation
with degree $k$. Here is the index of all the nodes defined as $\mathcal{I} =\left\{ 1, \ldots, N \right\}  $, where $N$ is the total number of nodes.

We denote the discretized surface, $\Gamma _{h} \left[ \mathbf{x}\left( t \right)  \right] $, to be the numerical approximation of the surface $\Gamma \left( t \right) $. As an initial condition do we construct the initial nodes, $
 x_{j}( 0 )  = p_{j} \forall j \in  \mathcal{I}  $, s.t.
 $\Gamma _{h}^{0  } = \Gamma _{h} \left[ \mathbf{x}\left( 0 \right)   \right] $ interpolates the initial surface, $\Gamma ^{0}$ in the nodal points $p_{j} \in \Gamma ^{0}, \  j \in  \mathcal{I} $.

We denote the finite element basis on $\Gamma _{h}\left[ \mathbf{x} \right] $ to have the form,
\[
 \phi_{j} \left[ \mathbf{x} \right]: \Gamma_{h}\left[ \mathbf{x} \right] \mapsto \mathbb{R} , \quad   j \in  \mathcal{I} .
\]
which in fact satisfies the identity, \[
\phi _{j} \left[ \mathbf{x} \right] \left( x_{i} \right)  = \delta _{ij} \quad \forall i,j \in  \mathcal{I} .
\]

Finally, we can now define the so-called evolving finite element space of $\Gamma _{h}\left[ \mathbf{x}\left( t \right)  \right] $ as, \[
    \begin{split}
S_{h}\left[ \mathbf{x} \right]   & = S_{h}\left( \Gamma _{h}\left[ \mathbf{x} \right]  \right) \\
 & = span\left\{ \sum_{j \in  \mathcal{I}  }^{}  c_{j} \phi _{j}\left[ \mathbf{x} \right]  \mid  c_{j} \in \mathbb{R}   \right\}.
    \end{split}
\]

The discretized surface mapping, $X_{h}: \Gamma_{h} ^{0} \mapsto  \Gamma _{h} \left[ \mathbf{x}\left( t \right)  \right] $, is denoted as \[
X_{h}\left( p_{h},t \right) = \sum_{j \in \mathcal{I} }^{}  x_{j}\left( t \right)  \phi \left[ x\left( 0 \right)  \right] \left( p_{h} \right), \  p_{h} \in \Gamma ^{0}_{h},
\]
and obviously satisfied $X_{h}\left( p_{h},0 \right)  = p_{h} \forall p_{h} \in \Gamma _{h}$, the evolutionary property $X_{h} \left( p_{j}, t \right)  = x_{j}\left( t \right)  \forall j \in \mathcal{I} $, and
\[
    \begin{split}
\Gamma_{h} \left[ \mathbf{x}\left( t \right)  \right]  & = \Gamma_{h}\left[ X_{h}\left( .,t \right)  \right] \\
&=\left\{ X_{h}\left( p_{h},t \right): p_{h} \in  \Gamma ^{0}_{h} \right\}.
    \end{split}
\]














