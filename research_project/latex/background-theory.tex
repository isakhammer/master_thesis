

\section{Background Theory}%
\label{sec:differential_geometry}

The section is highly inspired by the notation used in \cite{kovacs2021convergent}.
Let some initial surface $\Gamma^{0} \subset \mathbb{R} ^3  $ smooth compact and oriented surface with no boundary where we can assign a unique point $p \in \Gamma ^{0}$. We define define the time evolutionary surface to be on the form,
\[
    \begin{split}
\Gamma  = \Gamma \left( t \right) & = \Gamma \left( \chi \left( p,t \right)  \right) \\
                                  &= \left\{ \chi \left( p,t \right): \ p \in \Gamma^{0}  \right\}
    \end{split}
\]
transformed via the smooth mapping
\[
\chi : \Gamma^{0} \times  \left[ 0,T \right]  \mapsto  \mathbb{R} ^3.
\]
We will define a unique evolutionary point $x \in \Gamma \left( t \right) $ based on the smooth mapping $\chi \left( p,t \right) = x$. A way to imagine this is to have a initial point in $\Gamma ^{0}$ and the mapping $\chi $ describes how this point will deform over time. The outer unit normal vector field of $\Gamma \left( t \right) $ is defined as the mapping $\nu : \Gamma \mapsto
\mathbb{R} ^{3}$. The velocity $v$ for a point $x = \chi \left( p,t \right) $ can be defined as
\begin{equation}
    \label{eq:vel}
\frac{\partial }{\partial  t}  \chi \left( p,t \right)  = v\left( x,t \right).
.\end{equation}
Given a model of the velocity $v$ can we solve the differential equation \eqref{eq:vel} and determine the evolution of a point on the surface $\Gamma $
Assume we have a function $f: \Gamma \mapsto \mathbb{R}  $. We can then denote the material derivative as,
\[
    \begin{split}
\frac{D}{Dt}  f\left( x,t \right)  & = \frac{d}{dt} f \left( \chi \left( p,t \right) , t \right) \\
&= \frac{\partial f\left( x,t \right) }{\partial  t} + \frac{\partial f\left( x,t \right) }{\partial  x}  \frac{\partial \chi \left( p,t \right) }{\partial  t}   \\
    \end{split}
\]
for $x \in \Gamma \left( t \right)  $.
\todo[inline]{ Not sure about the second line }
We can then finally define the tangential gradient mapping (project gradient) $\nabla _{\Gamma } f: \Gamma \mapsto \mathbb{R} ^{3}$ and the divergence mapping $\nabla _{\Gamma } \cdot  f: \Gamma \mapsto \mathbb{R} $ s.t.
\[
\begin{split}
\nabla_{\Gamma } f & = \nabla f - \left<\nu, \nabla f \right> \nu. \\
\nabla_{\Gamma } \cdot f &= \text{TODO}  \\
\end{split}
\]
and the Laplace-Beltrami operator $\Delta _{\Gamma }f: \Gamma \mapsto \mathbb{R}$ s.t. $ \Delta _{\Gamma } f  = \nabla _{\Gamma } \cdot  \nabla _{\Gamma }f$.

A method to compute the mean curvature and the so-called Frobenius norm of matrix $A$ involves applying the
extended Weingarten map, $ A\left( x \right) = \nabla \nu \left( x \right) $, s.t. these identities holds \cite{kovacs2021convergent},
\begin{equation*}
    \begin{split}
    H & = tr(A) = k_{1} + k_{2} \\
    \left\lvert A \right\rvert^{2}  & = k_{1}^2 + k_{2}^2
    \end{split}
.\end{equation*}

Now, let some arbitrary energy functional have the form
\[
\mathcal{J} = \int_{\Gamma }^{} \varphi,.
\]
For instance, the case in \eqref{eq:WE} is $\varphi = H ^2$.
Using the definition from \cite{bonito2010parametric, troltzsch2010optimal} can we define the shape derivative of some energy
functional $\mathcal{J} \left( \Gamma \left( t \right)  \right)  $ towards any directions $ w \forall w \in \mathcal{V}  $  to be the limit

% Det e no som skurra i dennar notasjonen
\begin{equation*}
    \begin{split}
d\mathcal{J} \left( \Gamma \left( t \right) ; w  \right)  & = \lim_{t \to 0} \frac{\mathcal{J}\left( \Gamma \left( t \right)  \right) - \mathcal{J} (  \Gamma \left( 0 \right))}{t} \\
&= \left(  \varphi ,w\right)_{\Gamma \left( t \right) }  \\
    \end{split}
.\end{equation*}
\todo[inline]{ Dennar definisjonen e litt rar, sjekk Troltzsch ch2.6 istedenfor  } The notation used here is $\left( v,w \right)_{\Gamma } = \int_{\Gamma }^{}  vw   $ for some $ v,w \in L^{2}\left( \Gamma \right) $ .

Our goal is to develop the evolutionary dynamics by minimizing the energy functional \eqref{eq:WE}.

To minimize our energy functional \eqref{eq:WE} of the surface dynamics will we utilize a method called gradient flows. Gradient flows in surface partial differential equation (PDE) are used to solve physical problems where the surface is changing due to some external force. The PDE describes how the surface changes over time in response to this force, thus allowing us to model real-world phenomena such as fluid and heat flow. The gradient of the surface PDE determines the direction and magnitude of the change over time, while the PDE itself may contain additional terms that modify or influence the solution. \cite{dogan2007discrete} An alternative approach would be to solve the problem using standard shape optimization techniques using $\Gamma$ as a variable surface \cite{dalphin2014study}.
We say that the $L^{2}$  gradient flow is defined as
\[
\left( \dot{\chi } , w \right)_{\Gamma \left( t \right)  } = -d\mathcal{J}\left( \Gamma \left( t \right); w  \right).
\]




