

\section{Background Theory}%
\label{sec:differential_geometry}


\subsection{Differential Calculus}%
\label{sub:differential_calculus}

This subsection is inspired by the notation used in \cite{kovacs2021convergent, dougan2012first}.
Let some initial surface $\Gamma^{0} \subset \mathbb{R} ^3  $ smooth compact and oriented surface with no boundary where we can assign a unique point $p \in \Gamma ^{0}$. We define define the time evolutionary surface to be on the form,
\[
    \begin{split}
\Gamma  = \Gamma \left( t \right) & = \Gamma \left( X \left(\cdot ,t \right)  \right) \\
                                  &= \left\{ X \left( p,t \right): \ p \in \Gamma^{0}  \right\}
    \end{split}
\]
transformed via the smooth mapping,
\[
X : \Gamma^{0} \times  \left[ 0,T \right]  \mapsto  \mathbb{R} ^3.
\]
An important regularity result is that if $\Gamma ^{0}$ is of class $C^{\infty}$, then $\Gamma $ is also of class $C^{\infty}$ for $\forall t \in \left[ 0,T \right] $ \cite{sokolowski1992introduction, dougan2012first}.
\todo[inline]{ Formally in \cite[p 48]{sokolowski1992introduction}, it might be an idea to formulate normal unit-vector regularity as $C^{\infty}$ }

We will define a unique evolutionary point $x \in \Gamma \left( t \right) $ based on the smooth mapping $X \left( p,t \right) = x$. A way to imagine this is to have a initial point in $\Gamma ^{0}$ and the mapping $X $ describes how this point will deform over time. The outer unit normal vector field of $\Gamma \left( t \right) $ is defined as the mapping $\nu : \Gamma \mapsto
\mathbb{R} ^{3}$.

Using the notation presented in \cite{dougan2012first} and \cite{kovacs2021convergent} can we define the basic surface differential operators. Consider a scalar function, $u: \Gamma \mapsto \mathbb{R} $, and a vector-valued function, $\hat{u}: \Gamma  \mapsto \mathbb{R} ^3$. We can then denote $ \nabla _{\Gamma } u: \Gamma \mapsto \mathbb{R} ^{3}$ as the tangential operator,
$$
\nabla_{\Gamma
} u  = \nabla u - \left<\nu, \nabla u \right> \nu.
$$
\todo[inline]{May be an idea to define a extension $\widetilde{u} \mid _{\Gamma }$ and look into regularity. See definitions in \cite{dziuk2013finite}.  }
Similarly, for the vector-valued function is the operator defined s.t.
$$\nabla_{\Gamma } \hat{u} = \left( \nabla_{\Gamma } u_{1},\nabla_{\Gamma } u_{2},\nabla_{\Gamma } u_{3}   \right)^{T}.$$ The surface divergence for a vector-valued function is defined as \[
\nabla_{\Gamma } \cdot \hat{u} = \nabla  \cdot \hat{u} - \nu^{T} D \hat{u} \cdot \nu
\]
Here $D\hat{u}$ denotes the Jacobian of $\hat{u}$. Similarly, the Laplace-Beltrami operator $\Delta _{\Gamma }u  : \Gamma \mapsto \mathbb{R}$  is proven to have the form \cite[Lemma 1]{xu2003eulerian},
\begin{equation*}
    \begin{split}
 \Delta _{\Gamma } u  & = \nabla _{\Gamma } \cdot  \nabla _{\Gamma }u \\
 &=  \Delta u  - \nu ^{T} D^2 u \cdot \nu - H \partial _{\nu } u
    \end{split}
.\end{equation*}
Here is $D^2u$ denotes as the Hessian of the scalar function $u$. In the case of a vector valued function is the operator defined as \[
\Delta _{\Gamma } \hat{u} = \left( \Delta _{\Gamma } u_{1}, \Delta _{\Gamma } u_{2}, \Delta _{\Gamma } u_{3} \right)^{T}
\]
A method to compute the mean curvature and the so-called Frobenius norm of matrix $A$ involves applying the
extended Weingarten map, $ A\left( x \right) = \nabla_{\Gamma } \nu \left( x \right) $, s.t. these identities holds ,
\begin{equation*}
    \begin{split}
    H & = tr(A) = k_{1} + k_{2}, \\
    \left\lvert A \right\rvert^{2}  & = k_{1}^2 + k_{2}^2,
    \end{split}
.\end{equation*}
see \cite{kovacs2021convergent}.
We may also want to use these definitions to introduce the following identities ,
\[
    \begin{split}
         \partial _{\nu } H & = - \left\lvert A \right\rvert ^{2}, \\
    \nabla _{\Gamma } H & = \Delta  _{\Gamma } \nu  + \left\lvert A \right\rvert ^2 \nu.  \\
    \end{split}
\]
Again, see Lemma 3.3 and Lemma 3.2 in \cite{dougan2012first}.

\subsection{Evolutionary Surface Dynamics}%
\label{sub:evolutionary_equations}

In this section will we develop a framework evolutionary surface dynamics.

First of all, we can denote the velocity $v: \Gamma \mapsto \mathbb{R} ^3$ to be
\begin{equation}
    \label{eq:vel}
\frac{dx }{ d t}  = v\left( x \right) \quad \forall x \in \Gamma \left( t \right) .
\end{equation}
Given a model of the velocity $v$ can we solve the ordinary differential equation (ODE) \eqref{eq:vel} and determine the evolution of a point on the surface $\Gamma\left( t \right)  $. In this article will we assume that the velocity only has a
normal component to the surface, i.e., it exists a scalar function $V: \Gamma \mapsto \mathbb{R} $ s.t. $v = V \nu  $.

% Define shape derivatives
Recall that the point $x = X \left( p,t \right)  $ is arising from the smooth mapping from the point $p $ in  $\Gamma ^{0} $ to $\Gamma \left( t \right) $. Now, let some arbitrary energy functional have the form,
\[
\mathcal{J}\left( \Gamma\left( t \right)   \right)  = \int_{\Gamma\left( t \right)  }^{} \varphi \left( x  \right) .
\]
For instance, in the case presented in \eqref{eq:WE} we define $\varphi = H ^2$.

Based on \cite{dougan2012first},the shape derivative of this energy functional at some time $t$ in the direction of the velocity $v\left( x \right) $ from \eqref{eq:vel} is defined as \[
d \mathcal{J} \left( \Gamma \left( t \right) ; v  \right) = \lim_{\varepsilon  \to 0} \frac{\mathcal{J}  \left( \Gamma \left( t + \varepsilon  \right) \right)  - \mathcal{J}\left( \Gamma \left( t  \right) \right)      }{\varepsilon }.
\]
For a more detailed description of the shape derivative, see \cite[Definition 2.19]{sokolowski1992introduction}.

Assume we have a scalar function $f: \Gamma\left( t \right)  \mapsto \mathbb{R}  $. Similarly, as for the shape derivative, we can now denote the material derivative at time $t$ in the direction of the velocity $v\left( x \right) $ as
\[
    \begin{split}
\frac{D}{Dt}  f\left( x,t; v \right)  & = \frac{d}{dt} f \left( X \left( p,t \right) , t \right) \\
&= \lim_{\varepsilon \to 0}  \frac{f \left( X \left( p, t + \varepsilon  \right)  \right) - f \left( X \left( p, t  \right)  \right) }{ \varepsilon },
    \end{split}
\]
see \cite[Definition 2.74]{sokolowski1992introduction}.


We denote the $L^2\left( \Gamma  \right)  $ as the space of all functions that are square-integrable with respect to the surface measure, i.e., \[
   L^{2}\left( \Gamma   \right)  = \left\{ u: \Gamma \mapsto \mathbb{R}  \mid  \int_{\Gamma }^{}    \left\lvert u \right\rvert ^{2} < \infty   \right\} \\
\]
Let $u,v \in L^{2}\left( \Gamma  \right) $, then can we define the norm and the inner-product \[
    \begin{split}
        \| u \|_{ L^{2}\left( \Gamma  \right)  }^{ 2 } & = \int_{\Gamma }^{} \left\lvert u \right\rvert ^2 \\
        \left( u, v \right)_{L^2\left( \Gamma  \right) } &= \int_{\Gamma }^{} uv  \\
    \end{split}
\]
In this paper will we also the shorthand notation $\| u \|_{ L^2\left( \Gamma  \right)  }^{  }  = \| u \|_{ \Gamma  }^{  } $ and $\left( u,v \right)_{L^2\left( \Gamma  \right) } = \left( u,v \right) _{\Gamma } $.
The Sobolev space $H^1\left( \Gamma  \right) $ is defined as the space of all functions and its first weak derivative with a finite $L^{2}$-norm, i.e,
\[
H^{1}\left( \Gamma  \right) = \left\{ f: \Gamma \mapsto \mathbb{R}   \mid  \int_{\Gamma }^{}
\left\lvert f \right\rvert ^2  + \left\lvert \nabla_\Gamma  f \right\rvert ^2 < \infty \right\},
\]
with the following norm and inner product $u,v \in H^{1}\left( \Gamma  \right) $,
\[
    \begin{split}
        \| u \|_{ H ^{1}\left( \Gamma  \right)  }^{  }  & = \| u \|_{ \Gamma  }^{  }  + \| \nabla_ \Gamma u \|_{ \Gamma  }^{  },  \\
        \left( u, v \right)_{H^1\left( \Gamma  \right) } &= \left( u,v \right) _{\Gamma } + \left( \nabla_\Gamma u, \nabla _{\Gamma } v  \right) _{\Gamma }.   \\
    \end{split}
\]

If we have a vector-valued function that $u: \Gamma  \mapsto  \mathbb{R} ^{3} $ where each element is in $H^1\left( \Gamma  \right) $ or $L^2\left( \Gamma  \right) $, then do we denote is as a member of respectively $\left[ H^{1}\left( \Gamma
\right)   \right]^3 $ or $\left[ L^2\left( \Gamma  \right)  \right] ^3$.




The method we will use in this paper to minimize the energy functional \eqref{eq:WE} is to compute the so-called gradient flow. The fundamental idea of the gradient flow is to give rise of evolutionary dynamics to decrease the overall energy
functional both in space and time, i.e., $\mathcal{J}\left( \Gamma \left( t_{2} \right)  \right) <   \mathcal{J}\left( \Gamma \left( t_{1} \right)\right)$ for all  $t_{2} > t_{1}$. For more information about gradient flows, see
    \cite{dogan2007discrete, dogan2005finite}. Now assume we have the velocity defined in \eqref{eq:vel} to be $v \in \left[ L^{2}\left( \Gamma  \right)  \right]^3 $, then we define the $L^2$  gradient flow s.t. \[
      \left( v,\varphi  \right) _{\Gamma  }  = - d \mathcal{J} \left( \Gamma ; \varphi  \right) \  \forall \varphi \in \left[ L^2\left( \Gamma  \right)   \right] ^3.
    \]
    It turns out, if $v \neq 0$ then is this equivalent to
    \begin{equation}
    \label{eq:gradient_flow}
d \mathcal{J} \left( \Gamma ; v \right) = -\| v \|_{ L^2\left( \Gamma  \right)  }^{^2  } < 0.
    .\end{equation}
Hence, we finally have a toolbox which can be used to model evolutionary dynamics for moving surfaces.








