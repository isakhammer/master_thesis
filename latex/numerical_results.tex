

% Hey
\newpage
\section{Numerical Results}%
\label{sec:numerical_results}


A visualization of  can be seen on figure .
For all experiments carried out will we define $\alpha  = 1 $. We also choose the penalty parameter $ \gamma = 1$, even though the only requirement according to the results in subsection \ref{ssub:coercitivity} is that $\gamma \ge  \frac{1}{2}$. The numerical experiments was done on the mesh discretization illustrated in figure \ref{fig:mesh_discretization}.
The FEM software package used in the implementation was gridap written in the Julia programming language \cite{verdugo22, julia17}.
On the numerical experiments shown on figures \ref{fig:man_conv},  the error norms is shown numerically that there exists a $C > 0$ such
that


\newpage
\subsection{Manufactured Solution}%
\label{sub:manufactured_solution}
We want to apply the well known method of manufactured solution to solve this problem.
Let us consider the manufactured solution. On the experiment did we choose

\begin{equation}
    \label{eq:man_sol}
u\left( x; L,m,r \right) = \cos\left(  m\cdot \frac{2\pi}{L}  x_{1}\right)  \cos \left(r\cdot  \frac{2\pi}{L} x_{2} \right) \quad \text{ on }   \Omega =  \left( 0,L  \right)^{2}
.\end{equation}

\begin{figure}[tbh!]
    \centering
    \includegraphics[width=0.5\textwidth]{figures/model/l_1.0_m_1_r_1n_30_grid.png}
    \caption{example of mesh on $ \omega =  \left( 0,1  \right)^{2}$ where $h=\frac{1}{2^{4}}$  }
    \label{fig:sol_l1_m1_r1}
\end{figure}



\subsubsection{First Example, $L=1$, $m=1$, $r=1$}%
\label{sub:first_example}

\begin{figure}[tbh!]
    \centering
    \includegraphics[width=0.5\textwidth]{figures/model/l_1.0_m_1_r_1n_100_sol.png}
    \caption{Example, $L=1$, $m=1$, $r=1$}
    \label{fig:sol_l1_m1_r1}
\end{figure}

\begin{figure}
    \centering
    \begin{minipage}{.5\linewidth}
    \centering
    \subfloat[]{\label{fig:ex1_conv:a}\includegraphics[scale=.25]{figures/convergence/L_1.0_m_1_r_1/conv_order_2_gamma_9.0.png}}
    \end{minipage}%
    \begin{minipage}{.5\linewidth}
    \centering
    \subfloat[]{\label{fig:ex1_conv:b}\includegraphics[scale=.25]{figures/convergence/L_1.0_m_1_r_1/conv_order_3_gamma_18.0.png}}
    \end{minipage}\par\medskip
    \centering
    \subfloat[]{\label{fig:ex1_conv:c}\includegraphics[scale=.25]{figures/convergence/L_1.0_m_1_r_1/conv_order_4_gamma_30.0.png}}
    \caption{ Example, $L=1$, $m=1$, $r=1$. Polynomials $\mathcal{P}_{k} $ with order $k$ . Figures \ref{fig:ex1_conv:a}, \ref{fig:ex1_conv:b} and \ref{fig:ex1_conv:c} has respectively the order $k=2,3, 4$ with penalty parameters $\gamma = 9,18,30 $.  }
    \label{fig:ex1_conv}
\end{figure}

Illustration can be seen on figure \ref{fig:ex1_conv} with convergence rate \ref{fig:ex1_conv}. The following table is this

\begin{table}
  \begin{tabular}{rrrrrrr}
    \hline\hline
    \textbf{$h/{L} $} & \textbf{$L^2$ norm} & \textbf{EOC} & \textbf{$H_1$ norm} & \textbf{EOC} & \textbf{energy norm} & \textbf{EOC} \\\hline
    $\frac{1}{4}$ & 2.708E-01 &  & 6.068E-01 &  & 2.316E+00 &  \\
    $\frac{1}{8}$ & 6.547E-02 & 2.048E+00 & 1.524E-01 & 1.993E+00 & 1.043E+00 & 1.151E+00 \\
    $\frac{1}{16}$ & 1.621E-02 & 2.014E+00 & 3.796E-02 & 2.006E+00 & 5.078E-01 & 1.039E+00 \\
    $\frac{1}{32}$ & 4.041E-03 & 2.004E+00 & 9.475E-03 & 2.002E+00 & 2.523E-01 & 1.009E+00 \\
    $\frac{1}{64}$ & 1.010E-03 & 2.001E+00 & 2.367E-03 & 2.001E+00 & 1.260E-01 & 1.002E+00 \\
    $\frac{1}{128}$ & 2.524E-04 & 2.000E+00 & 5.918E-04 & 2.000E+00 & 6.296E-02 & 1.001E+00 \\\hline\hline
  \end{tabular}
\end{table}

\begin{table}
  \begin{tabular}{rrrrrrr}
    \hline\hline
    \textbf{$h/{L} $} & \textbf{$L^2$ norm} & \textbf{EOC} & \textbf{$H_1$ norm} & \textbf{EOC} & \textbf{energy norm} & \textbf{EOC} \\\hline
    $\frac{1}{4}$ & 3.363E-03 &  & 7.471E-02 &  & 3.487E+00 &  \\
    $\frac{1}{8}$ & 2.055E-04 & 4.032E+00 & 9.572E-03 & 2.964E+00 & 8.712E-01 & 2.001E+00 \\
    $\frac{1}{16}$ & 1.292E-05 & 3.991E+00 & 1.217E-03 & 2.975E+00 & 2.127E-01 & 2.035E+00 \\
    $\frac{1}{32}$ & 8.153E-07 & 3.986E+00 & 1.538E-04 & 2.984E+00 & 5.232E-02 & 2.023E+00 \\
    $\frac{1}{64}$ & 5.118E-08 & 3.994E+00 & 1.935E-05 & 2.991E+00 & 1.296E-02 & 2.013E+00 \\
    $\frac{1}{128}$ & 4.052E-07 & -2.985E+00 & 2.466E-06 & 2.972E+00 & 3.226E-03 & 2.007E+00 \\\hline\hline
  \end{tabular}
\end{table}

\begin{table}
  \begin{tabular}{rrrrrrr}
    \hline\hline
    \textbf{$h/{L} $} & \textbf{$L^2$ norm} & \textbf{EOC} & \textbf{$H_1$ norm} & \textbf{EOC} & \textbf{energy norm} & \textbf{EOC} \\\hline
    $\frac{1}{4}$ & 1.529E-03 &  & 7.869E-03 &  & 6.700E-02 &  \\
    $\frac{1}{8}$ & 6.050E-05 & 4.660E+00 & 5.437E-04 & 3.855E+00 & 7.205E-03 & 3.217E+00 \\
    $\frac{1}{16}$ & 2.000E-06 & 4.919E+00 & 3.482E-05 & 3.965E+00 & 8.594E-04 & 3.068E+00 \\
    $\frac{1}{32}$ & 6.335E-08 & 4.980E+00 & 2.189E-06 & 3.992E+00 & 1.063E-04 & 3.016E+00 \\
    $\frac{1}{64}$ & 2.828E-08 & 1.164E+00 & 1.432E-07 & 3.934E+00 & 1.325E-05 & 3.003E+00 \\
    $\frac{1}{128}$ & 6.097E-07 & -4.430E+00 & 9.776E-07 & -2.772E+00 & 1.965E-06 & 2.753E+00 \\\hline\hline
  \end{tabular}
\end{table}



\subsubsection{Second Example, $L=1$, $m=7$, $r=3$}%
\label{sub:second_example}

\begin{figure}[tbh!]
    \centering
    \includegraphics[width=0.5\textwidth]{figures/model/l_1.0_m_7_r_3n_100_sol.png}
    \caption{Illustration of the manufactured solution \eqref{eq:man_sol}   with the parameters $L=1$, $m=7$ and $r=3$ in $\Omega = (0,1)^2$}
    \label{fig:sol_l1_m7_r3}
\end{figure}

\begin{figure}
    \centering
    \begin{minipage}{.5\linewidth}
    \centering
    \subfloat[]{\label{fig:ex2_conv:a}\includegraphics[scale=.25]{figures/convergence/L_1.0_m_7_r_3/conv_order_2_gamma_9.0.png}}
    \end{minipage}%
    \begin{minipage}{.5\linewidth}
    \centering
    \subfloat[]{\label{fig:ex2_conv:b}\includegraphics[scale=.25]{figures/convergence/L_1.0_m_7_r_3/conv_order_3_gamma_18.0.png}}
    \end{minipage}\par\medskip
    \centering
    \subfloat[]{\label{fig:ex2_conv:c}\includegraphics[scale=.25]{figures/convergence/L_1.0_m_7_r_3/conv_order_4_gamma_30.0.png}}
    \caption{ Example, $L=1$, $m=1$, $r=1$. Polynomials $\mathcal{P}_{k} $ with order $k$ . Figures \ref{fig:ex2_conv:a}, \ref{fig:ex2_conv:b} and \ref{fig:ex2_conv:c} has respectively the order $k=2,3, 4$ with penalty parameters $\gamma = 9,18,30 $.  }
    \label{fig:ex2_conv}
\end{figure}

Illustration can be seen on figure \ref{fig:ex2_conv} with convergence rate \ref{fig:ex2_conv}. The following table is

\begin{table}
  \begin{tabular}{rrrrrrr}
    \hline\hline
    \textbf{$h/{L} $} & \textbf{$L^2$ norm} & \textbf{EOC} & \textbf{$H_1$ norm} & \textbf{EOC} & \textbf{energy norm} & \textbf{EOC} \\\hline
    $\frac{1}{4}$ & 2.708E-01 &  & 6.068E-01 &  & 2.316E+00 &  \\
    $\frac{1}{8}$ & 6.547E-02 & 2.048E+00 & 1.524E-01 & 1.993E+00 & 1.043E+00 & 1.151E+00 \\
    $\frac{1}{16}$ & 1.621E-02 & 2.014E+00 & 3.796E-02 & 2.006E+00 & 5.078E-01 & 1.039E+00 \\
    $\frac{1}{32}$ & 4.041E-03 & 2.004E+00 & 9.475E-03 & 2.002E+00 & 2.523E-01 & 1.009E+00 \\
    $\frac{1}{64}$ & 1.010E-03 & 2.001E+00 & 2.367E-03 & 2.001E+00 & 1.260E-01 & 1.002E+00 \\
    $\frac{1}{128}$ & 2.524E-04 & 2.000E+00 & 5.918E-04 & 2.000E+00 & 6.296E-02 & 1.001E+00 \\\hline\hline
  \end{tabular}
\end{table}

\begin{table}
  \begin{tabular}{rrrrrrr}
    \hline\hline
    \textbf{$h/{L} $} & \textbf{$L^2$ norm} & \textbf{EOC} & \textbf{$H_1$ norm} & \textbf{EOC} & \textbf{energy norm} & \textbf{EOC} \\\hline
    $\frac{1}{4}$ & 3.363E-03 &  & 7.471E-02 &  & 3.487E+00 &  \\
    $\frac{1}{8}$ & 2.055E-04 & 4.032E+00 & 9.572E-03 & 2.964E+00 & 8.712E-01 & 2.001E+00 \\
    $\frac{1}{16}$ & 1.292E-05 & 3.991E+00 & 1.217E-03 & 2.975E+00 & 2.127E-01 & 2.035E+00 \\
    $\frac{1}{32}$ & 8.153E-07 & 3.986E+00 & 1.538E-04 & 2.984E+00 & 5.232E-02 & 2.023E+00 \\
    $\frac{1}{64}$ & 5.118E-08 & 3.994E+00 & 1.935E-05 & 2.991E+00 & 1.296E-02 & 2.013E+00 \\
    $\frac{1}{128}$ & 4.052E-07 & -2.985E+00 & 2.466E-06 & 2.972E+00 & 3.226E-03 & 2.007E+00 \\\hline\hline
  \end{tabular}
\end{table}

\begin{table}
  \begin{tabular}{rrrrrrr}
    \hline\hline
    \textbf{$h/{L} $} & \textbf{$L^2$ norm} & \textbf{EOC} & \textbf{$H_1$ norm} & \textbf{EOC} & \textbf{energy norm} & \textbf{EOC} \\\hline
    $\frac{1}{4}$ & 1.529E-03 &  & 7.869E-03 &  & 6.700E-02 &  \\
    $\frac{1}{8}$ & 6.050E-05 & 4.660E+00 & 5.437E-04 & 3.855E+00 & 7.205E-03 & 3.217E+00 \\
    $\frac{1}{16}$ & 2.000E-06 & 4.919E+00 & 3.482E-05 & 3.965E+00 & 8.594E-04 & 3.068E+00 \\
    $\frac{1}{32}$ & 6.335E-08 & 4.980E+00 & 2.189E-06 & 3.992E+00 & 1.063E-04 & 3.016E+00 \\
    $\frac{1}{64}$ & 2.828E-08 & 1.164E+00 & 1.432E-07 & 3.934E+00 & 1.325E-05 & 3.003E+00 \\
    $\frac{1}{128}$ & 6.097E-07 & -4.430E+00 & 9.776E-07 & -2.772E+00 & 1.965E-06 & 2.753E+00 \\\hline\hline
  \end{tabular}
\end{table}



% \subsubsection{Third Example, $L=2 \pi $, $m=1$, $r=1$}%
% \label{sub:first_example}

% \begin{figure}[tbh!]
%     \centering
%     \includegraphics[width=0.5\textwidth]{figures/model/l_6.28_m_1_r_1n_100_sol.png}
%     \caption{Illustration of the manufactured solution \eqref{eq:man_sol}   with the parameters $L=2\pi$, $m=1$ and $r=1$ in $\Omega = (0,2\pi)^2$}
%     \label{fig:sol_l2pi_m1_r1}
% \end{figure}

% Illustration can be seen on figure \ref{fig:sol_l2pi_m1_r1}

% \begin{figure}
%     \centering
%     \begin{minipage}{.5\linewidth}
%     \centering
%     \subfloat[]{\label{fig:ex3_conv:a}\includegraphics[scale=.2]{figures/convergence/L_6.28_m_1_r_1/conv_order_2_gamma_9.0.png}}
%     \end{minipage}%
%     \begin{minipage}{.5\linewidth}
%     \centering
%     \subfloat[]{\label{fig:ex3_conv:b}\includegraphics[scale=.2]{figures/convergence/L_6.28_m_1_r_1/conv_order_3_gamma_18.0.png}}
%     \end{minipage}\par\medskip
%     \centering
%     \subfloat[]{\label{fig:ex3_conv:c}\includegraphics[scale=.2]{figures/convergence/L_6.28_m_1_r_1/conv_order_4_gamma_30.0.png}}
%     \caption{ Example, $L=1$, $m=1$, $r=1$. Polynomials $\mathcal{P}_{k} $ with order $k$ . Figures \ref{fig:ex3_conv:a}, \ref{fig:ex3_conv:b} and \ref{fig:ex3_conv:c} has respectively the order $k=2,3, 4$ with penalty parameters $\gamma = 9,18,30 $.  }
%     \label{fig:ex3_conv}
% \end{figure}

% Illustration can be seen on figure \ref{fig:ex3_conv} with convergence rate \ref{fig:ex3_conv}. The following table is


% here is tables
% \begin{table}
  \begin{tabular}{rrrrrrr}
    \hline\hline
    \textbf{$h/{L} $} & \textbf{$L^2$ norm} & \textbf{EOC} & \textbf{$H_1$ norm} & \textbf{EOC} & \textbf{energy norm} & \textbf{EOC} \\\hline
    $\frac{1}{4}$ & 2.708E-01 &  & 6.068E-01 &  & 2.316E+00 &  \\
    $\frac{1}{8}$ & 6.547E-02 & 2.048E+00 & 1.524E-01 & 1.993E+00 & 1.043E+00 & 1.151E+00 \\
    $\frac{1}{16}$ & 1.621E-02 & 2.014E+00 & 3.796E-02 & 2.006E+00 & 5.078E-01 & 1.039E+00 \\
    $\frac{1}{32}$ & 4.041E-03 & 2.004E+00 & 9.475E-03 & 2.002E+00 & 2.523E-01 & 1.009E+00 \\
    $\frac{1}{64}$ & 1.010E-03 & 2.001E+00 & 2.367E-03 & 2.001E+00 & 1.260E-01 & 1.002E+00 \\
    $\frac{1}{128}$ & 2.524E-04 & 2.000E+00 & 5.918E-04 & 2.000E+00 & 6.296E-02 & 1.001E+00 \\\hline\hline
  \end{tabular}
\end{table}

% \begin{table}
  \begin{tabular}{rrrrrrr}
    \hline\hline
    \textbf{$h/{L} $} & \textbf{$L^2$ norm} & \textbf{EOC} & \textbf{$H_1$ norm} & \textbf{EOC} & \textbf{energy norm} & \textbf{EOC} \\\hline
    $\frac{1}{4}$ & 3.363E-03 &  & 7.471E-02 &  & 3.487E+00 &  \\
    $\frac{1}{8}$ & 2.055E-04 & 4.032E+00 & 9.572E-03 & 2.964E+00 & 8.712E-01 & 2.001E+00 \\
    $\frac{1}{16}$ & 1.292E-05 & 3.991E+00 & 1.217E-03 & 2.975E+00 & 2.127E-01 & 2.035E+00 \\
    $\frac{1}{32}$ & 8.153E-07 & 3.986E+00 & 1.538E-04 & 2.984E+00 & 5.232E-02 & 2.023E+00 \\
    $\frac{1}{64}$ & 5.118E-08 & 3.994E+00 & 1.935E-05 & 2.991E+00 & 1.296E-02 & 2.013E+00 \\
    $\frac{1}{128}$ & 4.052E-07 & -2.985E+00 & 2.466E-06 & 2.972E+00 & 3.226E-03 & 2.007E+00 \\\hline\hline
  \end{tabular}
\end{table}

% \begin{table}
  \begin{tabular}{rrrrrrr}
    \hline\hline
    \textbf{$h/{L} $} & \textbf{$L^2$ norm} & \textbf{EOC} & \textbf{$H_1$ norm} & \textbf{EOC} & \textbf{energy norm} & \textbf{EOC} \\\hline
    $\frac{1}{4}$ & 1.529E-03 &  & 7.869E-03 &  & 6.700E-02 &  \\
    $\frac{1}{8}$ & 6.050E-05 & 4.660E+00 & 5.437E-04 & 3.855E+00 & 7.205E-03 & 3.217E+00 \\
    $\frac{1}{16}$ & 2.000E-06 & 4.919E+00 & 3.482E-05 & 3.965E+00 & 8.594E-04 & 3.068E+00 \\
    $\frac{1}{32}$ & 6.335E-08 & 4.980E+00 & 2.189E-06 & 3.992E+00 & 1.063E-04 & 3.016E+00 \\
    $\frac{1}{64}$ & 2.828E-08 & 1.164E+00 & 1.432E-07 & 3.934E+00 & 1.325E-05 & 3.003E+00 \\
    $\frac{1}{128}$ & 6.097E-07 & -4.430E+00 & 9.776E-07 & -2.772E+00 & 1.965E-06 & 2.753E+00 \\\hline\hline
  \end{tabular}
\end{table}



