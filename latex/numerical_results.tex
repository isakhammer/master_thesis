

% Hey
\newpage
\section{Numerical Results}%
\label{sec:numerical_results}


A visualization of  can be seen on figure .
For all experiments carried out will we define $\alpha  = 1 $. We also choose the penalty parameter $ \gamma = 1$, even though the only requirement according to the results in subsection \ref{ssub:coercitivity} is that $\gamma \ge  \frac{1}{2}$. The numerical experiments was done on the mesh discretization illustrated in figure \ref{fig:mesh_discretization}.
The FEM software package used in the implementation was gridap written in the Julia programming language \cite{verdugo22, julia17}.
On the numerical experiments shown on figures \ref{fig:man_conv},  the error norms is shown numerically that there exists a $C > 0$ such
that


\newpage
\subsection{Manufactured Solution}%
\label{sub:manufactured_solution}
We want to apply the well known method of manufactured solution to solve this problem.
Let us consider the manufactured solution. On the experiment did we choose

\begin{equation}
    \label{eq:man_sol}
u\left( x; L,m,r \right) = \cos\left(  m\cdot \frac{2\pi}{L}  x_{1}\right)  \cos \left(r\cdot  \frac{2\pi}{L} x_{2} \right) \quad \text{ on }   \Omega =  \left( 0,L  \right)^{2}
.\end{equation}

\begin{figure}[tbh!]
    \centering
    \includegraphics[width=0.5\textwidth]{figures/model/L_1.0_m_1_r_1n_30_grid.png}
    \caption{Example of mesh on $ \Omega =  \left( 0,1  \right)^{2}$ where $h=\frac{1}{2^{4}}$  }
    \label{fig:sol_l1_m1_r1}
\end{figure}


\subsubsection{First Example, $L=1$, $m=1$, $r=1$}%
\label{sub:first_example}

\begin{figure}[tbh!]
    \centering
    \includegraphics[width=0.5\textwidth]{figures/model/l_1.0_m_1_r_1n_100_sol.png}
    \caption{Example, $L=1$, $m=1$, $r=1$}
    \label{fig:sol_l1_m1_r1}
\end{figure}


Illustration can be seen on figure \ref{fig:sol_l1_m1_r1}

\subsubsection{Second Example, $L=1$, $m=7$, $r=3$}%
\label{sub:second_example}

\begin{figure}[tbh!]
    \centering
    \includegraphics[width=0.5\textwidth]{figures/model/l_1.0_m_7_r_3n_100_sol.png}
    \caption{Illustration of the manufactured solution \eqref{eq:man_sol}   with the parameters $L=1$, $m=7$ and $r=3$ in $\Omega = (0,1)^2$}
    \label{fig:sol_l1_m7_r3}
\end{figure}

Illustration can be seen on figure \ref{fig:sol_l1_m7_r3}

\subsubsection{Third Example, $L=2 \pi $, $m=1$, $r=1$}%
\label{sub:first_example}
\begin{figure}[tbh!]
    \centering
    \includegraphics[width=0.5\textwidth]{figures/model/l_6.28_m_1_r_1n_100_sol.png}
    \caption{Illustration of the manufactured solution \eqref{eq:man_sol}   with the parameters $L=2\pi$, $m=1$ and $r=1$ in $\Omega = (0,2\pi)^2$}
    \label{fig:sol_l2pi_m1_r1}
\end{figure}

Illustration can be seen on figure \ref{fig:sol_l2pi_m1_r1}



% \caption{Illustration of mesh discretization on the unit square $\Omega = (0,2 \pi)^{2} \subseteq  \mathbb{R} ^{2}$
%     with triangulation $\mathcal{T} _{h}$. The chosen instances had the diameters $h = \left\{ 2\pi  /10, 2\pi /20,2\pi /40 \right\}$ in figures \ref{fig:mesh_discretization:a},
%     \ref{fig:mesh_discretization:b} and \ref{fig:mesh_discretization:c}.
% }

% \label{fig:mesh_discretization}
% \end{figure}


% \begin{figure}
%     \centering
% \begin{minipage}{.5\linewidth}
% \centering
% \subfloat[]{\label{fig:man_conv:a}\includegraphics[scale=.05]{figures/convergence/convergence_d_2.png}}
% \end{minipage}%
% \begin{minipage}{.5\linewidth}
% \centering
% \subfloat[]{\label{fig:man_conv:b}\includegraphics[scale=.05]{figures/convergence/convergence_d_3.png}}
% \end{minipage}\par\medskip
% \centering
% \subfloat[]{\label{fig:man_conv:c}\includegraphics[scale=.05]{figures/convergence/convergence_d_4.png}}


% \caption{ Convergence plots for $\| u - u_{h} \|_{ L_{2}\left( \Omega  \right)  }^{  } \le Ch^{p_{1}} $ and $\| u - u_{h} \|_{ H^{1}\left( \Omega  \right)   }^{  }\le Ch^{p_{2}} $  using Lagrangian elements with polynomials $\mathcal{P}_{k} $ with
%     order $k$ .
%     Figures \ref{fig:man_conv:a},
% \ref{fig:man_conv:b} and \ref{fig:man_conv:c} has respectively the order $k=2,3, 4$.  }
% \label{fig:man_conv}
% \end{figure}


% \begin{figure}
% \begin{center}
% \begin{tabular}{|c|c|}
% \toprule

% \textbf{A} & \adjustimage{height=8cm,valign=m}{PlotA} \\
% \midrule
% \textbf{B} & \adjustimage{height=8cm,valign=m}{PlotB} \\
% \midrule
% \textbf{C} & \adjustimage{height=8cm,valign=m}{PlotC} \\

% \bottomrule
% \end{tabular}
% \end{center}
% \caption{I don't want a table: Andrew Cashner's way} \label{faketable:mul}

% % \end{figure}

% \begin{table}
  \begin{tabular}{rrrrr}
    \hline\hline
    \textbf{h} & \textbf{$L_2$} & \textbf{$H^1$} & \textbf{$log_2(e^{2h}_{L^2(\Omega )}/e^{h}_{L^2(\Omega )}) $} & \textbf{$log_2(e^{2h}_{H_1(\Omega )}/e^{h}_{H_1(\Omega )}) $} \\\hline
    7.854E-01 & 9.380E+00 & 2.269E+00 &  &  \\
    3.927E-01 & 4.891E-01 & 1.410E-01 & -2.954E+00 & -2.778E+00 \\
    1.963E-01 & 1.180E-01 & 3.939E-02 & -1.422E+00 & -1.276E+00 \\
    9.817E-02 & 3.695E-02 & 1.107E-02 & -1.161E+00 & -1.269E+00 \\\hline\hline
  \end{tabular}
\end{table}


